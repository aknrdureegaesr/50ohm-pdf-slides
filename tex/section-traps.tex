
\section{Traps}
\label{section:traps}
\begin{frame}%STARTCONTENT

\only<1>{
\begin{PQuestion}{AG109}{Welche Antennenart ist hier dargestellt?  }{Sperrkreis-Dipol}
{Einband-Dipol mit Oberwellenfilter}
{Dipol mit Gleichwellenfilter}
{Saugkreis-Dipol}
{\DARCimage{1.0\linewidth}{234include}}\end{PQuestion}

}
\only<2>{
\begin{PQuestion}{AG109}{Welche Antennenart ist hier dargestellt?  }{\textbf{\textcolor{DARCgreen}{Sperrkreis-Dipol}}}
{Einband-Dipol mit Oberwellenfilter}
{Dipol mit Gleichwellenfilter}
{Saugkreis-Dipol}
{\DARCimage{1.0\linewidth}{234include}}\end{PQuestion}

}
\end{frame}

\begin{frame}
\only<1>{
\begin{QQuestion}{AG110}{Ein Parallelresonanzkreis (Trap) in jeder Dipolhälfte~...}{erhöht die effiziente Nutzung des jeweiligen Frequenzbereichs.}
{erlaubt eine Nutzung der Antenne für mindestens zwei Frequenzbereiche.}
{beschränkt die Nutzbarkeit der Antenne auf einen Frequenzbereich.}
{ermöglicht die Unterdrückung der Harmonischen.}
\end{QQuestion}

}
\only<2>{
\begin{QQuestion}{AG110}{Ein Parallelresonanzkreis (Trap) in jeder Dipolhälfte~...}{erhöht die effiziente Nutzung des jeweiligen Frequenzbereichs.}
{\textbf{\textcolor{DARCgreen}{erlaubt eine Nutzung der Antenne für mindestens zwei Frequenzbereiche.}}}
{beschränkt die Nutzbarkeit der Antenne auf einen Frequenzbereich.}
{ermöglicht die Unterdrückung der Harmonischen.}
\end{QQuestion}

}
\end{frame}

\begin{frame}
\only<1>{
\begin{PQuestion}{AG113}{Wenn man diese Mehrband-Antenne auf \qty{14}{\MHz} erregt, dann wirken die LC-Resonanzkreise~...}{als Vergrößerung des Strahlungswiderstands der Antenne.}
{als Sperrkreise für die Erregerfrequenz.}
{als induktive Verlängerung des Strahlers.}
{als kapazitive Verkürzung des Strahlers.}
{\DARCimage{1.0\linewidth}{235include}}\end{PQuestion}

}
\only<2>{
\begin{PQuestion}{AG113}{Wenn man diese Mehrband-Antenne auf \qty{14}{\MHz} erregt, dann wirken die LC-Resonanzkreise~...}{als Vergrößerung des Strahlungswiderstands der Antenne.}
{als Sperrkreise für die Erregerfrequenz.}
{als induktive Verlängerung des Strahlers.}
{\textbf{\textcolor{DARCgreen}{als kapazitive Verkürzung des Strahlers.}}}
{\DARCimage{1.0\linewidth}{235include}}\end{PQuestion}

}
\end{frame}

\begin{frame}
\only<1>{
\begin{PQuestion}{AG112}{Wenn man diese Mehrband-Antenne auf \qty{7}{\MHz} erregt, dann wirken die LC-Resonanzkreise~...}{als Vergrößerung des Strahlungswiderstands der Antenne.}
{als induktive Verlängerung des Strahlers.}
{als kapazitive Verkürzung des Strahlers.}
{als Sperrkreise für die Erregerfrequenz.}
{\DARCimage{1.0\linewidth}{235include}}\end{PQuestion}

}
\only<2>{
\begin{PQuestion}{AG112}{Wenn man diese Mehrband-Antenne auf \qty{7}{\MHz} erregt, dann wirken die LC-Resonanzkreise~...}{als Vergrößerung des Strahlungswiderstands der Antenne.}
{als induktive Verlängerung des Strahlers.}
{als kapazitive Verkürzung des Strahlers.}
{\textbf{\textcolor{DARCgreen}{als Sperrkreise für die Erregerfrequenz.}}}
{\DARCimage{1.0\linewidth}{235include}}\end{PQuestion}

}
\end{frame}

\begin{frame}
\only<1>{
\begin{PQuestion}{AG116}{Sie wollen eine Zweibandantenne für \qty{160}{\m} und \qty{80}{\m} selbst bauen. Welche der folgenden Antworten enthält die richtige Drahtlänge $l$ zwischen den Traps und die richtige Resonanzfrequenz $f_{\symup{res}}$ der Schwingkreise?}{$l$ beträgt zirka \qty{80}{\m}, $f_{\symup{res}}$ liegt bei zirka \qty{3,65}{\MHz}.}
{$l$ beträgt zirka \qty{40}{\m}, $f_{\symup{res}}$ liegt bei zirka \qty{3,65}{\MHz}.}
{$l$ beträgt zirka \qty{40}{\m}, $f_{\symup{res}}$ liegt bei zirka \qty{1,85}{\MHz}.}
{$l$ beträgt zirka \qty{80}{\m}, $f_{\symup{res}}$ liegt bei zirka \qty{1,85}{\MHz}.}
{\DARCimage{1.0\linewidth}{236include}}\end{PQuestion}

}
\only<2>{
\begin{PQuestion}{AG116}{Sie wollen eine Zweibandantenne für \qty{160}{\m} und \qty{80}{\m} selbst bauen. Welche der folgenden Antworten enthält die richtige Drahtlänge $l$ zwischen den Traps und die richtige Resonanzfrequenz $f_{\symup{res}}$ der Schwingkreise?}{$l$ beträgt zirka \qty{80}{\m}, $f_{\symup{res}}$ liegt bei zirka \qty{3,65}{\MHz}.}
{\textbf{\textcolor{DARCgreen}{$l$ beträgt zirka \qty{40}{\m}, $f_{\symup{res}}$ liegt bei zirka \qty{3,65}{\MHz}.}}}
{$l$ beträgt zirka \qty{40}{\m}, $f_{\symup{res}}$ liegt bei zirka \qty{1,85}{\MHz}.}
{$l$ beträgt zirka \qty{80}{\m}, $f_{\symup{res}}$ liegt bei zirka \qty{1,85}{\MHz}.}
{\DARCimage{1.0\linewidth}{236include}}\end{PQuestion}

}
\end{frame}

\begin{frame}
\only<1>{
\begin{PQuestion}{AG111}{Wenn man diese Mehrband-Antenne auf \qty{3,5}{\MHz} erregt, dann wirken die LC-Resonanzkreise~...}{als Vergrößerung des Strahlungswiderstands der Antenne.}
{als Sperrkreise für die Erregerfrequenz.}
{als kapazitive Verkürzung des Strahlers.}
{als induktive Verlängerung des Strahlers.}
{\DARCimage{1.0\linewidth}{235include}}\end{PQuestion}

}
\only<2>{
\begin{PQuestion}{AG111}{Wenn man diese Mehrband-Antenne auf \qty{3,5}{\MHz} erregt, dann wirken die LC-Resonanzkreise~...}{als Vergrößerung des Strahlungswiderstands der Antenne.}
{als Sperrkreise für die Erregerfrequenz.}
{als kapazitive Verkürzung des Strahlers.}
{\textbf{\textcolor{DARCgreen}{als induktive Verlängerung des Strahlers.}}}
{\DARCimage{1.0\linewidth}{235include}}\end{PQuestion}

}
\end{frame}

\begin{frame}
\only<1>{
\begin{PQuestion}{AG115}{Das folgende Bild stellt einen Dreiband-Dipol für die Frequenzbänder \num{20}, \num{15} und \qty{10}{\m} dar. Die mit b gekennzeichneten Schwingkreise sind abgestimmt auf:}{\qty{21,2}{\MHz}}
{\qty{10,1}{\MHz}}
{\qty{14,2}{\MHz}}
{\qty{29,0}{\MHz}}
{\DARCimage{1.0\linewidth}{237include}}\end{PQuestion}

}
\only<2>{
\begin{PQuestion}{AG115}{Das folgende Bild stellt einen Dreiband-Dipol für die Frequenzbänder \num{20}, \num{15} und \qty{10}{\m} dar. Die mit b gekennzeichneten Schwingkreise sind abgestimmt auf:}{\qty{21,2}{\MHz}}
{\qty{10,1}{\MHz}}
{\qty{14,2}{\MHz}}
{\textbf{\textcolor{DARCgreen}{\qty{29,0}{\MHz}}}}
{\DARCimage{1.0\linewidth}{237include}}\end{PQuestion}

}
\end{frame}

\begin{frame}
\only<1>{
\begin{PQuestion}{AG114}{Das folgende Bild stellt einen Dreiband-Dipol für die Frequenzbänder \num{20}, \num{15} und \qty{10}{\m} dar. Die mit a gekennzeichneten Schwingkreise sind abgestimmt auf:}{\qty{10,1}{\MHz}}
{\qty{21,2}{\MHz}}
{\qty{14,2}{\MHz}}
{\qty{29,0}{\MHz}}
{\DARCimage{1.0\linewidth}{237include}}\end{PQuestion}

}
\only<2>{
\begin{PQuestion}{AG114}{Das folgende Bild stellt einen Dreiband-Dipol für die Frequenzbänder \num{20}, \num{15} und \qty{10}{\m} dar. Die mit a gekennzeichneten Schwingkreise sind abgestimmt auf:}{\qty{10,1}{\MHz}}
{\textbf{\textcolor{DARCgreen}{\qty{21,2}{\MHz}}}}
{\qty{14,2}{\MHz}}
{\qty{29,0}{\MHz}}
{\DARCimage{1.0\linewidth}{237include}}\end{PQuestion}

}
\end{frame}%ENDCONTENT
