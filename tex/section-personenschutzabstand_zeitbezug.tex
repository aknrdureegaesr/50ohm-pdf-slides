
\section{Zeitbezug beim Personenschutzabstand}
\label{section:personenschutzabstand_zeitbezug}
\begin{frame}%STARTCONTENT

\frametitle{Zeitbezug}
\begin{itemize}
  \item In der \enquote{26. Verordnung zur Durchführung des Bundes-Immissionsschutzgesetzes} wird ein zeitlicher Bezug zur Einhaltung der Feldstärke-Gernzwerte hinzugefügt
  \item Es muss nach drei Fällen für Grenzwerte unterschieden werden
  \end{itemize}
\end{frame}

\begin{frame}
\frametitle{6-Minuten-Intervalle}
\begin{itemize}
  \item Da nicht ständig gesendet wird, Verwendung des quadratischen Mittels der Feldstärke (V/m) über 6 Minuten
  \item Grenzwerte sind frequenzabhängig
  \item z.B. \qty{28}{\volt}/m bei \qty{14}{\mega\hertz}
  \item Berechnung erfolgt mit Näherungsformel (im nächsten Abschnitt)
  \end{itemize}
\end{frame}

\begin{frame}
\frametitle{Momentaner Spitzenwert}
\begin{itemize}
  \item Maximaler momentaner Spitzenwert
  \item Elektrische Feldstärke in kV/m
  \item Grenzwerte sind frequenzabhängig
  \item z.B. \qty{5}{\kilo\volt}/m bei \qty{14}{\mega\hertz}
  \end{itemize}
\end{frame}

\begin{frame}
\frametitle{Gepulste Felder}
\begin{itemize}
  \item Schnelles Ein- und Ausschalten
  \item Als Faktor für den momentanen Spitzenwert oder das 6-Minuten-Intervall
  \item Grenzwerte sind frequenzabhängig
  \item z.B. 32-fache des 6-Minuten-Intervalls bei \qty{14}{\mega\hertz}
  \end{itemize}
\end{frame}

\begin{frame}
\only<1>{
\begin{QQuestion}{EK102}{Mit welchem zeitlichen Bezug ist die Feldstärke für die Einhaltung der Grenzwerte der 26. Verordnung zur Durchführung des Bundes-Immissionsschutzgesetzes (Verordnung über elektromagnetische Felder - 26. BImSchV) zu betrachten?}{Quadratisch gemittelt über 6 Minuten für Grenzwerte nach Anhang 1b, als kurzfristiger Effektivwert für Grenzwerte nach Anhang 1a und als momentaner Spitzenwert für Grenzwerte nach Anhang 3}
{Quadratisch gemittelt über 3 Minuten für Grenzwerte nach Anhang 1b, als kurzfristiger Effektivwert für Grenzwerte nach Anhang 1a und als momentaner Spitzenwert für Grenzwerte nach Anhang 3}
{Tagsüber maximale Momentanwerte und in den Nachtstunden zwischen Einbruch der Dunkelheit und Sonnenaufgang quadratisch gemittelt über 6 Minuten}
{Tagsüber maximale Momentanwerte und in den Nachtstunden zwischen Einbruch der Dunkelheit und Sonnenaufgang quadratisch gemittelt über 3 Minuten}
\end{QQuestion}

}
\only<2>{
\begin{QQuestion}{EK102}{Mit welchem zeitlichen Bezug ist die Feldstärke für die Einhaltung der Grenzwerte der 26. Verordnung zur Durchführung des Bundes-Immissionsschutzgesetzes (Verordnung über elektromagnetische Felder - 26. BImSchV) zu betrachten?}{\textbf{\textcolor{DARCgreen}{Quadratisch gemittelt über 6 Minuten für Grenzwerte nach Anhang 1b, als kurzfristiger Effektivwert für Grenzwerte nach Anhang 1a und als momentaner Spitzenwert für Grenzwerte nach Anhang 3}}}
{Quadratisch gemittelt über 3 Minuten für Grenzwerte nach Anhang 1b, als kurzfristiger Effektivwert für Grenzwerte nach Anhang 1a und als momentaner Spitzenwert für Grenzwerte nach Anhang 3}
{Tagsüber maximale Momentanwerte und in den Nachtstunden zwischen Einbruch der Dunkelheit und Sonnenaufgang quadratisch gemittelt über 6 Minuten}
{Tagsüber maximale Momentanwerte und in den Nachtstunden zwischen Einbruch der Dunkelheit und Sonnenaufgang quadratisch gemittelt über 3 Minuten}
\end{QQuestion}

}
\end{frame}

\begin{frame}
\frametitle{Körperhilfen}
\begin{itemize}
  \item Aktive Körperhilfen (z.B. Herzschrittmacher) dürfen nicht in elektrische Felder gebracht werden, deren Stärke die Grenzwerte der aktiven Körperhilfe überschreiten
  \item Der Genzwert ist hier immer der maximale Momentanwert
  \end{itemize}
\end{frame}

\begin{frame}
\only<1>{
\begin{QQuestion}{EK103}{Zum Schutz von Personen in elektromagnetischen Feldern sind in bestimmten Fällen auch Grenzwerte für aktive Körperhilfen einzuhalten. Mit welchem zeitlichen Bezug ist die Feldstärke hierbei zu betrachten?}{Quadratisch gemittelt über 6 Minuten}
{Als minimaler Momentanwert}
{Als maximaler Momentanwert}
{Quadratisch gemittelt über 3 Minuten}
\end{QQuestion}

}
\only<2>{
\begin{QQuestion}{EK103}{Zum Schutz von Personen in elektromagnetischen Feldern sind in bestimmten Fällen auch Grenzwerte für aktive Körperhilfen einzuhalten. Mit welchem zeitlichen Bezug ist die Feldstärke hierbei zu betrachten?}{Quadratisch gemittelt über 6 Minuten}
{Als minimaler Momentanwert}
{\textbf{\textcolor{DARCgreen}{Als maximaler Momentanwert}}}
{Quadratisch gemittelt über 3 Minuten}
\end{QQuestion}

}
\end{frame}%ENDCONTENT
