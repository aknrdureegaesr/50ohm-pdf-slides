
\section{Heißleiter und Kaltleiter}
\label{section:widerstand_ntc_ptc}
\begin{frame}%STARTCONTENT

\frametitle{Heißleiter}
\begin{columns}
    \begin{column}{0.48\textwidth}
    \begin{itemize}
  \item Heißleiter ist ein temperaturabhängiger Widerstand
  \item Englisch: Negative Temperature Coefficient Thermistor (\emph{NTC})
  \item Leitet bei \emph{hohen Temperaturen} elektrischen Strom besser
  \end{itemize}

    \end{column}
   \begin{column}{0.48\textwidth}
       
\begin{figure}
    \DARCimage{0.85\linewidth}{125include}
    \caption{\scriptsize Schaltzeichen eines NTC-Widerstands}
    \label{e_ntc}
\end{figure}


   \end{column}
\end{columns}

\end{frame}

\begin{frame}
\only<1>{
\begin{PQuestion}{EC109}{Welches Bauteil hat folgendes Schaltzeichen?}{VDR}
{PTC}
{LDR}
{NTC}
{\DARCimage{0.5\linewidth}{125include}}\end{PQuestion}

}
\only<2>{
\begin{PQuestion}{EC109}{Welches Bauteil hat folgendes Schaltzeichen?}{VDR}
{PTC}
{LDR}
{\textbf{\textcolor{DARCgreen}{NTC}}}
{\DARCimage{0.5\linewidth}{125include}}\end{PQuestion}

}
\end{frame}

\begin{frame}
\only<1>{
\begin{question2x2}{EC110}{Welches der folgenden Bauteile ist ein NTC-Widerstand?}{\DARCimage{1.0\linewidth}{128include}}
{\DARCimage{1.0\linewidth}{126include}}
{\DARCimage{1.0\linewidth}{127include}}
{\DARCimage{1.0\linewidth}{125include}}
\end{question2x2}

}
\only<2>{
\begin{question2x2}{EC110}{Welches der folgenden Bauteile ist ein NTC-Widerstand?}{\DARCimage{1.0\linewidth}{128include}}
{\DARCimage{1.0\linewidth}{126include}}
{\DARCimage{1.0\linewidth}{127include}}
{\textbf{\textcolor{DARCgreen}{\DARCimage{1.0\linewidth}{125include}}}}
\end{question2x2}

}
\end{frame}

\begin{frame}
\frametitle{Kaltleiter}
\begin{columns}
    \begin{column}{0.48\textwidth}
    \begin{itemize}
  \item Kaltleiter ist ein temperaturabhängiger Widerstand
  \item Englisch: Positive Temperature Coefficient Thermistor (\emph{PTC})
  \item Leitet bei \emph{tiefen Temperaturen} elektrischen Strom besser
  \end{itemize}

    \end{column}
   \begin{column}{0.48\textwidth}
       
\begin{figure}
    \DARCimage{0.85\linewidth}{126include}
    \caption{\scriptsize Schaltzeichen eines PTC-Widerstands}
    \label{e_ptc}
\end{figure}


   \end{column}
\end{columns}

\end{frame}

\begin{frame}
\only<1>{
\begin{question2x2}{EC111}{Welches der folgenden Schaltsymbole stellt einen PTC-Widerstand dar?}{\DARCimage{1.0\linewidth}{126include}}
{\DARCimage{1.0\linewidth}{125include}}
{\DARCimage{1.0\linewidth}{127include}}
{\DARCimage{1.0\linewidth}{128include}}
\end{question2x2}

}
\only<2>{
\begin{question2x2}{EC111}{Welches der folgenden Schaltsymbole stellt einen PTC-Widerstand dar?}{\textbf{\textcolor{DARCgreen}{\DARCimage{1.0\linewidth}{126include}}}}
{\DARCimage{1.0\linewidth}{125include}}
{\DARCimage{1.0\linewidth}{127include}}
{\DARCimage{1.0\linewidth}{128include}}
\end{question2x2}

}
\end{frame}%ENDCONTENT
