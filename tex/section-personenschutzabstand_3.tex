
\section{Personenschutzabstand III}
\label{section:personenschutzabstand_3}
\begin{frame}%STARTCONTENT

\only<1>{
\begin{QQuestion}{AK102}{Durch welche Größe sind Beträge der elektrischen und magnetischen Feldstärke eines elektromagnetischen Feldes im Fernfeld miteinander verknüpft?}{Durch die Aufbauhöhe der Antenne}
{Durch den Wellenwiderstand im jeweiligen Medium }
{Durch die Ausbreitungsbedingungen in der Ionosphäre}
{Durch die Polarisationsrichtung der verwendeten Antenne}
\end{QQuestion}

}
\only<2>{
\begin{QQuestion}{AK102}{Durch welche Größe sind Beträge der elektrischen und magnetischen Feldstärke eines elektromagnetischen Feldes im Fernfeld miteinander verknüpft?}{Durch die Aufbauhöhe der Antenne}
{\textbf{\textcolor{DARCgreen}{Durch den Wellenwiderstand im jeweiligen Medium }}}
{Durch die Ausbreitungsbedingungen in der Ionosphäre}
{Durch die Polarisationsrichtung der verwendeten Antenne}
\end{QQuestion}

}
\end{frame}

\begin{frame}
\only<1>{
\begin{QQuestion}{AK104}{Wie errechnen Sie die Leistung am Einspeisepunkt der Antenne (Antenneneingangsleistung) bei bekannter Senderausgangsleistung?}{Die Antenneneingangsleistung ist der Spitzenwert der Senderausgangsleistung, also: $P_{\symup{Ant}}~=~\sqrt{2\cdot P_{\symup{Sender}}}$}
{Antenneneingangsleistung und Senderausgangsleistung sind gleich, da die Kabelverluste bei Amateurfunkstationen vernachlässigbar klein sind, d. h. es gilt: $P_{\symup{Ant}}~=~P_{\symup{Sender}}$}
{Sie ermitteln die Verluste zwischen Senderausgang und Antenneneingang und berechnen aus dieser Dämpfung einen Dämpfungsfaktor$~D$; die Antenneneingangsleistung ist dann: $P_{\symup{Ant}}~=~D\cdot P_{\symup{Sender}}$}
{Die Antenneneingangsleistung ist der Spitzen-Spitzen-Wert der Senderausgangsleistung, also: $P_{\symup{Ant}}~=~2\cdot\sqrt{2\cdot P_{\symup{Sender}}}$}
\end{QQuestion}

}
\only<2>{
\begin{QQuestion}{AK104}{Wie errechnen Sie die Leistung am Einspeisepunkt der Antenne (Antenneneingangsleistung) bei bekannter Senderausgangsleistung?}{Die Antenneneingangsleistung ist der Spitzenwert der Senderausgangsleistung, also: $P_{\symup{Ant}}~=~\sqrt{2\cdot P_{\symup{Sender}}}$}
{Antenneneingangsleistung und Senderausgangsleistung sind gleich, da die Kabelverluste bei Amateurfunkstationen vernachlässigbar klein sind, d. h. es gilt: $P_{\symup{Ant}}~=~P_{\symup{Sender}}$}
{\textbf{\textcolor{DARCgreen}{Sie ermitteln die Verluste zwischen Senderausgang und Antenneneingang und berechnen aus dieser Dämpfung einen Dämpfungsfaktor$~D$; die Antenneneingangsleistung ist dann: $P_{\symup{Ant}}~=~D\cdot P_{\symup{Sender}}$}}}
{Die Antenneneingangsleistung ist der Spitzen-Spitzen-Wert der Senderausgangsleistung, also: $P_{\symup{Ant}}~=~2\cdot\sqrt{2\cdot P_{\symup{Sender}}}$}
\end{QQuestion}

}
\end{frame}

\begin{frame}
\only<1>{
\begin{QQuestion}{AK115}{Eine Amateurfunkstelle sendet in FM mit einer äquivalenten Strahlungsleistung (ERP) von \qty{100}{\W}. Wie groß ist die Feldstärke im freien Raum in einer Entfernung von \qty{100}{\m}?}{\qty{0,7}{\V}/m}
{\qty{0,5}{\V}/m}
{\qty{0,43}{\V}/m}
{\qty{0,55}{\V}/m}
\end{QQuestion}

}
\only<2>{
\begin{QQuestion}{AK115}{Eine Amateurfunkstelle sendet in FM mit einer äquivalenten Strahlungsleistung (ERP) von \qty{100}{\W}. Wie groß ist die Feldstärke im freien Raum in einer Entfernung von \qty{100}{\m}?}{\textbf{\textcolor{DARCgreen}{\qty{0,7}{\V}/m}}}
{\qty{0,5}{\V}/m}
{\qty{0,43}{\V}/m}
{\qty{0,55}{\V}/m}
\end{QQuestion}

}
\end{frame}

\begin{frame}
\frametitle{Lösungsweg}
\begin{itemize}
  \item gegeben: $P_{ERP} = 100W$
  \item gegeben: $d = 100m$
  \item gesucht: $E$
  \end{itemize}
    \pause
    $P_{EIRP} = P_{ERP} \cdot 1,64 = 100W \cdot 1,64 = 164W$
    \pause
    $E = \frac{\sqrt{30Ω \cdot P_{EIRP}}}{d} = \frac{\sqrt{30Ω \cdot 164W}}{100m} = 0,7\frac{V}{m}$



\end{frame}

\begin{frame}
\only<1>{
\begin{QQuestion}{AK114}{Eine vertikale Dipol-Antenne wird mit \qty{10}{\W} Sendeleistung im \qty{70}{\cm}-Band direkt gespeist. Welche elektrische Feldstärke ergibt sich bei Freiraumausbreitung in \qty{10}{\m} Entfernung in etwa?}{\qty{1,7}{\V}/m}
{\qty{8,9}{\V}/m}
{\qty{0,4}{\V}/m}
{\qty{2,2}{\V}/m}
\end{QQuestion}

}
\only<2>{
\begin{QQuestion}{AK114}{Eine vertikale Dipol-Antenne wird mit \qty{10}{\W} Sendeleistung im \qty{70}{\cm}-Band direkt gespeist. Welche elektrische Feldstärke ergibt sich bei Freiraumausbreitung in \qty{10}{\m} Entfernung in etwa?}{\qty{1,7}{\V}/m}
{\qty{8,9}{\V}/m}
{\qty{0,4}{\V}/m}
{\textbf{\textcolor{DARCgreen}{\qty{2,2}{\V}/m}}}
\end{QQuestion}

}
\end{frame}

\begin{frame}
\frametitle{Lösungsweg}
\begin{itemize}
  \item gegeben: $P_{ERP} = 10W$
  \item gegeben: $d = 10m$
  \item gesucht: $E$
  \end{itemize}
    \pause
    $P_{EIRP} = P_{ERP} \cdot 1,64 = 10W \cdot 1,64 = 16,4W$
    \pause
    $E = \frac{\sqrt{30Ω \cdot P_{EIRP}}}{d} = \frac{\sqrt{30Ω \cdot 16,4W}}{10m} = 2,2\frac{V}{m}$



\end{frame}

\begin{frame}
\only<1>{
\begin{QQuestion}{AK113}{Eine Yagi-Uda-Antenne mit \qty{12,15}{\dBi} Antennengewinn wird mit \qty{250}{\W} Sendeleistung im \qty{2}{\m}-Band direkt gespeist. Welche elektrische Feldstärke ergibt sich bei Freiraumausbreitung in \qty{30}{\m} Entfernung in etwa?}{\qty{9,1}{\V}/m}
{\qty{11,7}{\V}/m}
{\qty{15,0}{\V}/m}
{\qty{10,1}{\V}/m}
\end{QQuestion}

}
\only<2>{
\begin{QQuestion}{AK113}{Eine Yagi-Uda-Antenne mit \qty{12,15}{\dBi} Antennengewinn wird mit \qty{250}{\W} Sendeleistung im \qty{2}{\m}-Band direkt gespeist. Welche elektrische Feldstärke ergibt sich bei Freiraumausbreitung in \qty{30}{\m} Entfernung in etwa?}{\qty{9,1}{\V}/m}
{\textbf{\textcolor{DARCgreen}{\qty{11,7}{\V}/m}}}
{\qty{15,0}{\V}/m}
{\qty{10,1}{\V}/m}
\end{QQuestion}

}
\end{frame}

\begin{frame}
\frametitle{Lösungsweg}
\begin{itemize}
  \item gegeben: $g_i = 12,15dBi$
  \item gegeben: $P_A = 250W$
  \item gegeben: $d = 30m$
  \item gesucht: $E$
  \end{itemize}
    \pause
    $G_i = 10^{\frac{g_i}{10dB}} = 10^{\frac{12,15dBi}{10dB}} = 16,4$
    \pause
    $E = \frac{\sqrt{30Ω \cdot P_A \cdot G_i}}{d} = \frac{\sqrt{30Ω \cdot 250W \cdot 16,4}}{30m} = \frac{350V}{30m} \approx 11,7\frac{V}{m}$



\end{frame}

\begin{frame}
\only<1>{
\begin{QQuestion}{AK107}{Sie betreiben eine Amateurfunkstelle auf dem \qty{2}{\m}-Band im Modulationsverfahren FM mit einer Rundstrahlantenne mit \qty{6}{\decibel} Gewinn bezogen auf einen Dipol. Wie hoch darf die maximale Ausgangsleistung Ihres Senders unter Vernachlässigung der Kabeldämpfung sein, wenn der Grenzwert für den Personenschutz \qty{28}{\V\per\m} und der zur Verfügung stehende Sicherheitsabstand \qty{5}{\m} beträgt?}{ca. \qty{75}{\W}}
{ca. \qty{100}{\W}}
{ca. \qty{160}{\W}}
{ca. \qty{265}{\W}}
\end{QQuestion}

}
\only<2>{
\begin{QQuestion}{AK107}{Sie betreiben eine Amateurfunkstelle auf dem \qty{2}{\m}-Band im Modulationsverfahren FM mit einer Rundstrahlantenne mit \qty{6}{\decibel} Gewinn bezogen auf einen Dipol. Wie hoch darf die maximale Ausgangsleistung Ihres Senders unter Vernachlässigung der Kabeldämpfung sein, wenn der Grenzwert für den Personenschutz \qty{28}{\V\per\m} und der zur Verfügung stehende Sicherheitsabstand \qty{5}{\m} beträgt?}{ca. \qty{75}{\W}}
{\textbf{\textcolor{DARCgreen}{ca. \qty{100}{\W}}}}
{ca. \qty{160}{\W}}
{ca. \qty{265}{\W}}
\end{QQuestion}

}
\end{frame}

\begin{frame}
\frametitle{Lösungsweg}
\begin{itemize}
  \item gegeben: $g_d = 6dBd$
  \item gegeben: $E = 28\frac{V}{m}$
  \item gegeben: $d = 5m$
  \item gesucht: $P_S$
  \end{itemize}
    \pause
    $E = \frac{\sqrt{30Ω \cdot P_{EIRP}}}{d} \Rightarrow P_{EIRP} = \frac{(E \cdot d)^2}{30Ω} = \frac{(28\frac{V}{m} \cdot 5m)^2}{30Ω} = 653W$
    \pause
    $P_{EIRP} = P_S \cdot 10^{\frac{g_d -- a + 2,15dB}{10dB}} \Rightarrow P_S = \frac{P_{EIRP}}{10^{\frac{g_d -- a + 2,15dB}{10dB}}} = \frac{653W}{10^{\frac{6dBd -- 0 + 2,15dB}{10dB}}} = \frac{653W}{6,53} \approx 100W$



\end{frame}%ENDCONTENT
