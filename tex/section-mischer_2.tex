
\section{Mischer II}
\label{section:mischer_2}
\begin{frame}%STARTCONTENT

\only<1>{
\begin{QQuestion}{AF212}{In welchem Bereich der Steuerkennlinie arbeitet die Mischstufe eines Überlagerungsempfängers? }{Sie arbeitet im nichtlinearen Bereich.}
{Sie arbeitet im kapazitiven Bereich.}
{Sie arbeitet im induktiven Bereich.}
{Sie arbeitet im linearen Bereich.}
\end{QQuestion}

}
\only<2>{
\begin{QQuestion}{AF212}{In welchem Bereich der Steuerkennlinie arbeitet die Mischstufe eines Überlagerungsempfängers? }{\textbf{\textcolor{DARCgreen}{Sie arbeitet im nichtlinearen Bereich.}}}
{Sie arbeitet im kapazitiven Bereich.}
{Sie arbeitet im induktiven Bereich.}
{Sie arbeitet im linearen Bereich.}
\end{QQuestion}

}
\end{frame}

\begin{frame}
\only<1>{
\begin{QQuestion}{AF213}{Durch welchen Mischer werden unerwünschte Ausgangssignale am stärksten unterdrückt?}{additiver Diodenmischer}
{Balancemischer}
{Dualtransistormischer}
{Doppeldiodenmischer}
\end{QQuestion}

}
\only<2>{
\begin{QQuestion}{AF213}{Durch welchen Mischer werden unerwünschte Ausgangssignale am stärksten unterdrückt?}{additiver Diodenmischer}
{\textbf{\textcolor{DARCgreen}{Balancemischer}}}
{Dualtransistormischer}
{Doppeldiodenmischer}
\end{QQuestion}

}
\end{frame}

\begin{frame}
\only<1>{
\begin{QQuestion}{AF214}{Welche Mischerschaltung unterdrückt am wirksamsten unerwünschte Mischprodukte und Frequenzen?}{Ein Eintakt-Transistormischer}
{Ein unbalancierter Produktdetektor}
{Ein balancierter Ringmischer}
{Ein additiver Diodenmischer}
\end{QQuestion}

}
\only<2>{
\begin{QQuestion}{AF214}{Welche Mischerschaltung unterdrückt am wirksamsten unerwünschte Mischprodukte und Frequenzen?}{Ein Eintakt-Transistormischer}
{Ein unbalancierter Produktdetektor}
{\textbf{\textcolor{DARCgreen}{Ein balancierter Ringmischer}}}
{Ein additiver Diodenmischer}
\end{QQuestion}

}
\end{frame}%ENDCONTENT
