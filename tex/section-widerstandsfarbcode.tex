
\section{Widerstandsfarbcode}
\label{section:widerstandsfarbcode}
\begin{frame}%STARTCONTENT

\begin{columns}
    \begin{column}{0.48\textwidth}
    
\begin{figure}
    \DARCimage{0.85\linewidth}{665include}
    \caption{\scriptsize  Ein Widerstand mit 4 Farbringen}
    \label{n_widerstandsfarbcodes}
\end{figure}


    \end{column}
   \begin{column}{0.48\textwidth}
       \begin{table}
\begin{DARCtabular}{Xlll}
    Farbe  &Wert  &Multiplikator  &Toleranz   \\
     Silber  & --  & 0,01  &  $\pm$ \qty{10}{\percent}   \\
     Gold  & --  & 0,1 &  $\pm$ \qty{5}{\percent}   \\
     Schwarz  & 0  & 1  & --   \\
     Braun  & 1  & 10  &  $\pm$ \qty{1}{\percent}   \\
     Rot  & 2  & 100  &  $\pm$ \qty{2}{\percent}   \\
     Orange & 3  & 1000  & --   \\
     Gelb  & 4  & 10000  & --   \\
     Grün  & 5  & 100000  & --   \\
     Blau  & 6  & 1000000  &  $\pm$ \qty{0,25}{\percent}  \\
     Violett  & 7  & 10000000  &  $\pm$ \qty{0,1}{\percent}  \\
     Grau  & 8  & 100000000  & --   \\
     Weiß  & 9  & 1000000000  & --   \\
     Keine  & --  & --  &  $\pm$ \qty{20}{\percent}  \\
\end{DARCtabular}
\caption{Widerstandsfarbcodes Tabelle}
\label{n_widerstandsfarbcodes_tabelle}
\end{table}

   \end{column}
\end{columns}

\end{frame}

\begin{frame}
\frametitle{Toleranz}
\begin{itemize}
  \item Abweichung vom tatsächlichen Wert
  \item Beispiel: silber bedeutet  $\pm$ \qty{10}{\percent}
  \item \qty{10}{\percent} von 47 kΩ = 4,7 kΩ
  \item Widerstandswert zwischen 42,3 kΩ und 51,7 kΩ
  \end{itemize}
\end{frame}

\begin{frame}
\only<1>{
\begin{QQuestion}{NC107}{Die Farbringe gelb, violett und orange auf einem Widerstand mit 4 Farbringen bedeuten einen Widerstandswert von~...}{\qty{4,7}{\kohm}.}
{\qty{47}{\kohm}.}
{\qty{470}{\kohm}.}
{\qty{4,7}{\Mohm}.}
\end{QQuestion}

}
\only<2>{
\begin{QQuestion}{NC107}{Die Farbringe gelb, violett und orange auf einem Widerstand mit 4 Farbringen bedeuten einen Widerstandswert von~...}{\qty{4,7}{\kohm}.}
{\textbf{\textcolor{DARCgreen}{\qty{47}{\kohm}.}}}
{\qty{470}{\kohm}.}
{\qty{4,7}{\Mohm}.}
\end{QQuestion}

}
\end{frame}

\begin{frame}
\only<1>{
\begin{QQuestion}{NC105}{Die Farbringe gelb, violett und rot auf einem Widerstand mit 4 Farbringen bedeuten einen Widerstandswert von~...}{\qty{4,7}{\Mohm}.}
{\qty{47}{\kohm}.}
{\qty{470}{\kohm}.}
{\qty{4,7}{\kohm}.}
\end{QQuestion}

}
\only<2>{
\begin{QQuestion}{NC105}{Die Farbringe gelb, violett und rot auf einem Widerstand mit 4 Farbringen bedeuten einen Widerstandswert von~...}{\qty{4,7}{\Mohm}.}
{\qty{47}{\kohm}.}
{\qty{470}{\kohm}.}
{\textbf{\textcolor{DARCgreen}{\qty{4,7}{\kohm}.}}}
\end{QQuestion}

}
\end{frame}

\begin{frame}
\only<1>{
\begin{QQuestion}{NC106}{Die Farbringe rot, violett und orange auf einem Widerstand mit 4 Farbringen bedeuten einen Widerstandswert von~...}{\qty{2,7}{\kohm}.}
{\qty{27}{\kohm}.}
{\qty{270}{\kohm}.}
{\qty{2,7}{\Mohm}.}
\end{QQuestion}

}
\only<2>{
\begin{QQuestion}{NC106}{Die Farbringe rot, violett und orange auf einem Widerstand mit 4 Farbringen bedeuten einen Widerstandswert von~...}{\qty{2,7}{\kohm}.}
{\textbf{\textcolor{DARCgreen}{\qty{27}{\kohm}.}}}
{\qty{270}{\kohm}.}
{\qty{2,7}{\Mohm}.}
\end{QQuestion}

}
\end{frame}

\begin{frame}
\only<1>{
\begin{QQuestion}{NC104}{Die Farbringe rot, violett und rot auf einem Widerstand mit 4 Farbringen bedeuten einen Widerstandswert von~...}{\qty{27}{\kohm}.}
{\qty{2,7}{\kohm}.}
{\qty{270}{\kohm}.}
{\qty{2,7}{\Mohm}.}
\end{QQuestion}

}
\only<2>{
\begin{QQuestion}{NC104}{Die Farbringe rot, violett und rot auf einem Widerstand mit 4 Farbringen bedeuten einen Widerstandswert von~...}{\qty{27}{\kohm}.}
{\textbf{\textcolor{DARCgreen}{\qty{2,7}{\kohm}.}}}
{\qty{270}{\kohm}.}
{\qty{2,7}{\Mohm}.}
\end{QQuestion}

}
\end{frame}

\begin{frame}
\only<1>{
\begin{QQuestion}{NC103}{Welche drei Farbringe hat ein \qty{1,2}{\kohm} Widerstand am Anfang, wenn vier Farbringe verwendet werden?}{Braun, rot, rot}
{Rot, orange, braun}
{Braun, rot, orange}
{Rot, braun, rot}
\end{QQuestion}

}
\only<2>{
\begin{QQuestion}{NC103}{Welche drei Farbringe hat ein \qty{1,2}{\kohm} Widerstand am Anfang, wenn vier Farbringe verwendet werden?}{\textbf{\textcolor{DARCgreen}{Braun, rot, rot}}}
{Rot, orange, braun}
{Braun, rot, orange}
{Rot, braun, rot}
\end{QQuestion}

}
\end{frame}

\begin{frame}
\only<1>{
\begin{QQuestion}{NC102}{Welchem Multiplikator entspricht ein grüner Farbring auf einem Widerstand mit 4 Farbringen?}{\num{10000}}
{\num{100000}}
{\num{1000000}}
{\num{10000000}}
\end{QQuestion}

}
\only<2>{
\begin{QQuestion}{NC102}{Welchem Multiplikator entspricht ein grüner Farbring auf einem Widerstand mit 4 Farbringen?}{\num{10000}}
{\textbf{\textcolor{DARCgreen}{\num{100000}}}}
{\num{1000000}}
{\num{10000000}}
\end{QQuestion}

}
\end{frame}

\begin{frame}
\only<1>{
\begin{QQuestion}{NC108}{Welche Toleranz weist ein Widerstand mit 4 Farbcodes auf, wenn der vierte Farbring ein silberner Farbring ist?}{ $\pm$\qty{5}{\percent}}
{ $\pm$\qty{10}{\percent}}
{ $\pm$\qty{0,1}{\percent}}
{ $\pm$\qty{1}{\percent}}
\end{QQuestion}

}
\only<2>{
\begin{QQuestion}{NC108}{Welche Toleranz weist ein Widerstand mit 4 Farbcodes auf, wenn der vierte Farbring ein silberner Farbring ist?}{ $\pm$\qty{5}{\percent}}
{\textbf{\textcolor{DARCgreen}{ $\pm$\qty{10}{\percent}}}}
{ $\pm$\qty{0,1}{\percent}}
{ $\pm$\qty{1}{\percent}}
\end{QQuestion}

}
\end{frame}

\begin{frame}
\only<1>{
\begin{QQuestion}{NC109}{Welche Toleranz weist ein Widerstand mit 4 Farbcodes auf, wenn der vierte Farbring ein goldener Farbring ist?}{ $\pm$\qty{0,5}{\percent}}
{ $\pm$\qty{5}{\percent}}
{ $\pm$\qty{0,1}{\percent}}
{ $\pm$\qty{1}{\percent}}
\end{QQuestion}

}
\only<2>{
\begin{QQuestion}{NC109}{Welche Toleranz weist ein Widerstand mit 4 Farbcodes auf, wenn der vierte Farbring ein goldener Farbring ist?}{ $\pm$\qty{0,5}{\percent}}
{\textbf{\textcolor{DARCgreen}{ $\pm$\qty{5}{\percent}}}}
{ $\pm$\qty{0,1}{\percent}}
{ $\pm$\qty{1}{\percent}}
\end{QQuestion}

}
\end{frame}

\begin{frame}
\only<1>{
\begin{QQuestion}{NC110}{Welche Toleranz weist ein Widerstand mit 4 Farbcodes auf, wenn der vierte Farbring braun ist?}{$\pm$\qty{5}{\percent}}
{$\pm$\qty{0,1}{\percent}}
{$\pm$\qty{1}{\percent}}
{$\pm$\qty{10}{\percent}}
\end{QQuestion}

}
\only<2>{
\begin{QQuestion}{NC110}{Welche Toleranz weist ein Widerstand mit 4 Farbcodes auf, wenn der vierte Farbring braun ist?}{$\pm$\qty{5}{\percent}}
{$\pm$\qty{0,1}{\percent}}
{\textbf{\textcolor{DARCgreen}{$\pm$\qty{1}{\percent}}}}
{$\pm$\qty{10}{\percent}}
\end{QQuestion}

}
\end{frame}%ENDCONTENT
