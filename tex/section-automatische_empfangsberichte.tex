
\section{Automatische Empfangsberichte}
\label{section:automatische_empfangsberichte}
\begin{frame}%STARTCONTENT
\begin{itemize}
  \item Mittels Digimodes empfangene Rufzeichen können an Plattformen geschickt werden
  \item Diese lassen sich auf einer Karte mit empfangenen Band darstellen
  \item Zum Testen der eigenen Ausbreitungsbedingungen
  \end{itemize}

\end{frame}

\begin{frame}
\frametitle{WSPR}
\begin{itemize}
  \item \emph{Weak Signal Progagation Reporter Network}
  \item QRP-Digimode, der rein zum Testen der eigenen Ausbreitungsbedingungen entwickelt wurde
  \item Es ist kein 2-Wege-QSO möglich
  \item Sehr langsame Übertragung mit hoher Fehlerkorrektur
  \item 1 Minute Senden, mehrere Minuten empfangen
  \item Ergebnisse werden an Server geschickt und lassen sich auf WSPRnet (\textcolor{DARCblue}{\faLink~\href{https://www.wsprnet.org/drupal/wsprnet/map}{www.wsprnet.org/drupal/wsprnet/map}}) darstellen
  \end{itemize}
\end{frame}

\begin{frame}
\only<1>{
\begin{QQuestion}{EE405}{Wie können Sie automatische Empfangsberichte zu Aussendungen erhalten, z.~B. um die Reichweite ihrer Sendeanlage zu testen?}{Durch Aussendung einer Nachricht mittels geeignetem digitalen Verfahren (z.~B. CW oder WSPR) und Suche nach Ihrem Rufzeichen auf passenden Internetplattformen}
{Durch Aussendung einer Nachricht mittels geeignetem digitalen Verfahren (z.~B. CW oder WSPR) unter Angabe Ihrer E-Mail-Adresse und der Anzahl der maximal gewünschten Empfangsberichte}
{Durch Aussendung Ihres Rufzeichens mittels Telegrafie (5~WPM) mit dem Zusatz \glqq AUTO RSVP\grqq{} (vom französischen \glqq répondez s'il vous pla\^it\grqq{}) und Abhören der \qty{10}{\kHz} höher gelegenen Frequenz}
{Durch Aussendung Ihres Rufzeichens mittels Telegrafie (12~WPM) mit dem Zusatz \glqq R\grqq{} (für Report) und Abhören der \qty{10}{\kHz} tiefer gelegenen Frequenz}
\end{QQuestion}

}
\only<2>{
\begin{QQuestion}{EE405}{Wie können Sie automatische Empfangsberichte zu Aussendungen erhalten, z.~B. um die Reichweite ihrer Sendeanlage zu testen?}{\textbf{\textcolor{DARCgreen}{Durch Aussendung einer Nachricht mittels geeignetem digitalen Verfahren (z.~B. CW oder WSPR) und Suche nach Ihrem Rufzeichen auf passenden Internetplattformen}}}
{Durch Aussendung einer Nachricht mittels geeignetem digitalen Verfahren (z.~B. CW oder WSPR) unter Angabe Ihrer E-Mail-Adresse und der Anzahl der maximal gewünschten Empfangsberichte}
{Durch Aussendung Ihres Rufzeichens mittels Telegrafie (5~WPM) mit dem Zusatz \glqq AUTO RSVP\grqq{} (vom französischen \glqq répondez s'il vous pla\^it\grqq{}) und Abhören der \qty{10}{\kHz} höher gelegenen Frequenz}
{Durch Aussendung Ihres Rufzeichens mittels Telegrafie (12~WPM) mit dem Zusatz \glqq R\grqq{} (für Report) und Abhören der \qty{10}{\kHz} tiefer gelegenen Frequenz}
\end{QQuestion}

}
\end{frame}%ENDCONTENT
