
\section{Aurora II}
\label{section:aurora_2}
\begin{frame}%STARTCONTENT

\frametitle{Foliensatz in Arbeit}
2024-04-28: Die Inhalte werden noch aufbereitet.

Derzeit sind in diesem Abschnitt nur die Fragen sortiert enthalten.

Für das Selbststudium verweisen wir aktuell auf den Abschnitt Wellenausbreitung im DARC Online Lehrgang (\textcolor{DARCblue}{\faLink~\href{https://www.darc.de/der-club/referate/ajw/lehrgang-te/e09/}{www.darc.de/der-club/referate/ajw/lehrgang-te/e09/}}) für die Prüfung bis Juni 2024. Bis auf die Fragen hat sich an der Thematik nichts geändert. Das Thema war bisher Stoff der Klasse~E und wurde mit der neuen Prüfungsordnung auf alle drei Klassen aufgeteilt.

\end{frame}

\begin{frame}
\frametitle{Auftreten von Aurora}

\end{frame}

\begin{frame}
\only<1>{
\begin{QQuestion}{AH302}{In welchem ionosphärischen Bereich treten gelegentlich Aurora-Erscheinungen auf?}{In der E-Region in der Nähe der Pole}
{In der F-Region}
{In der E-Region in der Nähe des Äquators.}
{In der D-Region}
\end{QQuestion}

}
\only<2>{
\begin{QQuestion}{AH302}{In welchem ionosphärischen Bereich treten gelegentlich Aurora-Erscheinungen auf?}{\textbf{\textcolor{DARCgreen}{In der E-Region in der Nähe der Pole}}}
{In der F-Region}
{In der E-Region in der Nähe des Äquators.}
{In der D-Region}
\end{QQuestion}

}
\end{frame}

\begin{frame}
\only<1>{
\begin{QQuestion}{AH303}{Was ist die Ursache für Aurora-Erscheinungen?}{Das Eindringen geladener Teilchen von der Sonne in die Atmosphäre der Polarregionen.}
{Eine hohe Sonnenfleckenzahl.}
{Eine niedrige Sonnenfleckenzahl.}
{Das Eindringen starker Meteoritenschauer in die Atmosphäre der Polarregionen.}
\end{QQuestion}

}
\only<2>{
\begin{QQuestion}{AH303}{Was ist die Ursache für Aurora-Erscheinungen?}{\textbf{\textcolor{DARCgreen}{Das Eindringen geladener Teilchen von der Sonne in die Atmosphäre der Polarregionen.}}}
{Eine hohe Sonnenfleckenzahl.}
{Eine niedrige Sonnenfleckenzahl.}
{Das Eindringen starker Meteoritenschauer in die Atmosphäre der Polarregionen.}
\end{QQuestion}

}
\end{frame}

\begin{frame}
\only<1>{
\begin{QQuestion}{AH306}{In welche Himmelsrichtung muss eine Funkstation in Europa ihre VHF-Antenne drehen, um eine Verbindung über \glqq Aurora\grqq{} abzuwickeln?}{Süden}
{Norden}
{Osten}
{Westen}
\end{QQuestion}

}
\only<2>{
\begin{QQuestion}{AH306}{In welche Himmelsrichtung muss eine Funkstation in Europa ihre VHF-Antenne drehen, um eine Verbindung über \glqq Aurora\grqq{} abzuwickeln?}{Süden}
{\textbf{\textcolor{DARCgreen}{Norden}}}
{Osten}
{Westen}
\end{QQuestion}

}
\end{frame}

\begin{frame}
\frametitle{Nutzung für Wellenausbreitung}
\end{frame}

\begin{frame}
\only<1>{
\begin{QQuestion}{AH304}{Beim Auftreten von Polarlichtern lassen sich auf den Amateurfunkbändern über \qty{30}{\MHz} beträchtliche Überreichweiten erzielen, weil~...}{stark ionisierte Bereiche auftreten, die Funkwellen reflektieren.}
{starke Magnetfelder auftreten, die Funkwellen reflektieren.}
{starke Inversionsfelder auftreten, die Funkwellen reflektieren.}
{starke sporadische D-Regionen auftreten, die Funkwellen reflektieren.}
\end{QQuestion}

}
\only<2>{
\begin{QQuestion}{AH304}{Beim Auftreten von Polarlichtern lassen sich auf den Amateurfunkbändern über \qty{30}{\MHz} beträchtliche Überreichweiten erzielen, weil~...}{\textbf{\textcolor{DARCgreen}{stark ionisierte Bereiche auftreten, die Funkwellen reflektieren.}}}
{starke Magnetfelder auftreten, die Funkwellen reflektieren.}
{starke Inversionsfelder auftreten, die Funkwellen reflektieren.}
{starke sporadische D-Regionen auftreten, die Funkwellen reflektieren.}
\end{QQuestion}

}
\end{frame}

\begin{frame}
\only<1>{
\begin{QQuestion}{AH305}{Was meint ein Funkamateur damit, wenn er angibt, dass er auf dem \qty{2}{\m}-Band eine Aurora-Verbindung mit Schottland gehabt hat?}{Die Verbindung ist durch Verstärkung der polaren Nordlichter mittels Ultrakurzwellen zustande gekommen (Reflexion von ionisiertem Polarlicht).}
{Die Verbindung ist durch Beugung von Ultrakurzwellen an Lichtquellen der Polarregion zustande gekommen (Beugung an ionisierten Polarschichten).}
{Die Verbindung ist durch Reflexion von Ultrakurzwellen an polaren Nordlichtern zustande gekommen (Reflexion an polaren Ionisationserscheinungen).}
{Die Verbindung ist durch Reflexion von verbrummten Ultrakurzwellen am Polarkreis zustande gekommen (Reflexion an Ionisationserscheinungen des Polarkreises).}
\end{QQuestion}

}
\only<2>{
\begin{QQuestion}{AH305}{Was meint ein Funkamateur damit, wenn er angibt, dass er auf dem \qty{2}{\m}-Band eine Aurora-Verbindung mit Schottland gehabt hat?}{Die Verbindung ist durch Verstärkung der polaren Nordlichter mittels Ultrakurzwellen zustande gekommen (Reflexion von ionisiertem Polarlicht).}
{Die Verbindung ist durch Beugung von Ultrakurzwellen an Lichtquellen der Polarregion zustande gekommen (Beugung an ionisierten Polarschichten).}
{\textbf{\textcolor{DARCgreen}{Die Verbindung ist durch Reflexion von Ultrakurzwellen an polaren Nordlichtern zustande gekommen (Reflexion an polaren Ionisationserscheinungen).}}}
{Die Verbindung ist durch Reflexion von verbrummten Ultrakurzwellen am Polarkreis zustande gekommen (Reflexion an Ionisationserscheinungen des Polarkreises).}
\end{QQuestion}

}
\end{frame}

\begin{frame}
\only<1>{
\begin{QQuestion}{AH307}{Welches der folgenden Übertragungsverfahren eignet sich am besten für Auroraverbindungen?}{SSB}
{CW}
{FM}
{RTTY}
\end{QQuestion}

}
\only<2>{
\begin{QQuestion}{AH307}{Welches der folgenden Übertragungsverfahren eignet sich am besten für Auroraverbindungen?}{SSB}
{\textbf{\textcolor{DARCgreen}{CW}}}
{FM}
{RTTY}
\end{QQuestion}

}
\end{frame}

\begin{frame}
\only<1>{
\begin{QQuestion}{AH308}{Wie wirkt sich \glqq Aurora\grqq{} auf die Signalqualität eines Funksignals aus?}{CW-Signale haben einen flatternden und verbrummten Ton.}
{CW-Signale haben einen besseren Ton.}
{Die Lesbarkeit von Fonie-Signalen verbessert sich.}
{CW- und Fonie-Signale haben ein Echo.}
\end{QQuestion}

}
\only<2>{
\begin{QQuestion}{AH308}{Wie wirkt sich \glqq Aurora\grqq{} auf die Signalqualität eines Funksignals aus?}{\textbf{\textcolor{DARCgreen}{CW-Signale haben einen flatternden und verbrummten Ton.}}}
{CW-Signale haben einen besseren Ton.}
{Die Lesbarkeit von Fonie-Signalen verbessert sich.}
{CW- und Fonie-Signale haben ein Echo.}
\end{QQuestion}

}
\end{frame}%ENDCONTENT
