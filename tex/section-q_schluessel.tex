
\section{Q-Schlüssel}
\label{section:q_schluessel}
\begin{frame}%STARTCONTENT

\frametitle{Q-Gruppen}
\begin{columns}
    \begin{column}{0.48\textwidth}
    \begin{itemize}
  \item In Radio Regulations (RR) definiert
  \item 3 Zeichen lang
  \item Nutzung als Frage, Antwort oder Aufforderung
  \end{itemize}

    \end{column}
   \begin{column}{0.48\textwidth}
       
    \pause
    Wird ein Fragezeichen angehangen, wird aus einer Aussage eine Frage:

QTH Berlin $\rightarrow$ QTH?




   \end{column}
\end{columns}

\end{frame}

\begin{frame}
\only<1>{
\begin{QQuestion}{VA407}{In welchem internationalen Regelwerk können die Bedeutungen der im Amateurfunk gebräuchlichen Q-Gruppen nachgeschlagen werden? In den~...}{Empfehlungen der Internationalen Organisation für Normung (ISO)}
{Empfehlungen der International Amateur Radio Union (IARU)}
{Standards des European Telecommunications Standards Institute (ETSI)}
{Radio Regulations (RR)}
\end{QQuestion}

}
\only<2>{
\begin{QQuestion}{VA407}{In welchem internationalen Regelwerk können die Bedeutungen der im Amateurfunk gebräuchlichen Q-Gruppen nachgeschlagen werden? In den~...}{Empfehlungen der Internationalen Organisation für Normung (ISO)}
{Empfehlungen der International Amateur Radio Union (IARU)}
{Standards des European Telecommunications Standards Institute (ETSI)}
{\textbf{\textcolor{DARCgreen}{Radio Regulations (RR)}}}
\end{QQuestion}

}
\end{frame}

\begin{frame}\begin{table}
\begin{DARCtabular}{lll}
      & Bedeutung  & Merkhilfe   \\
     QRN  & Atmosphärische Störungen  & \emph{N}atürliche Störung   \\
     QRM  & Ich werde gestört  & \emph{M}enschengemachte Störung   \\
     QRO  & Erhöhen Sie die Sendeleistung!  & Ein paar Watt \emph{o}bendrauf legen   \\
     QRP  & Senken Sie die Sendeleistung!  & \emph{P}iano (sanft, leise), \emph{P}ssst!  \\
     QRT  & Stellen Sie die Übermittlung ein!  & \emph{T}erminal (Ende)   \\
     QRV  & Ich bin bereit  & \emph{v}orbereitet   \\
\end{DARCtabular}
\caption{Alle prüfungsrelevanten Q-Gruppen in der Übersicht mit Merkhilfen}
\label{n_q_gruppen_1}
\end{table}

\end{frame}

\begin{frame}\begin{table}
\begin{DARCtabular}{lll}
      & Bedeutung  & Merkhilfe   \\
     QRX?  & Wann rufen Sie mich wieder?  & Zeitpunkt \emph{X}   \\
     QRZ?  & Wer hat mich gerufen?  & Bitte nenne das Rufzeichen ein \emph{z}weites Mal!   \\
     QSB  & Fading, Schwankungen  & \emph{B}ergauf, \emph{B}ergab   \\
     QSL  & Ich bestätige den Empfang  & Ich habe ge\emph{l}oggt   \\
     QSO?  & Erreichen Sie Station~...~?  &    \\
     QSY  & Frequenzwechsel  & Change Frequenc\emph{y}   \\
     QTH  & Mein Standort  & \emph{H}ome, \emph{H}eimat   \\
\end{DARCtabular}
\caption{Alle prüfungsrelevanten Q-Gruppen in der Übersicht mit Merkhilfen}
\label{n_q_gruppen_2}
\end{table}

\end{frame}

\begin{frame}
\only<1>{
\begin{QQuestion}{BB204}{Was bedeuten die Q-Gruppen \glqq QRV\grqq{}, \glqq QRM?\grqq{} und \glqq QTH\grqq{}?}{Ich habe nichts mehr für Sie. Werden Sie gestört? Mein Standort ist...}
{Senden Sie eine Reihe V. Soll ich mehr Sendeleistung anwenden? Ihre gesendeten Töne sind kaum hörbar.}
{Ich bin bereit. Werden Sie gestört? Mein Standort ist...}
{Ich habe nichts mehr für Sie. Mein Standort ist... Ich bin bereit.}
\end{QQuestion}

}
\only<2>{
\begin{QQuestion}{BB204}{Was bedeuten die Q-Gruppen \glqq QRV\grqq{}, \glqq QRM?\grqq{} und \glqq QTH\grqq{}?}{Ich habe nichts mehr für Sie. Werden Sie gestört? Mein Standort ist...}
{Senden Sie eine Reihe V. Soll ich mehr Sendeleistung anwenden? Ihre gesendeten Töne sind kaum hörbar.}
{\textbf{\textcolor{DARCgreen}{Ich bin bereit. Werden Sie gestört? Mein Standort ist...}}}
{Ich habe nichts mehr für Sie. Mein Standort ist... Ich bin bereit.}
\end{QQuestion}

}
\end{frame}

\begin{frame}
\only<1>{
\begin{QQuestion}{BB203}{Was bedeuten die Q-Gruppen \glqq QRT\grqq{}, \glqq QRZ?\grqq{} und \glqq QSL?\grqq{}?}{Ich habe nichts für Sie. Von wem werde ich gerufen? Können Sie den Empfang bestätigen??}
{Stellen Sie die Übermittlung ein. Ich bin bereit. Schicken Sie eine QSL-Karte?}
{Stellen Sie die Übermittlung ein. Wie ist Ihr Standort? Können Sie den Empfang bestätigen??}
{Stellen Sie die Übermittlung ein. Von wem werde ich gerufen? Können Sie den Empfang bestätigen??}
\end{QQuestion}

}
\only<2>{
\begin{QQuestion}{BB203}{Was bedeuten die Q-Gruppen \glqq QRT\grqq{}, \glqq QRZ?\grqq{} und \glqq QSL?\grqq{}?}{Ich habe nichts für Sie. Von wem werde ich gerufen? Können Sie den Empfang bestätigen??}
{Stellen Sie die Übermittlung ein. Ich bin bereit. Schicken Sie eine QSL-Karte?}
{Stellen Sie die Übermittlung ein. Wie ist Ihr Standort? Können Sie den Empfang bestätigen??}
{\textbf{\textcolor{DARCgreen}{Stellen Sie die Übermittlung ein. Von wem werde ich gerufen? Können Sie den Empfang bestätigen??}}}
\end{QQuestion}

}
\end{frame}

\begin{frame}
\only<1>{
\begin{QQuestion}{BE115}{Was bedeutet die Betriebsabkürzung \glqq QRZ?\grqq{} im Amateurfunk?}{Von wem werde ich gerufen? In Pile-Ups auch: Aufruf weiterer Stationen}
{Sind Sie bereit? Im Contest auch: Alles aufgenommen?}
{Wie ist Ihr Standort? Im Contest auch: Wie ist Ihre Entfernung?}
{Können Sie den Empfang bestätigen? In Pile-Ups auch: Aufforderung zum Rapport}
\end{QQuestion}

}
\only<2>{
\begin{QQuestion}{BE115}{Was bedeutet die Betriebsabkürzung \glqq QRZ?\grqq{} im Amateurfunk?}{\textbf{\textcolor{DARCgreen}{Von wem werde ich gerufen? In Pile-Ups auch: Aufruf weiterer Stationen}}}
{Sind Sie bereit? Im Contest auch: Alles aufgenommen?}
{Wie ist Ihr Standort? Im Contest auch: Wie ist Ihre Entfernung?}
{Können Sie den Empfang bestätigen? In Pile-Ups auch: Aufforderung zum Rapport}
\end{QQuestion}

}
\end{frame}

\begin{frame}
\only<1>{
\begin{QQuestion}{BB202}{Was bedeuten die Q-Gruppen \glqq QRO?\grqq{}, \glqq QSO?\grqq{} und \glqq QRX?\grqq{}?}{Soll ich die Sendeleistung erhöhen? Können Sie direkt Funkverkehr aufnehmen mit ...? Wann werden Sie mich wieder rufen?}
{Soll ich meine Sendeleistung erhöhen? Haben Sie noch etwas für mich vorliegen? Werden Sie mich wieder rufen?}
{Soll ich die Sendeleistung verringern? Haben Sie noch etwas für mich vorliegen? Können Sie direkt Funkverkehr aufnehmen mit ...?}
{Haben Sie noch etwas für mich vorliegen? Können Sie direkt Funkverkehr aufnehmen mit ...? Wann werden Sie mich wieder rufen?}
\end{QQuestion}

}
\only<2>{
\begin{QQuestion}{BB202}{Was bedeuten die Q-Gruppen \glqq QRO?\grqq{}, \glqq QSO?\grqq{} und \glqq QRX?\grqq{}?}{\textbf{\textcolor{DARCgreen}{Soll ich die Sendeleistung erhöhen? Können Sie direkt Funkverkehr aufnehmen mit ...? Wann werden Sie mich wieder rufen?}}}
{Soll ich meine Sendeleistung erhöhen? Haben Sie noch etwas für mich vorliegen? Werden Sie mich wieder rufen?}
{Soll ich die Sendeleistung verringern? Haben Sie noch etwas für mich vorliegen? Können Sie direkt Funkverkehr aufnehmen mit ...?}
{Haben Sie noch etwas für mich vorliegen? Können Sie direkt Funkverkehr aufnehmen mit ...? Wann werden Sie mich wieder rufen?}
\end{QQuestion}

}
\end{frame}

\begin{frame}
\only<1>{
\begin{QQuestion}{BB205}{Wie verhalten Sie sich, wenn Sie \glqq PSE QRP\grqq{} aufnehmen?}{Sie senden eine Bestätigungskarte an die Gegenstation.}
{Sie erhöhen die Sendeleistung.}
{Sie wechseln die Frequenz.}
{Sie verringern die Sendeleistung.}
\end{QQuestion}

}
\only<2>{
\begin{QQuestion}{BB205}{Wie verhalten Sie sich, wenn Sie \glqq PSE QRP\grqq{} aufnehmen?}{Sie senden eine Bestätigungskarte an die Gegenstation.}
{Sie erhöhen die Sendeleistung.}
{Sie wechseln die Frequenz.}
{\textbf{\textcolor{DARCgreen}{Sie verringern die Sendeleistung.}}}
\end{QQuestion}

}
\end{frame}

\begin{frame}
\only<1>{
\begin{QQuestion}{BB201}{Was bedeuten die Q-Gruppen \glqq QRM\grqq{}, \glqq QRN\grqq{} und \glqq QSB?\grqq{}?}{Ich werde gestört. Ich habe atmosphärische Störungen. Schwankt die Stärke meiner Zeichen?}
{Ich habe Störungen. Sie haben Schwankungen Ihrer Zeichen. Werden Sie gestört?}
{Ich habe atmosphärische Störungen. Ich werde gestört. Schwankt die Stärke meiner Zeichen?}
{Die Stärke Ihrer Zeichen schwankt. Ich werde gestört. Haben Sie atmosphärische Störungen?}
\end{QQuestion}

}
\only<2>{
\begin{QQuestion}{BB201}{Was bedeuten die Q-Gruppen \glqq QRM\grqq{}, \glqq QRN\grqq{} und \glqq QSB?\grqq{}?}{\textbf{\textcolor{DARCgreen}{Ich werde gestört. Ich habe atmosphärische Störungen. Schwankt die Stärke meiner Zeichen?}}}
{Ich habe Störungen. Sie haben Schwankungen Ihrer Zeichen. Werden Sie gestört?}
{Ich habe atmosphärische Störungen. Ich werde gestört. Schwankt die Stärke meiner Zeichen?}
{Die Stärke Ihrer Zeichen schwankt. Ich werde gestört. Haben Sie atmosphärische Störungen?}
\end{QQuestion}

}
\end{frame}

\begin{frame}
\only<1>{
\begin{QQuestion}{BB206}{Wie verhalten Sie sich, wenn Sie \glqq PSE QSY ...\grqq{} aufnehmen?}{Sie erhöhen die Sendeleistung.}
{Sie wechseln die Frequenz.}
{Sie verringern die Sendeleistung.}
{Sie senden eine Bestätigungskarte an die Gegenstation.}
\end{QQuestion}

}
\only<2>{
\begin{QQuestion}{BB206}{Wie verhalten Sie sich, wenn Sie \glqq PSE QSY ...\grqq{} aufnehmen?}{Sie erhöhen die Sendeleistung.}
{\textbf{\textcolor{DARCgreen}{Sie wechseln die Frequenz.}}}
{Sie verringern die Sendeleistung.}
{Sie senden eine Bestätigungskarte an die Gegenstation.}
\end{QQuestion}

}
\end{frame}

\begin{frame}
\only<1>{
\begin{QQuestion}{BE107}{Sie tätigen einen allgemeinen Anruf in Telefonie auf \qty{145,500}{MHz}. Dieser wird von einer Gegenstation beantwortet. Wie sollten Sie das darauffolgende Funkgespräch fortsetzen? Ich fasse mich kurz und schlage~...}{QSY vor.}
{QRV vor.}
{QSB vor.}
{QRZ vor.}
\end{QQuestion}

}
\only<2>{
\begin{QQuestion}{BE107}{Sie tätigen einen allgemeinen Anruf in Telefonie auf \qty{145,500}{MHz}. Dieser wird von einer Gegenstation beantwortet. Wie sollten Sie das darauffolgende Funkgespräch fortsetzen? Ich fasse mich kurz und schlage~...}{\textbf{\textcolor{DARCgreen}{QSY vor.}}}
{QRV vor.}
{QSB vor.}
{QRZ vor.}
\end{QQuestion}

}
\end{frame}%ENDCONTENT
