
\section{Quellencodierung}
\label{section:quellencodierung}
\begin{frame}%STARTCONTENT
\begin{itemize}
  \item Bei der digitalen Übertragung möchte man das Frequenzspektrum möglichst effizient nutzen.
  \item Dies erreicht man durch die Kompression der Nutzdaten, die sogenannte Quellencodierung.
  \item Dabei werden Redundanzen (z. B. Wiederholungen) oder Irrelevanzen (weniger wichtige Informationsteile) aus dem Datenstrom entfernt.
  \end{itemize}

\begin{figure}
    \DARCimage{0.85\linewidth}{675include}
    \caption{\scriptsize Quellencodierer}
    \label{quellencodierer}
\end{figure}

\end{frame}

\begin{frame}
\only<1>{
\begin{QQuestion}{AE408}{Wodurch kann die Datenmenge einer zu übertragenden Nachricht reduziert werden?}{Kanalcodierung}
{Quellencodierung}
{Synchronisation}
{Mehrfachzugriff}
\end{QQuestion}

}
\only<2>{
\begin{QQuestion}{AE408}{Wodurch kann die Datenmenge einer zu übertragenden Nachricht reduziert werden?}{Kanalcodierung}
{\textbf{\textcolor{DARCgreen}{Quellencodierung}}}
{Synchronisation}
{Mehrfachzugriff}
\end{QQuestion}

}
\end{frame}%ENDCONTENT
