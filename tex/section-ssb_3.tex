
\section{Einseitenbandmodulation (SSB) III}
\label{section:ssb_3}
\begin{frame}%STARTCONTENT

\only<1>{
\begin{QQuestion}{AE205}{Ein übermoduliertes SSB-Sendesignal führt zu~...}{Splatter-Erscheinungen.}
{Kreuzmodulation.}
{verminderten Seitenbändern.}
{überhöhtem Hub.}
\end{QQuestion}

}
\only<2>{
\begin{QQuestion}{AE205}{Ein übermoduliertes SSB-Sendesignal führt zu~...}{\textbf{\textcolor{DARCgreen}{Splatter-Erscheinungen.}}}
{Kreuzmodulation.}
{verminderten Seitenbändern.}
{überhöhtem Hub.}
\end{QQuestion}

}
\end{frame}

\begin{frame}
\only<1>{
\begin{PQuestion}{AE207}{Das folgende Oszillogramm zeigt~...}{ein typisches \qty{100}{\percent}-AM-Signal.}
{ein typisches Einton-FM-Testsignal.}
{ein typisches Zweiton-SSB-Testsignal.}
{ein typisches CW-Signal.}
{\DARCimage{1.0\linewidth}{43include}}\end{PQuestion}

}
\only<2>{
\begin{PQuestion}{AE207}{Das folgende Oszillogramm zeigt~...}{ein typisches \qty{100}{\percent}-AM-Signal.}
{ein typisches Einton-FM-Testsignal.}
{\textbf{\textcolor{DARCgreen}{ein typisches Zweiton-SSB-Testsignal.}}}
{ein typisches CW-Signal.}
{\DARCimage{1.0\linewidth}{43include}}\end{PQuestion}

}
\end{frame}

\begin{frame}
\only<1>{
\begin{QQuestion}{AE208}{Um Bandbreite einzusparen, sollte der Frequenzumfang eines NF-Sprachsignals, das an einen SSB-Modulator angelegt wird,~...}{\qty{1,8}{\kHz} nicht überschreiten.}
{\qty{2,7}{\kHz} nicht überschreiten.}
{\qty{800}{\Hz} nicht überschreiten.}
{\qty{15}{\kHz} nicht überschreiten.}
\end{QQuestion}

}
\only<2>{
\begin{QQuestion}{AE208}{Um Bandbreite einzusparen, sollte der Frequenzumfang eines NF-Sprachsignals, das an einen SSB-Modulator angelegt wird,~...}{\qty{1,8}{\kHz} nicht überschreiten.}
{\textbf{\textcolor{DARCgreen}{\qty{2,7}{\kHz} nicht überschreiten.}}}
{\qty{800}{\Hz} nicht überschreiten.}
{\qty{15}{\kHz} nicht überschreiten.}
\end{QQuestion}

}
\end{frame}

\begin{frame}
\only<1>{
\begin{QQuestion}{AE209}{Wie groß sollte der Abstand der Sendefrequenz zwischen zwei SSB-Signalen sein, um gegenseitige Störungen in SSB-Telefonie auf ein Mindestmaß zu begrenzen?}{\qty{25}{\kHz}}
{\qty{12,5}{\kHz}}
{\qty{3}{\kHz}}
{\qty{455}{\kHz}}
\end{QQuestion}

}
\only<2>{
\begin{QQuestion}{AE209}{Wie groß sollte der Abstand der Sendefrequenz zwischen zwei SSB-Signalen sein, um gegenseitige Störungen in SSB-Telefonie auf ein Mindestmaß zu begrenzen?}{\qty{25}{\kHz}}
{\qty{12,5}{\kHz}}
{\textbf{\textcolor{DARCgreen}{\qty{3}{\kHz}}}}
{\qty{455}{\kHz}}
\end{QQuestion}

}
\end{frame}

\begin{frame}
\only<1>{
\begin{QQuestion}{AE213}{Welche Aufgabe hat der Equalizer in einem SSB-Sender?}{Er dient zur Anpassung des Mikrofonfrequenzgangs an den Operator.}
{Er dient zur Erzeugung des SSB-Signals.}
{Er dient zur Unterdrückung von Oberschwingungen im Sendesignal.}
{Er dient zur Erhöhung der Trägerunterdrückung.}
\end{QQuestion}

}
\only<2>{
\begin{QQuestion}{AE213}{Welche Aufgabe hat der Equalizer in einem SSB-Sender?}{\textbf{\textcolor{DARCgreen}{Er dient zur Anpassung des Mikrofonfrequenzgangs an den Operator.}}}
{Er dient zur Erzeugung des SSB-Signals.}
{Er dient zur Unterdrückung von Oberschwingungen im Sendesignal.}
{Er dient zur Erhöhung der Trägerunterdrückung.}
\end{QQuestion}

}
\end{frame}%ENDCONTENT
