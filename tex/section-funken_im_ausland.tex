
\section{Funken im Ausland}
\label{section:funken_im_ausland}
\begin{frame}%STARTCONTENT

\begin{columns}
    \begin{column}{0.48\textwidth}
    \begin{itemize}
  \item Funkbetrieb im Ausland unter bestimmten Voraussetzungen möglich
  \item Abkommen zwischen vielen Staaten
  \end{itemize}

    \end{column}
   \begin{column}{0.48\textwidth}
       \begin{itemize}
  \item Beim vorübergehenden Aufenthalt
  \item Gegenseitige Anerkennung von Amateurfunkzeugnissen
  \end{itemize}

   \end{column}
\end{columns}

\end{frame}

\begin{frame}
\begin{columns}
    \begin{column}{0.48\textwidth}
    \begin{itemize}
  \item Mitgliedsstaaten der \emph{Europäische Konferenz der Verwaltungen für Post und Telekommunikation} (Conférence Européenne des Administrations des Postes et des Télécommunications, CEPT)
  \item Mehrere Empfehlungen in der Tabelle rechts
  \end{itemize}

    \end{column}
   \begin{column}{0.48\textwidth}
       \begin{table}
\begin{DARCtabular}{ll}
     CEPT-Empfehlung  & Erläuterung   \\
     ECC Report 89  & Klasse~N   \\
     ERC-Report 32  & Grundlage für ECC Report (05) 06   \\
     ECC Report (05) 06  & Klasse~E   \\
     T/R 61-01  & Klasse~A   \\
     T/R 61-02  & HAREC   \\
\end{DARCtabular}
\caption{CEPT-Empfehlungen}
\label{n_funken_im_ausland_cept_empfehlungen}
\end{table}

   \end{column}
\end{columns}

\end{frame}

\begin{frame}
\only<1>{
\begin{QQuestion}{VB104}{Welche Bedeutung haben die CEPT-Empfehlungen T/R 61-01 und T/R 61-02 sowie der ERC-Report 32 und die ECC-Empfehlung (05)06 für den Amateurfunk? Sie bilden die Grundlage für...}{den grenzüberschreitenden Warenverkehr von Amateurfunkgeräten in der Europäischen Union und weiteren umsetzenden Ländern.}
{den Funkverkehr zwischen den umsetzenden Ländern und die Harmonisierung der nationalen Frequenzzuweisungen für den Amateurfunkdienst.}
{den Amateurfunkverkehr in den umsetzenden Ländern und die weltweite Anerkennung von Amateurfunkzeugnissen.}
{den vorübergehenden Amateurfunkbetrieb und die gegenseitige Anerkennung von Amateurfunkzeugnissen in den umsetzenden Ländern.}
\end{QQuestion}

}
\only<2>{
\begin{QQuestion}{VB104}{Welche Bedeutung haben die CEPT-Empfehlungen T/R 61-01 und T/R 61-02 sowie der ERC-Report 32 und die ECC-Empfehlung (05)06 für den Amateurfunk? Sie bilden die Grundlage für...}{den grenzüberschreitenden Warenverkehr von Amateurfunkgeräten in der Europäischen Union und weiteren umsetzenden Ländern.}
{den Funkverkehr zwischen den umsetzenden Ländern und die Harmonisierung der nationalen Frequenzzuweisungen für den Amateurfunkdienst.}
{den Amateurfunkverkehr in den umsetzenden Ländern und die weltweite Anerkennung von Amateurfunkzeugnissen.}
{\textbf{\textcolor{DARCgreen}{den vorübergehenden Amateurfunkbetrieb und die gegenseitige Anerkennung von Amateurfunkzeugnissen in den umsetzenden Ländern.}}}
\end{QQuestion}

}
\end{frame}

\begin{frame}
\frametitle{Vorübergehender Amateurfunkbetrieb im Ausland}
\begin{itemize}
  \item Darf nur in Staaten durchgeführt werden, die die CEPT-Regelungen anwenden
  \item Kein fester Wohnsitz
  \item Vorübergehend bis 3 Monate
  \end{itemize}

\end{frame}

\begin{frame}
\only<1>{
\begin{QQuestion}{VB106}{Darf ein Funkamateur mit einer \glqq CEPT-Novice-Amateurfunkgenehmigung\grqq{} in allen CEPT-Ländern Amateurfunkverkehr abwickeln?}{Nein, nur in Ländern, die die ECC-Empfehlung (05)06 umgesetzt haben, sofern er dort keinen festen Wohnsitz hat.}
{Ja, alle CEPT-Mitgliedsländer müssen die ECC-Empfehlung (05)06 anwenden.}
{Nein, die Anwendung der ECC-Empfehlung~(05)06 ist nur in den Mitgliedsstaaten der Europäischen Union zulässig.}
{Ja, er muss sich aber an die Amateurfunkregelungen des Heimatlandes halten.}
\end{QQuestion}

}
\only<2>{
\begin{QQuestion}{VB106}{Darf ein Funkamateur mit einer \glqq CEPT-Novice-Amateurfunkgenehmigung\grqq{} in allen CEPT-Ländern Amateurfunkverkehr abwickeln?}{\textbf{\textcolor{DARCgreen}{Nein, nur in Ländern, die die ECC-Empfehlung (05)06 umgesetzt haben, sofern er dort keinen festen Wohnsitz hat.}}}
{Ja, alle CEPT-Mitgliedsländer müssen die ECC-Empfehlung (05)06 anwenden.}
{Nein, die Anwendung der ECC-Empfehlung~(05)06 ist nur in den Mitgliedsstaaten der Europäischen Union zulässig.}
{Ja, er muss sich aber an die Amateurfunkregelungen des Heimatlandes halten.}
\end{QQuestion}

}
\end{frame}

\begin{frame}
\only<1>{
\begin{QQuestion}{VB107}{Darf ein deutscher Funkamateur mit einer Amateurfunkzulassung der Klasse~A in allen CEPT-Ländern Amateurfunkverkehr abwickeln?}{Ja, alle CEPT-Mitgliedsländer müssen die Empfehlung T/R 61-01 anwenden.}
{Nein, nur in den Ländern, die die Empfehlung T/R 61-01 umgesetzt haben, sofern er dort keinen festen Wohnsitz hat.}
{Nein, die Anwendung der CEPT-Empfehlung T/R~61-01 ist nur in den Mitgliedsstaaten der Europäischen Union zulässig.}
{Ja, er muss sich aber an die Amateurfunkregelungen des Heimatlandes halten.}
\end{QQuestion}

}
\only<2>{
\begin{QQuestion}{VB107}{Darf ein deutscher Funkamateur mit einer Amateurfunkzulassung der Klasse~A in allen CEPT-Ländern Amateurfunkverkehr abwickeln?}{Ja, alle CEPT-Mitgliedsländer müssen die Empfehlung T/R 61-01 anwenden.}
{\textbf{\textcolor{DARCgreen}{Nein, nur in den Ländern, die die Empfehlung T/R 61-01 umgesetzt haben, sofern er dort keinen festen Wohnsitz hat.}}}
{Nein, die Anwendung der CEPT-Empfehlung T/R~61-01 ist nur in den Mitgliedsstaaten der Europäischen Union zulässig.}
{Ja, er muss sich aber an die Amateurfunkregelungen des Heimatlandes halten.}
\end{QQuestion}

}
\end{frame}

\begin{frame}
\only<1>{
\begin{QQuestion}{VB109}{Wie lange darf ein Funkamateur vorübergehend Amateurfunkverkehr im Ausland je Aufenthalt durchführen, wenn die CEPT-Regelung Anwendung findet?}{Bis zu 9~Monaten}
{Bis zu 3~Monaten}
{Bis zu 6~Monaten}
{Bis zu einem Jahr}
\end{QQuestion}

}
\only<2>{
\begin{QQuestion}{VB109}{Wie lange darf ein Funkamateur vorübergehend Amateurfunkverkehr im Ausland je Aufenthalt durchführen, wenn die CEPT-Regelung Anwendung findet?}{Bis zu 9~Monaten}
{\textbf{\textcolor{DARCgreen}{Bis zu 3~Monaten}}}
{Bis zu 6~Monaten}
{Bis zu einem Jahr}
\end{QQuestion}

}
\end{frame}

\begin{frame}
\frametitle{Weitere CEPT-Anerkennung}
\begin{itemize}
  \item Es gibt Staaten, die nicht der CEPT angehören, aber die Regelungen teilweise oder ganz anwenden
  \item Beispiele: USA und Australien
  \end{itemize}
\end{frame}

\begin{frame}
\only<1>{
\begin{QQuestion}{VB108}{Darf ein Funkamateur mit einer deutschen Zulassung zur Teilnahme am Amateurfunkdienst der Klasse~A oder E auch in Nicht-CEPT-Ländern auf Grundlage der CEPT-Empfehlungen T/R 61-01 bzw. ECC (05)06 Amateurfunkverkehr abwickeln?}{Ja, wenn diese Länder die entsprechende CEPT-Empfehlung anwenden.}
{Ja, weltweit, da die ITU die CEPT-Empfehlungen für allgemeingültig erklärt hat.}
{Nein, die Anwendung der CEPT-Empfehlungen ist Mitgliedern der CEPT vorbehalten.}
{Nein, die deutsche Amateurfunkzulassung ist auf Mitglieder der CEPT beschränkt.}
\end{QQuestion}

}
\only<2>{
\begin{QQuestion}{VB108}{Darf ein Funkamateur mit einer deutschen Zulassung zur Teilnahme am Amateurfunkdienst der Klasse~A oder E auch in Nicht-CEPT-Ländern auf Grundlage der CEPT-Empfehlungen T/R 61-01 bzw. ECC (05)06 Amateurfunkverkehr abwickeln?}{\textbf{\textcolor{DARCgreen}{Ja, wenn diese Länder die entsprechende CEPT-Empfehlung anwenden.}}}
{Ja, weltweit, da die ITU die CEPT-Empfehlungen für allgemeingültig erklärt hat.}
{Nein, die Anwendung der CEPT-Empfehlungen ist Mitgliedern der CEPT vorbehalten.}
{Nein, die deutsche Amateurfunkzulassung ist auf Mitglieder der CEPT beschränkt.}
\end{QQuestion}

}
\end{frame}

\begin{frame}
\frametitle{Rufzeichen}
\begin{itemize}
  \item Kennzeichnung durch zusätzlichen Rufzeichenpräfix vor eigenem Rufzeichen
  \item Ggf. je nach Amateurfunkklasse
  \item Muss vor Verwendung im jeweiligen Land nachgeschlagen werden
  \end{itemize}

\end{frame}

\begin{frame}
\only<1>{
\begin{QQuestion}{BD213}{Wie muss die Rufzeichennennung von DO7PR bei der Nutzung der \glqq CEPT-Novice-Amateurfunkgenehmigung\grqq{} in der Schweiz erfolgen?}{DO7PR/HB9}
{DO7PR/HB3}
{HB3/DO7PR}
{HB9/DO7PR}
\end{QQuestion}

}
\only<2>{
\begin{QQuestion}{BD213}{Wie muss die Rufzeichennennung von DO7PR bei der Nutzung der \glqq CEPT-Novice-Amateurfunkgenehmigung\grqq{} in der Schweiz erfolgen?}{DO7PR/HB9}
{DO7PR/HB3}
{\textbf{\textcolor{DARCgreen}{HB3/DO7PR}}}
{HB9/DO7PR}
\end{QQuestion}

}

\end{frame}

\begin{frame}
\only<1>{
\begin{QQuestion}{BD214}{Wie muss die Rufzeichennennung von DL9MJ bei der Nutzung der \glqq CEPT-Amateurfunkgenehmigung\grqq{} in der Schweiz erfolgen?}{HB3/DL9MJ}
{DL9MJ/HB9}
{DL9MJ/HB3}
{HB9/DL9MJ}
\end{QQuestion}

}
\only<2>{
\begin{QQuestion}{BD214}{Wie muss die Rufzeichennennung von DL9MJ bei der Nutzung der \glqq CEPT-Amateurfunkgenehmigung\grqq{} in der Schweiz erfolgen?}{HB3/DL9MJ}
{DL9MJ/HB9}
{DL9MJ/HB3}
{\textbf{\textcolor{DARCgreen}{HB9/DL9MJ}}}
\end{QQuestion}

}

\end{frame}

\begin{frame}
\frametitle{Auslandsbetrieb mit Klasse N}
Nicht erlaubt

\end{frame}

\begin{frame}
\only<1>{
\begin{QQuestion}{VB105}{Darf ein Funkamateur mit einer Amateurfunkzulassung der Klasse N in allen CEPT-Ländern Amateurfunkverkehr abwickeln?}{Ja, der Betrieb in allen CEPT-Ländern ist zulässig, wenn sich der Funkamateur an die Bestimmungen seines Heimatlandes hält.}
{Nein, die Zulassungsklasse N ist nur innerhalb der Europäischen Union gültig.}
{Nein, die Zulassungsklasse N ist nur in Deutschland gültig.}
{Ja, der Betrieb in allen CEPT-Ländern ist zulässig, wenn sich der Funkamateur an die Bestimmungen seines Gastlandes hält.}
\end{QQuestion}

}
\only<2>{
\begin{QQuestion}{VB105}{Darf ein Funkamateur mit einer Amateurfunkzulassung der Klasse N in allen CEPT-Ländern Amateurfunkverkehr abwickeln?}{Ja, der Betrieb in allen CEPT-Ländern ist zulässig, wenn sich der Funkamateur an die Bestimmungen seines Heimatlandes hält.}
{Nein, die Zulassungsklasse N ist nur innerhalb der Europäischen Union gültig.}
{\textbf{\textcolor{DARCgreen}{Nein, die Zulassungsklasse N ist nur in Deutschland gültig.}}}
{Ja, der Betrieb in allen CEPT-Ländern ist zulässig, wenn sich der Funkamateur an die Bestimmungen seines Gastlandes hält.}
\end{QQuestion}

}
\end{frame}

\begin{frame}
\frametitle{Ausländische Funkamateure in Deutschland}
\begin{itemize}
  \item Gegenseitige CEPT-Regelungen
  \item Vorübergehender Funkbetrieb bis 3 Monate
  \item Ohne festen Wohnsitz
  \item Je nach Klasse Präfix \emph{DL/} oder \emph{DO/}
  \end{itemize}
\end{frame}

\begin{frame}
\only<1>{
\begin{QQuestion}{VB110}{Wie muss ein Funkamateur aus einem Land, das die CEPT-Empfehlung T/R 61-01 oder die ECC-Empfehlung (05)06 anwendet, sein Heimatrufzeichen beim Betrieb einer Amateurfunkstelle in Deutschland ergänzen? Je nach Klasse des Funkamateurs wird~...}{/DL oder /DO angehängt.}
{DL/ oder DO/ vorangestellt.}
{DE/ oder DP/ vorangestellt.}
{/DE oder /DP angehängt.}
\end{QQuestion}

}
\only<2>{
\begin{QQuestion}{VB110}{Wie muss ein Funkamateur aus einem Land, das die CEPT-Empfehlung T/R 61-01 oder die ECC-Empfehlung (05)06 anwendet, sein Heimatrufzeichen beim Betrieb einer Amateurfunkstelle in Deutschland ergänzen? Je nach Klasse des Funkamateurs wird~...}{/DL oder /DO angehängt.}
{\textbf{\textcolor{DARCgreen}{DL/ oder DO/ vorangestellt.}}}
{DE/ oder DP/ vorangestellt.}
{/DE oder /DP angehängt.}
\end{QQuestion}

}
\end{frame}

\begin{frame}
\only<1>{
\begin{QQuestion}{BD212}{Sie hören die Amateurfunkstation mit dem Rufzeichen DL/G3MM. Welcher der nachfolgenden Sachverhalte trifft zu?}{Dem Funkamateur G3MM ist es aufgrund einer Gastzulassung gestattet, in Deutschland Amateurfunk auszuüben.}
{Der englischen Station G3MM ist es aufgrund der CEPT-Empfehlungen gestattet, vorübergehend in Deutschland Amateurfunk auszuüben.}
{Der englischen Station G3MM ist es aufgrund der CEPT-Empfehlungen gestattet, dauerhaft in Deutschland Amateurfunk auszuüben.}
{Der englischen Station G3MM ist es aufgrund der Radio Regulations (RR) gestattet, vorübergehend in Deutschland Amateurfunk auszuüben.}
\end{QQuestion}

}
\only<2>{
\begin{QQuestion}{BD212}{Sie hören die Amateurfunkstation mit dem Rufzeichen DL/G3MM. Welcher der nachfolgenden Sachverhalte trifft zu?}{Dem Funkamateur G3MM ist es aufgrund einer Gastzulassung gestattet, in Deutschland Amateurfunk auszuüben.}
{\textbf{\textcolor{DARCgreen}{Der englischen Station G3MM ist es aufgrund der CEPT-Empfehlungen gestattet, vorübergehend in Deutschland Amateurfunk auszuüben.}}}
{Der englischen Station G3MM ist es aufgrund der CEPT-Empfehlungen gestattet, dauerhaft in Deutschland Amateurfunk auszuüben.}
{Der englischen Station G3MM ist es aufgrund der Radio Regulations (RR) gestattet, vorübergehend in Deutschland Amateurfunk auszuüben.}
\end{QQuestion}

}
\end{frame}

\begin{frame}
\frametitle{Klubstationen}
\begin{itemize}
  \item CEPT-Regelungen gelten nur für persönliche Rufzeichen
  \item Klubstationen bedürfen einer Gastgenehmigung
  \end{itemize}
\end{frame}

\begin{frame}
\frametitle{Gastzulassung}
\begin{itemize}
  \item In Ländern, in denen die CEPT-Empfehlung nicht angewandt wird
  \item Bei der Behörde des Gastlandes beantragen
  \item Ggf. muss auch für Funkgeräte eine entsprechende Erlaubnis eingeholt werden
  \end{itemize}

\end{frame}

\begin{frame}
\only<1>{
\begin{QQuestion}{VB114}{Ist der vorübergehende Betrieb einer deutschen Klubstation nach CEPT-Empfehlung T/R~ 61-01 in einem Land erlaubt, welches diese Empfehlung anwendet?}{Nein, der Betrieb einer Klubstation bedarf der Beantragung einer Gastgenehmigung.}
{Ja, aber nur, wenn die Klubstation im Ausland an keinem festen Standort betrieben wird.}
{Ja, der Betrieb einer Klubstation ist zulässig, wenn der zuständigen Außenstelle der Bundesnetzagentur der vorgesehene Standort im Ausland vorher mitgeteilt worden ist.}
{Nein, weil es in den übrigen CEPT-Ländern keine Klubstationen gibt.}
\end{QQuestion}

}
\only<2>{
\begin{QQuestion}{VB114}{Ist der vorübergehende Betrieb einer deutschen Klubstation nach CEPT-Empfehlung T/R~ 61-01 in einem Land erlaubt, welches diese Empfehlung anwendet?}{\textbf{\textcolor{DARCgreen}{Nein, der Betrieb einer Klubstation bedarf der Beantragung einer Gastgenehmigung.}}}
{Ja, aber nur, wenn die Klubstation im Ausland an keinem festen Standort betrieben wird.}
{Ja, der Betrieb einer Klubstation ist zulässig, wenn der zuständigen Außenstelle der Bundesnetzagentur der vorgesehene Standort im Ausland vorher mitgeteilt worden ist.}
{Nein, weil es in den übrigen CEPT-Ländern keine Klubstationen gibt.}
\end{QQuestion}

}
\end{frame}

\begin{frame}
\only<1>{
\begin{QQuestion}{VB113}{Was hat ein Funkamateur zu veranlassen, wenn er eine Amateurfunkstelle in einem Land betreiben will, das die CEPT-Empfehlung nicht anwendet?}{Er muss bei der zuständigen Behörde des Landes eine Gastzulassung beantragen.}
{Er muss eine besondere Genehmigung der Bundesnetzagentur einholen.}
{Nichts, wenn das Gastland die IARU-Empfehlungen anwendet.}
{Nichts, da aufgrund von Gegenseitigkeitsabkommen der vorübergehende Betrieb genehmigt ist.}
\end{QQuestion}

}
\only<2>{
\begin{QQuestion}{VB113}{Was hat ein Funkamateur zu veranlassen, wenn er eine Amateurfunkstelle in einem Land betreiben will, das die CEPT-Empfehlung nicht anwendet?}{\textbf{\textcolor{DARCgreen}{Er muss bei der zuständigen Behörde des Landes eine Gastzulassung beantragen.}}}
{Er muss eine besondere Genehmigung der Bundesnetzagentur einholen.}
{Nichts, wenn das Gastland die IARU-Empfehlungen anwendet.}
{Nichts, da aufgrund von Gegenseitigkeitsabkommen der vorübergehende Betrieb genehmigt ist.}
\end{QQuestion}

}
\end{frame}

\begin{frame}
\frametitle{Nationale Regelungen}
\begin{itemize}
  \item In CEPT-Ländern gibt es unterschiedliche nationale Regelungen
  \item z.B. ist das in Deutschland freigegebene 6m-Band im Ausland beschränkt
  \item Bestimmungen und Auflagen des Gastlandes beachten
  \end{itemize}

\end{frame}

\begin{frame}
\only<1>{
\begin{QQuestion}{VB111}{Welche Regelungen sind beim Betrieb einer Amateurfunkstelle im Ausland zu beachten, wenn das besuchte Land die CEPT-Empfehlungen T/R~61-01 und (05)06 umgesetzt hat?}{Die Bestimmungen des Gastlandes, aber nur, wenn der Funkamateur sich dauerhaft dort aufhält. Mobil betriebene Funkstellen (z.~B. auf der Durchreise) können wie in Deutschland genutzt werden.}
{Die zutreffende CEPT-Empfehlung und die im Gastland geltenden Bestimmungen und Auflagen. Man muss sich z.~B. mit der Sendeleistung den Bestimmungen des Gastlandes anpassen.}
{Die zutreffende CEPT-Empfehlung und die im Heimatland geltenden Bestimmungen und Auflagen. Man muss sich z.~B. mit der Sendeleistung nicht den Bestimmungen des Gastlandes anpassen.}
{In Ländern der Europäischen Union (EU) gelten die gleichen Gesetze wie in Deutschland. Außerhalb der EU sind die jeweiligen nationalen Gesetze zu beachten.}
\end{QQuestion}

}
\only<2>{
\begin{QQuestion}{VB111}{Welche Regelungen sind beim Betrieb einer Amateurfunkstelle im Ausland zu beachten, wenn das besuchte Land die CEPT-Empfehlungen T/R~61-01 und (05)06 umgesetzt hat?}{Die Bestimmungen des Gastlandes, aber nur, wenn der Funkamateur sich dauerhaft dort aufhält. Mobil betriebene Funkstellen (z.~B. auf der Durchreise) können wie in Deutschland genutzt werden.}
{\textbf{\textcolor{DARCgreen}{Die zutreffende CEPT-Empfehlung und die im Gastland geltenden Bestimmungen und Auflagen. Man muss sich z.~B. mit der Sendeleistung den Bestimmungen des Gastlandes anpassen.}}}
{Die zutreffende CEPT-Empfehlung und die im Heimatland geltenden Bestimmungen und Auflagen. Man muss sich z.~B. mit der Sendeleistung nicht den Bestimmungen des Gastlandes anpassen.}
{In Ländern der Europäischen Union (EU) gelten die gleichen Gesetze wie in Deutschland. Außerhalb der EU sind die jeweiligen nationalen Gesetze zu beachten.}
\end{QQuestion}

}
\end{frame}

\begin{frame}
\only<1>{
\begin{QQuestion}{VB112}{Darf ein Funkamateur mit einer deutschen Amateurfunkzulassung auch im Ausland Amateurfunkverkehr auf dem \qty{6}{\m}-Band durchführen?}{Die Genehmigung für den Betrieb im \qty{6}{\m}-Band muss aus der Amateurfunkzulassung ersichtlich sein.}
{Der Funkamateur hat sich an die Bestimmungen des Gastlandes im Rahmen seiner CEPT-Amateurfunkgenehmigung zu halten.}
{Der Funkamateur muss dazu eine CEPT-Amateurfunkgenehmigung im Gastland beantragen.}
{Der Betrieb im \qty{6}{\m}-Band ist im Ausland nicht zulässig.}
\end{QQuestion}

}
\only<2>{
\begin{QQuestion}{VB112}{Darf ein Funkamateur mit einer deutschen Amateurfunkzulassung auch im Ausland Amateurfunkverkehr auf dem \qty{6}{\m}-Band durchführen?}{Die Genehmigung für den Betrieb im \qty{6}{\m}-Band muss aus der Amateurfunkzulassung ersichtlich sein.}
{\textbf{\textcolor{DARCgreen}{Der Funkamateur hat sich an die Bestimmungen des Gastlandes im Rahmen seiner CEPT-Amateurfunkgenehmigung zu halten.}}}
{Der Funkamateur muss dazu eine CEPT-Amateurfunkgenehmigung im Gastland beantragen.}
{Der Betrieb im \qty{6}{\m}-Band ist im Ausland nicht zulässig.}
\end{QQuestion}

}
\end{frame}

\begin{frame}
\frametitle{Umzug ins Ausland}
\begin{itemize}
  \item Wohnsitz länger als 3 Monate oder dauerhaft im Ausland
  \item Vereinfachte Beantragung der Amateurfunkzulassung
  \item Meistens keine erneute Prüfung
  \item Klasse~E Amateurfunkzeugnis $\rightarrow$ CEPT-Novice-Amateurfunk-Prüfungsbescheinigung
  \item Klasse~A Amateurfunkzeugnis $\rightarrow$ Harmonized Amateur Radio Examination Certificate (HAREC)
  \end{itemize}

\end{frame}

\begin{frame}
\only<1>{
\begin{QQuestion}{VB102}{Was versteht man unter dem Begriff HAREC (Harmonized Amateur Radio Examination Certificate)?}{Es ist eine harmonisierte CEPT-Novice-Amateurfunkgenehmigung gemäß der ECC-Empfehlung~(05)06.}
{Es ist eine harmonisierte CEPT-Amateurfunkgenehmigung gemäß der CEPT-Empfehlung T/R~61-01.}
{Es ist eine harmonisierte CEPT-Novice-Amateurfunkprüfungsbescheinigung gemäß dem ERC-Report~32. Das Amateurfunkzeugnis der Klasse~E entspricht dieser Empfehlung.}
{Es ist eine harmonisierte Amateurfunkprüfungsbescheinigung gemäß der CEPT-Empfehlung T/R~61-02.
Das Amateurfunkzeugnis der Klasse~A entspricht dieser Empfehlung.}
\end{QQuestion}

}
\only<2>{
\begin{QQuestion}{VB102}{Was versteht man unter dem Begriff HAREC (Harmonized Amateur Radio Examination Certificate)?}{Es ist eine harmonisierte CEPT-Novice-Amateurfunkgenehmigung gemäß der ECC-Empfehlung~(05)06.}
{Es ist eine harmonisierte CEPT-Amateurfunkgenehmigung gemäß der CEPT-Empfehlung T/R~61-01.}
{Es ist eine harmonisierte CEPT-Novice-Amateurfunkprüfungsbescheinigung gemäß dem ERC-Report~32. Das Amateurfunkzeugnis der Klasse~E entspricht dieser Empfehlung.}
{\textbf{\textcolor{DARCgreen}{Es ist eine harmonisierte Amateurfunkprüfungsbescheinigung gemäß der CEPT-Empfehlung T/R~61-02.
Das Amateurfunkzeugnis der Klasse~A entspricht dieser Empfehlung.}}}
\end{QQuestion}

}
\end{frame}

\begin{frame}
\only<1>{
\begin{QQuestion}{VB103}{Was ist eine HAREC-Bescheinigung? Das Dokument~...}{berechtigt den Funkamateur zur Durchführung von vorübergehendem Amateurfunkbetrieb nach der CEPT-Empfehlung T/R~61-01.}
{bescheinigt eine erfolgreich abgelegte Prüfung der Klasse~A nach der CEPT-Empfehlung T/R~61-02. Damit kann ein Funkamateur in den beteiligten Ländern eine Amateurfunkzulassung erhalten.}
{bescheinigt eine erfolgreich abgelegte Prüfung der Klasse~E nach ECC~(05)06. Damit kann ein Funkamateur in den beteiligten Ländern eine Amateurfunkzulassung erhalten.}
{erhalten Funkamateure, die die freiwillige Morseprüfung erfolgreich abgelegt haben.}
\end{QQuestion}

}
\only<2>{
\begin{QQuestion}{VB103}{Was ist eine HAREC-Bescheinigung? Das Dokument~...}{berechtigt den Funkamateur zur Durchführung von vorübergehendem Amateurfunkbetrieb nach der CEPT-Empfehlung T/R~61-01.}
{\textbf{\textcolor{DARCgreen}{bescheinigt eine erfolgreich abgelegte Prüfung der Klasse~A nach der CEPT-Empfehlung T/R~61-02. Damit kann ein Funkamateur in den beteiligten Ländern eine Amateurfunkzulassung erhalten.}}}
{bescheinigt eine erfolgreich abgelegte Prüfung der Klasse~E nach ECC~(05)06. Damit kann ein Funkamateur in den beteiligten Ländern eine Amateurfunkzulassung erhalten.}
{erhalten Funkamateure, die die freiwillige Morseprüfung erfolgreich abgelegt haben.}
\end{QQuestion}

}
\end{frame}

\begin{frame}
\only<1>{
\begin{QQuestion}{VB101}{Welche der folgenden Aussagen zur \glqq CEPT-Novice-Amateurfunk-Prüfungsbescheinigung\grqq{} ist richtig?}{Inhaber der CEPT-Novice-Amateurfunk-Prüfungsbescheinigung dürfen grundsätzlich keinen Amateurfunkbetrieb im Ausland durchführen.}
{Sie berechtigt den Inhaber zur Durchführung von Amateurfunkbetrieb im Ausland nach den deutschen Bestimmungen.}
{Sie berechtigt den Inhaber zur Durchführung von Amateurfunkbetrieb gemäß der CEPT-Empfehlung T/R~61-01.}
{Sie kann die Erteilung einer entsprechenden Novice-Individualgenehmigung für Funkamateure in einem anderen Land vereinfachen.}
\end{QQuestion}

}
\only<2>{
\begin{QQuestion}{VB101}{Welche der folgenden Aussagen zur \glqq CEPT-Novice-Amateurfunk-Prüfungsbescheinigung\grqq{} ist richtig?}{Inhaber der CEPT-Novice-Amateurfunk-Prüfungsbescheinigung dürfen grundsätzlich keinen Amateurfunkbetrieb im Ausland durchführen.}
{Sie berechtigt den Inhaber zur Durchführung von Amateurfunkbetrieb im Ausland nach den deutschen Bestimmungen.}
{Sie berechtigt den Inhaber zur Durchführung von Amateurfunkbetrieb gemäß der CEPT-Empfehlung T/R~61-01.}
{\textbf{\textcolor{DARCgreen}{Sie kann die Erteilung einer entsprechenden Novice-Individualgenehmigung für Funkamateure in einem anderen Land vereinfachen.}}}
\end{QQuestion}

}
\end{frame}%ENDCONTENT
