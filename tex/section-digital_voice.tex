
\section{Digital Voice (DV)}
\label{section:digital_voice}
\begin{frame}%STARTCONTENT
\begin{itemize}
  \item Auch Sprache kann digital übertragen werden
  \item z.\,B. mit den Übertragungsverfahren DMR, D-Star, C4FM und M17
  \item Sprachsignale werden vor der Übertragung in einen Datenstrom umgewandelt
  \end{itemize}
\end{frame}

\begin{frame}
\frametitle{TDMA}
Time Division Multiple Access -- Zeitmultiplexverfahren
\begin{columns}
    \begin{column}{0.48\textwidth}
    \begin{itemize}
  \item Übertragung mehrerer Datenströme in schnell abwechselnder Folge
  \item Zwei oder mehr Sprachverbindungen nutzen quasi gleichzeitig dieselbe Frequenz
  \end{itemize}

    \end{column}
   \begin{column}{0.48\textwidth}
       
\begin{figure}
    \DARCimage{0.85\linewidth}{474include}
    \caption{\scriptsize TDMA mit drei Verbindungen auf einer Frequenz}
    \label{n_digital_voice_tdma}
\end{figure}


   \end{column}
\end{columns}

\end{frame}

\begin{frame}
\frametitle{Einstellungen}
Es sind für digitale Sprache oft mehr Einstellungen zu berücksichtigen als zum Beispiel bei einer FM-Verbindung. Zum Beispiel:

\begin{itemize}
  \item Sprechgruppe (Talkgroup)
  \item Raum oder Reflektor zum Zusammenschalten von Relaisfunkstellen
  \item TDMA-Zeitschlitz
  \item Color-Code
  \end{itemize}
\end{frame}

\begin{frame} 
\only<1>{
\begin{QQuestion}{NE404}{Welche Übertragungsverfahren für digitalen Sprechfunk sind im Amateurfunk gebräuchlich?}{FM-Sprechfunk, RTTY, D-STAR, JS8, Olivia}
{AM-Sprechfunk, FM-Sprechfunk, SSB-Sprechfunk, Olivia, SSTV}
{DMR, D-STAR, C4FM, M17, FreeDV}
{SSB-Sprechfunk, FT8, DMR, PSK31, SSTV}
\end{QQuestion}

}
\only<2>{
\begin{QQuestion}{NE404}{Welche Übertragungsverfahren für digitalen Sprechfunk sind im Amateurfunk gebräuchlich?}{FM-Sprechfunk, RTTY, D-STAR, JS8, Olivia}
{AM-Sprechfunk, FM-Sprechfunk, SSB-Sprechfunk, Olivia, SSTV}
{\textbf{\textcolor{DARCgreen}{DMR, D-STAR, C4FM, M17, FreeDV}}}
{SSB-Sprechfunk, FT8, DMR, PSK31, SSTV}
\end{QQuestion}

}
\end{frame}

\begin{frame} 
\only<1>{
\begin{QQuestion}{NE307}{Welche Übertragungsverfahren werden bei VHF/UHF-Handfunkgeräten üblicherweise verwendet?}{CW-Morsetelegrafie, FT8, D-STAR}
{FM-Sprechfunk, DMR, D-STAR}
{SSB-Sprechfunk, DMR, RTTY}
{AM-Sprechfunk, C4FM, FT8}
\end{QQuestion}

}
\only<2>{
\begin{QQuestion}{NE307}{Welche Übertragungsverfahren werden bei VHF/UHF-Handfunkgeräten üblicherweise verwendet?}{CW-Morsetelegrafie, FT8, D-STAR}
{\textbf{\textcolor{DARCgreen}{FM-Sprechfunk, DMR, D-STAR}}}
{SSB-Sprechfunk, DMR, RTTY}
{AM-Sprechfunk, C4FM, FT8}
\end{QQuestion}

}
\end{frame}

\begin{frame} 
\only<1>{
\begin{QQuestion}{NE403}{Ist es bei bestimmten digitalen Verfahren zur Sprachübertragung (z.~B. DMR oder TETRA) möglich, mehrere Sprechverbindungen gleichzeitig auf derselben Frequenz innerhalb eines Empfangsgebiets abzuwickeln?}{Nein. Sprachübertragungen können nicht in Datenpakete aufgeteilt werden.}
{Ja. Die Sendeleistung wird zur Verbesserung der digitalen Fehlerkorrektur erhöht.}
{Nein. Zeitgleich stattfindende digitale Übertragungen stören sich prinzipbedingt gegenseitig.}
{Ja. Die Sprachdaten werden abwechselnd in periodischen, kurzen Zeitschlitzen übertragen.}
\end{QQuestion}

}
\only<2>{
\begin{QQuestion}{NE403}{Ist es bei bestimmten digitalen Verfahren zur Sprachübertragung (z.~B. DMR oder TETRA) möglich, mehrere Sprechverbindungen gleichzeitig auf derselben Frequenz innerhalb eines Empfangsgebiets abzuwickeln?}{Nein. Sprachübertragungen können nicht in Datenpakete aufgeteilt werden.}
{Ja. Die Sendeleistung wird zur Verbesserung der digitalen Fehlerkorrektur erhöht.}
{Nein. Zeitgleich stattfindende digitale Übertragungen stören sich prinzipbedingt gegenseitig.}
{\textbf{\textcolor{DARCgreen}{Ja. Die Sprachdaten werden abwechselnd in periodischen, kurzen Zeitschlitzen übertragen.}}}
\end{QQuestion}

}
\end{frame}

\begin{frame} 
\only<1>{
\begin{QQuestion}{NE402}{Sie möchten an einer Funkrunde mittels digitaler Sprachübertragung (z.~B. C4FM, DMR oder D-Star) über ein Repeaternetzwerk teilnehmen. Worauf müssen Sie neben der Wahl des Übertragungsverfahrens, der Frequenz und der Modulation achten?}{Alle Stationen müssen sich in Funkreichweite desselben Repeaters befinden.}
{Sie müssen die gleiche Firmwareversion wie das Repeaternetzwerk verwenden.}
{Alle Stationen müssen die gleiche Stationskennung, z.~B. DMR-ID, einstellen.}
{Sie müssen geeignete Parameter, z.~B. Reflektor, Zeitschlitz oder Color-Code, wählen.}
\end{QQuestion}

}
\only<2>{
\begin{QQuestion}{NE402}{Sie möchten an einer Funkrunde mittels digitaler Sprachübertragung (z.~B. C4FM, DMR oder D-Star) über ein Repeaternetzwerk teilnehmen. Worauf müssen Sie neben der Wahl des Übertragungsverfahrens, der Frequenz und der Modulation achten?}{Alle Stationen müssen sich in Funkreichweite desselben Repeaters befinden.}
{Sie müssen die gleiche Firmwareversion wie das Repeaternetzwerk verwenden.}
{Alle Stationen müssen die gleiche Stationskennung, z.~B. DMR-ID, einstellen.}
{\textbf{\textcolor{DARCgreen}{Sie müssen geeignete Parameter, z.~B. Reflektor, Zeitschlitz oder Color-Code, wählen.}}}
\end{QQuestion}

}
\end{frame}%ENDCONTENT
