\documentclass[aspectratio = 169]{beamer}

\usepackage{../../common/beamerthemedarc}
\PassOptionsToPackage{fleqn}{amsmath}
\usepackage{amsmath}
\usepackage{amssymb}
\usepackage{csquotes}
\usepackage[greek,german]{babel}
\usepackage{siunitx}
\usepackage{tabularray}
\usepackage{fontawesome5}
\usepackage{cancel}
\makeatletter
\let\ang\relax
\makeatother
\usepackage{texvc}

\DeclareSIUnit{\baud}{Bd}
\DeclareSIUnit{\dBi}{dBi}
\DeclareSIUnit{\dBm}{dBm}
\DeclareSIUnit{\dBu}{dBu}
\DeclareSIUnit{\dBV}{dbV}
\DeclareSIUnit{\dBW}{dBW}
\DeclareSIUnit{\ppm}{ppm}
\DeclareSIUnit{\pps}{pps}
\DeclareSIUnit{\CPM}{CPM}
\DeclareSIUnit{\WPM}{WPM}
\DeclareSIUnit\noop{\relax}

\definecolor{DARCgreen}{RGB}{0,155,110}
\definecolor{DARCorange}{RGB}{250,180,0}
\definecolor{DARCred}{RGB}{225,80,35}
\definecolor{DARCblue}{RGB}{0,160,220}
\definecolor{DARCgray}{RGB}{190,190,190}
\definecolor{DARClightgray}{HTML/cmyk}{e8e9e8/.11,.07,.09,0}

\definecolor{unit-title}{RGB}{155,205,185}
\definecolor{attention-title}{RGB}{255,230,170}
\definecolor{danger-title}{RGB}{245,175,150}
\definecolor{hint-title}{RGB}{185,220,245}
\definecolor{advanced-title}{RGB}{225,230,235}

\definecolor{DARCdarkblue}{RGB}{43, 103, 140}

\definecolor{r}{rgb}{0.98, 0.84, 0.65}
\definecolor{rsilver}{rgb}{0.75, 0.75, 0.75}
\definecolor{rgold}{rgb}{0.85, 0.65, 0.13}
\definecolor{rbrown}{rgb}{0.59, 0.29, 0.0}
\definecolor{rblack}{rgb}{0,0,0}
\definecolor{rred}{rgb}{0.89, 0.0, 0.13}
\definecolor{rorange}{rgb}{0.93, 0.53, 0.18}
\definecolor{ryellow}{rgb}{1.0, 0.96, 0.0}
\definecolor{rgreen}{rgb}{0.0, 0.42, 0.24}
\definecolor{rblue}{rgb}{0.06, 0.2, 0.65}
\definecolor{rviolet}{rgb}{0.56, 0.0, 1.0}
\definecolor{rgrey}{rgb}{0.33, 0.33, 0.33}
\definecolor{rwhite}{rgb}{0.96, 0.96, 0.96}

\tcbset{QSO/.style={
	enhanced,boxrule=0pt,
	frame empty,colback=DARClightgray,fontupper=\footnotesize,
	interior code={
		\fill[tcbcolback](frame.north west)to(frame.north east)to([yshift=-\ht\strutbox]interior.north east)to(interior.south east)to(interior.south west)to([yshift=-\ht\strutbox]interior.north west)--cycle;
	}
}}

\newcommand{\QSOown}[1]{\begin{tcolorbox}[
		QSO,
		leftrule=\ht\strutbox,
		right skip=.1\linewidth
]%
#1%
\end{tcolorbox}}

\newcommand{\QSOother}[1]{\begin{tcolorbox}[
		QSO,
		rightrule=\ht\strutbox,
		left skip=.1\linewidth]%
		#1%
\end{tcolorbox}}

\newenvironment{QQuestion}[6]{
    \begin{block}{#1}
        #2
        \begin{itemize}
            \item[A:] #3
            \item[B:] #4
            \item[C:] #5
            \item[D:] #6
        \end{itemize}
    \end{block}
}

\newenvironment{PQuestion}[7]{
    \begin{block}{#1}
        \begin{columns}
            \begin{column}{0.48\linewidth}
                #7
            \end{column}
            \begin{column}{0.48\linewidth}
                #2
                \begin{itemize}
                    \item[A:] #3
                    \item[B:] #4
                    \item[C:] #5
                    \item[D:] #6
                \end{itemize}
            \end{column}
        \end{columns}
    \end{block}
}

\newenvironment{question_md}[6]{
    \Question{#1}{#2}{#3}{#4}{#5}{#6}
}

\newenvironment{question2x2}[6]{
    \begin{block}{#1}
        #2
        \begin{columns}
            \begin{column}{0.48\linewidth}
                \begin{itemize}
                    \item[A:] #3
                    \item[B:] #4
                \end{itemize}
            \end{column}
            \begin{column}{0.48\linewidth}
                \begin{itemize}
                    \item[C:] #5
                    \item[D:] #6
                \end{itemize}
            \end{column}
        \end{columns}
    \end{block}
}

%\newenvironment{DARCtabular}[1]{
%    \begin{tblr}{hlines, colspec={#1}}
%}{
%    \end{tblr}
%}

\NewTblrEnviron{DARCtabular}

\SetTblrInner[DARCtabular]{
    row{odd} = {DARCblue!30},
    row{even} = {DARCblue!50},
    row{1} = {bg=DARCblue, fg=white, font=\bfseries}
}

\SetTblrOuter[DARCtabular]{
    baseline=B
}

\input{../../common/settings.tex}

\begin{document}

% Last Line important!

\title{DARC Amateurfunklehrgang Klasse A}
\author{Wellenausbreitung}
\institute{Deutscher Amateur Radio Club e.\,V.}
\begin{frame}
\maketitle
\end{frame}

\section{Troposphäre III}
\label{section:troposphaere_3}
\begin{frame}%STARTCONTENT

\frametitle{Foliensatz in Arbeit}
2024-04-28: Die Inhalte werden noch aufbereitet.

Derzeit sind in diesem Abschnitt nur die Fragen sortiert enthalten.

Für das Selbststudium verweisen wir aktuell auf den Abschnitt Wellenausbreitung im DARC Online Lehrgang (\textcolor{DARCblue}{\faLink~\href{https://www.darc.de/der-club/referate/ajw/lehrgang-te/e09/}{www.darc.de/der-club/referate/ajw/lehrgang-te/e09/}}) für die Prüfung bis Juni 2024. Bis auf die Fragen hat sich an der Thematik nichts geändert. Das Thema war bisher Stoff der Klasse~E und wurde mit der neuen Prüfungsordnung auf alle drei Klassen aufgeteilt.

\end{frame}

\begin{frame}
\frametitle{Lage der Ionosphären-Schichten}

\begin{figure}
    \DARCimage{0.85\linewidth}{731include}
    \caption{\scriptsize Für den Amateurfunk relevante Schichten in der Atmosphäre}
    \label{e_atmosphaeren_schichten}
\end{figure}

\end{frame}

\begin{frame}
\only<1>{
\begin{QQuestion}{AH105}{In welcher Höhe befindet sich die für die Fernausbreitung (DX) wichtige F1-Region während der Tagesstunden? Sie befindet sich in ungefähr~...}{\qtyrange{200}{450}{\km} Höhe.}
{\qtyrange{90}{130}{\km} Höhe.}
{\qtyrange{50}{90}{\km} Höhe.}
{\qtyrange{130}{200}{\km} Höhe.}
\end{QQuestion}

}
\only<2>{
\begin{QQuestion}{AH105}{In welcher Höhe befindet sich die für die Fernausbreitung (DX) wichtige F1-Region während der Tagesstunden? Sie befindet sich in ungefähr~...}{\qtyrange{200}{450}{\km} Höhe.}
{\qtyrange{90}{130}{\km} Höhe.}
{\qtyrange{50}{90}{\km} Höhe.}
{\textbf{\textcolor{DARCgreen}{\qtyrange{130}{200}{\km} Höhe.}}}
\end{QQuestion}

}
\end{frame}

\begin{frame}
\only<1>{
\begin{QQuestion}{AH106}{In welcher Höhe befindet sich die für die Fernausbreitung (DX) wichtige F2-Region während der Tagesstunden an einem Sommertag? Sie befindet sich in ungefähr~...}{\qtyrange{130}{200}{\km} Höhe.}
{\qtyrange{250}{450}{\km} Höhe.}
{\qtyrange{90}{130}{\km} Höhe.}
{\qtyrange{50}{90}{\km} Höhe.}
\end{QQuestion}

}
\only<2>{
\begin{QQuestion}{AH106}{In welcher Höhe befindet sich die für die Fernausbreitung (DX) wichtige F2-Region während der Tagesstunden an einem Sommertag? Sie befindet sich in ungefähr~...}{\qtyrange{130}{200}{\km} Höhe.}
{\textbf{\textcolor{DARCgreen}{\qtyrange{250}{450}{\km} Höhe.}}}
{\qtyrange{90}{130}{\km} Höhe.}
{\qtyrange{50}{90}{\km} Höhe.}
\end{QQuestion}

}
\end{frame}

\begin{frame}
\frametitle{Effekte der Ionisierung}
\end{frame}

\begin{frame}
\only<1>{
\begin{QQuestion}{AH101}{Welcher Effekt sorgt hauptsächlich dafür, dass ionosphärische Regionen Funkwellen zur Erde ablenken können?}{Die von der Sonne ausgehende Infrarotstrahlung ionisiert~-~je nach Strahlungsintensität~-~die Moleküle in den verschiedenen Regionen.}
{Die von der Sonne ausgehende UV-Strahlung ionisiert~-~je nach Strahlungsintensität~-~die Moleküle in den verschiedenen Regionen.}
{Die von der Sonne ausgehende UV-Strahlung aktiviert~-~je nach Strahlungsintensität~-~die Sauerstoffatome in den verschiedenen Regionen.}
{Die von der Sonne ausgehende Infrarotstrahlung aktiviert~-~je nach Strahlungsintensität~-~die Sauerstoffatome in den verschiedenen Regionen.}
\end{QQuestion}

}
\only<2>{
\begin{QQuestion}{AH101}{Welcher Effekt sorgt hauptsächlich dafür, dass ionosphärische Regionen Funkwellen zur Erde ablenken können?}{Die von der Sonne ausgehende Infrarotstrahlung ionisiert~-~je nach Strahlungsintensität~-~die Moleküle in den verschiedenen Regionen.}
{\textbf{\textcolor{DARCgreen}{Die von der Sonne ausgehende UV-Strahlung ionisiert~-~je nach Strahlungsintensität~-~die Moleküle in den verschiedenen Regionen.}}}
{Die von der Sonne ausgehende UV-Strahlung aktiviert~-~je nach Strahlungsintensität~-~die Sauerstoffatome in den verschiedenen Regionen.}
{Die von der Sonne ausgehende Infrarotstrahlung aktiviert~-~je nach Strahlungsintensität~-~die Sauerstoffatome in den verschiedenen Regionen.}
\end{QQuestion}

}
\end{frame}

\begin{frame}
\only<1>{
\begin{QQuestion}{AH108}{Zu welcher Jahres- und Tageszeit hat die F2-Region ihre größte Höhe? Sie hat ihre größte Höhe~...}{im Sommer um Mitternacht.}
{im Sommer zur Mittagszeit.}
{im Frühjahr und Herbst zur Dämmerungszeit.}
{im Winter zur Mittagszeit.}
\end{QQuestion}

}
\only<2>{
\begin{QQuestion}{AH108}{Zu welcher Jahres- und Tageszeit hat die F2-Region ihre größte Höhe? Sie hat ihre größte Höhe~...}{im Sommer um Mitternacht.}
{\textbf{\textcolor{DARCgreen}{im Sommer zur Mittagszeit.}}}
{im Frühjahr und Herbst zur Dämmerungszeit.}
{im Winter zur Mittagszeit.}
\end{QQuestion}

}
\end{frame}

\begin{frame}
\only<1>{
\begin{QQuestion}{AH221}{Massiv erhöhte UV- und Röntgenstrahlung, wie sie vor allem durch starke Sonneneruptionen hervorgerufen wird, beeinflusst in der Ionosphäre vor allem~...}{die F2-Region, die dann so stark ionisiert wird, dass fast die gesamte KW-Ausstrahlung reflektiert wird.}
{die D-Region, die die Kurzwellen-Signale dann so massiv dämpft, dass keine Ausbreitung über die Raumwelle mehr möglich ist.}
{die E-Region, die dann für die höheren Frequenzen durchlässiger wird und durch Refraktion (Brechung) in der F2-Region für gute Ausbreitungsbedingungen sorgt.}
{die F1-Region, die durch Absorption der höheren Frequenzen die Refraktion (Brechung) an der F2-Region behindert.}
\end{QQuestion}

}
\only<2>{
\begin{QQuestion}{AH221}{Massiv erhöhte UV- und Röntgenstrahlung, wie sie vor allem durch starke Sonneneruptionen hervorgerufen wird, beeinflusst in der Ionosphäre vor allem~...}{die F2-Region, die dann so stark ionisiert wird, dass fast die gesamte KW-Ausstrahlung reflektiert wird.}
{\textbf{\textcolor{DARCgreen}{die D-Region, die die Kurzwellen-Signale dann so massiv dämpft, dass keine Ausbreitung über die Raumwelle mehr möglich ist.}}}
{die E-Region, die dann für die höheren Frequenzen durchlässiger wird und durch Refraktion (Brechung) in der F2-Region für gute Ausbreitungsbedingungen sorgt.}
{die F1-Region, die durch Absorption der höheren Frequenzen die Refraktion (Brechung) an der F2-Region behindert.}
\end{QQuestion}

}
\end{frame}

\begin{frame}
\frametitle{Polarisation}
\end{frame}

\begin{frame}
\only<1>{
\begin{QQuestion}{AH219}{Wie wird die Polarisation einer elektromagnetischen Welle bei der Ausbreitung über die Raumwelle beeinflusst?}{Die Polarisation der ausgesendeten Wellen wird in der Ionosphäre stets um \qty{90}{\degree} gedreht.}
{Die Polarisation der ausgesendeten Wellen bleibt bei der Refraktion (Brechung) in der Ionosphäre stets unverändert.}
{Die Polarisation der ausgesendeten Wellen wird bei der Refraktion (Brechung) in der Ionosphäre stets verändert.}
{Die Polarisation der ausgesendeten Wellen wird bei jedem Sprung (Hop) in der Ionosphäre um \qty{90}{\degree} gedreht.}
\end{QQuestion}

}
\only<2>{
\begin{QQuestion}{AH219}{Wie wird die Polarisation einer elektromagnetischen Welle bei der Ausbreitung über die Raumwelle beeinflusst?}{Die Polarisation der ausgesendeten Wellen wird in der Ionosphäre stets um \qty{90}{\degree} gedreht.}
{Die Polarisation der ausgesendeten Wellen bleibt bei der Refraktion (Brechung) in der Ionosphäre stets unverändert.}
{\textbf{\textcolor{DARCgreen}{Die Polarisation der ausgesendeten Wellen wird bei der Refraktion (Brechung) in der Ionosphäre stets verändert.}}}
{Die Polarisation der ausgesendeten Wellen wird bei jedem Sprung (Hop) in der Ionosphäre um \qty{90}{\degree} gedreht.}
\end{QQuestion}

}
\end{frame}

\begin{frame}
\frametitle{Funkwellen-Ausbreitung}
\end{frame}

\begin{frame}
\only<1>{
\begin{QQuestion}{AH201}{Welches der nachstehend aufgeführten Bänder ist für KW-Verbindungen zwischen Hamburg und München um die Mittagszeit herum üblicherweise gut geeignet?}{\qty{40}{\m}-Band}
{\qty{160}{\m}-Band}
{\qty{80}{\m}-Band}
{\qty{15}{\m}-Band}
\end{QQuestion}

}
\only<2>{
\begin{QQuestion}{AH201}{Welches der nachstehend aufgeführten Bänder ist für KW-Verbindungen zwischen Hamburg und München um die Mittagszeit herum üblicherweise gut geeignet?}{\textbf{\textcolor{DARCgreen}{\qty{40}{\m}-Band}}}
{\qty{160}{\m}-Band}
{\qty{80}{\m}-Band}
{\qty{15}{\m}-Band}
\end{QQuestion}

}
\end{frame}

\begin{frame}
\only<1>{
\begin{QQuestion}{AH203}{Welche der folgenden Frequenzbänder können in den Nachtstunden am ehesten für weltweite Funkverbindungen genutzt werden?}{\qty{160}{\m}, \qty{80}{\m} und \qty{40}{\m}}
{\qty{40}{\m}, \qty{20}{\m} und \qty{15}{\m}}
{\qty{40}{\m}, \qty{17}{\m} und \qty{6}{\m}}
{ \qty{30}{\m}, \qty{12}{\m} und \qty{10}{\m}}
\end{QQuestion}

}
\only<2>{
\begin{QQuestion}{AH203}{Welche der folgenden Frequenzbänder können in den Nachtstunden am ehesten für weltweite Funkverbindungen genutzt werden?}{\textbf{\textcolor{DARCgreen}{\qty{160}{\m}, \qty{80}{\m} und \qty{40}{\m}}}}
{\qty{40}{\m}, \qty{20}{\m} und \qty{15}{\m}}
{\qty{40}{\m}, \qty{17}{\m} und \qty{6}{\m}}
{ \qty{30}{\m}, \qty{12}{\m} und \qty{10}{\m}}
\end{QQuestion}

}
\end{frame}

\begin{frame}
\only<1>{
\begin{QQuestion}{AH107}{Für die DX-Kurzwellenausbreitung über die Raumwelle ist die F1-Region~...}{erwünscht, weil sie durch zusätzliche Reflexion die Wirkung der F2-Region verstärken kann.}
{meist unerwünscht, weil sie durch Abdeckung die Ausbreitung durch Refraktion (Brechung) an der F2-Region verhindern kann.}
{nicht von großer Bedeutung, weil sie vor allem für die höheren Frequenzen durchlässig ist.}
{von großer Bedeutung, weil sie die Dämpfung in der E-Region senkt und damit die Sprungdistanz vergrößert.}
\end{QQuestion}

}
\only<2>{
\begin{QQuestion}{AH107}{Für die DX-Kurzwellenausbreitung über die Raumwelle ist die F1-Region~...}{erwünscht, weil sie durch zusätzliche Reflexion die Wirkung der F2-Region verstärken kann.}
{\textbf{\textcolor{DARCgreen}{meist unerwünscht, weil sie durch Abdeckung die Ausbreitung durch Refraktion (Brechung) an der F2-Region verhindern kann.}}}
{nicht von großer Bedeutung, weil sie vor allem für die höheren Frequenzen durchlässig ist.}
{von großer Bedeutung, weil sie die Dämpfung in der E-Region senkt und damit die Sprungdistanz vergrößert.}
\end{QQuestion}

}
\end{frame}

\begin{frame}
\frametitle{Troposphärische Ausbreitung}
\end{frame}

\begin{frame}
\only<1>{
\begin{QQuestion}{AH309}{Überhorizontverbindungen im VHF/UHF-Bereich kommen unter anderem zustande durch~...}{Polarisationsdrehungen in der Troposphäre an Gewitterfronten.}
{Reflexion der Wellen in der Troposphäre durch das Auftreten sporadischer D-Regionen.}
{Polarisationsdrehungen in der Troposphäre bei hoch liegender Bewölkung.}
{troposphärische Duct-Übertragung beim Auftreten von Inversionsschichten.}
\end{QQuestion}

}
\only<2>{
\begin{QQuestion}{AH309}{Überhorizontverbindungen im VHF/UHF-Bereich kommen unter anderem zustande durch~...}{Polarisationsdrehungen in der Troposphäre an Gewitterfronten.}
{Reflexion der Wellen in der Troposphäre durch das Auftreten sporadischer D-Regionen.}
{Polarisationsdrehungen in der Troposphäre bei hoch liegender Bewölkung.}
{\textbf{\textcolor{DARCgreen}{troposphärische Duct-Übertragung beim Auftreten von Inversionsschichten.}}}
\end{QQuestion}

}
\end{frame}%ENDCONTENT


\section{Mehrwegeausbreitung}
\label{section:mehrwegeausbreitung}
\begin{frame}%STARTCONTENT

\begin{columns}
    \begin{column}{0.48\textwidth}
    \begin{itemize}
  \item Funksignal gelangt auf mehr als einem Weg vom Sender zum Empfänger
  \item Reflexion an Gebäude, Gelände, Flugzeuge etc.
  \item Refraktion an Ionossphäre bei Kurzwelle
  \item Führt zu Interferenz mit Verstärkung oder Auslöschung des Signals
  \end{itemize}

    \end{column}
   \begin{column}{0.48\textwidth}
       
   \end{column}
\end{columns}

\end{frame}

\begin{frame}\end{frame}%ENDCONTENT


\section{Aurora II}
\label{section:aurora_2}
\begin{frame}%STARTCONTENT

\frametitle{Foliensatz in Arbeit}
2024-04-28: Die Inhalte werden noch aufbereitet.

Derzeit sind in diesem Abschnitt nur die Fragen sortiert enthalten.

Für das Selbststudium verweisen wir aktuell auf den Abschnitt Wellenausbreitung im DARC Online Lehrgang (\textcolor{DARCblue}{\faLink~\href{https://www.darc.de/der-club/referate/ajw/lehrgang-te/e09/}{www.darc.de/der-club/referate/ajw/lehrgang-te/e09/}}) für die Prüfung bis Juni 2024. Bis auf die Fragen hat sich an der Thematik nichts geändert. Das Thema war bisher Stoff der Klasse~E und wurde mit der neuen Prüfungsordnung auf alle drei Klassen aufgeteilt.

\end{frame}

\begin{frame}
\frametitle{Auftreten von Aurora}

\end{frame}

\begin{frame}
\only<1>{
\begin{QQuestion}{AH302}{In welchem ionosphärischen Bereich treten gelegentlich Aurora-Erscheinungen auf?}{In der E-Region in der Nähe der Pole}
{In der F-Region}
{In der E-Region in der Nähe des Äquators.}
{In der D-Region}
\end{QQuestion}

}
\only<2>{
\begin{QQuestion}{AH302}{In welchem ionosphärischen Bereich treten gelegentlich Aurora-Erscheinungen auf?}{\textbf{\textcolor{DARCgreen}{In der E-Region in der Nähe der Pole}}}
{In der F-Region}
{In der E-Region in der Nähe des Äquators.}
{In der D-Region}
\end{QQuestion}

}
\end{frame}

\begin{frame}
\only<1>{
\begin{QQuestion}{AH303}{Was ist die Ursache für Aurora-Erscheinungen?}{Das Eindringen geladener Teilchen von der Sonne in die Atmosphäre der Polarregionen.}
{Eine hohe Sonnenfleckenzahl.}
{Eine niedrige Sonnenfleckenzahl.}
{Das Eindringen starker Meteoritenschauer in die Atmosphäre der Polarregionen.}
\end{QQuestion}

}
\only<2>{
\begin{QQuestion}{AH303}{Was ist die Ursache für Aurora-Erscheinungen?}{\textbf{\textcolor{DARCgreen}{Das Eindringen geladener Teilchen von der Sonne in die Atmosphäre der Polarregionen.}}}
{Eine hohe Sonnenfleckenzahl.}
{Eine niedrige Sonnenfleckenzahl.}
{Das Eindringen starker Meteoritenschauer in die Atmosphäre der Polarregionen.}
\end{QQuestion}

}
\end{frame}

\begin{frame}
\only<1>{
\begin{QQuestion}{AH306}{In welche Himmelsrichtung muss eine Funkstation in Europa ihre VHF-Antenne drehen, um eine Verbindung über \glqq Aurora\grqq{} abzuwickeln?}{Süden}
{Norden}
{Osten}
{Westen}
\end{QQuestion}

}
\only<2>{
\begin{QQuestion}{AH306}{In welche Himmelsrichtung muss eine Funkstation in Europa ihre VHF-Antenne drehen, um eine Verbindung über \glqq Aurora\grqq{} abzuwickeln?}{Süden}
{\textbf{\textcolor{DARCgreen}{Norden}}}
{Osten}
{Westen}
\end{QQuestion}

}
\end{frame}

\begin{frame}
\frametitle{Nutzung für Wellenausbreitung}
\end{frame}

\begin{frame}
\only<1>{
\begin{QQuestion}{AH304}{Beim Auftreten von Polarlichtern lassen sich auf den Amateurfunkbändern über \qty{30}{\MHz} beträchtliche Überreichweiten erzielen, weil~...}{stark ionisierte Bereiche auftreten, die Funkwellen reflektieren.}
{starke Magnetfelder auftreten, die Funkwellen reflektieren.}
{starke Inversionsfelder auftreten, die Funkwellen reflektieren.}
{starke sporadische D-Regionen auftreten, die Funkwellen reflektieren.}
\end{QQuestion}

}
\only<2>{
\begin{QQuestion}{AH304}{Beim Auftreten von Polarlichtern lassen sich auf den Amateurfunkbändern über \qty{30}{\MHz} beträchtliche Überreichweiten erzielen, weil~...}{\textbf{\textcolor{DARCgreen}{stark ionisierte Bereiche auftreten, die Funkwellen reflektieren.}}}
{starke Magnetfelder auftreten, die Funkwellen reflektieren.}
{starke Inversionsfelder auftreten, die Funkwellen reflektieren.}
{starke sporadische D-Regionen auftreten, die Funkwellen reflektieren.}
\end{QQuestion}

}
\end{frame}

\begin{frame}
\only<1>{
\begin{QQuestion}{AH305}{Was meint ein Funkamateur damit, wenn er angibt, dass er auf dem \qty{2}{\m}-Band eine Aurora-Verbindung mit Schottland gehabt hat?}{Die Verbindung ist durch Verstärkung der polaren Nordlichter mittels Ultrakurzwellen zustande gekommen (Reflexion von ionisiertem Polarlicht).}
{Die Verbindung ist durch Beugung von Ultrakurzwellen an Lichtquellen der Polarregion zustande gekommen (Beugung an ionisierten Polarschichten).}
{Die Verbindung ist durch Reflexion von Ultrakurzwellen an polaren Nordlichtern zustande gekommen (Reflexion an polaren Ionisationserscheinungen).}
{Die Verbindung ist durch Reflexion von verbrummten Ultrakurzwellen am Polarkreis zustande gekommen (Reflexion an Ionisationserscheinungen des Polarkreises).}
\end{QQuestion}

}
\only<2>{
\begin{QQuestion}{AH305}{Was meint ein Funkamateur damit, wenn er angibt, dass er auf dem \qty{2}{\m}-Band eine Aurora-Verbindung mit Schottland gehabt hat?}{Die Verbindung ist durch Verstärkung der polaren Nordlichter mittels Ultrakurzwellen zustande gekommen (Reflexion von ionisiertem Polarlicht).}
{Die Verbindung ist durch Beugung von Ultrakurzwellen an Lichtquellen der Polarregion zustande gekommen (Beugung an ionisierten Polarschichten).}
{\textbf{\textcolor{DARCgreen}{Die Verbindung ist durch Reflexion von Ultrakurzwellen an polaren Nordlichtern zustande gekommen (Reflexion an polaren Ionisationserscheinungen).}}}
{Die Verbindung ist durch Reflexion von verbrummten Ultrakurzwellen am Polarkreis zustande gekommen (Reflexion an Ionisationserscheinungen des Polarkreises).}
\end{QQuestion}

}
\end{frame}

\begin{frame}
\only<1>{
\begin{QQuestion}{AH307}{Welches der folgenden Übertragungsverfahren eignet sich am besten für Auroraverbindungen?}{SSB}
{CW}
{FM}
{RTTY}
\end{QQuestion}

}
\only<2>{
\begin{QQuestion}{AH307}{Welches der folgenden Übertragungsverfahren eignet sich am besten für Auroraverbindungen?}{SSB}
{\textbf{\textcolor{DARCgreen}{CW}}}
{FM}
{RTTY}
\end{QQuestion}

}
\end{frame}

\begin{frame}
\only<1>{
\begin{QQuestion}{AH308}{Wie wirkt sich \glqq Aurora\grqq{} auf die Signalqualität eines Funksignals aus?}{CW-Signale haben einen flatternden und verbrummten Ton.}
{CW-Signale haben einen besseren Ton.}
{Die Lesbarkeit von Fonie-Signalen verbessert sich.}
{CW- und Fonie-Signale haben ein Echo.}
\end{QQuestion}

}
\only<2>{
\begin{QQuestion}{AH308}{Wie wirkt sich \glqq Aurora\grqq{} auf die Signalqualität eines Funksignals aus?}{\textbf{\textcolor{DARCgreen}{CW-Signale haben einen flatternden und verbrummten Ton.}}}
{CW-Signale haben einen besseren Ton.}
{Die Lesbarkeit von Fonie-Signalen verbessert sich.}
{CW- und Fonie-Signale haben ein Echo.}
\end{QQuestion}

}
\end{frame}%ENDCONTENT


\section{Sporadic-E III}
\label{section:sporadic_e_3}
\begin{frame}%STARTCONTENT

\frametitle{Foliensatz in Arbeit}
2024-04-28: Die Inhalte werden noch aufbereitet.

Derzeit sind in diesem Abschnitt nur die Fragen sortiert enthalten.

Für das Selbststudium verweisen wir aktuell auf den Abschnitt Wellenausbreitung im DARC Online Lehrgang (\textcolor{DARCblue}{\faLink~\href{https://www.darc.de/der-club/referate/ajw/lehrgang-te/e09/}{www.darc.de/der-club/referate/ajw/lehrgang-te/e09/}}) für die Prüfung bis Juni 2024. Bis auf die Fragen hat sich an der Thematik nichts geändert. Das Thema war bisher Stoff der Klasse~E und wurde mit der neuen Prüfungsordnung auf alle drei Klassen aufgeteilt.

\end{frame}

\begin{frame}
\only<1>{
\begin{QQuestion}{AH301}{Bei \glqq Sporadic~E\grqq{}-Ausbreitung werden Wellen im VHF-Bereich gebrochen an~...}{besonders stark ionisierten Bereichen der E-Region.}
{Inversionen am unteren Rand der E-Region.}
{geomagnetischen Störungen am unteren Rand der E-Region.}
{Ionisationsspuren von Meteoriten in der E-Region.}
\end{QQuestion}

}
\only<2>{
\begin{QQuestion}{AH301}{Bei \glqq Sporadic~E\grqq{}-Ausbreitung werden Wellen im VHF-Bereich gebrochen an~...}{\textbf{\textcolor{DARCgreen}{besonders stark ionisierten Bereichen der E-Region.}}}
{Inversionen am unteren Rand der E-Region.}
{geomagnetischen Störungen am unteren Rand der E-Region.}
{Ionisationsspuren von Meteoriten in der E-Region.}
\end{QQuestion}

}
\end{frame}

\begin{frame}
\only<1>{
\begin{QQuestion}{AH214}{Wie groß ist in etwa die maximale Entfernung, die ein KW-Signal bei Refraktion (Brechung) in der E-Region auf der Erdoberfläche mit einem Sprung (Hop) überbrücken kann? Sie beträgt etwa~...}{\qty{9000}{\km}}
{\qty{1100}{\km}}
{\qty{4500}{\km}}
{\qty{2200}{\km}}
\end{QQuestion}

}
\only<2>{
\begin{QQuestion}{AH214}{Wie groß ist in etwa die maximale Entfernung, die ein KW-Signal bei Refraktion (Brechung) in der E-Region auf der Erdoberfläche mit einem Sprung (Hop) überbrücken kann? Sie beträgt etwa~...}{\qty{9000}{\km}}
{\qty{1100}{\km}}
{\qty{4500}{\km}}
{\textbf{\textcolor{DARCgreen}{\qty{2200}{\km}}}}
\end{QQuestion}

}

\end{frame}

\begin{frame}
\only<1>{
\begin{QQuestion}{AH220}{Wie wirkt sich \glqq Sporadic~E\grqq{} auf die höheren Kurzwellenbänder aus?}{Bei Überseeverbindungen tritt Flatterfading auf.}
{Die Signale werden stark verbrummt empfangen.}
{Die \glqq tote Zone\grqq{} wird reduziert oder verschwindet ganz.}
{Die ionosphärische Ausbreitung fällt komplett aus.}
\end{QQuestion}

}
\only<2>{
\begin{QQuestion}{AH220}{Wie wirkt sich \glqq Sporadic~E\grqq{} auf die höheren Kurzwellenbänder aus?}{Bei Überseeverbindungen tritt Flatterfading auf.}
{Die Signale werden stark verbrummt empfangen.}
{\textbf{\textcolor{DARCgreen}{Die \glqq tote Zone\grqq{} wird reduziert oder verschwindet ganz.}}}
{Die ionosphärische Ausbreitung fällt komplett aus.}
\end{QQuestion}

}
\end{frame}%ENDCONTENT


\section{Ionosphäre III}
\label{section:ionosphaere_3}
\begin{frame}%STARTCONTENT

\frametitle{Foliensatz in Arbeit}
2024-04-28: Die Inhalte werden noch aufbereitet.

Derzeit sind in diesem Abschnitt nur die Fragen sortiert enthalten.

Für das Selbststudium verweisen wir aktuell auf den Abschnitt Wellenausbreitung im DARC Online Lehrgang (\textcolor{DARCblue}{\faLink~\href{https://www.darc.de/der-club/referate/ajw/lehrgang-te/e09/}{www.darc.de/der-club/referate/ajw/lehrgang-te/e09/}}) für die Prüfung bis Juni 2024. Bis auf die Fragen hat sich an der Thematik nichts geändert. Das Thema war bisher Stoff der Klasse~E und wurde mit der neuen Prüfungsordnung auf alle drei Klassen aufgeteilt.

\end{frame}

\begin{frame}
\frametitle{Solarer Flux}

\only<1>{
\begin{QQuestion}{AH102}{Der solare Flux~F~...}{ist die gemessene Energieausstrahlung der Sonne im GHz-Bereich. Fluxwerte über 100 führen zu einem stark erhöhten Ionisationsgrad in der Ionosphäre und zu einer erheblich verbesserten Fernausbreitung auf den höheren Kurzwellenbändern.}
{ist die gemessene Energieausstrahlung der Sonne im Kurzwellenbereich. Fluxwerte über 60 führen zu einem stark erhöhten Ionisationsgrad in der Ionosphäre und zu einer erheblich verbesserten Fernausbreitung auf den höheren Kurzwellenbändern.}
{wird aus der Sonnenfleckenrelativzahl R abgeleitet und ist ein Indikator für die Aktivität der Sonne. Fluxwerte über 100 führen zu einem stark erhöhten Ionisationsgrad der D-Region und damit zu einer erheblichen Verschlechterung der Fernausbreitung auf den Kurzwellenbändern.}
{wird aus der Sonnenfleckenrelativzahl R abgeleitet und ist ein Indikator für die Aktivität der Sonne. Fluxwerte über 60 führen zu einem stark erhöhten Ionisationsgrad in der Ionosphäre und zu einer erheblich verbesserten Fernausbreitung auf den höheren Kurzwellenbändern.}
\end{QQuestion}

}
\only<2>{
\begin{QQuestion}{AH102}{Der solare Flux~F~...}{\textbf{\textcolor{DARCgreen}{ist die gemessene Energieausstrahlung der Sonne im GHz-Bereich. Fluxwerte über 100 führen zu einem stark erhöhten Ionisationsgrad in der Ionosphäre und zu einer erheblich verbesserten Fernausbreitung auf den höheren Kurzwellenbändern.}}}
{ist die gemessene Energieausstrahlung der Sonne im Kurzwellenbereich. Fluxwerte über 60 führen zu einem stark erhöhten Ionisationsgrad in der Ionosphäre und zu einer erheblich verbesserten Fernausbreitung auf den höheren Kurzwellenbändern.}
{wird aus der Sonnenfleckenrelativzahl R abgeleitet und ist ein Indikator für die Aktivität der Sonne. Fluxwerte über 100 führen zu einem stark erhöhten Ionisationsgrad der D-Region und damit zu einer erheblichen Verschlechterung der Fernausbreitung auf den Kurzwellenbändern.}
{wird aus der Sonnenfleckenrelativzahl R abgeleitet und ist ein Indikator für die Aktivität der Sonne. Fluxwerte über 60 führen zu einem stark erhöhten Ionisationsgrad in der Ionosphäre und zu einer erheblich verbesserten Fernausbreitung auf den höheren Kurzwellenbändern.}
\end{QQuestion}

}
\end{frame}

\begin{frame}
\frametitle{Wellenausbreitung an D- und E-Schicht}
\end{frame}

\begin{frame}
\only<1>{
\begin{QQuestion}{AH103}{In welcher Höhe befindet sich die für die Fernausbreitung wichtige D-Region? Sie befindet sich in ungefähr~...}{\qtyrange{130}{200}{\km} Höhe.}
{\qtyrange{9}{130}{\km} Höhe.}
{\qtyrange{50}{90}{\km} Höhe.}
{\qtyrange{250}{450}{\km} Höhe.}
\end{QQuestion}

}
\only<2>{
\begin{QQuestion}{AH103}{In welcher Höhe befindet sich die für die Fernausbreitung wichtige D-Region? Sie befindet sich in ungefähr~...}{\qtyrange{130}{200}{\km} Höhe.}
{\qtyrange{9}{130}{\km} Höhe.}
{\textbf{\textcolor{DARCgreen}{\qtyrange{50}{90}{\km} Höhe.}}}
{\qtyrange{250}{450}{\km} Höhe.}
\end{QQuestion}

}
\end{frame}

\begin{frame}
\only<1>{
\begin{QQuestion}{AH104}{In welcher Höhe befindet sich die für die Fernausbreitung wichtige E-Region? Sie befindet sich in ungefähr~...}{\qtyrange{90}{130}{\km} Höhe.}
{\qtyrange{50}{90}{\km} Höhe.}
{\qtyrange{130}{200}{\km} Höhe.}
{\qtyrange{250}{450}{\km} Höhe.}
\end{QQuestion}

}
\only<2>{
\begin{QQuestion}{AH104}{In welcher Höhe befindet sich die für die Fernausbreitung wichtige E-Region? Sie befindet sich in ungefähr~...}{\textbf{\textcolor{DARCgreen}{\qtyrange{90}{130}{\km} Höhe.}}}
{\qtyrange{50}{90}{\km} Höhe.}
{\qtyrange{130}{200}{\km} Höhe.}
{\qtyrange{250}{450}{\km} Höhe.}
\end{QQuestion}

}
\end{frame}

\begin{frame}
\only<1>{
\begin{QQuestion}{AH202}{Welches dieser Frequenzbänder kann im Sonnenfleckenminimum am ehesten für tägliche Weitverkehrsverbindungen verwendet werden?}{\qty{3,5}{\MHz}~(\qty{80}{\m}-Band)}
{\qty{14}{\MHz}~(\qty{20}{\m}-Band)}
{\qty{28}{\MHz}~(\qty{10}{\m}-Band)}
{\qty{1,8}{\MHz}~(\qty{160}{\m}-Band)}
\end{QQuestion}

}
\only<2>{
\begin{QQuestion}{AH202}{Welches dieser Frequenzbänder kann im Sonnenfleckenminimum am ehesten für tägliche Weitverkehrsverbindungen verwendet werden?}{\qty{3,5}{\MHz}~(\qty{80}{\m}-Band)}
{\textbf{\textcolor{DARCgreen}{\qty{14}{\MHz}~(\qty{20}{\m}-Band)}}}
{\qty{28}{\MHz}~(\qty{10}{\m}-Band)}
{\qty{1,8}{\MHz}~(\qty{160}{\m}-Band)}
\end{QQuestion}

}
\end{frame}%ENDCONTENT


\section{Tote Zone II}
\label{section:tote_zone_2}
\begin{frame}%STARTCONTENT

\frametitle{Tote Zone}
\begin{columns}
    \begin{column}{0.48\textwidth}
    \begin{itemize}
  \item Je höher die Frequenz, desto größer ist der Radius der toten Zone
  \item Insbesondere auf den höheren Bändern kann es zur Fehlannahme einer freien Frequenz kommen
  \end{itemize}

    \end{column}
   \begin{column}{0.48\textwidth}
       
\begin{figure}
    \DARCimage{0.85\linewidth}{741include}
    \caption{\scriptsize Tote Zone}
    \label{a_tote_zone}
\end{figure}


   \end{column}
\end{columns}

\end{frame}

\begin{frame}
\only<1>{
\begin{QQuestion}{AH215}{Eine Aussendung auf dem \qty{20}{\m}-Band kann von der Funkstelle~A in einer Entfernung von \qty{1500}{\km}, nicht jedoch von der Funkstelle~B in \qty{60}{\km} Entfernung empfangen werden. Der Grund hierfür ist, dass~...}{zwei in verschiedenen ionosphärischen Regionen reflektierte Wellen mit auslöschender Phase bei Funkstelle~B eintreffen.}
{die Boden- und Raumwellen sich bei Funkstelle~B gegenseitig aufheben.}
{die Funkstelle B die Bodenwelle nicht mehr und die Raumwelle noch nicht empfangen kann.}
{bei Funkstelle~B der Mögel-Dellinger-Effekt aufgetreten ist.}
\end{QQuestion}

}
\only<2>{
\begin{QQuestion}{AH215}{Eine Aussendung auf dem \qty{20}{\m}-Band kann von der Funkstelle~A in einer Entfernung von \qty{1500}{\km}, nicht jedoch von der Funkstelle~B in \qty{60}{\km} Entfernung empfangen werden. Der Grund hierfür ist, dass~...}{zwei in verschiedenen ionosphärischen Regionen reflektierte Wellen mit auslöschender Phase bei Funkstelle~B eintreffen.}
{die Boden- und Raumwellen sich bei Funkstelle~B gegenseitig aufheben.}
{\textbf{\textcolor{DARCgreen}{die Funkstelle B die Bodenwelle nicht mehr und die Raumwelle noch nicht empfangen kann.}}}
{bei Funkstelle~B der Mögel-Dellinger-Effekt aufgetreten ist.}
\end{QQuestion}

}
\end{frame}%ENDCONTENT


\section{Interferenz}
\label{section:interferenz}
\begin{frame}%STARTCONTENT

\only<1>{
\begin{QQuestion}{AH222}{Welcher Effekt tritt ein, wenn das Signal eines Senders auf zwei unterschiedlichen Wegen zum Empfänger gelangt?}{Es kommt zu Frequenzveränderungen beider Signale.}
{Es kommt zu Reflexionen der beiden Signale.}
{Es kommt zu Interferenzen der beiden Signale.}
{Es kommt zu Beugungseffekten bei beiden Signalen.}
\end{QQuestion}

}
\only<2>{
\begin{QQuestion}{AH222}{Welcher Effekt tritt ein, wenn das Signal eines Senders auf zwei unterschiedlichen Wegen zum Empfänger gelangt?}{Es kommt zu Frequenzveränderungen beider Signale.}
{Es kommt zu Reflexionen der beiden Signale.}
{\textbf{\textcolor{DARCgreen}{Es kommt zu Interferenzen der beiden Signale.}}}
{Es kommt zu Beugungseffekten bei beiden Signalen.}
\end{QQuestion}

}
\end{frame}%ENDCONTENT


\section{Sprungdistanz II}
\label{section:sprungdistanz_2}
\begin{frame}%STARTCONTENT

\begin{columns}
    \begin{column}{0.48\textwidth}
    \begin{itemize}
  \item Bisher: Sprungdistanz durch Abstrahlwinkel verändern
  \item Auch zu beachten:
  \item \emph{Höhe der ionisierten Region}
  \item \emph{die Tageszeit} wegen der unterschiedlichen Schichten
  \item \emph{genutzte Frequenz} wegen unterschiedlicher Refraktionseigenschaften an den Schichten
  \end{itemize}

    \end{column}
   \begin{column}{0.48\textwidth}
       
   \end{column}
\end{columns}

\end{frame}

\begin{frame}
\only<1>{
\begin{QQuestion}{AH212}{Was hat \underline{keine} Auswirkungen auf die Sprungentfernung?}{Die Änderung der Frequenz des ausgesendeten Signals.}
{Die Änderung der Strahlungsleistung.}
{Die Tageszeit.}
{Die aktuelle Höhe der ionisierten Regionen.}
\end{QQuestion}

}
\only<2>{
\begin{QQuestion}{AH212}{Was hat \underline{keine} Auswirkungen auf die Sprungentfernung?}{Die Änderung der Frequenz des ausgesendeten Signals.}
{\textbf{\textcolor{DARCgreen}{Die Änderung der Strahlungsleistung.}}}
{Die Tageszeit.}
{Die aktuelle Höhe der ionisierten Regionen.}
\end{QQuestion}

}
\end{frame}

\begin{frame}
\only<1>{
\begin{QQuestion}{AH213}{Wie groß ist in etwa die maximale Entfernung, die ein KW-Signal bei Refraktion (Brechung) an der F2-Region auf der Erdoberfläche mit einem Sprung (Hop) überbrücken kann?}{Etwa \qty{8000}{\km}.}
{Etwa \qty{2000}{\km}.}
{Etwa \qty{12000}{\km}.}
{Etwa \qty{4000}{\km}.}
\end{QQuestion}

}
\only<2>{
\begin{QQuestion}{AH213}{Wie groß ist in etwa die maximale Entfernung, die ein KW-Signal bei Refraktion (Brechung) an der F2-Region auf der Erdoberfläche mit einem Sprung (Hop) überbrücken kann?}{Etwa \qty{8000}{\km}.}
{Etwa \qty{2000}{\km}.}
{Etwa \qty{12000}{\km}.}
{\textbf{\textcolor{DARCgreen}{Etwa \qty{4000}{\km}.}}}
\end{QQuestion}

}
\end{frame}%ENDCONTENT


\section{MUF und LUF II}
\label{section:muf_luf_2}
\begin{frame}%STARTCONTENT

\frametitle{Höchste brauchbare Frequenz (MUF)}
\begin{columns}
    \begin{column}{0.48\textwidth}
    \begin{itemize}
  \item Höchste Frequenz mit der eine Verbindung über die Raumwelle hergestellt werden kann
  \item Abhängig vom Abstrahlwinkel
  \item $MUF \approx \frac{f_c}{\sin(\alpha)}$
  \end{itemize}

    \end{column}
   \begin{column}{0.48\textwidth}
       
   \end{column}
\end{columns}

\end{frame}

\begin{frame}
\only<1>{
\begin{QQuestion}{AH206}{Die höchste Frequenz, bei der eine Kommunikation zwischen zwei Funkstellen über Raumwelle möglich ist, wird als~...}{höchste durchlässige Frequenz bezeichnet (LUF).}
{optimale Arbeitsfrequenz bezeichnet (f$_{opt}$, FOT).
}
{kritische Frequenz bezeichnet (f$_{krit}$, foF2).}
{höchste nutzbare Frequenz bezeichnet (MUF).}
\end{QQuestion}

}
\only<2>{
\begin{QQuestion}{AH206}{Die höchste Frequenz, bei der eine Kommunikation zwischen zwei Funkstellen über Raumwelle möglich ist, wird als~...}{höchste durchlässige Frequenz bezeichnet (LUF).}
{optimale Arbeitsfrequenz bezeichnet (f$_{opt}$, FOT).
}
{kritische Frequenz bezeichnet (f$_{krit}$, foF2).}
{\textbf{\textcolor{DARCgreen}{höchste nutzbare Frequenz bezeichnet (MUF).}}}
\end{QQuestion}

}
\end{frame}

\begin{frame}
\only<1>{
\begin{QQuestion}{AH207}{Wenn sich elektromagnetische Wellen zwischen zwei Orten durch ionosphärische Brechung ausbreiten, dann ist die MUF~...}{die vorgeschriebene nutzbare Frequenz.}
{der Mittelwert aus der höchsten und niedrigsten brauchbaren Frequenz.}
{die niedrigste brauchbare Frequenz.}
{die höchste brauchbare Frequenz.}
\end{QQuestion}

}
\only<2>{
\begin{QQuestion}{AH207}{Wenn sich elektromagnetische Wellen zwischen zwei Orten durch ionosphärische Brechung ausbreiten, dann ist die MUF~...}{die vorgeschriebene nutzbare Frequenz.}
{der Mittelwert aus der höchsten und niedrigsten brauchbaren Frequenz.}
{die niedrigste brauchbare Frequenz.}
{\textbf{\textcolor{DARCgreen}{die höchste brauchbare Frequenz.}}}
\end{QQuestion}

}
\end{frame}

\begin{frame}
\frametitle{Kritische Frequenz}
\begin{itemize}
  \item Bei \qty{90}{\degree} Abstrahlwinkel muss das Signal in der Ionosphäre eine \qty{180}{\degree}-Wendung vollziehen
  \item Kritische Frequenz f<sub>c</sub> bei der das Signal reflektiert wird
  \item MUF liefgt höher als f<sub>c</sub>, da in der Regel nicht senkrecht nach oben gesendet wird
  \end{itemize}

\end{frame}

\begin{frame}
\only<1>{
\begin{QQuestion}{AH208}{Die höchste brauchbare Frequenz (MUF) für eine Funkstrecke~...}{liegt tiefer als die kritische Frequenz, und zwar um so mehr, je steiler die Sendeantenne abstrahlt bzw. die Empfangsantenne aufnimmt.}
{liegt tiefer als die kritische Frequenz, und zwar um so mehr, je flacher die Sendeantenne abstrahlt bzw. die Empfangsantenne aufnimmt.}
{liegt höher als die kritische Frequenz, und zwar um so mehr, je flacher die Sendeantenne abstrahlt bzw. die Empfangsantenne aufnimmt.}
{ist nicht davon abhängig, wie flach die Sendeantenne abstrahlt bzw. die Empfangsantenne aufnimmt, sondern nur vom Zustand der Ionosphäre.}
\end{QQuestion}

}
\only<2>{
\begin{QQuestion}{AH208}{Die höchste brauchbare Frequenz (MUF) für eine Funkstrecke~...}{liegt tiefer als die kritische Frequenz, und zwar um so mehr, je steiler die Sendeantenne abstrahlt bzw. die Empfangsantenne aufnimmt.}
{liegt tiefer als die kritische Frequenz, und zwar um so mehr, je flacher die Sendeantenne abstrahlt bzw. die Empfangsantenne aufnimmt.}
{\textbf{\textcolor{DARCgreen}{liegt höher als die kritische Frequenz, und zwar um so mehr, je flacher die Sendeantenne abstrahlt bzw. die Empfangsantenne aufnimmt.}}}
{ist nicht davon abhängig, wie flach die Sendeantenne abstrahlt bzw. die Empfangsantenne aufnimmt, sondern nur vom Zustand der Ionosphäre.}
\end{QQuestion}

}
\end{frame}

\begin{frame}
\frametitle{Optimale Frequenz}
\begin{itemize}
  \item Kommerzielle Frequenzplanung verwendet eine \emph{Frequency of optimal transmition}, optimale Sendefrequenz
  \item Frequenz, die auf einem bestimmten Signalweg statistisch an \qty{90}{\percent} aller Tage eine Funkverbindung erlaubt
  \item Liegt \qty{15}{\percent} unter dem monatlichen Mittel der MUF
  \item $f_{\textrm{opt}} = \textrm{MUF}\cdot 0,85$
  \item Spielt für Amateurfunk keine große Rolle, da keine dauerhafte Verbindung aufgebaut wird
  \item Im Amateurfunk wird bis nahe an der MUF gearbeitet
  \end{itemize}
\end{frame}

\begin{frame}
\only<1>{
\begin{QQuestion}{AH209}{Wie groß ist die höchste nutzbare Frequenz (MUF) und die optimale Frequenz $f_{\symup{opt}}$, wenn die Antenne in einem Winkel von $45^\circ$ schräg nach oben strahlt und die kritische Frequenz $f_{k}$ \qty{3}{\MHz} beträgt?}{Die MUF liegt bei \qty{2,1}{\MHz} und $f_{\symup{opt}}$ bei \qty{2,5}{\MHz}.}
{Die MUF liegt bei \qty{2,1}{\MHz} und $f_{\symup{opt}}$ bei \qty{1,8}{\MHz}.}
{Die MUF liegt bei \qty{4,2}{\MHz} und $f_{\symup{opt}}$ bei \qty{3,6}{\MHz}.}
{Die MUF liegt bei \qty{4,2}{\MHz} und $f_{\symup{opt}}$ bei \qty{4,9}{\MHz}.}
\end{QQuestion}

}
\only<2>{
\begin{QQuestion}{AH209}{Wie groß ist die höchste nutzbare Frequenz (MUF) und die optimale Frequenz $f_{\symup{opt}}$, wenn die Antenne in einem Winkel von $45^\circ$ schräg nach oben strahlt und die kritische Frequenz $f_{k}$ \qty{3}{\MHz} beträgt?}{Die MUF liegt bei \qty{2,1}{\MHz} und $f_{\symup{opt}}$ bei \qty{2,5}{\MHz}.}
{Die MUF liegt bei \qty{2,1}{\MHz} und $f_{\symup{opt}}$ bei \qty{1,8}{\MHz}.}
{\textbf{\textcolor{DARCgreen}{Die MUF liegt bei \qty{4,2}{\MHz} und $f_{\symup{opt}}$ bei \qty{3,6}{\MHz}.}}}
{Die MUF liegt bei \qty{4,2}{\MHz} und $f_{\symup{opt}}$ bei \qty{4,9}{\MHz}.}
\end{QQuestion}

}
\end{frame}

\begin{frame}
\frametitle{Lösungsweg}
\begin{columns}
    \begin{column}{0.48\textwidth}
    \begin{itemize}
  \item gegeben: $\alpha = 45\degree$
  \item gegeben: $f_c = 3MHz$
  \end{itemize}

    \end{column}
   \begin{column}{0.48\textwidth}
       \begin{itemize}
  \item gesucht: MUF
  \item gesucht: $f_{\textrm{opt}}$
  \end{itemize}

   \end{column}
\end{columns}
\begin{columns}
    \begin{column}{0.48\textwidth}
    
    \pause
    \begin{equation}\begin{split} \nonumber MUF &\approx \frac{f_c}{\sin(\alpha)}\\ &\approx \frac{3MHz}{0,71}\\ &\approx 4,2MHz \end{split}\end{equation}




    \end{column}
   \begin{column}{0.48\textwidth}
       
    \pause
    \begin{equation}\begin{split} \nonumber f_{\textrm{opt}} &= \textrm{MUF}\cdot 0,85\\ &= 4,2MHz \cdot 0,85\\ &= 3,6MHz \end{split}\end{equation}




   \end{column}
\end{columns}

\end{frame}

\begin{frame}
\frametitle{Niedrigste brauchbare Frequenz (LUF)}
Niedrigste Frequenz mit der eine Verbindung über die Raumwelle hergestellt werden kann

\end{frame}

\begin{frame}
\only<1>{
\begin{QQuestion}{AH210}{Die LUF für eine Funkstrecke ist~...}{die niedrigste brauchbare Frequenz, bei der eine Verbindung über die Raumwelle hergestellt werden kann.}
{der Mittelwert der höchsten und niedrigsten brauchbaren Frequenz, bei der eine Verbindung über die Raumwelle hergestellt werden kann.}
{die gemessene brauchbare Frequenz, bei der eine Verbindung über die Raumwelle hergestellt werden kann.}
{die brauchbarste Frequenz, bei der eine Verbindung über die Raumwelle hergestellt werden kann.}
\end{QQuestion}

}
\only<2>{
\begin{QQuestion}{AH210}{Die LUF für eine Funkstrecke ist~...}{\textbf{\textcolor{DARCgreen}{die niedrigste brauchbare Frequenz, bei der eine Verbindung über die Raumwelle hergestellt werden kann.}}}
{der Mittelwert der höchsten und niedrigsten brauchbaren Frequenz, bei der eine Verbindung über die Raumwelle hergestellt werden kann.}
{die gemessene brauchbare Frequenz, bei der eine Verbindung über die Raumwelle hergestellt werden kann.}
{die brauchbarste Frequenz, bei der eine Verbindung über die Raumwelle hergestellt werden kann.}
\end{QQuestion}

}
\end{frame}

\begin{frame}
\only<1>{
\begin{QQuestion}{AH211}{Was bedeutet die Aussage: \glqq Die LUF für eine Funkstrecke liegt bei \qty{6}{\MHz}\grqq{}?}{Die mittlere Frequenz, die für Verbindungen über die Raumwelle genutzt werden kann, liegt bei \qty{6}{\MHz}.}
{Die höchste Frequenz, die für Verbindungen über die Raumwelle als noch brauchbar angesehen wird, liegt bei \qty{6}{\MHz}.}
{Die niedrigste Frequenz, die für Verbindungen über die Raumwelle als noch brauchbar angesehen wird, liegt bei \qty{6}{\MHz}.}
{Die optimale Frequenz, die für Verbindungen über die Raumwelle genutzt werden kann, liegt bei \qty{6}{\MHz}.}
\end{QQuestion}

}
\only<2>{
\begin{QQuestion}{AH211}{Was bedeutet die Aussage: \glqq Die LUF für eine Funkstrecke liegt bei \qty{6}{\MHz}\grqq{}?}{Die mittlere Frequenz, die für Verbindungen über die Raumwelle genutzt werden kann, liegt bei \qty{6}{\MHz}.}
{Die höchste Frequenz, die für Verbindungen über die Raumwelle als noch brauchbar angesehen wird, liegt bei \qty{6}{\MHz}.}
{\textbf{\textcolor{DARCgreen}{Die niedrigste Frequenz, die für Verbindungen über die Raumwelle als noch brauchbar angesehen wird, liegt bei \qty{6}{\MHz}.}}}
{Die optimale Frequenz, die für Verbindungen über die Raumwelle genutzt werden kann, liegt bei \qty{6}{\MHz}.}
\end{QQuestion}

}
\end{frame}%ENDCONTENT


\section{Kritische Frequenz}
\label{section:kritische_frequenz}
\begin{frame}%STARTCONTENT

\frametitle{Kritische Frequenz}
Wiederholung:

\begin{itemize}
  \item Bei \qty{90}{\degree} Abstrahlwinkel muss das Signal in der Ionosphäre eine \qty{180}{\degree}-Wendung vollziehen
  \item Kritische Frequenz f<sub>c</sub> bei der das Signal reflektiert wird
  \item MUF liefgt höher als f<sub>c</sub>, da in der Regel nicht senkrecht nach oben gesendet wird
  \end{itemize}

\end{frame}

\begin{frame}\begin{itemize}
  \item Kritische Frequenz ist je nach ionosphärische Region, dem Ort und der Zeit unterschiedlich
  \item Formelzeichen: f<sub>O</sub>
  \item Ergänzt durch die Schicht, z.B. f<sub>O</sub>F2
  \end{itemize}

\end{frame}

\begin{frame}
\only<1>{
\begin{QQuestion}{AH204}{Die kritische Frequenz der F2-Region (foF2) ist die~...}{niedrigste Frequenz, die bei senkrechter Abstrahlung von der F2-Region noch zur Erde zurückgeworfen wird.}
{höchste Frequenz, die bei senkrechter Abstrahlung von der F2-Region noch zur Erde zurückgeworfen wird.}
{höchste Frequenz, die bei waagerechter Abstrahlung von der F2-Region noch zur Erde zurückgeworfen wird.}
{niedrigste Frequenz, die bei waagerechter Abstrahlung von der F2-Region noch zur Erde zurückgeworfen wird.}
\end{QQuestion}

}
\only<2>{
\begin{QQuestion}{AH204}{Die kritische Frequenz der F2-Region (foF2) ist die~...}{niedrigste Frequenz, die bei senkrechter Abstrahlung von der F2-Region noch zur Erde zurückgeworfen wird.}
{\textbf{\textcolor{DARCgreen}{höchste Frequenz, die bei senkrechter Abstrahlung von der F2-Region noch zur Erde zurückgeworfen wird.}}}
{höchste Frequenz, die bei waagerechter Abstrahlung von der F2-Region noch zur Erde zurückgeworfen wird.}
{niedrigste Frequenz, die bei waagerechter Abstrahlung von der F2-Region noch zur Erde zurückgeworfen wird.}
\end{QQuestion}

}
\end{frame}

\begin{frame}
\only<1>{
\begin{QQuestion}{AH205}{Angenommen, die kritische Frequenz der F2-Region (foF2) liegt bei \qty{12}{\MHz}. Welche Aussage ist dann richtig? Bei Einstrahlung in die Ionosphäre unter einem Winkel von~...}{\qty{45}{\degree} liegt die niedrigste noch zur Erde zurückgeworfene Signalfrequenz bei \qty{12}{\MHz}.}
{\qty{90}{\degree} liegt die niedrigste noch zur Erde zurückgeworfene Signalfrequenz bei \qty{12}{\MHz}.}
{\qty{45}{\degree} liegt die höchste noch zur Erde zurückgeworfene Signalfrequenz bei \qty{12}{\MHz}}
{\qty{90}{\degree} liegt die höchste noch zur Erde zurückgeworfene Signalfrequenz bei \qty{12}{\MHz}.}
\end{QQuestion}

}
\only<2>{
\begin{QQuestion}{AH205}{Angenommen, die kritische Frequenz der F2-Region (foF2) liegt bei \qty{12}{\MHz}. Welche Aussage ist dann richtig? Bei Einstrahlung in die Ionosphäre unter einem Winkel von~...}{\qty{45}{\degree} liegt die niedrigste noch zur Erde zurückgeworfene Signalfrequenz bei \qty{12}{\MHz}.}
{\qty{90}{\degree} liegt die niedrigste noch zur Erde zurückgeworfene Signalfrequenz bei \qty{12}{\MHz}.}
{\qty{45}{\degree} liegt die höchste noch zur Erde zurückgeworfene Signalfrequenz bei \qty{12}{\MHz}}
{\textbf{\textcolor{DARCgreen}{\qty{90}{\degree} liegt die höchste noch zur Erde zurückgeworfene Signalfrequenz bei \qty{12}{\MHz}.}}}
\end{QQuestion}

}
\end{frame}%ENDCONTENT


\section{Langer und kurzer Weg II}
\label{section:langer_kurzer_weg_2}
\begin{frame}%STARTCONTENT
\begin{itemize}
  \item Bei einer Richtantenne ist der Drehwinkel der Hauptstrahlrichtung entscheidend für das zu erreichende Funkziel
  \item Eine geradlinige Verbindung zwischen zwei Orten auf einer Kugel verläuft immer entlang des Großkreises
  \item Ein anderer Ort kann somit über zwei Drehrichtungen erreicht werden
  \item Die Strecke ist dabei unterschiedlich lang
  \item Der Drehwinkel unterscheidet sich dabei um \qty{180}{\degree}
  \item Beispiel: von Berlin nach Sidney/Australien ist der kurze Weg bei \qty{315}{\degree}, der lange Weg bei \qty{75}{\degree}
  \end{itemize}
\end{frame}

\begin{frame}
\only<1>{
\begin{QQuestion}{AH216}{Wie erkennt ein Funkamateur in der Regel, dass er mit \glqq PY\grqq{} auf dem indirekten und somit längeren Weg gearbeitet hat?}{Aus der Stellung seiner Richtantenne erkennt er, dass diese der Richtung des kürzesten Weges nach Brasilien um \qty{180}{\degree} entgegengesetzt ist. Das heißt, er hat \glqq PY\grqq{} auf dem \glqq langen Weg\grqq{} gearbeitet.}
{Durch die verhallte Tonlage der Verbindung erkennt er, dass diese in zwei Richtungen nach Brasilien stattgefunden hat. Das heißt, er hat \glqq PY\grqq{} nicht nur direkt, sondern auf einem längeren Weg gearbeitet.}
{Aus der Stellung seiner Richtantenne erkennt er, dass diese in Richtung des längeren Weges nach Brasilien eingesetzt ist. Das heißt, er hat \glqq PY\grqq{} auf dem direkten Weg gearbeitet.}
{Durch die verhallte Tonlage der Verbindung nach Brasilien, Ausbreitung der Funkwellen über zwei entgegengesetzte Wege.}
\end{QQuestion}

}
\only<2>{
\begin{QQuestion}{AH216}{Wie erkennt ein Funkamateur in der Regel, dass er mit \glqq PY\grqq{} auf dem indirekten und somit längeren Weg gearbeitet hat?}{\textbf{\textcolor{DARCgreen}{Aus der Stellung seiner Richtantenne erkennt er, dass diese der Richtung des kürzesten Weges nach Brasilien um \qty{180}{\degree} entgegengesetzt ist. Das heißt, er hat \glqq PY\grqq{} auf dem \glqq langen Weg\grqq{} gearbeitet.}}}
{Durch die verhallte Tonlage der Verbindung erkennt er, dass diese in zwei Richtungen nach Brasilien stattgefunden hat. Das heißt, er hat \glqq PY\grqq{} nicht nur direkt, sondern auf einem längeren Weg gearbeitet.}
{Aus der Stellung seiner Richtantenne erkennt er, dass diese in Richtung des längeren Weges nach Brasilien eingesetzt ist. Das heißt, er hat \glqq PY\grqq{} auf dem direkten Weg gearbeitet.}
{Durch die verhallte Tonlage der Verbindung nach Brasilien, Ausbreitung der Funkwellen über zwei entgegengesetzte Wege.}
\end{QQuestion}

}
\end{frame}

\begin{frame}
\frametitle{Rechnung}
Für den langen Weg

\begin{itemize}
  \item Bei Drehwinkel zwischen \qty{0}{\degree} und \qty{180}{\degree}: Drehwinkel + \qty{180}{\degree}
  \item Bei Drehwinkel zwischen \qty{180}{\degree} und \qty{360}{\degree}: Drehwinkel -- \qty{180}{\degree}
  \end{itemize}

\end{frame}

\begin{frame}
\only<1>{
\begin{QQuestion}{AH217}{Eine Amateurfunkstation in Frankfurt/Main will eine Verbindung nach Tokio auf dem langen Weg herstellen. Auf welchen Winkel gegen Nord (Azimut) muss der Funkamateur seinen Kurzwellenbeam drehen, wenn die Beamrichtung für den kurzen Weg \qty{38}{\degree} beträgt? Er muss die Antenne drehen auf~...}{\qty{122}{\degree}}
{\qty{322}{\degree}}
{\qty{218}{\degree}}
{\qty{308}{\degree}}
\end{QQuestion}

}
\only<2>{
\begin{QQuestion}{AH217}{Eine Amateurfunkstation in Frankfurt/Main will eine Verbindung nach Tokio auf dem langen Weg herstellen. Auf welchen Winkel gegen Nord (Azimut) muss der Funkamateur seinen Kurzwellenbeam drehen, wenn die Beamrichtung für den kurzen Weg \qty{38}{\degree} beträgt? Er muss die Antenne drehen auf~...}{\qty{122}{\degree}}
{\qty{322}{\degree}}
{\textbf{\textcolor{DARCgreen}{\qty{218}{\degree}}}}
{\qty{308}{\degree}}
\end{QQuestion}

}
\end{frame}

\begin{frame}
\only<1>{
\begin{QQuestion}{AH218}{Eine Amateurfunkstation in Frankfurt/Main will eine Verbindung nach Buenos Aires auf dem langen Weg herstellen. Auf welchen Winkel gegen Nord (Azimut) muss der Funkamateur seinen Kurzwellenbeam drehen, wenn die Beamrichtung für den kurzen Weg \qty{231}{\degree} beträgt? Er muss die Antenne drehen auf~...}{\qty{51}{\degree}}
{\qty{321}{\degree}}
{\qty{141}{\degree}}
{\qty{129}{\degree}}
\end{QQuestion}

}
\only<2>{
\begin{QQuestion}{AH218}{Eine Amateurfunkstation in Frankfurt/Main will eine Verbindung nach Buenos Aires auf dem langen Weg herstellen. Auf welchen Winkel gegen Nord (Azimut) muss der Funkamateur seinen Kurzwellenbeam drehen, wenn die Beamrichtung für den kurzen Weg \qty{231}{\degree} beträgt? Er muss die Antenne drehen auf~...}{\textbf{\textcolor{DARCgreen}{\qty{51}{\degree}}}}
{\qty{321}{\degree}}
{\qty{141}{\degree}}
{\qty{129}{\degree}}
\end{QQuestion}

}
\end{frame}%ENDCONTENT


\section{Scatter}
\label{section:scatter}
\begin{frame}%STARTCONTENT
\begin{itemize}
  \item \emph{Scatter}: Besondere Formen der Reflexion und Streuung eines Funksignals
  \item Damit können größere Entfernungen überbrückt werden
  \end{itemize}
\end{frame}

\begin{frame}
\frametitle{Regenscatter}
\begin{itemize}
  \item Englisch \emph{Rainscatter}
  \item Streuung an Regentropfen in alle Richtungen (Rayleigh-Streuung)
  \item Tropfengröße muss zur Wellenlänge passen: 6- und 3-cm-Band
  \item Antenne wird auf Regenwolke gehalten
  \item Rauer Ton in SSB- und CW-Signalen (ähnlich Aurora)
  \end{itemize}

\end{frame}

\begin{frame}
\only<1>{
\begin{QQuestion}{AH311}{Um welche Art von Überreichweiten handelt es sich bei Regenscatter (Rainscatter)?}{Streuungen von Mikrowellen, insbesondere im \qty{23}{\cm}-Band, an Regentropfen und Hagelkörnern.}
{Reflexionen in den VHF- und UHF-Bereichen an größeren Regentropfen.}
{Streuungen von Mikrowellen, insbesondere im \qty{3}{\cm}-Band, an Regen- und Gewitterwolken.}
{Reflexionen im \qty{13}{\cm}-Band bei Eisregen.}
\end{QQuestion}

}
\only<2>{
\begin{QQuestion}{AH311}{Um welche Art von Überreichweiten handelt es sich bei Regenscatter (Rainscatter)?}{Streuungen von Mikrowellen, insbesondere im \qty{23}{\cm}-Band, an Regentropfen und Hagelkörnern.}
{Reflexionen in den VHF- und UHF-Bereichen an größeren Regentropfen.}
{\textbf{\textcolor{DARCgreen}{Streuungen von Mikrowellen, insbesondere im \qty{3}{\cm}-Band, an Regen- und Gewitterwolken.}}}
{Reflexionen im \qty{13}{\cm}-Band bei Eisregen.}
\end{QQuestion}

}
\end{frame}

\begin{frame}
\frametitle{Backscatter}
\begin{itemize}
  \item Brechung der Raumwelle zurück zum Empfänger
  \item Vor allem während der Dämmerung
  \item Starke und schnell schwankende Signalstärke (\emph{Flatterfading}, flutter fading)
  \end{itemize}
\end{frame}

\begin{frame}
\only<1>{
\begin{QQuestion}{AH223}{Was ist für ein \glqq Backscatter-Signal\grqq{} charakteristisch?}{hohe Signalstärken}
{Pfeif- und Knattergeräusche}
{schnelle, unregelmäßige Feldstärkeschwankungen (Flatterfading)}
{breitbandiges Rauschen}
\end{QQuestion}

}
\only<2>{
\begin{QQuestion}{AH223}{Was ist für ein \glqq Backscatter-Signal\grqq{} charakteristisch?}{hohe Signalstärken}
{Pfeif- und Knattergeräusche}
{\textbf{\textcolor{DARCgreen}{schnelle, unregelmäßige Feldstärkeschwankungen (Flatterfading)}}}
{breitbandiges Rauschen}
\end{QQuestion}

}
\end{frame}

\begin{frame}
\frametitle{Aircraft-Scatter}
\begin{itemize}
  \item Reflexion (also eigentlich kein Scatter) von VHF, UHF und SHF an Flugzeugen
  \item Flugzeug muss auf Verbindungslinie zwischen Sender und Empfänger sein
  \item Recht kurze Verbindung aufgrund der schnellen Bewegung des Flugzeugs
  \end{itemize}

\end{frame}

\begin{frame}
\only<1>{
\begin{QQuestion}{AH310}{Was versteht man unter Aircraft-Scatter (AS)?}{Überhorizontverbindungen im VHF- und UHF-Bereich durch Reflexionen an Funkfeuern.}
{Das Beobachten des Funkverkehrs von Flugzeugen mit Hilfe von Amateurfunkgeräten und Antennen.}
{Überhorizontverbindungen im VHF-, UHF- und SHF-Bereich durch Reflexion an Flugzeugen.}
{Betrieb einer Amateurfunkstelle an Bord eines Flugzeuges.}
\end{QQuestion}

}
\only<2>{
\begin{QQuestion}{AH310}{Was versteht man unter Aircraft-Scatter (AS)?}{Überhorizontverbindungen im VHF- und UHF-Bereich durch Reflexionen an Funkfeuern.}
{Das Beobachten des Funkverkehrs von Flugzeugen mit Hilfe von Amateurfunkgeräten und Antennen.}
{\textbf{\textcolor{DARCgreen}{Überhorizontverbindungen im VHF-, UHF- und SHF-Bereich durch Reflexion an Flugzeugen.}}}
{Betrieb einer Amateurfunkstelle an Bord eines Flugzeuges.}
\end{QQuestion}

}
\end{frame}%ENDCONTENT


\title{DARC Amateurfunklehrgang Klasse A}
\author{Strom, Spannung, Widerstand, Leistung, Energie}
\institute{Deutscher Amateur Radio Club e.\,V.}
\begin{frame}
\maketitle
\end{frame}

\section{Physikalische Stromrichtung}
\label{section:physikalische_stromrichtung}
\begin{frame}%STARTCONTENT
\begin{itemize}
  \item Technische Stromrichtung vom Plus-Pol zum Minus-Pol
  \item In der Wissenschaft hat sich später erst herausgestellt, dass sich in Metallen die negativ geladenen Teilchen (Elekotronen) bewegen
  \item Elektronen werden vom Minus-Pol der Spannungsquelle abgestoßen und vom Plus-Pol angezogen
  \item Die \emph{Physikalische Stromrichtung} ist entegegen gesetzt zur technischen Stromrichtung
  \end{itemize}

\end{frame}

\begin{frame}
\only<1>{
\begin{question2x2}{AB601}{Welches Bild zeigt die physikalische Stromrichtung korrekt an?}{\DARCimage{1.0\linewidth}{646include}}
{\DARCimage{1.0\linewidth}{647include}}
{\DARCimage{1.0\linewidth}{557include}}
{\DARCimage{1.0\linewidth}{558include}}
\end{question2x2}

}
\only<2>{
\begin{question2x2}{AB601}{Welches Bild zeigt die physikalische Stromrichtung korrekt an?}{\textbf{\textcolor{DARCgreen}{\DARCimage{1.0\linewidth}{646include}}}}
{\DARCimage{1.0\linewidth}{647include}}
{\DARCimage{1.0\linewidth}{557include}}
{\DARCimage{1.0\linewidth}{558include}}
\end{question2x2}

}
\end{frame}%ENDCONTENT


\section{Strom- und Spannungsmessung III}
\label{section:strom_spannung_messung_3}
\begin{frame}%STARTCONTENT

\frametitle{Foliensatz in Arbeit}
2024-04-28: Die Inhalte werden noch aufbereitet.

Derzeit sind in diesem Abschnitt nur die Fragen sortiert enthalten.

Für das Selbststudium verweisen wir aktuell auf den Abschnitt Messtechnik im DARC Online Lehrgang (\textcolor{DARCblue}{\faLink~\href{https://www.darc.de/der-club/referate/ajw/lehrgang-ta/a16/}{www.darc.de/der-club/referate/ajw/lehrgang-ta/a16/}}) für die Prüfung bis Juni 2024. Bis auf die Fragen hat sich an der Thematik nichts geändert.

\end{frame}

\begin{frame}
\frametitle{Strom- und Spannungsmessung}
\end{frame}

\begin{frame}
\only<1>{
\begin{PQuestion}{AI101}{Wie sollten Strom- und Spannungsmessgeräte zur Feststellung der Gleichstrom-Eingangsleistung des dargestellten Endverstärkers (PA) angeordnet werden?}{Spannungsmessgerät bei 3, Strommessgerät bei 1.}
{Spannungsmessgerät bei 1, Strommessgerät bei 2.}
{Spannungsmessgerät bei 1, Strommessgerät bei 3.}
{Spannungsmessgerät bei 3, Strommessgerät bei 4.}
{\DARCimage{1.0\linewidth}{443include}}\end{PQuestion}

}
\only<2>{
\begin{PQuestion}{AI101}{Wie sollten Strom- und Spannungsmessgeräte zur Feststellung der Gleichstrom-Eingangsleistung des dargestellten Endverstärkers (PA) angeordnet werden?}{Spannungsmessgerät bei 3, Strommessgerät bei 1.}
{Spannungsmessgerät bei 1, Strommessgerät bei 2.}
{\textbf{\textcolor{DARCgreen}{Spannungsmessgerät bei 1, Strommessgerät bei 3.}}}
{Spannungsmessgerät bei 3, Strommessgerät bei 4.}
{\DARCimage{1.0\linewidth}{443include}}\end{PQuestion}

}
\end{frame}

\begin{frame}
\only<1>{
\begin{PQuestion}{AI102}{Für die Messung der Gleichstrom-Eingangsleistung werden verschiedene Messgeräte verwendet. Bei welchen der Instrumente in der Abbildung handelt es sich um Strommessgeräte?}{2, 3 und 4}
{1, 2 und 3}
{2, 4 und 1}
{1, 3 und 4}
{\DARCimage{1.0\linewidth}{443include}}\end{PQuestion}

}
\only<2>{
\begin{PQuestion}{AI102}{Für die Messung der Gleichstrom-Eingangsleistung werden verschiedene Messgeräte verwendet. Bei welchen der Instrumente in der Abbildung handelt es sich um Strommessgeräte?}{\textbf{\textcolor{DARCgreen}{2, 3 und 4}}}
{1, 2 und 3}
{2, 4 und 1}
{1, 3 und 4}
{\DARCimage{1.0\linewidth}{443include}}\end{PQuestion}

}
\end{frame}

\begin{frame}
\frametitle{Messgenauigkeit}
\end{frame}

\begin{frame}
\only<1>{
\begin{QQuestion}{AI103}{Ein Spannungs- und ein Strommessgerät werden für die Ermittlung der Gleichstromeingangsleistung einer Schaltung verwendet. Das Spannungsmessgerät zeigt \qty{10}{\V}, das Strommessgerät \qty{10}{\A} an. Falls beide dabei im Rahmen ihrer Messgenauigkeit jeweils einen um \qty{5}{\percent} zu geringen Wert anzeigen würden, würde man die elektrische Leistung um~...}{\qty{5}{\percent} zu hoch bestimmen.}
{\qty{5}{\percent} zu niedrig bestimmen.}
{\qty{10,25}{\percent} zu hoch bestimmen.}
{\qty{9,75}{\percent} zu niedrig bestimmen.}
\end{QQuestion}

}
\only<2>{
\begin{QQuestion}{AI103}{Ein Spannungs- und ein Strommessgerät werden für die Ermittlung der Gleichstromeingangsleistung einer Schaltung verwendet. Das Spannungsmessgerät zeigt \qty{10}{\V}, das Strommessgerät \qty{10}{\A} an. Falls beide dabei im Rahmen ihrer Messgenauigkeit jeweils einen um \qty{5}{\percent} zu geringen Wert anzeigen würden, würde man die elektrische Leistung um~...}{\qty{5}{\percent} zu hoch bestimmen.}
{\qty{5}{\percent} zu niedrig bestimmen.}
{\qty{10,25}{\percent} zu hoch bestimmen.}
{\textbf{\textcolor{DARCgreen}{\qty{9,75}{\percent} zu niedrig bestimmen.}}}
\end{QQuestion}

}
\end{frame}

\begin{frame}
\frametitle{Lösungsweg}
\begin{itemize}
  \item Prozentrechnung – die absoluten Werte sind nicht relevant
  \item gegeben: $U_{\textrm{Abw}}$ mit \qty{95}{\percent} vom Realwert
  \item gegeben: $I_{\textrm{Abw}}$ mit \qty{95}{\percent} vom Realwert
  \item gesucht: Abweichung der Leistung $P = U \cdot I$
  \end{itemize}
    \pause
    \begin{equation}\begin{split} \nonumber P_{\textrm{Abw}} &= 100\% -- (U_{\textrm{Abw}} \cdot I_{\textrm{Abw}})\\ &= 100\% -- (95\% \cdot 95\%)\\ &= 100\% -- 90,25\%\\ &= 9,75\% \end{split}\end{equation}



\end{frame}

\begin{frame}
\frametitle{Strom durch Multimeter}
\end{frame}

\begin{frame}
\only<1>{
\begin{QQuestion}{AI104}{Für ein digitales Multimeter ist folgende Angabe im Datenblatt zu finden: Kleinste Auflösung \qty{100}{\micro\V}, Innenwiderstand \qty{10}{\Mohm} in allen Messbereichen. Sie messen eine Spannung von \qty{0,5}{\V}. Welcher Strom fließt dabei durch das Multimeter?}{50~nA}
{10~nA}
{500~nA}
{200~nA}
\end{QQuestion}

}
\only<2>{
\begin{QQuestion}{AI104}{Für ein digitales Multimeter ist folgende Angabe im Datenblatt zu finden: Kleinste Auflösung \qty{100}{\micro\V}, Innenwiderstand \qty{10}{\Mohm} in allen Messbereichen. Sie messen eine Spannung von \qty{0,5}{\V}. Welcher Strom fließt dabei durch das Multimeter?}{\textbf{\textcolor{DARCgreen}{50~nA}}}
{10~nA}
{500~nA}
{200~nA}
\end{QQuestion}

}
\end{frame}

\begin{frame}
\frametitle{Lösungsweg}
\begin{itemize}
  \item gegeben: $U = 0,5V$
  \item gegeben: $R = 10M\Omega$
  \item gesucht: $I$
  \end{itemize}
    \pause
    \begin{equation} \nonumber I = \frac{U}{R} = \frac{0,5V}{10M\Omega} = 50nA \end{equation}



\end{frame}

\begin{frame}
\frametitle{Thermoumformer}
\begin{itemize}
  \item Messgerät, bei dem die abgestrahlte Wärme an einem Widerstand gemessen wird
  \item Aus der abgestrahlten Wärme wird mit einem Thermoelement eine Gleichspannung erzeugt, die gemessen werden kann
  \item Wird dann eingesetzt, wenn eine elektrische Messung nicht möglich ist, z.B. bei nicht-periodischen Signalen
  \end{itemize}
\end{frame}

\begin{frame}
\only<1>{
\begin{QQuestion}{AI105}{Zur genauen Messung der effektiven Leistung eines modulierten Signals bis in den oberen GHz-Bereich eignet sich~...}{ein Digitalmultimeter.}
{ein Oszillograf.}
{ein Messgerät mit Diodentastkopf.}
{ein Messgerät mit Thermoumformer.}
\end{QQuestion}

}
\only<2>{
\begin{QQuestion}{AI105}{Zur genauen Messung der effektiven Leistung eines modulierten Signals bis in den oberen GHz-Bereich eignet sich~...}{ein Digitalmultimeter.}
{ein Oszillograf.}
{ein Messgerät mit Diodentastkopf.}
{\textbf{\textcolor{DARCgreen}{ein Messgerät mit Thermoumformer.}}}
\end{QQuestion}

}
\end{frame}%ENDCONTENT


\section{Oszilloskop II}
\label{section:oszilloskop_2}
\begin{frame}%STARTCONTENT

\frametitle{Foliensatz in Arbeit}
2024-04-28: Die Inhalte werden noch aufbereitet.

Derzeit sind in diesem Abschnitt nur die Fragen sortiert enthalten.

Für das Selbststudium verweisen wir aktuell auf den Abschnitt Wellenausbreitung im DARC Online Lehrgang (\textcolor{DARCblue}{\faLink~\href{https://www.darc.de/der-club/referate/ajw/lehrgang-te/e09/}{www.darc.de/der-club/referate/ajw/lehrgang-te/e09/}}) für die Prüfung bis Juni 2024. Bis auf die Fragen hat sich an der Thematik nichts geändert. Das Thema war bisher Stoff der Klasse~E und wurde mit der neuen Prüfungsordnung auf alle drei Klassen aufgeteilt.

\end{frame}

\begin{frame}
\only<1>{
\begin{QQuestion}{AI301}{Welches Gerät kann für die Prüfung von Signalverläufen verwendet werden?}{Absorptionsfrequenzmesser}
{Oszilloskop}
{Frequenzzähler}
{Dipmeter}
\end{QQuestion}

}
\only<2>{
\begin{QQuestion}{AI301}{Welches Gerät kann für die Prüfung von Signalverläufen verwendet werden?}{Absorptionsfrequenzmesser}
{\textbf{\textcolor{DARCgreen}{Oszilloskop}}}
{Frequenzzähler}
{Dipmeter}
\end{QQuestion}

}
\end{frame}

\begin{frame}
\only<1>{
\begin{QQuestion}{AI302}{Was benötigt ein Oszilloskop zur Darstellung stehender Bilder?}{X-Vorteiler}
{Triggereinrichtung}
{Y-Vorteiler}
{Frequenzmarken-Generator}
\end{QQuestion}

}
\only<2>{
\begin{QQuestion}{AI302}{Was benötigt ein Oszilloskop zur Darstellung stehender Bilder?}{X-Vorteiler}
{\textbf{\textcolor{DARCgreen}{Triggereinrichtung}}}
{Y-Vorteiler}
{Frequenzmarken-Generator}
\end{QQuestion}

}
\end{frame}

\begin{frame}
\only<1>{
\begin{QQuestion}{AI303}{Die Pulsbreite wird mit einem Oszilloskop bei~...}{\qty{90}{\percent} des Spitzenwertes gemessen.}
{\qty{50}{\percent} des Spitzenwertes gemessen.}
{\qty{70}{\percent} des Spitzenwertes gemessen.}
{\qty{10}{\percent} des Spitzenwertes gemessen.}
\end{QQuestion}

}
\only<2>{
\begin{QQuestion}{AI303}{Die Pulsbreite wird mit einem Oszilloskop bei~...}{\qty{90}{\percent} des Spitzenwertes gemessen.}
{\textbf{\textcolor{DARCgreen}{\qty{50}{\percent} des Spitzenwertes gemessen.}}}
{\qty{70}{\percent} des Spitzenwertes gemessen.}
{\qty{10}{\percent} des Spitzenwertes gemessen.}
\end{QQuestion}

}
\end{frame}

\begin{frame}
\only<1>{
\begin{QQuestion}{AI304}{Womit misst man am einfachsten die Hüllkurvenform eines HF-Signals? Mit einem~...}{breitbandigen Detektor und Kopfhörer.}
{hochohmigen Vielfachinstrument in Stellung AC.}
{empfindlichen SWR-Meter in Stellung Wellenmessung.}
{breitbandigen Oszilloskop.}
\end{QQuestion}

}
\only<2>{
\begin{QQuestion}{AI304}{Womit misst man am einfachsten die Hüllkurvenform eines HF-Signals? Mit einem~...}{breitbandigen Detektor und Kopfhörer.}
{hochohmigen Vielfachinstrument in Stellung AC.}
{empfindlichen SWR-Meter in Stellung Wellenmessung.}
{\textbf{\textcolor{DARCgreen}{breitbandigen Oszilloskop.}}}
\end{QQuestion}

}
\end{frame}

\begin{frame}
\only<1>{
\begin{PQuestion}{AI305}{Das folgende Bild zeigt das Zweiton-SSB-Ausgangssignal eines KW-Senders, das mit einem Oszilloskop ausreichender Bandbreite über einen 1:1-Tastkopf direkt an der angeschlossenen künstlichen \qty{50}{\ohm}-Antenne gemessen wurde. Welche Ausgangsleistung (PEP) liefert der Sender?}{\qty{144}{\W}}
{\qty{36}{\W}}
{\qty{100}{\W}}
{\qty{1600}{\W}}
{\DARCimage{1.0\linewidth}{108include}}\end{PQuestion}

}
\only<2>{
\begin{PQuestion}{AI305}{Das folgende Bild zeigt das Zweiton-SSB-Ausgangssignal eines KW-Senders, das mit einem Oszilloskop ausreichender Bandbreite über einen 1:1-Tastkopf direkt an der angeschlossenen künstlichen \qty{50}{\ohm}-Antenne gemessen wurde. Welche Ausgangsleistung (PEP) liefert der Sender?}{\qty{144}{\W}}
{\qty{36}{\W}}
{\textbf{\textcolor{DARCgreen}{\qty{100}{\W}}}}
{\qty{1600}{\W}}
{\DARCimage{1.0\linewidth}{108include}}\end{PQuestion}

}
\end{frame}

\begin{frame}
\frametitle{Lösungsweg}
\begin{itemize}
  \item gegeben: $R=50\Omega$
  \item gegeben: (aus Darstellung) $\^{U} = 100V$
  \item gesucht: $P_{\textrm{PEP}}$
  \end{itemize}
    \pause
    \begin{equation}\begin{split} \nonumber P_{\textrm{PEP}} &= \frac{U_{\textrm{eff}}^2}{R} = \frac{(\frac{100V}{\sqrt{2}})^2}{50\Omega}\\ &=\frac{\frac{(100V)^2}{2}}{50\Omega} = \frac{5000V^2}{50\Omega} = 100W \end{split}\end{equation}



\end{frame}

\begin{frame}
\only<1>{
\begin{PQuestion}{AI306}{Das folgende Bild zeigt das Zweiton-SSB-Ausgangssignal eines KW-Senders, das mit einem Oszilloskop ausreichender Bandbreite über einen 10:1-Tastkopf direkt an der angeschlossenen künstlichen 50 Ohm-Antenne gemessen wurde. Welche Ausgangsleistung (PEP) liefert der Sender?}{\qty{36}{\W}}
{\qty{72}{\W}}
{\qty{144}{\W}}
{\qty{400}{\W}}
{\DARCimage{1.0\linewidth}{43include}}\end{PQuestion}

}
\only<2>{
\begin{PQuestion}{AI306}{Das folgende Bild zeigt das Zweiton-SSB-Ausgangssignal eines KW-Senders, das mit einem Oszilloskop ausreichender Bandbreite über einen 10:1-Tastkopf direkt an der angeschlossenen künstlichen 50 Ohm-Antenne gemessen wurde. Welche Ausgangsleistung (PEP) liefert der Sender?}{\textbf{\textcolor{DARCgreen}{\qty{36}{\W}}}}
{\qty{72}{\W}}
{\qty{144}{\W}}
{\qty{400}{\W}}
{\DARCimage{1.0\linewidth}{43include}}\end{PQuestion}

}
\end{frame}

\begin{frame}
\frametitle{Lösungsweg}
\begin{itemize}
  \item gegeben: $R=50\Omega$
  \item gegeben: (aus Darstellung mit 10:1-Tastkopf) $\^{U} = 6V\cdot 10$
  \item gesucht: $P_{\textrm{PEP}}$
  \end{itemize}
    \pause
    \begin{equation}\begin{split} \nonumber P_{\textrm{PEP}} &= \frac{U_{\textrm{eff}}^2}{R} = \frac{(\frac{6V\cdot 10}{\sqrt{2}})^2}{50\Omega}\\ &=\frac{\frac{(60V)^2}{2}}{50\Omega} = 36W \end{split}\end{equation}



\end{frame}%ENDCONTENT


\section{Leiterwiderstand}
\label{section:leiterwiderstand}
\begin{frame}%STARTCONTENT

\frametitle{Foliensatz in Arbeit}
2024-04-28: Die Inhalte werden noch aufbereitet.

Derzeit sind in diesem Abschnitt nur die Fragen sortiert enthalten.

Für das Selbststudium verweisen wir aktuell auf den Abschnitt Messtechnik im DARC Online Lehrgang (\textcolor{DARCblue}{\faLink~\href{https://www.darc.de/der-club/referate/ajw/lehrgang-ta/a16/}{www.darc.de/der-club/referate/ajw/lehrgang-ta/a16/}}) für die Prüfung bis Juni 2024. Bis auf die Fragen hat sich an der Thematik nichts geändert.

\end{frame}

\begin{frame}
\frametitle{Widerstand von Drähten}
\begin{columns}
    \begin{column}{0.48\textwidth}
    $R = \frac{\rho\cdot l}{A_{\textrm{Dr}}}$

\begin{itemize}
  \item $l$: Drahtlänge
  \item $A_{\textrm{Dr}}$: Drahtquerschnitt
  \item $\rho$: Spezifischer Widerstand in Ωmm<sup>2</sup>/m
  \end{itemize}

    \end{column}
   \begin{column}{0.48\textwidth}
       
    \pause
    Kupfer: 0,018

Aluminium: 0,028

Gold: 0,022

Silber: 0,016

Zink: 0,11

Eisen: 0,1

Messing: 0,07




   \end{column}
\end{columns}

\end{frame}

\begin{frame}
\only<1>{
\begin{QQuestion}{AB101}{Welchen Widerstand hat ein Kupferdraht etwa, wenn der verwendete Draht eine Länge von \qty{1,8}{\m} und einen Durchmesser von \qty{0,2}{\mm} hat?}{\qty{0,26}{\ohm}}
{\qty{56,0}{\ohm}}
{\qty{1,02}{\ohm}}
{\qty{0,16}{\ohm}}
\end{QQuestion}

}
\only<2>{
\begin{QQuestion}{AB101}{Welchen Widerstand hat ein Kupferdraht etwa, wenn der verwendete Draht eine Länge von \qty{1,8}{\m} und einen Durchmesser von \qty{0,2}{\mm} hat?}{\qty{0,26}{\ohm}}
{\qty{56,0}{\ohm}}
{\textbf{\textcolor{DARCgreen}{\qty{1,02}{\ohm}}}}
{\qty{0,16}{\ohm}}
\end{QQuestion}

}
\end{frame}

\begin{frame}
\frametitle{Lösungsweg}
\begin{itemize}
  \item gegeben: $l = 1,8m$
  \item gegeben: $d = 0,2mm$
  \item gegeben: $\rho = 0,018 \frac{\Omega mm^2}{m}$
  \item gesucht: $R$
  \end{itemize}
    \pause
    \begin{equation} \nonumber A_{\textrm{Dr}} = \frac{d^2\cdot \pi}{4} = \frac{(0,2mm)^2 \cdot \pi}{4} = \frac{\pi}{100}mm^2 = 0,0314mm^2 \end{equation}
    \pause
    \begin{equation} \nonumber R = \frac{\rho\cdot l}{A_{\textrm{Dr}}} = \frac{0,018 \frac{\Omega mm^2}{m} \cdot 1,8m}{0,0314mm^2} \approx 1,02\Omega \end{equation}



\end{frame}

\begin{frame}
\only<1>{
\begin{QQuestion}{AB102}{Zwischen den Enden eines Kupferdrahtes mit einem Querschnitt von \qty{0,5}{\mm\squared} messen Sie einen Widerstand von \qty{1,5}{\ohm}. Wie lang ist der Draht etwa?}{\qty{41,7}{\m}}
{\qty{3,0}{\m}}
{\qty{4,2}{\m}}
{\qty{16,5}{\m}}
\end{QQuestion}

}
\only<2>{
\begin{QQuestion}{AB102}{Zwischen den Enden eines Kupferdrahtes mit einem Querschnitt von \qty{0,5}{\mm\squared} messen Sie einen Widerstand von \qty{1,5}{\ohm}. Wie lang ist der Draht etwa?}{\textbf{\textcolor{DARCgreen}{\qty{41,7}{\m}}}}
{\qty{3,0}{\m}}
{\qty{4,2}{\m}}
{\qty{16,5}{\m}}
\end{QQuestion}

}
\end{frame}

\begin{frame}
\frametitle{Lösungsweg}
\begin{itemize}
  \item gegeben: $A_{\textrm{Dr}} = 0,5mm^2$
  \item gegeben: $R = 1,5\Omega$
  \item gegeben: $\rho = 0,018 \frac{\Omega mm^2}{m}$
  \item gesucht: $l$
  \end{itemize}
    \pause
    \begin{equation}\begin{align} \nonumber R &= \frac{\rho\cdot l}{A_{\textrm{Dr}}}\\ \nonumber \Rightarrow l &= \frac{R\cdot A_{\textrm{Dr}}}{\rho} = \frac{1,5\Omega \cdot 0,5mm^2}{0,018 \frac{\Omega mm^2}{m}} \approx 41,7m \end{align}\end{equation}



\end{frame}

\begin{frame}
\frametitle{Temperaturkoeffizient}
\begin{itemize}
  \item Widerstand von Metallen steigt bei zunehemender Temperatur
  \end{itemize}
\end{frame}

\begin{frame}
\only<1>{
\begin{QQuestion}{AB103}{Wie ändert sich der Widerstand eines Metalls mit der Temperatur im Regelfall?}{Der Widerstand steigt mit zunehmender Temperatur, d. h. der Temperaturkoeffizient ist positiv.}
{Der Widerstand sinkt mit zunehmender Temperatur, d. h. der Temperaturkoeffizient ist negativ.}
{Der Widerstand ändert sich nicht mit zunehmender Temperatur, d. h. der Temperaturkoeffizient ist Null.}
{Der Widerstand oszilliert mit zunehmender Temperatur, d. h. der Temperaturkoeffizient ist komplex.}
\end{QQuestion}

}
\only<2>{
\begin{QQuestion}{AB103}{Wie ändert sich der Widerstand eines Metalls mit der Temperatur im Regelfall?}{\textbf{\textcolor{DARCgreen}{Der Widerstand steigt mit zunehmender Temperatur, d. h. der Temperaturkoeffizient ist positiv.}}}
{Der Widerstand sinkt mit zunehmender Temperatur, d. h. der Temperaturkoeffizient ist negativ.}
{Der Widerstand ändert sich nicht mit zunehmender Temperatur, d. h. der Temperaturkoeffizient ist Null.}
{Der Widerstand oszilliert mit zunehmender Temperatur, d. h. der Temperaturkoeffizient ist komplex.}
\end{QQuestion}

}
\end{frame}%ENDCONTENT


\section{Leistung beim Wechselstrom}
\label{section:wechselstrom_leistung}
\begin{frame}%STARTCONTENT
\begin{itemize}
  \item Berechnung mit Effektivwert
  \item $U_{\textrm{eff}} = \frac{\^{U}}{\sqrt{2}}$
  \item $I_{\textrm{eff}} = \frac{\^{I}}{\sqrt{2}}$
  \end{itemize}
\end{frame}

\begin{frame}
\only<1>{
\begin{QQuestion}{AB301}{Ein sinusförmiger Wechselstrom mit einer Amplitude $I_{\symup{max}}$ von 0,5 Ampere fließt durch einen Widerstand von \qty{20}{\ohm}. Wieviel Leistung wird in Wärme umgesetzt?}{\qty{3,5}{\W}}
{\qty{5,0}{\W}}
{\qty{10}{\W}}
{\qty{2,5}{\W}}
\end{QQuestion}

}
\only<2>{
\begin{QQuestion}{AB301}{Ein sinusförmiger Wechselstrom mit einer Amplitude $I_{\symup{max}}$ von 0,5 Ampere fließt durch einen Widerstand von \qty{20}{\ohm}. Wieviel Leistung wird in Wärme umgesetzt?}{\qty{3,5}{\W}}
{\qty{5,0}{\W}}
{\qty{10}{\W}}
{\textbf{\textcolor{DARCgreen}{\qty{2,5}{\W}}}}
\end{QQuestion}

}
\end{frame}

\begin{frame}
\frametitle{Lösungsweg}
\begin{itemize}
  \item gegeben: $I_{\textrm{max}} = 0,5A$
  \item gegeben: $R = 20\Omega$
  \item gesucht: $P$
  \end{itemize}
    \pause
    \begin{equation}\begin{split} \nonumber P &=  I^2 \cdot R = (\frac{I_{\textrm{max}}}{\sqrt{2}})^2 \cdot R\\ &= \frac{(0,5A)^2}{2} \cdot 20\Omega \\ &= \frac{1}{8}A^2 \cdot 20\Omega = 2,5W \end{split}\end{equation}



\end{frame}%ENDCONTENT


\section{Dezibel II}
\label{section:dezibel_2}
\begin{frame}%STARTCONTENT

\frametitle{Leistungsverhältnis}
Faktor 10

$p = 10\cdot \log_{10}(\frac{P}{1mW})\textrm{dBm}$

$p = 10\cdot \log_{10}(\frac{P}{1W})\textrm{dBW}$

$0\textrm{dBm}$ liegt bei $P = 1mW$ vor.

$0\textrm{dBW}$ liegt bei $P = 1W$ vor.

\end{frame}

\begin{frame}
\only<1>{
\begin{QQuestion}{AA110}{Welcher Leistung entsprechen die Pegel \qty{0}{\dBm}, \qty{3}{\dBm} und \qty{20}{\dBm}?}{\qty{0}{\mW}, \qty{30}{\mW}, \qty{200}{\mW}}
{\qty{1}{\mW}, \qty{1,4}{\mW}, \qty{10}{\mW}}
{\qty{1}{\mW}, \qty{2}{\mW}, \qty{100}{\mW}}
{\qty{0}{\mW}, \qty{3}{\mW}, \qty{20}{\mW}}
\end{QQuestion}

}
\only<2>{
\begin{QQuestion}{AA110}{Welcher Leistung entsprechen die Pegel \qty{0}{\dBm}, \qty{3}{\dBm} und \qty{20}{\dBm}?}{\qty{0}{\mW}, \qty{30}{\mW}, \qty{200}{\mW}}
{\qty{1}{\mW}, \qty{1,4}{\mW}, \qty{10}{\mW}}
{\textbf{\textcolor{DARCgreen}{\qty{1}{\mW}, \qty{2}{\mW}, \qty{100}{\mW}}}}
{\qty{0}{\mW}, \qty{3}{\mW}, \qty{20}{\mW}}
\end{QQuestion}

}
\end{frame}

\begin{frame}
\only<1>{
\begin{QQuestion}{AA105}{Einer Leistungsverstärkung von 40 entsprechen~...}{\qty{32}{\decibel}.}
{\qty{36,8}{\decibel}.}
{\qty{16}{\decibel}.}
{\qty{73,8}{\decibel}.}
\end{QQuestion}

}
\only<2>{
\begin{QQuestion}{AA105}{Einer Leistungsverstärkung von 40 entsprechen~...}{\qty{32}{\decibel}.}
{\qty{36,8}{\decibel}.}
{\textbf{\textcolor{DARCgreen}{\qty{16}{\decibel}.}}}
{\qty{73,8}{\decibel}.}
\end{QQuestion}

}
\end{frame}

\begin{frame}
\frametitle{Spannungsverhältnis}
Faktor 20

$u = 20\cdot \log_{10}(\frac{U}{0,775V})\textrm{dBu}$

$0\textrm{dBu}$ liegt bei $U = 0,775V$ vor.

$0\textrm{dBV}$ liegt bei $U = 1V$ vor.

$0\textrm{dBµV}$ liegt bei $U = 1µV$ vor.

\end{frame}

\begin{frame}
\only<1>{
\begin{QQuestion}{AA111}{Einem Spannungsverhältnis von 15 entsprechen~...}{\qty{54}{\decibel}.}
{\qty{15}{\decibel}.}
{\qty{23,5}{\decibel}.}
{\qty{11,7}{\decibel}.}
\end{QQuestion}

}
\only<2>{
\begin{QQuestion}{AA111}{Einem Spannungsverhältnis von 15 entsprechen~...}{\qty{54}{\decibel}.}
{\qty{15}{\decibel}.}
{\textbf{\textcolor{DARCgreen}{\qty{23,5}{\decibel}.}}}
{\qty{11,7}{\decibel}.}
\end{QQuestion}

}
\end{frame}

\begin{frame}
\frametitle{Berechnungen}
\end{frame}

\begin{frame}
\only<1>{
\begin{QQuestion}{AA108}{Der Ausgangspegel eines Senders beträgt \qty{20}{\dBW}. Dies entspricht einer Ausgangsleistung von~...}{$10^{20}$~W.}
{$10^{0,5}$~W.}
{$10^2$~W.}
{$10^1$~W.}
\end{QQuestion}

}
\only<2>{
\begin{QQuestion}{AA108}{Der Ausgangspegel eines Senders beträgt \qty{20}{\dBW}. Dies entspricht einer Ausgangsleistung von~...}{$10^{20}$~W.}
{$10^{0,5}$~W.}
{\textbf{\textcolor{DARCgreen}{$10^2$~W.}}}
{$10^1$~W.}
\end{QQuestion}

}
\end{frame}

\begin{frame}
\frametitle{Lösungsweg}
\begin{itemize}
  \item gegeben: $p = 20\textrm{dBW}$
  \item gesucht: $P$
  \end{itemize}
    \pause
    \begin{equation}\begin{align} \nonumber p &= 10\cdot \log_{10}(\frac{P}{1W})\textrm{dBW}\\ \nonumber \Rightarrow P &= 10^{\frac{p}{10}} \cdot 1W = 10^{\frac{20\textrm{dBW}}{10}} \cdot 1W = 10^2W \end{align}\end{equation}



\end{frame}

\begin{frame}
\only<1>{
\begin{QQuestion}{AA107}{Ein Sender mit \qty{1}{\W} Ausgangsleistung ist an eine Endstufe mit einer Verstärkung von \qty{10}{\decibel} angeschlossen. Wie groß ist der Ausgangspegel der Endstufe?}{\qty{3}{\dBW}}
{\qty{1}{\dBW}}
{\qty{10}{\dBW}}
{\qty{20}{\dBW}}
\end{QQuestion}

}
\only<2>{
\begin{QQuestion}{AA107}{Ein Sender mit \qty{1}{\W} Ausgangsleistung ist an eine Endstufe mit einer Verstärkung von \qty{10}{\decibel} angeschlossen. Wie groß ist der Ausgangspegel der Endstufe?}{\qty{3}{\dBW}}
{\qty{1}{\dBW}}
{\textbf{\textcolor{DARCgreen}{\qty{10}{\dBW}}}}
{\qty{20}{\dBW}}
\end{QQuestion}

}
\end{frame}

\begin{frame}
\only<1>{
\begin{QQuestion}{AA109}{Ein Sender mit \qty{1}{\W} Ausgangsleistung ist an eine Endstufe mit einer Verstärkung von \qty{10}{\decibel} angeschlossen. Wie groß ist der Ausgangspegel der Endstufe?}{\qty{10}{\dBm}}
{\qty{30}{\dBm}}
{\qty{20}{\dBm}}
{\qty{40}{\dBm}}
\end{QQuestion}

}
\only<2>{
\begin{QQuestion}{AA109}{Ein Sender mit \qty{1}{\W} Ausgangsleistung ist an eine Endstufe mit einer Verstärkung von \qty{10}{\decibel} angeschlossen. Wie groß ist der Ausgangspegel der Endstufe?}{\qty{10}{\dBm}}
{\qty{30}{\dBm}}
{\qty{20}{\dBm}}
{\textbf{\textcolor{DARCgreen}{\qty{40}{\dBm}}}}
\end{QQuestion}

}
\end{frame}

\begin{frame}
\frametitle{Lösungsweg}
1W = 1000mW

\qty{10}{\dB} = Faktor 10

1000mW $\cdot$ 10 = 10000mW = 40dBm

\end{frame}

\begin{frame}
\only<1>{
\begin{QQuestion}{AA106}{Ein HF-Leistungsverstärker hat eine Verstärkung von \qty{16}{\decibel} mit maximal \qty{100}{\W} Ausgangsleistung. Welche HF-Ausgangsleistung ist zu erwarten, wenn der Verstärker mit \qty{1}{\W} HF-Eingangsleistung angesteuert wird?}{\qty{40}{\W}}
{\qty{4}{\W}}
{\qty{16}{\W}}
{\qty{20}{\W}}
\end{QQuestion}

}
\only<2>{
\begin{QQuestion}{AA106}{Ein HF-Leistungsverstärker hat eine Verstärkung von \qty{16}{\decibel} mit maximal \qty{100}{\W} Ausgangsleistung. Welche HF-Ausgangsleistung ist zu erwarten, wenn der Verstärker mit \qty{1}{\W} HF-Eingangsleistung angesteuert wird?}{\textbf{\textcolor{DARCgreen}{\qty{40}{\W}}}}
{\qty{4}{\W}}
{\qty{16}{\W}}
{\qty{20}{\W}}
\end{QQuestion}

}
\end{frame}

\begin{frame}
\frametitle{Lösungsweg}
\begin{itemize}
  \item 16dB = 10dB + 6dB = 10 $\cdot$ 4 = 40
  \item 1W $\cdot$ 40 = 40W
  \end{itemize}
\end{frame}

\begin{frame}
\only<1>{
\begin{QQuestion}{AA112}{Der Pegelwert \qty{120}{\decibel}$\upmu$V/m entspricht einer elektrischen Feldstärke von~...}{\qty{1000}{\kV}/m.}
{\qty{0,78}{\V}/m.}
{\qty{41,6}{\V}/m.}
{\qty{1}{\V}/m.}
\end{QQuestion}

}
\only<2>{
\begin{QQuestion}{AA112}{Der Pegelwert \qty{120}{\decibel}$\upmu$V/m entspricht einer elektrischen Feldstärke von~...}{\qty{1000}{\kV}/m.}
{\qty{0,78}{\V}/m.}
{\qty{41,6}{\V}/m.}
{\textbf{\textcolor{DARCgreen}{\qty{1}{\V}/m.}}}
\end{QQuestion}

}
\end{frame}

\begin{frame}
\frametitle{Lösungsweg}
\begin{itemize}
  \item gegeben: $u = 120\textrm{dBµV}/m$
  \item gesucht: $U$
  \end{itemize}
    \pause
    \begin{equation}\begin{align} \nonumber u &= 20\cdot \log_{10}(\frac{U}{1\textrm{µV}})\textrm{\textrm{dBµV}}\\ \nonumber \Rightarrow U &= 10^{\frac{p}{20}} \cdot 1\textrm{µV} = 10^{\frac{120\textrm{dBµV}/m}{20}} \cdot 1\textrm{µV} = 1V/m \end{align}\end{equation}
    \pause
    In der Literatur ist oft zu finden: 120dBµV = 1V



\end{frame}%ENDCONTENT


\section{Ladung und Energie}
\label{section:ladung_energie}
\begin{frame}%STARTCONTENT

\frametitle{Elektrische Ladung}
Strom über Zeit

$Q = I\cdot t$

in Amperesekunde (As)

\end{frame}

\begin{frame}
\only<1>{
\begin{QQuestion}{AA102}{Welche Einheit wird üblicherweise für die elektrische Ladung verwendet?}{Kilowatt (kW)}
{Amperesekunde (As)}
{Joule (J)}
{Ampere (A)}
\end{QQuestion}

}
\only<2>{
\begin{QQuestion}{AA102}{Welche Einheit wird üblicherweise für die elektrische Ladung verwendet?}{Kilowatt (kW)}
{\textbf{\textcolor{DARCgreen}{Amperesekunde (As)}}}
{Joule (J)}
{Ampere (A)}
\end{QQuestion}

}
\end{frame}

\begin{frame}
\frametitle{Elektrische Energie}
Leistung über Zeit

$W = P\cdot t$

in Joule (J) oder Wattstunden (Wh)

\end{frame}

\begin{frame}
\only<1>{
\begin{QQuestion}{AA103}{Welche Einheit wird üblicherweise für die Energie verwendet?}{Volt (V) bzw. Watt pro Ampere (W/A)}
{Joule (J) bzw. Wattstunden (Wh)}
{Watt (W) bzw. Joule pro Stunde (J/h)}
{Watt (W) bzw. Voltampere (VA)}
\end{QQuestion}

}
\only<2>{
\begin{QQuestion}{AA103}{Welche Einheit wird üblicherweise für die Energie verwendet?}{Volt (V) bzw. Watt pro Ampere (W/A)}
{\textbf{\textcolor{DARCgreen}{Joule (J) bzw. Wattstunden (Wh)}}}
{Watt (W) bzw. Joule pro Stunde (J/h)}
{Watt (W) bzw. Voltampere (VA)}
\end{QQuestion}

}
\end{frame}

\begin{frame}
\only<1>{
\begin{QQuestion}{AB502}{Eine Stromversorgung nimmt bei einer Spannung von \qty{230}{\V} einen Strom von \qty{0,63}{\A} auf. Wieviel Energie wird bei einer Betriebsdauer von 7 Stunden umgesetzt?}{\qty{1,01}{\kWh}}
{\qty{0,14}{\kWh}}
{\qty{2,56}{\kWh}}
{\qty{20,7}{\kWh}}
\end{QQuestion}

}
\only<2>{
\begin{QQuestion}{AB502}{Eine Stromversorgung nimmt bei einer Spannung von \qty{230}{\V} einen Strom von \qty{0,63}{\A} auf. Wieviel Energie wird bei einer Betriebsdauer von 7 Stunden umgesetzt?}{\textbf{\textcolor{DARCgreen}{\qty{1,01}{\kWh}}}}
{\qty{0,14}{\kWh}}
{\qty{2,56}{\kWh}}
{\qty{20,7}{\kWh}}
\end{QQuestion}

}
\end{frame}

\begin{frame}
\frametitle{Lösungweg}
\begin{itemize}
  \item gegeben: $U = 230V$
  \item gegeben: $I = 0,63A$
  \item gegeben: $t = 7h$
  \item gesucht: $W$
  \end{itemize}
    \pause
    \begin{equation} \nonumber W = P\cdot t = U\cdot I\cdot t = 230V\cdot 0,63A\cdot 7h = 1,01kWh \end{equation}



\end{frame}

\begin{frame}
\only<1>{
\begin{PQuestion}{AB503}{Wie viel Energie wird vom Widerstand innerhalb einer Stunde in Wärme umgewandelt?}{\qty{0,5}{\W\hour} bzw. \qty{1800}{\J}}
{\qty{2}{\W\hour} bzw. \qty{7200}{\J}}
{\qty{0,1}{\W\hour} bzw. \qty{360}{\J}}
{\qty{1}{\W\hour} bzw. \qty{3600}{\J}}
{\DARCimage{1.0\linewidth}{556include}}\end{PQuestion}

}
\only<2>{
\begin{PQuestion}{AB503}{Wie viel Energie wird vom Widerstand innerhalb einer Stunde in Wärme umgewandelt?}{\qty{0,5}{\W\hour} bzw. \qty{1800}{\J}}
{\qty{2}{\W\hour} bzw. \qty{7200}{\J}}
{\qty{0,1}{\W\hour} bzw. \qty{360}{\J}}
{\textbf{\textcolor{DARCgreen}{\qty{1}{\W\hour} bzw. \qty{3600}{\J}}}}
{\DARCimage{1.0\linewidth}{556include}}\end{PQuestion}

}
\end{frame}

\begin{frame}
\frametitle{Lösungsweg}
\begin{itemize}
  \item gegeben: $U = 10V$
  \item gegeben: $R = 100\Omega$
  \item gegeben: $t = 1h$
  \item gesucht: $W$
  \end{itemize}
    \pause
    \begin{equation} \nonumber W = P\cdot t = \frac{U^2}{R} \cdot t = \frac{(10V)^2}{100\Omega}\cdot 1h = 1Wh \end{equation}



\end{frame}%ENDCONTENT


\title{DARC Amateurfunklehrgang Klasse A}
\author{Bauelemente}
\institute{Deutscher Amateur Radio Club e.\,V.}
\begin{frame}
\maketitle
\end{frame}

\section{Phase}
\label{section:phase}
\begin{frame}%STARTCONTENT

\only<1>{
\begin{PQuestion}{AB302}{Welche Antwort enthält die richtigen Phasenwinkel der dargestellten sinusförmigen Wechselspannung an der mit X$_3$ bezeichneten Stelle?}{$\dfrac{3\pi}{2}; \qty{270}{\degree}$}
{$\dfrac{\pi}{3}; \qty{270}{\degree}$}
{$3\pi; \qty{180}{\degree}$}
{$\dfrac{3\pi}{4}; \qty{135}{\degree}$}
{\DARCimage{1.0\linewidth}{207include}}\end{PQuestion}

}
\only<2>{
\begin{PQuestion}{AB302}{Welche Antwort enthält die richtigen Phasenwinkel der dargestellten sinusförmigen Wechselspannung an der mit X$_3$ bezeichneten Stelle?}{\textbf{\textcolor{DARCgreen}{$\dfrac{3\pi}{2}; \qty{270}{\degree}$}}}
{$\dfrac{\pi}{3}; \qty{270}{\degree}$}
{$3\pi; \qty{180}{\degree}$}
{$\dfrac{3\pi}{4}; \qty{135}{\degree}$}
{\DARCimage{1.0\linewidth}{207include}}\end{PQuestion}

}
\end{frame}

\begin{frame}
\only<1>{
\begin{PQuestion}{AB303}{Der Betrag der Phasendifferenz zwischen den beiden in der Abbildung dargestellten Sinussignalen ist~...}{\qty{45}{\degree}. }
{\qty{0}{\degree}.}
{\qty{90}{\degree}.}
{\qty{180}{\degree}.}
{\DARCimage{1.0\linewidth}{208include}}\end{PQuestion}

}
\only<2>{
\begin{PQuestion}{AB303}{Der Betrag der Phasendifferenz zwischen den beiden in der Abbildung dargestellten Sinussignalen ist~...}{\textbf{\textcolor{DARCgreen}{\qty{45}{\degree}. }}}
{\qty{0}{\degree}.}
{\qty{90}{\degree}.}
{\qty{180}{\degree}.}
{\DARCimage{1.0\linewidth}{208include}}\end{PQuestion}

}
\end{frame}%ENDCONTENT


\section{Kondensator II}
\label{section:kondensator_2}
\begin{frame}%STARTCONTENT

\only<1>{
\begin{QQuestion}{AC101}{Ein verlustloser Kondensator wird an eine Wechselspannungsquelle angeschlossen. Welche Phasenverschiebung zwischen Spannung und Strom stellt sich ein?}{Der Strom eilt der Spannung um \qty{90}{\degree} voraus.}
{Die Spannung eilt dem Strom um \qty{90}{\degree} voraus.}
{Die Spannung eilt dem Strom um \qty{45}{\degree} voraus.}
{Der Strom eilt der Spannung um \qty{45}{\degree} voraus.}
\end{QQuestion}

}
\only<2>{
\begin{QQuestion}{AC101}{Ein verlustloser Kondensator wird an eine Wechselspannungsquelle angeschlossen. Welche Phasenverschiebung zwischen Spannung und Strom stellt sich ein?}{\textbf{\textcolor{DARCgreen}{Der Strom eilt der Spannung um \qty{90}{\degree} voraus.}}}
{Die Spannung eilt dem Strom um \qty{90}{\degree} voraus.}
{Die Spannung eilt dem Strom um \qty{45}{\degree} voraus.}
{Der Strom eilt der Spannung um \qty{45}{\degree} voraus.}
\end{QQuestion}

}
\end{frame}

\begin{frame}
\only<1>{
\begin{QQuestion}{AC102}{Welches Vorzeichen hat der Blindwiderstand eines idealen Kondensators und von welchen physikalischen Größen hängt er ab? Der Blindwiderstand ist~...}{positiv und unabhängig von der Kapazität und der anliegenden Frequenz.}
{negativ und unabhängig von der Kapazität und der anliegenden Frequenz.}
{positiv und abhängig von der Kapazität und der anliegenden Frequenz.}
{negativ und abhängig von der Kapazität und der anliegenden Frequenz.}
\end{QQuestion}

}
\only<2>{
\begin{QQuestion}{AC102}{Welches Vorzeichen hat der Blindwiderstand eines idealen Kondensators und von welchen physikalischen Größen hängt er ab? Der Blindwiderstand ist~...}{positiv und unabhängig von der Kapazität und der anliegenden Frequenz.}
{negativ und unabhängig von der Kapazität und der anliegenden Frequenz.}
{positiv und abhängig von der Kapazität und der anliegenden Frequenz.}
{\textbf{\textcolor{DARCgreen}{negativ und abhängig von der Kapazität und der anliegenden Frequenz.}}}
\end{QQuestion}

}
\end{frame}

\begin{frame}
\only<1>{
\begin{QQuestion}{AC103}{Welcher der folgenden Widerstände hat keine Wärmeverluste?}{Der NTC-Widerstand}
{Der Metalloxidwiderstand}
{Der Wirkwiderstand}
{Der Blindwiderstand}
\end{QQuestion}

}
\only<2>{
\begin{QQuestion}{AC103}{Welcher der folgenden Widerstände hat keine Wärmeverluste?}{Der NTC-Widerstand}
{Der Metalloxidwiderstand}
{Der Wirkwiderstand}
{\textbf{\textcolor{DARCgreen}{Der Blindwiderstand}}}
\end{QQuestion}

}
\end{frame}

\begin{frame}
\only<1>{
\begin{QQuestion}{AC104}{Wie groß ist der Betrag des kapazitiven Blindwiderstands eines Kondensators mit \qty{10}{\pF} bei einer Frequenz von \qty{100}{\MHz}?}{\qty{318}{\ohm}}
{\qty{1,59}{\kohm}}
{\qty{159}{\ohm}}
{\qty{31,8}{\ohm}}
\end{QQuestion}

}
\only<2>{
\begin{QQuestion}{AC104}{Wie groß ist der Betrag des kapazitiven Blindwiderstands eines Kondensators mit \qty{10}{\pF} bei einer Frequenz von \qty{100}{\MHz}?}{\qty{318}{\ohm}}
{\qty{1,59}{\kohm}}
{\textbf{\textcolor{DARCgreen}{\qty{159}{\ohm}}}}
{\qty{31,8}{\ohm}}
\end{QQuestion}

}
\end{frame}

\begin{frame}
\frametitle{Lösungsweg}
\begin{itemize}
  \item gegeben: $C = 10pF$
  \item gegeben: $f = 100MHz$
  \item gesucht: $X_{\textrm{C}}$
  \end{itemize}
    \pause
    \begin{equation}\begin{split}\nonumber X_{\textrm{C}} &= \frac{1}{\omega \cdot C} = \frac{1}{2\pi \cdot f \cdot C}\\ &= \frac{1}{2\pi \cdot 100MHz \cdot 10pF}\\ &\approx 159\Omega \end{split}\end{equation}



\end{frame}

\begin{frame}
\only<1>{
\begin{QQuestion}{AC106}{Wie groß ist der Betrag des kapazitiven Blindwiderstands eines Kondensators mit \qty{100}{\pF} bei einer Frequenz von \qty{100}{\MHz}?}{ca. \qty{15,9}{\ohm}}
{ca. \qty{159}{\ohm}}
{ca. \qty{31,8}{\ohm}}
{ca. \qty{3,2}{\ohm}}
\end{QQuestion}

}
\only<2>{
\begin{QQuestion}{AC106}{Wie groß ist der Betrag des kapazitiven Blindwiderstands eines Kondensators mit \qty{100}{\pF} bei einer Frequenz von \qty{100}{\MHz}?}{\textbf{\textcolor{DARCgreen}{ca. \qty{15,9}{\ohm}}}}
{ca. \qty{159}{\ohm}}
{ca. \qty{31,8}{\ohm}}
{ca. \qty{3,2}{\ohm}}
\end{QQuestion}

}
\end{frame}

\begin{frame}
\frametitle{Lösungsweg}
\begin{itemize}
  \item gegeben: $C = 100pF$
  \item gegeben: $f = 100MHz$
  \item gesucht: $X_{\textrm{C}}$
  \end{itemize}
    \pause
    \begin{equation}\begin{split}\nonumber X_{\textrm{C}} &= \frac{1}{\omega \cdot C} = \frac{1}{2\pi \cdot f \cdot C}\\ &= \frac{1}{2\pi \cdot 100MHz \cdot 100pF}\\ &\approx 15,9\Omega \end{split}\end{equation}



\end{frame}

\begin{frame}
\only<1>{
\begin{QQuestion}{AC105}{Wie groß ist der Betrag des kapazitiven Blindwiderstands eines Kondensators mit \qty{50}{\pF} bei einer Frequenz von \qty{145}{\MHz} ?}{ca. \qty{69}{\ohm}}
{ca. \qty{0,045}{\ohm}}
{ca. \qty{18,2}{\kohm}}
{ca. \qty{22}{\ohm}}
\end{QQuestion}

}
\only<2>{
\begin{QQuestion}{AC105}{Wie groß ist der Betrag des kapazitiven Blindwiderstands eines Kondensators mit \qty{50}{\pF} bei einer Frequenz von \qty{145}{\MHz} ?}{ca. \qty{69}{\ohm}}
{ca. \qty{0,045}{\ohm}}
{ca. \qty{18,2}{\kohm}}
{\textbf{\textcolor{DARCgreen}{ca. \qty{22}{\ohm}}}}
\end{QQuestion}

}
\end{frame}

\begin{frame}
\frametitle{Lösungsweg}
\begin{itemize}
  \item gegeben: $C = 50pF$
  \item gegeben: $f = 145MHz$
  \item gesucht: $X_{\textrm{C}}$
  \end{itemize}
    \pause
    \begin{equation}\begin{split}\nonumber X_{\textrm{C}} &= \frac{1}{\omega \cdot C} = \frac{1}{2\pi \cdot f \cdot C}\\ &= \frac{1}{2\pi \cdot 145MHz \cdot 50pF}\\ &\approx 22\Omega \end{split}\end{equation}



\end{frame}

\begin{frame}
\only<1>{
\begin{QQuestion}{AC107}{Wie groß ist der Betrag des kapazitiven Blindwiderstands eines Kondensators mit \qty{100}{\pF} bei einer Frequenz von \qty{435}{\MHz} ?}{ca. \qty{3,7}{\ohm}}
{ca. \qty{0,27}{\ohm}}
{ca. \qty{27,3}{\kohm}}
{ca. \qty{11,5}{\ohm}}
\end{QQuestion}

}
\only<2>{
\begin{QQuestion}{AC107}{Wie groß ist der Betrag des kapazitiven Blindwiderstands eines Kondensators mit \qty{100}{\pF} bei einer Frequenz von \qty{435}{\MHz} ?}{\textbf{\textcolor{DARCgreen}{ca. \qty{3,7}{\ohm}}}}
{ca. \qty{0,27}{\ohm}}
{ca. \qty{27,3}{\kohm}}
{ca. \qty{11,5}{\ohm}}
\end{QQuestion}

}
\end{frame}

\begin{frame}
\frametitle{Lösungsweg}
\begin{itemize}
  \item gegeben: $C = 100pF$
  \item gegeben: $f = 435MHz$
  \item gesucht: $X_{\textrm{C}}$
  \end{itemize}
    \pause
    \begin{equation}\begin{split}\nonumber X_{\textrm{C}} &= \frac{1}{\omega \cdot C} = \frac{1}{2\pi \cdot f \cdot C}\\ &= \frac{1}{2\pi \cdot 435MHz \cdot 100pF}\\ &\approx 3,7\Omega \end{split}\end{equation}



\end{frame}

\begin{frame}
\only<1>{
\begin{QQuestion}{AC108}{An einem unbekannten Kondensator liegt eine Wechselspannung mit \qty{16}{\V} und \qty{50}{\Hz}. Es wird ein Strom von \qty{32}{\mA} gemessen. Welche Kapazität hat der Kondensator?}{ca. \qty{0,637}{\micro\F}}
{ca. \qty{6,37}{\micro\F}}
{ca. \qty{0,45}{\micro\F}}
{ca. \qty{4,5}{\micro\F}}
\end{QQuestion}

}
\only<2>{
\begin{QQuestion}{AC108}{An einem unbekannten Kondensator liegt eine Wechselspannung mit \qty{16}{\V} und \qty{50}{\Hz}. Es wird ein Strom von \qty{32}{\mA} gemessen. Welche Kapazität hat der Kondensator?}{ca. \qty{0,637}{\micro\F}}
{\textbf{\textcolor{DARCgreen}{ca. \qty{6,37}{\micro\F}}}}
{ca. \qty{0,45}{\micro\F}}
{ca. \qty{4,5}{\micro\F}}
\end{QQuestion}

}
\end{frame}

\begin{frame}
\frametitle{Lösungsweg}
\begin{columns}
    \begin{column}{0.48\textwidth}
    \begin{itemize}
  \item gegeben: $U = 16V$
  \item gegeben: $I = 32mA$
  \end{itemize}

    \end{column}
   \begin{column}{0.48\textwidth}
       \begin{itemize}
  \item gegeben: $f = 50Hz$
  \item gesucht: $C$
  \end{itemize}

   \end{column}
\end{columns}
    \pause
    $X_{\textrm{C}} = \frac{U}{I} = \frac{16V}{32mA} = 500\Omega$
    \pause
    \begin{equation}\begin{align}\nonumber X_{\textrm{C}} &= \frac{1}{\omega \cdot C} \\ \nonumber \Rightarrow C &= \frac{1}{\omega \cdot X_{\textrm{C}}} = \frac{1}{2\pi \cdot f \cdot X_{\textrm{C}}}\\ \nonumber &= \frac{1}{2\pi \cdot 50Hz \cdot 500\Omega}\\ \nonumber &\approx 6,37\mu F\end{align}\end{equation}



\end{frame}

\begin{frame}
\only<1>{
\begin{QQuestion}{AC109}{Kommt es in einem von Wechselstrom durchflossenen realen Kondensator zu Verlusten?}{Ja, infolge von Verlusten in Dielektrikum und Zuleitung}
{Nein, beim Kondensator handelt es sich  um eine reine Blindleistung.}
{Ja, infolge des Blindwiderstands}
{Nein, bei Wechselstrom treten keine Verluste auf.}
\end{QQuestion}

}
\only<2>{
\begin{QQuestion}{AC109}{Kommt es in einem von Wechselstrom durchflossenen realen Kondensator zu Verlusten?}{\textbf{\textcolor{DARCgreen}{Ja, infolge von Verlusten in Dielektrikum und Zuleitung}}}
{Nein, beim Kondensator handelt es sich  um eine reine Blindleistung.}
{Ja, infolge des Blindwiderstands}
{Nein, bei Wechselstrom treten keine Verluste auf.}
\end{QQuestion}

}
\end{frame}

\begin{frame}
\only<1>{
\begin{QQuestion}{AC110}{Neben dem kapazitiven Blindwiderstand treten im von Wechselstrom durchflossenen Kondensator auch Verluste auf, die rechnerisch in einem parallelgeschalteten Verlustwiderstand zusammengefasst werden können. Die Kondensatorverluste werden oft durch~...}{den relativen Blindwiderstand in Ohm pro Farad angegeben, mit dem die Kondensatorgüte berechnet werden kann.}
{den relativen Verlustwiderstand in Ohm pro Farad angegeben, mit dem die Kondensatorgüte berechnet werden kann.}
{den Verlustfaktor tan $\delta$ angegeben, der dem Kehrwert des Gütefaktors entspricht.}
{den Verlustfaktor cos $\phi$ angegeben, der dem Kehrwert des Gütefaktors entspricht.}
\end{QQuestion}

}
\only<2>{
\begin{QQuestion}{AC110}{Neben dem kapazitiven Blindwiderstand treten im von Wechselstrom durchflossenen Kondensator auch Verluste auf, die rechnerisch in einem parallelgeschalteten Verlustwiderstand zusammengefasst werden können. Die Kondensatorverluste werden oft durch~...}{den relativen Blindwiderstand in Ohm pro Farad angegeben, mit dem die Kondensatorgüte berechnet werden kann.}
{den relativen Verlustwiderstand in Ohm pro Farad angegeben, mit dem die Kondensatorgüte berechnet werden kann.}
{\textbf{\textcolor{DARCgreen}{den Verlustfaktor tan $\delta$ angegeben, der dem Kehrwert des Gütefaktors entspricht.}}}
{den Verlustfaktor cos $\phi$ angegeben, der dem Kehrwert des Gütefaktors entspricht.}
\end{QQuestion}

}
\end{frame}

\begin{frame}
\only<1>{
\begin{QQuestion}{AC111}{An einem Kondensator mit einer Kapazität von \qty{1}{\micro\F} wird ein NF-Signal mit \qty{10}{\kHz} und \qty{12}{\V}$_{\textrm{eff}}$ angelegt. Wie groß ist die aufgenommene Wirkleistung im eingeschwungenen Zustand?}{\qty{0,9}{\W}}
{Näherungsweise \qty{0}{\W}}
{\qty{0,75}{\W}}
{\qty{9}{\W}}
\end{QQuestion}

}
\only<2>{
\begin{QQuestion}{AC111}{An einem Kondensator mit einer Kapazität von \qty{1}{\micro\F} wird ein NF-Signal mit \qty{10}{\kHz} und \qty{12}{\V}$_{\textrm{eff}}$ angelegt. Wie groß ist die aufgenommene Wirkleistung im eingeschwungenen Zustand?}{\qty{0,9}{\W}}
{\textbf{\textcolor{DARCgreen}{Näherungsweise \qty{0}{\W}}}}
{\qty{0,75}{\W}}
{\qty{9}{\W}}
\end{QQuestion}

}

\end{frame}%ENDCONTENT


\section{Spule II}
\label{section:spule_2}
\begin{frame}%STARTCONTENT

\only<1>{
\begin{QQuestion}{AA101}{Welche Einheit wird üblicherweise für die Impedanz verwendet?}{Farad}
{Ohm}
{Siemens}
{Henry}
\end{QQuestion}

}
\only<2>{
\begin{QQuestion}{AA101}{Welche Einheit wird üblicherweise für die Impedanz verwendet?}{Farad}
{\textbf{\textcolor{DARCgreen}{Ohm}}}
{Siemens}
{Henry}
\end{QQuestion}

}
\end{frame}

\begin{frame}
\only<1>{
\begin{QQuestion}{AC201}{In einer idealen Induktivität, die an einer Wechselspannungsquelle angeschlossen ist, eilt der Strom der angelegten Spannung~...}{um \qty{45}{\degree} voraus.}
{um \qty{90}{\degree} nach.}
{um \qty{45}{\degree} nach.}
{um \qty{90}{\degree} voraus.}
\end{QQuestion}

}
\only<2>{
\begin{QQuestion}{AC201}{In einer idealen Induktivität, die an einer Wechselspannungsquelle angeschlossen ist, eilt der Strom der angelegten Spannung~...}{um \qty{45}{\degree} voraus.}
{\textbf{\textcolor{DARCgreen}{um \qty{90}{\degree} nach.}}}
{um \qty{45}{\degree} nach.}
{um \qty{90}{\degree} voraus.}
\end{QQuestion}

}
\end{frame}

\begin{frame}
\only<1>{
\begin{QQuestion}{AC202}{Welches Vorzeichen hat der Blindwiderstand einer idealen Spule und von welchen physikalischen Größen hängt er ab? Der Blindwiderstand ist~...}{negativ und abhängig von der Induktivität und der anliegenden Frequenz.}
{positiv und unabhängig von der Induktivität und der anliegenden Frequenz.}
{positiv und abhängig von der Induktivität und der anliegenden Frequenz.}
{negativ und unabhängig von der Induktivität und der anliegenden Frequenz.}
\end{QQuestion}

}
\only<2>{
\begin{QQuestion}{AC202}{Welches Vorzeichen hat der Blindwiderstand einer idealen Spule und von welchen physikalischen Größen hängt er ab? Der Blindwiderstand ist~...}{negativ und abhängig von der Induktivität und der anliegenden Frequenz.}
{positiv und unabhängig von der Induktivität und der anliegenden Frequenz.}
{\textbf{\textcolor{DARCgreen}{positiv und abhängig von der Induktivität und der anliegenden Frequenz.}}}
{negativ und unabhängig von der Induktivität und der anliegenden Frequenz.}
\end{QQuestion}

}
\end{frame}

\begin{frame}
\only<1>{
\begin{QQuestion}{AC203}{Beim Anlegen einer Gleichspannung $U$ = \qty{1}{\V} an eine Spule messen Sie einen Strom. Wird der Strom beim Anlegen von einer Wechselspannung mit $U_{\symup{eff}}$ = \qty{1}{\V} größer oder kleiner?}{Beim Betrieb mit Gleichspannung wirkt nur der Gleichstromwiderstand der Spule. Beim Betrieb mit Wechselspannung wird der induktive Widerstand $X_{\symup{L}}$ wirksam und erhöht den Gesamtwiderstand. Der Strom wird kleiner.}
{Beim Betrieb mit Gleichspannung wirkt nur der Gleichstromwiderstand der Spule. Beim Betrieb mit Wechselspannung wirkt nur der kleinere induktive Widerstand $X_{\symup{L}}$. Der Strom wird größer.}
{Beim Betrieb mit Gleich- oder Wechselspannung wirkt nur der ohmsche Widerstand $X_{\symup{L}}$ der Spule. Der Strom bleibt gleich.}
{Beim Betrieb mit Wechselspannung wirkt nur der Wechselstromwiderstand der Spule. Beim Betrieb mit Gleichspannung wird nur der ohmsche Widerstand $X_{\symup{L}}$ wirksam. Der Strom wird größer.}
\end{QQuestion}

}
\only<2>{
\begin{QQuestion}{AC203}{Beim Anlegen einer Gleichspannung $U$ = \qty{1}{\V} an eine Spule messen Sie einen Strom. Wird der Strom beim Anlegen von einer Wechselspannung mit $U_{\symup{eff}}$ = \qty{1}{\V} größer oder kleiner?}{\textbf{\textcolor{DARCgreen}{Beim Betrieb mit Gleichspannung wirkt nur der Gleichstromwiderstand der Spule. Beim Betrieb mit Wechselspannung wird der induktive Widerstand $X_{\symup{L}}$ wirksam und erhöht den Gesamtwiderstand. Der Strom wird kleiner.}}}
{Beim Betrieb mit Gleichspannung wirkt nur der Gleichstromwiderstand der Spule. Beim Betrieb mit Wechselspannung wirkt nur der kleinere induktive Widerstand $X_{\symup{L}}$. Der Strom wird größer.}
{Beim Betrieb mit Gleich- oder Wechselspannung wirkt nur der ohmsche Widerstand $X_{\symup{L}}$ der Spule. Der Strom bleibt gleich.}
{Beim Betrieb mit Wechselspannung wirkt nur der Wechselstromwiderstand der Spule. Beim Betrieb mit Gleichspannung wird nur der ohmsche Widerstand $X_{\symup{L}}$ wirksam. Der Strom wird größer.}
\end{QQuestion}

}
\end{frame}

\begin{frame}
\only<1>{
\begin{QQuestion}{AC204}{Wie groß ist der Betrag des induktiven Blindwiderstands einer Spule mit \qty{3}{\micro\H} Induktivität bei einer Frequenz von \qty{100}{\MHz}?}{ca. \qty{1885}{\ohm}}
{ca. \qty{942,0}{\ohm}}
{ca. \qty{1885}{\kohm}}
{ca. \qty{1,942}{\ohm}}
\end{QQuestion}

}
\only<2>{
\begin{QQuestion}{AC204}{Wie groß ist der Betrag des induktiven Blindwiderstands einer Spule mit \qty{3}{\micro\H} Induktivität bei einer Frequenz von \qty{100}{\MHz}?}{\textbf{\textcolor{DARCgreen}{ca. \qty{1885}{\ohm}}}}
{ca. \qty{942,0}{\ohm}}
{ca. \qty{1885}{\kohm}}
{ca. \qty{1,942}{\ohm}}
\end{QQuestion}

}
\end{frame}

\begin{frame}
\frametitle{Lösungsweg}
\begin{itemize}
  \item gegeben: $L = 3\mu H$
  \item gegeben: $f = 100MHz$
  \item gesucht: $X_{\textrm{L}}$
  \end{itemize}
    \pause
    \begin{equation}\begin{split}\nonumber X_{\textrm{L}} &= \omega \cdot L = 2\pi \cdot f \cdot L\\ &= 2\pi \cdot 100MHz \cdot 3\mu H\\ &\approx 1885\Omega \end{split}\end{equation}



\end{frame}

\begin{frame}
\only<1>{
\begin{QQuestion}{AC205}{Wie groß ist die Induktivität einer Spule mit 14 Windungen, die auf einen Kern mit einer Induktivitätskonstante ($A_{\symup{L}}$-Wert) von \qty{1,5}{\nano\H} gewickelt ist?}{\qty{0,294}{\micro\H}}
{\qty{2,94}{\micro\H}}
{\qty{29,4}{\nano\H}}
{\qty{2,94}{\nano\H}}
\end{QQuestion}

}
\only<2>{
\begin{QQuestion}{AC205}{Wie groß ist die Induktivität einer Spule mit 14 Windungen, die auf einen Kern mit einer Induktivitätskonstante ($A_{\symup{L}}$-Wert) von \qty{1,5}{\nano\H} gewickelt ist?}{\textbf{\textcolor{DARCgreen}{\qty{0,294}{\micro\H}}}}
{\qty{2,94}{\micro\H}}
{\qty{29,4}{\nano\H}}
{\qty{2,94}{\nano\H}}
\end{QQuestion}

}
\end{frame}

\begin{frame}
\frametitle{Lösungsweg}
\begin{itemize}
  \item gegeben: $N = 14$
  \item gegeben: $A_{\textrm{L}} = 1,5nH$
  \item gesucht: $L$
  \end{itemize}
    \pause
    \begin{equation}\begin{split}\nonumber L &= N^2 \cdot A_{\textrm{L}}\\ &= 14^2 \cdot 1,5nH\\ &= 0,294\mu H \end{split}\end{equation}



\end{frame}

\begin{frame}
\only<1>{
\begin{QQuestion}{AC206}{Wie groß ist die Induktivität einer Spule mit 300 Windungen, die auf einen Kern mit einer Induktivitätskonstante ($A_{\symup{L}}$-Wert) von \qty{1250}{\nano\H} gewickelt ist?}{\qty{112,5}{\mH}}
{\qty{112,5}{\micro\H}}
{\qty{11,25}{\mH}}
{\qty{1,125}{\mH}}
\end{QQuestion}

}
\only<2>{
\begin{QQuestion}{AC206}{Wie groß ist die Induktivität einer Spule mit 300 Windungen, die auf einen Kern mit einer Induktivitätskonstante ($A_{\symup{L}}$-Wert) von \qty{1250}{\nano\H} gewickelt ist?}{\textbf{\textcolor{DARCgreen}{\qty{112,5}{\mH}}}}
{\qty{112,5}{\micro\H}}
{\qty{11,25}{\mH}}
{\qty{1,125}{\mH}}
\end{QQuestion}

}
\end{frame}

\begin{frame}
\frametitle{Lösungsweg}
\begin{itemize}
  \item gegeben: $N = 300$
  \item gegeben: $A_{\textrm{L}} = 1250nH$
  \item gesucht: $L$
  \end{itemize}
    \pause
    \begin{equation}\begin{split}\nonumber L &= N^2 \cdot A_{\textrm{L}}\\ &= 300^2 \cdot 1250nH\\ &= 112,5mH \end{split}\end{equation}



\end{frame}

\begin{frame}
\only<1>{
\begin{QQuestion}{AC207}{Mit einem Ringkern, dessen Induktivitätskonstante ($A_{\symup{L}}$-Wert) mit \qty{250}{\nano\H} angegeben ist, soll eine Spule mit einer Induktivität von \qty{2}{\mH} hergestellt werden. Wie groß ist die erforderliche Windungszahl etwa?}{89}
{3}
{2828}
{53}
\end{QQuestion}

}
\only<2>{
\begin{QQuestion}{AC207}{Mit einem Ringkern, dessen Induktivitätskonstante ($A_{\symup{L}}$-Wert) mit \qty{250}{\nano\H} angegeben ist, soll eine Spule mit einer Induktivität von \qty{2}{\mH} hergestellt werden. Wie groß ist die erforderliche Windungszahl etwa?}{\textbf{\textcolor{DARCgreen}{89}}}
{3}
{2828}
{53}
\end{QQuestion}

}
\end{frame}

\begin{frame}
\frametitle{Lösungsweg}
\begin{itemize}
  \item gegeben: $L = 2mH$
  \item gegeben: $A_{\textrm{L}} = 250nH$
  \item gesucht: $N$
  \end{itemize}
    \pause
    \begin{equation}\begin{align}\nonumber L &= N^2 \cdot A_{\textrm{L}}\\ \nonumber N &= \sqrt{\frac{L}{A_{\textrm{L}}}} = \sqrt{\frac{2mH}{250nH}} \\ \nonumber &= 89\ \textrm{Windungen} \end{align}\end{equation}



\end{frame}

\begin{frame}
\only<1>{
\begin{QQuestion}{AC208}{Ein Spulenkern hat eine Induktivitätskonstante ($A_{\symup{L}}$-Wert) von \qty{30}{\nano\H}. Wie groß ist die erforderliche Windungszahl zur Herstellung einer Induktivität von \qty{12}{\micro\H} in etwa?}{400}
{20}
{360}
{6}
\end{QQuestion}

}
\only<2>{
\begin{QQuestion}{AC208}{Ein Spulenkern hat eine Induktivitätskonstante ($A_{\symup{L}}$-Wert) von \qty{30}{\nano\H}. Wie groß ist die erforderliche Windungszahl zur Herstellung einer Induktivität von \qty{12}{\micro\H} in etwa?}{400}
{\textbf{\textcolor{DARCgreen}{20}}}
{360}
{6}
\end{QQuestion}

}
\end{frame}

\begin{frame}
\frametitle{Lösungsweg}
\begin{itemize}
  \item gegeben: $L = 12\mu H$
  \item gegeben: $A_{\textrm{L}} = 30nH$
  \item gesucht: $N$
  \end{itemize}
    \pause
    \begin{equation}\begin{align}\nonumber L &= N^2 \cdot A_{\textrm{L}}\\ \nonumber N &= \sqrt{\frac{L}{A_{\textrm{L}}}} = \sqrt{\frac{12\mu H}{30nH}} \\ \nonumber &= 20\ \textrm{Windungen} \end{align}\end{equation}



\end{frame}

\begin{frame}
\only<1>{
\begin{QQuestion}{AC209}{Neben dem induktiven Blindwiderstand treten in der mit Wechselstrom durchflossenen Spule auch Verluste auf, die rechnerisch in einem seriellen Verlustwiderstand zusammengefasst werden können. Als Maß für die Verluste in einer Spule wird auch~...}{der Verlustfaktor tan $\delta$ angegeben, der dem Kehrwert des Gütefaktors entspricht.}
{der relative Verlustwiderstand in Ohm pro Henry angegeben, mit dem die Spulengüte berechnet werden kann.}
{der relative Blindwiderstand in Ohm pro Henry angegeben, mit dem die Spulengüte berechnet werden kann.}
{der Verlustfaktor cos $\varphi$ angegeben, der dem Kehrwert des Gütefaktors entspricht.}
\end{QQuestion}

}
\only<2>{
\begin{QQuestion}{AC209}{Neben dem induktiven Blindwiderstand treten in der mit Wechselstrom durchflossenen Spule auch Verluste auf, die rechnerisch in einem seriellen Verlustwiderstand zusammengefasst werden können. Als Maß für die Verluste in einer Spule wird auch~...}{\textbf{\textcolor{DARCgreen}{der Verlustfaktor tan $\delta$ angegeben, der dem Kehrwert des Gütefaktors entspricht.}}}
{der relative Verlustwiderstand in Ohm pro Henry angegeben, mit dem die Spulengüte berechnet werden kann.}
{der relative Blindwiderstand in Ohm pro Henry angegeben, mit dem die Spulengüte berechnet werden kann.}
{der Verlustfaktor cos $\varphi$ angegeben, der dem Kehrwert des Gütefaktors entspricht.}
\end{QQuestion}

}
\end{frame}

\begin{frame}
\only<1>{
\begin{QQuestion}{AC210}{Um die Abstrahlungen der Spule eines abgestimmten Schwingkreises zu verringern, sollte die Spule~...}{einen abgestimmten Kunststoffkern aufweisen.}
{einen hohlen Kupferkern aufweisen.}
{in einem isolierenden Kunststoffgehäuse untergebracht werden.}
{in einem leitenden Metallgehäuse untergebracht werden.}
\end{QQuestion}

}
\only<2>{
\begin{QQuestion}{AC210}{Um die Abstrahlungen der Spule eines abgestimmten Schwingkreises zu verringern, sollte die Spule~...}{einen abgestimmten Kunststoffkern aufweisen.}
{einen hohlen Kupferkern aufweisen.}
{in einem isolierenden Kunststoffgehäuse untergebracht werden.}
{\textbf{\textcolor{DARCgreen}{in einem leitenden Metallgehäuse untergebracht werden.}}}
\end{QQuestion}

}
\end{frame}

\begin{frame}
\only<1>{
\begin{PQuestion}{AC211}{Das folgende Bild zeigt einen Kern, um den ein Kabel für den Bau einer Drossel gewickelt ist. Der Kern sollte üblicherweise aus~...}{Stahl bestehen.}
{Kunststoff bestehen.}
{Ferrit bestehen.}
{diamagnetischem Material bestehen.}
{\DARCimage{0.5\linewidth}{40include}}\end{PQuestion}

}
\only<2>{
\begin{PQuestion}{AC211}{Das folgende Bild zeigt einen Kern, um den ein Kabel für den Bau einer Drossel gewickelt ist. Der Kern sollte üblicherweise aus~...}{Stahl bestehen.}
{Kunststoff bestehen.}
{\textbf{\textcolor{DARCgreen}{Ferrit bestehen.}}}
{diamagnetischem Material bestehen.}
{\DARCimage{0.5\linewidth}{40include}}\end{PQuestion}

}
\end{frame}%ENDCONTENT


\section{Übertrager II}
\label{section:uebertrager_2}
\begin{frame}%STARTCONTENT

\only<1>{
\begin{QQuestion}{AC301}{Durch Gegeninduktion wird in einer Spule eine Spannung erzeugt, wenn~...}{sich die Spule in einem konstanten Magnetfeld befindet.}
{ein veränderlicher Strom durch die Spule fließt und sich dabei ein dielektrischer Gegenstand innerhalb der Spule befindet. }
{ein konstanter Gleichstrom durch eine magnetisch gekoppelte benachbarte Spule fließt.}
{ein veränderlicher Strom durch eine magnetisch gekoppelte benachbarte Spule fließt.}
\end{QQuestion}

}
\only<2>{
\begin{QQuestion}{AC301}{Durch Gegeninduktion wird in einer Spule eine Spannung erzeugt, wenn~...}{sich die Spule in einem konstanten Magnetfeld befindet.}
{ein veränderlicher Strom durch die Spule fließt und sich dabei ein dielektrischer Gegenstand innerhalb der Spule befindet. }
{ein konstanter Gleichstrom durch eine magnetisch gekoppelte benachbarte Spule fließt.}
{\textbf{\textcolor{DARCgreen}{ein veränderlicher Strom durch eine magnetisch gekoppelte benachbarte Spule fließt.}}}
\end{QQuestion}

}
\end{frame}

\begin{frame}
\only<1>{
\begin{QQuestion}{AC302}{Ein Transformator setzt die Spannung von \qty{230}{\V} auf \qty{6}{\V} herunter und liefert dabei einen Strom von \qty{1,15}{\A}. Wie groß ist der dadurch in der Primärwicklung zu erwartende Strom bei Vernachlässigung der Verluste?}{\qty{22,7}{\mA}}
{\qty{30}{\mA}}
{\qty{0,83}{\mA}}
{\qty{33,3}{\mA}}
\end{QQuestion}

}
\only<2>{
\begin{QQuestion}{AC302}{Ein Transformator setzt die Spannung von \qty{230}{\V} auf \qty{6}{\V} herunter und liefert dabei einen Strom von \qty{1,15}{\A}. Wie groß ist der dadurch in der Primärwicklung zu erwartende Strom bei Vernachlässigung der Verluste?}{\qty{22,7}{\mA}}
{\textbf{\textcolor{DARCgreen}{\qty{30}{\mA}}}}
{\qty{0,83}{\mA}}
{\qty{33,3}{\mA}}
\end{QQuestion}

}
\end{frame}

\begin{frame}
\frametitle{Lösungsweg}
\begin{itemize}
  \item gegeben: $U_P = 230V$
  \item gegeben: $U_S = 6V$
  \item gegeben: $I_S = 1,15A$
  \item gesucht: $I_P$
  \end{itemize}
    \pause
    \begin{equation}\begin{align} \nonumber \frac{U_P}{U_S} &= \frac{I_S}{I_P} \\ \nonumber \Rightarrow I_P &= \frac{I_S \cdot U_S}{U_P} = \frac{1,15A \cdot 6V}{230V} \\ \nonumber &= 30mA \end{align}\end{equation}



\end{frame}

\begin{frame}
\only<1>{
\begin{PQuestion}{AC303}{In dieser Schaltung beträgt $R$=\qty{16}{\kohm}. Die Impedanz zwischen den Anschlüssen a und b beträgt im Idealfall~...}{\qty{1}{\kohm}.}
{\qty{64}{\kohm}.}
{\qty{16}{\kohm}.}
{\qty{4}{\kohm}.}
{\DARCimage{0.75\linewidth}{303include}}\end{PQuestion}

}
\only<2>{
\begin{PQuestion}{AC303}{In dieser Schaltung beträgt $R$=\qty{16}{\kohm}. Die Impedanz zwischen den Anschlüssen a und b beträgt im Idealfall~...}{\textbf{\textcolor{DARCgreen}{\qty{1}{\kohm}.}}}
{\qty{64}{\kohm}.}
{\qty{16}{\kohm}.}
{\qty{4}{\kohm}.}
{\DARCimage{0.75\linewidth}{303include}}\end{PQuestion}

}
\end{frame}

\begin{frame}
\frametitle{Lösungsweg}
\begin{itemize}
  \item gegeben: $Z_S = 16k\Omega$
  \item gegeben: $ü = \frac{1}{4}$
  \item gesucht: $Z_P$
  \end{itemize}
    \pause
    \begin{equation}\begin{align} \nonumber ü &= \sqrt{\frac{Z_P}{Z_S}} \\ \nonumber \Rightarrow Z_P &= ü^2 \cdot Z_S = \frac{1^2}{4^2} \cdot 16k\Omega \\ \nonumber &= \frac{16k\Omega}{16} = 1k\Omega \end{align}\end{equation}



\end{frame}

\begin{frame}
\only<1>{
\begin{PQuestion}{AC304}{In dieser Schaltung beträgt $R$=\qty{6,4}{\kohm}. Die Impedanz zwischen den Anschlüssen a und b beträgt im Idealfall~...}{\qty{6,4}{\kohm}.}
{\qty{26}{\kohm}.}
{\qty{0,4}{\kohm}.}
{\qty{1,6}{\kohm}.}
{\DARCimage{0.75\linewidth}{303include}}\end{PQuestion}

}
\only<2>{
\begin{PQuestion}{AC304}{In dieser Schaltung beträgt $R$=\qty{6,4}{\kohm}. Die Impedanz zwischen den Anschlüssen a und b beträgt im Idealfall~...}{\qty{6,4}{\kohm}.}
{\qty{26}{\kohm}.}
{\textbf{\textcolor{DARCgreen}{\qty{0,4}{\kohm}.}}}
{\qty{1,6}{\kohm}.}
{\DARCimage{0.75\linewidth}{303include}}\end{PQuestion}

}
\end{frame}

\begin{frame}
\frametitle{Lösungsweg}
\begin{itemize}
  \item gegeben: $Z_S = 6,4k\Omega$
  \item gegeben: $ü = \frac{1}{4}$
  \item gesucht: $Z_P$
  \end{itemize}
    \pause
    \begin{equation}\begin{align} \nonumber ü &= \sqrt{\frac{Z_P}{Z_S}} \\ \nonumber \Rightarrow Z_P &= ü^2 \cdot Z_S = \frac{1^2}{4^2} \cdot 6,4k\Omega \\ \nonumber &= \frac{6,4k\Omega}{16} = 0,4k\Omega \end{align}\end{equation}



\end{frame}

\begin{frame}
\only<1>{
\begin{QQuestion}{AC305}{Für die Anpassung einer Antenne mit einem Fußpunktwiderstand von \qty{450}{\ohm} an eine \qty{50}{\ohm}-Übertragungsleitung sollte ein Übertrager mit einem Windungsverhältnis von~...}{4:1 verwendet werden.}
{3:1 verwendet werden.}
{9:1 verwendet werden.}
{16:1 verwendet werden.}
\end{QQuestion}

}
\only<2>{
\begin{QQuestion}{AC305}{Für die Anpassung einer Antenne mit einem Fußpunktwiderstand von \qty{450}{\ohm} an eine \qty{50}{\ohm}-Übertragungsleitung sollte ein Übertrager mit einem Windungsverhältnis von~...}{4:1 verwendet werden.}
{\textbf{\textcolor{DARCgreen}{3:1 verwendet werden.}}}
{9:1 verwendet werden.}
{16:1 verwendet werden.}
\end{QQuestion}

}
\end{frame}

\begin{frame}
\frametitle{Lösungsweg}
\begin{itemize}
  \item gegeben: $Z_P = 450\Omega$
  \item gegeben: $Z_S = 50\Omega$
  \item gesucht: $ü$
  \end{itemize}
    \pause
    \begin{equation}\begin{split} \nonumber ü &= \sqrt{\frac{Z_P}{Z_S}} = \sqrt{\frac{450\Omega}{50\Omega}} \\ &= \sqrt{\frac{9}{1}} = \frac{3}{1} \end{split}\end{equation}



\end{frame}

\begin{frame}
\only<1>{
\begin{QQuestion}{AC306}{Für die Anpassung einer \qty{50}{\ohm} Übertragungsleitung an eine endgespeiste Halbwellenantenne mit einem Fußpunktwiderstand von \qty{2,5}{\kohm} wird ein Übertrager verwendet. Er sollte in etwa ein Windungverhältnis von~...}{1:3 aufweisen.}
{1:7 aufweisen.}
{1:49 aufweisen.}
{1:14 aufweisen.}
\end{QQuestion}

}
\only<2>{
\begin{QQuestion}{AC306}{Für die Anpassung einer \qty{50}{\ohm} Übertragungsleitung an eine endgespeiste Halbwellenantenne mit einem Fußpunktwiderstand von \qty{2,5}{\kohm} wird ein Übertrager verwendet. Er sollte in etwa ein Windungverhältnis von~...}{1:3 aufweisen.}
{\textbf{\textcolor{DARCgreen}{1:7 aufweisen.}}}
{1:49 aufweisen.}
{1:14 aufweisen.}
\end{QQuestion}

}
\end{frame}

\begin{frame}
\frametitle{Lösungsweg}
\begin{itemize}
  \item gegeben: $Z_P = 50\Omega$
  \item gegeben: $Z_S = 2,5k\Omega$
  \item gesucht: $ü$
  \end{itemize}
    \pause
    \begin{equation}\begin{split} \nonumber ü &= \sqrt{\frac{Z_P}{Z_S}} = \sqrt{\frac{50\Omega}{2,5k\Omega}} \\ &= \sqrt{\frac{1}{50}} \approx \frac{1}{7} \end{split}\end{equation}



\end{frame}

\begin{frame}
\only<1>{
\begin{QQuestion}{AC307}{Eine Transformatorwicklung hat einen Drahtdurchmesser von \qty{0,5}{\mm}. Die zulässige Stromdichte beträgt \qty[per-mode=symbol]{2,5}{\A\per\mm\squared}. Wie groß ist der zulässige Strom?}{ca. \qty{0,49}{\A}}
{ca. \qty{1,96}{\A}}
{ca. \qty{1,25}{\A}}
{ca. \qty{0,19}{\A}}
\end{QQuestion}

}
\only<2>{
\begin{QQuestion}{AC307}{Eine Transformatorwicklung hat einen Drahtdurchmesser von \qty{0,5}{\mm}. Die zulässige Stromdichte beträgt \qty[per-mode=symbol]{2,5}{\A\per\mm\squared}. Wie groß ist der zulässige Strom?}{\textbf{\textcolor{DARCgreen}{ca. \qty{0,49}{\A}}}}
{ca. \qty{1,96}{\A}}
{ca. \qty{1,25}{\A}}
{ca. \qty{0,19}{\A}}
\end{QQuestion}

}
\end{frame}

\begin{frame}
\frametitle{Lösungsweg}
\begin{itemize}
  \item gegeben: $d = 0,5mm$
  \item gegeben: Stromdichte $\frac{I}{A} = \frac{2,5A}{1mm^2}$
  \item gesucht: $I_{max}$
  \end{itemize}
    \pause
    $A_{Dr} = \frac{d^2 \cdot \pi}{4} = \frac{(0,5mm)^2 \cdot \pi}{4} \approx 0,196mm^2$
    \pause
    $I_{max} = \frac{I}{A} \cdot A_{Dr} = \frac{2,5A}{1mm^2} \cdot 0,196mm^2 = 0,49A$



\end{frame}%ENDCONTENT


\section{Diode II}
\label{section:diode_2}
\begin{frame}%STARTCONTENT

\only<1>{
\begin{QQuestion}{AC401}{Ein in Durchlassrichtung betriebener PN-Übergang ermöglicht~...}{den Elektronenfluss von P nach N.}
{die Halbierung des Stromflusses.}
{keinen Stromfluss.}
{den Elektronenfluss von N nach P.}
\end{QQuestion}

}
\only<2>{
\begin{QQuestion}{AC401}{Ein in Durchlassrichtung betriebener PN-Übergang ermöglicht~...}{den Elektronenfluss von P nach N.}
{die Halbierung des Stromflusses.}
{keinen Stromfluss.}
{\textbf{\textcolor{DARCgreen}{den Elektronenfluss von N nach P.}}}
\end{QQuestion}

}
\end{frame}

\begin{frame}
\only<1>{
\begin{QQuestion}{AC403}{Wie verhält sich die Durchlassspannung einer Diode in Abhängigkeit von der Temperatur?}{Die Spannung ist unabhängig von der Temperatur.}
{Die Spannung sinkt bei steigender Temperatur.}
{Die Spannung oszilliert mit steigender Temperatur.}
{Die Spannung steigt bei steigender Temperatur.}
\end{QQuestion}

}
\only<2>{
\begin{QQuestion}{AC403}{Wie verhält sich die Durchlassspannung einer Diode in Abhängigkeit von der Temperatur?}{Die Spannung ist unabhängig von der Temperatur.}
{\textbf{\textcolor{DARCgreen}{Die Spannung sinkt bei steigender Temperatur.}}}
{Die Spannung oszilliert mit steigender Temperatur.}
{Die Spannung steigt bei steigender Temperatur.}
\end{QQuestion}

}
\end{frame}

\begin{frame}
\only<1>{
\begin{QQuestion}{AC404}{Wie verhält sich die Kapazität einer Kapazitätsdiode (Varicap)?}{Sie nimmt mit abnehmender Sperrspannung zu.}
{Sie nimmt mit abnehmendem Durchlassstrom zu.}
{Sie nimmt mit zunehmender Sperrspannung zu.}
{Sie nimmt mit zunehmendem Durchlassstrom zu.}
\end{QQuestion}

}
\only<2>{
\begin{QQuestion}{AC404}{Wie verhält sich die Kapazität einer Kapazitätsdiode (Varicap)?}{\textbf{\textcolor{DARCgreen}{Sie nimmt mit abnehmender Sperrspannung zu.}}}
{Sie nimmt mit abnehmendem Durchlassstrom zu.}
{Sie nimmt mit zunehmender Sperrspannung zu.}
{Sie nimmt mit zunehmendem Durchlassstrom zu.}
\end{QQuestion}

}
\end{frame}

\begin{frame}
\only<1>{
\begin{PQuestion}{AC405}{Das folgende Signal wird als $U_1$ an den Eingang der Schaltung mit Siliziumdioden gelegt. Wie sieht das zugehörige Ausgangssignal $U_2$ aus?}{\DARCimage{1.0\linewidth}{17include}}
{\DARCimage{1.0\linewidth}{15include}}
{\DARCimage{1.0\linewidth}{16include}}
{\DARCimage{1.0\linewidth}{14include}}
{\DARCimage{1.0\linewidth}{13include}}\end{PQuestion}

}
\only<2>{
\begin{PQuestion}{AC405}{Das folgende Signal wird als $U_1$ an den Eingang der Schaltung mit Siliziumdioden gelegt. Wie sieht das zugehörige Ausgangssignal $U_2$ aus?}{\DARCimage{1.0\linewidth}{17include}}
{\DARCimage{1.0\linewidth}{15include}}
{\DARCimage{1.0\linewidth}{16include}}
{\textbf{\textcolor{DARCgreen}{\DARCimage{1.0\linewidth}{14include}}}}
{\DARCimage{1.0\linewidth}{13include}}\end{PQuestion}

}
\end{frame}

\begin{frame}
\only<1>{
\begin{PQuestion}{AC406}{Das folgende Signal wird als $U_1$ an den Eingang der Schaltung mit Germaniumdioden gelegt. Wie sieht das zugehörige Ausgangssignal $U_2$ aus?}{\DARCimage{1.0\linewidth}{17include}}
{\DARCimage{1.0\linewidth}{15include}}
{\DARCimage{1.0\linewidth}{16include}}
{\DARCimage{1.0\linewidth}{14include}}
{\DARCimage{1.0\linewidth}{13include}}\end{PQuestion}

}
\only<2>{
\begin{PQuestion}{AC406}{Das folgende Signal wird als $U_1$ an den Eingang der Schaltung mit Germaniumdioden gelegt. Wie sieht das zugehörige Ausgangssignal $U_2$ aus?}{\textbf{\textcolor{DARCgreen}{\DARCimage{1.0\linewidth}{17include}}}}
{\DARCimage{1.0\linewidth}{15include}}
{\DARCimage{1.0\linewidth}{16include}}
{\DARCimage{1.0\linewidth}{14include}}
{\DARCimage{1.0\linewidth}{13include}}\end{PQuestion}

}
\end{frame}

\begin{frame}
\only<1>{
\begin{QQuestion}{AC407}{Welches Bauteil kann durch Lichteinfall elektrischen Strom erzeugen?}{Blindwiderstand}
{Fotowiderstand}
{Kapazitätsdiode}
{Fotodiode}
\end{QQuestion}

}
\only<2>{
\begin{QQuestion}{AC407}{Welches Bauteil kann durch Lichteinfall elektrischen Strom erzeugen?}{Blindwiderstand}
{Fotowiderstand}
{Kapazitätsdiode}
{\textbf{\textcolor{DARCgreen}{Fotodiode}}}
\end{QQuestion}

}
\end{frame}

\begin{frame}
\only<1>{
\begin{QQuestion}{AC408}{Die Hauptfunktion eines Optokopplers ist~...}{die galvanische Entkopplung zweier Stromkreise durch Licht.}
{die Erzeugung von hochfrequentem Wechselstrom durch Licht.}
{die Signalanzeige durch Licht.}
{die Erzeugung von Gleichstrom durch Licht.}
\end{QQuestion}

}
\only<2>{
\begin{QQuestion}{AC408}{Die Hauptfunktion eines Optokopplers ist~...}{\textbf{\textcolor{DARCgreen}{die galvanische Entkopplung zweier Stromkreise durch Licht.}}}
{die Erzeugung von hochfrequentem Wechselstrom durch Licht.}
{die Signalanzeige durch Licht.}
{die Erzeugung von Gleichstrom durch Licht.}
\end{QQuestion}

}
\end{frame}%ENDCONTENT


\section{Transistor II}
\label{section:transistor_2}
\begin{frame}%STARTCONTENT

\frametitle{Bipolarer Transistor}
\end{frame}

\begin{frame}
\only<1>{
\begin{QQuestion}{AC501}{Ein bipolarer Transistor ist~...}{thermisch gesteuert.}
{spannungsgesteuert.}
{stromgesteuert.}
{feldgesteuert.}
\end{QQuestion}

}
\only<2>{
\begin{QQuestion}{AC501}{Ein bipolarer Transistor ist~...}{thermisch gesteuert.}
{spannungsgesteuert.}
{\textbf{\textcolor{DARCgreen}{stromgesteuert.}}}
{feldgesteuert.}
\end{QQuestion}

}
\end{frame}

\begin{frame}
\only<1>{
\begin{QQuestion}{AC503}{Mit welchem Anschluss ist der p-dotierte Bereich eines NPN-Transistors verbunden?}{Gehäuse}
{Kollektor}
{Emitter}
{Basis}
\end{QQuestion}

}
\only<2>{
\begin{QQuestion}{AC503}{Mit welchem Anschluss ist der p-dotierte Bereich eines NPN-Transistors verbunden?}{Gehäuse}
{Kollektor}
{Emitter}
{\textbf{\textcolor{DARCgreen}{Basis}}}
\end{QQuestion}

}
\end{frame}

\begin{frame}
\only<1>{
\begin{QQuestion}{AC504}{Mit welchem Anschluss ist der n-dotierte Bereich eines PNP-Transistors verbunden?}{Kollektor}
{Emitter}
{Basis}
{Gehäuse}
\end{QQuestion}

}
\only<2>{
\begin{QQuestion}{AC504}{Mit welchem Anschluss ist der n-dotierte Bereich eines PNP-Transistors verbunden?}{Kollektor}
{Emitter}
{\textbf{\textcolor{DARCgreen}{Basis}}}
{Gehäuse}
\end{QQuestion}

}
\end{frame}

\begin{frame}
\only<1>{
\begin{QQuestion}{AC505}{Bei einem bipolaren Transistor in leitendem Zustand befindet sich der Basis-Emitter-PN-Übergang~...}{in Sperrrichtung.}
{im Leerlauf.}
{im Kurzschluss.}
{in Durchlassrichtung.}
\end{QQuestion}

}
\only<2>{
\begin{QQuestion}{AC505}{Bei einem bipolaren Transistor in leitendem Zustand befindet sich der Basis-Emitter-PN-Übergang~...}{in Sperrrichtung.}
{im Leerlauf.}
{im Kurzschluss.}
{\textbf{\textcolor{DARCgreen}{in Durchlassrichtung.}}}
\end{QQuestion}

}
\end{frame}

\begin{frame}
\frametitle{Rechnungen}
\end{frame}

\begin{frame}
\only<1>{
\begin{PQuestion}{AC515}{Die Betriebsspannung beträgt \qty{12}{\V}, der Kollektorstrom soll \qty{5}{\mA} betragen, die Gleichstromverstärkung des Transistors beträgt 298. Berechnen Sie den Vorwiderstand $R_1$.}{ca. \qty{680}{\kohm}}
{ca. \qty{715}{\kohm}}
{ca. \qty{68}{\kohm}}
{ca. \qty{2,3}{\kohm}}
{\DARCimage{1.0\linewidth}{360include}}\end{PQuestion}

}
\only<2>{
\begin{PQuestion}{AC515}{Die Betriebsspannung beträgt \qty{12}{\V}, der Kollektorstrom soll \qty{5}{\mA} betragen, die Gleichstromverstärkung des Transistors beträgt 298. Berechnen Sie den Vorwiderstand $R_1$.}{\textbf{\textcolor{DARCgreen}{ca. \qty{680}{\kohm}}}}
{ca. \qty{715}{\kohm}}
{ca. \qty{68}{\kohm}}
{ca. \qty{2,3}{\kohm}}
{\DARCimage{1.0\linewidth}{360include}}\end{PQuestion}

}
\end{frame}

\begin{frame}
\frametitle{Lösungsweg}
\begin{itemize}
  \item gegeben: $U = 12V$
  \item gegeben: $I_{\textrm{C}} = 5mA$
  \item gegeben: $B = 298$
  \item gegeben: $U_{\textrm{BE}} = 0,6V$
  \item gesucht: $R_1$
  \end{itemize}
    \pause
    $B = \frac{I_{\textrm{C}}}{I_{\textrm{B}}} \Rightarrow I_{\textrm{B}} = \frac{I_{\textrm{C}}}{B} = \frac{5mA}{298} = 16,779\mu A$
    \pause
    $R_1 = \frac{U-U_{\textrm{BE}}}{I_{\textrm{B}}} = \frac{12V -- 0,6V}{16,779\mu A} \approx 680k\Omega$



\end{frame}

\begin{frame}
\only<1>{
\begin{PQuestion}{AC518}{Die Betriebsspannung beträgt \qty{10}{\V}, der Kollektorstrom soll \qty{2}{\mA} betragen, die Gleichstromverstärkung des Transistors beträgt 200. Durch den Querwiderstand $R_2$ soll der zehnfache Basisstrom fließen. Berechnen Sie den Vorwiderstand $R_1$.}{ca. \qty{76,4}{\kohm}}
{ca. \qty{940}{\kohm}}
{ca. \qty{85,5}{\kohm}}
{ca. \qty{540}{\kohm}}
{\DARCimage{1.0\linewidth}{361include}}\end{PQuestion}

}
\only<2>{
\begin{PQuestion}{AC518}{Die Betriebsspannung beträgt \qty{10}{\V}, der Kollektorstrom soll \qty{2}{\mA} betragen, die Gleichstromverstärkung des Transistors beträgt 200. Durch den Querwiderstand $R_2$ soll der zehnfache Basisstrom fließen. Berechnen Sie den Vorwiderstand $R_1$.}{ca. \qty{76,4}{\kohm}}
{ca. \qty{940}{\kohm}}
{\textbf{\textcolor{DARCgreen}{ca. \qty{85,5}{\kohm}}}}
{ca. \qty{540}{\kohm}}
{\DARCimage{1.0\linewidth}{361include}}\end{PQuestion}

}
\end{frame}

\begin{frame}
\frametitle{Lösungsweg}
\begin{columns}
    \begin{column}{0.48\textwidth}
    \begin{itemize}
  \item gegeben: $U = 10V$
  \item gegeben: $I_{\textrm{C}} = 2mA$
  \item gegeben: $B = 200$
  \end{itemize}

    \end{column}
   \begin{column}{0.48\textwidth}
       \begin{itemize}
  \item gegeben: $U_{\textrm{R2}} = 0,6$
  \item gegeben: $I_{\textrm{R2}} = 10 \cdot I_{\textrm{B}}$
  \item gesucht: $R_1$
  \end{itemize}

   \end{column}
\end{columns}
    \pause
    $B = \frac{I_{\textrm{C}}}{I_{\textrm{B}}} \Rightarrow I_{\textrm{B}} = \frac{I_{\textrm{C}}}{B} = \frac{2mA}{200} = 10\mu A$
    \pause
    $U_{\textrm{R1}} = U -- U_{\textrm{R2}} = 10V -- 0,6V = 9,4V$
    \pause
    $I_{\textrm{R1}} = I_{\textrm{B}} + I_{\textrm{R2}} = I_{\textrm{B}} + 10 \cdot I_{\textrm{B}} = 110\mu A$
    \pause
    $R_1 = \frac{U_{\textrm{R1}}}{I_{\textrm{R1}}} = \frac{9,4V}{110\mu A} \approx 85,5k\Omega$



\end{frame}

\begin{frame}
\only<1>{
\begin{PQuestion}{AC517}{Die Betriebsspannung beträgt \qty{10}{\V}, der Kollektorstrom soll \qty{2}{\mA} betragen, die Gleichstromverstärkung des Transistors beträgt 200. Durch den Querwiderstand $R_2$ soll der zehnfache Basisstrom fließen. Am Emitterwiderstand soll \qty{1}{\V} abfallen. Berechnen Sie den Vorwiderstand $R_1$.}{ca. \qty{540}{\kohm}}
{ca. \qty{76,4}{\kohm}}
{ca. \qty{85,5}{\kohm}}
{ca. \qty{940}{\kohm}}
{\DARCimage{1.0\linewidth}{362include}}\end{PQuestion}

}
\only<2>{
\begin{PQuestion}{AC517}{Die Betriebsspannung beträgt \qty{10}{\V}, der Kollektorstrom soll \qty{2}{\mA} betragen, die Gleichstromverstärkung des Transistors beträgt 200. Durch den Querwiderstand $R_2$ soll der zehnfache Basisstrom fließen. Am Emitterwiderstand soll \qty{1}{\V} abfallen. Berechnen Sie den Vorwiderstand $R_1$.}{ca. \qty{540}{\kohm}}
{\textbf{\textcolor{DARCgreen}{ca. \qty{76,4}{\kohm}}}}
{ca. \qty{85,5}{\kohm}}
{ca. \qty{940}{\kohm}}
{\DARCimage{1.0\linewidth}{362include}}\end{PQuestion}

}
\end{frame}

\begin{frame}
\frametitle{Lösungsweg}
\begin{columns}
    \begin{column}{0.48\textwidth}
    \begin{itemize}
  \item gegeben: $U = 10V$
  \item gegeben: $I_{\textrm{C}} = 2mA$
  \item gegeben: $B = 200$
  \end{itemize}

    \end{column}
   \begin{column}{0.48\textwidth}
       \begin{itemize}
  \item gegeben: $U_{\textrm{BE}} = 0,6V$
  \item gegeben: $U_{\textrm{RE}} = 1V$
  \item gegeben: $I_{\textrm{R2}} = 10 \cdot I_{\textrm{B}}$
  \end{itemize}

   \end{column}
\end{columns}

\begin{itemize}
  \item gesucht: $R_1$
  \end{itemize}
    \pause
    $B = \frac{I_{\textrm{C}}}{I_{\textrm{B}}} \Rightarrow I_{\textrm{B}} = \frac{I_{\textrm{C}}}{B} = \frac{2mA}{200} = 10\mu A$
    \pause
    $U_{\textrm{R2}} = U_{\textrm{BE}} + U_{R_{\textrm{E}}} = 0,6V + 1V = 1,6V$
    \pause
    $U_{\textrm{R1}} = U -- U_{\textrm{R2}} = 10V -- 1,6V = 8,4V$
    \pause
    $I_{\textrm{R1}} = I_{\textrm{B}} + I_{\textrm{R2}} = I_{\textrm{B}} + 10 \cdot I_{\textrm{B}} = 110\mu A$
    \pause
    $R_1 = \frac{U_{\textrm{R1}}}{I_{\textrm{R1}}} = \frac{8,4V}{110\mu A} \approx 76,4k\Omega$



\end{frame}

\begin{frame}
\only<1>{
\begin{PQuestion}{AC516}{Warum soll bei dem gezeigten Basisspannungsteiler der Strom durch $R_2$ etwa 10-mal größer als der Basisstrom sein?}{Damit sich der Basisstrom bei Erwärmung nicht ändert.}
{Damit der Arbeitspunkt stabil bleibt.}
{Damit $R_2$ eine Stromgegenkopplung bewirkt.}
{Damit $R_2$ eine Spannungsgegenkopplung bewirkt}
{\DARCimage{1.0\linewidth}{361include}}\end{PQuestion}

}
\only<2>{
\begin{PQuestion}{AC516}{Warum soll bei dem gezeigten Basisspannungsteiler der Strom durch $R_2$ etwa 10-mal größer als der Basisstrom sein?}{Damit sich der Basisstrom bei Erwärmung nicht ändert.}
{\textbf{\textcolor{DARCgreen}{Damit der Arbeitspunkt stabil bleibt.}}}
{Damit $R_2$ eine Stromgegenkopplung bewirkt.}
{Damit $R_2$ eine Spannungsgegenkopplung bewirkt}
{\DARCimage{1.0\linewidth}{361include}}\end{PQuestion}

}
\end{frame}

\begin{frame}
\only<1>{
\begin{PQuestion}{AC519}{Was passiert, wenn der Widerstand $R_1$ durch eine fehlerhafte Lötstelle an einer Seite keinen Kontakt mehr zur Schaltung hat? Welche Beschreibung trifft zu?}{Es fließt Kurzschlussstrom. Der Transistor wird zerstört.}
{Es fließt kein Kollektorstrom mehr. Die Kollektorspannung steigt auf die Betriebsspannung an.}
{Der Kollektorstrom wird nur durch $R_{\symup{C}}$ begrenzt. Die Kollektorspannung sinkt auf zirka \qty{0,1}{\V}.}
{Der Kollektorstrom steigt stark an. Die Kollektorspannung erhöht sich.}
{\DARCimage{1.0\linewidth}{365include}}\end{PQuestion}

}
\only<2>{
\begin{PQuestion}{AC519}{Was passiert, wenn der Widerstand $R_1$ durch eine fehlerhafte Lötstelle an einer Seite keinen Kontakt mehr zur Schaltung hat? Welche Beschreibung trifft zu?}{Es fließt Kurzschlussstrom. Der Transistor wird zerstört.}
{\textbf{\textcolor{DARCgreen}{Es fließt kein Kollektorstrom mehr. Die Kollektorspannung steigt auf die Betriebsspannung an.}}}
{Der Kollektorstrom wird nur durch $R_{\symup{C}}$ begrenzt. Die Kollektorspannung sinkt auf zirka \qty{0,1}{\V}.}
{Der Kollektorstrom steigt stark an. Die Kollektorspannung erhöht sich.}
{\DARCimage{1.0\linewidth}{365include}}\end{PQuestion}

}
\end{frame}

\begin{frame}
\only<1>{
\begin{PQuestion}{AC520}{Was passiert, wenn der Widerstand $R_2$ durch eine fehlerhafte Lötstelle an einer Seite keinen Kontakt mehr zur Schaltung hat? In welcher Antwort sind beide Aussagen richtig?}{Es fließt kein Kollektorstrom mehr. Die Kollektorspannung steigt auf die Betriebsspannung an.}
{Es fließt Kurzschlussstrom. Der Transistor wird zerstört.}
{Der Kollektorstrom wird nur durch $R_{\symup{C}}$ begrenzt. Die Kollektorspannung sinkt auf zirka \qty{0,1}{\V}.}
{Der Kollektorstrom steigt stark an. Die Kollektorspannung erhöht sich.}
{\DARCimage{1.0\linewidth}{364include}}\end{PQuestion}

}
\only<2>{
\begin{PQuestion}{AC520}{Was passiert, wenn der Widerstand $R_2$ durch eine fehlerhafte Lötstelle an einer Seite keinen Kontakt mehr zur Schaltung hat? In welcher Antwort sind beide Aussagen richtig?}{Es fließt kein Kollektorstrom mehr. Die Kollektorspannung steigt auf die Betriebsspannung an.}
{Es fließt Kurzschlussstrom. Der Transistor wird zerstört.}
{\textbf{\textcolor{DARCgreen}{Der Kollektorstrom wird nur durch $R_{\symup{C}}$ begrenzt. Die Kollektorspannung sinkt auf zirka \qty{0,1}{\V}.}}}
{Der Kollektorstrom steigt stark an. Die Kollektorspannung erhöht sich.}
{\DARCimage{1.0\linewidth}{364include}}\end{PQuestion}

}
\end{frame}

\begin{frame}
\frametitle{Feldeffekttransistor}
\end{frame}

\begin{frame}
\only<1>{
\begin{QQuestion}{AC502}{Ein Feldeffekttransistor ist~...}{stromgesteuert.}
{spannungsgesteuert.}
{leistungsgesteuert.}
{optisch gesteuert.}
\end{QQuestion}

}
\only<2>{
\begin{QQuestion}{AC502}{Ein Feldeffekttransistor ist~...}{stromgesteuert.}
{\textbf{\textcolor{DARCgreen}{spannungsgesteuert.}}}
{leistungsgesteuert.}
{optisch gesteuert.}
\end{QQuestion}

}
\end{frame}

\begin{frame}
\only<1>{
\begin{PQuestion}{AC506}{Welches Bauteil wird durch das Schaltzeichen symbolisiert?}{Lautsprecher}
{Bipolartransistor}
{Diode}
{Feldeffekttransistor}
{\DARCimage{0.25\linewidth}{561include}}\end{PQuestion}

}
\only<2>{
\begin{PQuestion}{AC506}{Welches Bauteil wird durch das Schaltzeichen symbolisiert?}{Lautsprecher}
{Bipolartransistor}
{Diode}
{\textbf{\textcolor{DARCgreen}{Feldeffekttransistor}}}
{\DARCimage{0.25\linewidth}{561include}}\end{PQuestion}

}
\end{frame}

\begin{frame}
\only<1>{
\begin{PQuestion}{AC513}{Wie bezeichnet man die Anschlüsse des abgebildeten Transistors?}{1: Drain, 2: Source, 3: Gate}
{1: Anode, 2: Kollektor, 3: Gate}
{1: Anode, 2: Kathode, 3: Gate}
{1: Kollektor, 2: Emitter, 3: Basis}
{\DARCimage{0.25\linewidth}{376include}}\end{PQuestion}

}
\only<2>{
\begin{PQuestion}{AC513}{Wie bezeichnet man die Anschlüsse des abgebildeten Transistors?}{\textbf{\textcolor{DARCgreen}{1: Drain, 2: Source, 3: Gate}}}
{1: Anode, 2: Kollektor, 3: Gate}
{1: Anode, 2: Kathode, 3: Gate}
{1: Kollektor, 2: Emitter, 3: Basis}
{\DARCimage{0.25\linewidth}{376include}}\end{PQuestion}

}
\end{frame}

\begin{frame}
\only<1>{
\begin{QQuestion}{AC512}{Wie lauten die Bezeichnungen der Anschlüsse eines Feldeffekttransistors?}{Emitter, Basis, Kollektor}
{Drain, Gate, Source}
{Emitter, Drain, Source}
{Gate, Source, Kollektor}
\end{QQuestion}

}
\only<2>{
\begin{QQuestion}{AC512}{Wie lauten die Bezeichnungen der Anschlüsse eines Feldeffekttransistors?}{Emitter, Basis, Kollektor}
{\textbf{\textcolor{DARCgreen}{Drain, Gate, Source}}}
{Emitter, Drain, Source}
{Gate, Source, Kollektor}
\end{QQuestion}

}
\end{frame}

\begin{frame}
\only<1>{
\begin{QQuestion}{AC514}{Wie erfolgt die Steuerung des Stroms im Feldeffekttransistor (FET)?}{Die Gate-Source-Spannung steuert den Widerstand des Kanals zwischen Source und Drain.}
{Die Gate-Source-Spannung steuert den Gatestrom.}
{Der Gatestrom steuert den Drainstrom.}
{Der Gatestrom steuert den Widerstand des Kanals zwischen Source und Drain.}
\end{QQuestion}

}
\only<2>{
\begin{QQuestion}{AC514}{Wie erfolgt die Steuerung des Stroms im Feldeffekttransistor (FET)?}{\textbf{\textcolor{DARCgreen}{Die Gate-Source-Spannung steuert den Widerstand des Kanals zwischen Source und Drain.}}}
{Die Gate-Source-Spannung steuert den Gatestrom.}
{Der Gatestrom steuert den Drainstrom.}
{Der Gatestrom steuert den Widerstand des Kanals zwischen Source und Drain.}
\end{QQuestion}

}
\end{frame}

\begin{frame}
\frametitle{Bauarten FET}
\end{frame}

\begin{frame}
\only<1>{
\begin{PQuestion}{AC507}{Welche Bezeichnungen für die Bauelemente sind richtig?}{1: Selbstleitender N-Kanal-Sperrschicht-FET
2: Selbstleitender P-Kanal-Sperrschicht-FET}
{1: Selbstsperrender N-Kanal-Sperrschicht-FET
2: Selbstsperrender P-Kanal-Sperrschicht-FET}
{1: Selbstleitender P-Kanal-Sperrschicht-FET
2: Selbstleitender N-Kanal-Sperrschicht-FET}
{1: Selbstsperrender P-Kanal-Sperrschicht-FET
2: Selbstsperrender N-Kanal-Sperrschicht-FET}
{\DARCimage{1.0\linewidth}{271include}}\end{PQuestion}

}
\only<2>{
\begin{PQuestion}{AC507}{Welche Bezeichnungen für die Bauelemente sind richtig?}{\textbf{\textcolor{DARCgreen}{1: Selbstleitender N-Kanal-Sperrschicht-FET
2: Selbstleitender P-Kanal-Sperrschicht-FET}}}
{1: Selbstsperrender N-Kanal-Sperrschicht-FET
2: Selbstsperrender P-Kanal-Sperrschicht-FET}
{1: Selbstleitender P-Kanal-Sperrschicht-FET
2: Selbstleitender N-Kanal-Sperrschicht-FET}
{1: Selbstsperrender P-Kanal-Sperrschicht-FET
2: Selbstsperrender N-Kanal-Sperrschicht-FET}
{\DARCimage{1.0\linewidth}{271include}}\end{PQuestion}

}
\end{frame}

\begin{frame}
\only<1>{
\begin{PQuestion}{AC508}{Der folgende Transistor ist ein~...}{Selbstleitender P-Kanal-Isolierschicht-FET (MOSFET).}
{Selbstsperrender P-Kanal-Isolierschicht-FET (MOSFET).}
{Selbstleitender N-Kanal-Isolierschicht-FET (MOSFET).}
{Selbstsperrender N-Kanal-Isolierschicht-FET (MOSFET).}
{\DARCimage{0.25\linewidth}{272include}}\end{PQuestion}

}
\only<2>{
\begin{PQuestion}{AC508}{Der folgende Transistor ist ein~...}{Selbstleitender P-Kanal-Isolierschicht-FET (MOSFET).}
{Selbstsperrender P-Kanal-Isolierschicht-FET (MOSFET).}
{Selbstleitender N-Kanal-Isolierschicht-FET (MOSFET).}
{\textbf{\textcolor{DARCgreen}{Selbstsperrender N-Kanal-Isolierschicht-FET (MOSFET).}}}
{\DARCimage{0.25\linewidth}{272include}}\end{PQuestion}

}
\end{frame}

\begin{frame}
\only<1>{
\begin{question2x2}{AC509}{Welcher der folgenden Transistoren ist ein selbstsperrender N-Kanal-MOSFET?}{\DARCimage{1.0\linewidth}{275include}}
{\DARCimage{1.0\linewidth}{274include}}
{\DARCimage{1.0\linewidth}{273include}}
{\DARCimage{1.0\linewidth}{276include}}
\end{question2x2}

}
\only<2>{
\begin{question2x2}{AC509}{Welcher der folgenden Transistoren ist ein selbstsperrender N-Kanal-MOSFET?}{\DARCimage{1.0\linewidth}{275include}}
{\DARCimage{1.0\linewidth}{274include}}
{\DARCimage{1.0\linewidth}{273include}}
{\textbf{\textcolor{DARCgreen}{\DARCimage{1.0\linewidth}{276include}}}}
\end{question2x2}

}
\end{frame}

\begin{frame}
\only<1>{
\begin{question2x2}{AC510}{Welcher der folgenden Transistoren ist ein selbstleitender N-Kanal-MOSFET?}{\DARCimage{1.0\linewidth}{275include}}
{\DARCimage{1.0\linewidth}{273include}}
{\DARCimage{1.0\linewidth}{274include}}
{\DARCimage{1.0\linewidth}{276include}}
\end{question2x2}

}
\only<2>{
\begin{question2x2}{AC510}{Welcher der folgenden Transistoren ist ein selbstleitender N-Kanal-MOSFET?}{\DARCimage{1.0\linewidth}{275include}}
{\DARCimage{1.0\linewidth}{273include}}
{\textbf{\textcolor{DARCgreen}{\DARCimage{1.0\linewidth}{274include}}}}
{\DARCimage{1.0\linewidth}{276include}}
\end{question2x2}

}
\end{frame}

\begin{frame}
\only<1>{
\begin{question2x2}{AC511}{Welcher der folgenden Transistoren ist ein selbstleitender P-Kanal-MOSFET?}{\DARCimage{1.0\linewidth}{274include}}
{\DARCimage{1.0\linewidth}{273include}}
{\DARCimage{1.0\linewidth}{276include}}
{\DARCimage{1.0\linewidth}{275include}}
\end{question2x2}

}
\only<2>{
\begin{question2x2}{AC511}{Welcher der folgenden Transistoren ist ein selbstleitender P-Kanal-MOSFET?}{\DARCimage{1.0\linewidth}{274include}}
{\textbf{\textcolor{DARCgreen}{\DARCimage{1.0\linewidth}{273include}}}}
{\DARCimage{1.0\linewidth}{276include}}
{\DARCimage{1.0\linewidth}{275include}}
\end{question2x2}

}
\end{frame}

\begin{frame}
\frametitle{Rechnungen}
\end{frame}

\begin{frame}
\only<1>{
\begin{PQuestion}{AC521}{Wie groß ist die Gate-Source-Spannung in der gezeichneten Schaltung? $U_{\symup{B}} = \qty{44}{\V}$; $R_1 = 10~k\Omega$; $R_2 = 1~k\Omega$; $R_3 = 2,2~k\Omega$~...}{\qty{4}{\V}}
{\qty{8}{\V}}
{\qty{0,7}{\V} }
{\qty{4,4}{\V}}
{\DARCimage{0.5\linewidth}{345include}}\end{PQuestion}

}
\only<2>{
\begin{PQuestion}{AC521}{Wie groß ist die Gate-Source-Spannung in der gezeichneten Schaltung? $U_{\symup{B}} = \qty{44}{\V}$; $R_1 = 10~k\Omega$; $R_2 = 1~k\Omega$; $R_3 = 2,2~k\Omega$~...}{\textbf{\textcolor{DARCgreen}{\qty{4}{\V}}}}
{\qty{8}{\V}}
{\qty{0,7}{\V} }
{\qty{4,4}{\V}}
{\DARCimage{0.5\linewidth}{345include}}\end{PQuestion}

}
\end{frame}

\begin{frame}
\frametitle{Lösungsweg}
\begin{columns}
    \begin{column}{0.48\textwidth}
    \begin{itemize}
  \item gegeben: $U_{\textrm{B}} = 44V$
  \item gegeben: $R_1 = 10k\Omega$
  \item gegeben: $R_2 = 1k\Omega$
  \item gegeben: $R_3 = 2,2k\Omega$
  \item gesucht: $U_{\textrm{GS}}$
  \item Ansatz: Spannungsteiler über $R_1$ und $R_2$, mit $U_{\textrm{GS}} = U_{\textrm{R2}}$
  \end{itemize}

    \end{column}
   \begin{column}{0.48\textwidth}
       
    \pause
    \begin{equation}\begin{align}\nonumber \frac{U_{\textrm{R2}}}{U_{\textrm{B}}} &= \frac{R_2}{R_1+R_2}\\ \nonumber \Rightarrow U_{\textrm{R2}} &= \frac{R_2}{R_1+R_2} \cdot U_{\textrm{G}}\\ \nonumber &= \frac{1k\Omega}{10k\Omega+1k\Omega} \cdot 44V\\ \nonumber &= \frac{1}{11} \cdot 44V = 4V \end{align}\end{equation}




   \end{column}
\end{columns}

\end{frame}

\begin{frame}
\only<1>{
\begin{PQuestion}{AC522}{Wie groß muss $R_2$ gewählt werden, damit sich eine Spannung von \qty{2,8}{\V} zwischen Gate und Source einstellt? $U_{\symup{B}}$=\qty{44}{\V}; $R_1$=\qty{10}{\kohm}; $R_3$=\qty{2,2}{\kohm}~...}{ca. \qty{820}{\ohm}}
{ca. \qty{1405}{\ohm}}
{ca. \qty{68}{\ohm}}
{ca. \qty{680}{\ohm}}
{\DARCimage{0.5\linewidth}{345include}}\end{PQuestion}

}
\only<2>{
\begin{PQuestion}{AC522}{Wie groß muss $R_2$ gewählt werden, damit sich eine Spannung von \qty{2,8}{\V} zwischen Gate und Source einstellt? $U_{\symup{B}}$=\qty{44}{\V}; $R_1$=\qty{10}{\kohm}; $R_3$=\qty{2,2}{\kohm}~...}{ca. \qty{820}{\ohm}}
{ca. \qty{1405}{\ohm}}
{ca. \qty{68}{\ohm}}
{\textbf{\textcolor{DARCgreen}{ca. \qty{680}{\ohm}}}}
{\DARCimage{0.5\linewidth}{345include}}\end{PQuestion}

}
\end{frame}

\begin{frame}
\frametitle{Lösungsweg}
\begin{columns}
    \begin{column}{0.48\textwidth}
    \begin{itemize}
  \item gegeben: $U_{\textrm{B}} = 44V$
  \item gegeben: $R_1 = 10k\Omega$
  \item gegeben: $R_3 = 2,2k\Omega$
  \item gegeben: $U_{\textrm{GS}} = U_{\textrm{R2}} = 2,8V$
  \item gegeben: $U_{\textrm{B}} = U_{\textrm{R1}} + U_{\textrm{R2}}$
  \item gesucht: $R_2$
  \end{itemize}

    \end{column}
   \begin{column}{0.48\textwidth}
       
    \pause
    \begin{equation}\begin{align}\nonumber \frac{U_{\textrm{R1}}}{U_{\textrm{R2}}} &= \frac{R_1}{R_2}\\ \nonumber \Rightarrow R_2 &= R_1 \cdot \frac{U_{\textrm{R2}}}{U_{\textrm{R1}}}\\ \nonumber &= R_1 \cdot \frac{U_{\textrm{R2}}}{U_{\textrm{B}}-U_{\textrm{GS}}}\\ \nonumber &= 10k\Omega \cdot \frac{2,8V}{44V-2,8V}\\ \nonumber &\approx 680\Omega \end{align}\end{equation}




   \end{column}
\end{columns}

\end{frame}

\begin{frame}
\only<1>{
\begin{QQuestion}{AC523}{Welche Verlustleistung erzeugt ein Power-MOS-FET mit $R_{\symup{DSon}}$ = \qty{4}{\m\ohm} bei einem Strom von \qty{25}{\A}?}{\qty{2,5}{\W}}
{\qty{1}{\W}}
{\qty{0,1}{\W}}
{\qty{6,25}{\W}}
\end{QQuestion}

}
\only<2>{
\begin{QQuestion}{AC523}{Welche Verlustleistung erzeugt ein Power-MOS-FET mit $R_{\symup{DSon}}$ = \qty{4}{\m\ohm} bei einem Strom von \qty{25}{\A}?}{\textbf{\textcolor{DARCgreen}{\qty{2,5}{\W}}}}
{\qty{1}{\W}}
{\qty{0,1}{\W}}
{\qty{6,25}{\W}}
\end{QQuestion}

}
\end{frame}

\begin{frame}
\frametitle{Lösungsweg}
\begin{itemize}
  \item gegeben: $R_{\textrm{DSon}} = 4m\Omega$
  \item gegeben: $I = 25A$
  \item gesucht: $P$
  \end{itemize}
    \pause
    $P = I^2 \cdot R = 25^2A \cdot 4m\Omega = 2,5W$



\end{frame}

\begin{frame}
\frametitle{Freilaufdiode}
\end{frame}

\begin{frame}
\only<1>{
\begin{question2x2}{AC524}{In welcher der folgenden Schaltungen ist die Freilaufdiode richtig eingesetzt?}{\DARCimage{1.0\linewidth}{429include}}
{\DARCimage{1.0\linewidth}{427include}}
{\DARCimage{1.0\linewidth}{428include}}
{\DARCimage{1.0\linewidth}{426include}}
\end{question2x2}

}
\only<2>{
\begin{question2x2}{AC524}{In welcher der folgenden Schaltungen ist die Freilaufdiode richtig eingesetzt?}{\DARCimage{1.0\linewidth}{429include}}
{\DARCimage{1.0\linewidth}{427include}}
{\DARCimage{1.0\linewidth}{428include}}
{\textbf{\textcolor{DARCgreen}{\DARCimage{1.0\linewidth}{426include}}}}
\end{question2x2}

}
\end{frame}%ENDCONTENT


\section{Halbleiter II}
\label{section:halbleiter_2}
\begin{frame}%STARTCONTENT

\only<1>{
\begin{QQuestion}{AC402}{Wie verhalten sich die Elektronen in einem in Durchlassrichtung betriebenen PN-Übergang?}{Sie wandern von N nach P.}
{Sie wandern von P nach N.}
{Sie bleiben im N-Bereich.}
{Sie zerfallen beim Übergang.}
\end{QQuestion}

}
\only<2>{
\begin{QQuestion}{AC402}{Wie verhalten sich die Elektronen in einem in Durchlassrichtung betriebenen PN-Übergang?}{\textbf{\textcolor{DARCgreen}{Sie wandern von N nach P.}}}
{Sie wandern von P nach N.}
{Sie bleiben im N-Bereich.}
{Sie zerfallen beim Übergang.}
\end{QQuestion}

}
\end{frame}

\begin{frame}
\only<1>{
\begin{QQuestion}{AB104}{Was versteht man unter Halbleitermaterialien?}{Einige Stoffe (z. B. Silizium) sind in reinem Zustand bei Raumtemperatur gute Elektrolyten. Durch geringfügige Zusätze von geeigneten anderen Stoffen (z. B. Bismut, Tellur) kann man daraus entweder N-leitendes- oder P-leitendes Material für Anoden bzw. Kathoden von Batterien herstellen.}
{Einige Stoffe (z. B. Silizium) sind in reinem Zustand bei Raumtemperatur gute Leiter. Durch geringfügige Zusätze von geeigneten anderen Stoffen (z. B. Bor, Phosphor) oder bei hohen Temperaturen nimmt jedoch ihre Leitfähigkeit ab.}
{Einige Stoffe (z. B. Silizium) sind in reinem Zustand bei Raumtemperatur gute Leiter. Durch geringfügige Zusätze von geeigneten anderen Stoffen (z. B. Bismut, Tellur) fällt ihr Widerstand auf den halben Wert.}
{Einige Stoffe (z. B. Silizium) sind in reinem Zustand bei Raumtemperatur gute Isolatoren. Durch geringfügige Zusätze von geeigneten anderen Stoffen (z. B. Bor, Phosphor) oder bei hohen Temperaturen werden sie jedoch zu Leitern.}
\end{QQuestion}

}
\only<2>{
\begin{QQuestion}{AB104}{Was versteht man unter Halbleitermaterialien?}{Einige Stoffe (z. B. Silizium) sind in reinem Zustand bei Raumtemperatur gute Elektrolyten. Durch geringfügige Zusätze von geeigneten anderen Stoffen (z. B. Bismut, Tellur) kann man daraus entweder N-leitendes- oder P-leitendes Material für Anoden bzw. Kathoden von Batterien herstellen.}
{Einige Stoffe (z. B. Silizium) sind in reinem Zustand bei Raumtemperatur gute Leiter. Durch geringfügige Zusätze von geeigneten anderen Stoffen (z. B. Bor, Phosphor) oder bei hohen Temperaturen nimmt jedoch ihre Leitfähigkeit ab.}
{Einige Stoffe (z. B. Silizium) sind in reinem Zustand bei Raumtemperatur gute Leiter. Durch geringfügige Zusätze von geeigneten anderen Stoffen (z. B. Bismut, Tellur) fällt ihr Widerstand auf den halben Wert.}
{\textbf{\textcolor{DARCgreen}{Einige Stoffe (z. B. Silizium) sind in reinem Zustand bei Raumtemperatur gute Isolatoren. Durch geringfügige Zusätze von geeigneten anderen Stoffen (z. B. Bor, Phosphor) oder bei hohen Temperaturen werden sie jedoch zu Leitern.}}}
\end{QQuestion}

}
\end{frame}

\begin{frame}
\only<1>{
\begin{QQuestion}{AB105}{Was versteht man unter Dotierung?}{Das Einbringen von magnetischen Nord- oder Südpolen in einen Halbleitergrundstoff, um die Induktivität zu erhöhen.}
{Das Entfernen von Atomen aus dem Halbleitergrundstoff, um die elektrische Leitfähigkeit zu senken.}
{Das Einbringen von chemisch anderswertigen Fremdatomen in einen Halbleitergrundstoff, um freie Ladungsträger zur Verfügung zu stellen.}
{Das Entfernen von Verunreinigungen aus einem Halbleitergrundstoff, um Elektronen zu generieren.}
\end{QQuestion}

}
\only<2>{
\begin{QQuestion}{AB105}{Was versteht man unter Dotierung?}{Das Einbringen von magnetischen Nord- oder Südpolen in einen Halbleitergrundstoff, um die Induktivität zu erhöhen.}
{Das Entfernen von Atomen aus dem Halbleitergrundstoff, um die elektrische Leitfähigkeit zu senken.}
{\textbf{\textcolor{DARCgreen}{Das Einbringen von chemisch anderswertigen Fremdatomen in einen Halbleitergrundstoff, um freie Ladungsträger zur Verfügung zu stellen.}}}
{Das Entfernen von Verunreinigungen aus einem Halbleitergrundstoff, um Elektronen zu generieren.}
\end{QQuestion}

}
\end{frame}

\begin{frame}
\only<1>{
\begin{QQuestion}{AB106}{N-leitendes Halbleitermaterial ist gekennzeichnet durch~...}{einen Überschuss an beweglichen Elektronenlöchern.}
{ein Fehlen von Dotierungsatomen.}
{ein Fehlen von Atomen im Gitter des Halbleiterkristalls.}
{einen Überschuss an beweglichen Elektronen.}
\end{QQuestion}

}
\only<2>{
\begin{QQuestion}{AB106}{N-leitendes Halbleitermaterial ist gekennzeichnet durch~...}{einen Überschuss an beweglichen Elektronenlöchern.}
{ein Fehlen von Dotierungsatomen.}
{ein Fehlen von Atomen im Gitter des Halbleiterkristalls.}
{\textbf{\textcolor{DARCgreen}{einen Überschuss an beweglichen Elektronen.}}}
\end{QQuestion}

}
\end{frame}

\begin{frame}
\only<1>{
\begin{QQuestion}{AB107}{P-leitendes Halbleitermaterial ist gekennzeichnet durch~...}{einen Überschuss an beweglichen Elektronen.}
{ein Fehlen von Dotierungsatomen.}
{ein Fehlen von Atomen im Gitter des Halbleiterkristalls.}
{einen Überschuss an beweglichen Elektronenlöchern.}
\end{QQuestion}

}
\only<2>{
\begin{QQuestion}{AB107}{P-leitendes Halbleitermaterial ist gekennzeichnet durch~...}{einen Überschuss an beweglichen Elektronen.}
{ein Fehlen von Dotierungsatomen.}
{ein Fehlen von Atomen im Gitter des Halbleiterkristalls.}
{\textbf{\textcolor{DARCgreen}{einen Überschuss an beweglichen Elektronenlöchern.}}}
\end{QQuestion}

}
\end{frame}

\begin{frame}
\only<1>{
\begin{PQuestion}{AB108}{Das folgende Bild zeigt den prinzipiellen Aufbau einer Halbleiterdiode. Wie entsteht die Sperrschicht?}{An der Grenzschicht wandern Atome aus dem P-Teil in den N-Teil. Dadurch wird auf der P-Seite der Atommangel abgebaut, auf der N-Seite der Atommangel vergrößert. Es bildet sich auf beiden Seiten der Grenzfläche eine leitende Schicht.}
{An der Grenzschicht wandern Elektronen aus dem P-Teil in den N-Teil. Dadurch wird auf der P-Seite der Elektronenüberschuss teilweise abgebaut, auf der N-Seite der Elektronenmangel teilweise neutralisiert. Es bildet sich auf beiden Seiten der Grenzfläche eine isolierende Schicht.}
{An der Grenzschicht wandern Elektronen aus dem N-Teil in den P-Teil. Dadurch wird auf der N-Seite der Elektronenüberschuss teilweise abgebaut, auf der P-Seite der Elektronenmangel teilweise neutralisiert. Es bildet sich auf beiden Seiten der Grenzfläche eine isolierende Schicht.}
{An der Grenzschicht wandern Atome aus dem N-Teil in den P-Teil. Dadurch wird auf der N-Seite der Atommangel abgebaut, auf der P-Seite der Atommangel vergrößert. Es bildet sich auf beiden Seiten der Grenzfläche eine leitende Schicht.}
{\DARCimage{1.0\linewidth}{48include}}\end{PQuestion}

}
\only<2>{
\begin{PQuestion}{AB108}{Das folgende Bild zeigt den prinzipiellen Aufbau einer Halbleiterdiode. Wie entsteht die Sperrschicht?}{An der Grenzschicht wandern Atome aus dem P-Teil in den N-Teil. Dadurch wird auf der P-Seite der Atommangel abgebaut, auf der N-Seite der Atommangel vergrößert. Es bildet sich auf beiden Seiten der Grenzfläche eine leitende Schicht.}
{An der Grenzschicht wandern Elektronen aus dem P-Teil in den N-Teil. Dadurch wird auf der P-Seite der Elektronenüberschuss teilweise abgebaut, auf der N-Seite der Elektronenmangel teilweise neutralisiert. Es bildet sich auf beiden Seiten der Grenzfläche eine isolierende Schicht.}
{\textbf{\textcolor{DARCgreen}{An der Grenzschicht wandern Elektronen aus dem N-Teil in den P-Teil. Dadurch wird auf der N-Seite der Elektronenüberschuss teilweise abgebaut, auf der P-Seite der Elektronenmangel teilweise neutralisiert. Es bildet sich auf beiden Seiten der Grenzfläche eine isolierende Schicht.}}}
{An der Grenzschicht wandern Atome aus dem N-Teil in den P-Teil. Dadurch wird auf der N-Seite der Atommangel abgebaut, auf der P-Seite der Atommangel vergrößert. Es bildet sich auf beiden Seiten der Grenzfläche eine leitende Schicht.}
{\DARCimage{1.0\linewidth}{48include}}\end{PQuestion}

}
\end{frame}

\begin{frame}
\only<1>{
\begin{PQuestion}{AB109}{Wie verhält sich die Verarmungszone in der hier dargestellten Halbleiterdiode?}{Sie verändert sich nicht.}
{Sie verengt sich.}
{Sie erweitert sich.}
{Sie verschwindet. }
{\DARCimage{1.0\linewidth}{486include}}\end{PQuestion}

}
\only<2>{
\begin{PQuestion}{AB109}{Wie verhält sich die Verarmungszone in der hier dargestellten Halbleiterdiode?}{Sie verändert sich nicht.}
{Sie verengt sich.}
{\textbf{\textcolor{DARCgreen}{Sie erweitert sich.}}}
{Sie verschwindet. }
{\DARCimage{1.0\linewidth}{486include}}\end{PQuestion}

}
\end{frame}%ENDCONTENT


\section{Integrierte Schaltkreise}
\label{section:integrierte_schaltkreise}
\begin{frame}%STARTCONTENT

\only<1>{
\begin{QQuestion}{AC601}{Eine integrierte Schaltung ist~...}{eine miniaturisierte, aus SMD-Bauteilen aufgebaute Schaltung.}
{eine aus einzelnen Bauteilen aufgebaute vergossene Schaltung.}
{eine komplexe Schaltung auf einem Halbleitersubstrat.}
{die Zusammenschaltung einzelner Baugruppen zu einem elektronischen Gerät.}
\end{QQuestion}

}
\only<2>{
\begin{QQuestion}{AC601}{Eine integrierte Schaltung ist~...}{eine miniaturisierte, aus SMD-Bauteilen aufgebaute Schaltung.}
{eine aus einzelnen Bauteilen aufgebaute vergossene Schaltung.}
{\textbf{\textcolor{DARCgreen}{eine komplexe Schaltung auf einem Halbleitersubstrat.}}}
{die Zusammenschaltung einzelner Baugruppen zu einem elektronischen Gerät.}
\end{QQuestion}

}
\end{frame}

\begin{frame}
\frametitle{Monolithic Microwave Integrated Circuit (MMIC)}
\end{frame}

\begin{frame}
\only<1>{
\begin{QQuestion}{AC602}{Welche Bauteile sind in einem Monolithic Microwave Integrated Circuit (MMIC) enthalten?}{Ein MMIC enthält alle aktiven und passiven Bauteile auf einem Halbleiter-Substrat.}
{Ein MMIC enthält nur aktive Bauteile auf einem Halbleiter-Substrat.}
{Ein MMIC enthält nur passive Bauteile auf einem Halbleiter-Substrat.}
{Ein MMIC enthält alle aktiven und passiven Bauteile auf einer Leiterplatte.}
\end{QQuestion}

}
\only<2>{
\begin{QQuestion}{AC602}{Welche Bauteile sind in einem Monolithic Microwave Integrated Circuit (MMIC) enthalten?}{\textbf{\textcolor{DARCgreen}{Ein MMIC enthält alle aktiven und passiven Bauteile auf einem Halbleiter-Substrat.}}}
{Ein MMIC enthält nur aktive Bauteile auf einem Halbleiter-Substrat.}
{Ein MMIC enthält nur passive Bauteile auf einem Halbleiter-Substrat.}
{Ein MMIC enthält alle aktiven und passiven Bauteile auf einer Leiterplatte.}
\end{QQuestion}

}
\end{frame}

\begin{frame}
\only<1>{
\begin{QQuestion}{AC603}{Welchen Vorteil hat ein Monolithic Microwave Integrated Circuit (MMIC) gegenüber einem diskreten Transistorverstärker?}{Ein MMIC bietet einen hohen Eingangswiderstand und einen niedrigen Ausgangswiderstand.}
{Ein MMIC bietet schmalbandig eine hohe Verstärkung in einem Bauteil.}
{Ein MMIC bietet breitbandig eine hohe Verstärkung mit weniger Bauteilen.}
{Ein MMIC bietet einstellbare Eingangs- und Ausgangsimpedanz.}
\end{QQuestion}

}
\only<2>{
\begin{QQuestion}{AC603}{Welchen Vorteil hat ein Monolithic Microwave Integrated Circuit (MMIC) gegenüber einem diskreten Transistorverstärker?}{Ein MMIC bietet einen hohen Eingangswiderstand und einen niedrigen Ausgangswiderstand.}
{Ein MMIC bietet schmalbandig eine hohe Verstärkung in einem Bauteil.}
{\textbf{\textcolor{DARCgreen}{Ein MMIC bietet breitbandig eine hohe Verstärkung mit weniger Bauteilen.}}}
{Ein MMIC bietet einstellbare Eingangs- und Ausgangsimpedanz.}
\end{QQuestion}

}
\end{frame}

\begin{frame}
\only<1>{
\begin{QQuestion}{AC604}{Was ist typisch für einen Monolithic Microwave Integrated Circuit (MMIC)?}{Sie sind nur im Mikrowellenbereich einsetzbar.}
{Die Verstärkung ist bereits ab \qty{0}{\Hz} konstant.}
{Ein- und Ausgangsimpedanz entsprechen üblichen Leitungsimpedanzen (z. B. 50 Ohm).}
{Der Verstärkungsbereich ist schmalbandig.}
\end{QQuestion}

}
\only<2>{
\begin{QQuestion}{AC604}{Was ist typisch für einen Monolithic Microwave Integrated Circuit (MMIC)?}{Sie sind nur im Mikrowellenbereich einsetzbar.}
{Die Verstärkung ist bereits ab \qty{0}{\Hz} konstant.}
{\textbf{\textcolor{DARCgreen}{Ein- und Ausgangsimpedanz entsprechen üblichen Leitungsimpedanzen (z. B. 50 Ohm).}}}
{Der Verstärkungsbereich ist schmalbandig.}
\end{QQuestion}

}
\end{frame}

\begin{frame}
\only<1>{
\begin{PQuestion}{AF425}{Der optimale Arbeitspunkt des dargestellten MMIC ist mit \qty{4}{\volt} und \qty{10}{\mA} angegeben. Die Betriebsspannung beträgt \qty{13,5}{\volt}. Berechnen Sie den Vorwiderstand ($R_\text{BIAS}$).}{\qty{95}{\ohm}}
{\qty{1350}{\ohm}}
{\qty{950}{\ohm}}
{\qty{400}{\ohm}}
{\DARCimage{1.0\linewidth}{773include}}\end{PQuestion}

}
\only<2>{
\begin{PQuestion}{AF425}{Der optimale Arbeitspunkt des dargestellten MMIC ist mit \qty{4}{\volt} und \qty{10}{\mA} angegeben. Die Betriebsspannung beträgt \qty{13,5}{\volt}. Berechnen Sie den Vorwiderstand ($R_\text{BIAS}$).}{\qty{95}{\ohm}}
{\qty{1350}{\ohm}}
{\textbf{\textcolor{DARCgreen}{\qty{950}{\ohm}}}}
{\qty{400}{\ohm}}
{\DARCimage{1.0\linewidth}{773include}}\end{PQuestion}

}
\end{frame}

\begin{frame}
\frametitle{Lösungsweg}
\begin{itemize}
  \item gegeben: $U_{\textrm{D}} = 4V$
  \item gegeben: $U_{\textrm{CC}} = 13,5V$
  \item gegeben: $I_{\textrm{D}} = 10mA$
  \item gesucht: $R_{\textrm{BIAS}}$
  \end{itemize}
    \pause
    $R_{\textrm{BIAS}} = \frac{U_{\textrm{CC}} -- U_{\textrm{D}}}{I_{\textrm{D}}} = \frac{13,5V -4V}{10mA} = 950\Omega$



\end{frame}

\begin{frame}
\only<1>{
\begin{PQuestion}{AF426}{Berechnen Sie $R_\text{BIAS}$ für die dargestellte MMIC-Schaltung und wählen Sie den nächsten Normwert. $U_\text{CC}$~=~\qty{13,8}{\V}; $U_\text{D}$~=~\qty{4}{\V}; $I_\text{D}$~=~\qty{15}{\mA}}{\qty{820}{\ohm}}
{\qty{680}{\ohm}}
{\qty{270}{\ohm}}
{\qty{560}{\ohm}}
{\DARCimage{1.0\linewidth}{773include}}\end{PQuestion}

}
\only<2>{
\begin{PQuestion}{AF426}{Berechnen Sie $R_\text{BIAS}$ für die dargestellte MMIC-Schaltung und wählen Sie den nächsten Normwert. $U_\text{CC}$~=~\qty{13,8}{\V}; $U_\text{D}$~=~\qty{4}{\V}; $I_\text{D}$~=~\qty{15}{\mA}}{\qty{820}{\ohm}}
{\textbf{\textcolor{DARCgreen}{\qty{680}{\ohm}}}}
{\qty{270}{\ohm}}
{\qty{560}{\ohm}}
{\DARCimage{1.0\linewidth}{773include}}\end{PQuestion}

}
\end{frame}

\begin{frame}
\frametitle{Lösungsweg}
\begin{itemize}
  \item gegeben: $U_{\textrm{D}} = 4V$
  \item gegeben: $U_{\textrm{CC}} = 13,8V$
  \item gegeben: $I_{\textrm{D}} = 15mA$
  \item gesucht: $R_{\textrm{BIAS}}$
  \end{itemize}
    \pause
    $R_{\textrm{BIAS}} = \frac{U_{\textrm{CC}} -- U_{\textrm{D}}}{I_{\textrm{D}}} = \frac{13,8V -4V}{15mA} = 653,3\Omega \rightarrow 680\Omega$



\end{frame}

\begin{frame}
\only<1>{
\begin{PQuestion}{AF427}{Wieviel Wärmeleistung wird im MMIC in Wärme umgesetzt, wenn die Betriebsspannung \qty{9}{\V} beträgt und $R_\text{BIAS}$ einen Wert von \qty{470}{\ohm} hat?}{\qty{90}{\mW}}
{\qty{47}{\mW}}
{\qty{43}{\mW}}
{\qty{52}{\mW}}
{\DARCimage{1.0\linewidth}{773include}}\end{PQuestion}

}
\only<2>{
\begin{PQuestion}{AF427}{Wieviel Wärmeleistung wird im MMIC in Wärme umgesetzt, wenn die Betriebsspannung \qty{9}{\V} beträgt und $R_\text{BIAS}$ einen Wert von \qty{470}{\ohm} hat?}{\qty{90}{\mW}}
{\qty{47}{\mW}}
{\textbf{\textcolor{DARCgreen}{\qty{43}{\mW}}}}
{\qty{52}{\mW}}
{\DARCimage{1.0\linewidth}{773include}}\end{PQuestion}

}
\end{frame}

\begin{frame}
\frametitle{Lösungsweg}
\begin{itemize}
  \item gegeben: $U = 9V$
  \item gegeben: $R_{\textrm{BIAS}} = 470\Omega$
  \item gegeben: $U_{\textrm{D}} = 4V$
  \item gesucht: $P$
  \item Ansatz: Strom durch $R_{\textrm{BIAS}}$ ist überall gleich, weil kein anderer ohmschmer Verbraucher in der Schaltung vorhanden ist
  \end{itemize}
    \pause
    $I_{\textrm{D}} = \frac{U_{\textrm{BIAS}}}{R_{\textrm{BIAS}}} = \frac{U-U_{\textrm{D}}}{R_{\textrm{BIAS}}} = \frac{9V-4V}{470\Omega} = 10,64mA$
    \pause
    $P = U_{\textrm{D}} \cdot I_{\textrm{D}} = 4V \cdot 10,64mA \approx 43mW$



\end{frame}%ENDCONTENT


\title{DARC Amateurfunklehrgang Klasse A}
\author{Reihen- und Parallelschaltung von Bauelementen}
\institute{Deutscher Amateur Radio Club e.\,V.}
\begin{frame}
\maketitle
\end{frame}

\section{Widerstandsnetzwerke II}
\label{section:reihe_parallel_widerstandsnetz_2}
\begin{frame}%STARTCONTENT

\only<1>{
\begin{PQuestion}{AD106}{Wie groß ist die Spannung $U$, wenn durch $R_3$ ein Strom von \qty{1}{\mA} fließt und alle Widerstände $R_1$ bis $R_3$ je \qty{10}{\kohm} betragen? }{\qty{30}{\V}}
{\qty{20}{\V}}
{\qty{15}{\V}}
{\qty{40}{\V}}
{\DARCimage{1.0\linewidth}{398include}}\end{PQuestion}

}
\only<2>{
\begin{PQuestion}{AD106}{Wie groß ist die Spannung $U$, wenn durch $R_3$ ein Strom von \qty{1}{\mA} fließt und alle Widerstände $R_1$ bis $R_3$ je \qty{10}{\kohm} betragen? }{\textbf{\textcolor{DARCgreen}{\qty{30}{\V}}}}
{\qty{20}{\V}}
{\qty{15}{\V}}
{\qty{40}{\V}}
{\DARCimage{1.0\linewidth}{398include}}\end{PQuestion}

}
\end{frame}

\begin{frame}
\frametitle{Lösungsweg}
\begin{itemize}
  \item gegeben: $R_1 = R_2 = R_3 = 10kΩ$
  \item gegeben: $I_3 = I_2 = 1mA$
  \item gesucht: $U$
  \end{itemize}
    \pause
    $R_{ges} = R_1 + \frac{R_1 \cdot R_2}{R_1 + R_2} = 10kΩ + \frac{10kΩ \cdot 10kΩ}{10kΩ + 10kΩ} = 15kΩ$
    \pause
    $I_{ges} = I_2 + I_3 = 1mA + 1mA = 2mA$
    \pause
    $U = R_{ges} \cdot I_{ges} = 15kΩ \cdot 2mA = 30V$



\end{frame}

\begin{frame}
\only<1>{
\begin{PQuestion}{AD107}{Wie groß ist der Strom durch $R_3$, wenn $U$~=~\qty{15}{\V} und alle Widerstände $R_1$ bis $R_3$ je \qty{10}{\kohm} betragen?}{\qty{4,5}{\mA}}
{\qty{1,0}{\mA}}
{\qty{1,6}{\mA}}
{\qty{0,5}{\mA}}
{\DARCimage{1.0\linewidth}{398include}}\end{PQuestion}

}
\only<2>{
\begin{PQuestion}{AD107}{Wie groß ist der Strom durch $R_3$, wenn $U$~=~\qty{15}{\V} und alle Widerstände $R_1$ bis $R_3$ je \qty{10}{\kohm} betragen?}{\qty{4,5}{\mA}}
{\qty{1,0}{\mA}}
{\qty{1,6}{\mA}}
{\textbf{\textcolor{DARCgreen}{\qty{0,5}{\mA}}}}
{\DARCimage{1.0\linewidth}{398include}}\end{PQuestion}

}
\end{frame}

\begin{frame}
\frametitle{Lösungsweg}
\begin{itemize}
  \item gegeben: $R_1 = R_2 = R_3 = 10kΩ$
  \item gegeben: $U=15V$
  \item gesucht: $I_3$
  \end{itemize}
    \pause
    $R_{ges} = R_1 + \frac{R_1 \cdot R_2}{R_1 + R_2} = 10kΩ + \frac{10kΩ \cdot 10kΩ}{10kΩ + 10kΩ} = 15kΩ$
    \pause
    $\frac{U_3}{U} = \frac{R_{2\parallel 3}}{R_{ges}} \Rightarrow U_3 = \frac{R_{2\parallel 3}}{R_{ges}} \cdot U = \frac{5kΩ}{15kΩ} \cdot 15V = 5V$
    \pause
    $I_3 = \frac{U_3}{R_3} = \frac{5V}{10kΩ} = 0,5mA$



\end{frame}

\begin{frame}
\only<1>{
\begin{PQuestion}{AD108}{Welche Leistung tritt in $R_2$ auf, wenn $U$~=~\qty{15}{\V} und alle Widerstände $R_1$ bis $R_3$ je \qty{10}{\kohm} betragen? }{\qty{1,5}{\mW}}
{\qty{5,0}{\mW}}
{\qty{2,5}{\mW}}
{\qty{0,15}{\W}}
{\DARCimage{1.0\linewidth}{398include}}\end{PQuestion}

}
\only<2>{
\begin{PQuestion}{AD108}{Welche Leistung tritt in $R_2$ auf, wenn $U$~=~\qty{15}{\V} und alle Widerstände $R_1$ bis $R_3$ je \qty{10}{\kohm} betragen? }{\qty{1,5}{\mW}}
{\qty{5,0}{\mW}}
{\textbf{\textcolor{DARCgreen}{\qty{2,5}{\mW}}}}
{\qty{0,15}{\W}}
{\DARCimage{1.0\linewidth}{398include}}\end{PQuestion}

}
\end{frame}

\begin{frame}
\frametitle{Lösungsweg}
\begin{itemize}
  \item gegeben: $R_1 = R_2 = R_3 = 10kΩ$
  \item gegeben: $U=15V$
  \item gesucht: $P_2$
  \end{itemize}
    \pause
    $\frac{U_2}{U} = \frac{R_{2\parallel 3}}{R_{ges}} \Rightarrow U_2 = \frac{R_{2\parallel 3}}{R_{ges}} \cdot U = \frac{5kΩ}{15kΩ} \cdot 15V = 5V$
    \pause
    $P_2 = \frac{U_2^2}{R_2} = \frac{(5V)^2}{10kΩ} = 2,5mW$



\end{frame}

\begin{frame}
\only<1>{
\begin{PQuestion}{AD109}{In welchem Bereich liegt der Eingangswiderstand der folgenden Schaltung, wenn $R$ alle Werte von \qty{0}{\ohm} bis \qty{1}{\kohm} annehmen kann? }{\qtyrange{300}{367}{\ohm}}
{\qtyrange{300}{500}{\ohm}}
{\qtyrange{292}{367}{\ohm}}
{\qtyrange{267}{292}{\ohm}}
{\DARCimage{1.0\linewidth}{384include}}\end{PQuestion}

}
\only<2>{
\begin{PQuestion}{AD109}{In welchem Bereich liegt der Eingangswiderstand der folgenden Schaltung, wenn $R$ alle Werte von \qty{0}{\ohm} bis \qty{1}{\kohm} annehmen kann? }{\qtyrange{300}{367}{\ohm}}
{\qtyrange{300}{500}{\ohm}}
{\qtyrange{292}{367}{\ohm}}
{\textbf{\textcolor{DARCgreen}{\qtyrange{267}{292}{\ohm}}}}
{\DARCimage{1.0\linewidth}{384include}}\end{PQuestion}

}
\end{frame}

\begin{frame}
\frametitle{Lösungsweg}
\begin{columns}
    \begin{column}{0.48\textwidth}
    \begin{itemize}
  \item gegeben: $R = 0\dots 1kΩ$
  \item gegeben: $R_1 = 200Ω$
  \end{itemize}

    \end{column}
   \begin{column}{0.48\textwidth}
       \begin{itemize}
  \item gegeben: $R_2 = 100Ω$
  \item gegeben: $R_3 = 200Ω$
  \end{itemize}

   \end{column}
\end{columns}
    \pause
    $R_{ges} = R_1 + \frac{R_2 \cdot (R_3 + R)}{R_2 + (R_3 + R)}$
    \pause
    Bei $R = 0Ω$:

$R_{ges} = 200Ω + \frac{100Ω \cdot (200Ω + 0Ω)}{100Ω + 200Ω +0Ω} \approx 267Ω$
    \pause
    Bei $R = 1kΩ$:

$R_{ges} = 200Ω + \frac{100Ω \cdot (200Ω + 1kΩ)}{100Ω + 200Ω +1kΩ} \approx 292Ω$



\end{frame}

\begin{frame}
\only<1>{
\begin{PQuestion}{AD110}{Wenn $\textrm{R}_1$ und $\textrm{R}_3$ je \qty{2,2}{\kohm} haben und $\textrm{R}_2$ und $\textrm{R}_4$ je \qty{220}{\ohm} betragen, hat die Schaltung zwischen den Punkten a und b einen Gesamtwiderstand von~...}{\qty{1540}{\ohm}.}
{\qty{1210}{\ohm}.}
{\qty{4840}{\ohm}.}
{\qty{2420}{\ohm}.}
{\DARCimage{1.0\linewidth}{344include}}\end{PQuestion}

}
\only<2>{
\begin{PQuestion}{AD110}{Wenn $\textrm{R}_1$ und $\textrm{R}_3$ je \qty{2,2}{\kohm} haben und $\textrm{R}_2$ und $\textrm{R}_4$ je \qty{220}{\ohm} betragen, hat die Schaltung zwischen den Punkten a und b einen Gesamtwiderstand von~...}{\qty{1540}{\ohm}.}
{\textbf{\textcolor{DARCgreen}{\qty{1210}{\ohm}.}}}
{\qty{4840}{\ohm}.}
{\qty{2420}{\ohm}.}
{\DARCimage{1.0\linewidth}{344include}}\end{PQuestion}

}
\end{frame}

\begin{frame}
\frametitle{Lösungsweg}
\begin{itemize}
  \item gegeben: $R_1 = R_3 = 2,2kΩ$
  \item gegeben: $R_2 = R_4 = 220Ω$
  \item gesucht: $R_{ges}$
  \end{itemize}
    \pause
    $R_{ges} = \frac{(R_1 + R_2) \cdot (R_3 + R_4)}{(R_1 + R_2) + (R_3 + R_4)} = \frac{(2,2kΩ + 220Ω) \cdot (2,2kΩ + 220Ω)}{2,2kΩ + 220Ω + 2,2kΩ + 220Ω} = 1210Ω$



\end{frame}

\begin{frame}
\only<1>{
\begin{PQuestion}{AD114}{Wie groß ist die Spannung $U_2$ in der Schaltung mit folgenden Werten: $U_{\symup{B}}~=~\qty{12}{\V}$, $R_1~=~10~k\Omega$, $R_2~=~2,2~k\Omega$, $R_{\symup{L}}~=~8,2~k\Omega$}{\qty{2,2}{\V}}
{\qty{1,8}{\V}}
{\qty{5,4}{\V}}
{\qty{8,2}{\V}}
{\DARCimage{0.75\linewidth}{199include}}\end{PQuestion}

}
\only<2>{
\begin{PQuestion}{AD114}{Wie groß ist die Spannung $U_2$ in der Schaltung mit folgenden Werten: $U_{\symup{B}}~=~\qty{12}{\V}$, $R_1~=~10~k\Omega$, $R_2~=~2,2~k\Omega$, $R_{\symup{L}}~=~8,2~k\Omega$}{\qty{2,2}{\V}}
{\textbf{\textcolor{DARCgreen}{\qty{1,8}{\V}}}}
{\qty{5,4}{\V}}
{\qty{8,2}{\V}}
{\DARCimage{0.75\linewidth}{199include}}\end{PQuestion}

}
\end{frame}

\begin{frame}
\frametitle{Lösungsweg}
\begin{columns}
    \begin{column}{0.48\textwidth}
    \begin{itemize}
  \item gegeben: $R_1 = 10kΩ$
  \item gegeben: $R_2 = 2,2kΩ$
  \item gegeben: $R_L = 8,2kΩ$
  \end{itemize}

    \end{column}
   \begin{column}{0.48\textwidth}
       \begin{itemize}
  \item gegeben: $U_B = 12V$
  \item gesucht: $U_2$
  \end{itemize}

   \end{column}
\end{columns}
    \pause
    $\frac{U_2}{U_B} = \frac{R_{2\parallel L}}{R_{ges}}$

$R_{2\parallel L} = \frac{R_2 \cdot R_L}{R_2 + R_L} = \frac{2,2kΩ \cdot 8,2kΩ}{2,2kΩ + 8,2kΩ} = 1,74kΩ$

$R_{ges} = R_1 + R_{2\parallel L} = 10kΩ + 1,74kΩ = 11,74kΩ$
    \pause
    $U_2 = \frac{R_{2\parallel L}}{R_{ges}} \cdot U_B = \frac{1,74kΩ}{11,74kΩ} \cdot 12V \approx 1,8V$



\end{frame}%ENDCONTENT


\section{Spannungsteiler II}
\label{section:spannungsteiler_2}
\begin{frame}%STARTCONTENT

\only<1>{
\begin{PQuestion}{AD115}{Wenn der dargestellte Spannungsteiler mit $R_{\symup{L}}$ belastet wird, dann ergibt sich folgender Zusammenhang:}{$I_1$ sinkt, $R_2$ setzt mehr Leistung in Wärme um.}
{$I_1$ steigt, $R_2$ setzt mehr Leistung in Wärme um.}
{$I_1$ steigt, $R_1$ setzt mehr Leistung in Wärme um.}
{$I_2$ steigt, $R_1$ setzt weniger Leistung in Wärme um.}
{\DARCimage{0.75\linewidth}{199include}}\end{PQuestion}

}
\only<2>{
\begin{PQuestion}{AD115}{Wenn der dargestellte Spannungsteiler mit $R_{\symup{L}}$ belastet wird, dann ergibt sich folgender Zusammenhang:}{$I_1$ sinkt, $R_2$ setzt mehr Leistung in Wärme um.}
{$I_1$ steigt, $R_2$ setzt mehr Leistung in Wärme um.}
{\textbf{\textcolor{DARCgreen}{$I_1$ steigt, $R_1$ setzt mehr Leistung in Wärme um.}}}
{$I_2$ steigt, $R_1$ setzt weniger Leistung in Wärme um.}
{\DARCimage{0.75\linewidth}{199include}}\end{PQuestion}

}
\end{frame}

\begin{frame}\end{frame}%ENDCONTENT


\section{Brückenschaltung}
\label{section:brueckenschaltung}
\begin{frame}%STARTCONTENT

\only<1>{
\begin{PQuestion}{AD111}{In welchem Verhältnis müssen die Widerstände $R_1$ bis $R_4$ zueinander stehen, damit das Messinstrument im Brückenzweig keine Spannung anzeigt?}{$\dfrac{R_1}{R_4} = \dfrac{R_2}{R_3}$}
{$\dfrac{R_1}{R_2} = \dfrac{R_4}{R_3}$}
{$\dfrac{R_2}{R_1} = \dfrac{R_3}{R_4}$}
{$\dfrac{R_1}{R_2} = \dfrac{R_3}{R_4}$}
{\DARCimage{1.0\linewidth}{343include}}\end{PQuestion}

}
\only<2>{
\begin{PQuestion}{AD111}{In welchem Verhältnis müssen die Widerstände $R_1$ bis $R_4$ zueinander stehen, damit das Messinstrument im Brückenzweig keine Spannung anzeigt?}{$\dfrac{R_1}{R_4} = \dfrac{R_2}{R_3}$}
{$\dfrac{R_1}{R_2} = \dfrac{R_4}{R_3}$}
{$\dfrac{R_2}{R_1} = \dfrac{R_3}{R_4}$}
{\textbf{\textcolor{DARCgreen}{$\dfrac{R_1}{R_2} = \dfrac{R_3}{R_4}$}}}
{\DARCimage{1.0\linewidth}{343include}}\end{PQuestion}

}
\end{frame}

\begin{frame}
\only<1>{
\begin{PQuestion}{AD112}{Die Spannung an der Brückenschaltung beträgt \qty{10}{\V}. Alle Widerstände haben einen Wert von \qty{50}{\ohm}. Wie groß ist die Spannung zwischen A und B im Brückenzweig (gemessen von A nach B)?}{\qty{2,5}{\V}}
{\qty{-5}{\V}}
{\qty{5}{\V}}
{\qty{0}{\V}}
{\DARCimage{1.0\linewidth}{343include}}\end{PQuestion}

}
\only<2>{
\begin{PQuestion}{AD112}{Die Spannung an der Brückenschaltung beträgt \qty{10}{\V}. Alle Widerstände haben einen Wert von \qty{50}{\ohm}. Wie groß ist die Spannung zwischen A und B im Brückenzweig (gemessen von A nach B)?}{\qty{2,5}{\V}}
{\qty{-5}{\V}}
{\qty{5}{\V}}
{\textbf{\textcolor{DARCgreen}{\qty{0}{\V}}}}
{\DARCimage{1.0\linewidth}{343include}}\end{PQuestion}

}
\end{frame}

\begin{frame}
\only<1>{
\begin{PQuestion}{AD113}{Die Spannung an der Brückenschaltung beträgt \qty{11}{\V}. Die Widerstände haben folgende Werte: $R_1$ = \qty{1}{\kohm}; $R_2$ = \qty{10}{\kohm}; $R_3$ = \qty{10}{\kohm}; $R_4$ = \qty{1}{\kohm}. Wie groß ist die Spannung zwischen A und B im Brückenzweig (gemessen von A nach B)?}{$U_{AB} = \qty{-9}{\V}$}
{$U_{AB} = \qty{9}{\V}$ }
{$U_{AB} = \qty{10}{\V}$}
{$U_{AB} = \qty{-10}{\V}$}
{\DARCimage{1.0\linewidth}{343include}}\end{PQuestion}

}
\only<2>{
\begin{PQuestion}{AD113}{Die Spannung an der Brückenschaltung beträgt \qty{11}{\V}. Die Widerstände haben folgende Werte: $R_1$ = \qty{1}{\kohm}; $R_2$ = \qty{10}{\kohm}; $R_3$ = \qty{10}{\kohm}; $R_4$ = \qty{1}{\kohm}. Wie groß ist die Spannung zwischen A und B im Brückenzweig (gemessen von A nach B)?}{$U_{AB} = \qty{-9}{\V}$}
{\textbf{\textcolor{DARCgreen}{$U_{AB} = \qty{9}{\V}$ }}}
{$U_{AB} = \qty{10}{\V}$}
{$U_{AB} = \qty{-10}{\V}$}
{\DARCimage{1.0\linewidth}{343include}}\end{PQuestion}

}
\end{frame}

\begin{frame}
\frametitle{Lösungsweg}
\begin{itemize}
  \item gegeben: $R_1 = R_4 = 1kΩ$
  \item gegeben: $R_2 = R_3 = 10kΩ$
  \item gegeben: $U = 11V$
  \item gesucht: $U_{AB}$
  \end{itemize}
    \pause
    $\frac{U_A}{U} = \frac{R_1}{R_1 + R_2} \Rightarrow U_A = \frac{R_1}{R_1 + R_2} \cdot U = \frac{1kΩ}{1kΩ + 10kΩ} \cdot 11V = 1V$
    \pause
    $\frac{U_B}{U} = \frac{R_3}{R_3 + R_4} \Rightarrow U_B = \frac{R_3}{R_3 + R_4} \cdot U = \frac{10kΩ}{10kΩ + 1kΩ} \cdot 11V = 10V$
    \pause
    $U_{AB} = |U_A -- U_B| = |1V -- 10V| = 9V$



\end{frame}%ENDCONTENT


\section{Spule in Reihenschaltung}
\label{section:reihenschaltung_spule}
\begin{frame}%STARTCONTENT

\only<1>{
\begin{QQuestion}{AD102}{Wie groß ist die Gesamtinduktivität von drei in Reihe geschalteten Spulen von \qty{2200}{\nano\H}, \qty{0,033}{\mH} und \qty{150}{\micro\H}?}{\qty{205,0}{\nano\H}}
{\qty{155,5}{\micro\H}}
{\qty{185,2}{\micro\H}}
{\qty{205,0}{\micro\H}}
\end{QQuestion}

}
\only<2>{
\begin{QQuestion}{AD102}{Wie groß ist die Gesamtinduktivität von drei in Reihe geschalteten Spulen von \qty{2200}{\nano\H}, \qty{0,033}{\mH} und \qty{150}{\micro\H}?}{\qty{205,0}{\nano\H}}
{\qty{155,5}{\micro\H}}
{\textbf{\textcolor{DARCgreen}{\qty{185,2}{\micro\H}}}}
{\qty{205,0}{\micro\H}}
\end{QQuestion}

}
\end{frame}

\begin{frame}
\frametitle{Lösungsweg}
$L_{ges} = 2200nH + 0,033mH + 150µH = 185,2µH$

\end{frame}%ENDCONTENT


\section{Reihen- und Parallelschaltung gemischter Bauelemente}
\label{section:reihe_parallel_gemischt}
\begin{frame}%STARTCONTENT

\only<1>{
\begin{QQuestion}{AD101}{Wie groß ist die Gesamtkapazität, wenn drei Kondensatoren $C_1$ = \qty{0,10}{\nF}, $C_2$ = \qty{47}{\pF} und $C_3$ = \qty{22}{\pF} in Reihe geschaltet werden?}{\qty{0,13}{\nF}}
{\qty{13,0}{\pF}}
{\qty{169}{\pF}}
{\qty{16,9}{\pF}}
\end{QQuestion}

}
\only<2>{
\begin{QQuestion}{AD101}{Wie groß ist die Gesamtkapazität, wenn drei Kondensatoren $C_1$ = \qty{0,10}{\nF}, $C_2$ = \qty{47}{\pF} und $C_3$ = \qty{22}{\pF} in Reihe geschaltet werden?}{\qty{0,13}{\nF}}
{\textbf{\textcolor{DARCgreen}{\qty{13,0}{\pF}}}}
{\qty{169}{\pF}}
{\qty{16,9}{\pF}}
\end{QQuestion}

}
\end{frame}

\begin{frame}
\frametitle{Lösungsweg}
\begin{itemize}
  \item gegeben: $C_1 = 0,10nF$
  \item gegeben: $C_2 = 47pF$
  \item gegeben: $C_3 = 22pF$
  \item gesucht: $C_{\textrm{ges}}$
  \end{itemize}
    \pause
    \begin{equation}\begin{align}\nonumber \tfrac{1}{C_{\textrm{ges}}} &= \tfrac{1}{C_1} + \tfrac{1}{C_2} + \tfrac{1}{C_3} = \tfrac{1}{0,10nF} + \tfrac{1}{47pF} + \tfrac{1}{22pF}\\ \nonumber &= 7,67\cdot 10^{10} \tfrac{1}{F}\\ \nonumber \Rightarrow C_{\textrm{ges}} &= \frac{1}{7,67\cdot 10^{10} \frac{1}{F}} \approx 13,0pF \end{align}\end{equation}



\end{frame}

\begin{frame}
\only<1>{
\begin{PQuestion}{AD103}{Wie groß ist die Gesamtkapazität dieser Schaltung, wenn $C_1$ = \qty{0,1}{\nF}, $C_2$ = \qty{1,5}{\nF}, $C_3$ = \qty{220}{\pF} und die Eigenkapazität der Spule \qty{1}{\pF} beträgt?}{\qty{66}{\pF}}
{\qty{1821}{\pF}}
{\qty{1,6}{\nF}}
{\qty{1}{\pF}}
{\DARCimage{1.0\linewidth}{776include}}\end{PQuestion}

}
\only<2>{
\begin{PQuestion}{AD103}{Wie groß ist die Gesamtkapazität dieser Schaltung, wenn $C_1$ = \qty{0,1}{\nF}, $C_2$ = \qty{1,5}{\nF}, $C_3$ = \qty{220}{\pF} und die Eigenkapazität der Spule \qty{1}{\pF} beträgt?}{\qty{66}{\pF}}
{\textbf{\textcolor{DARCgreen}{\qty{1821}{\pF}}}}
{\qty{1,6}{\nF}}
{\qty{1}{\pF}}
{\DARCimage{1.0\linewidth}{776include}}\end{PQuestion}

}
\end{frame}

\begin{frame}
\frametitle{Lösungsweg}
\begin{itemize}
  \item gegeben: $C_1 = 0,1nF$
  \item gegeben: $C_2 = 1,5nF$
  \item gegeben: $C_3 = 220pF$
  \item gegeben: $C_{\textrm{L}} = 1pF$
  \item gesucht: $C_{\textrm{ges}}$
  \end{itemize}
    \pause
    \begin{equation}\begin{split}\nonumber C_{\textrm{ges}} &= C_1 + C_2 + C_3 + C_{\textrm{L}}\\ &= 0,1nF + 1,5nF + 220pF + 1pF\\ &= 1821pF \end{split}\end{equation}



\end{frame}

\begin{frame}
\only<1>{
\begin{QQuestion}{AD105}{Berechne den Betrag des Scheinwiderstands $Z$ für eine Reihenschaltung aus $R$~=~\qty{100}{\ohm} und $L$~=~\qty{100}{\micro\H} bei \qty{1}{\MHz}.}{$|Z|$~=~\qty{636}{\ohm}}
{$|Z|$~=~\qty{628}{\ohm}}
{$|Z|$~=~\qty{188}{\ohm}}
{$|Z|$~=~\qty{259}{\ohm}}
\end{QQuestion}

}
\only<2>{
\begin{QQuestion}{AD105}{Berechne den Betrag des Scheinwiderstands $Z$ für eine Reihenschaltung aus $R$~=~\qty{100}{\ohm} und $L$~=~\qty{100}{\micro\H} bei \qty{1}{\MHz}.}{\textbf{\textcolor{DARCgreen}{$|Z|$~=~\qty{636}{\ohm}}}}
{$|Z|$~=~\qty{628}{\ohm}}
{$|Z|$~=~\qty{188}{\ohm}}
{$|Z|$~=~\qty{259}{\ohm}}
\end{QQuestion}

}
\end{frame}

\begin{frame}
\frametitle{Lösungsweg}
\begin{itemize}
  \item gegeben: $R = 100\Omega$
  \item gegeben: $L = 100\mu H$
  \item gegeben: $f = 1MHz$
  \item gesucht: $Z$
  \end{itemize}
    \pause
    $X_{\textrm{L}} = \omega \cdot L = 2 \cdot \pi \cdot f \cdot L = 2 \cdot \pi \cdot 1MHz \cdot 100\mu H = 628\Omega$
    \pause
    $Z = \sqrt{R^2 + X^2} = \sqrt{(100\Omega)^2 + (628\Omega)^2} \approx 636\Omega$



\end{frame}

\begin{frame}
\only<1>{
\begin{QQuestion}{AD104}{Berechne den Betrag des Scheinwiderstands $Z$ für eine Reihenschaltung aus $R$~=~\qty{100}{\ohm} und $C$~=~\qty{1}{\nF} bei \qty{1}{\MHz}.}{$|Z|$~=~\qty{188}{\ohm}}
{$|Z|$~=~\qty{159}{\ohm}}
{$|Z|$~=~\qty{636}{\ohm}}
{$|Z|$~=~\qty{259}{\ohm}}
\end{QQuestion}

}
\only<2>{
\begin{QQuestion}{AD104}{Berechne den Betrag des Scheinwiderstands $Z$ für eine Reihenschaltung aus $R$~=~\qty{100}{\ohm} und $C$~=~\qty{1}{\nF} bei \qty{1}{\MHz}.}{\textbf{\textcolor{DARCgreen}{$|Z|$~=~\qty{188}{\ohm}}}}
{$|Z|$~=~\qty{159}{\ohm}}
{$|Z|$~=~\qty{636}{\ohm}}
{$|Z|$~=~\qty{259}{\ohm}}
\end{QQuestion}

}
\end{frame}

\begin{frame}
\frametitle{Lösungsweg}
\begin{itemize}
  \item gegeben: $R = 100\Omega$
  \item gegeben: $C = 100nF$
  \item gegeben: $f = 1MHz$
  \item gesucht: $Z$
  \end{itemize}
    \pause
    $X_{\textrm{C}} = \frac{1}{\omega \cdot C} = \frac{1}{2 \cdot \pi \cdot f \cdot C} = \frac{1}{2 \cdot \pi \cdot 1MHz \cdot 100nF} = 159\Omega$
    \pause
    $Z = \sqrt{R^2 + X^2} = \sqrt{(100\Omega)^2 + (159\Omega)^2} \approx 188\Omega$



\end{frame}%ENDCONTENT


\title{DARC Amateurfunklehrgang Klasse A}
\author{Strom- und Spannungsversorgung}
\institute{Deutscher Amateur Radio Club e.\,V.}
\begin{frame}
\maketitle
\end{frame}

\section{Stromquellen}
\label{section:stromquelle}
\begin{frame}%STARTCONTENT
\end{frame}%ENDCONTENT


\section{Innenwiderstand}
\label{section:innenwiderstand}
\begin{frame}%STARTCONTENT

\only<1>{
\begin{QQuestion}{AB201}{Welche Eigenschaften sollten Strom- und Spannungsquellen nach Möglichkeit aufweisen?}{Stromquellen sollten einen möglichst hohen Innenwiderstand und Spannungsquellen einen möglichst niedrigen Innenwiderstand haben.}
{Strom- und Spannungsquellen sollten einen möglichst niedrigen Innenwiderstand haben.}
{Strom- und Spannungsquellen sollten einen möglichst hohen Innenwiderstand haben.}
{Stromquellen sollten einen möglichst niedrigen Innenwiderstand und Spannungsquellen einen möglichst hohen Innenwiderstand haben.}
\end{QQuestion}

}
\only<2>{
\begin{QQuestion}{AB201}{Welche Eigenschaften sollten Strom- und Spannungsquellen nach Möglichkeit aufweisen?}{\textbf{\textcolor{DARCgreen}{Stromquellen sollten einen möglichst hohen Innenwiderstand und Spannungsquellen einen möglichst niedrigen Innenwiderstand haben.}}}
{Strom- und Spannungsquellen sollten einen möglichst niedrigen Innenwiderstand haben.}
{Strom- und Spannungsquellen sollten einen möglichst hohen Innenwiderstand haben.}
{Stromquellen sollten einen möglichst niedrigen Innenwiderstand und Spannungsquellen einen möglichst hohen Innenwiderstand haben.}
\end{QQuestion}

}
\end{frame}

\begin{frame}
\only<1>{
\begin{QQuestion}{AG401}{Welche Lastimpedanz ist für eine Leistungsanpassung erforderlich, wenn die Signalquelle eine Ausgangsimpedanz von \qty{50}{\ohm} hat? }{\qty{50}{\ohm}}
{1/\qty{50}{\ohm}}
{\qty{100}{\ohm}}
{\qty{200}{\ohm}}
\end{QQuestion}

}
\only<2>{
\begin{QQuestion}{AG401}{Welche Lastimpedanz ist für eine Leistungsanpassung erforderlich, wenn die Signalquelle eine Ausgangsimpedanz von \qty{50}{\ohm} hat? }{\textbf{\textcolor{DARCgreen}{\qty{50}{\ohm}}}}
{1/\qty{50}{\ohm}}
{\qty{100}{\ohm}}
{\qty{200}{\ohm}}
\end{QQuestion}

}
\end{frame}

\begin{frame}
\only<1>{
\begin{QQuestion}{AB202}{In welchem Zusammenhang müssen der Innenwiderstand $R_\textrm{i}$ einer Strom- oder Spannungsquelle und ein direkt daran angeschlossener Lastwiderstand $R_\textrm{L}$ stehen, damit Leistungsanpassung vorliegt?}{$R_\textrm{L} \ll R_\textrm{i}$}
{$R_\textrm{L} \gg R_\textrm{i}$}
{$R_\textrm{L} = R_\textrm{i}$}
{$R_\textrm{L} = \dfrac{1}{R_\textrm{i}}$}
\end{QQuestion}

}
\only<2>{
\begin{QQuestion}{AB202}{In welchem Zusammenhang müssen der Innenwiderstand $R_\textrm{i}$ einer Strom- oder Spannungsquelle und ein direkt daran angeschlossener Lastwiderstand $R_\textrm{L}$ stehen, damit Leistungsanpassung vorliegt?}{$R_\textrm{L} \ll R_\textrm{i}$}
{$R_\textrm{L} \gg R_\textrm{i}$}
{\textbf{\textcolor{DARCgreen}{$R_\textrm{L} = R_\textrm{i}$}}}
{$R_\textrm{L} = \dfrac{1}{R_\textrm{i}}$}
\end{QQuestion}

}
\end{frame}

\begin{frame}
\only<1>{
\begin{QQuestion}{AB203}{In welchem Zusammenhang müssen der Innenwiderstand $R_{\symup{i}}$ einer Spannungsquelle und ein direkt daran angeschlossener Lastwiderstand $R_{\symup{L}}$ stehen, damit Spannungsanpassung vorliegt?}{$R_{\symup{L}} \gg R_{\symup{i}}$}
{$R_{\symup{L}} \ll R_{\symup{i}}$}
{$R_{\symup{L}} = R_{\symup{i}}$}
{$R_{\symup{L}} = \frac{1}{R_{\symup{i}}}$}
\end{QQuestion}

}
\only<2>{
\begin{QQuestion}{AB203}{In welchem Zusammenhang müssen der Innenwiderstand $R_{\symup{i}}$ einer Spannungsquelle und ein direkt daran angeschlossener Lastwiderstand $R_{\symup{L}}$ stehen, damit Spannungsanpassung vorliegt?}{\textbf{\textcolor{DARCgreen}{$R_{\symup{L}} \gg R_{\symup{i}}$}}}
{$R_{\symup{L}} \ll R_{\symup{i}}$}
{$R_{\symup{L}} = R_{\symup{i}}$}
{$R_{\symup{L}} = \frac{1}{R_{\symup{i}}}$}
\end{QQuestion}

}
\end{frame}

\begin{frame}
\only<1>{
\begin{QQuestion}{AB204}{In welchem Zusammenhang müssen der Innenwiderstand $R_\textrm{i}$ einer Stromquelle und ein direkt daran angeschlossener Lastwiderstand $R_\textrm{L}$ stehen, damit Stromanpassung vorliegt?}{$R_{\textrm{L}} \ll R_{\textrm{i}}$}
{$R_{\textrm{L}} \gg R_{\textrm{i}}$}
{$R_{\textrm{L}} = R_{\textrm{i}}$}
{$R_{\textrm{L}} = \dfrac{1}{R_{\textrm{i}}}$}
\end{QQuestion}

}
\only<2>{
\begin{QQuestion}{AB204}{In welchem Zusammenhang müssen der Innenwiderstand $R_\textrm{i}$ einer Stromquelle und ein direkt daran angeschlossener Lastwiderstand $R_\textrm{L}$ stehen, damit Stromanpassung vorliegt?}{\textbf{\textcolor{DARCgreen}{$R_{\textrm{L}} \ll R_{\textrm{i}}$}}}
{$R_{\textrm{L}} \gg R_{\textrm{i}}$}
{$R_{\textrm{L}} = R_{\textrm{i}}$}
{$R_{\textrm{L}} = \dfrac{1}{R_{\textrm{i}}}$}
\end{QQuestion}

}
\end{frame}

\begin{frame}
\only<1>{
\begin{QQuestion}{AB207}{Die Leerlaufspannung einer Gleichspannungsquelle beträgt \qty{13,5}{\V}. Wenn die Spannungsquelle einen Strom von \qty{2}{\A} abgibt, sinkt die Klemmenspannung auf \qty{13}{\V}. Wie groß ist der Innenwiderstand der Spannungsquelle?}{\qty{4}{\ohm}}
{\qty{6,75}{\ohm}}
{\qty{0,25}{\ohm}}
{\qty{1}{\ohm}}
\end{QQuestion}

}
\only<2>{
\begin{QQuestion}{AB207}{Die Leerlaufspannung einer Gleichspannungsquelle beträgt \qty{13,5}{\V}. Wenn die Spannungsquelle einen Strom von \qty{2}{\A} abgibt, sinkt die Klemmenspannung auf \qty{13}{\V}. Wie groß ist der Innenwiderstand der Spannungsquelle?}{\qty{4}{\ohm}}
{\qty{6,75}{\ohm}}
{\textbf{\textcolor{DARCgreen}{\qty{0,25}{\ohm}}}}
{\qty{1}{\ohm}}
\end{QQuestion}

}
\end{frame}

\begin{frame}
\frametitle{Lösungsweg}
\begin{itemize}
  \item gegeben: $U_0 = 13,5V$
  \item gegeben: $U_{Kl} = 13V$
  \item gegeben: $I = 2A$
  \item gesucht: $R_i$
  \end{itemize}
    \pause
    $R_i = \frac{U_i}{I} = \frac{U_0-U_{Kl}}{I} = \frac{13,5V-13V}{2A} = 0,25Ω$



\end{frame}

\begin{frame}
\only<1>{
\begin{QQuestion}{AB208}{Die Leerlaufspannung einer Gleichspannungsquelle beträgt \qty{13,8}{\V}. Wenn die Spannungsquelle einen Strom von \qty{20}{\A} abgibt, bleibt die Klemmenspannung auf \qty{13,6}{\V}. Wie groß ist der Innenwiderstand der Spannungsquelle?}{\qty{20}{\m}$\Omega$}
{\qty{10}{\m}$\Omega$}
{\qty{0,2}{\ohm}}
{\qty{0,1}{\ohm}}
\end{QQuestion}

}
\only<2>{
\begin{QQuestion}{AB208}{Die Leerlaufspannung einer Gleichspannungsquelle beträgt \qty{13,8}{\V}. Wenn die Spannungsquelle einen Strom von \qty{20}{\A} abgibt, bleibt die Klemmenspannung auf \qty{13,6}{\V}. Wie groß ist der Innenwiderstand der Spannungsquelle?}{\qty{20}{\m}$\Omega$}
{\textbf{\textcolor{DARCgreen}{\qty{10}{\m}$\Omega$}}}
{\qty{0,2}{\ohm}}
{\qty{0,1}{\ohm}}
\end{QQuestion}

}
\end{frame}

\begin{frame}
\frametitle{Lösungsweg}
\begin{itemize}
  \item gegeben: $U_0 = 13,8V$
  \item gegeben: $U_{Kl} = 13,6V$
  \item gegeben: $I = 20A$
  \item gesucht: $R_i$
  \end{itemize}
    \pause
    $R_i = \frac{U_i}{I} = \frac{U_0-U_{Kl}}{I} = \frac{13,8V-13,6V}{20A} = 10mΩ$



\end{frame}

\begin{frame}
\only<1>{
\begin{QQuestion}{AB206}{Die Leerlaufspannung einer Gleichspannungsquelle beträgt \qty{13,5}{\V}. Wenn die Spannungsquelle einen Strom von \qty{0,9}{\A} abgibt, sinkt die Klemmenspannung auf \qty{12,4}{\V}. Wie groß ist der Innenwiderstand der Spannungsquelle?}{\qty{0,99}{\ohm}}
{\qty{0,82}{\ohm}}
{\qty{1,22}{\ohm}}
{\qty{15,0}{\ohm}}
\end{QQuestion}

}
\only<2>{
\begin{QQuestion}{AB206}{Die Leerlaufspannung einer Gleichspannungsquelle beträgt \qty{13,5}{\V}. Wenn die Spannungsquelle einen Strom von \qty{0,9}{\A} abgibt, sinkt die Klemmenspannung auf \qty{12,4}{\V}. Wie groß ist der Innenwiderstand der Spannungsquelle?}{\qty{0,99}{\ohm}}
{\qty{0,82}{\ohm}}
{\textbf{\textcolor{DARCgreen}{\qty{1,22}{\ohm}}}}
{\qty{15,0}{\ohm}}
\end{QQuestion}

}
\end{frame}

\begin{frame}
\frametitle{Lösungsweg}
\begin{itemize}
  \item gegeben: $U_0 = 13,5V$
  \item gegeben: $U_{Kl} = 12,4V$
  \item gegeben: $I = 0,9A$
  \item gesucht: $R_i$
  \end{itemize}
    \pause
    $R_i = \frac{U_i}{I} = \frac{U_0-U_{Kl}}{I} = \frac{13,5V-12,4V}{0,9A} = 1,22Ω$



\end{frame}

\begin{frame}
\only<1>{
\begin{QQuestion}{AB205}{Die Leerlaufspannung einer Spannungsquelle beträgt \qty{5,0}{\V}. Schließt man einen Belastungswiderstand mit \qty{1,2}{\ohm} an, so geht die Klemmenspannung der Spannungsquelle auf \qty{4,8}{\V} zurück. Wie hoch ist der Innenwiderstand der Spannungsquelle?}{\qty{8,2}{\ohm}}
{\qty{0,05}{\ohm}}
{\qty{0,17}{\ohm}}
{\qty{0,25}{\ohm}}
\end{QQuestion}

}
\only<2>{
\begin{QQuestion}{AB205}{Die Leerlaufspannung einer Spannungsquelle beträgt \qty{5,0}{\V}. Schließt man einen Belastungswiderstand mit \qty{1,2}{\ohm} an, so geht die Klemmenspannung der Spannungsquelle auf \qty{4,8}{\V} zurück. Wie hoch ist der Innenwiderstand der Spannungsquelle?}{\qty{8,2}{\ohm}}
{\textbf{\textcolor{DARCgreen}{\qty{0,05}{\ohm}}}}
{\qty{0,17}{\ohm}}
{\qty{0,25}{\ohm}}
\end{QQuestion}

}
\end{frame}

\begin{frame}
\frametitle{Lösungsweg}
\begin{itemize}
  \item gegeben: $U_0 = 5,0V$
  \item gegeben: $U_{Kl} = 4,8V$
  \item gegeben: $R_L = 1,2Ω$
  \item gesucht: $R_i$
  \end{itemize}
    \pause
    $I = \frac{U_{Kl}}{R_L} = \frac{4,8V}{1,2Ω} = 4A$
    \pause
    $R_i = \frac{U_i}{I} = \frac{U_0-U_{Kl}}{I} = \frac{5,0V-4,8V}{4A} = 0,05Ω$



\end{frame}%ENDCONTENT


\section{Akkus}
\label{section:akku}
\begin{frame}%STARTCONTENT

\only<1>{
\begin{PQuestion}{AB209}{Folgende Schaltung eines Akkus besteht aus Zellen von je \qty{2}{\V}. Jede Zelle kann \qty{10}{\A\hour} Ladung liefern. Welche Daten hat der Akku?}{\qty{2}{\V}/\qty{10}{\A\hour}}
{\qty{12}{\V}/\qty{60}{\A\hour}}
{\qty{12}{\V}/\qty{10}{\A\hour}}
{\qty{2}{\V}/\qty{60}{\A\hour}}
{\DARCimage{1.0\linewidth}{431include}}\end{PQuestion}

}
\only<2>{
\begin{PQuestion}{AB209}{Folgende Schaltung eines Akkus besteht aus Zellen von je \qty{2}{\V}. Jede Zelle kann \qty{10}{\A\hour} Ladung liefern. Welche Daten hat der Akku?}{\qty{2}{\V}/\qty{10}{\A\hour}}
{\qty{12}{\V}/\qty{60}{\A\hour}}
{\textbf{\textcolor{DARCgreen}{\qty{12}{\V}/\qty{10}{\A\hour}}}}
{\qty{2}{\V}/\qty{60}{\A\hour}}
{\DARCimage{1.0\linewidth}{431include}}\end{PQuestion}

}
\end{frame}

\begin{frame}
\frametitle{Lösungsweg}
\begin{itemize}
  \item gegeben: $U = 2V$
  \item gegeben: $Q = 10Ah$
  \item gegeben: $N = 6$
  \item gesucht: $U_{ges}, Q_{ges}$
  \end{itemize}
    \pause
    $U_{ges} = N \cdot U = 6 \cdot 2V = 12V$
    \pause
    $Q_{ges} = Q =10Ah$



\end{frame}

\begin{frame}
\only<1>{
\begin{QQuestion}{AB210}{Auf dem Akku-Pack eines Handfunksprechgerätes stehen folgende Angaben: \qty{7,4}{\V} - \qty{2200}{\milli\A\hour} - \qty{16,28}{\W\hour}. Welcher Begriff ist für die Angabe \qty{2200}{\milli\A\hour} zutreffend.}{Nennkapazität}
{Nennleistung}
{maximaler Ladestrom pro Stunde}
{maximaler Entladestrom pro Stunde}
\end{QQuestion}

}
\only<2>{
\begin{QQuestion}{AB210}{Auf dem Akku-Pack eines Handfunksprechgerätes stehen folgende Angaben: \qty{7,4}{\V} - \qty{2200}{\milli\A\hour} - \qty{16,28}{\W\hour}. Welcher Begriff ist für die Angabe \qty{2200}{\milli\A\hour} zutreffend.}{\textbf{\textcolor{DARCgreen}{Nennkapazität}}}
{Nennleistung}
{maximaler Ladestrom pro Stunde}
{maximaler Entladestrom pro Stunde}
\end{QQuestion}

}
\end{frame}

\begin{frame}
\only<1>{
\begin{QQuestion}{AB211}{Wie lange könnte man idealerweise mit einem voll geladenen Akku mit \qty{60}{\A\hour} einen Amateurfunkempfänger betreiben, bis dieser auf \qty{10}{\percent} seiner Kapazität entladen ist und einen Strom von \qty{0,8}{\A} aufnimmt?}{74~Stunden und 60~Minuten}
{43~Stunden und 12~Minuten}
{67~Stunden und 30~Minuten}
{48~Stunden und 0~Minuten}
\end{QQuestion}

}
\only<2>{
\begin{QQuestion}{AB211}{Wie lange könnte man idealerweise mit einem voll geladenen Akku mit \qty{60}{\A\hour} einen Amateurfunkempfänger betreiben, bis dieser auf \qty{10}{\percent} seiner Kapazität entladen ist und einen Strom von \qty{0,8}{\A} aufnimmt?}{74~Stunden und 60~Minuten}
{43~Stunden und 12~Minuten}
{\textbf{\textcolor{DARCgreen}{67~Stunden und 30~Minuten}}}
{48~Stunden und 0~Minuten}
\end{QQuestion}

}
\end{frame}

\begin{frame}
\frametitle{Lösungsweg}
\begin{itemize}
  \item gegeben: $Q_{max} = 60Ah$
  \item gegeben: $Q_{10\%} = 0,1 \cdot Q_{max} = 6Ah$
  \item gegeben: $I = 0,8A$
  \item gesucht: $t$
  \end{itemize}
    \pause
    $Q = I \cdot t \Rightarrow t = \frac{Q}{I} = \frac{Q_{max} -- Q_{10\%}}{I} = \frac{54Ah}{0,8A} = 67,5h$



\end{frame}

\begin{frame}
\only<1>{
\begin{QQuestion}{AB501}{Ein \qty{12}{\V} Akku hat eine Kapazität von \qty{5}{\A\hour}. Welcher speicherbaren Energie entspricht das?}{\qty{5,0}{\W\hour}}
{\qty{12,0}{\W\hour}}
{\qty{2,4}{\W\hour}}
{\qty{60,0}{\W\hour}}
\end{QQuestion}

}
\only<2>{
\begin{QQuestion}{AB501}{Ein \qty{12}{\V} Akku hat eine Kapazität von \qty{5}{\A\hour}. Welcher speicherbaren Energie entspricht das?}{\qty{5,0}{\W\hour}}
{\qty{12,0}{\W\hour}}
{\qty{2,4}{\W\hour}}
{\textbf{\textcolor{DARCgreen}{\qty{60,0}{\W\hour}}}}
\end{QQuestion}

}
\end{frame}

\begin{frame}
\frametitle{Lösungsweg}
\begin{itemize}
  \item gegeben: $U = 12V$
  \item gegeben: $Q = 5Ah$
  \item gesucht: $W$
  \end{itemize}
    \pause
    $W = P \cdot t = U \cdot I \cdot t = U \cdot Q = 12V \cdot 5Ah = 60,0Wh$



\end{frame}%ENDCONTENT


\section{Photovoltaik}
\label{section:photovoltaik}
\begin{frame}%STARTCONTENT

\only<1>{
\begin{QQuestion}{AB212}{Was ist die primäre Aufgabe einer Solarzelle?}{Die Umwandlung von elektrischer Energie in Strahlungsenergie.}
{Die Umwandlung von Strahlungsenergie in elektrische Energie.}
{Die Umwandlung von Strahlungsenergie in thermische Energie.}
{Die Umwandlung von thermischer Energie in Strahlungsenergie.}
\end{QQuestion}

}
\only<2>{
\begin{QQuestion}{AB212}{Was ist die primäre Aufgabe einer Solarzelle?}{Die Umwandlung von elektrischer Energie in Strahlungsenergie.}
{\textbf{\textcolor{DARCgreen}{Die Umwandlung von Strahlungsenergie in elektrische Energie.}}}
{Die Umwandlung von Strahlungsenergie in thermische Energie.}
{Die Umwandlung von thermischer Energie in Strahlungsenergie.}
\end{QQuestion}

}
\end{frame}

\begin{frame}
\only<1>{
\begin{PQuestion}{AD301}{Ein Photovoltaikmodul besteht aus vier parallel geschalteten Reihen von je 30 Solarzellen mit je Zelle \qty{0,6}{\V} Leerlaufspannung und \qty{1}{\A} Kurzschlussstrom. Welche Leerlaufspannung und welchen Kurzschlussstrom liefert das Modul?}{Leerlaufspannung: \qty{18}{\V}, Kurzschlussstrom: \qty{4}{\A}}
{Leerlaufspannung: \qty{18}{\V}, Kurzschlussstrom: \qty{30}{\A}}
{Leerlaufspannung: \qty{2,4}{\V}, Kurzschlussstrom: \qty{4}{\A}}
{Leerlaufspannung: \qty{2,4}{\V}, Kurzschlussstrom: \qty{30}{\A}}
{\DARCimage{1.0\linewidth}{46include}}\end{PQuestion}

}
\only<2>{
\begin{PQuestion}{AD301}{Ein Photovoltaikmodul besteht aus vier parallel geschalteten Reihen von je 30 Solarzellen mit je Zelle \qty{0,6}{\V} Leerlaufspannung und \qty{1}{\A} Kurzschlussstrom. Welche Leerlaufspannung und welchen Kurzschlussstrom liefert das Modul?}{\textbf{\textcolor{DARCgreen}{Leerlaufspannung: \qty{18}{\V}, Kurzschlussstrom: \qty{4}{\A}}}}
{Leerlaufspannung: \qty{18}{\V}, Kurzschlussstrom: \qty{30}{\A}}
{Leerlaufspannung: \qty{2,4}{\V}, Kurzschlussstrom: \qty{4}{\A}}
{Leerlaufspannung: \qty{2,4}{\V}, Kurzschlussstrom: \qty{30}{\A}}
{\DARCimage{1.0\linewidth}{46include}}\end{PQuestion}

}
\end{frame}

\begin{frame}
\frametitle{Lösungweg}
\begin{itemize}
  \item gegeben: $U_0 = 0,6V$
  \item gegeben: $I_k = 1A$
  \item gegeben: $N_R = 30, N_P = 4$
  \item gesucht: $U_{0,ges}, I_{k,ges}$
  \end{itemize}
    \pause
    $U_{0,ges} = N_R \cdot U_0 = 30 \cdot 0,6V = 18V$
    \pause
    $I_{0,ges} = N_P \cdot I_k = 4 \cdot 1A = 4A$



\end{frame}%ENDCONTENT


\section{Spannungswandler}
\label{section:spannungswandler}
\begin{frame}%STARTCONTENT
\end{frame}%ENDCONTENT


\section{Gleichrichter II}
\label{section:gleichrichter_2}
\begin{frame}%STARTCONTENT

\only<1>{
\begin{PQuestion}{AD302}{Berechnen Sie für diese Schaltung die Leerlaufspannung an den Klemmen A - B.}{Zirka \qty{42}{\V}}
{Zirka \qty{15}{\V}}
{Zirka \qty{30}{\V}}
{Zirka \qty{21}{\V}}
{\DARCimage{1.0\linewidth}{198include}}\end{PQuestion}

}
\only<2>{
\begin{PQuestion}{AD302}{Berechnen Sie für diese Schaltung die Leerlaufspannung an den Klemmen A - B.}{Zirka \qty{42}{\V}}
{Zirka \qty{15}{\V}}
{Zirka \qty{30}{\V}}
{\textbf{\textcolor{DARCgreen}{Zirka \qty{21}{\V}}}}
{\DARCimage{1.0\linewidth}{198include}}\end{PQuestion}

}
\end{frame}

\begin{frame}
\frametitle{Lösungsweg}
\begin{itemize}
  \item gegeben: $U_{eff} = 15V$
  \item gesucht: $\hat{U}$
  \end{itemize}
    \pause
    $\hat{U} = U_{eff} \cdot \sqrt{2} = 15V \cdot 1,41 = 21,21V$



\end{frame}

\begin{frame}
\only<1>{
\begin{PQuestion}{AD303}{Welche Spannungsfestigkeit des Kondensators sollte mindestens gewählt werden, wenn das Transformationsverhältnis 20:1 beträgt und ein Sicherheitsaufschlag auf die Spannungsfestigkeit von \qty{50}{\percent} berücksichtigt werden soll?}{\qty{35}{\V}}
{\qty{16}{\V} }
{\qty{25}{\V} }
{\qty{10}{\V}}
{\DARCimage{1.0\linewidth}{29include}}\end{PQuestion}

}
\only<2>{
\begin{PQuestion}{AD303}{Welche Spannungsfestigkeit des Kondensators sollte mindestens gewählt werden, wenn das Transformationsverhältnis 20:1 beträgt und ein Sicherheitsaufschlag auf die Spannungsfestigkeit von \qty{50}{\percent} berücksichtigt werden soll?}{\qty{35}{\V}}
{\qty{16}{\V} }
{\textbf{\textcolor{DARCgreen}{\qty{25}{\V} }}}
{\qty{10}{\V}}
{\DARCimage{1.0\linewidth}{29include}}\end{PQuestion}

}
\end{frame}

\begin{frame}
\frametitle{Lösungsweg}
\begin{itemize}
  \item gegeben: $U_P = 230V$
  \item gegeben: $\"{u} = 20:1$
  \item gesucht: $\hat{U} + 50\%$
  \end{itemize}
    \pause
    $ü = \frac{U_P}{U_S} \Rightarrow U_S = \frac{U_P}{ü} = \frac{230V}{20} = 11,5V$
    \pause
    $\hat{U} = U_S \cdot \sqrt{2} = 11,5V \cdot 1,41 \approx 16,26V$
    \pause
    $\hat{U} + 50\% \approx 25V$



\end{frame}

\begin{frame}
\only<1>{
\begin{PQuestion}{AD304}{Bei einem Transformationsverhältnis von 5:1 sollte die Spannungsfestigkeit der Diode (max. Spannung plus \qty{20}{\percent} Sicherheitsaufschlag) in dieser Schaltung nicht weniger als~...}{\qty{90}{\V} betragen.}
{\qty{78}{\V} betragen.}
{\qty{156}{\V} betragen.}
{\qty{130}{\V} betragen.}
{\DARCimage{1.0\linewidth}{29include}}\end{PQuestion}

}
\only<2>{
\begin{PQuestion}{AD304}{Bei einem Transformationsverhältnis von 5:1 sollte die Spannungsfestigkeit der Diode (max. Spannung plus \qty{20}{\percent} Sicherheitsaufschlag) in dieser Schaltung nicht weniger als~...}{\qty{90}{\V} betragen.}
{\qty{78}{\V} betragen.}
{\textbf{\textcolor{DARCgreen}{\qty{156}{\V} betragen.}}}
{\qty{130}{\V} betragen.}
{\DARCimage{1.0\linewidth}{29include}}\end{PQuestion}

}
\end{frame}

\begin{frame}
\frametitle{Lösungsweg}
\begin{itemize}
  \item gegeben: $U_P = 230V$
  \item gegeben: $\"{u} = 5:1$
  \item gesucht: $U_{SS} + 20\%$
  \end{itemize}
    \pause
    $ü = \frac{U_P}{U_S} \Rightarrow U_S = \frac{U_P}{ü} = \frac{230V}{5} = 46V$
    \pause
    $\hat{U} = U_S \cdot \sqrt{2} = 46V \cdot 1,41 \approx 65,05V$
    \pause
    $U_{SS} + 20\% = 2 \cdot \hat{U} + 20\% \approx 156V$



\end{frame}%ENDCONTENT


\section{Brückengleichrichter}
\label{section:brueckengleichrichter}
\begin{frame}%STARTCONTENT

\only<1>{
\begin{question2x2}{AD305}{Welche der folgenden Auswahlantworten enthält die richtige Diodenanordnung und Polarität eines Brückengleichrichters?}{\DARCimage{1.0\linewidth}{69include}}
{\DARCimage{1.0\linewidth}{68include}}
{\DARCimage{1.0\linewidth}{67include}}
{\DARCimage{1.0\linewidth}{70include}}
\end{question2x2}

}
\only<2>{
\begin{question2x2}{AD305}{Welche der folgenden Auswahlantworten enthält die richtige Diodenanordnung und Polarität eines Brückengleichrichters?}{\DARCimage{1.0\linewidth}{69include}}
{\DARCimage{1.0\linewidth}{68include}}
{\textbf{\textcolor{DARCgreen}{\DARCimage{1.0\linewidth}{67include}}}}
{\DARCimage{1.0\linewidth}{70include}}
\end{question2x2}

}
\end{frame}

\begin{frame}
\only<1>{
\begin{PQuestion}{AD306}{Wie groß ist die Spannung am Siebkondensator $C_{\symup{S}}$ im Leerlauf, wenn die Netzwechselspannung von \qty{230}{\V} anliegt und das Windungsverhältnis 8:1 beträgt?}{etwa \qty{20}{\V}}
{etwa \qty{40}{\V}}
{etwa \qty{29}{\V}}
{etwa \qty{58}{\V}}
{\DARCimage{1.0\linewidth}{66include}}\end{PQuestion}

}
\only<2>{
\begin{PQuestion}{AD306}{Wie groß ist die Spannung am Siebkondensator $C_{\symup{S}}$ im Leerlauf, wenn die Netzwechselspannung von \qty{230}{\V} anliegt und das Windungsverhältnis 8:1 beträgt?}{etwa \qty{20}{\V}}
{\textbf{\textcolor{DARCgreen}{etwa \qty{40}{\V}}}}
{etwa \qty{29}{\V}}
{etwa \qty{58}{\V}}
{\DARCimage{1.0\linewidth}{66include}}\end{PQuestion}

}
\end{frame}

\begin{frame}
\frametitle{Lösungsweg}
\begin{itemize}
  \item gegeben: $U_P = 230V$
  \item gegeben: $\"{u} = 8:1$
  \item gegeben: $U_D = 0,6V$
  \item gesucht: $\hat{U}$
  \end{itemize}
    \pause
    $ü = \frac{U_P}{U_S} \Rightarrow U_S = \frac{U_P}{ü} = \frac{230V}{8} = 28,75V$
    \pause
    Im Leerlauf kann die Diodenspannung vernachlässigt werden.

$\hat{U} = U_S \cdot \sqrt{2} = 28,75V \cdot 1,41 \approx 40V$



\end{frame}%ENDCONTENT


\section{Vollweggleichrichter}
\label{section:vollweggleichrichter}
\begin{frame}%STARTCONTENT

\only<1>{
\begin{question2x2}{AD307}{Welche Gleichrichterschaltung erzeugt eine Vollweg-Gleichrichtung mit der angezeigten Polarität?}{\DARCimage{1.0\linewidth}{72include}}
{\DARCimage{1.0\linewidth}{71include}}
{\DARCimage{1.0\linewidth}{73include}}
{\DARCimage{1.0\linewidth}{74include}}
\end{question2x2}

}
\only<2>{
\begin{question2x2}{AD307}{Welche Gleichrichterschaltung erzeugt eine Vollweg-Gleichrichtung mit der angezeigten Polarität?}{\DARCimage{1.0\linewidth}{72include}}
{\textbf{\textcolor{DARCgreen}{\DARCimage{1.0\linewidth}{71include}}}}
{\DARCimage{1.0\linewidth}{73include}}
{\DARCimage{1.0\linewidth}{74include}}
\end{question2x2}

}
\end{frame}

\begin{frame}
\only<1>{
\begin{PQuestion}{AD308}{Welche Form hat die Ausgangsspannung der dargestellten Schaltung?}{\DARCimage{1\linewidth}{64include}}
{\DARCimage{1\linewidth}{62include}}
{\DARCimage{1\linewidth}{63include}}
{\DARCimage{1\linewidth}{61include}}
{\DARCimage{1\linewidth}{65include}}\end{PQuestion}

}
\only<2>{
\begin{PQuestion}{AD308}{Welche Form hat die Ausgangsspannung der dargestellten Schaltung?}{\DARCimage{1\linewidth}{64include}}
{\DARCimage{1\linewidth}{62include}}
{\DARCimage{1\linewidth}{63include}}
{\textbf{\textcolor{DARCgreen}{\DARCimage{1\linewidth}{61include}}}}
{\DARCimage{1\linewidth}{65include}}\end{PQuestion}

}
\end{frame}

\begin{frame}
\only<1>{
\begin{QQuestion}{AD310}{Welche Grundfrequenz hat die Ausgangsspannung eines Vollweggleichrichters, der an eine \qty{50}{\Hz}-Versorgung angeschlossen ist?}{\qty{100}{\Hz}}
{\qty{50}{\Hz}}
{\qty{25}{\Hz}}
{\qty{200}{\Hz}}
\end{QQuestion}

}
\only<2>{
\begin{QQuestion}{AD310}{Welche Grundfrequenz hat die Ausgangsspannung eines Vollweggleichrichters, der an eine \qty{50}{\Hz}-Versorgung angeschlossen ist?}{\textbf{\textcolor{DARCgreen}{\qty{100}{\Hz}}}}
{\qty{50}{\Hz}}
{\qty{25}{\Hz}}
{\qty{200}{\Hz}}
\end{QQuestion}

}
\end{frame}%ENDCONTENT


\section{Restwelligkeit}
\label{section:restwelligkeit}
\begin{frame}%STARTCONTENT

\only<1>{
\begin{PQuestion}{AD309}{Im folgenden Bild ist die Spannung am Ausgang einer Stromversorgung dargestellt. Die Restwelligkeit und die Brummfrequenz betragen~...}{\qty{3}{\V}; \qty{50}{\Hz}}
{\qty{3}{\V}; \qty{100}{\Hz}}
{\qty{13,5}{\V}~\pm\qty{1,5}{\V}; \qty{50}{\Hz}}
{\qty{13,5}{\V}~\pm\qty{1,5}{\V}; \qty{100}{\Hz}}
{\DARCimage{1.0\linewidth}{75include}}\end{PQuestion}

}
\only<2>{
\begin{PQuestion}{AD309}{Im folgenden Bild ist die Spannung am Ausgang einer Stromversorgung dargestellt. Die Restwelligkeit und die Brummfrequenz betragen~...}{\qty{3}{\V}; \qty{50}{\Hz}}
{\textbf{\textcolor{DARCgreen}{\qty{3}{\V}; \qty{100}{\Hz}}}}
{\qty{13,5}{\V}~\pm\qty{1,5}{\V}; \qty{50}{\Hz}}
{\qty{13,5}{\V}~\pm\qty{1,5}{\V}; \qty{100}{\Hz}}
{\DARCimage{1.0\linewidth}{75include}}\end{PQuestion}

}
\end{frame}%ENDCONTENT


\section{Schaltnetzteil II}
\label{section:schaltnetzteil_2}
\begin{frame}%STARTCONTENT

\only<1>{
\begin{PQuestion}{AD311}{Welche Funktion übernimmt der elektronische Schalter (Block E) des Schaltnetzteils?}{Gleichrichter}
{Überspannungsableiter}
{Impulsbreitenmodulator}
{Puls-Gleichrichter}
{\DARCimage{1.0\linewidth}{35include}}\end{PQuestion}

}
\only<2>{
\begin{PQuestion}{AD311}{Welche Funktion übernimmt der elektronische Schalter (Block E) des Schaltnetzteils?}{Gleichrichter}
{Überspannungsableiter}
{\textbf{\textcolor{DARCgreen}{Impulsbreitenmodulator}}}
{Puls-Gleichrichter}
{\DARCimage{1.0\linewidth}{35include}}\end{PQuestion}

}
\end{frame}

\begin{frame}
\only<1>{
\begin{PQuestion}{AD312}{Was ist der Hauptnachteil des dargestellten Schaltnetzteils?}{Der Transformator bewirkt hohe Verluste}
{Der elektronische Schalter in Block E erzeugt ein unerwünschtes Signalspektrum.}
{Der Brückengleichrichter erzeugt eine Spannung mit Restwelligkeit.}
{Die Diode am Ausgang muss hohe Frequenzen gleichrichten.}
{\DARCimage{1.0\linewidth}{35include}}\end{PQuestion}

}
\only<2>{
\begin{PQuestion}{AD312}{Was ist der Hauptnachteil des dargestellten Schaltnetzteils?}{Der Transformator bewirkt hohe Verluste}
{\textbf{\textcolor{DARCgreen}{Der elektronische Schalter in Block E erzeugt ein unerwünschtes Signalspektrum.}}}
{Der Brückengleichrichter erzeugt eine Spannung mit Restwelligkeit.}
{Die Diode am Ausgang muss hohe Frequenzen gleichrichten.}
{\DARCimage{1.0\linewidth}{35include}}\end{PQuestion}

}
\end{frame}

\begin{frame}
\only<1>{
\begin{QQuestion}{AD313}{In einem Amateurfunkempfänger werden etwa alle \qty{120}{\kHz} unerwünschte Signale festgestellt. Dies ist wahrscheinlich zurückzuführen auf~...}{einen schlecht entstörten Bürstenmotor.}
{unerwünschte Abstrahlungen eines Schaltnetzteils.}
{unerwünschte Abstrahlungen eines linearen Netzteils.}
{eine Amateurfunkstelle mit unzureichender Anpassung der Antenne.}
\end{QQuestion}

}
\only<2>{
\begin{QQuestion}{AD313}{In einem Amateurfunkempfänger werden etwa alle \qty{120}{\kHz} unerwünschte Signale festgestellt. Dies ist wahrscheinlich zurückzuführen auf~...}{einen schlecht entstörten Bürstenmotor.}
{\textbf{\textcolor{DARCgreen}{unerwünschte Abstrahlungen eines Schaltnetzteils.}}}
{unerwünschte Abstrahlungen eines linearen Netzteils.}
{eine Amateurfunkstelle mit unzureichender Anpassung der Antenne.}
\end{QQuestion}

}
\end{frame}

\begin{frame}
\only<1>{
\begin{question2x2}{AD314}{Welche der dargestellten Schaltungen könnte in den Netzeingang eines Schaltnetzteils eingebaut werden, um eine Verbreitung von Störungen in das Stromversorgungsnetz zu verringern?}{\DARCimage{1.0\linewidth}{370include}}
{\DARCimage{1.0\linewidth}{368include}}
{\DARCimage{1.0\linewidth}{369include}}
{\DARCimage{1.0\linewidth}{367include}}
\end{question2x2}

}
\only<2>{
\begin{question2x2}{AD314}{Welche der dargestellten Schaltungen könnte in den Netzeingang eines Schaltnetzteils eingebaut werden, um eine Verbreitung von Störungen in das Stromversorgungsnetz zu verringern?}{\DARCimage{1.0\linewidth}{370include}}
{\DARCimage{1.0\linewidth}{368include}}
{\DARCimage{1.0\linewidth}{369include}}
{\textbf{\textcolor{DARCgreen}{\DARCimage{1.0\linewidth}{367include}}}}
\end{question2x2}

}
\end{frame}%ENDCONTENT


\section{Spannungsstabilisierung}
\label{section:spannungsstabilisierung}
\begin{frame}%STARTCONTENT

\only<1>{
\begin{PQuestion}{AD315}{Wenn man folgendes Signal an den Eingang der gezeigten Schaltung anlegt, beträgt die Ausgangsspannung zwischen A und B ungefähr~...}{\qty{6,2}{\V}.}
{\qty{11,2}{\V}.}
{\qty{5}{\V}.}
{\qty{5,6}{\V}.}
{\DARCimage{1.0\linewidth}{490include}}\end{PQuestion}

}
\only<2>{
\begin{PQuestion}{AD315}{Wenn man folgendes Signal an den Eingang der gezeigten Schaltung anlegt, beträgt die Ausgangsspannung zwischen A und B ungefähr~...}{\qty{6,2}{\V}.}
{\qty{11,2}{\V}.}
{\textbf{\textcolor{DARCgreen}{\qty{5}{\V}.}}}
{\qty{5,6}{\V}.}
{\DARCimage{1.0\linewidth}{490include}}\end{PQuestion}

}
\end{frame}

\begin{frame}
\only<1>{
\begin{PQuestion}{AD316}{Welche Beziehung muss zwischen der Eingangsspannung und der Ausgangsspannung der folgenden Schaltung bestehen, damit der Linearspannungsregler IC1 eine stabilisierte Ausgangsspannung erzeugt?}{Die Eingangsspannung muss gleich der gewünschten Ausgangsspannung sein}
{Die Eingangsspannung muss größer als die gewünschte Ausgangsspannung sein.}
{Die Eingangsspannung muss mindestens doppelt so groß wie die gewünschte Ausgangsspannung sein.}
{Die Eingangsspannung muss kleiner als die gewünschte Ausgangsspannung sein.}
{\DARCimage{1.0\linewidth}{200include}}\end{PQuestion}

}
\only<2>{
\begin{PQuestion}{AD316}{Welche Beziehung muss zwischen der Eingangsspannung und der Ausgangsspannung der folgenden Schaltung bestehen, damit der Linearspannungsregler IC1 eine stabilisierte Ausgangsspannung erzeugt?}{Die Eingangsspannung muss gleich der gewünschten Ausgangsspannung sein}
{\textbf{\textcolor{DARCgreen}{Die Eingangsspannung muss größer als die gewünschte Ausgangsspannung sein.}}}
{Die Eingangsspannung muss mindestens doppelt so groß wie die gewünschte Ausgangsspannung sein.}
{Die Eingangsspannung muss kleiner als die gewünschte Ausgangsspannung sein.}
{\DARCimage{1.0\linewidth}{200include}}\end{PQuestion}

}
\end{frame}

\begin{frame}
\only<1>{
\begin{PQuestion}{AD317}{Bei dieser Schaltung mit einem \qty{12}{\V}-Festspannungsregler schwankt die Eingangsspannung zwischen \qty{15}{\V} und \qty{18}{\V}. Wie groß ist die Spannungsschwankung am Ausgang?}{Die Spannungsschwankung beträgt ca.~\qty{3}{\V}.}
{Die Spannungsschwankung beträgt nahezu null Volt.}
{Die Spannungsschwankung beträgt ca.~\qty{0,7}{\V}.}
{Die Spannungsschwankung liegt zwischen \qty{0,7}{\V} und \qty{3}{\V}.}
{\DARCimage{1.0\linewidth}{200include}}\end{PQuestion}

}
\only<2>{
\begin{PQuestion}{AD317}{Bei dieser Schaltung mit einem \qty{12}{\V}-Festspannungsregler schwankt die Eingangsspannung zwischen \qty{15}{\V} und \qty{18}{\V}. Wie groß ist die Spannungsschwankung am Ausgang?}{Die Spannungsschwankung beträgt ca.~\qty{3}{\V}.}
{\textbf{\textcolor{DARCgreen}{Die Spannungsschwankung beträgt nahezu null Volt.}}}
{Die Spannungsschwankung beträgt ca.~\qty{0,7}{\V}.}
{Die Spannungsschwankung liegt zwischen \qty{0,7}{\V} und \qty{3}{\V}.}
{\DARCimage{1.0\linewidth}{200include}}\end{PQuestion}

}
\end{frame}

\begin{frame}
\only<1>{
\begin{QQuestion}{AD319}{Ein linearer Spannungsregler stabilisiert eine Eingangsspannung von \qty{13,8}{\V} auf eine Ausgangsspannung von \qty{9}{\V}. Es fließt ein Ausgangsstrom von \qty{900}{\mA}. Wie groß ist die Verlustleistung im Spannungsregler?}{\qty{12,42}{\W}}
{\qty{8,10}{\W}}
{\qty{4,32}{\W}}
{\qty{1,53}{\W}}
\end{QQuestion}

}
\only<2>{
\begin{QQuestion}{AD319}{Ein linearer Spannungsregler stabilisiert eine Eingangsspannung von \qty{13,8}{\V} auf eine Ausgangsspannung von \qty{9}{\V}. Es fließt ein Ausgangsstrom von \qty{900}{\mA}. Wie groß ist die Verlustleistung im Spannungsregler?}{\qty{12,42}{\W}}
{\qty{8,10}{\W}}
{\textbf{\textcolor{DARCgreen}{\qty{4,32}{\W}}}}
{\qty{1,53}{\W}}
\end{QQuestion}

}
\end{frame}

\begin{frame}
\frametitle{Lösungsweg}
\begin{itemize}
  \item gegeben: $U_{zu} = 13,8V$
  \item gegeben: $U_{ab} = 9V$
  \item gegeben: $I = 900mA$
  \item gesucht: $P_V$
  \end{itemize}
    \pause
    $U_{IC1} = U_{zu} -- U_{ab} = 13,8V -- 9V = 4,8V$
    \pause
    $P_V = U_{IC1} \cdot I = 4,8V \cdot 900mA = 4,32W$



\end{frame}

\begin{frame}
\only<1>{
\begin{PQuestion}{AD318}{Wie groß ist die Verlustleistung im Linearspannungsregler IC1?}{\qty{2,5}{\W}}
{\qty{4,4}{\W}}
{\qty{7,9}{\W} }
{\qty{5,0}{\W}}
{\DARCimage{1.0\linewidth}{201include}}\end{PQuestion}

}
\only<2>{
\begin{PQuestion}{AD318}{Wie groß ist die Verlustleistung im Linearspannungsregler IC1?}{\qty{2,5}{\W}}
{\textbf{\textcolor{DARCgreen}{\qty{4,4}{\W}}}}
{\qty{7,9}{\W} }
{\qty{5,0}{\W}}
{\DARCimage{1.0\linewidth}{201include}}\end{PQuestion}

}
\end{frame}

\begin{frame}
\frametitle{Lösungsweg}
\begin{itemize}
  \item gegeben: $U_{zu} = 13,8V$
  \item gegeben: $U_{ab} = 5V$
  \item gegeben: $R_L = 10Ω$
  \item gesucht: $P_V$
  \end{itemize}
    \pause
    $I = \frac{U_{zu}}{R_L} = \frac{5V}{10Ω} = 500mA$
    \pause
    $U_{IC1} = U_{zu} -- U_{ab} = 13,8V -- 5V = 8,8V$
    \pause
    $P_V = U_{IC1} \cdot I = 8,8V \cdot 500mA = 4,4W$



\end{frame}

\begin{frame}
\only<1>{
\begin{QQuestion}{AD320}{Ein linearer Spannungsregler stabilisiert eine Eingangsspannung von \qty{13,8}{\V} auf eine Ausgangsspannung von \qty{5}{\V}. Es fließt ein Eingangsstrom von \qty{455}{\mA} und ein Ausgangsstrom von \qty{450}{\mA}. Wie groß ist der Wirkungsgrad?}{0,99}
{0,36}
{0,56}
{0,64}
\end{QQuestion}

}
\only<2>{
\begin{QQuestion}{AD320}{Ein linearer Spannungsregler stabilisiert eine Eingangsspannung von \qty{13,8}{\V} auf eine Ausgangsspannung von \qty{5}{\V}. Es fließt ein Eingangsstrom von \qty{455}{\mA} und ein Ausgangsstrom von \qty{450}{\mA}. Wie groß ist der Wirkungsgrad?}{0,99}
{\textbf{\textcolor{DARCgreen}{0,36}}}
{0,56}
{0,64}
\end{QQuestion}

}
\end{frame}

\begin{frame}
\frametitle{Lösungsweg}
\begin{itemize}
  \item gegeben: $U_{zu} = 13,8V$
  \item gegeben: $U_{ab} = 5V$
  \item gegeben: $I_{zu} = 455mA$
  \item gegeben: $I_{ab} = 450mA$
  \item gesucht: $\eta$
  \end{itemize}
    \pause
    $\eta = \frac{P_{ab}}{P_{zu}} = \frac{U_{ab} \cdot I_{ab}}{U_{zu} \cdot I_{zu}} = \frac{5V \cdot 450mA}{13,8V \cdot 455mA} \approx 0,36$



\end{frame}

\begin{frame}
\only<1>{
\begin{PQuestion}{AD321}{Wie groß ist der Wirkungsgrad $\left(\eta~=~\dfrac{P_{\symup{L}}}{P_{\symup{IN}}}\right)$ der dargestellten Spannungsstabilisierung, wenn durch den Lastwiderstand $R_{\symup{L}}$~=~\qty{470}{\ohm} ein Strom von $I_{\symup{L}}$~=~\qty{10}{\mA} und durch die Z-Diode ein Strom $I_{\symup{Z}}$~=~\qty{15}{\mA} fließt.}{0{,}21}
{0{,}34}
{0{,}17}
{0{,}14}
{\DARCimage{1.0\linewidth}{323include}}\end{PQuestion}

}
\only<2>{
\begin{PQuestion}{AD321}{Wie groß ist der Wirkungsgrad $\left(\eta~=~\dfrac{P_{\symup{L}}}{P_{\symup{IN}}}\right)$ der dargestellten Spannungsstabilisierung, wenn durch den Lastwiderstand $R_{\symup{L}}$~=~\qty{470}{\ohm} ein Strom von $I_{\symup{L}}$~=~\qty{10}{\mA} und durch die Z-Diode ein Strom $I_{\symup{Z}}$~=~\qty{15}{\mA} fließt.}{0{,}21}
{0{,}34}
{0{,}17}
{\textbf{\textcolor{DARCgreen}{0{,}14}}}
{\DARCimage{1.0\linewidth}{323include}}\end{PQuestion}

}
\end{frame}

\begin{frame}
\frametitle{Lösungsweg}
\begin{itemize}
  \item gegeben: $R_L = 470Ω$
  \item gegeben: $I_L = 10mA$
  \item gegeben: $I_Z = 15mA$
  \item gegeben: $U_{IN} = 13,8V$
  \item gesucht: $\eta = \frac{P_L}{P_{IN}}$
  \end{itemize}
    \pause
    $P_L = I_L^2 \cdot R_L = (10mA)^2 \cdot 470Ω = 47mW$
    \pause
    $P_{IN} = U_{IN} \cdot I_{IN} = U_{IN} \cdot (I_Z + I_L) = 13,8V \cdot (15mA + 10mA) = 345mW$
    \pause
    $\eta = \frac{P_L}{P_{IN}} = \frac{47mW}{345mW} \approx 0,14$



\end{frame}%ENDCONTENT


\section{Fernspeiseweiche}
\label{section:fernspeiseweiche}
\begin{frame}%STARTCONTENT

\only<1>{
\begin{QQuestion}{AD322}{Zu welchem Zweck wird ein Bias-T (Fernspeiseweiche) eingesetzt?}{Zur Verteilung der Gleichspannung auf zwei unterschiedliche Geräte.}
{Zur Gleichspannungsversorgung und HF-Signalübertragung über eine gemeinsame Leitung.}
{Zur Verteilung eines HF-Signals auf zwei Ausgänge.}
{Zur Übertragung von zwei unterschiedlichen Gleichspannungen über eine gemeinsame Leitung.}
\end{QQuestion}

}
\only<2>{
\begin{QQuestion}{AD322}{Zu welchem Zweck wird ein Bias-T (Fernspeiseweiche) eingesetzt?}{Zur Verteilung der Gleichspannung auf zwei unterschiedliche Geräte.}
{\textbf{\textcolor{DARCgreen}{Zur Gleichspannungsversorgung und HF-Signalübertragung über eine gemeinsame Leitung.}}}
{Zur Verteilung eines HF-Signals auf zwei Ausgänge.}
{Zur Übertragung von zwei unterschiedlichen Gleichspannungen über eine gemeinsame Leitung.}
\end{QQuestion}

}
\end{frame}

\begin{frame}
\only<1>{
\begin{PQuestion}{AD323}{Was stellt die folgende Schaltung dar? }{PI-Filter}
{Bandsperre}
{Bias-T}
{Netzfilter}
{\DARCimage{0.75\linewidth}{399include}}\end{PQuestion}

}
\only<2>{
\begin{PQuestion}{AD323}{Was stellt die folgende Schaltung dar? }{PI-Filter}
{Bandsperre}
{\textbf{\textcolor{DARCgreen}{Bias-T}}}
{Netzfilter}
{\DARCimage{0.75\linewidth}{399include}}\end{PQuestion}

}
\end{frame}

\begin{frame}
\only<1>{
\begin{PQuestion}{AD324}{Zu welchem Zweck dient $C_1$ in dem dargestellten Bias-T?}{Zur Verbesserung des Tiefpass-Verhaltens.}
{Zur Siebung der Gleichspannung.}
{Zur HF-Trennung von RX und LNA.}
{Zur Trennung der Gleichspannung vom Empfängereingang.}
{\DARCimage{0.75\linewidth}{399include}}\end{PQuestion}

}
\only<2>{
\begin{PQuestion}{AD324}{Zu welchem Zweck dient $C_1$ in dem dargestellten Bias-T?}{Zur Verbesserung des Tiefpass-Verhaltens.}
{Zur Siebung der Gleichspannung.}
{Zur HF-Trennung von RX und LNA.}
{\textbf{\textcolor{DARCgreen}{Zur Trennung der Gleichspannung vom Empfängereingang.}}}
{\DARCimage{0.75\linewidth}{399include}}\end{PQuestion}

}
\end{frame}

\begin{frame}
\only<1>{
\begin{PQuestion}{AD325}{Was ist bei der Dimensionierung der Spule in dem dargestellten Bias-T zu beachten?}{Temperaturkoeffizient}
{Spannungsfestigkeit}
{Strombelastbarkeit}
{Güte}
{\DARCimage{0.75\linewidth}{399include}}\end{PQuestion}

}
\only<2>{
\begin{PQuestion}{AD325}{Was ist bei der Dimensionierung der Spule in dem dargestellten Bias-T zu beachten?}{Temperaturkoeffizient}
{Spannungsfestigkeit}
{\textbf{\textcolor{DARCgreen}{Strombelastbarkeit}}}
{Güte}
{\DARCimage{0.75\linewidth}{399include}}\end{PQuestion}

}
\end{frame}%ENDCONTENT


\title{DARC Amateurfunklehrgang Klasse A}
\author{Grundlegende Schaltungen}
\institute{Deutscher Amateur Radio Club e.\,V.}
\begin{frame}
\maketitle
\end{frame}

\section{Schwingkreis II}
\label{section:schwingkreis_2}
\begin{frame}%STARTCONTENT

\only<1>{
\begin{PQuestion}{AD201}{Welche Grenzfrequenz ergibt sich bei einem Hochpass mit einem Widerstand von \qty{4,7}{\kohm} und einem Kondensator von \qty{2,2}{\nF}?}{\qty{154}{\Hz}}
{\qty{1,54}{\kHz}}
{\qty{154}{\kHz}}
{\qty{15,4}{\kHz}}
{\DARCimage{1.0\linewidth}{195include}}\end{PQuestion}

}
\only<2>{
\begin{PQuestion}{AD201}{Welche Grenzfrequenz ergibt sich bei einem Hochpass mit einem Widerstand von \qty{4,7}{\kohm} und einem Kondensator von \qty{2,2}{\nF}?}{\qty{154}{\Hz}}
{\qty{1,54}{\kHz}}
{\qty{154}{\kHz}}
{\textbf{\textcolor{DARCgreen}{\qty{15,4}{\kHz}}}}
{\DARCimage{1.0\linewidth}{195include}}\end{PQuestion}

}
\end{frame}

\begin{frame}
\frametitle{Lösungsweg}
\begin{itemize}
  \item gegeben: $R = 4,7kΩ$
  \item gegeben: $C = 2,2nF$
  \item gesucht: $f_g$
  \end{itemize}
    \pause
    $f_g = \frac{1}{2 \cdot \pi \cdot R \cdot C} = \frac{1}{2 \cdot \pi \cdot 4,7kΩ \cdot 2,2nF} = 15,4kHz$



\end{frame}

\begin{frame}
\only<1>{
\begin{PQuestion}{AD202}{Welche Grenzfrequenz ergibt sich bei einem Tiefpass mit einem Widerstand von \qty{10}{\kohm} und einem Kondensator von \qty{47}{\nF}?}{\qty{339}{\kHz}}
{\qty{3,39}{\kHz}}
{\qty{339}{\Hz}}
{\qty{33,9}{\Hz}}
{\DARCimage{1.0\linewidth}{175include}}\end{PQuestion}

}
\only<2>{
\begin{PQuestion}{AD202}{Welche Grenzfrequenz ergibt sich bei einem Tiefpass mit einem Widerstand von \qty{10}{\kohm} und einem Kondensator von \qty{47}{\nF}?}{\qty{339}{\kHz}}
{\qty{3,39}{\kHz}}
{\textbf{\textcolor{DARCgreen}{\qty{339}{\Hz}}}}
{\qty{33,9}{\Hz}}
{\DARCimage{1.0\linewidth}{175include}}\end{PQuestion}

}
\end{frame}

\begin{frame}
\frametitle{Lösungsweg}
\begin{itemize}
  \item gegeben: $R = 10kΩ$
  \item gegeben: $C = 47nF$
  \item gesucht: $f_g$
  \end{itemize}
    \pause
    $f_g = \frac{1}{2 \cdot \pi \cdot R \cdot C} = \frac{1}{2 \cdot \pi \cdot 10kΩ \cdot 47nF} = 339Hz$



\end{frame}

\begin{frame}
\only<1>{
\begin{PQuestion}{AD203}{Wo liegt die Grenzfrequenz des Audio-Verstärkers, wenn $R_{1}$ = \qty{4,7}{\kilo\ohm}, $C_1$ = \qty{6,8}{\nF} und $C_2$ = \qty{47}{\nF} betragen? Der Verstärker hat eine Grenzfrequenz von \qty{1}{\MHz} und die Impedanz des Eingangs PIN 2 ist mit \qty{1}{\Mohm} sehr hochohmig.}{ca. 5~kHz}
{ca. 720~Hz}
{ca. 2,7~kHz}
{ca. 294~Hz}
{\DARCimage{1.0\linewidth}{488include}}\end{PQuestion}

}
\only<2>{
\begin{PQuestion}{AD203}{Wo liegt die Grenzfrequenz des Audio-Verstärkers, wenn $R_{1}$ = \qty{4,7}{\kilo\ohm}, $C_1$ = \qty{6,8}{\nF} und $C_2$ = \qty{47}{\nF} betragen? Der Verstärker hat eine Grenzfrequenz von \qty{1}{\MHz} und die Impedanz des Eingangs PIN 2 ist mit \qty{1}{\Mohm} sehr hochohmig.}{\textbf{\textcolor{DARCgreen}{ca. 5~kHz}}}
{ca. 720~Hz}
{ca. 2,7~kHz}
{ca. 294~Hz}
{\DARCimage{1.0\linewidth}{488include}}\end{PQuestion}

}
\end{frame}

\begin{frame}
\frametitle{Lösungsweg}
\begin{itemize}
  \item gegeben: $R_1 = 4,7kΩ$
  \item gegeben: $C_1 = 6,8nF$
  \item gesucht: $f_g$
  \end{itemize}
    \pause
    $C_2$ und alle weiteren Angaben sind für den Tiefpass uninteressant.
    \pause
    $f_g = \frac{1}{2 \cdot \pi \cdot R_1 \cdot C_1} = \frac{1}{2 \cdot \pi \cdot 4,7kΩ \cdot 6,8nF} \approx 5kHz$



\end{frame}

\begin{frame}
\only<1>{
\begin{QQuestion}{AD206}{Was ist im Resonanzfall bei der Reihenschaltung einer Induktivität mit einer Kapazität erfüllt?}{Der Betrag des elektrischen Feldes in der Spule ist dann gleich dem Betrag des elektrischen Feldes im Kondensator.}
{Der Betrag des Verlustwiderstandes der Spule ist dann gleich dem Betrag des Verlustwiderstandes des Kondensators.}
{Der Betrag des induktiven Widerstands ist dann gleich dem Betrag des kapazitiven Widerstands.}
{Der Betrag des magnetischen Feldes in der Spule ist dann gleich dem Betrag des magnetischen Feldes im Kondensator.}
\end{QQuestion}

}
\only<2>{
\begin{QQuestion}{AD206}{Was ist im Resonanzfall bei der Reihenschaltung einer Induktivität mit einer Kapazität erfüllt?}{Der Betrag des elektrischen Feldes in der Spule ist dann gleich dem Betrag des elektrischen Feldes im Kondensator.}
{Der Betrag des Verlustwiderstandes der Spule ist dann gleich dem Betrag des Verlustwiderstandes des Kondensators.}
{\textbf{\textcolor{DARCgreen}{Der Betrag des induktiven Widerstands ist dann gleich dem Betrag des kapazitiven Widerstands.}}}
{Der Betrag des magnetischen Feldes in der Spule ist dann gleich dem Betrag des magnetischen Feldes im Kondensator.}
\end{QQuestion}

}
\end{frame}

\begin{frame}
\only<1>{
\begin{PQuestion}{AD207}{Bei der Resonanzfrequenz ist die Impedanz dieser Schaltung~...}{gleich dem Wirkwiderstand $R$.}
{unendlich hoch.}
{gleich dem kapazitiven Widerstand $X_{\symup{C}}$.}
{gleich dem induktiven Widerstand $X_{\symup{L}}$.}
{\DARCimage{1.0\linewidth}{181include}}\end{PQuestion}

}
\only<2>{
\begin{PQuestion}{AD207}{Bei der Resonanzfrequenz ist die Impedanz dieser Schaltung~...}{\textbf{\textcolor{DARCgreen}{gleich dem Wirkwiderstand $R$.}}}
{unendlich hoch.}
{gleich dem kapazitiven Widerstand $X_{\symup{C}}$.}
{gleich dem induktiven Widerstand $X_{\symup{L}}$.}
{\DARCimage{1.0\linewidth}{181include}}\end{PQuestion}

}
\end{frame}

\begin{frame}
\only<1>{
\begin{question2x2}{AD204}{Welcher Schwingkreis passt zu dem neben der jeweiligen Schaltung dargestellten Verlauf der Impedanz?}{\DARCimage{1.0\linewidth}{230include}}
{\DARCimage{1.0\linewidth}{231include}}
{\DARCimage{1.0\linewidth}{232include}}
{\DARCimage{1.0\linewidth}{826include}}
\end{question2x2}

}
\only<2>{
\begin{question2x2}{AD204}{Welcher Schwingkreis passt zu dem neben der jeweiligen Schaltung dargestellten Verlauf der Impedanz?}{\textbf{\textcolor{DARCgreen}{\DARCimage{1.0\linewidth}{230include}}}}
{\DARCimage{1.0\linewidth}{231include}}
{\DARCimage{1.0\linewidth}{232include}}
{\DARCimage{1.0\linewidth}{826include}}
\end{question2x2}

}
\end{frame}

\begin{frame}
\only<1>{
\begin{PQuestion}{AD208}{Welche Resonanzfrequenz $f_{\symup{res}}$ hat die Reihenschaltung einer Spule von \qty{1,2}{\micro\H} mit einem Kondensator von \qty{6,8}{\pF} und einem Widerstand von \qty{10}{\ohm}?}{\qty{55,7}{\MHz}}
{\qty{5,57}{\MHz}}
{\qty{557}{\MHz}}
{\qty{557}{\kHz}}
{\DARCimage{1.0\linewidth}{181include}}\end{PQuestion}

}
\only<2>{
\begin{PQuestion}{AD208}{Welche Resonanzfrequenz $f_{\symup{res}}$ hat die Reihenschaltung einer Spule von \qty{1,2}{\micro\H} mit einem Kondensator von \qty{6,8}{\pF} und einem Widerstand von \qty{10}{\ohm}?}{\textbf{\textcolor{DARCgreen}{\qty{55,7}{\MHz}}}}
{\qty{5,57}{\MHz}}
{\qty{557}{\MHz}}
{\qty{557}{\kHz}}
{\DARCimage{1.0\linewidth}{181include}}\end{PQuestion}

}
\end{frame}

\begin{frame}
\frametitle{Lösungsweg}
\begin{itemize}
  \item gegeben: $L = 1,2µH$
  \item gegeben: $C = 6,8pF$
  \item gegeben: $R = 10Ω$
  \item gesucht: $f_0$
  \end{itemize}
    \pause
    $f_0 = \frac{1}{2 \cdot \pi \cdot \sqrt{L \cdot C}} = \frac{1}{2 \cdot \pi \cdot \sqrt{1,2µH \cdot 6,8pF}} = 55,7MHz$
    \pause
    Widerstand $R$ wird zur Berechnung nicht benötigt.



\end{frame}

\begin{frame}
\only<1>{
\begin{PQuestion}{AD209}{Welche Resonanzfrequenz $f_{\symup{res}}$ hat die Reihenschaltung einer Spule von \qty{10}{\micro\H} mit einem Kondensator von \qty{1}{\nF} und einem Widerstand von \qty{0,1}{\kohm}?}{\qty{15,92}{\MHz}}
{\qty{159,2}{\kHz}}
{\qty{1,592}{\MHz}}
{\qty{15,92}{\kHz}}
{\DARCimage{1.0\linewidth}{181include}}\end{PQuestion}

}
\only<2>{
\begin{PQuestion}{AD209}{Welche Resonanzfrequenz $f_{\symup{res}}$ hat die Reihenschaltung einer Spule von \qty{10}{\micro\H} mit einem Kondensator von \qty{1}{\nF} und einem Widerstand von \qty{0,1}{\kohm}?}{\qty{15,92}{\MHz}}
{\qty{159,2}{\kHz}}
{\textbf{\textcolor{DARCgreen}{\qty{1,592}{\MHz}}}}
{\qty{15,92}{\kHz}}
{\DARCimage{1.0\linewidth}{181include}}\end{PQuestion}

}
\end{frame}

\begin{frame}
\frametitle{Lösungsweg}
\begin{itemize}
  \item gegeben: $L = 10µH$
  \item gegeben: $C = 1nF$
  \item gesucht: $f_0$
  \end{itemize}
    \pause
    $f_0 = \frac{1}{2 \cdot \pi \cdot \sqrt{L \cdot C}} = \frac{1}{2 \cdot \pi \cdot \sqrt{10µH \cdot 1nF}} = 1,592MHz$



\end{frame}

\begin{frame}
\only<1>{
\begin{PQuestion}{AD210}{Welche Resonanzfrequenz $f_{\symup{res}}$ hat die Reihenschaltung einer Spule von \qty{100}{\micro\H} mit einem Kondensator von \qty{0,01}{\micro\F} und einem Widerstand von \qty{100}{\ohm}?}{\qty{1,59}{\kHz}}
{\qty{15,9}{\kHz}}
{\qty{159}{\kHz}}
{\qty{1590}{\kHz}}
{\DARCimage{1.0\linewidth}{181include}}\end{PQuestion}

}
\only<2>{
\begin{PQuestion}{AD210}{Welche Resonanzfrequenz $f_{\symup{res}}$ hat die Reihenschaltung einer Spule von \qty{100}{\micro\H} mit einem Kondensator von \qty{0,01}{\micro\F} und einem Widerstand von \qty{100}{\ohm}?}{\qty{1,59}{\kHz}}
{\qty{15,9}{\kHz}}
{\textbf{\textcolor{DARCgreen}{\qty{159}{\kHz}}}}
{\qty{1590}{\kHz}}
{\DARCimage{1.0\linewidth}{181include}}\end{PQuestion}

}
\end{frame}

\begin{frame}
\frametitle{Lösungsweg}
\begin{itemize}
  \item gegeben: $L = 100µH$
  \item gegeben: $C = 0,01µF$
  \item gesucht: $f_0$
  \end{itemize}
    \pause
    $f_0 = \frac{1}{2 \cdot \pi \cdot \sqrt{L \cdot C}} = \frac{1}{2 \cdot \pi \cdot \sqrt{100µH \cdot 0,01µF}} = 159kHz$



\end{frame}

\begin{frame}
\only<1>{
\begin{PQuestion}{AD211}{Welche Resonanzfrequenz $f_{\symup{res}}$ hat die Parallelschaltung einer Spule von \qty{2,2}{\micro\H} mit einem Kondensator von \qty{56}{\pF} und einem Widerstand von \qty{10}{\kohm}?}{\qty{14,34}{\MHz}}
{\qty{143,4}{\MHz}}
{\qty{1,434}{\MHz}}
{\qty{143,4}{\kHz}}
{\DARCimage{1.0\linewidth}{233include}}\end{PQuestion}

}
\only<2>{
\begin{PQuestion}{AD211}{Welche Resonanzfrequenz $f_{\symup{res}}$ hat die Parallelschaltung einer Spule von \qty{2,2}{\micro\H} mit einem Kondensator von \qty{56}{\pF} und einem Widerstand von \qty{10}{\kohm}?}{\textbf{\textcolor{DARCgreen}{\qty{14,34}{\MHz}}}}
{\qty{143,4}{\MHz}}
{\qty{1,434}{\MHz}}
{\qty{143,4}{\kHz}}
{\DARCimage{1.0\linewidth}{233include}}\end{PQuestion}

}
\end{frame}

\begin{frame}
\frametitle{Lösungsweg}
\begin{itemize}
  \item gegeben: $L = 2,2µH$
  \item gegeben: $C = 56pF$
  \item gesucht: $f_0$
  \end{itemize}
    \pause
    $f_0 = \frac{1}{2 \cdot \pi \cdot \sqrt{L \cdot C}} = \frac{1}{2 \cdot \pi \cdot \sqrt{2,2µH \cdot 56pF}} = 14,34MHz$



\end{frame}

\begin{frame}
\only<1>{
\begin{PQuestion}{AD212}{Wie groß ist die Resonanzfrequenz dieser Schaltung, wenn die Kapazitäten $C_1$ = \qty{0,1}{\nF}, $C_2$ = \qty{1,5}{\nF}, $C_3$ = \qty{220}{\pF} und die Induktivität der Spule \qty{1,2}{\mH} betragen?}{\qty{10,77}{\kHz}}
{\qty{107,7}{\kHz}}
{\qty{1,077}{\kHz}}
{\qty{1,077}{\MHz}}
{\DARCimage{1.0\linewidth}{776include}}\end{PQuestion}

}
\only<2>{
\begin{PQuestion}{AD212}{Wie groß ist die Resonanzfrequenz dieser Schaltung, wenn die Kapazitäten $C_1$ = \qty{0,1}{\nF}, $C_2$ = \qty{1,5}{\nF}, $C_3$ = \qty{220}{\pF} und die Induktivität der Spule \qty{1,2}{\mH} betragen?}{\qty{10,77}{\kHz}}
{\textbf{\textcolor{DARCgreen}{\qty{107,7}{\kHz}}}}
{\qty{1,077}{\kHz}}
{\qty{1,077}{\MHz}}
{\DARCimage{1.0\linewidth}{776include}}\end{PQuestion}

}
\end{frame}

\begin{frame}
\frametitle{Lösungsweg}
\begin{itemize}
  \item gegeben: $C_1 = 0,1nF$
  \item gegeben: $C_2 = 1,5nF$
  \item gegeben: $C_3 = 220pF$
  \item gegeben: $L = 1,2mH$
  \item gesucht: $f_0$
  \end{itemize}
    \pause
    $C = C_1 + C_2 + C_3 = 0,1nF + 1,5nF + 220pF = 1,82nF$
    \pause
    $f_0 = \frac{1}{2 \cdot \pi \cdot \sqrt{L \cdot C}} = \frac{1}{2 \cdot \pi \cdot \sqrt{1,2mH \cdot 1,82nF}} = 107,7kHz$



\end{frame}

\begin{frame}
\only<1>{
\begin{QQuestion}{AD213}{Sie wollen die Resonanzfrequenz eines Schwingkreises vergrößern. Welche der folgenden Maßnahmen ist geeignet?}{Kleineren Spulenwert verwenden}
{Spule zusammenschieben}
{Ferritkern in die Spule einführen}
{Anzahl der Spulenwindungen erhöhen}
\end{QQuestion}

}
\only<2>{
\begin{QQuestion}{AD213}{Sie wollen die Resonanzfrequenz eines Schwingkreises vergrößern. Welche der folgenden Maßnahmen ist geeignet?}{\textbf{\textcolor{DARCgreen}{Kleineren Spulenwert verwenden}}}
{Spule zusammenschieben}
{Ferritkern in die Spule einführen}
{Anzahl der Spulenwindungen erhöhen}
\end{QQuestion}

}
\end{frame}

\begin{frame}
\only<1>{
\begin{QQuestion}{AD214}{Sie wollen die Resonanzfrequenz eines Schwingkreises vergrößern. Welche der folgenden Maßnahmen ist geeignet?}{Spule zusammenschieben}
{Anzahl der Spulenwindungen verringern}
{Größeren Spulenwert verwenden}
{Größeren Kondensatorwert verwenden}
\end{QQuestion}

}
\only<2>{
\begin{QQuestion}{AD214}{Sie wollen die Resonanzfrequenz eines Schwingkreises vergrößern. Welche der folgenden Maßnahmen ist geeignet?}{Spule zusammenschieben}
{\textbf{\textcolor{DARCgreen}{Anzahl der Spulenwindungen verringern}}}
{Größeren Spulenwert verwenden}
{Größeren Kondensatorwert verwenden}
\end{QQuestion}

}
\end{frame}

\begin{frame}
\only<1>{
\begin{QQuestion}{AD215}{Sie wollen die Resonanzfrequenz eines Schwingkreises verringern. Welche der folgenden Maßnahmen ist geeignet?}{Anzahl der Spulenwindungen verringern}
{Kleineren Spulenwert verwenden}
{Größeren Kondensatorwert verwenden}
{Spule auseinanderziehen}
\end{QQuestion}

}
\only<2>{
\begin{QQuestion}{AD215}{Sie wollen die Resonanzfrequenz eines Schwingkreises verringern. Welche der folgenden Maßnahmen ist geeignet?}{Anzahl der Spulenwindungen verringern}
{Kleineren Spulenwert verwenden}
{\textbf{\textcolor{DARCgreen}{Größeren Kondensatorwert verwenden}}}
{Spule auseinanderziehen}
\end{QQuestion}

}
\end{frame}

\begin{frame}
\only<1>{
\begin{QQuestion}{AD216}{Sie wollen die Resonanzfrequenz eines Schwingkreises verringern. Welche der folgenden Maßnahmen ist geeignet?}{Kleineren Spulenwert verwenden}
{Spule auseinanderziehen}
{Kleineren Kondensatorwert verwenden}
{Spule zusammenschieben}
\end{QQuestion}

}
\only<2>{
\begin{QQuestion}{AD216}{Sie wollen die Resonanzfrequenz eines Schwingkreises verringern. Welche der folgenden Maßnahmen ist geeignet?}{Kleineren Spulenwert verwenden}
{Spule auseinanderziehen}
{Kleineren Kondensatorwert verwenden}
{\textbf{\textcolor{DARCgreen}{Spule zusammenschieben}}}
\end{QQuestion}

}
\end{frame}

\begin{frame}
\only<1>{
\begin{QQuestion}{AD217}{Sie wollen die Resonanzfrequenz eines Schwingkreises verringern. Welche der folgenden Maßnahmen ist geeignet?}{Kleineren Spulenwert verwenden}
{Spule auseinanderziehen}
{Kleineren Kondensatorwert verwenden}
{Ferritkern in die Spule einführen}
\end{QQuestion}

}
\only<2>{
\begin{QQuestion}{AD217}{Sie wollen die Resonanzfrequenz eines Schwingkreises verringern. Welche der folgenden Maßnahmen ist geeignet?}{Kleineren Spulenwert verwenden}
{Spule auseinanderziehen}
{Kleineren Kondensatorwert verwenden}
{\textbf{\textcolor{DARCgreen}{Ferritkern in die Spule einführen}}}
\end{QQuestion}

}
\end{frame}

\begin{frame}
\only<1>{
\begin{PQuestion}{AD218}{Wie verändert sich die Frequenz des Schwingkreises in der folgenden Schaltung, wenn das Potentiometer mehr in Richtung X gedreht wird?}{Die Frequenz des Schwingkreises ändert sich nicht.}
{Die Frequenz des Schwingkreises sinkt.}
{Die Frequenz des Schwingkreises steigt.}
{Die Frequenz sinkt zunächst und steigt dann stark an.}
{\DARCimage{1.0\linewidth}{752include}}\end{PQuestion}

}
\only<2>{
\begin{PQuestion}{AD218}{Wie verändert sich die Frequenz des Schwingkreises in der folgenden Schaltung, wenn das Potentiometer mehr in Richtung X gedreht wird?}{Die Frequenz des Schwingkreises ändert sich nicht.}
{Die Frequenz des Schwingkreises sinkt.}
{\textbf{\textcolor{DARCgreen}{Die Frequenz des Schwingkreises steigt.}}}
{Die Frequenz sinkt zunächst und steigt dann stark an.}
{\DARCimage{1.0\linewidth}{752include}}\end{PQuestion}

}
\end{frame}

\begin{frame}
\only<1>{
\begin{PQuestion}{AD205}{Welche der nachfolgenden Beschreibungen trifft auf diese Schaltung zu und wie nennt man sie? }{Es handelt sich um eine Bandsperre. Frequenzen oberhalb der oberen Grenzfrequenz und Frequenzen unterhalb der unteren Grenzfrequenz werden durchgelassen. Sie bedämpft nur einen bestimmten Frequenzbereich.}
{Es handelt sich um einen Hochpass. Frequenzen unterhalb der Grenzfrequenz werden bedämpft, oberhalb der Grenzfrequenz durchgelassen.}
{Es handelt sich um einen Tiefpass. Frequenzen oberhalb der Grenzfrequenz werden bedämpft, unterhalb der Grenzfrequenz durchgelassen.}
{Es handelt sich um einen Bandpass. Frequenzen oberhalb der oberen Grenzfrequenz und Frequenzen unterhalb der unteren Grenzfrequenz werden bedämpft. Er lässt nur einen bestimmten Frequenzbereich passieren.}
{\DARCimage{1.0\linewidth}{785include}}\end{PQuestion}

}
\only<2>{
\begin{PQuestion}{AD205}{Welche der nachfolgenden Beschreibungen trifft auf diese Schaltung zu und wie nennt man sie? }{Es handelt sich um eine Bandsperre. Frequenzen oberhalb der oberen Grenzfrequenz und Frequenzen unterhalb der unteren Grenzfrequenz werden durchgelassen. Sie bedämpft nur einen bestimmten Frequenzbereich.}
{Es handelt sich um einen Hochpass. Frequenzen unterhalb der Grenzfrequenz werden bedämpft, oberhalb der Grenzfrequenz durchgelassen.}
{Es handelt sich um einen Tiefpass. Frequenzen oberhalb der Grenzfrequenz werden bedämpft, unterhalb der Grenzfrequenz durchgelassen.}
{\textbf{\textcolor{DARCgreen}{Es handelt sich um einen Bandpass. Frequenzen oberhalb der oberen Grenzfrequenz und Frequenzen unterhalb der unteren Grenzfrequenz werden bedämpft. Er lässt nur einen bestimmten Frequenzbereich passieren.}}}
{\DARCimage{1.0\linewidth}{785include}}\end{PQuestion}

}
\end{frame}

\begin{frame}
\only<1>{
\begin{PQuestion}{AD219}{Wie groß ist die Bandbreite in dem dargestellten Diagramm bei \qty{-60}{\decibel}?}{Etwa \qty{6,0}{\kHz}}
{Etwa \qty{6,5}{\kHz}}
{Etwa \qty{4,0}{\kHz}}
{Etwa \qty{2,5}{\kHz}}
{\DARCimage{1.0\linewidth}{38include}}\end{PQuestion}

}
\only<2>{
\begin{PQuestion}{AD219}{Wie groß ist die Bandbreite in dem dargestellten Diagramm bei \qty{-60}{\decibel}?}{Etwa \qty{6,0}{\kHz}}
{Etwa \qty{6,5}{\kHz}}
{\textbf{\textcolor{DARCgreen}{Etwa \qty{4,0}{\kHz}}}}
{Etwa \qty{2,5}{\kHz}}
{\DARCimage{1.0\linewidth}{38include}}\end{PQuestion}

}
\end{frame}

\begin{frame}
\only<1>{
\begin{QQuestion}{AD221}{Ein Quarzfilter mit einer \qty{3}{\decibel}-Bandbreite von \qty{2,7}{\kHz} eignet sich besonders zur Verwendung in einem Sendeempfänger für~...}{FM.}
{AM.}
{SSB.}
{CW.}
\end{QQuestion}

}
\only<2>{
\begin{QQuestion}{AD221}{Ein Quarzfilter mit einer \qty{3}{\decibel}-Bandbreite von \qty{2,7}{\kHz} eignet sich besonders zur Verwendung in einem Sendeempfänger für~...}{FM.}
{AM.}
{\textbf{\textcolor{DARCgreen}{SSB.}}}
{CW.}
\end{QQuestion}

}
\end{frame}

\begin{frame}
\only<1>{
\begin{QQuestion}{AD222}{Ein Quarzfilter mit einer \qty{3}{\decibel}-Bandbreite von \qty{500}{\Hz} eignet sich besonders zur Verwendung in einem Sendeempfänger für~...}{SSB.}
{CW.}
{AM.}
{FM.}
\end{QQuestion}

}
\only<2>{
\begin{QQuestion}{AD222}{Ein Quarzfilter mit einer \qty{3}{\decibel}-Bandbreite von \qty{500}{\Hz} eignet sich besonders zur Verwendung in einem Sendeempfänger für~...}{SSB.}
{\textbf{\textcolor{DARCgreen}{CW.}}}
{AM.}
{FM.}
\end{QQuestion}

}
\end{frame}

\begin{frame}
\only<1>{
\begin{QQuestion}{AD220}{Wie ergibt sich die Bandbreite $B$ eines Parallelschwingkreises aus der Resonanzkurve?}{Die Bandbreite ergibt sich aus der Differenz der beiden Frequenzen, bei denen die Spannung auf den 0,7-fachen Wert gegenüber der maximalen Spannung bei der Resonanzfrequenz abgesunken ist.}
{Die Bandbreite ergibt sich aus der Differenz der beiden Frequenzen, bei denen die Spannung auf den 0,5-fachen Wert gegenüber der maximalen Spannung bei der Resonanzfrequenz abgesunken ist.}
{Die Bandbreite ergibt sich aus der Multiplikation der Resonanzfrequenz mit dem Faktor 0,5.}
{Die Bandbreite ergibt sich aus der Multiplikation der Resonanzfrequenz mit dem Faktor 0,7.}
\end{QQuestion}

}
\only<2>{
\begin{QQuestion}{AD220}{Wie ergibt sich die Bandbreite $B$ eines Parallelschwingkreises aus der Resonanzkurve?}{\textbf{\textcolor{DARCgreen}{Die Bandbreite ergibt sich aus der Differenz der beiden Frequenzen, bei denen die Spannung auf den 0,7-fachen Wert gegenüber der maximalen Spannung bei der Resonanzfrequenz abgesunken ist.}}}
{Die Bandbreite ergibt sich aus der Differenz der beiden Frequenzen, bei denen die Spannung auf den 0,5-fachen Wert gegenüber der maximalen Spannung bei der Resonanzfrequenz abgesunken ist.}
{Die Bandbreite ergibt sich aus der Multiplikation der Resonanzfrequenz mit dem Faktor 0,5.}
{Die Bandbreite ergibt sich aus der Multiplikation der Resonanzfrequenz mit dem Faktor 0,7.}
\end{QQuestion}

}
\end{frame}

\begin{frame}
\only<1>{
\begin{QQuestion}{AD223}{Welche Bandbreite $B$ hat die Reihenschaltung einer Spule von \qty{100}{\micro\H} mit einem Kondensator von \qty{0,01}{\micro\F} und einem Widerstand von \qty{10}{\ohm}?}{\qty{1,59}{\kHz}}
{\qty{159}{\kHz}}
{\qty{15,9}{\kHz}}
{\qty{159}{\Hz}}
\end{QQuestion}

}
\only<2>{
\begin{QQuestion}{AD223}{Welche Bandbreite $B$ hat die Reihenschaltung einer Spule von \qty{100}{\micro\H} mit einem Kondensator von \qty{0,01}{\micro\F} und einem Widerstand von \qty{10}{\ohm}?}{\qty{1,59}{\kHz}}
{\qty{159}{\kHz}}
{\textbf{\textcolor{DARCgreen}{\qty{15,9}{\kHz}}}}
{\qty{159}{\Hz}}
\end{QQuestion}

}
\end{frame}

\begin{frame}
\frametitle{Lösungsweg}
\begin{itemize}
  \item gegeben: $L = 100µH$
  \item gegeben: $C = 0,01µF$
  \item gegeben: $R_S = 10Ω$
  \item gesucht: $B$
  \end{itemize}
    \pause
    $B = \frac{R_S}{2\cdot \pi \cdot L} = \frac{10Ω}{2\cdot \pi \cdot 100µH} = 15,9kHz$



\end{frame}

\begin{frame}
\only<1>{
\begin{QQuestion}{AD225}{Welchen Gütefaktor $Q$ hat die Reihenschaltung einer Spule von \qty{100}{\micro\H} mit einem Kondensator von \qty{0,01}{\micro\F} und einem Widerstand von \qty{10}{\ohm}?}{10}
{1}
{0,1}
{100}
\end{QQuestion}

}
\only<2>{
\begin{QQuestion}{AD225}{Welchen Gütefaktor $Q$ hat die Reihenschaltung einer Spule von \qty{100}{\micro\H} mit einem Kondensator von \qty{0,01}{\micro\F} und einem Widerstand von \qty{10}{\ohm}?}{\textbf{\textcolor{DARCgreen}{10}}}
{1}
{0,1}
{100}
\end{QQuestion}

}
\end{frame}

\begin{frame}
\frametitle{Lösungsweg}
\begin{itemize}
  \item gegeben: $L = 100µH$
  \item gegeben: $C = 0,01µF$
  \item gegeben: $R_S = 10Ω$
  \item gesucht: $Q$
  \end{itemize}
    \pause
    $f_0 = \frac{1}{2 \cdot \pi \cdot \sqrt{L \cdot C}} = \frac{1}{2 \cdot \pi \cdot \sqrt{100µH \cdot 0,01µF}} = 159,2kHz$
    \pause
    $B$ oder $X_L$ ausrechnen – $B$ haben wir schon vorher ausgerechnet

$B = \frac{R_S}{2\cdot \pi \cdot L} = \frac{10Ω}{2\cdot \pi \cdot 100µH} = 15,92kHz$
    \pause
    $Q = \frac{f_0}{B} = \frac{159,2kHz}{15,92kHz} = 10$



\end{frame}

\begin{frame}
\only<1>{
\begin{QQuestion}{AD224}{Welche Bandbreite $B$ hat die Parallelschaltung einer Spule von \qty{2,2}{\micro\H} mit einem Kondensator von \qty{56}{\pF} und einem Widerstand von \qty{1}{\kohm}?}{\qty{28,4}{\kHz}}
{\qty{28,4}{\MHz}}
{\qty{284}{\kHz}}
{\qty{2,84}{\MHz}}
\end{QQuestion}

}
\only<2>{
\begin{QQuestion}{AD224}{Welche Bandbreite $B$ hat die Parallelschaltung einer Spule von \qty{2,2}{\micro\H} mit einem Kondensator von \qty{56}{\pF} und einem Widerstand von \qty{1}{\kohm}?}{\qty{28,4}{\kHz}}
{\qty{28,4}{\MHz}}
{\qty{284}{\kHz}}
{\textbf{\textcolor{DARCgreen}{\qty{2,84}{\MHz}}}}
\end{QQuestion}

}
\end{frame}

\begin{frame}
\frametitle{Lösungsweg}
\begin{itemize}
  \item gegeben: $L = 2,2µH$
  \item gegeben: $C = 56pF$
  \item gegeben: $R_P = 1kΩ$
  \item gesucht: $B$
  \end{itemize}
    \pause
    $B = \frac{1}{2\cdot \pi \cdot R_P \cdot C} = \frac{1}{2\cdot \pi \cdot 1kΩ \cdot 56pF} = 2,84MHz$



\end{frame}

\begin{frame}
\only<1>{
\begin{QQuestion}{AD226}{Welchen Gütefaktor $Q$ hat die Parallelschaltung einer Spule von \qty{2,2}{\micro\H} mit einem Kondensator von \qty{56}{\pF} und einem Widerstand von \qty{1}{\kohm}?}{15}
{50}
{5}
{0,2}
\end{QQuestion}

}
\only<2>{
\begin{QQuestion}{AD226}{Welchen Gütefaktor $Q$ hat die Parallelschaltung einer Spule von \qty{2,2}{\micro\H} mit einem Kondensator von \qty{56}{\pF} und einem Widerstand von \qty{1}{\kohm}?}{15}
{50}
{\textbf{\textcolor{DARCgreen}{5}}}
{0,2}
\end{QQuestion}

}
\end{frame}

\begin{frame}
\frametitle{Lösungsweg}
\begin{itemize}
  \item gegeben: $L = 2,2µH$
  \item gegeben: $C = 56pF$
  \item gegeben: $R_P = 1kΩ$
  \item gesucht: $Q$
  \end{itemize}
    \pause
    $f_0 = \frac{1}{2 \cdot \pi \cdot \sqrt{L \cdot C}} = \frac{1}{2 \cdot \pi \cdot \sqrt{2,2µH \cdot 56pF}} = 14,34MHz$
    \pause
    $B$ oder $X_L$ ausrechnen – $B$ haben wir schon vorher ausgerechnet

$B = \frac{1}{2\cdot \pi \cdot R_P \cdot C} = \frac{1}{2\cdot \pi \cdot 1kΩ \cdot 56pF} = 2,842MHz$
    \pause
    $Q = \frac{f_0}{B} = \frac{14,34MHz}{2,842MHz} = 5$



\end{frame}

\begin{frame}
\only<1>{
\begin{QQuestion}{AD229}{Welche Kopplung eines Bandfilters wird \glqq kritische Kopplung\grqq{} genannt?}{Die Kopplung, bei der die Ausgangsspannung des Bandfilters das 0{,}707-fache der Eingangsspannung erreicht.}
{Die Kopplung, bei der die Resonanzkurve des Bandfilters ihre größtmögliche Breite hat.}
{Die Kopplung, bei der die Resonanzkurve des Bandfilters eine Welligkeit von \qty{3}{\decibel} (Höcker- zu Sattelspannung) zeigt.}
{Die Kopplung, bei der die Resonanzkurve ihre größte Breite hat und dabei am Resonanzmaximum noch völlig eben ist.}
\end{QQuestion}

}
\only<2>{
\begin{QQuestion}{AD229}{Welche Kopplung eines Bandfilters wird \glqq kritische Kopplung\grqq{} genannt?}{Die Kopplung, bei der die Ausgangsspannung des Bandfilters das 0{,}707-fache der Eingangsspannung erreicht.}
{Die Kopplung, bei der die Resonanzkurve des Bandfilters ihre größtmögliche Breite hat.}
{Die Kopplung, bei der die Resonanzkurve des Bandfilters eine Welligkeit von \qty{3}{\decibel} (Höcker- zu Sattelspannung) zeigt.}
{\textbf{\textcolor{DARCgreen}{Die Kopplung, bei der die Resonanzkurve ihre größte Breite hat und dabei am Resonanzmaximum noch völlig eben ist.}}}
\end{QQuestion}

}
\end{frame}

\begin{frame}
\only<1>{
\begin{PQuestion}{AD227}{Das folgende Bild zeigt ein induktiv gekoppeltes Bandfilter und vier seiner möglichen Übertragungskurven (a bis d). Welche der folgenden Aussagen ist richtig?}{Bei der Kurve c ist die Kopplung loser als bei der Kurve a.}
{Bei der Kurve b ist die Kopplung loser als bei der Kurve c.}
{Bei der Kurve a ist die Kopplung loser als bei der Kurve c.}
{Bei der Kurve b ist die Kopplung loser als bei der Kurve d.}
{\DARCimage{1.0\linewidth}{184include}}\end{PQuestion}

}
\only<2>{
\begin{PQuestion}{AD227}{Das folgende Bild zeigt ein induktiv gekoppeltes Bandfilter und vier seiner möglichen Übertragungskurven (a bis d). Welche der folgenden Aussagen ist richtig?}{\textbf{\textcolor{DARCgreen}{Bei der Kurve c ist die Kopplung loser als bei der Kurve a.}}}
{Bei der Kurve b ist die Kopplung loser als bei der Kurve c.}
{Bei der Kurve a ist die Kopplung loser als bei der Kurve c.}
{Bei der Kurve b ist die Kopplung loser als bei der Kurve d.}
{\DARCimage{1.0\linewidth}{184include}}\end{PQuestion}

}
\end{frame}

\begin{frame}
\only<1>{
\begin{PQuestion}{AD228}{Das folgende Bild zeigt ein typisches ZF-Filter und vier seiner möglichen Übertragungskurven (a bis d). Welche Kurve ergibt sich bei kritischer Kopplung und welche bei überkritischer Kopplung?}{Die Kurve c zeigt kritische, die Kurve b zeigt überkritische Kopplung.}
{Die Kurve a zeigt kritische, die Kurve b zeigt überkritische Kopplung.}
{Die Kurve b zeigt kritische, die Kurve a zeigt überkritische Kopplung.}
{Die Kurve d zeigt kritische, die Kurve c zeigt überkritische Kopplung.}
{\DARCimage{1.0\linewidth}{184include}}\end{PQuestion}

}
\only<2>{
\begin{PQuestion}{AD228}{Das folgende Bild zeigt ein typisches ZF-Filter und vier seiner möglichen Übertragungskurven (a bis d). Welche Kurve ergibt sich bei kritischer Kopplung und welche bei überkritischer Kopplung?}{Die Kurve c zeigt kritische, die Kurve b zeigt überkritische Kopplung.}
{Die Kurve a zeigt kritische, die Kurve b zeigt überkritische Kopplung.}
{\textbf{\textcolor{DARCgreen}{Die Kurve b zeigt kritische, die Kurve a zeigt überkritische Kopplung.}}}
{Die Kurve d zeigt kritische, die Kurve c zeigt überkritische Kopplung.}
{\DARCimage{1.0\linewidth}{184include}}\end{PQuestion}

}
\end{frame}%ENDCONTENT


\section{Spannungsgesteuerter Oszillator (VCO)}
\label{section:oszillator_vco}
\begin{frame}%STARTCONTENT

\only<1>{
\begin{QQuestion}{AD601}{Was versteht man unter einem VCO? Ein VCO ist ein~...}{quarzstabilisierter Referenzoszillator.}
{Oszillator, der mittels eines Drehkondensators abgestimmt wird.}
{spannungsgesteuerter Oszillator.}
{variabler Quarzoszillator.}
\end{QQuestion}

}
\only<2>{
\begin{QQuestion}{AD601}{Was versteht man unter einem VCO? Ein VCO ist ein~...}{quarzstabilisierter Referenzoszillator.}
{Oszillator, der mittels eines Drehkondensators abgestimmt wird.}
{\textbf{\textcolor{DARCgreen}{spannungsgesteuerter Oszillator.}}}
{variabler Quarzoszillator.}
\end{QQuestion}

}
\end{frame}

\begin{frame}
\only<1>{
\begin{QQuestion}{AD611}{Wenn HF-Signale unerwünscht auf einen VFO zurückkoppeln, kann dies zu~...}{Frequenzsynthese führen.}
{Frequenzinstabilität führen.}
{Gegenkopplung führen.}
{Mehrwegeausbreitung führen.}
\end{QQuestion}

}
\only<2>{
\begin{QQuestion}{AD611}{Wenn HF-Signale unerwünscht auf einen VFO zurückkoppeln, kann dies zu~...}{Frequenzsynthese führen.}
{\textbf{\textcolor{DARCgreen}{Frequenzinstabilität führen.}}}
{Gegenkopplung führen.}
{Mehrwegeausbreitung führen.}
\end{QQuestion}

}
\end{frame}%ENDCONTENT


\section{Temperaturkompensation von Oszillatoren}
\label{section:oszillator_tcxo_ocxo}
\begin{frame}%STARTCONTENT

\only<1>{
\begin{QQuestion}{AF215}{Wie sollte ein bereits temperaturkompensierter VFO innerhalb eines Gerätes verbaut werden, um eine möglichst optimale Frequenzstabilität zu gewährleisten?}{Er sollte möglichst gut thermisch isoliert zu anderen Wärmequellen im Gerät sein.}
{Er sollte auf einem eigenen Kühlkörper montiert sein.}
{Er sollte auf dem gleichen Kühlkörper wie der Leistungsverstärker angebracht werden.}
{Er sollte durch einen kleinen Ventilator separat gekühlt werden. }
\end{QQuestion}

}
\only<2>{
\begin{QQuestion}{AF215}{Wie sollte ein bereits temperaturkompensierter VFO innerhalb eines Gerätes verbaut werden, um eine möglichst optimale Frequenzstabilität zu gewährleisten?}{\textbf{\textcolor{DARCgreen}{Er sollte möglichst gut thermisch isoliert zu anderen Wärmequellen im Gerät sein.}}}
{Er sollte auf einem eigenen Kühlkörper montiert sein.}
{Er sollte auf dem gleichen Kühlkörper wie der Leistungsverstärker angebracht werden.}
{Er sollte durch einen kleinen Ventilator separat gekühlt werden. }
\end{QQuestion}

}
\end{frame}

\begin{frame}
\only<1>{
\begin{QQuestion}{AD602}{Unter einem TCXO versteht man einen~...}{temperaturkompensierten Quarzoszillator.}
{kapazitiv abgestimmten Quarzoszillator.}
{temperaturkompensierten LC-Oszillator.}
{Oszillator, der auf konstanter Temperatur gehalten wird.}
\end{QQuestion}

}
\only<2>{
\begin{QQuestion}{AD602}{Unter einem TCXO versteht man einen~...}{\textbf{\textcolor{DARCgreen}{temperaturkompensierten Quarzoszillator.}}}
{kapazitiv abgestimmten Quarzoszillator.}
{temperaturkompensierten LC-Oszillator.}
{Oszillator, der auf konstanter Temperatur gehalten wird.}
\end{QQuestion}

}
\end{frame}

\begin{frame}
\only<1>{
\begin{QQuestion}{AD603}{Wie nennt man einen temperaturkompensierten Quarzoszillator?}{VCO}
{OCXO}
{TCXO}
{VFO}
\end{QQuestion}

}
\only<2>{
\begin{QQuestion}{AD603}{Wie nennt man einen temperaturkompensierten Quarzoszillator?}{VCO}
{OCXO}
{\textbf{\textcolor{DARCgreen}{TCXO}}}
{VFO}
\end{QQuestion}

}
\end{frame}

\begin{frame}
\only<1>{
\begin{QQuestion}{AD605}{Welcher der angegebenen Oszillatoren hat die größte Frequenzstabilität?}{VCO}
{TCXO}
{OCXO}
{XO}
\end{QQuestion}

}
\only<2>{
\begin{QQuestion}{AD605}{Welcher der angegebenen Oszillatoren hat die größte Frequenzstabilität?}{VCO}
{TCXO}
{\textbf{\textcolor{DARCgreen}{OCXO}}}
{XO}
\end{QQuestion}

}
\end{frame}

\begin{frame}
\only<1>{
\begin{QQuestion}{AD604}{Welcher Oszillator ist für einen SSB-SDR-Sender im \qty{3}{\cm} Band geeignet?}{TCXO}
{VCO}
{LC-Oszillator}
{RC-Oszillator}
\end{QQuestion}

}
\only<2>{
\begin{QQuestion}{AD604}{Welcher Oszillator ist für einen SSB-SDR-Sender im \qty{3}{\cm} Band geeignet?}{\textbf{\textcolor{DARCgreen}{TCXO}}}
{VCO}
{LC-Oszillator}
{RC-Oszillator}
\end{QQuestion}

}
\end{frame}%ENDCONTENT


\section{GPS-disziplinierter Oszillator}
\label{section:oszillator_gpsdo}
\begin{frame}%STARTCONTENT

\only<1>{
\begin{QQuestion}{AD606}{Welche Eigenschaften besitzt ein GPSDO?}{Er hat eine hohe Kurz- und Langzeitstabilität durch ein externes Referenzsignal.}
{Er hat eine niedrige Kurz- und hohe Langzeitstabilität durch ein externes Referenzsignal.}
{Er hat eine hohe Kurz- und niedrige Langzeitstabilität durch ein internes Referenzsignal.}
{Er hat eine hohe Kurz- und Langzeitstabilität durch ein internes Referenzsignal.}
\end{QQuestion}

}
\only<2>{
\begin{QQuestion}{AD606}{Welche Eigenschaften besitzt ein GPSDO?}{\textbf{\textcolor{DARCgreen}{Er hat eine hohe Kurz- und Langzeitstabilität durch ein externes Referenzsignal.}}}
{Er hat eine niedrige Kurz- und hohe Langzeitstabilität durch ein externes Referenzsignal.}
{Er hat eine hohe Kurz- und niedrige Langzeitstabilität durch ein internes Referenzsignal.}
{Er hat eine hohe Kurz- und Langzeitstabilität durch ein internes Referenzsignal.}
\end{QQuestion}

}
\end{frame}%ENDCONTENT


\section{Spannungsstabilität von Oszillatoren}
\label{section:oszillator_spannungsstabilitaet}
\begin{frame}%STARTCONTENT

\only<1>{
\begin{QQuestion}{AD612}{Wie sollte die Gleichspannungsversorgung eines VFOs beschaffen sein, um Rückwirkungen nachfolgender HF-Leistungsverstärkerstufen zu verhindern?}{Die durch die PA hervorgerufenen HF-Überlagerungen auf der VFO-Stromversorgung müssen mit einem Hochpass gefiltert werden.}
{Sie muss möglichst direkt an die Spannungsversorgung der PA angekoppelt werden.}
{Sie darf nicht mit der Masseleitung der PA verbunden werden.}
{Sie muss gut gefiltert und von der Spannungsversorgung der PA entkoppelt werden.}
\end{QQuestion}

}
\only<2>{
\begin{QQuestion}{AD612}{Wie sollte die Gleichspannungsversorgung eines VFOs beschaffen sein, um Rückwirkungen nachfolgender HF-Leistungsverstärkerstufen zu verhindern?}{Die durch die PA hervorgerufenen HF-Überlagerungen auf der VFO-Stromversorgung müssen mit einem Hochpass gefiltert werden.}
{Sie muss möglichst direkt an die Spannungsversorgung der PA angekoppelt werden.}
{Sie darf nicht mit der Masseleitung der PA verbunden werden.}
{\textbf{\textcolor{DARCgreen}{Sie muss gut gefiltert und von der Spannungsversorgung der PA entkoppelt werden.}}}
\end{QQuestion}

}
\end{frame}

\begin{frame}
\only<1>{
\begin{QQuestion}{AD608}{Worauf ist bei der Spannungsversorgung eines VFO zu achten?}{Stromstabilisierte Gleichspannung}
{Unmittelbare Stromzufuhr vom Gleichrichter}
{Spannungsstabilisierte Gleichspannung}
{Stabilisierte Wechselspannung}
\end{QQuestion}

}
\only<2>{
\begin{QQuestion}{AD608}{Worauf ist bei der Spannungsversorgung eines VFO zu achten?}{Stromstabilisierte Gleichspannung}
{Unmittelbare Stromzufuhr vom Gleichrichter}
{\textbf{\textcolor{DARCgreen}{Spannungsstabilisierte Gleichspannung}}}
{Stabilisierte Wechselspannung}
\end{QQuestion}

}
\end{frame}

\begin{frame}
\only<1>{
\begin{QQuestion}{AD607}{Wie sollte der VFO in einem Sender betrieben werden, damit seine Frequenz stabil bleibt?}{Er sollte mit einer unstabilisierten Wechselspannung versorgt werden. }
{Er sollte in einem verlustarmen Teflongehäuse untergebracht sein.}
{Er sollte mit einer stabilisierten Gleichspannung versorgt werden.}
{Er sollte in einem Pertinaxgehäuse untergebracht sein.}
\end{QQuestion}

}
\only<2>{
\begin{QQuestion}{AD607}{Wie sollte der VFO in einem Sender betrieben werden, damit seine Frequenz stabil bleibt?}{Er sollte mit einer unstabilisierten Wechselspannung versorgt werden. }
{Er sollte in einem verlustarmen Teflongehäuse untergebracht sein.}
{\textbf{\textcolor{DARCgreen}{Er sollte mit einer stabilisierten Gleichspannung versorgt werden.}}}
{Er sollte in einem Pertinaxgehäuse untergebracht sein.}
\end{QQuestion}

}
\end{frame}

\begin{frame}
\only<1>{
\begin{QQuestion}{AD609}{Wodurch wird \glqq Chirp\grqq{} bei Morsetelegrafie hervorgerufen?}{Durch Amplitudenänderungen des Oszillators, weil die Tastung in der falschen Stufe erfolgt.}
{Durch Betriebsspannungsänderungen des Oszillators bei der Tastung.}
{Durch zu steile Flanken des Tastsignals.}
{Durch zu schnelle Tastung der Treiberstufe.}
\end{QQuestion}

}
\only<2>{
\begin{QQuestion}{AD609}{Wodurch wird \glqq Chirp\grqq{} bei Morsetelegrafie hervorgerufen?}{Durch Amplitudenänderungen des Oszillators, weil die Tastung in der falschen Stufe erfolgt.}
{\textbf{\textcolor{DARCgreen}{Durch Betriebsspannungsänderungen des Oszillators bei der Tastung.}}}
{Durch zu steile Flanken des Tastsignals.}
{Durch zu schnelle Tastung der Treiberstufe.}
\end{QQuestion}

}
\end{frame}%ENDCONTENT


\section{Oszillatorschaltungen}
\label{section:oszillator_schaltungen}
\begin{frame}%STARTCONTENT

\only<1>{
\begin{QQuestion}{AD613}{Welche Bedingungen müssen zur Erzeugung ungedämpfter Schwingungen in Oszillatoren erfüllt sein?}{Die Schleifenverstärkung des Signalwegs im Oszillator muss kleiner als 1 sein, und das entstehende Oszillatorsignal darf auf dem Rückkopplungsweg nicht in der Phase gedreht werden.}
{Die Grenzfrequenz des verwendeten Verstärkerelements muss mindestens der Schwingfrequenz des Oszillators entsprechen, und das entstehende Eingangssignal muss über den Rückkopplungsweg wieder gegenphasig zum Eingang zurückgeführt werden.}
{Das an einem Schaltungspunkt betrachtete Oszillatorsignal muss auf dem Signalweg im Oszillator so verstärkt und phasengedreht werden, dass es wieder gleichphasig und mit mindestens der gleichen Amplitude zum selben Punkt zurückgekoppelt wird.}
{Die Schleifenverstärkung des Signalwegs im Oszillator muss größer als 1 sein, und das Ausgangssignal muss über den Rückkopplungsweg in der Phase so gedreht werden, dass es gegenphasig zum Ausgangspunkt zurückgeführt wird.}
\end{QQuestion}

}
\only<2>{
\begin{QQuestion}{AD613}{Welche Bedingungen müssen zur Erzeugung ungedämpfter Schwingungen in Oszillatoren erfüllt sein?}{Die Schleifenverstärkung des Signalwegs im Oszillator muss kleiner als 1 sein, und das entstehende Oszillatorsignal darf auf dem Rückkopplungsweg nicht in der Phase gedreht werden.}
{Die Grenzfrequenz des verwendeten Verstärkerelements muss mindestens der Schwingfrequenz des Oszillators entsprechen, und das entstehende Eingangssignal muss über den Rückkopplungsweg wieder gegenphasig zum Eingang zurückgeführt werden.}
{\textbf{\textcolor{DARCgreen}{Das an einem Schaltungspunkt betrachtete Oszillatorsignal muss auf dem Signalweg im Oszillator so verstärkt und phasengedreht werden, dass es wieder gleichphasig und mit mindestens der gleichen Amplitude zum selben Punkt zurückgekoppelt wird.}}}
{Die Schleifenverstärkung des Signalwegs im Oszillator muss größer als 1 sein, und das Ausgangssignal muss über den Rückkopplungsweg in der Phase so gedreht werden, dass es gegenphasig zum Ausgangspunkt zurückgeführt wird.}
\end{QQuestion}

}
\end{frame}

\begin{frame}
\only<1>{
\begin{PQuestion}{AD614}{ Bei dieser Schaltung handelt es sich um~...}{einen Hochfrequenzverstärker in Emitterschaltung.}
{einen Hochfrequenzverstärker in Kollektorschaltung.}
{einen kapazitiv rückgekoppelten Dreipunkt-Oszillator.}
{einen Oberton-Oszillator in Kollektorschaltung.}
{\DARCimage{1.0\linewidth}{760include}}\end{PQuestion}

}
\only<2>{
\begin{PQuestion}{AD614}{ Bei dieser Schaltung handelt es sich um~...}{einen Hochfrequenzverstärker in Emitterschaltung.}
{einen Hochfrequenzverstärker in Kollektorschaltung.}
{\textbf{\textcolor{DARCgreen}{einen kapazitiv rückgekoppelten Dreipunkt-Oszillator.}}}
{einen Oberton-Oszillator in Kollektorschaltung.}
{\DARCimage{1.0\linewidth}{760include}}\end{PQuestion}

}
\end{frame}

\begin{frame}
\only<1>{
\begin{PQuestion}{AD616}{Welche Funktion haben die beiden Kondensatoren $C_1$ und $C_2$ in der folgenden Schaltung?}{$C_1$ kompensiert die Basis-Kollektor-Kapazität und $C_2$ die Basis-Emitter-Kapazität.}
{Sie bilden in der dargestellten Audionschaltung die notwendige Rückkopplung.}
{$C_1$ stabilisiert die Basisvorspannung und $C_2$ die Emittervorspannung.}
{Sie bilden im dargestellten LC-Oszillator einen kapazitiven Spannungsteiler zur Rückkopplung.}
{\DARCimage{1.0\linewidth}{761include}}\end{PQuestion}

}
\only<2>{
\begin{PQuestion}{AD616}{Welche Funktion haben die beiden Kondensatoren $C_1$ und $C_2$ in der folgenden Schaltung?}{$C_1$ kompensiert die Basis-Kollektor-Kapazität und $C_2$ die Basis-Emitter-Kapazität.}
{Sie bilden in der dargestellten Audionschaltung die notwendige Rückkopplung.}
{$C_1$ stabilisiert die Basisvorspannung und $C_2$ die Emittervorspannung.}
{\textbf{\textcolor{DARCgreen}{Sie bilden im dargestellten LC-Oszillator einen kapazitiven Spannungsteiler zur Rückkopplung.}}}
{\DARCimage{1.0\linewidth}{761include}}\end{PQuestion}

}
\end{frame}

\begin{frame}
\only<1>{
\begin{PQuestion}{AD617}{ Bei dieser Oszillatorschaltung handelt es sich um einen kapazitiv rückgekoppelten Quarz-Oszillator in~...}{Emitterschaltung. Der Quarz wird in Parallelresonanz betrieben.}
{Kollektorschaltung. Der Quarz schwingt auf dem dritten Oberton.}
{Kollektorschaltung. Der Quarz schwingt auf seiner Grundfrequenz.}
{Emitterschaltung. Der Quarz wird in Serienresonanz betrieben.}
{\DARCimage{0.75\linewidth}{497include}}\end{PQuestion}

}
\only<2>{
\begin{PQuestion}{AD617}{ Bei dieser Oszillatorschaltung handelt es sich um einen kapazitiv rückgekoppelten Quarz-Oszillator in~...}{Emitterschaltung. Der Quarz wird in Parallelresonanz betrieben.}
{Kollektorschaltung. Der Quarz schwingt auf dem dritten Oberton.}
{\textbf{\textcolor{DARCgreen}{Kollektorschaltung. Der Quarz schwingt auf seiner Grundfrequenz.}}}
{Emitterschaltung. Der Quarz wird in Serienresonanz betrieben.}
{\DARCimage{0.75\linewidth}{497include}}\end{PQuestion}

}
\end{frame}

\begin{frame}
\only<1>{
\begin{QQuestion}{AD610}{Wie sollte ein Oszillator im Regelfall ausgangsseitig betrieben werden?}{Er sollte an eine Pufferstufe angeschlossen sein.}
{Er sollte direkt an einen HF-Leistungsverstärker angeschlossen sein.}
{Er sollte an ein passives Hochpassfilter angeschlossen sein.}
{Er sollte an ein passives Notchfilter angeschlossen sein.}
\end{QQuestion}

}
\only<2>{
\begin{QQuestion}{AD610}{Wie sollte ein Oszillator im Regelfall ausgangsseitig betrieben werden?}{\textbf{\textcolor{DARCgreen}{Er sollte an eine Pufferstufe angeschlossen sein.}}}
{Er sollte direkt an einen HF-Leistungsverstärker angeschlossen sein.}
{Er sollte an ein passives Hochpassfilter angeschlossen sein.}
{Er sollte an ein passives Notchfilter angeschlossen sein.}
\end{QQuestion}

}
\end{frame}

\begin{frame}
\only<1>{
\begin{PQuestion}{AD615}{An welchem Punkt der Schaltung sollte die HF-Ausgangsleistung ausgekoppelt werden?}{Schaltungspunkt D}
{Schaltungspunkt A}
{Schaltungspunkt B}
{Schaltungspunkt C}
{\DARCimage{1.0\linewidth}{777include}}\end{PQuestion}

}
\only<2>{
\begin{PQuestion}{AD615}{An welchem Punkt der Schaltung sollte die HF-Ausgangsleistung ausgekoppelt werden?}{\textbf{\textcolor{DARCgreen}{Schaltungspunkt D}}}
{Schaltungspunkt A}
{Schaltungspunkt B}
{Schaltungspunkt C}
{\DARCimage{1.0\linewidth}{777include}}\end{PQuestion}

}
\end{frame}

\begin{frame}
\only<1>{
\begin{PQuestion}{AD619}{Für die Messung der Oszillatorfrequenz sollte der Tastkopf hier vorzugsweise am Punkt~...}{2 angelegt werden.}
{1 angelegt werden.}
{3 angelegt werden.}
{4 angelegt werden.}
{\DARCimage{1.0\linewidth}{498include}}\end{PQuestion}

}
\only<2>{
\begin{PQuestion}{AD619}{Für die Messung der Oszillatorfrequenz sollte der Tastkopf hier vorzugsweise am Punkt~...}{2 angelegt werden.}
{1 angelegt werden.}
{3 angelegt werden.}
{\textbf{\textcolor{DARCgreen}{4 angelegt werden.}}}
{\DARCimage{1.0\linewidth}{498include}}\end{PQuestion}

}
\end{frame}

\begin{frame}
\only<1>{
\begin{PQuestion}{AD618}{Welche Auswirkung hat die Messung der Oszillatorfrequenz mit einem Tastkopf an Punkt 3?}{Die Oszillatorfrequenz verändert sich.}
{Der Transistor wird überlastet.}
{Der Quarz wird überlastet.}
{Es gibt keine Auswirkungen.}
{\DARCimage{1.0\linewidth}{498include}}\end{PQuestion}

}
\only<2>{
\begin{PQuestion}{AD618}{Welche Auswirkung hat die Messung der Oszillatorfrequenz mit einem Tastkopf an Punkt 3?}{\textbf{\textcolor{DARCgreen}{Die Oszillatorfrequenz verändert sich.}}}
{Der Transistor wird überlastet.}
{Der Quarz wird überlastet.}
{Es gibt keine Auswirkungen.}
{\DARCimage{1.0\linewidth}{498include}}\end{PQuestion}

}
\end{frame}%ENDCONTENT


\section{Direkte digitale Synthese}
\label{section:oszillator_dds}
\begin{frame}%STARTCONTENT

\only<1>{
\begin{PQuestion}{AD620}{Um welche Art von Frequenzaufbereitung handelt es sich bei der dargestellten Schaltung?}{DDS (Direct Digital Synthesis)}
{PLL (Phase Locked Loop)}
{VCO (Voltage Controlled Oszillator)}
{VFO (Variable Frequency Oszillator)}
{\DARCimage{1.0\linewidth}{438include}}\end{PQuestion}

}
\only<2>{
\begin{PQuestion}{AD620}{Um welche Art von Frequenzaufbereitung handelt es sich bei der dargestellten Schaltung?}{\textbf{\textcolor{DARCgreen}{DDS (Direct Digital Synthesis)}}}
{PLL (Phase Locked Loop)}
{VCO (Voltage Controlled Oszillator)}
{VFO (Variable Frequency Oszillator)}
{\DARCimage{1.0\linewidth}{438include}}\end{PQuestion}

}
\end{frame}%ENDCONTENT


\section{Phasenregelschleife (PLL)}
\label{section:oszillator_pll}
\begin{frame}%STARTCONTENT

\only<1>{
\begin{QQuestion}{AD701}{Welche Baugruppen muss eine Phasenregelschleife (PLL) mindestens enthalten?}{Einen Phasenvergleicher, einen Hochpass und einen Frequenzteiler}
{Einen VCO, einen Hochpass und einen Phasenvergleicher}
{Einen Phasenvergleicher, einen Tiefpass und einen Frequenzteiler}
{Einen VCO, einen Tiefpass und einen Phasenvergleicher}
\end{QQuestion}

}
\only<2>{
\begin{QQuestion}{AD701}{Welche Baugruppen muss eine Phasenregelschleife (PLL) mindestens enthalten?}{Einen Phasenvergleicher, einen Hochpass und einen Frequenzteiler}
{Einen VCO, einen Hochpass und einen Phasenvergleicher}
{Einen Phasenvergleicher, einen Tiefpass und einen Frequenzteiler}
{\textbf{\textcolor{DARCgreen}{Einen VCO, einen Tiefpass und einen Phasenvergleicher}}}
\end{QQuestion}

}
\end{frame}

\begin{frame}
\only<1>{
\begin{PQuestion}{AD702}{Welche der nachfolgenden Aussagen ist richtig, wenn die im Bild dargestellte Regelschleife in stabilem Zustand ist?}{Die Frequenzen an den Punkten A und B sind gleich.}
{Die Frequenz an Punkt A ist höher als die Frequenz an Punkt B.}
{Die Frequenzen an den Punkten A und C sind gleich.}
{Die Frequenz an Punkt B ist höher als die Frequenz an Punkt C.}
{\DARCimage{1.0\linewidth}{45include}}\end{PQuestion}

}
\only<2>{
\begin{PQuestion}{AD702}{Welche der nachfolgenden Aussagen ist richtig, wenn die im Bild dargestellte Regelschleife in stabilem Zustand ist?}{\textbf{\textcolor{DARCgreen}{Die Frequenzen an den Punkten A und B sind gleich.}}}
{Die Frequenz an Punkt A ist höher als die Frequenz an Punkt B.}
{Die Frequenzen an den Punkten A und C sind gleich.}
{Die Frequenz an Punkt B ist höher als die Frequenz an Punkt C.}
{\DARCimage{1.0\linewidth}{45include}}\end{PQuestion}

}
\end{frame}

\begin{frame}
\only<1>{
\begin{QQuestion}{AD705}{Ein Frequenzsynthesizer soll eine einstellbare Frequenz mit hoher Frequenzgenauigkeit erzeugen. Die Genauigkeit und Stabilität der Ausgangsfrequenz eines Frequenzsynthesizers wird hauptsächlich bestimmt von~...}{den Eigenschaften des eingesetzten Quarzgenerators.}
{den Eigenschaften des spannungsgesteuerten Oszillators (VCO).}
{den Eigenschaften der eingesetzten Frequenzteiler.}
{den Eigenschaften des eingesetzten Phasenvergleichers.}
\end{QQuestion}

}
\only<2>{
\begin{QQuestion}{AD705}{Ein Frequenzsynthesizer soll eine einstellbare Frequenz mit hoher Frequenzgenauigkeit erzeugen. Die Genauigkeit und Stabilität der Ausgangsfrequenz eines Frequenzsynthesizers wird hauptsächlich bestimmt von~...}{\textbf{\textcolor{DARCgreen}{den Eigenschaften des eingesetzten Quarzgenerators.}}}
{den Eigenschaften des spannungsgesteuerten Oszillators (VCO).}
{den Eigenschaften der eingesetzten Frequenzteiler.}
{den Eigenschaften des eingesetzten Phasenvergleichers.}
\end{QQuestion}

}
\end{frame}

\begin{frame}
\only<1>{
\begin{PQuestion}{AD703}{Wie groß muss bei der folgenden Schaltung die Frequenz an Punkt A sein, wenn ein Kanalabstand von \qty{12,5}{\kHz} benötigt wird?}{\qty{11,64}{\Hz}}
{\qty{25}{\kHz}}
{\qty{1,25}{\kHz}}
{\qty{12,5}{\kHz}}
{\DARCimage{1.0\linewidth}{45include}}\end{PQuestion}

}
\only<2>{
\begin{PQuestion}{AD703}{Wie groß muss bei der folgenden Schaltung die Frequenz an Punkt A sein, wenn ein Kanalabstand von \qty{12,5}{\kHz} benötigt wird?}{\qty{11,64}{\Hz}}
{\qty{25}{\kHz}}
{\qty{1,25}{\kHz}}
{\textbf{\textcolor{DARCgreen}{\qty{12,5}{\kHz}}}}
{\DARCimage{1.0\linewidth}{45include}}\end{PQuestion}

}
\end{frame}

\begin{frame}
\only<1>{
\begin{PQuestion}{AD704}{Die Frequenz an Punkt A beträgt \qty{12,5}{\kHz}. Es sollen Ausgangsfrequenzen im Bereich von \qty{12,000}{\MHz} bis \qty{14,000}{\MHz} erzeugt werden. In welchem Bereich bewegt sich dabei das Teilerverhältnis n?}{960 bis 1120}
{300 bis 857}
{960 bis 857}
{300 bis 1120}
{\DARCimage{1.0\linewidth}{45include}}\end{PQuestion}

}
\only<2>{
\begin{PQuestion}{AD704}{Die Frequenz an Punkt A beträgt \qty{12,5}{\kHz}. Es sollen Ausgangsfrequenzen im Bereich von \qty{12,000}{\MHz} bis \qty{14,000}{\MHz} erzeugt werden. In welchem Bereich bewegt sich dabei das Teilerverhältnis n?}{\textbf{\textcolor{DARCgreen}{960 bis 1120}}}
{300 bis 857}
{960 bis 857}
{300 bis 1120}
{\DARCimage{1.0\linewidth}{45include}}\end{PQuestion}

}
\end{frame}

\begin{frame}
\frametitle{Lösungsweg}
\begin{itemize}
  \item gegeben: $f_{Osc} = 12,5kHz$
  \item gegeben: $f_{Out,low} = 12,000MHz$
  \item gegeben: $f_{Out,high} = 14,000MHz$
  \item gesucht: $:n$
  \end{itemize}
    \pause
    Bei $f_{Out,low} = 12,000MHz$:

$n = \frac{f_{Out,low}}{f_{Osc}} = \frac{12,000MHz}{12,5kHz} = 960$
    \pause
    Bei $f_{Out,high} = 14,000MHz$:

$n = \frac{f_{Out,high}}{f_{Osc}} = \frac{14,000MHz}{12,5kHz} = 1120$



\end{frame}%ENDCONTENT


\section{Frequenzvervielfacher II}
\label{section:frequenzvervielfacher_2}
\begin{frame}%STARTCONTENT

\only<1>{
\begin{QQuestion}{AF311}{Nach welchem Prinzip arbeitet die analoge Frequenzvervielfachung?}{Das jeweils um plus und minus \qty{90}{\degree} phasenverschobene Signal wird einem additiven Mischer zugeführt, der die gewünschte Oberschwingungen erzeugt.}
{Das Signal wird einer nicht linearen Verzerrerstufe zugeführt und die gewünschte Oberschwingungen ausgefiltert.}
{Das Signal wird gefiltert und einem Ringmischer zugeführt, der die gewünschte Oberschwingungen erzeugt.}
{Das jeweils um plus und minus \qty{90}{\degree} phasenverschobene Signal wird einem multiplikativen Mischer zugeführt, der die gewünschte Oberschwingungen erzeugt.}
\end{QQuestion}

}
\only<2>{
\begin{QQuestion}{AF311}{Nach welchem Prinzip arbeitet die analoge Frequenzvervielfachung?}{Das jeweils um plus und minus \qty{90}{\degree} phasenverschobene Signal wird einem additiven Mischer zugeführt, der die gewünschte Oberschwingungen erzeugt.}
{\textbf{\textcolor{DARCgreen}{Das Signal wird einer nicht linearen Verzerrerstufe zugeführt und die gewünschte Oberschwingungen ausgefiltert.}}}
{Das Signal wird gefiltert und einem Ringmischer zugeführt, der die gewünschte Oberschwingungen erzeugt.}
{Das jeweils um plus und minus \qty{90}{\degree} phasenverschobene Signal wird einem multiplikativen Mischer zugeführt, der die gewünschte Oberschwingungen erzeugt.}
\end{QQuestion}

}
\end{frame}

\begin{frame}
\only<1>{
\begin{QQuestion}{AF313}{Wie sollten Frequenzvervielfacher in einer Sendeeinrichtung aufgebaut und betrieben werden?}{Sie sollten am Ausgang ein Hochpassfilter für das vervielfachte Signal besitzen.}
{Sie sollten gut abgeschirmt sein, um unerwünschte Abstrahlungen zu minimieren.}
{Sie sollten unbedingt im linearen Kennlinienabschnitt betrieben werden}
{Sie sollten sehr gut gekühlt werden.}
\end{QQuestion}

}
\only<2>{
\begin{QQuestion}{AF313}{Wie sollten Frequenzvervielfacher in einer Sendeeinrichtung aufgebaut und betrieben werden?}{Sie sollten am Ausgang ein Hochpassfilter für das vervielfachte Signal besitzen.}
{\textbf{\textcolor{DARCgreen}{Sie sollten gut abgeschirmt sein, um unerwünschte Abstrahlungen zu minimieren.}}}
{Sie sollten unbedingt im linearen Kennlinienabschnitt betrieben werden}
{Sie sollten sehr gut gekühlt werden.}
\end{QQuestion}

}
\end{frame}

\begin{frame}
\only<1>{
\begin{PQuestion}{AF312}{Worum handelt es sich bei dieser Schaltung?}{Oszillator}
{Frequenzvervielfacher}
{Frequenzteiler}
{Selbstschwingende Mischstufe}
{\DARCimage{1.0\linewidth}{489include}}\end{PQuestion}

}
\only<2>{
\begin{PQuestion}{AF312}{Worum handelt es sich bei dieser Schaltung?}{Oszillator}
{\textbf{\textcolor{DARCgreen}{Frequenzvervielfacher}}}
{Frequenzteiler}
{Selbstschwingende Mischstufe}
{\DARCimage{1.0\linewidth}{489include}}\end{PQuestion}

}
\end{frame}

\begin{frame}
\only<1>{
\begin{QQuestion}{AF314}{Ein quarzgesteuertes Funkgerät mit einer Ausgangsfrequenz von \qty{432}{\MHz} verursacht Störungen bei \qty{144}{\MHz}. Der Quarzoszillator des Funkgeräts schwingt auf einer Grundfrequenz bei \qty{12}{\MHz}.  Bei welcher Vervielfachungskombination kann die Störfrequenz von \qty{144}{\MHz} auftreten?  }{Grundfrequenz $\cdot 2 \cdot 3 \cdot 3 \cdot 2$}
{Grundfrequenz $\cdot 2 \cdot 2 \cdot 3 \cdot 3$}
{Grundfrequenz $\cdot 3 \cdot 3 \cdot 2\cdot 2$}
{Grundfrequenz $\cdot 3 \cdot 2 \cdot 3 \cdot 2$}
\end{QQuestion}

}
\only<2>{
\begin{QQuestion}{AF314}{Ein quarzgesteuertes Funkgerät mit einer Ausgangsfrequenz von \qty{432}{\MHz} verursacht Störungen bei \qty{144}{\MHz}. Der Quarzoszillator des Funkgeräts schwingt auf einer Grundfrequenz bei \qty{12}{\MHz}.  Bei welcher Vervielfachungskombination kann die Störfrequenz von \qty{144}{\MHz} auftreten?  }{Grundfrequenz $\cdot 2 \cdot 3 \cdot 3 \cdot 2$}
{\textbf{\textcolor{DARCgreen}{Grundfrequenz $\cdot 2 \cdot 2 \cdot 3 \cdot 3$}}}
{Grundfrequenz $\cdot 3 \cdot 3 \cdot 2\cdot 2$}
{Grundfrequenz $\cdot 3 \cdot 2 \cdot 3 \cdot 2$}
\end{QQuestion}

}
\end{frame}

\begin{frame}
\frametitle{Lösungsweg}
\begin{itemize}
  \item gegeben: $f_{Sender} = 432MHz$
  \item gegeben: $f_{Grund} = 12MHz$
  \item gegeben: $f_{QRM} = 144MHz$
  \item gesucht: Vervielfachungskombination
  \end{itemize}
    \pause
    $n = \frac{f_{Sender}}{f_{QRM}} = \frac{432MHz}{144MHz} = 3$
    \pause
    Es ist nur die Kombination aus $\textrm{Grundfrequenz}\,\cdot 2\cdot 2\cdot 3\cdot 3$ möglich, da diese als letzte eine Verdreifachung der Frequenz vornimmt.



\end{frame}

\begin{frame}Gegenprobe:

$$\begin{split}f_{Sender} &= f_{Grund}\cdot 2\cdot 2\cdot 3\cdot 3\\ &= 12MHz\cdot 2\cdot 2\cdot 3\cdot 3\\ &= 24MHz\cdot 2\cdot 3\cdot 3\\ &= 48MHz\cdot 3\cdot 3\\ &= \bold{144MHz}\cdot 3\\ &= 432MHz\end{split}\end{equation}

\end{frame}%ENDCONTENT


\section{Konverter und Transverter II}
\label{section:transverter_2}
\begin{frame}%STARTCONTENT

\only<1>{
\begin{QQuestion}{AF301}{Mit welchen der folgenden Baugruppen kann aus einem \qty{5,3}{\MHz}-Signal ein \qty{14,3}{\MHz}-Signal erzeugt werden?}{Ein Mischer, ein \qty{9}{\MHz}-Oszillator und ein Bandfilter.}
{Ein Vervielfacher, ein selektiver Verstärker und ein Tiefpass.}
{Ein Phasenvergleicher, ein Oberwellenmischer und ein Hochpass.}
{Ein Frequenzteiler durch 3, ein Verachtfacher und ein Notchfilter.}
\end{QQuestion}

}
\only<2>{
\begin{QQuestion}{AF301}{Mit welchen der folgenden Baugruppen kann aus einem \qty{5,3}{\MHz}-Signal ein \qty{14,3}{\MHz}-Signal erzeugt werden?}{\textbf{\textcolor{DARCgreen}{Ein Mischer, ein \qty{9}{\MHz}-Oszillator und ein Bandfilter.}}}
{Ein Vervielfacher, ein selektiver Verstärker und ein Tiefpass.}
{Ein Phasenvergleicher, ein Oberwellenmischer und ein Hochpass.}
{Ein Frequenzteiler durch 3, ein Verachtfacher und ein Notchfilter.}
\end{QQuestion}

}
\end{frame}

\begin{frame}
\only<1>{
\begin{PQuestion}{AF501}{Zwischen welchen Frequenzen muss der Quarzoszillator umschaltbar sein, damit im \qty{70}{\cm}-Bereich die oberen \qty{4}{\MHz} durch diesen Konverter empfangen werden können? Die Oszillatorfrequenz $f_{\symup{OSZ}}$ soll jeweils unterhalb des Nutzsignals liegen.}{\qty{45,111}{\MHz} und \qty{45,333}{\MHz}}
{\qty{45,556}{\MHz} und \qty{45,778}{\MHz}}
{\qty{45,333}{\MHz} und \qty{45,556}{\MHz}}
{\qty{44,889}{\MHz} und \qty{45,111}{\MHz}}
{\DARCimage{1.0\linewidth}{85include}}\end{PQuestion}

}
\only<2>{
\begin{PQuestion}{AF501}{Zwischen welchen Frequenzen muss der Quarzoszillator umschaltbar sein, damit im \qty{70}{\cm}-Bereich die oberen \qty{4}{\MHz} durch diesen Konverter empfangen werden können? Die Oszillatorfrequenz $f_{\symup{OSZ}}$ soll jeweils unterhalb des Nutzsignals liegen.}{\qty{45,111}{\MHz} und \qty{45,333}{\MHz}}
{\qty{45,556}{\MHz} und \qty{45,778}{\MHz}}
{\textbf{\textcolor{DARCgreen}{\qty{45,333}{\MHz} und \qty{45,556}{\MHz}}}}
{\qty{44,889}{\MHz} und \qty{45,111}{\MHz}}
{\DARCimage{1.0\linewidth}{85include}}\end{PQuestion}

}
\end{frame}

\begin{frame}
\frametitle{Lösungsweg}
\begin{itemize}
  \item gegeben: $\Delta f_o = 440MHz -- 30MHz = 410MHz$
  \item gegeben: $\Delta f_u = 436MHz -- 28MHz = 408MHz$
  \item gegeben: $n = 9$
  \item gesucht: $f_{Osc,1}, f_{Osc,2}$
  \end{itemize}
    \pause
    $f_{Osc,1} = \frac{\Delta f_u}{n} = \frac{408MHz}{9} = 45,333MHz$

$f_{Osc,2} = \frac{\Delta f_o}{n} = \frac{410MHz}{9} = 45,556MHz$



\end{frame}

\begin{frame}
\only<1>{
\begin{PQuestion}{AF502}{Zwischen welchen Frequenzen muss der Quarzoszillator umschaltbar sein, damit im \qty{70}{\cm}-Bereich die unteren \qty{4}{\MHz} durch diesen Konverter empfangen werden können? Die Oszillatorfrequenz $f_{\symup{OSZ}}$ soll jeweils unterhalb des Nutzsignals liegen.}{\qty{44,444}{\MHz} und 
\qty{44,667}{\MHz}}
{\qty{44,667}{\MHz} und \qty{44,889}{\MHz}}
{\qty{44,889}{\MHz} und \qty{45,111}{\MHz}}
{\qty{45,111}{\MHz} und \qty{45,333}{\MHz}}
{\DARCimage{1.0\linewidth}{86include}}\end{PQuestion}

}
\only<2>{
\begin{PQuestion}{AF502}{Zwischen welchen Frequenzen muss der Quarzoszillator umschaltbar sein, damit im \qty{70}{\cm}-Bereich die unteren \qty{4}{\MHz} durch diesen Konverter empfangen werden können? Die Oszillatorfrequenz $f_{\symup{OSZ}}$ soll jeweils unterhalb des Nutzsignals liegen.}{\qty{44,444}{\MHz} und 
\qty{44,667}{\MHz}}
{\textbf{\textcolor{DARCgreen}{\qty{44,667}{\MHz} und \qty{44,889}{\MHz}}}}
{\qty{44,889}{\MHz} und \qty{45,111}{\MHz}}
{\qty{45,111}{\MHz} und \qty{45,333}{\MHz}}
{\DARCimage{1.0\linewidth}{86include}}\end{PQuestion}

}
\end{frame}

\begin{frame}
\frametitle{Lösungsweg}
\begin{itemize}
  \item gegeben: $\Delta f_o = 434MHz -- 30MHz = 404MHz$
  \item gegeben: $\Delta f_u = 430MHz -- 28MHz = 402MHz$
  \item gegeben: $n = 9$
  \item gesucht: $f_{Osc,1}, f_{Osc,2}$
  \end{itemize}
    \pause
    $f_{Osc,1} = \frac{\Delta f_u}{n} = \frac{402MHz}{9} = 44,6667MHz$

$f_{Osc,2} = \frac{\Delta f_o}{n} = \frac{404MHz}{9} = 44,889MHz$



\end{frame}%ENDCONTENT


\section{Kollektorschaltung}
\label{section:kollektorschaltung}
\begin{frame}%STARTCONTENT

\only<1>{
\begin{PQuestion}{AD401}{Bei dieser Schaltung handelt es sich um~...}{einen Oszillator in Kollektorschaltung.}
{einen Verstärker in Emitterschaltung.}
{einen Verstärker in Kollektorschaltung.}
{einen Oszillator in Emitterschaltung.}
{\DARCimage{1.0\linewidth}{140include}}\end{PQuestion}

}
\only<2>{
\begin{PQuestion}{AD401}{Bei dieser Schaltung handelt es sich um~...}{einen Oszillator in Kollektorschaltung.}
{einen Verstärker in Emitterschaltung.}
{\textbf{\textcolor{DARCgreen}{einen Verstärker in Kollektorschaltung.}}}
{einen Oszillator in Emitterschaltung.}
{\DARCimage{1.0\linewidth}{140include}}\end{PQuestion}

}
\end{frame}

\begin{frame}
\only<1>{
\begin{QQuestion}{AD405}{Welche Phasenverschiebung tritt zwischen den sinusförmigen Ein- und Ausgangsspannungen eines Transistorverstärkers in Kollektorschaltung auf?}{\qty{0}{\degree}}
{\qty{90}{\degree}}
{\qty{180}{\degree}}
{\qty{270}{\degree}}
\end{QQuestion}

}
\only<2>{
\begin{QQuestion}{AD405}{Welche Phasenverschiebung tritt zwischen den sinusförmigen Ein- und Ausgangsspannungen eines Transistorverstärkers in Kollektorschaltung auf?}{\textbf{\textcolor{DARCgreen}{\qty{0}{\degree}}}}
{\qty{90}{\degree}}
{\qty{180}{\degree}}
{\qty{270}{\degree}}
\end{QQuestion}

}
\end{frame}

\begin{frame}
\only<1>{
\begin{PQuestion}{AD402}{Was lässt sich über die Wechselspannungsverstärkung $v_U$ und die Phasenverschiebung $\varphi$ zwischen Ausgangs- und Eingangsspannung dieser Schaltung aussagen?}{$v_U$ ist groß (z.~B. 100~... 300) und $\varphi = \qty{0}{\degree}$.}
{$v_U$ ist klein (z.~B. 0,9~... 0,98) und $\varphi = \qty{0}{\degree}$.}
{$v_U$ ist klein (z.~B. 0,9~... 0,98) und $\varphi = \qty{180}{\degree}$.}
{$v_U$ ist groß (z.~B. 100~... 300) und $\varphi = \qty{180}{\degree}$.}
{\DARCimage{1.0\linewidth}{140include}}\end{PQuestion}

}
\only<2>{
\begin{PQuestion}{AD402}{Was lässt sich über die Wechselspannungsverstärkung $v_U$ und die Phasenverschiebung $\varphi$ zwischen Ausgangs- und Eingangsspannung dieser Schaltung aussagen?}{$v_U$ ist groß (z.~B. 100~... 300) und $\varphi = \qty{0}{\degree}$.}
{\textbf{\textcolor{DARCgreen}{$v_U$ ist klein (z.~B. 0,9~... 0,98) und $\varphi = \qty{0}{\degree}$.}}}
{$v_U$ ist klein (z.~B. 0,9~... 0,98) und $\varphi = \qty{180}{\degree}$.}
{$v_U$ ist groß (z.~B. 100~... 300) und $\varphi = \qty{180}{\degree}$.}
{\DARCimage{1.0\linewidth}{140include}}\end{PQuestion}

}
\end{frame}

\begin{frame}
\only<1>{
\begin{PQuestion}{AD403}{Die Ausgangsimpedanz dieser Schaltung ist~...}{in etwa gleich der Eingangsimpedanz und hochohmig.}
{in etwa gleich der Eingangsimpedanz und niederohmig.}
{sehr hoch im Vergleich zur Eingangsimpedanz.}
{sehr niedrig im Vergleich zur Eingangsimpedanz.}
{\DARCimage{1.0\linewidth}{140include}}\end{PQuestion}

}
\only<2>{
\begin{PQuestion}{AD403}{Die Ausgangsimpedanz dieser Schaltung ist~...}{in etwa gleich der Eingangsimpedanz und hochohmig.}
{in etwa gleich der Eingangsimpedanz und niederohmig.}
{sehr hoch im Vergleich zur Eingangsimpedanz.}
{\textbf{\textcolor{DARCgreen}{sehr niedrig im Vergleich zur Eingangsimpedanz.}}}
{\DARCimage{1.0\linewidth}{140include}}\end{PQuestion}

}
\end{frame}

\begin{frame}
\only<1>{
\begin{PQuestion}{AD404}{Diese Schaltung kann unter anderem als~...}{Pufferstufe zwischen Oszillator und Last verwendet werden.}
{Spannungsverstärker mit hoher Verstärkung verwendet werden.}
{Phasenumkehrstufe verwendet werden.}
{Frequenzvervielfacher verwendet werden.}
{\DARCimage{1.0\linewidth}{140include}}\end{PQuestion}

}
\only<2>{
\begin{PQuestion}{AD404}{Diese Schaltung kann unter anderem als~...}{\textbf{\textcolor{DARCgreen}{Pufferstufe zwischen Oszillator und Last verwendet werden.}}}
{Spannungsverstärker mit hoher Verstärkung verwendet werden.}
{Phasenumkehrstufe verwendet werden.}
{Frequenzvervielfacher verwendet werden.}
{\DARCimage{1.0\linewidth}{140include}}\end{PQuestion}

}
\end{frame}%ENDCONTENT


\section{Emitterschaltung}
\label{section:emitterschaltung}
\begin{frame}%STARTCONTENT

\only<1>{
\begin{PQuestion}{AD409}{Bei dieser Schaltung handelt es sich um~...}{einen Verstärker als Emitterfolger.}
{einen Verstärker in Emitterschaltung.}
{einen Verstärker in Kollektorschaltung.}
{einen Verstärker für Gleichspannung.}
{\DARCimage{1.0\linewidth}{136include}}\end{PQuestion}

}
\only<2>{
\begin{PQuestion}{AD409}{Bei dieser Schaltung handelt es sich um~...}{einen Verstärker als Emitterfolger.}
{\textbf{\textcolor{DARCgreen}{einen Verstärker in Emitterschaltung.}}}
{einen Verstärker in Kollektorschaltung.}
{einen Verstärker für Gleichspannung.}
{\DARCimage{1.0\linewidth}{136include}}\end{PQuestion}

}
\end{frame}

\begin{frame}
\only<1>{
\begin{PQuestion}{AD411}{Welche Funktion haben die Widerstände $R_1$ und $R_2$ in der folgenden Schaltung? Sie dienen zur~...}{Einstellung der Gegenkopplung.}
{Verhinderung von Phasendrehungen.}
{Verhinderung von Eigenschwingungen.}
{Einstellung der Basisvorspannung.}
{\DARCimage{1.0\linewidth}{137include}}\end{PQuestion}

}
\only<2>{
\begin{PQuestion}{AD411}{Welche Funktion haben die Widerstände $R_1$ und $R_2$ in der folgenden Schaltung? Sie dienen zur~...}{Einstellung der Gegenkopplung.}
{Verhinderung von Phasendrehungen.}
{Verhinderung von Eigenschwingungen.}
{\textbf{\textcolor{DARCgreen}{Einstellung der Basisvorspannung.}}}
{\DARCimage{1.0\linewidth}{137include}}\end{PQuestion}

}
\end{frame}

\begin{frame}
\only<1>{
\begin{PQuestion}{AD413}{Welche Funktion hat der Kondensator $C_1$ in der folgenden Schaltung? Er dient zur~...}{Einstellung der Vorspannung am Emitter.}
{Verringerung der Wechselspannungsverstärkung.}
{Stabilisierung des Arbeitspunktes des Transistors.}
{Maximierung der Wechselspannungsverstärkung.}
{\DARCimage{1.0\linewidth}{138include}}\end{PQuestion}

}
\only<2>{
\begin{PQuestion}{AD413}{Welche Funktion hat der Kondensator $C_1$ in der folgenden Schaltung? Er dient zur~...}{Einstellung der Vorspannung am Emitter.}
{Verringerung der Wechselspannungsverstärkung.}
{Stabilisierung des Arbeitspunktes des Transistors.}
{\textbf{\textcolor{DARCgreen}{Maximierung der Wechselspannungsverstärkung.}}}
{\DARCimage{1.0\linewidth}{138include}}\end{PQuestion}

}
\end{frame}

\begin{frame}
\only<1>{
\begin{PQuestion}{AD412}{Welche Funktion haben die Kondensatoren $C_1$ und $C_2$ in der folgenden Schaltung? Sie dienen zur~...}{Festlegung der oberen Grenzfrequenz.}
{Wechselstromkopplung und Gleichspannungsentkopplung.}
{Erzeugung der erforderlichen Phasenverschiebung.}
{Anhebung niederfrequenter Signalanteile.}
{\DARCimage{1.0\linewidth}{139include}}\end{PQuestion}

}
\only<2>{
\begin{PQuestion}{AD412}{Welche Funktion haben die Kondensatoren $C_1$ und $C_2$ in der folgenden Schaltung? Sie dienen zur~...}{Festlegung der oberen Grenzfrequenz.}
{\textbf{\textcolor{DARCgreen}{Wechselstromkopplung und Gleichspannungsentkopplung.}}}
{Erzeugung der erforderlichen Phasenverschiebung.}
{Anhebung niederfrequenter Signalanteile.}
{\DARCimage{1.0\linewidth}{139include}}\end{PQuestion}

}
\end{frame}

\begin{frame}
\only<1>{
\begin{QQuestion}{AD407}{Welche Phasenverschiebung tritt zwischen den sinusförmigen Ein- und Ausgangsspannungen eines Transistorverstärkers in Emitterschaltung auf?}{\qty{0}{\degree}}
{\qty{90}{\degree}}
{\qty{180}{\degree}}
{\qty{270}{\degree}}
\end{QQuestion}

}
\only<2>{
\begin{QQuestion}{AD407}{Welche Phasenverschiebung tritt zwischen den sinusförmigen Ein- und Ausgangsspannungen eines Transistorverstärkers in Emitterschaltung auf?}{\qty{0}{\degree}}
{\qty{90}{\degree}}
{\textbf{\textcolor{DARCgreen}{\qty{180}{\degree}}}}
{\qty{270}{\degree}}
\end{QQuestion}

}
\end{frame}

\begin{frame}
\only<1>{
\begin{PQuestion}{AD408}{Das Signal $U_{\symup{E}}$ wird auf den Eingang folgender Schaltung gegeben. In welcher Antwort sind alle dargestellten Signale phasenrichtig zugeordnet?}{\DARCimage{1.0\linewidth}{218include}}
{\DARCimage{1.0\linewidth}{219include}}
{\DARCimage{1.0\linewidth}{220include}}
{\DARCimage{1.0\linewidth}{221include}}
{\DARCimage{1.0\linewidth}{222include}}\end{PQuestion}

}
\only<2>{
\begin{PQuestion}{AD408}{Das Signal $U_{\symup{E}}$ wird auf den Eingang folgender Schaltung gegeben. In welcher Antwort sind alle dargestellten Signale phasenrichtig zugeordnet?}{\textbf{\textcolor{DARCgreen}{\DARCimage{1.0\linewidth}{218include}}}}
{\DARCimage{1.0\linewidth}{219include}}
{\DARCimage{1.0\linewidth}{220include}}
{\DARCimage{1.0\linewidth}{221include}}
{\DARCimage{1.0\linewidth}{222include}}\end{PQuestion}

}
\end{frame}

\begin{frame}
\only<1>{
\begin{PQuestion}{AD406}{An den Eingang dieser Schaltung wird das folgende Signal gelegt. Welches ist ein mögliches Ausgangssignal $U_{\symup{A}}$?}{\DARCimage{1.0\linewidth}{223include}}
{\DARCimage{1.0\linewidth}{228include}}
{\DARCimage{1.0\linewidth}{225include}}
{\DARCimage{1.0\linewidth}{226include}}
{\DARCimage{1.0\linewidth}{227include}}\end{PQuestion}

}
\only<2>{
\begin{PQuestion}{AD406}{An den Eingang dieser Schaltung wird das folgende Signal gelegt. Welches ist ein mögliches Ausgangssignal $U_{\symup{A}}$?}{\textbf{\textcolor{DARCgreen}{\DARCimage{1.0\linewidth}{223include}}}}
{\DARCimage{1.0\linewidth}{228include}}
{\DARCimage{1.0\linewidth}{225include}}
{\DARCimage{1.0\linewidth}{226include}}
{\DARCimage{1.0\linewidth}{227include}}\end{PQuestion}

}
\end{frame}

\begin{frame}
\only<1>{
\begin{PQuestion}{AD414}{Wie verhält sich die Spannungsverstärkung bei der folgenden Schaltung, wenn der Kondensator $C_1$ entfernt wird?}{Sie nimmt ab.}
{Sie bleibt konstant.}
{Sie nimmt zu.}
{Sie fällt auf Null ab.}
{\DARCimage{1.0\linewidth}{138include}}\end{PQuestion}

}
\only<2>{
\begin{PQuestion}{AD414}{Wie verhält sich die Spannungsverstärkung bei der folgenden Schaltung, wenn der Kondensator $C_1$ entfernt wird?}{\textbf{\textcolor{DARCgreen}{Sie nimmt ab.}}}
{Sie bleibt konstant.}
{Sie nimmt zu.}
{Sie fällt auf Null ab.}
{\DARCimage{1.0\linewidth}{138include}}\end{PQuestion}

}
\end{frame}

\begin{frame}
\only<1>{
\begin{PQuestion}{AD415}{Bei folgender Emitterschaltung wird die Schaltung ohne den Emitterkondensator betrieben. Auf welchen Betrag sinkt die Spannungsverstärkung ungefähr?}{1}
{1/10}
{10}
{0}
{\DARCimage{1.0\linewidth}{366include}}\end{PQuestion}

}
\only<2>{
\begin{PQuestion}{AD415}{Bei folgender Emitterschaltung wird die Schaltung ohne den Emitterkondensator betrieben. Auf welchen Betrag sinkt die Spannungsverstärkung ungefähr?}{1}
{1/10}
{\textbf{\textcolor{DARCgreen}{10}}}
{0}
{\DARCimage{1.0\linewidth}{366include}}\end{PQuestion}

}
\end{frame}

\begin{frame}
\only<1>{
\begin{PQuestion}{AD410}{Was lässt sich über die Wechselspannungsverstärkung $v_U$ und die Phasenverschiebung $\varphi$ zwischen Ausgangs- und Eingangsspannung dieser Schaltung aussagen?}{$v_U$ ist klein (z.~B. 0,9~... 0,98) und $\varphi = \qty{0}{\degree}$.}
{$v_U$ ist groß (z.~B. 100~... 300) und $\varphi = \qty{0}{\degree}$}
{$v_U$ ist klein (z.~B. 0,9~... 0,98) und $\varphi = \qty{180}{\degree}$.}
{$v_U$ ist groß (z.~B. 100~... 300) und $\varphi = \qty{180}{\degree}$.}
{\DARCimage{1.0\linewidth}{136include}}\end{PQuestion}

}
\only<2>{
\begin{PQuestion}{AD410}{Was lässt sich über die Wechselspannungsverstärkung $v_U$ und die Phasenverschiebung $\varphi$ zwischen Ausgangs- und Eingangsspannung dieser Schaltung aussagen?}{$v_U$ ist klein (z.~B. 0,9~... 0,98) und $\varphi = \qty{0}{\degree}$.}
{$v_U$ ist groß (z.~B. 100~... 300) und $\varphi = \qty{0}{\degree}$}
{$v_U$ ist klein (z.~B. 0,9~... 0,98) und $\varphi = \qty{180}{\degree}$.}
{\textbf{\textcolor{DARCgreen}{$v_U$ ist groß (z.~B. 100~... 300) und $\varphi = \qty{180}{\degree}$.}}}
{\DARCimage{1.0\linewidth}{136include}}\end{PQuestion}

}
\end{frame}%ENDCONTENT


\section{Verstärkerklassen}
\label{section:verstaerker_klasse}
\begin{frame}%STARTCONTENT

\only<1>{
\begin{PQuestion}{AD416}{Das folgende Bild zeigt eine idealisierte Steuerkennlinie eines Transistors mit vier eingezeichneten Arbeitspunkten $\text{AP}_1$ bis $\text{AP}_4$.  Welcher Arbeitspunkt ist welcher Verstärkerbetriebsart zuzuordnen?}{$\text{AP}_1$ ist kein geeigneter Verstärkerarbeitspunkt, $\text{AP}_2$ entspricht A-Betrieb, $\text{AP}_3$ entspricht B-Betrieb, $\text{AP}_4$ entspricht C-Betrieb.}
{$\text{AP}_1$ ist kein geeigneter Verstärkerarbeitspunkt, $\text{AP}_2$ entspricht C-Betrieb, $\text{AP}_3$ entspricht B-Betrieb, $\text{AP}_4$ entspricht A-Betrieb.}
{$\text{AP}_1$ entspricht C-Betrieb, $\text{AP}_2$ entspricht B-Betrieb, $\text{AP}_3$ entspricht AB-Betrieb, $\text{AP}_4$ entspricht A-Betrieb.}
{$\text{AP}_1$ entspricht A-Betrieb, $\text{AP}_2$ entspricht AB-Betrieb, $\text{AP}_3$ entspricht B-Betrieb, $\text{AP}_4$ entspricht C-Betrieb.}
{\DARCimage{0.75\linewidth}{377include}}\end{PQuestion}

}
\only<2>{
\begin{PQuestion}{AD416}{Das folgende Bild zeigt eine idealisierte Steuerkennlinie eines Transistors mit vier eingezeichneten Arbeitspunkten $\text{AP}_1$ bis $\text{AP}_4$.  Welcher Arbeitspunkt ist welcher Verstärkerbetriebsart zuzuordnen?}{$\text{AP}_1$ ist kein geeigneter Verstärkerarbeitspunkt, $\text{AP}_2$ entspricht A-Betrieb, $\text{AP}_3$ entspricht B-Betrieb, $\text{AP}_4$ entspricht C-Betrieb.}
{$\text{AP}_1$ ist kein geeigneter Verstärkerarbeitspunkt, $\text{AP}_2$ entspricht C-Betrieb, $\text{AP}_3$ entspricht B-Betrieb, $\text{AP}_4$ entspricht A-Betrieb.}
{\textbf{\textcolor{DARCgreen}{$\text{AP}_1$ entspricht C-Betrieb, $\text{AP}_2$ entspricht B-Betrieb, $\text{AP}_3$ entspricht AB-Betrieb, $\text{AP}_4$ entspricht A-Betrieb.}}}
{$\text{AP}_1$ entspricht A-Betrieb, $\text{AP}_2$ entspricht AB-Betrieb, $\text{AP}_3$ entspricht B-Betrieb, $\text{AP}_4$ entspricht C-Betrieb.}
{\DARCimage{0.75\linewidth}{377include}}\end{PQuestion}

}
\end{frame}

\begin{frame}
\only<1>{
\begin{QQuestion}{AD419}{Welche Merkmale hat ein HF-Leistungsverstärker im A-Betrieb?}{Wirkungsgrad \qtyrange{80}{87}{\percent}, hoher Oberschwingungsanteil, der Ruhestrom ist null.}
{Wirkungsgrad bis zu \qty{70}{\percent}, geringer Oberschwingungsanteil, geringer bis mittlerer Ruhestrom.}
{Wirkungsgrad bis zu \qty{80}{\percent}, geringer Oberschwingungsanteil, sehr geringer Ruhestrom.}
{Wirkungsgrad ca. \qty{40}{\percent}, sehr geringer Oberschwingungsanteil, hoher Ruhestrom.}
\end{QQuestion}

}
\only<2>{
\begin{QQuestion}{AD419}{Welche Merkmale hat ein HF-Leistungsverstärker im A-Betrieb?}{Wirkungsgrad \qtyrange{80}{87}{\percent}, hoher Oberschwingungsanteil, der Ruhestrom ist null.}
{Wirkungsgrad bis zu \qty{70}{\percent}, geringer Oberschwingungsanteil, geringer bis mittlerer Ruhestrom.}
{Wirkungsgrad bis zu \qty{80}{\percent}, geringer Oberschwingungsanteil, sehr geringer Ruhestrom.}
{\textbf{\textcolor{DARCgreen}{Wirkungsgrad ca. \qty{40}{\percent}, sehr geringer Oberschwingungsanteil, hoher Ruhestrom.}}}
\end{QQuestion}

}
\end{frame}

\begin{frame}
\only<1>{
\begin{QQuestion}{AD420}{Welche Merkmale hat ein HF-Leistungsverstärker im B-Betrieb?}{Wirkungsgrad ca. \qty{40}{\percent}, sehr geringer Oberschwingungsanteil, hoher Ruhestrom.}
{Wirkungsgrad bis zu \qty{70}{\percent}, geringer Oberschwingungsanteil, geringer bis mittlerer Ruhestrom.}
{Wirkungsgrad bis zu \qty{80}{\percent}, geringer Oberschwingungsanteil, sehr geringer Ruhestrom.}
{Wirkungsgrad \qtyrange{80}{87}{\percent}, hoher Oberschwingungsanteil, der Ruhestrom ist null.}
\end{QQuestion}

}
\only<2>{
\begin{QQuestion}{AD420}{Welche Merkmale hat ein HF-Leistungsverstärker im B-Betrieb?}{Wirkungsgrad ca. \qty{40}{\percent}, sehr geringer Oberschwingungsanteil, hoher Ruhestrom.}
{Wirkungsgrad bis zu \qty{70}{\percent}, geringer Oberschwingungsanteil, geringer bis mittlerer Ruhestrom.}
{\textbf{\textcolor{DARCgreen}{Wirkungsgrad bis zu \qty{80}{\percent}, geringer Oberschwingungsanteil, sehr geringer Ruhestrom.}}}
{Wirkungsgrad \qtyrange{80}{87}{\percent}, hoher Oberschwingungsanteil, der Ruhestrom ist null.}
\end{QQuestion}

}
\end{frame}

\begin{frame}
\only<1>{
\begin{QQuestion}{AD421}{Welche Merkmale hat ein HF-Leistungsverstärker im C-Betrieb?}{Wirkungsgrad ca. \qty{40}{\percent}, sehr geringer Oberschwingungsanteil, hoher Ruhestrom.}
{Wirkungsgrad bis zu \qty{70}{\percent}, geringer Oberschwingungsanteil, geringer bis mittlerer Ruhestrom.}
{Wirkungsgrad bis zu \qty{80}{\percent}, geringer Oberschwingungsanteil, sehr geringer Ruhestrom.}
{Wirkungsgrad \qtyrange{80}{87}{\percent}, hoher Oberschwingungsanteil, der Ruhestrom ist null.}
\end{QQuestion}

}
\only<2>{
\begin{QQuestion}{AD421}{Welche Merkmale hat ein HF-Leistungsverstärker im C-Betrieb?}{Wirkungsgrad ca. \qty{40}{\percent}, sehr geringer Oberschwingungsanteil, hoher Ruhestrom.}
{Wirkungsgrad bis zu \qty{70}{\percent}, geringer Oberschwingungsanteil, geringer bis mittlerer Ruhestrom.}
{Wirkungsgrad bis zu \qty{80}{\percent}, geringer Oberschwingungsanteil, sehr geringer Ruhestrom.}
{\textbf{\textcolor{DARCgreen}{Wirkungsgrad \qtyrange{80}{87}{\percent}, hoher Oberschwingungsanteil, der Ruhestrom ist null.}}}
\end{QQuestion}

}
\end{frame}

\begin{frame}
\only<1>{
\begin{QQuestion}{AD424}{Ein HF-Leistungsverstärker im A-Betrieb wird mit einer Drainspannung von \qty{50}{\volt} und einem Drainstrom von \qty{2}{\ampere} betrieben. Wie hoch ist die zu erwartende Ausgangsleistung des Verstärkers?}{$\approx$ \qty{85}{\W}}
{$\approx$ \qty{40}{\W}}
{$\approx$ \qty{60}{\W}}
{$\approx$ \qty{75}{\W}}
\end{QQuestion}

}
\only<2>{
\begin{QQuestion}{AD424}{Ein HF-Leistungsverstärker im A-Betrieb wird mit einer Drainspannung von \qty{50}{\volt} und einem Drainstrom von \qty{2}{\ampere} betrieben. Wie hoch ist die zu erwartende Ausgangsleistung des Verstärkers?}{$\approx$ \qty{85}{\W}}
{\textbf{\textcolor{DARCgreen}{$\approx$ \qty{40}{\W}}}}
{$\approx$ \qty{60}{\W}}
{$\approx$ \qty{75}{\W}}
\end{QQuestion}

}
\end{frame}

\begin{frame}
\frametitle{Lösungsweg}
\begin{itemize}
  \item gegeben: $U=50V$
  \item gegeben: $I = 2A$
  \item gegeben: $\eta_A \approx 40\%$
  \item gesucht: $P_{ab}$
  \end{itemize}
    \pause
    $P_{zu} = U \cdot I = 50V \cdot 2A = 100W$
    \pause
    $\eta_A = \frac{P_{ab}}{P_{zu}} \Rightarrow P_{ab} = \eta_A \cdot P_{zu} = 0,4 \cdot 100W = 40W$



\end{frame}

\begin{frame}
\only<1>{
\begin{QQuestion}{AD425}{Ein HF-Leistungsverstärker im C-Betrieb wird mit einer Drainspannung von \qty{50}{\V} und einem Drainstrom von \qty{2}{\A} betrieben. Wie hoch ist die zu erwartende Ausgangsleistung des Verstärkers?}{$\approx$ \qty{85}{\W}}
{$\approx$ \qty{70}{\W}}
{$\approx$ \qty{60}{\W}}
{$\approx$ \qty{40}{\W}}
\end{QQuestion}

}
\only<2>{
\begin{QQuestion}{AD425}{Ein HF-Leistungsverstärker im C-Betrieb wird mit einer Drainspannung von \qty{50}{\V} und einem Drainstrom von \qty{2}{\A} betrieben. Wie hoch ist die zu erwartende Ausgangsleistung des Verstärkers?}{\textbf{\textcolor{DARCgreen}{$\approx$ \qty{85}{\W}}}}
{$\approx$ \qty{70}{\W}}
{$\approx$ \qty{60}{\W}}
{$\approx$ \qty{40}{\W}}
\end{QQuestion}

}
\end{frame}

\begin{frame}
\frametitle{Lösungsweg}
\begin{itemize}
  \item gegeben: $U=50V$
  \item gegeben: $I = 2A$
  \item gegeben: $\eta_C \approx 85\%$
  \item gesucht: $P_{ab}$
  \end{itemize}
    \pause
    $P_{zu} = U \cdot I = 50V \cdot 2A = 100W$
    \pause
    $\eta_C = \frac{P_{ab}}{P_{zu}} \Rightarrow P_{ab} = \eta_C \cdot P_{zu} = 0,85 \cdot 100W = 85W$



\end{frame}

\begin{frame}
\only<1>{
\begin{QQuestion}{AD418}{In welcher Größenordnung liegt der Ruhestrom eines HF-Leistungsverstärkers im C-Betrieb?}{Bei etwa \qtyrange{70}{80}{\percent} des Stromes bei Nennleistung}
{Bei etwa \qtyrange{10}{20}{\percent} des Stromes bei Nennleistung}
{Bei null Ampere}
{Bei fast \qty{100}{\percent} des Stromes bei Nennleistung}
\end{QQuestion}

}
\only<2>{
\begin{QQuestion}{AD418}{In welcher Größenordnung liegt der Ruhestrom eines HF-Leistungsverstärkers im C-Betrieb?}{Bei etwa \qtyrange{70}{80}{\percent} des Stromes bei Nennleistung}
{Bei etwa \qtyrange{10}{20}{\percent} des Stromes bei Nennleistung}
{\textbf{\textcolor{DARCgreen}{Bei null Ampere}}}
{Bei fast \qty{100}{\percent} des Stromes bei Nennleistung}
\end{QQuestion}

}
\end{frame}

\begin{frame}
\only<1>{
\begin{QQuestion}{AD417}{Wie verhält sich der Kollektorstrom eines NPN-Transistors in einer HF-Verstärkerstufe im B-Betrieb, wenn die Basis-Emitterspannung erhöht wird?}{Er bleibt konstant.}
{Er verringert sich geringfügig.}
{Er nimmt erheblich zu.}
{Er nimmt erheblich ab.}
\end{QQuestion}

}
\only<2>{
\begin{QQuestion}{AD417}{Wie verhält sich der Kollektorstrom eines NPN-Transistors in einer HF-Verstärkerstufe im B-Betrieb, wenn die Basis-Emitterspannung erhöht wird?}{Er bleibt konstant.}
{Er verringert sich geringfügig.}
{\textbf{\textcolor{DARCgreen}{Er nimmt erheblich zu.}}}
{Er nimmt erheblich ab.}
\end{QQuestion}

}
\end{frame}

\begin{frame}
\only<1>{
\begin{QQuestion}{AD422}{In welchem Arbeitspunkt kann ein HF-Leistungsverstärker für einen SSB-Sender betrieben werden?}{A-, AB- oder B-Betrieb}
{AB-, B- oder C-Betrieb}
{B- oder C-Betrieb}
{A-, AB-, B- oder C-Betrieb}
\end{QQuestion}

}
\only<2>{
\begin{QQuestion}{AD422}{In welchem Arbeitspunkt kann ein HF-Leistungsverstärker für einen SSB-Sender betrieben werden?}{\textbf{\textcolor{DARCgreen}{A-, AB- oder B-Betrieb}}}
{AB-, B- oder C-Betrieb}
{B- oder C-Betrieb}
{A-, AB-, B- oder C-Betrieb}
\end{QQuestion}

}
\end{frame}

\begin{frame}
\only<1>{
\begin{QQuestion}{AJ218}{In welcher Arbeitspunkteinstellung darf die Endstufe eines SSB-Senders \underline{nicht} betrieben werden?}{A-Betrieb}
{C-Betrieb}
{B-Betrieb}
{AB-Betrieb}
\end{QQuestion}

}
\only<2>{
\begin{QQuestion}{AJ218}{In welcher Arbeitspunkteinstellung darf die Endstufe eines SSB-Senders \underline{nicht} betrieben werden?}{A-Betrieb}
{\textbf{\textcolor{DARCgreen}{C-Betrieb}}}
{B-Betrieb}
{AB-Betrieb}
\end{QQuestion}

}
\end{frame}

\begin{frame}
\only<1>{
\begin{QQuestion}{AD423}{Wenn ein linearer HF-Leistungsverstärker im AB-Betrieb durch ein SSB-Signal übersteuert wird, führt dies zu~...}{Splatter auf benachbarten Frequenzen.}
{parasitären Schwingungen des Verstärkers.}
{Frequenzsprüngen in der Sendefrequenz.}
{Chirp im Sendesignal.}
\end{QQuestion}

}
\only<2>{
\begin{QQuestion}{AD423}{Wenn ein linearer HF-Leistungsverstärker im AB-Betrieb durch ein SSB-Signal übersteuert wird, führt dies zu~...}{\textbf{\textcolor{DARCgreen}{Splatter auf benachbarten Frequenzen.}}}
{parasitären Schwingungen des Verstärkers.}
{Frequenzsprüngen in der Sendefrequenz.}
{Chirp im Sendesignal.}
\end{QQuestion}

}
\end{frame}

\begin{frame}
\only<1>{
\begin{QQuestion}{AF402}{Welcher Arbeitspunkt der Leistungsverstärkerstufe eines Senders erzeugt grundsätzlich den größten Oberschwingungsanteil?}{A-Betrieb}
{B-Betrieb}
{AB-Betrieb}
{C-Betrieb}
\end{QQuestion}

}
\only<2>{
\begin{QQuestion}{AF402}{Welcher Arbeitspunkt der Leistungsverstärkerstufe eines Senders erzeugt grundsätzlich den größten Oberschwingungsanteil?}{A-Betrieb}
{B-Betrieb}
{AB-Betrieb}
{\textbf{\textcolor{DARCgreen}{C-Betrieb}}}
\end{QQuestion}

}
\end{frame}

\begin{frame}
\only<1>{
\begin{QQuestion}{AF403}{Welche Maßnahmen sind für Ausgangsanpassschaltung und Ausgangsfilter eines HF-Verstärkers im C-Betrieb vorzunehmen? Beide müssen...}{in einem gut abschirmenden Metallgehäuse untergebracht werden.}
{in einem gut isolierten Kunststoffgehäuse untergebracht werden. }
{vor dem Verstärker eingebaut werden.}
{direkt an der Antenne befestigt werden.}
\end{QQuestion}

}
\only<2>{
\begin{QQuestion}{AF403}{Welche Maßnahmen sind für Ausgangsanpassschaltung und Ausgangsfilter eines HF-Verstärkers im C-Betrieb vorzunehmen? Beide müssen...}{\textbf{\textcolor{DARCgreen}{in einem gut abschirmenden Metallgehäuse untergebracht werden.}}}
{in einem gut isolierten Kunststoffgehäuse untergebracht werden. }
{vor dem Verstärker eingebaut werden.}
{direkt an der Antenne befestigt werden.}
\end{QQuestion}

}
\end{frame}%ENDCONTENT


\section{Verstärkungsleistung}
\label{section:verstaerkungsleistung}
\begin{frame}%STARTCONTENT

\only<1>{
\begin{QQuestion}{AD427}{Ein NF-Verstärker hebt die Eingangsspannung von \qty{1}{\mV} auf \qty{4}{\mV} Ausgangsspannung an. Eingangs- und Ausgangswiderstand sind gleich. Wie groß ist die Spannungsverstärkung des Verstärkers?}{\qty{6}{\decibel}}
{\qty{3}{\decibel}}
{\qty{12}{\decibel}}
{\qty{9}{\decibel}}
\end{QQuestion}

}
\only<2>{
\begin{QQuestion}{AD427}{Ein NF-Verstärker hebt die Eingangsspannung von \qty{1}{\mV} auf \qty{4}{\mV} Ausgangsspannung an. Eingangs- und Ausgangswiderstand sind gleich. Wie groß ist die Spannungsverstärkung des Verstärkers?}{\qty{6}{\decibel}}
{\qty{3}{\decibel}}
{\textbf{\textcolor{DARCgreen}{\qty{12}{\decibel}}}}
{\qty{9}{\decibel}}
\end{QQuestion}

}
\end{frame}

\begin{frame}
\frametitle{Lösungsweg}
\begin{itemize}
  \item gegeben: $U_1 = 1mV$
  \item gegeben: $U_2 = 4mV$
  \item gesucht: $g$
  \end{itemize}
    \pause
    $g = 20\cdot \log_{10}{(\frac{U_2}{U_1})}dB = 20\cdot \log_{10}{(\frac{4mV}{1mV})}dB = 12dB$



\end{frame}

\begin{frame}
\only<1>{
\begin{QQuestion}{AD428}{Ein Leistungsverstärker hebt die Eingangsleistung von \qty{2,5}{\W} auf \qty{38}{\W} Ausgangsleistung an. Dem entspricht eine Leistungsverstärkung von~...}{\qty{23,6}{\decibel}.}
{\qty{15,2}{\decibel}.}
{\qty{17,7}{\decibel}.}
{\qty{11,8}{\decibel}.}
\end{QQuestion}

}
\only<2>{
\begin{QQuestion}{AD428}{Ein Leistungsverstärker hebt die Eingangsleistung von \qty{2,5}{\W} auf \qty{38}{\W} Ausgangsleistung an. Dem entspricht eine Leistungsverstärkung von~...}{\qty{23,6}{\decibel}.}
{\qty{15,2}{\decibel}.}
{\qty{17,7}{\decibel}.}
{\textbf{\textcolor{DARCgreen}{\qty{11,8}{\decibel}.}}}
\end{QQuestion}

}
\end{frame}

\begin{frame}
\frametitle{Lösungsweg}
\begin{itemize}
  \item gegeben: $P_1 = 38W$
  \item gegeben: $P_2 = 2,5W$
  \item gesucht: $g$
  \end{itemize}
    \pause
    $g = 10\cdot \log_{10}{(\frac{P_2}{P_1})}dB = 10\cdot \log_{10}{(\frac{38W}{2,5W})}dB = 11,8dB$



\end{frame}

\begin{frame}
\only<1>{
\begin{QQuestion}{AD426}{Ein HF-Leistungsverstärker hat eine Verstärkung von \qty{16}{\decibel}. Welche HF-Ausgangsleistung ist zu erwarten, wenn der Verstärker mit \qty{1}{\W} HF-Eingangsleistung angesteuert wird?}{\qty{80}{\W}}
{\qty{40}{\W}}
{\qty{16}{\W}}
{\qty{20}{\W}}
\end{QQuestion}

}
\only<2>{
\begin{QQuestion}{AD426}{Ein HF-Leistungsverstärker hat eine Verstärkung von \qty{16}{\decibel}. Welche HF-Ausgangsleistung ist zu erwarten, wenn der Verstärker mit \qty{1}{\W} HF-Eingangsleistung angesteuert wird?}{\qty{80}{\W}}
{\textbf{\textcolor{DARCgreen}{\qty{40}{\W}}}}
{\qty{16}{\W}}
{\qty{20}{\W}}
\end{QQuestion}

}
\end{frame}

\begin{frame}
\frametitle{Lösungsweg}
\begin{itemize}
  \item gegeben: $g = 16dB$
  \item gegeben: $P_1 = 1W$
  \item gesucht: $P_2$
  \end{itemize}
    \pause
    $g = 16dB = 10dB + 6dB = 10 \cdot 4 = 40$
    \pause
    $P_2 = P_1 \cdot g = 1W \cdot 40 = 40W$



\end{frame}%ENDCONTENT


\section{Wirkungsgrad}
\label{section:verstaerker_wirkungsgrad}
\begin{frame}%STARTCONTENT

\only<1>{
\begin{QQuestion}{AD430}{Ein HF-Verstärker ist an eine \qty{12,5}{\V}-Gleichstrom-Versorgung angeschlossen. Wenn die HF-Ausgangsleistung des Verstärkers \qty{90}{\W} beträgt, zeigt das an die Stromversorgung angeschlossene Strommessgerät \qty{16}{\A} an. Der Wirkungsgrad des Verstärkers beträgt~...}{\qty{45}{\percent}.}
{\qty{55}{\percent}.}
{\qty{100}{\percent}.}
{\qty{222}{\percent}.}
\end{QQuestion}

}
\only<2>{
\begin{QQuestion}{AD430}{Ein HF-Verstärker ist an eine \qty{12,5}{\V}-Gleichstrom-Versorgung angeschlossen. Wenn die HF-Ausgangsleistung des Verstärkers \qty{90}{\W} beträgt, zeigt das an die Stromversorgung angeschlossene Strommessgerät \qty{16}{\A} an. Der Wirkungsgrad des Verstärkers beträgt~...}{\textbf{\textcolor{DARCgreen}{\qty{45}{\percent}.}}}
{\qty{55}{\percent}.}
{\qty{100}{\percent}.}
{\qty{222}{\percent}.}
\end{QQuestion}

}
\end{frame}

\begin{frame}
\frametitle{Lösungsweg}
\begin{itemize}
  \item gegeben: $U = 12,5V$
  \item gegeben: $I = 16A$
  \item gegeben: $P_{ab} = 90W$
  \item gesucht: $\eta$
  \end{itemize}
    \pause
    $P_{zu} = U \cdot I = 12,5V \cdot 16A = 200W$
    \pause
    $\eta = \frac{P_{ab}}{P_{zu}} = \frac{90W}{200W} = 45\%$



\end{frame}

\begin{frame}
\only<1>{
\begin{QQuestion}{AD429}{Eine Treiberstufe eines HF-Verstärkers braucht am Eingang eine Leistung von \qty{1}{\W}, um am Ausgang \qty{10}{\W} an die Endstufe abgeben zu können. Sie benötigt dazu eine Gleichstromleistung von \qty{25}{\W}. Wie hoch ist der Wirkungsgrad der Treiberstufe?}{\qty{10}{\percent}}
{\qty{25}{\percent}}
{\qty{15}{\percent}}
{\qty{40}{\percent}}
\end{QQuestion}

}
\only<2>{
\begin{QQuestion}{AD429}{Eine Treiberstufe eines HF-Verstärkers braucht am Eingang eine Leistung von \qty{1}{\W}, um am Ausgang \qty{10}{\W} an die Endstufe abgeben zu können. Sie benötigt dazu eine Gleichstromleistung von \qty{25}{\W}. Wie hoch ist der Wirkungsgrad der Treiberstufe?}{\qty{10}{\percent}}
{\qty{25}{\percent}}
{\qty{15}{\percent}}
{\textbf{\textcolor{DARCgreen}{\qty{40}{\percent}}}}
\end{QQuestion}

}
\end{frame}

\begin{frame}
\frametitle{Lösungsweg}
\begin{itemize}
  \item gegeben: $P_{ab} = 10W$
  \item gegeben: $P_{zu} = 25W$
  \item gesucht: $\eta$
  \end{itemize}
    \pause
    $\eta = \frac{P_{ab}}{P_{zu}} = \frac{25W}{10W} = 40\%$



\end{frame}%ENDCONTENT


\section{Linearverstärker}
\label{section:verstaerker_linearverstaerker}
\begin{frame}%STARTCONTENT

\only<1>{
\begin{QQuestion}{AD431}{Welche Eigenschaft besitzt ein Linearverstärker?}{Er ist nur für sinusförmige Signale geeignet.}
{Die Kurvenform am Ausgang entspricht der Kurvenform am Eingang.}
{Die Phasenlage zwischen Eingang und Ausgang beträgt immer \qty{180}{\degree}.}
{Die Amplitude am Ausgang entspricht der Amplitude am Eingang.}
\end{QQuestion}

}
\only<2>{
\begin{QQuestion}{AD431}{Welche Eigenschaft besitzt ein Linearverstärker?}{Er ist nur für sinusförmige Signale geeignet.}
{\textbf{\textcolor{DARCgreen}{Die Kurvenform am Ausgang entspricht der Kurvenform am Eingang.}}}
{Die Phasenlage zwischen Eingang und Ausgang beträgt immer \qty{180}{\degree}.}
{Die Amplitude am Ausgang entspricht der Amplitude am Eingang.}
\end{QQuestion}

}
\end{frame}%ENDCONTENT


\section{Eigenschwingung}
\label{section:verstaerker_eigenschwingung}
\begin{frame}%STARTCONTENT

\only<1>{
\begin{QQuestion}{AD432}{Was ist die Ursache für Eigenschwingungen eines Verstärkers?}{Unzulängliche Verstärkung}
{Kopplung zwischen Ausgang und Eingang}
{Zu hohe Restwelligkeit in der Stromversorgung}
{Unzulängliche Regelung der Stromversorgung}
\end{QQuestion}

}
\only<2>{
\begin{QQuestion}{AD432}{Was ist die Ursache für Eigenschwingungen eines Verstärkers?}{Unzulängliche Verstärkung}
{\textbf{\textcolor{DARCgreen}{Kopplung zwischen Ausgang und Eingang}}}
{Zu hohe Restwelligkeit in der Stromversorgung}
{Unzulängliche Regelung der Stromversorgung}
\end{QQuestion}

}
\end{frame}

\begin{frame}
\only<1>{
\begin{QQuestion}{AJ216}{Um die Gefahr von unerwünschten Eigenschwingungen in HF-Schaltungen zu verringern,~...}{sollte jede Stufe gut abgeschirmt sein.}
{sollten die Abschirmungen der einzelnen Stufen nicht miteinander verbunden werden.}
{sollten die Betriebsspannungen den einzelnen Stufen mit koaxialen oder verdrillten Leitungen zugeführt werden.}
{sollte die vollständige Schaltung in einem einzelnen Metallgehäuse untergebracht sein.}
\end{QQuestion}

}
\only<2>{
\begin{QQuestion}{AJ216}{Um die Gefahr von unerwünschten Eigenschwingungen in HF-Schaltungen zu verringern,~...}{\textbf{\textcolor{DARCgreen}{sollte jede Stufe gut abgeschirmt sein.}}}
{sollten die Abschirmungen der einzelnen Stufen nicht miteinander verbunden werden.}
{sollten die Betriebsspannungen den einzelnen Stufen mit koaxialen oder verdrillten Leitungen zugeführt werden.}
{sollte die vollständige Schaltung in einem einzelnen Metallgehäuse untergebracht sein.}
\end{QQuestion}

}
\end{frame}

\begin{frame}
\only<1>{
\begin{QQuestion}{AJ215}{Um die Wahrscheinlichkeit von Eigenschwingungen in einem Leistungsverstärker zu verringern,~...}{sollten die Ein- und Ausgangsschaltungen gut voneinander entkoppelt werden.}
{sollte Verstärkerausgang und Netzteil möglichst weit voneinander entfernt aufgebaut werden.}
{sollte die Versorgungsspannung über ein Netzfilter zugeführt werden.}
{sollte kein Schaltnetzteil als Stromversorgung verwendet werden.}
\end{QQuestion}

}
\only<2>{
\begin{QQuestion}{AJ215}{Um die Wahrscheinlichkeit von Eigenschwingungen in einem Leistungsverstärker zu verringern,~...}{\textbf{\textcolor{DARCgreen}{sollten die Ein- und Ausgangsschaltungen gut voneinander entkoppelt werden.}}}
{sollte Verstärkerausgang und Netzteil möglichst weit voneinander entfernt aufgebaut werden.}
{sollte die Versorgungsspannung über ein Netzfilter zugeführt werden.}
{sollte kein Schaltnetzteil als Stromversorgung verwendet werden.}
\end{QQuestion}

}
\end{frame}%ENDCONTENT


\section{Begrenzung der Verstärkerbandbreite}
\label{section:verstaerker_begrenzung_bandbreite}
\begin{frame}%STARTCONTENT

\only<1>{
\begin{QQuestion}{AD433}{Welche Baugruppe sollte für die Begrenzung der NF-Bandbreite eines Mikrofonverstärkers verwendet werden?}{Notchfilter}
{Bandpassfilter}
{Hochpassfilter}
{Amplitudenbegrenzer}
\end{QQuestion}

}
\only<2>{
\begin{QQuestion}{AD433}{Welche Baugruppe sollte für die Begrenzung der NF-Bandbreite eines Mikrofonverstärkers verwendet werden?}{Notchfilter}
{\textbf{\textcolor{DARCgreen}{Bandpassfilter}}}
{Hochpassfilter}
{Amplitudenbegrenzer}
\end{QQuestion}

}
\end{frame}%ENDCONTENT


\title{DARC Amateurfunklehrgang Klasse A}
\author{Modulation}
\institute{Deutscher Amateur Radio Club e.\,V.}
\begin{frame}
\maketitle
\end{frame}

\section{Amplitudenmodulation (AM) II}
\label{section:am_2}
\begin{frame}%STARTCONTENT

\only<1>{
\begin{question2x2}{AE201}{In welcher Abbildung ist AM mit einem Modulationsgrad von \qty{100}{\percent} dargestellt?}{\DARCimage{1.0\linewidth}{24include}}
{\DARCimage{1.0\linewidth}{27include}}
{\DARCimage{1.0\linewidth}{28include}}
{\DARCimage{1.0\linewidth}{26include}}
\end{question2x2}

}
\only<2>{
\begin{question2x2}{AE201}{In welcher Abbildung ist AM mit einem Modulationsgrad von \qty{100}{\percent} dargestellt?}{\textbf{\textcolor{DARCgreen}{\DARCimage{1.0\linewidth}{24include}}}}
{\DARCimage{1.0\linewidth}{27include}}
{\DARCimage{1.0\linewidth}{28include}}
{\DARCimage{1.0\linewidth}{26include}}
\end{question2x2}

}
\end{frame}

\begin{frame}
\only<1>{
\begin{PQuestion}{AE202}{Das folgende Oszillogramm zeigt ein AM-Signal. Der Modulationsgrad beträgt hier ca.}{\qty{50}{\percent}.}
{\qty{33}{\percent}.}
{\qty{67}{\percent}.}
{\qty{75}{\percent}.}
{\DARCimage{1.0\linewidth}{78include}}\end{PQuestion}

}
\only<2>{
\begin{PQuestion}{AE202}{Das folgende Oszillogramm zeigt ein AM-Signal. Der Modulationsgrad beträgt hier ca.}{\textbf{\textcolor{DARCgreen}{\qty{50}{\percent}.}}}
{\qty{33}{\percent}.}
{\qty{67}{\percent}.}
{\qty{75}{\percent}.}
{\DARCimage{1.0\linewidth}{78include}}\end{PQuestion}

}
\end{frame}

\begin{frame}
\only<1>{
\begin{question2x2}{AE203}{Welches Bild stellt die Übermodulation eines AM-Signals dar?}{\DARCimage{1.0\linewidth}{24include}}
{\DARCimage{1.0\linewidth}{28include}}
{\DARCimage{1.0\linewidth}{25include}}
{\DARCimage{1.0\linewidth}{26include}}
\end{question2x2}

}
\only<2>{
\begin{question2x2}{AE203}{Welches Bild stellt die Übermodulation eines AM-Signals dar?}{\DARCimage{1.0\linewidth}{24include}}
{\textbf{\textcolor{DARCgreen}{\DARCimage{1.0\linewidth}{28include}}}}
{\DARCimage{1.0\linewidth}{25include}}
{\DARCimage{1.0\linewidth}{26include}}
\end{question2x2}

}
\end{frame}

\begin{frame}
\only<1>{
\begin{QQuestion}{AE204}{Um Seitenband-Splatter zu vermeiden, sollte der Modulationsgrad eines AM-Signals unter~...}{\qty{100}{\percent}~liegen.}
{\qty{50}{\percent}~liegen.}
{\qty{75}{\percent}~liegen.}
{\qty{25}{\percent}~liegen.}
\end{QQuestion}

}
\only<2>{
\begin{QQuestion}{AE204}{Um Seitenband-Splatter zu vermeiden, sollte der Modulationsgrad eines AM-Signals unter~...}{\textbf{\textcolor{DARCgreen}{\qty{100}{\percent}~liegen.}}}
{\qty{50}{\percent}~liegen.}
{\qty{75}{\percent}~liegen.}
{\qty{25}{\percent}~liegen.}
\end{QQuestion}

}
\end{frame}%ENDCONTENT


\section{Einseitenbandmodulation (SSB) III}
\label{section:ssb_3}
\begin{frame}%STARTCONTENT

\only<1>{
\begin{QQuestion}{AE205}{Ein übermoduliertes SSB-Sendesignal führt zu~...}{Splatter-Erscheinungen.}
{Kreuzmodulation.}
{verminderten Seitenbändern.}
{überhöhtem Hub.}
\end{QQuestion}

}
\only<2>{
\begin{QQuestion}{AE205}{Ein übermoduliertes SSB-Sendesignal führt zu~...}{\textbf{\textcolor{DARCgreen}{Splatter-Erscheinungen.}}}
{Kreuzmodulation.}
{verminderten Seitenbändern.}
{überhöhtem Hub.}
\end{QQuestion}

}
\end{frame}

\begin{frame}
\only<1>{
\begin{PQuestion}{AE207}{Das folgende Oszillogramm zeigt~...}{ein typisches \qty{100}{\percent}-AM-Signal.}
{ein typisches Einton-FM-Testsignal.}
{ein typisches Zweiton-SSB-Testsignal.}
{ein typisches CW-Signal.}
{\DARCimage{1.0\linewidth}{43include}}\end{PQuestion}

}
\only<2>{
\begin{PQuestion}{AE207}{Das folgende Oszillogramm zeigt~...}{ein typisches \qty{100}{\percent}-AM-Signal.}
{ein typisches Einton-FM-Testsignal.}
{\textbf{\textcolor{DARCgreen}{ein typisches Zweiton-SSB-Testsignal.}}}
{ein typisches CW-Signal.}
{\DARCimage{1.0\linewidth}{43include}}\end{PQuestion}

}
\end{frame}

\begin{frame}
\only<1>{
\begin{QQuestion}{AE208}{Um Bandbreite einzusparen, sollte der Frequenzumfang eines NF-Sprachsignals, das an einen SSB-Modulator angelegt wird,~...}{\qty{1,8}{\kHz} nicht überschreiten.}
{\qty{2,7}{\kHz} nicht überschreiten.}
{\qty{800}{\Hz} nicht überschreiten.}
{\qty{15}{\kHz} nicht überschreiten.}
\end{QQuestion}

}
\only<2>{
\begin{QQuestion}{AE208}{Um Bandbreite einzusparen, sollte der Frequenzumfang eines NF-Sprachsignals, das an einen SSB-Modulator angelegt wird,~...}{\qty{1,8}{\kHz} nicht überschreiten.}
{\textbf{\textcolor{DARCgreen}{\qty{2,7}{\kHz} nicht überschreiten.}}}
{\qty{800}{\Hz} nicht überschreiten.}
{\qty{15}{\kHz} nicht überschreiten.}
\end{QQuestion}

}
\end{frame}

\begin{frame}
\only<1>{
\begin{QQuestion}{AE209}{Wie groß sollte der Abstand der Sendefrequenz zwischen zwei SSB-Signalen sein, um gegenseitige Störungen in SSB-Telefonie auf ein Mindestmaß zu begrenzen?}{\qty{25}{\kHz}}
{\qty{12,5}{\kHz}}
{\qty{3}{\kHz}}
{\qty{455}{\kHz}}
\end{QQuestion}

}
\only<2>{
\begin{QQuestion}{AE209}{Wie groß sollte der Abstand der Sendefrequenz zwischen zwei SSB-Signalen sein, um gegenseitige Störungen in SSB-Telefonie auf ein Mindestmaß zu begrenzen?}{\qty{25}{\kHz}}
{\qty{12,5}{\kHz}}
{\textbf{\textcolor{DARCgreen}{\qty{3}{\kHz}}}}
{\qty{455}{\kHz}}
\end{QQuestion}

}
\end{frame}

\begin{frame}
\only<1>{
\begin{QQuestion}{AE213}{Welche Aufgabe hat der Equalizer in einem SSB-Sender?}{Er dient zur Anpassung des Mikrofonfrequenzgangs an den Operator.}
{Er dient zur Erzeugung des SSB-Signals.}
{Er dient zur Unterdrückung von Oberschwingungen im Sendesignal.}
{Er dient zur Erhöhung der Trägerunterdrückung.}
\end{QQuestion}

}
\only<2>{
\begin{QQuestion}{AE213}{Welche Aufgabe hat der Equalizer in einem SSB-Sender?}{\textbf{\textcolor{DARCgreen}{Er dient zur Anpassung des Mikrofonfrequenzgangs an den Operator.}}}
{Er dient zur Erzeugung des SSB-Signals.}
{Er dient zur Unterdrückung von Oberschwingungen im Sendesignal.}
{Er dient zur Erhöhung der Trägerunterdrückung.}
\end{QQuestion}

}
\end{frame}%ENDCONTENT


\section{Frequenzmodulation (FM) III}
\label{section:fm_3}
\begin{frame}%STARTCONTENT

\only<1>{
\begin{QQuestion}{AE301}{Wie beeinflusst die Frequenz eines sinusförmigen Modulationssignals den HF-Träger bei Frequenzmodulation?}{Wie weit sich die Trägerfrequenz ändert.}
{Wie schnell sich die Trägeramplitude ändert.}
{In welcher Häufigkeit sich der HF-Träger ändert.
}
{Wie weit sich die Trägeramplitude ändert.}
\end{QQuestion}

}
\only<2>{
\begin{QQuestion}{AE301}{Wie beeinflusst die Frequenz eines sinusförmigen Modulationssignals den HF-Träger bei Frequenzmodulation?}{Wie weit sich die Trägerfrequenz ändert.}
{Wie schnell sich die Trägeramplitude ändert.}
{\textbf{\textcolor{DARCgreen}{In welcher Häufigkeit sich der HF-Träger ändert.
}}}
{Wie weit sich die Trägeramplitude ändert.}
\end{QQuestion}

}
\end{frame}

\begin{frame}
\only<1>{
\begin{QQuestion}{AE302}{Welches der nachfolgenden Übertragungsverfahren weist die geringste Störanfälligkeit gegenüber Impulsstörungen durch Funkenbildung in Elektromotoren auf?}{AM-Sprechfunk, weil hier die wichtige Information in den Amplituden der beiden Seitenbänder enthalten ist.}
{CW-Morsetelegrafie, weil hier die wichtige Information in der Amplitude von zwei Seitenbändern liegt.}
{SSB-Sprechfunk, weil hier die wichtige Information in der Amplitude eines Seitenbandes enthalten ist.}
{FM-Sprechfunk, weil hier die wichtige Information nicht in der Amplitude enthalten ist.}
\end{QQuestion}

}
\only<2>{
\begin{QQuestion}{AE302}{Welches der nachfolgenden Übertragungsverfahren weist die geringste Störanfälligkeit gegenüber Impulsstörungen durch Funkenbildung in Elektromotoren auf?}{AM-Sprechfunk, weil hier die wichtige Information in den Amplituden der beiden Seitenbänder enthalten ist.}
{CW-Morsetelegrafie, weil hier die wichtige Information in der Amplitude von zwei Seitenbändern liegt.}
{SSB-Sprechfunk, weil hier die wichtige Information in der Amplitude eines Seitenbandes enthalten ist.}
{\textbf{\textcolor{DARCgreen}{FM-Sprechfunk, weil hier die wichtige Information nicht in der Amplitude enthalten ist.}}}
\end{QQuestion}

}
\end{frame}

\begin{frame}
\only<1>{
\begin{QQuestion}{AE303}{Eine Quarzoszillator-Schaltung mit Kapazitätsdiode ermöglicht es~...}{Zweiseitenbandmodulation zu erzeugen.}
{Frequenzmodulation zu erzeugen.}
{Einseitenbandmodulation zu erzeugen.}
{Amplitudenmodulation zu erzeugen.}
\end{QQuestion}

}
\only<2>{
\begin{QQuestion}{AE303}{Eine Quarzoszillator-Schaltung mit Kapazitätsdiode ermöglicht es~...}{Zweiseitenbandmodulation zu erzeugen.}
{\textbf{\textcolor{DARCgreen}{Frequenzmodulation zu erzeugen.}}}
{Einseitenbandmodulation zu erzeugen.}
{Amplitudenmodulation zu erzeugen.}
\end{QQuestion}

}
\end{frame}

\begin{frame}
\only<1>{
\begin{QQuestion}{AE304}{Eine zu hohe Modulationsfrequenz eines FM-Senders führt dazu,~...}{dass die Sendeendstufe übersteuert wird.}
{dass die HF-Bandbreite zu groß wird.}
{dass Verzerrungen auf Grund unerwünschter Unterdrückung der Trägerfrequenz auftreten.}
{dass Verzerrungen auf Grund gegenseitiger Auslöschung der Seitenbänder auftreten.}
\end{QQuestion}

}
\only<2>{
\begin{QQuestion}{AE304}{Eine zu hohe Modulationsfrequenz eines FM-Senders führt dazu,~...}{dass die Sendeendstufe übersteuert wird.}
{\textbf{\textcolor{DARCgreen}{dass die HF-Bandbreite zu groß wird.}}}
{dass Verzerrungen auf Grund unerwünschter Unterdrückung der Trägerfrequenz auftreten.}
{dass Verzerrungen auf Grund gegenseitiger Auslöschung der Seitenbänder auftreten.}
\end{QQuestion}

}
\end{frame}

\begin{frame}
\only<1>{
\begin{QQuestion}{AE305}{Was bewirkt die Erhöhung des Hubes eines frequenzmodulierten Senders im Empfänger?}{Eine geringere Lautstärke}
{Eine größere Sprachkomprimierung}
{Eine Verringerung des Signal-Rausch-Abstandes}
{Eine größere Lautstärke}
\end{QQuestion}

}
\only<2>{
\begin{QQuestion}{AE305}{Was bewirkt die Erhöhung des Hubes eines frequenzmodulierten Senders im Empfänger?}{Eine geringere Lautstärke}
{Eine größere Sprachkomprimierung}
{Eine Verringerung des Signal-Rausch-Abstandes}
{\textbf{\textcolor{DARCgreen}{Eine größere Lautstärke}}}
\end{QQuestion}

}
\end{frame}

\begin{frame}
\only<1>{
\begin{QQuestion}{AE306}{Eine FM-Telefonie-Aussendung mit zu großem Hub führt möglicherweise~...}{zu unerwünschter Begrenzung des Trägerfrequenzsignals.}
{zur Verminderung der Ausgangsleistung.}
{zu Nachbarkanalstörungen.}
{zur Auslöschung der Seitenbänder.}
\end{QQuestion}

}
\only<2>{
\begin{QQuestion}{AE306}{Eine FM-Telefonie-Aussendung mit zu großem Hub führt möglicherweise~...}{zu unerwünschter Begrenzung des Trägerfrequenzsignals.}
{zur Verminderung der Ausgangsleistung.}
{\textbf{\textcolor{DARCgreen}{zu Nachbarkanalstörungen.}}}
{zur Auslöschung der Seitenbänder.}
\end{QQuestion}

}
\end{frame}

\begin{frame}
\only<1>{
\begin{QQuestion}{AE307}{Zu starke Ansteuerung des Modulators führt bei Frequenzmodulation zur~...}{Überlastung des Netzteils.}
{Übersteuerung der HF-Endstufe.}
{Verzerrung des HF-Sendesignals.}
{Erhöhung der HF-Bandbreite.}
\end{QQuestion}

}
\only<2>{
\begin{QQuestion}{AE307}{Zu starke Ansteuerung des Modulators führt bei Frequenzmodulation zur~...}{Überlastung des Netzteils.}
{Übersteuerung der HF-Endstufe.}
{Verzerrung des HF-Sendesignals.}
{\textbf{\textcolor{DARCgreen}{Erhöhung der HF-Bandbreite.}}}
\end{QQuestion}

}
\end{frame}

\begin{frame}
\only<1>{
\begin{QQuestion}{AE308}{Wie groß ist die Bandbreite eines FM-Signals bei einer Modulationsfrequenz von \qty{2,7}{\kHz} und einem Hub von \qty{2,5}{\kHz} nach der Carson-Formel?}{\qty{5,5}{\kHz}}
{\qty{12,5}{\kHz}}
{\qty{10,4}{\kHz}}
{\qty{2,5}{\kHz}}
\end{QQuestion}

}
\only<2>{
\begin{QQuestion}{AE308}{Wie groß ist die Bandbreite eines FM-Signals bei einer Modulationsfrequenz von \qty{2,7}{\kHz} und einem Hub von \qty{2,5}{\kHz} nach der Carson-Formel?}{\qty{5,5}{\kHz}}
{\qty{12,5}{\kHz}}
{\textbf{\textcolor{DARCgreen}{\qty{10,4}{\kHz}}}}
{\qty{2,5}{\kHz}}
\end{QQuestion}

}
\end{frame}

\begin{frame}
\frametitle{Lösungsweg}
\begin{itemize}
  \item gegeben: $f_{mod max} = 2,7kHz$
  \item gegeben: $\Delta f_T = 2,5kHz$
  \item gesucht: $B$
  \end{itemize}
    \pause
    $B \approx 2 \cdot (\Delta f_T + f_{mod max}) = 2 \cdot (2,5kHz + 2,7kHz) = 10,4kHz$



\end{frame}

\begin{frame}
\only<1>{
\begin{QQuestion}{AE309}{Ein Träger von \qty{145}{\MHz} wird mit der NF-Frequenz von \qty{2}{\kHz} und einem Hub von \qty{1,8}{\kHz} frequenzmoduliert. Welche Bandbreite hat das modulierte Signal ungefähr? Die Bandbreite beträgt ungefähr~...}{\qty{12}{\kHz}}
{\qty{3,8}{\kHz}}
{\qty{5,8}{\kHz}}
{\qty{7,6}{\kHz}}
\end{QQuestion}

}
\only<2>{
\begin{QQuestion}{AE309}{Ein Träger von \qty{145}{\MHz} wird mit der NF-Frequenz von \qty{2}{\kHz} und einem Hub von \qty{1,8}{\kHz} frequenzmoduliert. Welche Bandbreite hat das modulierte Signal ungefähr? Die Bandbreite beträgt ungefähr~...}{\qty{12}{\kHz}}
{\qty{3,8}{\kHz}}
{\qty{5,8}{\kHz}}
{\textbf{\textcolor{DARCgreen}{\qty{7,6}{\kHz}}}}
\end{QQuestion}

}
\end{frame}

\begin{frame}
\frametitle{Lösungsweg}
\begin{itemize}
  \item gegeben: $f_{mod max} = 2kHz$
  \item gegeben: $\Delta f_T = 1,8kHz$
  \item gesucht: $B$
  \end{itemize}
    \pause
    $B \approx 2 \cdot (\Delta f_T + f_{mod max}) = 2 \cdot (1,8kHz + 2kHz) = 7,6kHz$



\end{frame}

\begin{frame}
\only<1>{
\begin{QQuestion}{AE310}{Der typische Spitzenhub eines NBFM-Signals im \qty{12,5}{\kHz} Kanalraster beträgt~...}{\qty{25}{\kHz}.}
{\qty{2,5}{\kHz}.}
{\qty{6,25}{\kHz}.}
{\qty{12,5}{\kHz}.}
\end{QQuestion}

}
\only<2>{
\begin{QQuestion}{AE310}{Der typische Spitzenhub eines NBFM-Signals im \qty{12,5}{\kHz} Kanalraster beträgt~...}{\qty{25}{\kHz}.}
{\textbf{\textcolor{DARCgreen}{\qty{2,5}{\kHz}.}}}
{\qty{6,25}{\kHz}.}
{\qty{12,5}{\kHz}.}
\end{QQuestion}

}
\end{frame}

\begin{frame}
\only<1>{
\begin{QQuestion}{AE311}{Die Bandbreite eines FM-Signals soll \qty{10}{\kHz} nicht überschreiten. Der Hub beträgt \qty{2,5}{\kHz}. Wie groß ist dabei die höchste Modulationsfrequenz?}{\qty{3}{\kHz}}
{\qty{1,5}{\kHz}}
{\qty{2,5}{\kHz}}
{\qty{2}{\kHz}}
\end{QQuestion}

}
\only<2>{
\begin{QQuestion}{AE311}{Die Bandbreite eines FM-Signals soll \qty{10}{\kHz} nicht überschreiten. Der Hub beträgt \qty{2,5}{\kHz}. Wie groß ist dabei die höchste Modulationsfrequenz?}{\qty{3}{\kHz}}
{\qty{1,5}{\kHz}}
{\textbf{\textcolor{DARCgreen}{\qty{2,5}{\kHz}}}}
{\qty{2}{\kHz}}
\end{QQuestion}

}
\end{frame}

\begin{frame}
\frametitle{Lösungsweg}
\begin{itemize}
  \item gegeben: $B = 10kHz$
  \item gegeben: $\Delta f_T = 2,5kHz$
  \item gesucht: $f_{mod max}$
  \end{itemize}
    \pause
    $B \approx 2 \cdot (\Delta f_T + f_{mod max}) \Rightarrow f_{mod max} = \frac{B}{2} -- \Delta f_T = \frac{10kHz}{2} -- 2,5kHz = 2,5kHz$



\end{frame}

\begin{frame}
\only<1>{
\begin{QQuestion}{AE312}{Die Bandbreite eines FM-Senders soll \qty{10}{\kHz} nicht überschreiten. Wie hoch darf der Frequenzhub bei einer Modulationsfrequenz von \qty{2,7}{\kHz} maximal sein?}{\qty{4,6}{\kHz}}
{\qty{7,7}{\kHz}}
{\qty{2,3}{\kHz}}
{\qty{12,7}{\kHz}}
\end{QQuestion}

}
\only<2>{
\begin{QQuestion}{AE312}{Die Bandbreite eines FM-Senders soll \qty{10}{\kHz} nicht überschreiten. Wie hoch darf der Frequenzhub bei einer Modulationsfrequenz von \qty{2,7}{\kHz} maximal sein?}{\qty{4,6}{\kHz}}
{\qty{7,7}{\kHz}}
{\textbf{\textcolor{DARCgreen}{\qty{2,3}{\kHz}}}}
{\qty{12,7}{\kHz}}
\end{QQuestion}

}
\end{frame}

\begin{frame}
\frametitle{Lösungsweg}
\begin{itemize}
  \item gegeben: $B = 10kHz$
  \item gegeben: $f_{mod max} = 2,7kHz$
  \item gesucht: $\Delta f_T$
  \end{itemize}
    \pause
    $B \approx 2 \cdot (\Delta f_T + f_{mod max}) \Rightarrow \Delta f_T = \frac{B}{2} -- f_{mod max} = \frac{10kHz}{2} -- 2,7kHz = 2,3kHz$



\end{frame}%ENDCONTENT


\section{Phasenmodulation (PM)}
\label{section:pm}
\begin{frame}%STARTCONTENT

\only<1>{
\begin{QQuestion}{AE313}{Welche Antwort beschreibt die Modulationsart \glqq PM\grqq{}?}{Die Phase eines Trägersignals wird anhand eines zu übertragenden Signals verändert.}
{Die Amplitude eines Trägersignals wird anhand eines zu übertragenden Signals verändert.}
{Die Polarisation eines Trägersignals wird anhand eines zu übertragenden Signals verändert.}
{Die Richtung eines Trägersignals wird anhand eines zu übertragenden Signals verändert.}
\end{QQuestion}

}
\only<2>{
\begin{QQuestion}{AE313}{Welche Antwort beschreibt die Modulationsart \glqq PM\grqq{}?}{\textbf{\textcolor{DARCgreen}{Die Phase eines Trägersignals wird anhand eines zu übertragenden Signals verändert.}}}
{Die Amplitude eines Trägersignals wird anhand eines zu übertragenden Signals verändert.}
{Die Polarisation eines Trägersignals wird anhand eines zu übertragenden Signals verändert.}
{Die Richtung eines Trägersignals wird anhand eines zu übertragenden Signals verändert.}
\end{QQuestion}

}
\end{frame}%ENDCONTENT


\section{Bandbreite III}
\label{section:bandreite_3}
\begin{frame}%STARTCONTENT

\only<1>{
\begin{PQuestion}{AE101}{Welcher Wert ist in folgender Aussage für X einzusetzen? Die \glqq belegte Bandbreite\grqq{} ist gemäß der Amateurfunkverordnung die Frequenzbandbreite, bei der die unterhalb ihrer unteren und oberhalb ihrer oberen Frequenzgrenzen ausgesendeten mittleren Leistungen jeweils X an der gesamten mittleren Leistung betragen.}{\qty{0,5}{\percent}}
{\qty{1}{\percent}}
{\qty{5}{\percent}}
{\qty{10}{\percent}}
{\DARCimage{1.0\linewidth}{612include}}\end{PQuestion}

}
\only<2>{
\begin{PQuestion}{AE101}{Welcher Wert ist in folgender Aussage für X einzusetzen? Die \glqq belegte Bandbreite\grqq{} ist gemäß der Amateurfunkverordnung die Frequenzbandbreite, bei der die unterhalb ihrer unteren und oberhalb ihrer oberen Frequenzgrenzen ausgesendeten mittleren Leistungen jeweils X an der gesamten mittleren Leistung betragen.}{\textbf{\textcolor{DARCgreen}{\qty{0,5}{\percent}}}}
{\qty{1}{\percent}}
{\qty{5}{\percent}}
{\qty{10}{\percent}}
{\DARCimage{1.0\linewidth}{612include}}\end{PQuestion}

}
\end{frame}%ENDCONTENT


\section{Dynamikkompressor II}
\label{section:dynamik_compressor_2}
\begin{frame}%STARTCONTENT

\only<1>{
\begin{QQuestion}{AE211}{Welche Aufgabe hat der Dynamik-Kompressor in einem SSB-Sender?}{Die Reichweite in CW wird erhöht.}
{Die mittlere Sendeleistung wird abgesenkt.}
{Der Dynamikbereich des Modulationssignals wird erhöht.}
{Die mittlere Sendeleistung wird verzerrungsarm angehoben.}
\end{QQuestion}

}
\only<2>{
\begin{QQuestion}{AE211}{Welche Aufgabe hat der Dynamik-Kompressor in einem SSB-Sender?}{Die Reichweite in CW wird erhöht.}
{Die mittlere Sendeleistung wird abgesenkt.}
{Der Dynamikbereich des Modulationssignals wird erhöht.}
{\textbf{\textcolor{DARCgreen}{Die mittlere Sendeleistung wird verzerrungsarm angehoben.}}}
\end{QQuestion}

}
\end{frame}

\begin{frame}
\only<1>{
\begin{QQuestion}{AE212}{Welche Folge hat eine zu hohe Kompressionseinstellung des Dynamik-Kompressors im SSB-Sender?}{Die Trägerunterdrückung nimmt ab.}
{Die Verständlichkeit des Audiosignals auf der Empfängerseite nimmt ab.}
{Die Modulation des Senders führt zur Zerstörung der Endstufe.}
{Das Signal kann im Empfänger nicht demoduliert werden.}
\end{QQuestion}

}
\only<2>{
\begin{QQuestion}{AE212}{Welche Folge hat eine zu hohe Kompressionseinstellung des Dynamik-Kompressors im SSB-Sender?}{Die Trägerunterdrückung nimmt ab.}
{\textbf{\textcolor{DARCgreen}{Die Verständlichkeit des Audiosignals auf der Empfängerseite nimmt ab.}}}
{Die Modulation des Senders führt zur Zerstörung der Endstufe.}
{Das Signal kann im Empfänger nicht demoduliert werden.}
\end{QQuestion}

}
\end{frame}

\begin{frame}
\only<1>{
\begin{QQuestion}{AE210}{Was versteht man unter einem NF-Dynamik-Kompressor?}{Signalprozessor zur Abtastung des HF-Signals}
{Sprachprozessor zur Erhöhung des Dynamikumfangs in der Modulation}
{Sprachprozessor zur Verringerung des Dynamikumfangs in der Modulation}
{Signalprozessor zur Abtastung des ZF-Signals}
\end{QQuestion}

}
\only<2>{
\begin{QQuestion}{AE210}{Was versteht man unter einem NF-Dynamik-Kompressor?}{Signalprozessor zur Abtastung des HF-Signals}
{Sprachprozessor zur Erhöhung des Dynamikumfangs in der Modulation}
{\textbf{\textcolor{DARCgreen}{Sprachprozessor zur Verringerung des Dynamikumfangs in der Modulation}}}
{Signalprozessor zur Abtastung des ZF-Signals}
\end{QQuestion}

}
\end{frame}%ENDCONTENT


\title{DARC Amateurfunklehrgang Klasse A}
\author{Empfänger}
\institute{Deutscher Amateur Radio Club e.\,V.}
\begin{frame}
\maketitle
\end{frame}

\section{Überlagerungsempfänger (Einfachsuper) II}
\label{section:ueberlagerungsempfaenger_einfachsuper_2}
\begin{frame}%STARTCONTENT

\only<1>{
\begin{QQuestion}{AF115}{Wodurch wird die Nahselektion eines Superhet-Empfängers bestimmt?}{Durch die ZF-Verstärkung}
{Durch den Bandpass auf der Empfangsfrequenz}
{Durch die ZF-Filter}
{Durch den Empfangsvorverstärker}
\end{QQuestion}

}
\only<2>{
\begin{QQuestion}{AF115}{Wodurch wird die Nahselektion eines Superhet-Empfängers bestimmt?}{Durch die ZF-Verstärkung}
{Durch den Bandpass auf der Empfangsfrequenz}
{\textbf{\textcolor{DARCgreen}{Durch die ZF-Filter}}}
{Durch den Empfangsvorverstärker}
\end{QQuestion}

}
\end{frame}%ENDCONTENT


\section{Mischer II}
\label{section:mischer_2}
\begin{frame}%STARTCONTENT

\only<1>{
\begin{QQuestion}{AF212}{In welchem Bereich der Steuerkennlinie arbeitet die Mischstufe eines Überlagerungsempfängers? }{Sie arbeitet im nichtlinearen Bereich.}
{Sie arbeitet im kapazitiven Bereich.}
{Sie arbeitet im induktiven Bereich.}
{Sie arbeitet im linearen Bereich.}
\end{QQuestion}

}
\only<2>{
\begin{QQuestion}{AF212}{In welchem Bereich der Steuerkennlinie arbeitet die Mischstufe eines Überlagerungsempfängers? }{\textbf{\textcolor{DARCgreen}{Sie arbeitet im nichtlinearen Bereich.}}}
{Sie arbeitet im kapazitiven Bereich.}
{Sie arbeitet im induktiven Bereich.}
{Sie arbeitet im linearen Bereich.}
\end{QQuestion}

}
\end{frame}

\begin{frame}
\only<1>{
\begin{QQuestion}{AF213}{Durch welchen Mischer werden unerwünschte Ausgangssignale am stärksten unterdrückt?}{additiver Diodenmischer}
{Balancemischer}
{Dualtransistormischer}
{Doppeldiodenmischer}
\end{QQuestion}

}
\only<2>{
\begin{QQuestion}{AF213}{Durch welchen Mischer werden unerwünschte Ausgangssignale am stärksten unterdrückt?}{additiver Diodenmischer}
{\textbf{\textcolor{DARCgreen}{Balancemischer}}}
{Dualtransistormischer}
{Doppeldiodenmischer}
\end{QQuestion}

}
\end{frame}

\begin{frame}
\only<1>{
\begin{QQuestion}{AF214}{Welche Mischerschaltung unterdrückt am wirksamsten unerwünschte Mischprodukte und Frequenzen?}{Ein Eintakt-Transistormischer}
{Ein unbalancierter Produktdetektor}
{Ein balancierter Ringmischer}
{Ein additiver Diodenmischer}
\end{QQuestion}

}
\only<2>{
\begin{QQuestion}{AF214}{Welche Mischerschaltung unterdrückt am wirksamsten unerwünschte Mischprodukte und Frequenzen?}{Ein Eintakt-Transistormischer}
{Ein unbalancierter Produktdetektor}
{\textbf{\textcolor{DARCgreen}{Ein balancierter Ringmischer}}}
{Ein additiver Diodenmischer}
\end{QQuestion}

}
\end{frame}%ENDCONTENT


\section{Spiegelfrequenzen}
\label{section:spiegelfrequenzen}
\begin{frame}%STARTCONTENT

\only<1>{
\begin{PQuestion}{AF201}{Welche Differenz liegt zwischen der HF-Nutzfrequenz und der Spiegelfrequenz?}{Das Doppelte der HF-Nutzfrequenz}
{Das Doppelte der ZF}
{Das Dreifache der ZF}
{Die HF-Nutzfrequenz plus der ZF}
{\DARCimage{1.0\linewidth}{80include}}\end{PQuestion}

}
\only<2>{
\begin{PQuestion}{AF201}{Welche Differenz liegt zwischen der HF-Nutzfrequenz und der Spiegelfrequenz?}{Das Doppelte der HF-Nutzfrequenz}
{\textbf{\textcolor{DARCgreen}{Das Doppelte der ZF}}}
{Das Dreifache der ZF}
{Die HF-Nutzfrequenz plus der ZF}
{\DARCimage{1.0\linewidth}{80include}}\end{PQuestion}

}
\end{frame}

\begin{frame}
\only<1>{
\begin{PQuestion}{AF202}{Der VCO schwingt auf \qty{134,9}{\MHz}. Die Empfangsfrequenz soll \qty{145,6}{\MHz} betragen. Welche Spiegelfrequenz kann Störungen beim Empfang verursachen?}{\qty{280,5}{\MHz} }
{\qty{134,9}{\MHz}}
{\qty{124,2}{\MHz}}
{\qty{156,3}{\MHz}}
{\DARCimage{1.0\linewidth}{80include}}\end{PQuestion}

}
\only<2>{
\begin{PQuestion}{AF202}{Der VCO schwingt auf \qty{134,9}{\MHz}. Die Empfangsfrequenz soll \qty{145,6}{\MHz} betragen. Welche Spiegelfrequenz kann Störungen beim Empfang verursachen?}{\qty{280,5}{\MHz} }
{\qty{134,9}{\MHz}}
{\textbf{\textcolor{DARCgreen}{\qty{124,2}{\MHz}}}}
{\qty{156,3}{\MHz}}
{\DARCimage{1.0\linewidth}{80include}}\end{PQuestion}

}
\end{frame}

\begin{frame}
\frametitle{Lösungsweg}
\begin{itemize}
  \item gegeben: $f_{OSZ} = 134,9MHz$
  \item gegeben: $f_E = 145,6MHz$
  \item gesucht: $f_S$
  \end{itemize}
    \pause
    $f_S = 2 \cdot f_{OSZ} -- f_E = 2 \cdot 134,9MHz -- 145,6MHz = 124,2MHz$



\end{frame}

\begin{frame}
\only<1>{
\begin{QQuestion}{AF203}{Der Quarzoszillator schwingt auf \qty{39}{\MHz}. Die Empfangsfrequenz soll \qty{28,3}{\MHz} betragen. Auf welcher Frequenz ist mit Spiegelfrequenzstörungen zu rechnen?}{\qty{67,3}{\MHz}}
{\qty{39}{\MHz}}
{\qty{49,7}{\MHz}}
{\qty{17,6}{\MHz}}
\end{QQuestion}

}
\only<2>{
\begin{QQuestion}{AF203}{Der Quarzoszillator schwingt auf \qty{39}{\MHz}. Die Empfangsfrequenz soll \qty{28,3}{\MHz} betragen. Auf welcher Frequenz ist mit Spiegelfrequenzstörungen zu rechnen?}{\qty{67,3}{\MHz}}
{\qty{39}{\MHz}}
{\textbf{\textcolor{DARCgreen}{\qty{49,7}{\MHz}}}}
{\qty{17,6}{\MHz}}
\end{QQuestion}

}
\end{frame}

\begin{frame}
\frametitle{Lösungsweg}
\begin{itemize}
  \item gegeben: $f_{OSZ} = 39MHz$
  \item gegeben: $f_E = 28,3MHz$
  \item gesucht: $f_S$
  \end{itemize}
    \pause
    $f_S = 2 \cdot f_{OSZ} -- f_E = 2 \cdot 39MHz -- 28,3MHz = 49,7MHz$



\end{frame}

\begin{frame}
\only<1>{
\begin{QQuestion}{AF204}{Wodurch wird beim Überlagerungsempfänger die Spiegelfrequenzdämpfung bestimmt?}{Durch die Selektion im ZF-Bereich}
{Durch die Demodulatorkennlinie}
{Durch die Vorselektion}
{Durch den Tiefpass im Audioverstärker}
\end{QQuestion}

}
\only<2>{
\begin{QQuestion}{AF204}{Wodurch wird beim Überlagerungsempfänger die Spiegelfrequenzdämpfung bestimmt?}{Durch die Selektion im ZF-Bereich}
{Durch die Demodulatorkennlinie}
{\textbf{\textcolor{DARCgreen}{Durch die Vorselektion}}}
{Durch den Tiefpass im Audioverstärker}
\end{QQuestion}

}
\end{frame}

\begin{frame}
\only<1>{
\begin{QQuestion}{AF106}{Welche Frequenzdifferenz besteht bei einem Einfachsuper immer zwischen der Empfangsfrequenz und der Spiegelfrequenz?}{Die doppelte Empfangsfrequenz}
{Die Frequenz des lokalen Oszillators}
{Die doppelte ZF}
{Die ZF}
\end{QQuestion}

}
\only<2>{
\begin{QQuestion}{AF106}{Welche Frequenzdifferenz besteht bei einem Einfachsuper immer zwischen der Empfangsfrequenz und der Spiegelfrequenz?}{Die doppelte Empfangsfrequenz}
{Die Frequenz des lokalen Oszillators}
{\textbf{\textcolor{DARCgreen}{Die doppelte ZF}}}
{Die ZF}
\end{QQuestion}

}
\end{frame}

\begin{frame}
\only<1>{
\begin{PQuestion}{AF107}{Ein Einfachsuperhet-Empfänger ist auf \qty{14,24}{\MHz} eingestellt. Der Lokaloszillator schwingt mit \qty{24,94}{\MHz} und liegt mit dieser Frequenz über der ZF. Wo können Spiegelfrequenzstörungen auftreten?}{\qty{10,7}{\MHz}}
{\qty{35,64}{\MHz}}
{\qty{3,54}{\MHz}}
{\qty{24,94}{\MHz}}
{\DARCimage{1.0\linewidth}{590include}}\end{PQuestion}

}
\only<2>{
\begin{PQuestion}{AF107}{Ein Einfachsuperhet-Empfänger ist auf \qty{14,24}{\MHz} eingestellt. Der Lokaloszillator schwingt mit \qty{24,94}{\MHz} und liegt mit dieser Frequenz über der ZF. Wo können Spiegelfrequenzstörungen auftreten?}{\qty{10,7}{\MHz}}
{\textbf{\textcolor{DARCgreen}{\qty{35,64}{\MHz}}}}
{\qty{3,54}{\MHz}}
{\qty{24,94}{\MHz}}
{\DARCimage{1.0\linewidth}{590include}}\end{PQuestion}

}
\end{frame}

\begin{frame}
\frametitle{Lösungsweg}
\begin{itemize}
  \item gegeben: $f_{OSZ} = 24,94MHz$
  \item gegeben: $f_E = 14,24MHz$
  \item gesucht: $f_S$
  \end{itemize}
    \pause
    $f_S = 2 \cdot f_{OSZ} -- f_E = 2 \cdot 24,94MHz -- 14,24MHz = 35,64MHz$



\end{frame}

\begin{frame}
\only<1>{
\begin{PQuestion}{AF108}{Ein Einfachsuper hat eine ZF von \qty{10,7}{\MHz} und ist auf \qty{28,5}{\MHz} abgestimmt. Der Oszillator des Empfängers schwingt oberhalb der Empfangsfrequenz. Welche Frequenz hat die Spiegelfrequenz?}{\qty{17,8}{\MHz}}
{\qty{7,1}{\MHz}}
{\qty{39,2}{\MHz}}
{\qty{49,9}{\MHz}}
{\DARCimage{1.0\linewidth}{590include}}\end{PQuestion}

}
\only<2>{
\begin{PQuestion}{AF108}{Ein Einfachsuper hat eine ZF von \qty{10,7}{\MHz} und ist auf \qty{28,5}{\MHz} abgestimmt. Der Oszillator des Empfängers schwingt oberhalb der Empfangsfrequenz. Welche Frequenz hat die Spiegelfrequenz?}{\qty{17,8}{\MHz}}
{\qty{7,1}{\MHz}}
{\qty{39,2}{\MHz}}
{\textbf{\textcolor{DARCgreen}{\qty{49,9}{\MHz}}}}
{\DARCimage{1.0\linewidth}{590include}}\end{PQuestion}

}
\end{frame}

\begin{frame}
\frametitle{Lösungsweg}
\begin{itemize}
  \item gegeben: $f_{ZF} = 10,7MHz$
  \item gegeben: $f_E = 28,5MHz$
  \item gesucht: $f_S$
  \end{itemize}
    \pause
    Bei $f_E < f_{OSZ}$:

$f_S = f_E + 2 \cdot f_{ZF} = 28,5MHz + 2 \cdot 10,7MHz = 49,9MHz$



\end{frame}

\begin{frame}
\only<1>{
\begin{QQuestion}{AF109}{Welchen Vorteil haben Kurzwellenempfänger mit einer sehr hohen ersten ZF-Frequenz (z.~B. \qty{50}{\MHz})?}{Man erhält einen Empfänger für Kurzwelle und gleichzeitig für Ultrakurzwelle.}
{Filter für \qty{50}{\MHz} haben eine höhere Trennschärfe als Filter mit niedrigerer Frequenz.}
{Ein solcher Empfänger hat eine höhere Großsignalfestigkeit.}
{Die Spiegelfrequenz liegt sehr weit außerhalb des Empfangsbereichs.}
\end{QQuestion}

}
\only<2>{
\begin{QQuestion}{AF109}{Welchen Vorteil haben Kurzwellenempfänger mit einer sehr hohen ersten ZF-Frequenz (z.~B. \qty{50}{\MHz})?}{Man erhält einen Empfänger für Kurzwelle und gleichzeitig für Ultrakurzwelle.}
{Filter für \qty{50}{\MHz} haben eine höhere Trennschärfe als Filter mit niedrigerer Frequenz.}
{Ein solcher Empfänger hat eine höhere Großsignalfestigkeit.}
{\textbf{\textcolor{DARCgreen}{Die Spiegelfrequenz liegt sehr weit außerhalb des Empfangsbereichs.}}}
\end{QQuestion}

}
\end{frame}

\begin{frame}
\only<1>{
\begin{QQuestion}{AF110}{Wodurch wird beim Überlagerungsempfänger mit einer ZF die Spiegelfrequenzunterdrückung hauptsächlich bestimmt?}{Durch die Verstärkung der ZF}
{Durch die Höhe der ZF}
{Durch die Bandbreite der ZF-Filter}
{Durch die NF-Bandbreite}
\end{QQuestion}

}
\only<2>{
\begin{QQuestion}{AF110}{Wodurch wird beim Überlagerungsempfänger mit einer ZF die Spiegelfrequenzunterdrückung hauptsächlich bestimmt?}{Durch die Verstärkung der ZF}
{\textbf{\textcolor{DARCgreen}{Durch die Höhe der ZF}}}
{Durch die Bandbreite der ZF-Filter}
{Durch die NF-Bandbreite}
\end{QQuestion}

}
\end{frame}

\begin{frame}
\only<1>{
\begin{QQuestion}{AF111}{Welchen Vorteil bietet eine hohe erste Zwischenfrequenz bei Überlagerungsempfängern?}{Sie reduziert Beeinflussungen des lokalen Oszillators durch Empfangssignale.}
{Sie ermöglicht eine hohe Spiegelfrequenzunterdrückung.}
{Sie vermeidet eine hohe Spiegelfrequenzunterdrückung.}
{Sie ermöglicht eine gute Nahselektion. }
\end{QQuestion}

}
\only<2>{
\begin{QQuestion}{AF111}{Welchen Vorteil bietet eine hohe erste Zwischenfrequenz bei Überlagerungsempfängern?}{Sie reduziert Beeinflussungen des lokalen Oszillators durch Empfangssignale.}
{\textbf{\textcolor{DARCgreen}{Sie ermöglicht eine hohe Spiegelfrequenzunterdrückung.}}}
{Sie vermeidet eine hohe Spiegelfrequenzunterdrückung.}
{Sie ermöglicht eine gute Nahselektion. }
\end{QQuestion}

}
\end{frame}%ENDCONTENT


\section{Doppelüberlagerungsempfänger (Doppelsuper)}
\label{section:doppelueberlagerungsempfaenger_doppelsuper}
\begin{frame}%STARTCONTENT

\only<1>{
\begin{QQuestion}{AF112}{Welche Aussage ist für einen Doppelsuper richtig?}{Mit einer hohen ersten ZF erreicht man leicht eine gute Spiegelfrequenzunterdrückung.}
{Das von der Antenne aufgenommene Signal bleibt bis zum Demodulator in seiner Frequenz erhalten.}
{Mit einer niedrigen ersten ZF erreicht man leicht eine gute Spiegelfrequenzunterdrückung.}
{Mit einer niedrigen zweiten ZF erreicht man leicht eine gute Spiegelfrequenzunterdrückung.}
\end{QQuestion}

}
\only<2>{
\begin{QQuestion}{AF112}{Welche Aussage ist für einen Doppelsuper richtig?}{\textbf{\textcolor{DARCgreen}{Mit einer hohen ersten ZF erreicht man leicht eine gute Spiegelfrequenzunterdrückung.}}}
{Das von der Antenne aufgenommene Signal bleibt bis zum Demodulator in seiner Frequenz erhalten.}
{Mit einer niedrigen ersten ZF erreicht man leicht eine gute Spiegelfrequenzunterdrückung.}
{Mit einer niedrigen zweiten ZF erreicht man leicht eine gute Spiegelfrequenzunterdrückung.}
\end{QQuestion}

}
\end{frame}

\begin{frame}
\only<1>{
\begin{QQuestion}{AF113}{Welche Aussage ist für einen Doppelsuper richtig?}{Durch eine hohe erste ZF erreicht man leicht eine hohe Empfindlichkeit.}
{Mit einer niedrigen ersten ZF erreicht man leicht gute Werte bei der Kreuzmodulation.}
{Mit einer niedrigen zweiten ZF erreicht man leicht eine gute Trennschärfe.}
{Durch eine niedrige zweite ZF erreicht man leicht eine gute Spiegelselektion.}
\end{QQuestion}

}
\only<2>{
\begin{QQuestion}{AF113}{Welche Aussage ist für einen Doppelsuper richtig?}{Durch eine hohe erste ZF erreicht man leicht eine hohe Empfindlichkeit.}
{Mit einer niedrigen ersten ZF erreicht man leicht gute Werte bei der Kreuzmodulation.}
{\textbf{\textcolor{DARCgreen}{Mit einer niedrigen zweiten ZF erreicht man leicht eine gute Trennschärfe.}}}
{Durch eine niedrige zweite ZF erreicht man leicht eine gute Spiegelselektion.}
\end{QQuestion}

}
\end{frame}

\begin{frame}
\only<1>{
\begin{QQuestion}{AF114}{Welche Beziehungen der Zwischenfrequenzen zueinander sind für einen Kurzwellen-Doppelsuper vorteilhaft?}{Die 1. ZF liegt niedriger als die maximale Empfangsfrequenz. Nach der Filterung im Roofing-Filter (1. ZF) wird auf eine höhere 2. ZF heraufgemischt.}
{Die 1. ZF liegt höher als das Doppelte der maximalen Empfangsfrequenz. Nach der Filterung im Roofing-Filter (1. ZF) wird auf eine niedrigere 2. ZF heruntergemischt.}
{Die 1. ZF liegt unter der niedrigsten Empfangsfrequenz. Die 2. ZF liegt über der höchsten Empfangsfrequenz. }
{Die 1. ZF darf maximal die Hälfte der höchsten Empfangsfrequenz betragen. Die 2. ZF liegt höher als das Doppelte der niedrigsten Empfangsfrequenz.}
\end{QQuestion}

}
\only<2>{
\begin{QQuestion}{AF114}{Welche Beziehungen der Zwischenfrequenzen zueinander sind für einen Kurzwellen-Doppelsuper vorteilhaft?}{Die 1. ZF liegt niedriger als die maximale Empfangsfrequenz. Nach der Filterung im Roofing-Filter (1. ZF) wird auf eine höhere 2. ZF heraufgemischt.}
{\textbf{\textcolor{DARCgreen}{Die 1. ZF liegt höher als das Doppelte der maximalen Empfangsfrequenz. Nach der Filterung im Roofing-Filter (1. ZF) wird auf eine niedrigere 2. ZF heruntergemischt.}}}
{Die 1. ZF liegt unter der niedrigsten Empfangsfrequenz. Die 2. ZF liegt über der höchsten Empfangsfrequenz. }
{Die 1. ZF darf maximal die Hälfte der höchsten Empfangsfrequenz betragen. Die 2. ZF liegt höher als das Doppelte der niedrigsten Empfangsfrequenz.}
\end{QQuestion}

}
\end{frame}

\begin{frame}
\only<1>{
\begin{QQuestion}{AF116}{Wie groß sollte die Bandbreite des Filters für die 1. ZF in einem Doppelsuper sein?}{Mindestens so groß wie die größte benötigte Bandbreite der vorgesehenen Betriebsarten.}
{Mindestens so groß wie die doppelte Bandbreite der jeweiligen Betriebsart.}
{Mindestens so groß wie das breiteste zu empfangende Amateurband.}
{Sie muss den vollen Abstimmbereich des Empfängers umfassen.}
\end{QQuestion}

}
\only<2>{
\begin{QQuestion}{AF116}{Wie groß sollte die Bandbreite des Filters für die 1. ZF in einem Doppelsuper sein?}{\textbf{\textcolor{DARCgreen}{Mindestens so groß wie die größte benötigte Bandbreite der vorgesehenen Betriebsarten.}}}
{Mindestens so groß wie die doppelte Bandbreite der jeweiligen Betriebsart.}
{Mindestens so groß wie das breiteste zu empfangende Amateurband.}
{Sie muss den vollen Abstimmbereich des Empfängers umfassen.}
\end{QQuestion}

}
\end{frame}

\begin{frame}
\only<1>{
\begin{PQuestion}{AF209}{Folgende Schaltung stellt einen Doppelsuper dar. Welche Funktion haben die drei mit X, Y und Z gekennzeichneten Blöcke?}{X und Y sind Produktdetektoren, Z ist ein HF-Mischer.}
{X ist ein Mischer, Y ist ein Produktdetektor, Z ist ein Mischer.}
{X und Y sind Mischer, Z ist ein Produktdetektor.}
{X und Y sind Balancemischer, Z ist ein ZF-Verstärker.}
{\DARCimage{1.0\linewidth}{82include}}\end{PQuestion}

}
\only<2>{
\begin{PQuestion}{AF209}{Folgende Schaltung stellt einen Doppelsuper dar. Welche Funktion haben die drei mit X, Y und Z gekennzeichneten Blöcke?}{X und Y sind Produktdetektoren, Z ist ein HF-Mischer.}
{X ist ein Mischer, Y ist ein Produktdetektor, Z ist ein Mischer.}
{\textbf{\textcolor{DARCgreen}{X und Y sind Mischer, Z ist ein Produktdetektor.}}}
{X und Y sind Balancemischer, Z ist ein ZF-Verstärker.}
{\DARCimage{1.0\linewidth}{82include}}\end{PQuestion}

}
\end{frame}

\begin{frame}
\only<1>{
\begin{PQuestion}{AF117}{Folgende Schaltung stellt einen Doppelsuper dar. Welche Funktion haben die drei mit X, Y und Z gekennzeichneten Blöcke?}{X ist ein BFO, Y ist ein CO und Z ein VFO.}
{X ist ein VFO, Y ist ein BFO und Z ein CO.}
{X ist ein VFO, Y ist ein CO und Z ein BFO.}
{X ist ein BFO, Y ist ein VFO und Z ein CO.}
{\DARCimage{1.0\linewidth}{83include}}\end{PQuestion}

}
\only<2>{
\begin{PQuestion}{AF117}{Folgende Schaltung stellt einen Doppelsuper dar. Welche Funktion haben die drei mit X, Y und Z gekennzeichneten Blöcke?}{X ist ein BFO, Y ist ein CO und Z ein VFO.}
{X ist ein VFO, Y ist ein BFO und Z ein CO.}
{\textbf{\textcolor{DARCgreen}{X ist ein VFO, Y ist ein CO und Z ein BFO.}}}
{X ist ein BFO, Y ist ein VFO und Z ein CO.}
{\DARCimage{1.0\linewidth}{83include}}\end{PQuestion}

}
\end{frame}

\begin{frame}
\only<1>{
\begin{PQuestion}{AF210}{Welchen Frequenzbereich kann der VFO des im folgenden Blockschaltbild gezeichneten HF-Teils eines Empfängers haben?}{\qtyrange{20}{47}{\MHz} oder \qtyrange{62}{89}{\MHz}}
{\qtyrange{20}{47}{\MHz} oder \qtyrange{53}{80}{\MHz}}
{\qtyrange{23}{41}{\MHz} oder \qtyrange{53}{80}{\MHz}}
{\qtyrange{23}{41}{\MHz} oder \qtyrange{62}{89}{\MHz}}
{\DARCimage{1.0\linewidth}{93include}}\end{PQuestion}

}
\only<2>{
\begin{PQuestion}{AF210}{Welchen Frequenzbereich kann der VFO des im folgenden Blockschaltbild gezeichneten HF-Teils eines Empfängers haben?}{\qtyrange{20}{47}{\MHz} oder \qtyrange{62}{89}{\MHz}}
{\textbf{\textcolor{DARCgreen}{\qtyrange{20}{47}{\MHz} oder \qtyrange{53}{80}{\MHz}}}}
{\qtyrange{23}{41}{\MHz} oder \qtyrange{53}{80}{\MHz}}
{\qtyrange{23}{41}{\MHz} oder \qtyrange{62}{89}{\MHz}}
{\DARCimage{1.0\linewidth}{93include}}\end{PQuestion}

}
\end{frame}

\begin{frame}
\frametitle{Lösungsweg}
\begin{itemize}
  \item gegeben: $f_E = 3\dots30MHz$
  \item gegeben: $f_{ZF1} = 50MHz$
  \item gesucht: $f_{OSZ}$
  \end{itemize}
    \pause
    $f_{ZF} = |f_E − f_{OSZ}| \Rightarrow f_{OSZ} = f_{ZF} \pm f_{E}$
    \pause
    \begin{enumerate}
  \item[1] Lösung: $f_{OSZ} = f_{ZF} + f_{E} = 50MHz + 3\dots30MHz = 53\dots80MHz$
  \item[2] Lösung: $_{OSZ} = f_{ZF} -- f_{E} = 50MHz -- 3\dots30MHz = 47\dots20MHz$
  \end{enumerate}


\end{frame}

\begin{frame}
\only<1>{
\begin{PQuestion}{AF120}{Welche Frequenzen können die drei Oszillatoren des im folgenden Blockschaltbild gezeichneten Empfängers haben, wenn eine Frequenz von \qty{3,65}{\MHz} empfangen wird? Bei welcher Antwort sind alle drei Frequenzen richtig?}{VFO:~\qty{23,65}{\MHz} CO1:~\qty{59}{\MHz} CO2:~\qty{8,545}{\MHz}}
{VFO:~\qty{46,35}{\MHz} CO1:~\qty{41}{\MHz} CO2:~\qty{9,455}{\MHz}}
{VFO:~\qty{46,35}{\MHz} CO1:~\qty{41}{\MHz} CO2:~\qty{9,545}{\MHz}}
{VFO:~\qty{46,35}{\MHz} CO1:~\qty{40,545}{\MHz} CO2:~\qty{9,455}{\MHz}}
{\DARCimage{1.0\linewidth}{90include}}\end{PQuestion}

}
\only<2>{
\begin{PQuestion}{AF120}{Welche Frequenzen können die drei Oszillatoren des im folgenden Blockschaltbild gezeichneten Empfängers haben, wenn eine Frequenz von \qty{3,65}{\MHz} empfangen wird? Bei welcher Antwort sind alle drei Frequenzen richtig?}{VFO:~\qty{23,65}{\MHz} CO1:~\qty{59}{\MHz} CO2:~\qty{8,545}{\MHz}}
{\textbf{\textcolor{DARCgreen}{VFO:~\qty{46,35}{\MHz} CO1:~\qty{41}{\MHz} CO2:~\qty{9,455}{\MHz}}}}
{VFO:~\qty{46,35}{\MHz} CO1:~\qty{41}{\MHz} CO2:~\qty{9,545}{\MHz}}
{VFO:~\qty{46,35}{\MHz} CO1:~\qty{40,545}{\MHz} CO2:~\qty{9,455}{\MHz}}
{\DARCimage{1.0\linewidth}{90include}}\end{PQuestion}

}
\end{frame}

\begin{frame}
\frametitle{Lösungsweg}
\begin{columns}
    \begin{column}{0.48\textwidth}
    \begin{itemize}
  \item gegeben: $f_{E} = 3,65MHz$
  \item gegeben: $f_{ZF1} = 50MHz$
  \end{itemize}

    \end{column}
   \begin{column}{0.48\textwidth}
       \begin{itemize}
  \item gegeben: $f_{ZF2} = 9MHz$
  \item gegeben: $f_{NF} = 455kHz$
  \end{itemize}

   \end{column}
\end{columns}

\begin{itemize}
  \item gesucht: $f_{VFO}, f_{CO1}, f_{CO2}$
  \end{itemize}
    \pause
    $f_{ZF} = \begin{cases}f_E + f_{OSZ}\\ f_{OSZ} -- f_E\\ f_E -- f_{OSZ}\end{cases} \Rightarrow f_{OSZ} = \begin{cases}f_{ZF} -- f_E\\ f_E + f_{ZF}\\ f_E -- f_{ZF}\end{cases}$
    \pause
    $f_{VFO} = f_{ZF} -- f_E = 50MHz -- 3,65MHz = 46,35MHz$

$f_{VFO} = f_E \pm f_{ZF1} = 3,65MHz \pm 50MHz = \begin{cases}53,65MHz\\ \cancel{-46,35MHz}\end{cases}$



\end{frame}

\begin{frame}
    \pause
    $f_{CO1} = f_{ZF2} -- f_{ZF1} = 9MHz -- 50MHz = \cancel{-41MHz}$

$f_{CO1} = f_{ZF1} \pm f_{ZF2} = 50MHz \pm 9MHz = \begin{cases}59MHz\\ 41MHz\end{cases}$
    \pause
    $f_{CO2} = f_{NF} -- f_{ZF2} = 455kHz -- 9MHz = \cancel{-8,545MHz}$

$f_{CO2} = f_{ZF2} \pm f_{NF} = 9MHz \pm 455kHz = \begin{cases}9,455MHz\\ 8,545MHz\end{cases}$
    \pause
    VFO: $\bold{46,35MHz} \And 53,65MHz$, CO1: $\bold{41MHz} \And 59MHz$, CO2: $8,545MHz \And \bold{9,455MHz}$



\end{frame}

\begin{frame}
\only<1>{
\begin{PQuestion}{AF118}{Ein Doppelsuper hat eine erste ZF von \qty{9}{\MHz} und eine zweite ZF von \qty{460}{\kHz}. Die Empfangsfrequenz soll \qty{21,1}{\MHz} sein. Welche Frequenzen sind für den VFO und den CO erforderlich, wenn der VFO oberhalb und der CO unterhalb des jeweiligen Mischer-Eingangssignals schwingen sollen?}{Der VFO muss bei \qty{12,1}{\MHz} und der CO bei \qty{9,46}{\MHz} schwingen.}
{Der VFO muss bei \qty{30,1}{\MHz} und der CO bei \qty{8,54}{\MHz} schwingen.}
{Der VFO muss bei \qty{12,1}{\MHz} und der CO bei \qty{8,54}{\MHz} schwingen.}
{Der VFO muss bei \qty{30,1}{\MHz} und der CO bei \qty{9,46}{\MHz} schwingen.}
{\DARCimage{1.0\linewidth}{84include}}\end{PQuestion}

}
\only<2>{
\begin{PQuestion}{AF118}{Ein Doppelsuper hat eine erste ZF von \qty{9}{\MHz} und eine zweite ZF von \qty{460}{\kHz}. Die Empfangsfrequenz soll \qty{21,1}{\MHz} sein. Welche Frequenzen sind für den VFO und den CO erforderlich, wenn der VFO oberhalb und der CO unterhalb des jeweiligen Mischer-Eingangssignals schwingen sollen?}{Der VFO muss bei \qty{12,1}{\MHz} und der CO bei \qty{9,46}{\MHz} schwingen.}
{\textbf{\textcolor{DARCgreen}{Der VFO muss bei \qty{30,1}{\MHz} und der CO bei \qty{8,54}{\MHz} schwingen.}}}
{Der VFO muss bei \qty{12,1}{\MHz} und der CO bei \qty{8,54}{\MHz} schwingen.}
{Der VFO muss bei \qty{30,1}{\MHz} und der CO bei \qty{9,46}{\MHz} schwingen.}
{\DARCimage{1.0\linewidth}{84include}}\end{PQuestion}

}
\end{frame}

\begin{frame}
\frametitle{Lösungsweg}
\begin{columns}
    \begin{column}{0.48\textwidth}
    \begin{itemize}
  \item gegeben: $f_{E} = 21,1MHz$
  \item gegeben: $f_{ZF1} = 9MHz$
  \end{itemize}

    \end{column}
   \begin{column}{0.48\textwidth}
       \begin{itemize}
  \item gegeben: $f_{ZF2} = 460kHz$
  \end{itemize}

   \end{column}
\end{columns}

\begin{itemize}
  \item gesucht: $f_{VFO} \gt f_E, f_{CO} \lt f_{ZF1}$
  \end{itemize}
    \pause
    $f_{ZF} = \begin{cases}f_{OSZ} -- f_E\\ f_E -- f_{OSZ}\end{cases} \Rightarrow f_{OSZ} = \begin{cases}f_E + f_{ZF}\\ f_E -- f_{ZF}\end{cases}$
    \pause
    $f_{VFO} = f_E + f_{ZF1} = 21,1MHz + 9MHz = 30,1MHz$
    \pause
    $f_{CO} = f_{ZF1} -- f_{ZF2} = 9MHz -- 460kHz = 8,54MHz$



\end{frame}

\begin{frame}
\only<1>{
\begin{PQuestion}{AF119}{Ein Doppelsuper hat eine erste ZF von \qty{10,7}{\MHz} und eine zweite ZF von \qty{460}{\kHz}. Die Empfangsfrequenz soll \qty{28}{\MHz} sein. Welche Frequenzen sind für den VFO und den CO erforderlich, wenn die Oszillatoren oberhalb der Mischer-Eingangssignale schwingen sollen?}{Der VFO muss bei \qty{38,70}{\MHz} und der CO bei \qty{11,16}{\MHz} schwingen.}
{Der VFO muss bei \qty{10,24}{\MHz} und der CO bei \qty{17,30}{\MHz} schwingen.}
{Der VFO muss bei \qty{17,3}{\MHz} und der CO bei \qty{10,24}{\MHz} schwingen.}
{Der VFO muss bei \qty{28,460}{\MHz} und der CO bei \qty{39,16}{\MHz} schwingen.}
{\DARCimage{1.0\linewidth}{84include}}\end{PQuestion}

}
\only<2>{
\begin{PQuestion}{AF119}{Ein Doppelsuper hat eine erste ZF von \qty{10,7}{\MHz} und eine zweite ZF von \qty{460}{\kHz}. Die Empfangsfrequenz soll \qty{28}{\MHz} sein. Welche Frequenzen sind für den VFO und den CO erforderlich, wenn die Oszillatoren oberhalb der Mischer-Eingangssignale schwingen sollen?}{\textbf{\textcolor{DARCgreen}{Der VFO muss bei \qty{38,70}{\MHz} und der CO bei \qty{11,16}{\MHz} schwingen.}}}
{Der VFO muss bei \qty{10,24}{\MHz} und der CO bei \qty{17,30}{\MHz} schwingen.}
{Der VFO muss bei \qty{17,3}{\MHz} und der CO bei \qty{10,24}{\MHz} schwingen.}
{Der VFO muss bei \qty{28,460}{\MHz} und der CO bei \qty{39,16}{\MHz} schwingen.}
{\DARCimage{1.0\linewidth}{84include}}\end{PQuestion}

}
\end{frame}

\begin{frame}
\frametitle{Lösungsweg}
\begin{columns}
    \begin{column}{0.48\textwidth}
    \begin{itemize}
  \item gegeben: $f_{E} = 28MHz$
  \item gegeben: $f_{ZF1} = 10,7MHz$
  \end{itemize}

    \end{column}
   \begin{column}{0.48\textwidth}
       \begin{itemize}
  \item gegeben: $f_{ZF2} = 460kHz$
  \end{itemize}

   \end{column}
\end{columns}

\begin{itemize}
  \item gesucht: $f_{VFO} \gt f_E, f_{CO} \gt f_{ZF1}$
  \end{itemize}
    \pause
    $f_{ZF} = \begin{cases}f_{OSZ} -- f_E\\ f_E -- f_{OSZ}\end{cases} \Rightarrow f_{OSZ} = \begin{cases}f_E + f_{ZF}\\ f_E -- f_{ZF}\end{cases}$
    \pause
    $f_{VFO} = f_E + f_{ZF1} = 28MHz + 10,7MHz = 38,70MHz$
    \pause
    $f_{CO} = f_{ZF1} + f_{ZF2} = 10,7MHz + 460kHz = 11,16MHz$



\end{frame}%ENDCONTENT


\section{Trennschärfe II}
\label{section:trennschaerfe_2}
\begin{frame}%STARTCONTENT

\only<1>{
\begin{QQuestion}{AF208}{Welches der folgenden Bandpassfilter verfügt bei jeweils gleicher Mittenfrequenz am ehesten über die geringste Bandbreite und höchste Flankensteilheit?}{RC-Filter}
{LC-Filter}
{Keramikfilter}
{Quarzfilter}
\end{QQuestion}

}
\only<2>{
\begin{QQuestion}{AF208}{Welches der folgenden Bandpassfilter verfügt bei jeweils gleicher Mittenfrequenz am ehesten über die geringste Bandbreite und höchste Flankensteilheit?}{RC-Filter}
{LC-Filter}
{Keramikfilter}
{\textbf{\textcolor{DARCgreen}{Quarzfilter}}}
\end{QQuestion}

}
\end{frame}

\begin{frame}
\only<1>{
\begin{QQuestion}{AF206}{Welche ungefähren Werte sollte die Bandbreite der ZF-Verstärker eines Amateurfunkempfängers für folgende Übertragungsverfahren aufweisen: SSB-Sprechfunk, RTTY (Shift \qty{170}{\Hz}), FM-Sprechfunk?}{SSB:~\qty{2,7}{\kHz}; RTTY:~\qty{340}{\Hz}; FM:~\qty{3,6}{\kHz}}
{SSB:~\qty{6}{\kHz}; RTTY:~\qty{1,5}{\kHz}; FM:~\qty{12}{\kHz}}
{SSB:~\qty{2,7}{\kHz}; RTTY:~\qty{500}{\Hz}; FM:~\qty{12}{\kHz}}
{SSB:~\qty{3,6}{\kHz}; RTTY:~\qty{170}{\Hz}; FM:~\qty{25}{\kHz}}
\end{QQuestion}

}
\only<2>{
\begin{QQuestion}{AF206}{Welche ungefähren Werte sollte die Bandbreite der ZF-Verstärker eines Amateurfunkempfängers für folgende Übertragungsverfahren aufweisen: SSB-Sprechfunk, RTTY (Shift \qty{170}{\Hz}), FM-Sprechfunk?}{SSB:~\qty{2,7}{\kHz}; RTTY:~\qty{340}{\Hz}; FM:~\qty{3,6}{\kHz}}
{SSB:~\qty{6}{\kHz}; RTTY:~\qty{1,5}{\kHz}; FM:~\qty{12}{\kHz}}
{\textbf{\textcolor{DARCgreen}{SSB:~\qty{2,7}{\kHz}; RTTY:~\qty{500}{\Hz}; FM:~\qty{12}{\kHz}}}}
{SSB:~\qty{3,6}{\kHz}; RTTY:~\qty{170}{\Hz}; FM:~\qty{25}{\kHz}}
\end{QQuestion}

}
\end{frame}

\begin{frame}
\only<1>{
\begin{QQuestion}{AF205}{Welche Baugruppe eines Empfängers bestimmt die Trennschärfe?}{Der Oszillatorschwingkreis in der Mischstufe}
{Das Oberwellenfilter im ZF-Verstärker}
{Die Filter im ZF-Verstärker}
{Die PLL-Frequenzaufbereitung}
\end{QQuestion}

}
\only<2>{
\begin{QQuestion}{AF205}{Welche Baugruppe eines Empfängers bestimmt die Trennschärfe?}{Der Oszillatorschwingkreis in der Mischstufe}
{Das Oberwellenfilter im ZF-Verstärker}
{\textbf{\textcolor{DARCgreen}{Die Filter im ZF-Verstärker}}}
{Die PLL-Frequenzaufbereitung}
\end{QQuestion}

}
\end{frame}

\begin{frame}
\only<1>{
\begin{PQuestion}{AF207}{Für welche Signale ist die untenstehende Durchlasskurve eines Empfängerfilters geeignet?}{FM-Signale}
{AM-Signale}
{OFDM-Signale}
{SSB-Signale}
{\DARCimage{1.0\linewidth}{38include}}\end{PQuestion}

}
\only<2>{
\begin{PQuestion}{AF207}{Für welche Signale ist die untenstehende Durchlasskurve eines Empfängerfilters geeignet?}{FM-Signale}
{AM-Signale}
{OFDM-Signale}
{\textbf{\textcolor{DARCgreen}{SSB-Signale}}}
{\DARCimage{1.0\linewidth}{38include}}\end{PQuestion}

}
\end{frame}%ENDCONTENT


\section{BFO II}
\label{section:bfo_2}
\begin{frame}%STARTCONTENT

\only<1>{
\begin{QQuestion}{AF211}{Wie groß sollte die Differenz zwischen der BFO-Frequenz und der letzten ZF für den Empfang von CW-Signalen ungefähr sein?}{\qty{800}{\Hz}}
{die halbe Zwischenfrequenz}
{die doppelte Zwischenfrequenz}
{\qty{4}{\kHz}}
\end{QQuestion}

}
\only<2>{
\begin{QQuestion}{AF211}{Wie groß sollte die Differenz zwischen der BFO-Frequenz und der letzten ZF für den Empfang von CW-Signalen ungefähr sein?}{\textbf{\textcolor{DARCgreen}{\qty{800}{\Hz}}}}
{die halbe Zwischenfrequenz}
{die doppelte Zwischenfrequenz}
{\qty{4}{\kHz}}
\end{QQuestion}

}
\end{frame}

\begin{frame}
\only<1>{
\begin{QQuestion}{AF216}{Für die Demodulation von SSB-Signalen im Kurzwellenbereich wird ein Hilfsträgeroszillator verwendet. Welcher der folgenden Oszillatoren ist hierfür am besten geeignet?}{LC-Oszillator mit Parallelschwingkreis}
{quarzgesteuerter Oszillator}
{LC-Oszillator mit Reihenschwingkreis}
{RC-Oszillator}
\end{QQuestion}

}
\only<2>{
\begin{QQuestion}{AF216}{Für die Demodulation von SSB-Signalen im Kurzwellenbereich wird ein Hilfsträgeroszillator verwendet. Welcher der folgenden Oszillatoren ist hierfür am besten geeignet?}{LC-Oszillator mit Parallelschwingkreis}
{\textbf{\textcolor{DARCgreen}{quarzgesteuerter Oszillator}}}
{LC-Oszillator mit Reihenschwingkreis}
{RC-Oszillator}
\end{QQuestion}

}
\end{frame}%ENDCONTENT


\section{Inter- und Kreuzmodulation}
\label{section:intermodulation_kreuzmodulation}
\begin{frame}%STARTCONTENT

\only<1>{
\begin{QQuestion}{AF217}{Welches Phänomen tritt bei einem gleichzeitigen Empfang zweier Signale an einer nicht linear arbeitenden Empfängerstufe auf?}{Dopplereffekt }
{Frequenzmodulation}
{erhöhter Signal-Rausch-Abstand}
{Intermodulation}
\end{QQuestion}

}
\only<2>{
\begin{QQuestion}{AF217}{Welches Phänomen tritt bei einem gleichzeitigen Empfang zweier Signale an einer nicht linear arbeitenden Empfängerstufe auf?}{Dopplereffekt }
{Frequenzmodulation}
{erhöhter Signal-Rausch-Abstand}
{\textbf{\textcolor{DARCgreen}{Intermodulation}}}
\end{QQuestion}

}
\end{frame}

\begin{frame}
\only<1>{
\begin{QQuestion}{AF219}{Wodurch wird Kreuzmodulation verursacht?}{Durch Übermodulation oder zu großen Hub.}
{Durch Reflexion der Oberwellen im Empfangsverstärker.}
{Durch die Übersteuerung eines Verstärkers.}
{Durch Vermischung eines starken unerwünschten Signals mit dem Nutzsignal.}
\end{QQuestion}

}
\only<2>{
\begin{QQuestion}{AF219}{Wodurch wird Kreuzmodulation verursacht?}{Durch Übermodulation oder zu großen Hub.}
{Durch Reflexion der Oberwellen im Empfangsverstärker.}
{Durch die Übersteuerung eines Verstärkers.}
{\textbf{\textcolor{DARCgreen}{Durch Vermischung eines starken unerwünschten Signals mit dem Nutzsignal.}}}
\end{QQuestion}

}
\end{frame}

\begin{frame}
\only<1>{
\begin{QQuestion}{AF222}{Wodurch kann die Qualität eines empfangenen Signals beispielsweise verringert werden?  }{Durch Betrieb des Empfängers an einem linear geregelten Netzteil}
{Durch Batteriebetrieb des Empfängers}
{Durch eine zu niedrige Rauschzahl des Empfängers}
{Durch starke HF-Signale auf einer sehr nahen Frequenz }
\end{QQuestion}

}
\only<2>{
\begin{QQuestion}{AF222}{Wodurch kann die Qualität eines empfangenen Signals beispielsweise verringert werden?  }{Durch Betrieb des Empfängers an einem linear geregelten Netzteil}
{Durch Batteriebetrieb des Empfängers}
{Durch eine zu niedrige Rauschzahl des Empfängers}
{\textbf{\textcolor{DARCgreen}{Durch starke HF-Signale auf einer sehr nahen Frequenz }}}
\end{QQuestion}

}
\end{frame}

\begin{frame}
\only<1>{
\begin{QQuestion}{AF218}{Was ist die Hauptursache für Intermodulationsprodukte in einem Empfänger?}{Es wird ein zu schmalbandiger Preselektor verwendet.}
{Der Empfänger ist nicht genau auf die Frequenz eingestellt.}
{Es wird ein zu schmalbandiges Quarzfilter verwendet.}
{Die HF-Stufe wird bei zunehmend großen Eingangssignalen zunehmend nichtlinear.}
\end{QQuestion}

}
\only<2>{
\begin{QQuestion}{AF218}{Was ist die Hauptursache für Intermodulationsprodukte in einem Empfänger?}{Es wird ein zu schmalbandiger Preselektor verwendet.}
{Der Empfänger ist nicht genau auf die Frequenz eingestellt.}
{Es wird ein zu schmalbandiges Quarzfilter verwendet.}
{\textbf{\textcolor{DARCgreen}{Die HF-Stufe wird bei zunehmend großen Eingangssignalen zunehmend nichtlinear.}}}
\end{QQuestion}

}
\end{frame}

\begin{frame}
\only<1>{
\begin{question2x2}{AF223}{Welche Konfiguration wäre für die Unterdrückung eines unerwünschten Störsignals am Eingang eines Empfängers hilfreich?}{\DARCimage{1.0\linewidth}{436include}}
{\DARCimage{1.0\linewidth}{435include}}
{\DARCimage{1.0\linewidth}{434include}}
{\DARCimage{1.0\linewidth}{437include}}
\end{question2x2}

}
\only<2>{
\begin{question2x2}{AF223}{Welche Konfiguration wäre für die Unterdrückung eines unerwünschten Störsignals am Eingang eines Empfängers hilfreich?}{\DARCimage{1.0\linewidth}{436include}}
{\DARCimage{1.0\linewidth}{435include}}
{\textbf{\textcolor{DARCgreen}{\DARCimage{1.0\linewidth}{434include}}}}
{\DARCimage{1.0\linewidth}{437include}}
\end{question2x2}

}
\end{frame}

\begin{frame}
\only<1>{
\begin{QQuestion}{AF221}{Welche Empfängereigenschaft beurteilt man mit dem Interception Point IP$_3$?}{Signal-Rausch-Verhältnis}
{Trennschärfe}
{Grenzempfindlichkeit}
{Großsignalfestigkeit}
\end{QQuestion}

}
\only<2>{
\begin{QQuestion}{AF221}{Welche Empfängereigenschaft beurteilt man mit dem Interception Point IP$_3$?}{Signal-Rausch-Verhältnis}
{Trennschärfe}
{Grenzempfindlichkeit}
{\textbf{\textcolor{DARCgreen}{Großsignalfestigkeit}}}
\end{QQuestion}

}
\end{frame}

\begin{frame}
\only<1>{
\begin{QQuestion}{AF220}{Wodurch erreicht man eine Verringerung von Intermodulation und Kreuzmodulation beim Empfang?}{Einschalten der Rauschsperre}
{Einschalten des Vorverstärkers}
{Einschalten des Noise-Blankers}
{Einschalten eines Dämpfungsgliedes vor den Empfängereingang}
\end{QQuestion}

}
\only<2>{
\begin{QQuestion}{AF220}{Wodurch erreicht man eine Verringerung von Intermodulation und Kreuzmodulation beim Empfang?}{Einschalten der Rauschsperre}
{Einschalten des Vorverstärkers}
{Einschalten des Noise-Blankers}
{\textbf{\textcolor{DARCgreen}{Einschalten eines Dämpfungsgliedes vor den Empfängereingang}}}
\end{QQuestion}

}
\end{frame}%ENDCONTENT


\section{Begrenzerverstärker}
\label{section:begrenzerverstaerker}
\begin{frame}%STARTCONTENT

\only<1>{
\begin{QQuestion}{AF226}{Welche Aufgabe hat der Begrenzerverstärker in einem FM-Empfänger?}{Er verringert das Vorstufenrauschen.}
{Er begrenzt das Ausgangssignal ab einem bestimmten Pegel des Eingangssignals zur Unterdrückung von AM-Störungen.   }
{Er begrenzt den Hub für den FM-Demodulator.}
{Er bewirkt eine vollständige ZF-Trägerunterdrückung zur Vermeidung von AM-Störungen.}
\end{QQuestion}

}
\only<2>{
\begin{QQuestion}{AF226}{Welche Aufgabe hat der Begrenzerverstärker in einem FM-Empfänger?}{Er verringert das Vorstufenrauschen.}
{\textbf{\textcolor{DARCgreen}{Er begrenzt das Ausgangssignal ab einem bestimmten Pegel des Eingangssignals zur Unterdrückung von AM-Störungen.   }}}
{Er begrenzt den Hub für den FM-Demodulator.}
{Er bewirkt eine vollständige ZF-Trägerunterdrückung zur Vermeidung von AM-Störungen.}
\end{QQuestion}

}
\end{frame}%ENDCONTENT


\section{Low Noise Block (LNB)}
\label{section:low_noise_block}
\begin{frame}%STARTCONTENT

\only<1>{
\begin{PQuestion}{AF230}{Sie empfangen das Signal eines Satelliten auf \qty{10}{\GHz}. Die Kabellänge zwischen LNB und Empfänger beträgt \qty{20}{\m}. Warum ist die Kabeldämpfung trotz der hohen Empfangsfrequenz eher vernachlässigbar? }{Durch die Fernspeisespannung, die den LNB versorgt, sinkt die Kabeldämpfung.}
{Der LNB verstärkt das Empfangssignal und mischt dieses auf eine niedrigere Frequenz, auf der die Kabeldämpfung geringer ist. }
{Durch die Mischung des Empfangssignals mit der TCXO-Frequenz wird nur noch das Basisband übertragen. }
{Der LNB demoduliert das Signal. Die entstehende NF ist unempfindlich gegen Kabeldämpfung.}
{\DARCimage{1.0\linewidth}{470include}}\end{PQuestion}

}
\only<2>{
\begin{PQuestion}{AF230}{Sie empfangen das Signal eines Satelliten auf \qty{10}{\GHz}. Die Kabellänge zwischen LNB und Empfänger beträgt \qty{20}{\m}. Warum ist die Kabeldämpfung trotz der hohen Empfangsfrequenz eher vernachlässigbar? }{Durch die Fernspeisespannung, die den LNB versorgt, sinkt die Kabeldämpfung.}
{\textbf{\textcolor{DARCgreen}{Der LNB verstärkt das Empfangssignal und mischt dieses auf eine niedrigere Frequenz, auf der die Kabeldämpfung geringer ist. }}}
{Durch die Mischung des Empfangssignals mit der TCXO-Frequenz wird nur noch das Basisband übertragen. }
{Der LNB demoduliert das Signal. Die entstehende NF ist unempfindlich gegen Kabeldämpfung.}
{\DARCimage{1.0\linewidth}{470include}}\end{PQuestion}

}
\end{frame}

\begin{frame}
\only<1>{
\begin{PQuestion}{AF231}{Der LNB einer Satellitenempfangsanlage kann mit zwei unterschiedlichen Betriebsspannungen arbeiten. Was passiert, wenn die Versorgungsspannung am Bias-T im dargestellten Blockschaltbild auf \qty{18}{\V} erhöht wird?}{Der LNB schaltet auf einen anderen Satelliten um. }
{Der LNB schaltet die Polarisation um. }
{Der LNB wird durch Überspannung beschädigt. }
{Der LNB schaltet den Empfangsbereich um. }
{\DARCimage{1.0\linewidth}{470include}}\end{PQuestion}

}
\only<2>{
\begin{PQuestion}{AF231}{Der LNB einer Satellitenempfangsanlage kann mit zwei unterschiedlichen Betriebsspannungen arbeiten. Was passiert, wenn die Versorgungsspannung am Bias-T im dargestellten Blockschaltbild auf \qty{18}{\V} erhöht wird?}{Der LNB schaltet auf einen anderen Satelliten um. }
{\textbf{\textcolor{DARCgreen}{Der LNB schaltet die Polarisation um. }}}
{Der LNB wird durch Überspannung beschädigt. }
{Der LNB schaltet den Empfangsbereich um. }
{\DARCimage{1.0\linewidth}{470include}}\end{PQuestion}

}
\end{frame}%ENDCONTENT


\section{S-Meter}
\label{section:s_meter}
\begin{frame}%STARTCONTENT
\begin{itemize}
  \item Bis S9: Eine S-Stufe entspricht 6dB
  \item 6dB: $2\cdot U$ oder $4\cdot P$
  \end{itemize}
\end{frame}

\begin{frame}
\only<1>{
\begin{QQuestion}{AA113}{Wie groß ist der Unterschied zwischen den S-Stufen S4 und S7 in dB?}{\qty{18}{\decibel}}
{\qty{9}{\decibel}}
{\qty{15}{\decibel}}
{\qty{3}{\decibel}}
\end{QQuestion}

}
\only<2>{
\begin{QQuestion}{AA113}{Wie groß ist der Unterschied zwischen den S-Stufen S4 und S7 in dB?}{\textbf{\textcolor{DARCgreen}{\qty{18}{\decibel}}}}
{\qty{9}{\decibel}}
{\qty{15}{\decibel}}
{\qty{3}{\decibel}}
\end{QQuestion}

}
\end{frame}

\begin{frame}
\frametitle{Lösungsweg}
\begin{itemize}
  \item von S3 bis S7 sind 3-Stufen
  \item $3\cdot 6dB = 18dB$
  \end{itemize}
\end{frame}

\begin{frame}
\only<1>{
\begin{QQuestion}{AF104}{Ein Funkamateur kommt laut S-Meter mit S7 an. Dann schaltet dieser seine Endstufe ein und bittet um einen erneuten Rapport. Das S-Meter zeigt nun S9+\qty{8}{\decibel} an. Um welchen Faktor hat der Funkamateur seine Leistung erhöht?}{100-fach}
{20-fach}
{10-fach}
{120-fach}
\end{QQuestion}

}
\only<2>{
\begin{QQuestion}{AF104}{Ein Funkamateur kommt laut S-Meter mit S7 an. Dann schaltet dieser seine Endstufe ein und bittet um einen erneuten Rapport. Das S-Meter zeigt nun S9+\qty{8}{\decibel} an. Um welchen Faktor hat der Funkamateur seine Leistung erhöht?}{\textbf{\textcolor{DARCgreen}{100-fach}}}
{20-fach}
{10-fach}
{120-fach}
\end{QQuestion}

}
\end{frame}

\begin{frame}
\frametitle{Lösungsweg}
\begin{itemize}
  \item von S7 auf S9+8dB sind 6dB+6dB+8dB = 20dB
  \item 20dB entsprechen der 100-fachen Leistung
  \end{itemize}
\end{frame}

\begin{frame}
\only<1>{
\begin{QQuestion}{AF101}{Um wie viele S-Stufen müsste die S-Meter-Anzeige Ihres Empfängers steigen, wenn Ihr Partner die Sendeleistung von \qty{25}{\W} auf \qty{100}{\W} erhöht?}{Um zwei S-Stufen}
{Um eine S-Stufe}
{Um vier S-Stufen}
{Um acht S-Stufen}
\end{QQuestion}

}
\only<2>{
\begin{QQuestion}{AF101}{Um wie viele S-Stufen müsste die S-Meter-Anzeige Ihres Empfängers steigen, wenn Ihr Partner die Sendeleistung von \qty{25}{\W} auf \qty{100}{\W} erhöht?}{Um zwei S-Stufen}
{\textbf{\textcolor{DARCgreen}{Um eine S-Stufe}}}
{Um vier S-Stufen}
{Um acht S-Stufen}
\end{QQuestion}

}
\end{frame}

\begin{frame}
\frametitle{Lösungsweg}
\begin{itemize}
  \item von 25W auf 100W sind $\frac{100W}{25W} = 4$-fache Leistung
  \item 4-fache Leistung entsprich einer S-Stufe
  \end{itemize}
\end{frame}

\begin{frame}
\only<1>{
\begin{QQuestion}{AF102}{Um wie viel S-Stufen müsste die S-Meter-Anzeige Ihres Empfängers steigen, wenn Ihr Funkpartner die Sendeleistung von \qty{100}{\W} auf \qty{400}{\W} erhöht?}{Um eine S-Stufe}
{Um zwei S-Stufen}
{Um vier S-Stufen}
{Um acht S-Stufen}
\end{QQuestion}

}
\only<2>{
\begin{QQuestion}{AF102}{Um wie viel S-Stufen müsste die S-Meter-Anzeige Ihres Empfängers steigen, wenn Ihr Funkpartner die Sendeleistung von \qty{100}{\W} auf \qty{400}{\W} erhöht?}{\textbf{\textcolor{DARCgreen}{Um eine S-Stufe}}}
{Um zwei S-Stufen}
{Um vier S-Stufen}
{Um acht S-Stufen}
\end{QQuestion}

}
\end{frame}

\begin{frame}
\frametitle{Lösungsweg}
\begin{itemize}
  \item von 100W auf 400W sind $\frac{400W}{100W} = 4$-fache Leistung
  \item 4-fache Leistung entspricht einer S-Stufe
  \end{itemize}
\end{frame}

\begin{frame}
\only<1>{
\begin{QQuestion}{AF103}{Ein Funkamateur erhöht seine Sendeleistung von 10 auf \qty{100}{\W}. Vor der Leistungserhöhung zeigte Ihr S-Meter genau S8. Auf welchen Wert müsste die Anzeige Ihres S-Meters nach der Leistungserhöhung ansteigen?}{S9}
{S9+\qty{7}{\decibel}}
{S9+\qty{4}{\decibel}}
{S9+\qty{9}{\decibel}}
\end{QQuestion}

}
\only<2>{
\begin{QQuestion}{AF103}{Ein Funkamateur erhöht seine Sendeleistung von 10 auf \qty{100}{\W}. Vor der Leistungserhöhung zeigte Ihr S-Meter genau S8. Auf welchen Wert müsste die Anzeige Ihres S-Meters nach der Leistungserhöhung ansteigen?}{S9}
{S9+\qty{7}{\decibel}}
{\textbf{\textcolor{DARCgreen}{S9+\qty{4}{\decibel}}}}
{S9+\qty{9}{\decibel}}
\end{QQuestion}

}
\end{frame}

\begin{frame}
\frametitle{Lösungsweg}
\begin{itemize}
  \item von 10W auf 100W sind $\frac{100W}{10W} = 10$-fache Leistung
  \item 10-fache Leistung entspricht 10dB
  \item von S8 auf S9 sind 6dB
  \item die restlichen 4dB kommen als +4dB oben drauf
  \end{itemize}
\end{frame}

\begin{frame}
\only<1>{
\begin{QQuestion}{AA114}{Wie stark ist die Empfängereingangsspannung abgesunken, wenn die S-Meter-Anzeige durch Änderung der Ausbreitungsbedingungen von S9+\qty{20}{\decibel} auf S8 zurückgeht? Die Empfängereingangsspannung sinkt um~...}{\qty{6}{\decibel}.}
{\qty{23}{\decibel}.}
{\qty{26}{\decibel}.}
{\qty{20}{\decibel}.}
\end{QQuestion}

}
\only<2>{
\begin{QQuestion}{AA114}{Wie stark ist die Empfängereingangsspannung abgesunken, wenn die S-Meter-Anzeige durch Änderung der Ausbreitungsbedingungen von S9+\qty{20}{\decibel} auf S8 zurückgeht? Die Empfängereingangsspannung sinkt um~...}{\qty{6}{\decibel}.}
{\qty{23}{\decibel}.}
{\textbf{\textcolor{DARCgreen}{\qty{26}{\decibel}.}}}
{\qty{20}{\decibel}.}
\end{QQuestion}

}
\end{frame}

\begin{frame}
\frametitle{Lösungsweg}
\begin{itemize}
  \item von S9+20dB auf S8 sind 26dB
  \end{itemize}
\end{frame}

\begin{frame}
\only<1>{
\begin{QQuestion}{AF105}{Durch \glqq Fading\grqq{} sinkt die S-Meter-Anzeige von S9 auf S8. Auf welchen Wert sinkt dabei die Empfänger-Eingangsspannung ab, wenn bei S9 am Empfängereingang \qty{50}{\micro\V} anliegen? Die Empfänger-Eingangsspannung sinkt auf}{\qty{25}{\micro\V}}
{\qty{37}{\micro\V}}
{\qty{40}{\micro\V}}
{\qty{30}{\micro\V}}
\end{QQuestion}

}
\only<2>{
\begin{QQuestion}{AF105}{Durch \glqq Fading\grqq{} sinkt die S-Meter-Anzeige von S9 auf S8. Auf welchen Wert sinkt dabei die Empfänger-Eingangsspannung ab, wenn bei S9 am Empfängereingang \qty{50}{\micro\V} anliegen? Die Empfänger-Eingangsspannung sinkt auf}{\textbf{\textcolor{DARCgreen}{\qty{25}{\micro\V}}}}
{\qty{37}{\micro\V}}
{\qty{40}{\micro\V}}
{\qty{30}{\micro\V}}
\end{QQuestion}

}
\end{frame}

\begin{frame}
\frametitle{Lösungsweg}
\end{frame}%ENDCONTENT


\section{Dämpfungsglieder}
\label{section:daempfungsglieder}
\begin{frame}%STARTCONTENT

\only<1>{
\begin{PQuestion}{AD801}{Was zeigt diese Schaltung?}{Dämpfungsglied}
{Verstärker}
{Hochpass}
{Tiefpass}
{\DARCimage{1.0\linewidth}{554include}}\end{PQuestion}

}
\only<2>{
\begin{PQuestion}{AD801}{Was zeigt diese Schaltung?}{\textbf{\textcolor{DARCgreen}{Dämpfungsglied}}}
{Verstärker}
{Hochpass}
{Tiefpass}
{\DARCimage{1.0\linewidth}{554include}}\end{PQuestion}

}
\end{frame}

\begin{frame}
\only<1>{
\begin{PQuestion}{AD802}{Was zeigt diese Schaltung?}{Tiefpass}
{Verstärker}
{Hochpass}
{Dämpfungsglied}
{\DARCimage{1.0\linewidth}{555include}}\end{PQuestion}

}
\only<2>{
\begin{PQuestion}{AD802}{Was zeigt diese Schaltung?}{Tiefpass}
{Verstärker}
{Hochpass}
{\textbf{\textcolor{DARCgreen}{Dämpfungsglied}}}
{\DARCimage{1.0\linewidth}{555include}}\end{PQuestion}

}
\end{frame}

\begin{frame}
\only<1>{
\begin{PQuestion}{AD803}{Dargestellt ist ein \qty{20}{\decibel} Dämpfungsglied. Wie groß ist das Leistungsverhältnis zwischen der Eingangsleistung $P_{\symup{IN}}$ und der Leistung am Lastwiderstand $P_{\symup{RL}}$?}{50}
{10}
{20}
{100}
{\DARCimage{1.0\linewidth}{341include}}\end{PQuestion}

}
\only<2>{
\begin{PQuestion}{AD803}{Dargestellt ist ein \qty{20}{\decibel} Dämpfungsglied. Wie groß ist das Leistungsverhältnis zwischen der Eingangsleistung $P_{\symup{IN}}$ und der Leistung am Lastwiderstand $P_{\symup{RL}}$?}{50}
{10}
{20}
{\textbf{\textcolor{DARCgreen}{100}}}
{\DARCimage{1.0\linewidth}{341include}}\end{PQuestion}

}
\end{frame}

\begin{frame}
\frametitle{Lösungweg}
\begin{itemize}
  \item 20dB entsprechen einer Leistungdämpfung mit dem Faktor 100
  \end{itemize}
\end{frame}

\begin{frame}
\only<1>{
\begin{PQuestion}{AD804}{Dargestellt ist ein \qty{6}{\decibel} Dämpfungsglied. Wie groß ist das Leistungsverhältnis zwischen der Eingangsleistung $P_{\symup{IN}}$ und der Leistung am Lastwiderstand $P_{\symup{RL}}$?}{2}
{4}
{3}
{6}
{\DARCimage{1.0\linewidth}{424include}}\end{PQuestion}

}
\only<2>{
\begin{PQuestion}{AD804}{Dargestellt ist ein \qty{6}{\decibel} Dämpfungsglied. Wie groß ist das Leistungsverhältnis zwischen der Eingangsleistung $P_{\symup{IN}}$ und der Leistung am Lastwiderstand $P_{\symup{RL}}$?}{2}
{\textbf{\textcolor{DARCgreen}{4}}}
{3}
{6}
{\DARCimage{1.0\linewidth}{424include}}\end{PQuestion}

}
\end{frame}

\begin{frame}
\frametitle{Lösungsweg}
\begin{itemize}
  \item 6dB entsprechen einer Leistungsdämpfung mit dem Faktor 4
  \end{itemize}
\end{frame}

\begin{frame}
\only<1>{
\begin{PQuestion}{AD805}{Dargestellt ist ein symmetrisches \qty{50}{\ohm} Dämpfungsglied. Welche Impedanz ist zwischen $a$ und $b$ messbar, wenn $R_{\symup{L}}$~=~\qty{50}{\ohm} beträgt?}{$R_1$ + $R_2$ + \qty{50}{\ohm}}
{\qty{100}{\ohm}}
{$R_1$ + \qty{50}{\ohm}}
{\qty{50}{\ohm}}
{\DARCimage{1.0\linewidth}{423include}}\end{PQuestion}

}
\only<2>{
\begin{PQuestion}{AD805}{Dargestellt ist ein symmetrisches \qty{50}{\ohm} Dämpfungsglied. Welche Impedanz ist zwischen $a$ und $b$ messbar, wenn $R_{\symup{L}}$~=~\qty{50}{\ohm} beträgt?}{$R_1$ + $R_2$ + \qty{50}{\ohm}}
{\qty{100}{\ohm}}
{$R_1$ + \qty{50}{\ohm}}
{\textbf{\textcolor{DARCgreen}{\qty{50}{\ohm}}}}
{\DARCimage{1.0\linewidth}{423include}}\end{PQuestion}

}
\end{frame}

\begin{frame}
\frametitle{Lösungsweg}
\begin{itemize}
  \item Die Impedanz für die Gesamtschaltung ändert sich nicht – also 50Ω
  \end{itemize}
\end{frame}

\begin{frame}
\only<1>{
\begin{PQuestion}{AD806}{In einem \qty{50}{\ohm} System wird in ein symmetrisches \qty{20}{\decibel} Dämpfungsglied die Leistung von \qty{100}{\W} eingespeist. Der Widerstand $R_{\symup{L}}$~=~\qty{50}{\ohm} ist an das Dämpfungsglied angepasst. Welche Leistung wird insgesamt im Dämpfungsglied in Wärme umgesetzt?}{\qty{99}{\W}}
{\qty{50}{\W}}
{\qty{2}{\W}}
{\qty{1}{\W}}
{\DARCimage{1.0\linewidth}{342include}}\end{PQuestion}

}
\only<2>{
\begin{PQuestion}{AD806}{In einem \qty{50}{\ohm} System wird in ein symmetrisches \qty{20}{\decibel} Dämpfungsglied die Leistung von \qty{100}{\W} eingespeist. Der Widerstand $R_{\symup{L}}$~=~\qty{50}{\ohm} ist an das Dämpfungsglied angepasst. Welche Leistung wird insgesamt im Dämpfungsglied in Wärme umgesetzt?}{\textbf{\textcolor{DARCgreen}{\qty{99}{\W}}}}
{\qty{50}{\W}}
{\qty{2}{\W}}
{\qty{1}{\W}}
{\DARCimage{1.0\linewidth}{342include}}\end{PQuestion}

}
\end{frame}

\begin{frame}
\frametitle{Lösungsweg}
\begin{itemize}
  \item gegeben: $P_1 = 100W$
  \item gegeben: $a = 20dB$
  \item gesucht: $\Delta P = P_2 -- P_1$
  \end{itemize}
    \pause
    $$\begin{align}\nonumber a &= 10 \cdot \log_{10}{(\frac{P_1}{P_2})}dB\\ \nonumber \Rightarrow \frac{a}{10} &= \log_{10}{(\frac{P_1}{P_2})}dB\\ \nonumber \Rightarrow 10^{\frac{a}{10}} &= \frac{P_1}{P_2}\\ \nonumber \Rightarrow P_2 &= \frac{P_1}{10^{\frac{a}{10}}}\end{align}$$



\end{frame}

\begin{frame}
    \pause
    $P_2 = \frac{P_1}{10^{\frac{a}{10}}} = \frac{100W}{10^{\frac{20}{10}}} = 1W$
    \pause
    $\Delta P = P_2 -- P_1 = 100W -- 1W = 99W$



\end{frame}%ENDCONTENT


\section{Automatische Verstärkungsregelung (AGC) II}
\label{section:agc_2}
\begin{frame}%STARTCONTENT

\only<1>{
\begin{QQuestion}{AF224}{Was bewirkt die AGC (Automatic Gain Control) bei einem starken Eingangssignal?}{Sie reduziert die Amplitude des BFO.}
{Sie reduziert die Amplitude des VFO.}
{Sie reduziert die Verstärkung von Verstärkerstufen im Empfangsteil.}
{Sie erhöht die Verstärkung von Verstärkerstufen im Empfangsteil.}
\end{QQuestion}

}
\only<2>{
\begin{QQuestion}{AF224}{Was bewirkt die AGC (Automatic Gain Control) bei einem starken Eingangssignal?}{Sie reduziert die Amplitude des BFO.}
{Sie reduziert die Amplitude des VFO.}
{\textbf{\textcolor{DARCgreen}{Sie reduziert die Verstärkung von Verstärkerstufen im Empfangsteil.}}}
{Sie erhöht die Verstärkung von Verstärkerstufen im Empfangsteil.}
\end{QQuestion}

}
\end{frame}%ENDCONTENT


\section{SNR und Rauschzahl}
\label{section:snr_rauschzahl}
\begin{frame}%STARTCONTENT

\only<1>{
\begin{QQuestion}{AF227}{Was bedeutet Signal-Rausch-Abstand (SNR) bei einem Empfänger?}{Er gibt an, in welchem Verhältnis das Rauschsignal stärker ist als das Nutzsignal.}
{Er gibt an, in welchem Verhältnis das Nutzsignal stärker ist als das Rauschsignal.}
{Es ist der Frequenzabstand zwischen Empfangssignal und Störsignal.}
{Es ist der Abstand zwischen Empfangsfrequenz und Spiegelfrequenz.}
\end{QQuestion}

}
\only<2>{
\begin{QQuestion}{AF227}{Was bedeutet Signal-Rausch-Abstand (SNR) bei einem Empfänger?}{Er gibt an, in welchem Verhältnis das Rauschsignal stärker ist als das Nutzsignal.}
{\textbf{\textcolor{DARCgreen}{Er gibt an, in welchem Verhältnis das Nutzsignal stärker ist als das Rauschsignal.}}}
{Es ist der Frequenzabstand zwischen Empfangssignal und Störsignal.}
{Es ist der Abstand zwischen Empfangsfrequenz und Spiegelfrequenz.}
\end{QQuestion}

}
\end{frame}

\begin{frame}
\only<1>{
\begin{QQuestion}{AF228}{Was bedeutet die Rauschzahl von \qty{1,8}{\decibel} bei einem UHF-Vorverstärker?}{Das Rauschen des Ausgangssignals ist um \qty{1,8}{\decibel} niedriger als das Rauschen des Eingangssignals.}
{Das Ausgangssignal des Vorverstärkers hat ein um \qty{1,8}{\decibel} höheres Signal-Rausch-Verhältnis als das Eingangssignal.}
{Das Ausgangssignal des Vorverstärkers hat ein um \qty{1,8}{\decibel} geringeres Signal-Rausch-Verhältnis als das Eingangssignal.}
{Die Verstärkung des Nutzsignals beträgt \qty{1,8}{\decibel}, die Rauschleistung bleibt unverändert.}
\end{QQuestion}

}
\only<2>{
\begin{QQuestion}{AF228}{Was bedeutet die Rauschzahl von \qty{1,8}{\decibel} bei einem UHF-Vorverstärker?}{Das Rauschen des Ausgangssignals ist um \qty{1,8}{\decibel} niedriger als das Rauschen des Eingangssignals.}
{Das Ausgangssignal des Vorverstärkers hat ein um \qty{1,8}{\decibel} höheres Signal-Rausch-Verhältnis als das Eingangssignal.}
{\textbf{\textcolor{DARCgreen}{Das Ausgangssignal des Vorverstärkers hat ein um \qty{1,8}{\decibel} geringeres Signal-Rausch-Verhältnis als das Eingangssignal.}}}
{Die Verstärkung des Nutzsignals beträgt \qty{1,8}{\decibel}, die Rauschleistung bleibt unverändert.}
\end{QQuestion}

}
\end{frame}

\begin{frame}
\only<1>{
\begin{QQuestion}{AF229}{Was bedeutet die Rauschzahl F~=~2 bei einem UHF-Vorverstärker?}{Das Ausgangssignal des Verstärkers hat ein um \qty{6}{\decibel} höheres Signal-Rausch-Verhältnis als das Eingangssignal.}
{Das Ausgangssignal des Verstärkers hat ein um \qty{3}{\decibel} höheres Signal-Rausch-Verhältnis als das Eingangssignal.}
{Das Ausgangssignal des Verstärkers hat ein um \qty{6}{\decibel} geringeres Signal-Rausch-Verhältnis als das Eingangssignal.}
{Das Ausgangssignal des Verstärkers hat ein um \qty{3}{\decibel} geringeres Signal-Rausch-Verhältnis als das Eingangssignal.}
\end{QQuestion}

}
\only<2>{
\begin{QQuestion}{AF229}{Was bedeutet die Rauschzahl F~=~2 bei einem UHF-Vorverstärker?}{Das Ausgangssignal des Verstärkers hat ein um \qty{6}{\decibel} höheres Signal-Rausch-Verhältnis als das Eingangssignal.}
{Das Ausgangssignal des Verstärkers hat ein um \qty{3}{\decibel} höheres Signal-Rausch-Verhältnis als das Eingangssignal.}
{Das Ausgangssignal des Verstärkers hat ein um \qty{6}{\decibel} geringeres Signal-Rausch-Verhältnis als das Eingangssignal.}
{\textbf{\textcolor{DARCgreen}{Das Ausgangssignal des Verstärkers hat ein um \qty{3}{\decibel} geringeres Signal-Rausch-Verhältnis als das Eingangssignal.}}}
\end{QQuestion}

}
\end{frame}%ENDCONTENT


\section{Rauschen}
\label{section:rauschen}
\begin{frame}%STARTCONTENT

\only<1>{
\begin{QQuestion}{AB408}{Für Messzwecke speisen Sie in den Antenneneingang Ihres Empfängers ein gleichmäßig über alle Frequenzen verteiltes Rauschsignal aus einem Messender ein (weißes Rauschen). Welche Aussage über die Leistung, die man beim Empfang dieses Signals misst, stimmt?}{Sie ist umgekehrt proportional zur Bandbreite des Empfängers.}
{Sie ist proportional zur Bandbreite des Empfängers.}
{Sie ist proportional zum Signal-Rausch-Abstand des Empfängers}
{Sie ist umgekehrt proportional zum Eingangswiderstand des Empfängers.}
\end{QQuestion}

}
\only<2>{
\begin{QQuestion}{AB408}{Für Messzwecke speisen Sie in den Antenneneingang Ihres Empfängers ein gleichmäßig über alle Frequenzen verteiltes Rauschsignal aus einem Messender ein (weißes Rauschen). Welche Aussage über die Leistung, die man beim Empfang dieses Signals misst, stimmt?}{Sie ist umgekehrt proportional zur Bandbreite des Empfängers.}
{\textbf{\textcolor{DARCgreen}{Sie ist proportional zur Bandbreite des Empfängers.}}}
{Sie ist proportional zum Signal-Rausch-Abstand des Empfängers}
{Sie ist umgekehrt proportional zum Eingangswiderstand des Empfängers.}
\end{QQuestion}

}
\end{frame}

\begin{frame}
\only<1>{
\begin{QQuestion}{AB409}{Wie verhält sich der Pegel des thermischen Rauschens am Empfängerausgang, wenn von einem Quarzfilter mit einer Bandbreite von \qty{2,5}{\kHz} auf ein Quarzfilter mit einer Bandbreite von \qty{0,5}{\kHz} mit gleicher Durchlassdämpfung und Flankensteilheit umgeschaltet wird? Der Rauschleistungspegel~...}{verringert sich um etwa \qty{14}{\decibel}.}
{erhöht sich um etwa \qty{7}{\decibel}.}
{verringert sich um etwa \qty{7}{\decibel}.}
{erhöht sich um etwa \qty{14}{\decibel}.}
\end{QQuestion}

}
\only<2>{
\begin{QQuestion}{AB409}{Wie verhält sich der Pegel des thermischen Rauschens am Empfängerausgang, wenn von einem Quarzfilter mit einer Bandbreite von \qty{2,5}{\kHz} auf ein Quarzfilter mit einer Bandbreite von \qty{0,5}{\kHz} mit gleicher Durchlassdämpfung und Flankensteilheit umgeschaltet wird? Der Rauschleistungspegel~...}{verringert sich um etwa \qty{14}{\decibel}.}
{erhöht sich um etwa \qty{7}{\decibel}.}
{\textbf{\textcolor{DARCgreen}{verringert sich um etwa \qty{7}{\decibel}.}}}
{erhöht sich um etwa \qty{14}{\decibel}.}
\end{QQuestion}

}
\end{frame}

\begin{frame}
\frametitle{Lösungsweg}
\begin{itemize}
  \item gegeben: $B_1 = 2,5kHz$
  \item gegeben: $B_2 = 0,5kHz$
  \item gesucht: $\Delta P_R$
  \end{itemize}
    \pause
    $\Delta P_R = 10 \cdot \log_{10}{(\frac{B_1}{B_2})}dB = 10 \cdot \log_{10}{(\frac{2,5kHz}{0,5kHz})}dB \approx 7dB$



\end{frame}%ENDCONTENT


\section{Squelch II}
\label{section:squelch_2}
\begin{frame}%STARTCONTENT

\only<1>{
\begin{QQuestion}{AF225}{Welche Signale steuern gewöhnlich die Empfängerstummschaltung (Squelch)?}{Es sind die ZF- oder NF-Signale.}
{Es ist das HF-Signal der Eingangsstufe.}
{Es ist das Signal des VFO.}
{Es ist das Signal des BFO.}
\end{QQuestion}

}
\only<2>{
\begin{QQuestion}{AF225}{Welche Signale steuern gewöhnlich die Empfängerstummschaltung (Squelch)?}{\textbf{\textcolor{DARCgreen}{Es sind die ZF- oder NF-Signale.}}}
{Es ist das HF-Signal der Eingangsstufe.}
{Es ist das Signal des VFO.}
{Es ist das Signal des BFO.}
\end{QQuestion}

}
\end{frame}%ENDCONTENT


\section{Demodulator}
\label{section:demodulator}
\begin{frame}%STARTCONTENT

\only<1>{
\begin{PQuestion}{AD501}{Bei dieser Schaltung handelt es sich um einen~...}{FM-Demodulator.}
{SSB-Modulator.}
{Hüllkurvendemodulator zur Demodulation von AM-Signalen.}
{Produktdetektor zur Demodulation von SSB Signalen.}
{\DARCimage{1.0\linewidth}{141include}}\end{PQuestion}

}
\only<2>{
\begin{PQuestion}{AD501}{Bei dieser Schaltung handelt es sich um einen~...}{FM-Demodulator.}
{SSB-Modulator.}
{\textbf{\textcolor{DARCgreen}{Hüllkurvendemodulator zur Demodulation von AM-Signalen.}}}
{Produktdetektor zur Demodulation von SSB Signalen.}
{\DARCimage{1.0\linewidth}{141include}}\end{PQuestion}

}
\end{frame}

\begin{frame}
\only<1>{
\begin{PQuestion}{AD502}{Am ZF-Eingang des Hüllkurvendemodulators liegt das dargestellte Signal an. Welches der folgenden Signale zeigt sich an dem mit X bezeichneten Punkt der Schaltung?}{\DARCimage{1.0\linewidth}{145include}}
{\DARCimage{1.0\linewidth}{146include}}
{\DARCimage{1.0\linewidth}{144include}}
{\DARCimage{1.0\linewidth}{147include}}
{\DARCimage{1.0\linewidth}{607include}}\end{PQuestion}

}
\only<2>{
\begin{PQuestion}{AD502}{Am ZF-Eingang des Hüllkurvendemodulators liegt das dargestellte Signal an. Welches der folgenden Signale zeigt sich an dem mit X bezeichneten Punkt der Schaltung?}{\DARCimage{1.0\linewidth}{145include}}
{\textbf{\textcolor{DARCgreen}{\DARCimage{1.0\linewidth}{146include}}}}
{\DARCimage{1.0\linewidth}{144include}}
{\DARCimage{1.0\linewidth}{147include}}
{\DARCimage{1.0\linewidth}{607include}}\end{PQuestion}

}
\end{frame}

\begin{frame}
\only<1>{
\begin{PQuestion}{AD504}{Bei dieser Schaltung handelt es sich um einen~...}{Diodendetektor zur Demodulation von SSB-Signalen.}
{Produktdetektor zur Demodulation von SSB-Signalen.}
{Produktdetektor zur Demodulation von FM-Signalen.}
{Flanken-Diskriminator zur Demodulation von FM-Signalen.}
{\DARCimage{1.0\linewidth}{149include}}\end{PQuestion}

}
\only<2>{
\begin{PQuestion}{AD504}{Bei dieser Schaltung handelt es sich um einen~...}{Diodendetektor zur Demodulation von SSB-Signalen.}
{Produktdetektor zur Demodulation von SSB-Signalen.}
{Produktdetektor zur Demodulation von FM-Signalen.}
{\textbf{\textcolor{DARCgreen}{Flanken-Diskriminator zur Demodulation von FM-Signalen.}}}
{\DARCimage{1.0\linewidth}{149include}}\end{PQuestion}

}
\end{frame}

\begin{frame}
\only<1>{
\begin{PQuestion}{AD505}{Bei dieser Schaltung handelt es sich um einen~...}{PLL-FM-Demodulator.}
{SSB-Demodulator mit PLL-gesteuertem BFO.}
{PLL-Abwärtsmischer.}
{AM-Modulator.}
{\DARCimage{1.0\linewidth}{77include}}\end{PQuestion}

}
\only<2>{
\begin{PQuestion}{AD505}{Bei dieser Schaltung handelt es sich um einen~...}{\textbf{\textcolor{DARCgreen}{PLL-FM-Demodulator.}}}
{SSB-Demodulator mit PLL-gesteuertem BFO.}
{PLL-Abwärtsmischer.}
{AM-Modulator.}
{\DARCimage{1.0\linewidth}{77include}}\end{PQuestion}

}
\end{frame}

\begin{frame}
\only<1>{
\begin{PQuestion}{AD506}{Bei dieser Schaltung handelt es sich um einen~...}{Hüllkurvendemodulator zur Demodulation von AM-Signalen.}
{Flankendemodulator zur Demodulation von FM-Signalen.}
{Produktdetektor zu Demodulation von SSB-Signalen.}
{Diskriminator zur Demodulation von FM-Signalen.}
{\DARCimage{1.0\linewidth}{153include}}\end{PQuestion}

}
\only<2>{
\begin{PQuestion}{AD506}{Bei dieser Schaltung handelt es sich um einen~...}{Hüllkurvendemodulator zur Demodulation von AM-Signalen.}
{Flankendemodulator zur Demodulation von FM-Signalen.}
{\textbf{\textcolor{DARCgreen}{Produktdetektor zu Demodulation von SSB-Signalen.}}}
{Diskriminator zur Demodulation von FM-Signalen.}
{\DARCimage{1.0\linewidth}{153include}}\end{PQuestion}

}
\end{frame}%ENDCONTENT


\section{Frequenzmessung II}
\label{section:frequenzmessung_2}
\begin{frame}%STARTCONTENT

\only<1>{
\begin{QQuestion}{AI511}{Womit kann die Frequenzanzeige eines durchstimmbaren Empfängers möglichst genau geprüft werden?}{Mit einem LC-Oszillator hoher Schwingkreisgüte.}
{Mit einem Quarzofen- oder GPS-synchronisierten Frequenzgenerator.}
{Mit den Oberschwingungen eines konstant belasteten Schaltnetzteils.}
{Mit einem temperaturstabiliserten RC-Oszillator.}
\end{QQuestion}

}
\only<2>{
\begin{QQuestion}{AI511}{Womit kann die Frequenzanzeige eines durchstimmbaren Empfängers möglichst genau geprüft werden?}{Mit einem LC-Oszillator hoher Schwingkreisgüte.}
{\textbf{\textcolor{DARCgreen}{Mit einem Quarzofen- oder GPS-synchronisierten Frequenzgenerator.}}}
{Mit den Oberschwingungen eines konstant belasteten Schaltnetzteils.}
{Mit einem temperaturstabiliserten RC-Oszillator.}
\end{QQuestion}

}
\end{frame}

\begin{frame}
\only<1>{
\begin{QQuestion}{AI504}{Eine Frequenzmessung wird genauer, wenn bei einem Frequenzzähler~...}{ein Vorteiler mit höherem Teilverhältnis benutzt wird.}
{der Hauptoszillator temperaturstabilisiert wird.}
{die Messdauer möglichst kurz gehalten wird.}
{das Eingangssignal gleichgerichtet wird.}
\end{QQuestion}

}
\only<2>{
\begin{QQuestion}{AI504}{Eine Frequenzmessung wird genauer, wenn bei einem Frequenzzähler~...}{ein Vorteiler mit höherem Teilverhältnis benutzt wird.}
{\textbf{\textcolor{DARCgreen}{der Hauptoszillator temperaturstabilisiert wird.}}}
{die Messdauer möglichst kurz gehalten wird.}
{das Eingangssignal gleichgerichtet wird.}
\end{QQuestion}

}
\end{frame}

\begin{frame}
\only<1>{
\begin{QQuestion}{AI502}{Was kann man mit einem passenden Dämpfungsglied und einem Frequenzzähler messen?}{Die Sendefrequenz eines CW-Senders}
{Den Modulationsindex eines FM-Senders}
{Die Ausdehnung des Seitenbandes eines SSB-Senders}
{Den Frequenzhub eines FM-Senders}
\end{QQuestion}

}
\only<2>{
\begin{QQuestion}{AI502}{Was kann man mit einem passenden Dämpfungsglied und einem Frequenzzähler messen?}{\textbf{\textcolor{DARCgreen}{Die Sendefrequenz eines CW-Senders}}}
{Den Modulationsindex eines FM-Senders}
{Die Ausdehnung des Seitenbandes eines SSB-Senders}
{Den Frequenzhub eines FM-Senders}
\end{QQuestion}

}
\end{frame}

\begin{frame}
\only<1>{
\begin{QQuestion}{AI501}{Wenn die Frequenz eines Senders mit einem Frequenzzähler überprüft wird, ist~...}{der Zähler mit der Netzfrequenz zu synchronisieren.}
{ein Träger ohne Modulation zu verwenden.}
{der Zähler mit der Sendefrequenz zu synchronisieren.}
{eine analoge Modulation des Trägers zu verwenden.}
\end{QQuestion}

}
\only<2>{
\begin{QQuestion}{AI501}{Wenn die Frequenz eines Senders mit einem Frequenzzähler überprüft wird, ist~...}{der Zähler mit der Netzfrequenz zu synchronisieren.}
{\textbf{\textcolor{DARCgreen}{ein Träger ohne Modulation zu verwenden.}}}
{der Zähler mit der Sendefrequenz zu synchronisieren.}
{eine analoge Modulation des Trägers zu verwenden.}
\end{QQuestion}

}
\end{frame}

\begin{frame}
\only<1>{
\begin{QQuestion}{AI503}{Welche Konfiguration gewährleistet die höchste Genauigkeit bei der Prüfung der Trägerfrequenz eines FM-Senders?}{Absorptionsfrequenzmesser und modulierter Träger}
{Oszilloskop und unmodulierter Träger}
{Frequenzzähler und modulierter Träger}
{Frequenzzähler und unmodulierter Träger}
\end{QQuestion}

}
\only<2>{
\begin{QQuestion}{AI503}{Welche Konfiguration gewährleistet die höchste Genauigkeit bei der Prüfung der Trägerfrequenz eines FM-Senders?}{Absorptionsfrequenzmesser und modulierter Träger}
{Oszilloskop und unmodulierter Träger}
{Frequenzzähler und modulierter Träger}
{\textbf{\textcolor{DARCgreen}{Frequenzzähler und unmodulierter Träger}}}
\end{QQuestion}

}
\end{frame}

\begin{frame}
\only<1>{
\begin{QQuestion}{AI505}{Benutzt man bei einem Frequenzzähler eine Torzeit von 10 s anstelle von 1 s erhöht sich~...}{die Empfindlichkeit.}
{die Langzeitstabilität.}
{die Auflösung.}
{die Stabilität.}
\end{QQuestion}

}
\only<2>{
\begin{QQuestion}{AI505}{Benutzt man bei einem Frequenzzähler eine Torzeit von 10 s anstelle von 1 s erhöht sich~...}{die Empfindlichkeit.}
{die Langzeitstabilität.}
{\textbf{\textcolor{DARCgreen}{die Auflösung.}}}
{die Stabilität.}
\end{QQuestion}

}
\end{frame}%ENDCONTENT


\section{Frequenzgenauigkeit}
\label{section:frequenzgenauigkeit}
\begin{frame}%STARTCONTENT

\only<1>{
\begin{QQuestion}{AA115}{Eine Genauigkeit von \qty{1}{\ppm} bei einer Frequenz von \qty{435}{\MHz} entspricht~...}{\qty{43,5}{\Hz}.}
{\qty{435}{\Hz}.}
{\qty{4,35}{\MHz}.}
{\qty{4,35}{\kHz}.}
\end{QQuestion}

}
\only<2>{
\begin{QQuestion}{AA115}{Eine Genauigkeit von \qty{1}{\ppm} bei einer Frequenz von \qty{435}{\MHz} entspricht~...}{\qty{43,5}{\Hz}.}
{\textbf{\textcolor{DARCgreen}{\qty{435}{\Hz}.}}}
{\qty{4,35}{\MHz}.}
{\qty{4,35}{\kHz}.}
\end{QQuestion}

}
\end{frame}

\begin{frame}
\frametitle{Lösungsweg}
\begin{itemize}
  \item gegeben: $f = 435MHz$
  \item gesucht: $1pmm$ von $f$
  \end{itemize}
    \pause
    $435MHz \cdot frac{1}{10^6} = \frac{435\cdot \cancel{10^6}Hz}{\cancel{10^6}} = 435Hz$



\end{frame}

\begin{frame}
\only<1>{
\begin{QQuestion}{AA116}{Die Frequenzerzeugung eines Senders hat eine Genauigkeit von \qty{10}{\ppm}. Die digitale Anzeige zeigt eine Sendefrequenz von 14,200.\qty{000}{\MHz} an. In welchen Grenzen kann sich die tatsächliche Frequenz bewegen?}{Zwischen \qtyrange{14,199858}{14,200142}{\MHz}}
{Zwischen \qtyrange{14,199986}{14,200014}{\MHz}}
{Zwischen \qtyrange{14,199990}{14,200010}{\MHz}}
{Zwischen \qtyrange{14,198580}{14,201420}{\MHz}}
\end{QQuestion}

}
\only<2>{
\begin{QQuestion}{AA116}{Die Frequenzerzeugung eines Senders hat eine Genauigkeit von \qty{10}{\ppm}. Die digitale Anzeige zeigt eine Sendefrequenz von 14,200.\qty{000}{\MHz} an. In welchen Grenzen kann sich die tatsächliche Frequenz bewegen?}{\textbf{\textcolor{DARCgreen}{Zwischen \qtyrange{14,199858}{14,200142}{\MHz}}}}
{Zwischen \qtyrange{14,199986}{14,200014}{\MHz}}
{Zwischen \qtyrange{14,199990}{14,200010}{\MHz}}
{Zwischen \qtyrange{14,198580}{14,201420}{\MHz}}
\end{QQuestion}

}
\end{frame}

\begin{frame}
\frametitle{Lösungsweg}
\begin{itemize}
  \item gegeben: $f = 14,200.000MHz$
  \item gegeben: $\textrm{Abw.} = 10ppm$
  \item gesucht: $f_{min}, f_{max}$
  \end{itemize}
    \pause
    $f_{min} = f -- f \cdot \frac{10}{10^6} = 14,2MHz -- \frac{14,2\cdot \cancel{10^6}Hz\cdot 10}{\cancel{10^6}} = 14,2MHz -- 142Hz = 14,199858MHz$

$f_{max} = f + f \cdot \frac{10}{10^6} = 14,2MHz + \frac{14,2\cdot \cancel{10^6}Hz\cdot 10}{\cancel{10^6}} = 14,2MHz + 142Hz = 14,200142MHz$



\end{frame}

\begin{frame}
\only<1>{
\begin{QQuestion}{AI506}{Die relative Ungenauigkeit der digitalen Anzeige eines Empfängers beträgt \qty{0,01}{\percent}. Um wieviel Hertz kann die angezeigte Frequenz bei \qty{29}{\MHz} maximal abweichen?}{\qty{290}{\Hz}}
{\qty{2900}{\Hz}}
{\qty{29}{\Hz}}
{\qty{29}{\kHz}}
\end{QQuestion}

}
\only<2>{
\begin{QQuestion}{AI506}{Die relative Ungenauigkeit der digitalen Anzeige eines Empfängers beträgt \qty{0,01}{\percent}. Um wieviel Hertz kann die angezeigte Frequenz bei \qty{29}{\MHz} maximal abweichen?}{\qty{290}{\Hz}}
{\textbf{\textcolor{DARCgreen}{\qty{2900}{\Hz}}}}
{\qty{29}{\Hz}}
{\qty{29}{\kHz}}
\end{QQuestion}

}
\end{frame}

\begin{frame}
\frametitle{Lösungsweg}
\begin{itemize}
  \item gegeben: $f = 29MHz$
  \item gegeben: $\textrm{Abw.} = 0,01\%$
  \item gesucht: $\Delta f$
  \end{itemize}
    \pause
    $\Delta f = 29MHz \cdot 0,01\% = 29\cdot \cancel{10^6}Hz \cdot 100\cdot \cancel{10^{-6}} = 2900Hz$



\end{frame}

\begin{frame}
\only<1>{
\begin{QQuestion}{AI507}{Ein TRX mit einem eingebauten OCXO besitzt eine Anzeigegenauigkeit von $\pm$\qty{0,00001}{\percent}. Wie groß ist die maximale Abweichung, wenn eine Frequenz von \qty{14100}{\kHz} angezeigt wird?}{$\pm$ \qty{1,410}{\Hz}}
{$\pm$ \qty{0,141}{\Hz}}
{$\pm$ \qty{1,141}{\Hz}}
{$\pm$ \qty{114,1}{\Hz}}
\end{QQuestion}

}
\only<2>{
\begin{QQuestion}{AI507}{Ein TRX mit einem eingebauten OCXO besitzt eine Anzeigegenauigkeit von $\pm$\qty{0,00001}{\percent}. Wie groß ist die maximale Abweichung, wenn eine Frequenz von \qty{14100}{\kHz} angezeigt wird?}{\textbf{\textcolor{DARCgreen}{$\pm$ \qty{1,410}{\Hz}}}}
{$\pm$ \qty{0,141}{\Hz}}
{$\pm$ \qty{1,141}{\Hz}}
{$\pm$ \qty{114,1}{\Hz}}
\end{QQuestion}

}
\end{frame}

\begin{frame}
\frametitle{Lösungsweg}
\begin{itemize}
  \item gegeben: $f = 14100kHz$
  \item gegeben: $\textrm{Abw.} = \pm0,00001\%$
  \item gesucht: $\Delta f$
  \end{itemize}
    \pause
    $\Delta f = 14100kHz \cdot 0,00001\% = 14,1\cdot \cancel{10^6}Hz \cdot 0,1\cdot \cancel{10^{-6}} = 1,41Hz$



\end{frame}

\begin{frame}
\only<1>{
\begin{QQuestion}{AI508}{Ein Frequenzzähler misst auf $\pm$\qty{1}{\ppm} genau. Ist der Zähler auf den \qty{100}{\MHz}-Bereich eingestellt, so ist am oberen Ende dieses Bereiches eine Ungenauigkeit zu erwarten von~...}{$\pm$~\qty{1}{\Hz}.}
{$\pm$~\qty{10}{\Hz}.}
{$\pm$~\qty{1}{\kHz}.}
{$\pm$~\qty{100}{\Hz}.}
\end{QQuestion}

}
\only<2>{
\begin{QQuestion}{AI508}{Ein Frequenzzähler misst auf $\pm$\qty{1}{\ppm} genau. Ist der Zähler auf den \qty{100}{\MHz}-Bereich eingestellt, so ist am oberen Ende dieses Bereiches eine Ungenauigkeit zu erwarten von~...}{$\pm$~\qty{1}{\Hz}.}
{$\pm$~\qty{10}{\Hz}.}
{$\pm$~\qty{1}{\kHz}.}
{\textbf{\textcolor{DARCgreen}{$\pm$~\qty{100}{\Hz}.}}}
\end{QQuestion}

}
\end{frame}

\begin{frame}
\frametitle{Lösungsweg}
\begin{itemize}
  \item gegeben: $f = 100MHz$
  \item gegeben: $\textrm{Abw.} = \pm1ppm$
  \item gesucht: $\Delta f$
  \end{itemize}
    \pause
    $\Delta f = 100MHz \cdot \frac{1}{10^6} = \frac{100\cdot \cancel{10^6}Hz}{\cancel{10^6}} = 100Hz$



\end{frame}

\begin{frame}
\only<1>{
\begin{QQuestion}{AI509}{Mit einem auf \qty{10}{\ppm} genauen digitalen Frequenzzähler wird eine Frequenz von \qty{145}{\MHz} gemessen. In welchem Bereich liegt der vom Zähler angezeigte Frequenzwert?}{\qty{144,999275}{\MHz}~-~\qty{145,000725}{\MHz}}
{\qty{144,99565}{\MHz}~-~\qty{145,00435}{\MHz}}
{\qty{144,9971}{\MHz}~-~\qty{145,0029}{\MHz}}
{\qty{144,99855}{\MHz}~-~\qty{145,00145}{\MHz}}
\end{QQuestion}

}
\only<2>{
\begin{QQuestion}{AI509}{Mit einem auf \qty{10}{\ppm} genauen digitalen Frequenzzähler wird eine Frequenz von \qty{145}{\MHz} gemessen. In welchem Bereich liegt der vom Zähler angezeigte Frequenzwert?}{\qty{144,999275}{\MHz}~-~\qty{145,000725}{\MHz}}
{\qty{144,99565}{\MHz}~-~\qty{145,00435}{\MHz}}
{\qty{144,9971}{\MHz}~-~\qty{145,0029}{\MHz}}
{\textbf{\textcolor{DARCgreen}{\qty{144,99855}{\MHz}~-~\qty{145,00145}{\MHz}}}}
\end{QQuestion}

}
\end{frame}

\begin{frame}
\frametitle{Lösungsweg}
\begin{itemize}
  \item gegeben: $f = 145MHz$
  \item gegeben: $\textrm{Abw.} = 10ppm$
  \item gesucht: $f_{min},f_{max}$
  \end{itemize}
    \pause
    $\Delta f = 145MHz \cdot \frac{10}{10^6} = \frac{145\cdot \cancel{10^6}Hz \cdot 10}{\cancel{10^6}} = 1450Hz$
    \pause
    $f_{min} = f -- \Delta f = 145MHz -- 1450Hz = 144,99855MHz$

$f_{max} = f -- \Delta f = 145MHz + 1450Hz = 145,00145MHz$



\end{frame}

\begin{frame}
\only<1>{
\begin{QQuestion}{AI510}{Ein Transceivers zeigt Frequenzen im \qty{2}{\m}-Band auf \qty{1}{\ppm} genau an. Um wie viel kHz muss die an diesem Transceiver bei SSB-Betrieb (USB) eingestellte Sendefrequenz (Frequenz des unterdrückten Trägers) unterhalb von \qty{144,400}{\MHz} liegen, um das dort beginnende Bakensegment zu schützen, wenn die übertragene NF auf den Bereich \qty{300}{\Hz} bis \qty{2,7}{\kHz} beschränkt ist?}{\qty{0,144}{\kHz}}
{\qty{2,844}{\kHz}}
{\qty{1,42}{\kHz}}
{\qty{2,70}{\kHz}}
\end{QQuestion}

}
\only<2>{
\begin{QQuestion}{AI510}{Ein Transceivers zeigt Frequenzen im \qty{2}{\m}-Band auf \qty{1}{\ppm} genau an. Um wie viel kHz muss die an diesem Transceiver bei SSB-Betrieb (USB) eingestellte Sendefrequenz (Frequenz des unterdrückten Trägers) unterhalb von \qty{144,400}{\MHz} liegen, um das dort beginnende Bakensegment zu schützen, wenn die übertragene NF auf den Bereich \qty{300}{\Hz} bis \qty{2,7}{\kHz} beschränkt ist?}{\qty{0,144}{\kHz}}
{\textbf{\textcolor{DARCgreen}{\qty{2,844}{\kHz}}}}
{\qty{1,42}{\kHz}}
{\qty{2,70}{\kHz}}
\end{QQuestion}

}
\end{frame}

\begin{frame}
\frametitle{Lösungsweg}
\begin{itemize}
  \item gegeben: $f = 144,400MHz$
  \item gegeben: $\textrm{Abw.} = 1ppm$
  \item gegeben: $f_{B,max} = 2,7kHz$
  \item gesucht: $f_{B,max,Abw}$
  \end{itemize}
    \pause
    $\Delta f = 144,4MHz \cdot \frac{1}{10^6} = \frac{144,4\cdot \cancel{10^6}Hz}{\cancel{10^6}} = 144,4Hz$
    \pause
    $f_{B,max,Abw} = f_{B,max} + \Delta f = 2,7kHz + 144,4Hz = 2,8444kHz$



\end{frame}%ENDCONTENT


\title{DARC Amateurfunklehrgang Klasse A}
\author{Sender}
\institute{Deutscher Amateur Radio Club e.\,V.}
\begin{frame}
\maketitle
\end{frame}

\section{Modulatoren}
\label{section:modulatoren}
\begin{frame}%STARTCONTENT

\only<1>{
\begin{PQuestion}{AD503}{Bei dieser Schaltung ist der mit X bezeichnete Anschluss~...}{der Ausgang für das NF-Signal.}
{der Ausgang für eine Regelspannung.}
{der Ausgang für das Oszillatorsignal.}
{der Ausgang für das ZF-Signal.}
{\DARCimage{1.0\linewidth}{142include}}\end{PQuestion}

}
\only<2>{
\begin{PQuestion}{AD503}{Bei dieser Schaltung ist der mit X bezeichnete Anschluss~...}{der Ausgang für das NF-Signal.}
{\textbf{\textcolor{DARCgreen}{der Ausgang für eine Regelspannung.}}}
{der Ausgang für das Oszillatorsignal.}
{der Ausgang für das ZF-Signal.}
{\DARCimage{1.0\linewidth}{142include}}\end{PQuestion}

}
\end{frame}

\begin{frame}
\only<1>{
\begin{PQuestion}{AD507}{Bei dieser Schaltung handelt es sich um einen~...}{LSB-Modulator.}
{USB-Modulator.}
{FM-Modulator.}
{AM-Modulator.}
{\DARCimage{1.0\linewidth}{772include}}\end{PQuestion}

}
\only<2>{
\begin{PQuestion}{AD507}{Bei dieser Schaltung handelt es sich um einen~...}{LSB-Modulator.}
{USB-Modulator.}
{FM-Modulator.}
{\textbf{\textcolor{DARCgreen}{AM-Modulator.}}}
{\DARCimage{1.0\linewidth}{772include}}\end{PQuestion}

}
\end{frame}

\begin{frame}
\only<1>{
\begin{PQuestion}{AD508}{Bei dieser Schaltung handelt es sich um einen Modulator zur Erzeugung von~...}{AM-Signalen mit unterdrücktem Träger.}
{phasenmodulierten Signalen.}
{frequenzmodulierten Signalen.}
{AM-Signalen.}
{\DARCimage{1.0\linewidth}{155include}}\end{PQuestion}

}
\only<2>{
\begin{PQuestion}{AD508}{Bei dieser Schaltung handelt es sich um einen Modulator zur Erzeugung von~...}{AM-Signalen mit unterdrücktem Träger.}
{phasenmodulierten Signalen.}
{\textbf{\textcolor{DARCgreen}{frequenzmodulierten Signalen.}}}
{AM-Signalen.}
{\DARCimage{1.0\linewidth}{155include}}\end{PQuestion}

}
\end{frame}

\begin{frame}
\only<1>{
\begin{PQuestion}{AD509}{Was ermöglicht die abgebildete Schaltung?}{Die Erzeugung von Phasenmodulation}
{Die HF-Pegelbegrenzung und HF-Pegeleinstellung bei FM-Funkgeräten}
{Die Erzeugung von Amplitudenmodulation}
{Die Hubbegrenzung und Hubeinstellung bei FM-Funkgeräten}
{\DARCimage{1.0\linewidth}{44include}}\end{PQuestion}

}
\only<2>{
\begin{PQuestion}{AD509}{Was ermöglicht die abgebildete Schaltung?}{Die Erzeugung von Phasenmodulation}
{Die HF-Pegelbegrenzung und HF-Pegeleinstellung bei FM-Funkgeräten}
{Die Erzeugung von Amplitudenmodulation}
{\textbf{\textcolor{DARCgreen}{Die Hubbegrenzung und Hubeinstellung bei FM-Funkgeräten}}}
{\DARCimage{1.0\linewidth}{44include}}\end{PQuestion}

}
\end{frame}

\begin{frame}
\only<1>{
\begin{QQuestion}{AE206}{Welche Baugruppe sollte für die analoge Erzeugung eines unterdrückten Zweiseitenband-Trägersignals verwendet werden?}{Demodulator}
{Quarzfilter}
{Bandfilter}
{Balancemischer}
\end{QQuestion}

}
\only<2>{
\begin{QQuestion}{AE206}{Welche Baugruppe sollte für die analoge Erzeugung eines unterdrückten Zweiseitenband-Trägersignals verwendet werden?}{Demodulator}
{Quarzfilter}
{Bandfilter}
{\textbf{\textcolor{DARCgreen}{Balancemischer}}}
\end{QQuestion}

}
\end{frame}

\begin{frame}
\only<1>{
\begin{QQuestion}{AF302}{Welcher Mischertyp ist am besten geeignet, um ein Doppelseitenbandsignal mit unterdrücktem Träger zu erzeugen?}{Ein Mischer mit einem einzelnen FET}
{Ein Balancemischer}
{Ein Mischer mit einer Varaktordiode}
{Ein quarzgesteuerter Mischer}
\end{QQuestion}

}
\only<2>{
\begin{QQuestion}{AF302}{Welcher Mischertyp ist am besten geeignet, um ein Doppelseitenbandsignal mit unterdrücktem Träger zu erzeugen?}{Ein Mischer mit einem einzelnen FET}
{\textbf{\textcolor{DARCgreen}{Ein Balancemischer}}}
{Ein Mischer mit einer Varaktordiode}
{Ein quarzgesteuerter Mischer}
\end{QQuestion}

}
\end{frame}

\begin{frame}
\only<1>{
\begin{QQuestion}{AF303}{Wie kann mit analoger Technologie ein SSB-Signal erzeugt werden?}{In einem Balancemodulator wird ein Zweiseitenband-Signal erzeugt. Das Seitenbandfilter selektiert ein Seitenband heraus.}
{In einem Balancemodulator wird ein Zweiseitenband-Signal erzeugt. Ein auf die Trägerfrequenz abgestimmter Saugkreis filtert den Träger aus.}
{In einem Balancemodulator wird ein Zweiseitenband-Signal erzeugt. Ein auf die Trägerfrequenz abgestimmter Sperrkreis filtert den Träger aus.}
{In einem Balancemodulator wird ein Zweiseitenband-Signal erzeugt. In einem Frequenzteiler wird ein Seitenband abgespalten.}
\end{QQuestion}

}
\only<2>{
\begin{QQuestion}{AF303}{Wie kann mit analoger Technologie ein SSB-Signal erzeugt werden?}{\textbf{\textcolor{DARCgreen}{In einem Balancemodulator wird ein Zweiseitenband-Signal erzeugt. Das Seitenbandfilter selektiert ein Seitenband heraus.}}}
{In einem Balancemodulator wird ein Zweiseitenband-Signal erzeugt. Ein auf die Trägerfrequenz abgestimmter Saugkreis filtert den Träger aus.}
{In einem Balancemodulator wird ein Zweiseitenband-Signal erzeugt. Ein auf die Trägerfrequenz abgestimmter Sperrkreis filtert den Träger aus.}
{In einem Balancemodulator wird ein Zweiseitenband-Signal erzeugt. In einem Frequenzteiler wird ein Seitenband abgespalten.}
\end{QQuestion}

}
\end{frame}

\begin{frame}
\only<1>{
\begin{QQuestion}{AF304}{Bei üblichen analogen Methoden zur Aufbereitung eines SSB-Signals werden~...}{der Träger unterdrückt und ein Seitenband hinzugesetzt.}
{der Träger hinzugesetzt und ein Seitenband ausgefiltert.}
{der Träger unterdrückt und ein Seitenband ausgefiltert.}
{der Träger unterdrückt und beide Seitenbänder ausgefiltert.}
\end{QQuestion}

}
\only<2>{
\begin{QQuestion}{AF304}{Bei üblichen analogen Methoden zur Aufbereitung eines SSB-Signals werden~...}{der Träger unterdrückt und ein Seitenband hinzugesetzt.}
{der Träger hinzugesetzt und ein Seitenband ausgefiltert.}
{\textbf{\textcolor{DARCgreen}{der Träger unterdrückt und ein Seitenband ausgefiltert.}}}
{der Träger unterdrückt und beide Seitenbänder ausgefiltert.}
\end{QQuestion}

}
\end{frame}

\begin{frame}
\only<1>{
\begin{PQuestion}{AF305}{Dieses Blockschaltbild zeigt einen SSB-Sender. Die Stufe bei \glqq?\grqq{} ist ein...}{ZF-Notchfilter zur Unterdrückung des unerwünschten Seitenbands.}
{RC-Hochpass zur Unterdrückung des unteren Seitenbands.}
{RL-Tiefpass zur Unterdrückung des oberen Seitenbands.}
{Quarzfilter als Bandpass für das gewünschte Seitenband.}
{\DARCimage{1.0\linewidth}{98include}}\end{PQuestion}

}
\only<2>{
\begin{PQuestion}{AF305}{Dieses Blockschaltbild zeigt einen SSB-Sender. Die Stufe bei \glqq?\grqq{} ist ein...}{ZF-Notchfilter zur Unterdrückung des unerwünschten Seitenbands.}
{RC-Hochpass zur Unterdrückung des unteren Seitenbands.}
{RL-Tiefpass zur Unterdrückung des oberen Seitenbands.}
{\textbf{\textcolor{DARCgreen}{Quarzfilter als Bandpass für das gewünschte Seitenband.}}}
{\DARCimage{1.0\linewidth}{98include}}\end{PQuestion}

}
\end{frame}

\begin{frame}
\only<1>{
\begin{PQuestion}{AF306}{Welches Schaltungsteil ist in der folgenden Blockschaltung am Ausgang des NF-Verstärkers angeschlossen?}{symmetrisches Filter}
{Balancemischer}
{Dynamikkompressor}
{DSB-Filter}
{\DARCimage{1.0\linewidth}{500include}}\end{PQuestion}

}
\only<2>{
\begin{PQuestion}{AF306}{Welches Schaltungsteil ist in der folgenden Blockschaltung am Ausgang des NF-Verstärkers angeschlossen?}{symmetrisches Filter}
{\textbf{\textcolor{DARCgreen}{Balancemischer}}}
{Dynamikkompressor}
{DSB-Filter}
{\DARCimage{1.0\linewidth}{500include}}\end{PQuestion}

}
\end{frame}

\begin{frame}
\only<1>{
\begin{PQuestion}{AF307}{Die folgende Blockschaltung zeigt eine SSB-Aufbereitung mit einem \qty{9}{\MHz}-Quarzfilter. Welche Frequenz wird in der Schalterstellung USB mit der NF gemischt?}{\qty{9,0000}{\MHz}}
{\qty{8,9970}{\MHz}}
{\qty{8,9985}{\MHz}}
{\qty{9,0030}{\MHz}}
{\DARCimage{1.0\linewidth}{39include}}\end{PQuestion}

}
\only<2>{
\begin{PQuestion}{AF307}{Die folgende Blockschaltung zeigt eine SSB-Aufbereitung mit einem \qty{9}{\MHz}-Quarzfilter. Welche Frequenz wird in der Schalterstellung USB mit der NF gemischt?}{\qty{9,0000}{\MHz}}
{\qty{8,9970}{\MHz}}
{\textbf{\textcolor{DARCgreen}{\qty{8,9985}{\MHz}}}}
{\qty{9,0030}{\MHz}}
{\DARCimage{1.0\linewidth}{39include}}\end{PQuestion}

}
\end{frame}

\begin{frame}
\frametitle{Lösungsweg}
\begin{itemize}
  \item gegeben: $f_Q = 9MHz$
  \item gegeben: $f_{LSB} = 9,0015MHz$
  \item gesucht: $f_{USB}$
  \end{itemize}
    \pause
    $f_{USB} = f_Q -- (f_{LSB} -- f_Q) = 9MHz -- (9,0015MHz -- 9MHz) = 9MHz -- 0,0015MHz =8,9985MHz$



\end{frame}

\begin{frame}
\only<1>{
\begin{PQuestion}{AF308}{Bei dieser Schaltung handelt es sich um einen Modulator zur Erzeugung von~...}{AM-Signalen mit unterdrücktem Träger.}
{phasenmodulierten Signalen.}
{frequenzmodulierten Signalen.}
{LSB-Signalen.}
{\DARCimage{1.0\linewidth}{759include}}\end{PQuestion}

}
\only<2>{
\begin{PQuestion}{AF308}{Bei dieser Schaltung handelt es sich um einen Modulator zur Erzeugung von~...}{\textbf{\textcolor{DARCgreen}{AM-Signalen mit unterdrücktem Träger.}}}
{phasenmodulierten Signalen.}
{frequenzmodulierten Signalen.}
{LSB-Signalen.}
{\DARCimage{1.0\linewidth}{759include}}\end{PQuestion}

}
\end{frame}

\begin{frame}
\only<1>{
\begin{PQuestion}{AF309}{Wozu dienen $R_1$ und $C_1$ bei dieser Schaltung?  }{Sie dienen zur Einstellung des Modulationsgrades des erzeugten DSB-Signals.}
{Sie dienen zum Ausgleich von Frequenzgangs- und Laufzeitunterschieden.}
{Sie dienen zur Einstellung des Frequenzhubes mit Hilfe der ersten Trägernullstelle.}
{Sie dienen zur Einstellung der Trägerunterdrückung nach Betrag und Phase.}
{\DARCimage{1.0\linewidth}{762include}}\end{PQuestion}

}
\only<2>{
\begin{PQuestion}{AF309}{Wozu dienen $R_1$ und $C_1$ bei dieser Schaltung?  }{Sie dienen zur Einstellung des Modulationsgrades des erzeugten DSB-Signals.}
{Sie dienen zum Ausgleich von Frequenzgangs- und Laufzeitunterschieden.}
{Sie dienen zur Einstellung des Frequenzhubes mit Hilfe der ersten Trägernullstelle.}
{\textbf{\textcolor{DARCgreen}{Sie dienen zur Einstellung der Trägerunterdrückung nach Betrag und Phase.}}}
{\DARCimage{1.0\linewidth}{762include}}\end{PQuestion}

}
\end{frame}

\begin{frame}
\only<1>{
\begin{PQuestion}{AF310}{Dieser Schaltungsauszug ist Teil eines Senders. Welche Funktion hat die Diode?}{Sie stabilisiert die Betriebsspannung für den Oszillator, um diesen von der Stromversorgung der anderen Stufen zu entkoppeln.}
{Sie beeinflusst die Resonanzfrequenz des Schwingkreises in Abhängigkeit des NF-Spannungsverlaufs und moduliert so die Oszillatorfrequenz.}
{Sie begrenzt die Amplituden des Eingangssignals und vermeidet so die Übersteuerung der Oszillatorstufe.}
{Sie dient zur Erzeugung von Amplitudenmodulation in Abhängigkeit von den Frequenzen im Basisband.}
{\DARCimage{1.0\linewidth}{158include}}\end{PQuestion}

}
\only<2>{
\begin{PQuestion}{AF310}{Dieser Schaltungsauszug ist Teil eines Senders. Welche Funktion hat die Diode?}{Sie stabilisiert die Betriebsspannung für den Oszillator, um diesen von der Stromversorgung der anderen Stufen zu entkoppeln.}
{\textbf{\textcolor{DARCgreen}{Sie beeinflusst die Resonanzfrequenz des Schwingkreises in Abhängigkeit des NF-Spannungsverlaufs und moduliert so die Oszillatorfrequenz.}}}
{Sie begrenzt die Amplituden des Eingangssignals und vermeidet so die Übersteuerung der Oszillatorstufe.}
{Sie dient zur Erzeugung von Amplitudenmodulation in Abhängigkeit von den Frequenzen im Basisband.}
{\DARCimage{1.0\linewidth}{158include}}\end{PQuestion}

}
\end{frame}

\begin{frame}
\only<1>{
\begin{QQuestion}{AD510}{Welche Signale stehen am Ausgang eines symmetrisch eingestellten Balancemischers an?}{Der vollständige Träger}
{Viele Mischprodukte}
{Der verringerte Träger und ein Seitenband}
{Die zwei Seitenbänder}
\end{QQuestion}

}
\only<2>{
\begin{QQuestion}{AD510}{Welche Signale stehen am Ausgang eines symmetrisch eingestellten Balancemischers an?}{Der vollständige Träger}
{Viele Mischprodukte}
{Der verringerte Träger und ein Seitenband}
{\textbf{\textcolor{DARCgreen}{Die zwei Seitenbänder}}}
\end{QQuestion}

}
\end{frame}%ENDCONTENT


\section{Nicht-sinusförmige Signale}
\label{section:nicht_sinus_signale}
\begin{frame}%STARTCONTENT

\only<1>{
\begin{PQuestion}{AB403}{Eine periodische Schwingung, die wie das folgende Signal aussieht, besteht~...}{aus der Grundschwingung mit ganzzahligen Vielfachen dieser Frequenz (Oberschwingungen).}
{aus der Grundschwingung und Teilen dieser Frequenz (Unterschwingungen).}
{aus der Grundschwingung ohne weitere Frequenzen.}
{aus der Grundschwingung mit zufälligen Frequenzschwankungen.}
{\DARCimage{1.0\linewidth}{595include}}\end{PQuestion}

}
\only<2>{
\begin{PQuestion}{AB403}{Eine periodische Schwingung, die wie das folgende Signal aussieht, besteht~...}{\textbf{\textcolor{DARCgreen}{aus der Grundschwingung mit ganzzahligen Vielfachen dieser Frequenz (Oberschwingungen).}}}
{aus der Grundschwingung und Teilen dieser Frequenz (Unterschwingungen).}
{aus der Grundschwingung ohne weitere Frequenzen.}
{aus der Grundschwingung mit zufälligen Frequenzschwankungen.}
{\DARCimage{1.0\linewidth}{595include}}\end{PQuestion}

}
\end{frame}

\begin{frame}
\only<1>{
\begin{QQuestion}{AB401}{Was sind Harmonische?}{Harmonische sind ausschließlich die ungeradzahligen (1, 3, 5,~...) Vielfachen einer Frequenz.}
{Harmonische sind die ganzzahligen (1, 2, 3,~...) Teile einer Frequenz.}
{Harmonische sind die ganzzahligen (1, 2, 3,~...) Vielfachen einer Frequenz.}
{Harmonische sind ausschließlich die geradzahligen (2, 4, 6,~...) Teile einer Frequenz.}
\end{QQuestion}

}
\only<2>{
\begin{QQuestion}{AB401}{Was sind Harmonische?}{Harmonische sind ausschließlich die ungeradzahligen (1, 3, 5,~...) Vielfachen einer Frequenz.}
{Harmonische sind die ganzzahligen (1, 2, 3,~...) Teile einer Frequenz.}
{\textbf{\textcolor{DARCgreen}{Harmonische sind die ganzzahligen (1, 2, 3,~...) Vielfachen einer Frequenz.}}}
{Harmonische sind ausschließlich die geradzahligen (2, 4, 6,~...) Teile einer Frequenz.}
\end{QQuestion}

}
\end{frame}

\begin{frame}
\only<1>{
\begin{QQuestion}{AB402}{Die dritte Oberwelle entspricht~...}{der dritten Harmonischen.}
{der vierten Harmonischen.}
{der zweiten Harmonischen.}
{der zweiten ungeradzahligen Harmonischen.}
\end{QQuestion}

}
\only<2>{
\begin{QQuestion}{AB402}{Die dritte Oberwelle entspricht~...}{der dritten Harmonischen.}
{\textbf{\textcolor{DARCgreen}{der vierten Harmonischen.}}}
{der zweiten Harmonischen.}
{der zweiten ungeradzahligen Harmonischen.}
\end{QQuestion}

}
\end{frame}

\begin{frame}
\only<1>{
\begin{QQuestion}{AI615}{Mit welchem Messgerät kann man das Vorhandensein von Harmonischen nachweisen?}{Spektrumanalysator}
{Stehwellenmessgerät}
{Vektorieller Netzwerkanalysator (VNA)}
{Frequenzzähler}
\end{QQuestion}

}
\only<2>{
\begin{QQuestion}{AI615}{Mit welchem Messgerät kann man das Vorhandensein von Harmonischen nachweisen?}{\textbf{\textcolor{DARCgreen}{Spektrumanalysator}}}
{Stehwellenmessgerät}
{Vektorieller Netzwerkanalysator (VNA)}
{Frequenzzähler}
\end{QQuestion}

}
\end{frame}

\begin{frame}
\only<1>{
\begin{QQuestion}{AI614}{Mit welchem der folgenden Messinstrumente können die Amplituden der Harmonischen eines Signals gemessen werden? Sie können gemessen werden mit einem~...}{Breitbandpegelmesser.}
{Frequenzzähler.}
{Spektrumanalysator.}
{Multimeter.}
\end{QQuestion}

}
\only<2>{
\begin{QQuestion}{AI614}{Mit welchem der folgenden Messinstrumente können die Amplituden der Harmonischen eines Signals gemessen werden? Sie können gemessen werden mit einem~...}{Breitbandpegelmesser.}
{Frequenzzähler.}
{\textbf{\textcolor{DARCgreen}{Spektrumanalysator.}}}
{Multimeter.}
\end{QQuestion}

}
\end{frame}

\begin{frame}
\only<1>{
\begin{QQuestion}{AJ201}{Die zweite Harmonische der Frequenz \qty{3,730}{\MHz} befindet sich auf~...}{\qty{11,190}{\MHz}.}
{\qty{1,865}{\MHz}.}
{\qty{7,460}{\MHz}.}
{\qty{5,730}{\MHz}.}
\end{QQuestion}

}
\only<2>{
\begin{QQuestion}{AJ201}{Die zweite Harmonische der Frequenz \qty{3,730}{\MHz} befindet sich auf~...}{\qty{11,190}{\MHz}.}
{\qty{1,865}{\MHz}.}
{\textbf{\textcolor{DARCgreen}{\qty{7,460}{\MHz}.}}}
{\qty{5,730}{\MHz}.}
\end{QQuestion}

}
\end{frame}

\begin{frame}
\frametitle{Lösungsweg}
\begin{itemize}
  \item gegeben: $f = 3,730MHz$
  \item gesucht: $f$ der 2. Harmonischen
  \end{itemize}
    \pause
    $2 \cdot f = 2 \cdot 3,730MHz = 7,460MHz$



\end{frame}

\begin{frame}
\only<1>{
\begin{QQuestion}{AJ205}{Die zweite ungeradzahlige Harmonische der Frequenz \qty{144,690}{\MHz} ist~...}{\qty{723,450}{\MHz}.}
{\qty{289,380}{\MHz}.}
{\qty{145,000}{\MHz}.}
{\qty{434,070}{\MHz}.}
\end{QQuestion}

}
\only<2>{
\begin{QQuestion}{AJ205}{Die zweite ungeradzahlige Harmonische der Frequenz \qty{144,690}{\MHz} ist~...}{\qty{723,450}{\MHz}.}
{\qty{289,380}{\MHz}.}
{\qty{145,000}{\MHz}.}
{\textbf{\textcolor{DARCgreen}{\qty{434,070}{\MHz}.}}}
\end{QQuestion}

}
\end{frame}

\begin{frame}
\frametitle{Lösungsweg}
\begin{itemize}
  \item gegeben: $f = 144,690MHz$
  \item gesucht: $f$ als 2. ungeradzahlige Harmonische
  \end{itemize}
    \pause
    \begin{enumerate}
  \item[2] ungeradzahlige Harmonische = 3. Harmonische
  \end{enumerate}
$3 \cdot f = 3 \cdot 144,690MHz = 434,070MHz$



\end{frame}

\begin{frame}
\only<1>{
\begin{QQuestion}{AJ202}{Auf welche Frequenz müsste ein Empfänger eingestellt werden, um die dritte Harmonische einer nahen \qty{7,050}{\MHz}-Aussendung erkennen zu können?}{\qty{21,150}{\MHz}}
{\qty{14,100}{\MHz}}
{\qty{35,250}{\MHz}}
{\qty{28,200}{\MHz}}
\end{QQuestion}

}
\only<2>{
\begin{QQuestion}{AJ202}{Auf welche Frequenz müsste ein Empfänger eingestellt werden, um die dritte Harmonische einer nahen \qty{7,050}{\MHz}-Aussendung erkennen zu können?}{\textbf{\textcolor{DARCgreen}{\qty{21,150}{\MHz}}}}
{\qty{14,100}{\MHz}}
{\qty{35,250}{\MHz}}
{\qty{28,200}{\MHz}}
\end{QQuestion}

}
\end{frame}

\begin{frame}
\frametitle{Lösungsweg}
\begin{itemize}
  \item gegeben: $f = 7,050MHz$
  \item gesucht: $f$ als 3. Harmonische
  \end{itemize}
    \pause
    $3 \cdot f = 3 \cdot 7,050MHz = 21,150MHz$



\end{frame}

\begin{frame}
\only<1>{
\begin{QQuestion}{AJ206}{Auf welchen Frequenzen kann ein \qty{144,300}{\MHz} SSB-Sendesignal Störungen verursachen?}{\qty{432,900}{\MHz} und \qty{1298,700}{\MHz}}
{\qty{433,900}{\MHz} und \qty{1296,700}{\MHz}}
{\qty{438,900}{\MHz} und \qty{1290,700}{\MHz}}
{\qty{434,900}{\MHz} und \qty{1298,700}{\MHz}}
\end{QQuestion}

}
\only<2>{
\begin{QQuestion}{AJ206}{Auf welchen Frequenzen kann ein \qty{144,300}{\MHz} SSB-Sendesignal Störungen verursachen?}{\textbf{\textcolor{DARCgreen}{\qty{432,900}{\MHz} und \qty{1298,700}{\MHz}}}}
{\qty{433,900}{\MHz} und \qty{1296,700}{\MHz}}
{\qty{438,900}{\MHz} und \qty{1290,700}{\MHz}}
{\qty{434,900}{\MHz} und \qty{1298,700}{\MHz}}
\end{QQuestion}

}
\end{frame}

\begin{frame}
\frametitle{Lösungsweg}
\begin{itemize}
  \item gegeben: $f = 144,300MHz$
  \item gesucht: mehrere Harmonische
  \end{itemize}
    \pause
    $$\begin{align}\notag 2 \cdot 144,300MHz &= 288,600MHz\\ \notag 3 \cdot 144,300MHz &= \bold{432,900MHz}\\ \notag &\vdots\\ \notag 9 \cdot 144,300MHz &= \bold{1298,700MHz}\end{align}$$



\end{frame}%ENDCONTENT


\section{Leistungsverstärker}
\label{section:leistungsvertaerker}
\begin{frame}%STARTCONTENT

\only<1>{
\begin{PQuestion}{AF412}{Welche Art von Schaltung wird im folgenden Bild dargestellt? Es handelt sich um einen~...}{modulierbaren Oszillator.}
{selektiven Hochfrequenzverstärker.}
{Breitband-Gegentaktverstärker.}
{Breitband-Frequenzverdoppler.}
{\DARCimage{1.0\linewidth}{491include}}\end{PQuestion}

}
\only<2>{
\begin{PQuestion}{AF412}{Welche Art von Schaltung wird im folgenden Bild dargestellt? Es handelt sich um einen~...}{modulierbaren Oszillator.}
{selektiven Hochfrequenzverstärker.}
{\textbf{\textcolor{DARCgreen}{Breitband-Gegentaktverstärker.}}}
{Breitband-Frequenzverdoppler.}
{\DARCimage{1.0\linewidth}{491include}}\end{PQuestion}

}
\end{frame}

\begin{frame}
\only<1>{
\begin{PQuestion}{AF408}{Worum handelt es sich bei dieser Schaltung?}{Es handelt sich um einen selektiven HF-Verstärker.}
{Es handelt sich um einen selektiven Mischer.}
{Es handelt sich um einen breitbandigen NF-Verstärker.}
{Es handelt sich um einen frequenzvervielfachenden Oszillator.}
{\DARCimage{1.0\linewidth}{778include}}\end{PQuestion}

}
\only<2>{
\begin{PQuestion}{AF408}{Worum handelt es sich bei dieser Schaltung?}{\textbf{\textcolor{DARCgreen}{Es handelt sich um einen selektiven HF-Verstärker.}}}
{Es handelt sich um einen selektiven Mischer.}
{Es handelt sich um einen breitbandigen NF-Verstärker.}
{Es handelt sich um einen frequenzvervielfachenden Oszillator.}
{\DARCimage{1.0\linewidth}{778include}}\end{PQuestion}

}
\end{frame}

\begin{frame}
\only<1>{
\begin{PQuestion}{AF413}{Worum handelt es sich bei dieser Schaltung? Es handelt sich um einen...}{zweistufigen Breitband-HF-Verstärker.}
{selektiven Hochfrequenzverstärker.}
{Gegentakt-Verstärker im B-Betrieb.}
{zweistufigen LC-Oszillator.}
{\DARCimage{1.0\linewidth}{764include}}\end{PQuestion}

}
\only<2>{
\begin{PQuestion}{AF413}{Worum handelt es sich bei dieser Schaltung? Es handelt sich um einen...}{\textbf{\textcolor{DARCgreen}{zweistufigen Breitband-HF-Verstärker.}}}
{selektiven Hochfrequenzverstärker.}
{Gegentakt-Verstärker im B-Betrieb.}
{zweistufigen LC-Oszillator.}
{\DARCimage{1.0\linewidth}{764include}}\end{PQuestion}

}
\end{frame}

\begin{frame}
\only<1>{
\begin{PQuestion}{AF409}{Welchem Zweck dient die Anzapfung an X in der folgenden Schaltung?}{Sie bewirkt die notwendige Entkopplung für den Schwingungseinsatz der Oszillatorstufe.}
{Sie ermöglicht die Dreipunkt-Rückkopplung des Oszillators.}
{Sie dient zur Anpassung der Eingangsimpedanz dieser Stufe an die vorgelagerte Stufe.}
{Sie bewirkt eine stärkere Bedämpfung des Eingangsschwingkreises.}
{\DARCimage{1.0\linewidth}{779include}}\end{PQuestion}

}
\only<2>{
\begin{PQuestion}{AF409}{Welchem Zweck dient die Anzapfung an X in der folgenden Schaltung?}{Sie bewirkt die notwendige Entkopplung für den Schwingungseinsatz der Oszillatorstufe.}
{Sie ermöglicht die Dreipunkt-Rückkopplung des Oszillators.}
{\textbf{\textcolor{DARCgreen}{Sie dient zur Anpassung der Eingangsimpedanz dieser Stufe an die vorgelagerte Stufe.}}}
{Sie bewirkt eine stärkere Bedämpfung des Eingangsschwingkreises.}
{\DARCimage{1.0\linewidth}{779include}}\end{PQuestion}

}
\end{frame}

\begin{frame}
\only<1>{
\begin{PQuestion}{AF410}{Welchem Zweck dienen $C_1$ und $C_2$ in der folgenden Schaltung? Sie dienen zur...}{Impedanzanpassung. }
{Verhinderung der Schwingneigung.}
{Realisierung einer kapazitiven Dreipunktschaltung für den Oszillator.}
{Unterdrückung von Oberschwingungen.}
{\DARCimage{1.0\linewidth}{780include}}\end{PQuestion}

}
\only<2>{
\begin{PQuestion}{AF410}{Welchem Zweck dienen $C_1$ und $C_2$ in der folgenden Schaltung? Sie dienen zur...}{\textbf{\textcolor{DARCgreen}{Impedanzanpassung. }}}
{Verhinderung der Schwingneigung.}
{Realisierung einer kapazitiven Dreipunktschaltung für den Oszillator.}
{Unterdrückung von Oberschwingungen.}
{\DARCimage{1.0\linewidth}{780include}}\end{PQuestion}

}
\end{frame}

\begin{frame}
\only<1>{
\begin{PQuestion}{AF414}{Wozu dient der Transformator $T_1$ der folgenden Schaltung?}{Er dient der Anpassung des Ausgangswiderstandes der Emitterschaltung an den Eingang der folgenden Emitterschaltung.}
{Er dient der Anpassung des Ausgangswiderstandes der Emitterschaltung an den Eingang der folgenden Kollektorschaltung.}
{Er dient der Anpassung des Ausgangswiderstandes der Kollektorschaltung an den Eingang der folgenden Emitterschaltung.}
{Er dient der Anpassung des Ausgangswiderstandes der Kollektorschaltung an den Eingang der folgenden PA.}
{\DARCimage{1.0\linewidth}{765include}}\end{PQuestion}

}
\only<2>{
\begin{PQuestion}{AF414}{Wozu dient der Transformator $T_1$ der folgenden Schaltung?}{\textbf{\textcolor{DARCgreen}{Er dient der Anpassung des Ausgangswiderstandes der Emitterschaltung an den Eingang der folgenden Emitterschaltung.}}}
{Er dient der Anpassung des Ausgangswiderstandes der Emitterschaltung an den Eingang der folgenden Kollektorschaltung.}
{Er dient der Anpassung des Ausgangswiderstandes der Kollektorschaltung an den Eingang der folgenden Emitterschaltung.}
{Er dient der Anpassung des Ausgangswiderstandes der Kollektorschaltung an den Eingang der folgenden PA.}
{\DARCimage{1.0\linewidth}{765include}}\end{PQuestion}

}
\end{frame}

\begin{frame}
\only<1>{
\begin{PQuestion}{AF407}{Welche Funktion haben die mit X gekennzeichneten Bauteile in der folgenden Schaltung?}{Sie schützen den Transistor vor thermischer Überlastung.}
{Sie schützen den Transistor vor unerwünschten Rückkopplungen und filtern Eigenschwingungen des Transistors aus. }
{Sie dienen zur optimalen Einstellung des Arbeitspunktes für den Transistor.}
{Sie transformieren die Ausgangsimpedanz der vorhergehenden Stufe auf die Eingangsimpedanz des Transistors. }
{\DARCimage{1.0\linewidth}{768include}}\end{PQuestion}

}
\only<2>{
\begin{PQuestion}{AF407}{Welche Funktion haben die mit X gekennzeichneten Bauteile in der folgenden Schaltung?}{Sie schützen den Transistor vor thermischer Überlastung.}
{Sie schützen den Transistor vor unerwünschten Rückkopplungen und filtern Eigenschwingungen des Transistors aus. }
{Sie dienen zur optimalen Einstellung des Arbeitspunktes für den Transistor.}
{\textbf{\textcolor{DARCgreen}{Sie transformieren die Ausgangsimpedanz der vorhergehenden Stufe auf die Eingangsimpedanz des Transistors. }}}
{\DARCimage{1.0\linewidth}{768include}}\end{PQuestion}

}
\end{frame}

\begin{frame}
\only<1>{
\begin{PQuestion}{AF406}{Welche Funktion haben die mit X gekennzeichneten Bauteile in der folgenden Schaltung? Sie ~...}{dienen als Sperrkreis. }
{passen die Lastimpedanz an die gewünschte Impedanz für die Transistorschaltung an.}
{dienen der Trägerunterdrückung bei SSB-Modulation. }
{dienen als Bandsperre. }
{\DARCimage{1.0\linewidth}{769include}}\end{PQuestion}

}
\only<2>{
\begin{PQuestion}{AF406}{Welche Funktion haben die mit X gekennzeichneten Bauteile in der folgenden Schaltung? Sie ~...}{dienen als Sperrkreis. }
{\textbf{\textcolor{DARCgreen}{passen die Lastimpedanz an die gewünschte Impedanz für die Transistorschaltung an.}}}
{dienen der Trägerunterdrückung bei SSB-Modulation. }
{dienen als Bandsperre. }
{\DARCimage{1.0\linewidth}{769include}}\end{PQuestion}

}
\end{frame}

\begin{frame}
\only<1>{
\begin{PQuestion}{AF417}{Zu welchem Zweck dienen $T_1$ und $T_2$ in diesem HF-Leistungsverstärker?}{Zur Anpassung von \qty{50}{\ohm} an die hochohmige Eingangsimpedanz der Transistoren und die niederohmige Ausgangsimpedanz der Transistoren an \qty{50}{\ohm}.}
{Zur Anpassung von \qty{50}{\ohm} an die niederohmige Eingangsimpedanz der Transistoren und die niederohmige Ausgangsimpedanz der Transistoren an \qty{50}{\ohm}.}
{Zur Anpassung von \qty{50}{\ohm} an die niederohmige Eingangsimpedanz der Transistoren und die hochohmige Ausgangsimpedanz der Transistoren an \qty{50}{\ohm}.}
{Zur Anpassung von \qty{50}{\ohm} an die hochohmige Eingangsimpedanz der Transistoren und die hochohmige Ausgangsimpedanz der Transistoren an \qty{50}{\ohm}.}
{\DARCimage{1.0\linewidth}{786include}}\end{PQuestion}

}
\only<2>{
\begin{PQuestion}{AF417}{Zu welchem Zweck dienen $T_1$ und $T_2$ in diesem HF-Leistungsverstärker?}{Zur Anpassung von \qty{50}{\ohm} an die hochohmige Eingangsimpedanz der Transistoren und die niederohmige Ausgangsimpedanz der Transistoren an \qty{50}{\ohm}.}
{\textbf{\textcolor{DARCgreen}{Zur Anpassung von \qty{50}{\ohm} an die niederohmige Eingangsimpedanz der Transistoren und die niederohmige Ausgangsimpedanz der Transistoren an \qty{50}{\ohm}.}}}
{Zur Anpassung von \qty{50}{\ohm} an die niederohmige Eingangsimpedanz der Transistoren und die hochohmige Ausgangsimpedanz der Transistoren an \qty{50}{\ohm}.}
{Zur Anpassung von \qty{50}{\ohm} an die hochohmige Eingangsimpedanz der Transistoren und die hochohmige Ausgangsimpedanz der Transistoren an \qty{50}{\ohm}.}
{\DARCimage{1.0\linewidth}{786include}}\end{PQuestion}

}
\end{frame}

\begin{frame}
\only<1>{
\begin{QQuestion}{AF405}{Welche Funktion hat das Ausgangs-Pi-Filter eines HF-Senders?}{Es dient der besseren Oberwellenanpassung an die Antenne.}
{Es dient der Impedanztransformation und verbessert die Unterdrückung von Oberwellen.}
{Es dient der Verbesserung des Wirkungsgrads der Endstufe durch Änderung der ALC.}
{Es dient dem Schutz der Endstufe bei offener oder kurzgeschlossener Antennenbuchse.}
\end{QQuestion}

}
\only<2>{
\begin{QQuestion}{AF405}{Welche Funktion hat das Ausgangs-Pi-Filter eines HF-Senders?}{Es dient der besseren Oberwellenanpassung an die Antenne.}
{\textbf{\textcolor{DARCgreen}{Es dient der Impedanztransformation und verbessert die Unterdrückung von Oberwellen.}}}
{Es dient der Verbesserung des Wirkungsgrads der Endstufe durch Änderung der ALC.}
{Es dient dem Schutz der Endstufe bei offener oder kurzgeschlossener Antennenbuchse.}
\end{QQuestion}

}
\end{frame}

\begin{frame}
\only<1>{
\begin{QQuestion}{AF404}{Wozu dienen LC-Schaltungen unmittelbar hinter einem HF-Leistungsverstärker? Sie dienen zur...}{Unterdrückung des HF-Trägers bei SSB-Modulation. }
{optimalen Einstellung des Arbeitspunktes des HF-Leistungsverstärkers.}
{Verringerung der rücklaufenden Leistung bei Fehlanpassung der Antennenimpedanz.}
{frequenzabhängigen Transformation der Senderausgangsimpedanz auf die Antenneneingangsimpedanz und zur Unterdrückung von Oberschwingungen.}
\end{QQuestion}

}
\only<2>{
\begin{QQuestion}{AF404}{Wozu dienen LC-Schaltungen unmittelbar hinter einem HF-Leistungsverstärker? Sie dienen zur...}{Unterdrückung des HF-Trägers bei SSB-Modulation. }
{optimalen Einstellung des Arbeitspunktes des HF-Leistungsverstärkers.}
{Verringerung der rücklaufenden Leistung bei Fehlanpassung der Antennenimpedanz.}
{\textbf{\textcolor{DARCgreen}{frequenzabhängigen Transformation der Senderausgangsimpedanz auf die Antenneneingangsimpedanz und zur Unterdrückung von Oberschwingungen.}}}
\end{QQuestion}

}
\end{frame}

\begin{frame}
\only<1>{
\begin{QQuestion}{AF401}{Wie ist der Wirkungsgrad eines HF-Verstärkers definiert?}{Als Erhöhung der Ausgangsleistung  bezogen auf die Eingangsleistung.}
{Als Verhältnis der Stärke der erwünschten Aussendung zur Stärke der unerwünschten Aussendungen.}
{Als Verhältnis der HF-Leistung zu der Verlustleistung der Endstufenröhre bzw. des Endstufentransistors.}
{Als Verhältnis der HF-Ausgangsleistung zu der zugeführten Gleichstromleistung.}
\end{QQuestion}

}
\only<2>{
\begin{QQuestion}{AF401}{Wie ist der Wirkungsgrad eines HF-Verstärkers definiert?}{Als Erhöhung der Ausgangsleistung  bezogen auf die Eingangsleistung.}
{Als Verhältnis der Stärke der erwünschten Aussendung zur Stärke der unerwünschten Aussendungen.}
{Als Verhältnis der HF-Leistung zu der Verlustleistung der Endstufenröhre bzw. des Endstufentransistors.}
{\textbf{\textcolor{DARCgreen}{Als Verhältnis der HF-Ausgangsleistung zu der zugeführten Gleichstromleistung.}}}
\end{QQuestion}

}
\end{frame}

\begin{frame}
\only<1>{
\begin{PQuestion}{AF420}{Die Arbeitspunkteinstellung der LDMOS-Kurzwellen-PA erfolgt mit $R_3$. Wie verändert sich der Drainstrom, wenn $R_3$ in Richtung 3 verstellt wird?}{Der Drainstrom in beiden Transistoren verringert sich.}
{Der Drainstrom in beiden Transistoren erhöht sich.}
{Der Drainstrom steigt in $K_1$ und sinkt in $K_2$.}
{Der Drainstrom sinkt in $K_1$ und steigt in $K_2$.}
{\DARCimage{1.0\linewidth}{786include}}\end{PQuestion}

}
\only<2>{
\begin{PQuestion}{AF420}{Die Arbeitspunkteinstellung der LDMOS-Kurzwellen-PA erfolgt mit $R_3$. Wie verändert sich der Drainstrom, wenn $R_3$ in Richtung 3 verstellt wird?}{\textbf{\textcolor{DARCgreen}{Der Drainstrom in beiden Transistoren verringert sich.}}}
{Der Drainstrom in beiden Transistoren erhöht sich.}
{Der Drainstrom steigt in $K_1$ und sinkt in $K_2$.}
{Der Drainstrom sinkt in $K_1$ und steigt in $K_2$.}
{\DARCimage{1.0\linewidth}{786include}}\end{PQuestion}

}
\end{frame}

\begin{frame}
\only<1>{
\begin{PQuestion}{AF423}{Der Ruhestrom in der dargestellten VHF-LDMOS-PA soll erhöht werden. Welche Einstellungen sind vorzunehmen?}{$R_1$ in Richtung $U_\text{BIAS}$ und $R_2$ in Richtung GND verstellen.}
{$R_1$ und $R_2$ in Richtung GND verstellen.}
{$R_1$ und $R_2$ in Richtung $U_\text{BIAS}$ verstellen.}
{$R_1$ in Richtung GND und $R_2$ in Richtung $U_\text{BIAS}$ verstellen.}
{\DARCimage{1.0\linewidth}{783include}}\end{PQuestion}

}
\only<2>{
\begin{PQuestion}{AF423}{Der Ruhestrom in der dargestellten VHF-LDMOS-PA soll erhöht werden. Welche Einstellungen sind vorzunehmen?}{$R_1$ in Richtung $U_\text{BIAS}$ und $R_2$ in Richtung GND verstellen.}
{$R_1$ und $R_2$ in Richtung GND verstellen.}
{\textbf{\textcolor{DARCgreen}{$R_1$ und $R_2$ in Richtung $U_\text{BIAS}$ verstellen.}}}
{$R_1$ in Richtung GND und $R_2$ in Richtung $U_\text{BIAS}$ verstellen.}
{\DARCimage{1.0\linewidth}{783include}}\end{PQuestion}

}
\end{frame}

\begin{frame}
\only<1>{
\begin{PQuestion}{AF424}{Wie verändern sich die Drainströme in den beiden Endstufen-Transistoren, wenn der Schleifer von $R_4$ in Richtung $U_\text{BIAS}$ verstellt wird?}{Drainstrom in Transistor 1 sinkt und Drainstrom in Transistor 2 bleibt konstant.}
{Drainstrom in Transistor 1 steigt und Drainstrom in Transistor 2 steigt.}
{Drainstrom in Transistor 1 sinkt und Drainstrom in Transistor 2 sinkt.}
{Drainstrom in Transistor 1 steigt und Drainstrom in Transistor 2 bleibt konstant.}
{\DARCimage{1.0\linewidth}{784include}}\end{PQuestion}

}
\only<2>{
\begin{PQuestion}{AF424}{Wie verändern sich die Drainströme in den beiden Endstufen-Transistoren, wenn der Schleifer von $R_4$ in Richtung $U_\text{BIAS}$ verstellt wird?}{Drainstrom in Transistor 1 sinkt und Drainstrom in Transistor 2 bleibt konstant.}
{Drainstrom in Transistor 1 steigt und Drainstrom in Transistor 2 steigt.}
{Drainstrom in Transistor 1 sinkt und Drainstrom in Transistor 2 sinkt.}
{\textbf{\textcolor{DARCgreen}{Drainstrom in Transistor 1 steigt und Drainstrom in Transistor 2 bleibt konstant.}}}
{\DARCimage{1.0\linewidth}{784include}}\end{PQuestion}

}
\end{frame}

\begin{frame}
\only<1>{
\begin{PQuestion}{AF421}{Wie groß ist die Gate-Source-Spannung, wenn sich der Schleifer von $R_3$ am Anschlag~1 befindet?}{\qty{2,77}{\volt}}
{\qty{3,5}{\volt}}
{\qty{3,7}{\volt}}
{\qty{0,45}{\volt}}
{\DARCimage{1.0\linewidth}{786include}}\end{PQuestion}

}
\only<2>{
\begin{PQuestion}{AF421}{Wie groß ist die Gate-Source-Spannung, wenn sich der Schleifer von $R_3$ am Anschlag~1 befindet?}{\qty{2,77}{\volt}}
{\textbf{\textcolor{DARCgreen}{\qty{3,5}{\volt}}}}
{\qty{3,7}{\volt}}
{\qty{0,45}{\volt}}
{\DARCimage{1.0\linewidth}{786include}}\end{PQuestion}

}
\end{frame}

\begin{frame}
\frametitle{Lösungsweg}
\begin{columns}
    \begin{column}{0.48\textwidth}
    \begin{itemize}
  \item gegeben: $U_Z = 6,2V$
  \item gegeben: $R_2 = 270Ω$
  \item gegeben: $R_3 = 220Ω$
  \end{itemize}

    \end{column}
   \begin{column}{0.48\textwidth}
       \begin{itemize}
  \item gegeben: $R_4 = 6,8kΩ$
  \item gegeben: $R_6 = 150Ω$
  \item gesucht: $U_{GS}$
  \end{itemize}

   \end{column}
\end{columns}
    \pause
    $R_E = \frac{(R_3+R_6) \cdot R_4}{(R_3 + R_6) + R_4} = \frac{220Ω + 150Ω) \cdot 6,8kΩ}{220Ω + 150Ω + 6,8kΩ} = \frac{2,516MΩ^2}{7170Ω} = 351Ω$

$\frac{U_Z}{U_{GS}} = \frac{R_2 + R_E}{R_E} \Rightarrow \frac{6,2V}{U_{GS}} = \frac{270Ω+351Ω}{351Ω} = 1,77 \Rightarrow U_{GS} = \frac{6,2V}{1,77} = 3,50V$



\end{frame}

\begin{frame}
\only<1>{
\begin{PQuestion}{AF411}{Welchem Zweck dient X in der folgenden Schaltung?}{Zur Abstimmung}
{Zur HF-Entkopplung}
{Zur Wechselstromkopplung}
{Zur Kopplung mit der nächstfolgenden Stufe}
{\DARCimage{1.0\linewidth}{781include}}\end{PQuestion}

}
\only<2>{
\begin{PQuestion}{AF411}{Welchem Zweck dient X in der folgenden Schaltung?}{Zur Abstimmung}
{\textbf{\textcolor{DARCgreen}{Zur HF-Entkopplung}}}
{Zur Wechselstromkopplung}
{Zur Kopplung mit der nächstfolgenden Stufe}
{\DARCimage{1.0\linewidth}{781include}}\end{PQuestion}

}
\end{frame}

\begin{frame}
\only<1>{
\begin{PQuestion}{AF419}{Zu welchem Zweck dient die Schaltung der Spule, $C_2$ und $C_3$?}{Sie reduziert Brummspannungsanteile auf dem Sendesignal.}
{Sie reduziert HF-Anteile auf der Betriebsspannungsleitung.}
{Sie reduziert Oberschwingungen auf dem Sendesignal.}
{Sie wirkt als Pi-Filter für das Sendesignal.}
{\DARCimage{1.0\linewidth}{786include}}\end{PQuestion}

}
\only<2>{
\begin{PQuestion}{AF419}{Zu welchem Zweck dient die Schaltung der Spule, $C_2$ und $C_3$?}{Sie reduziert Brummspannungsanteile auf dem Sendesignal.}
{\textbf{\textcolor{DARCgreen}{Sie reduziert HF-Anteile auf der Betriebsspannungsleitung.}}}
{Sie reduziert Oberschwingungen auf dem Sendesignal.}
{Sie wirkt als Pi-Filter für das Sendesignal.}
{\DARCimage{1.0\linewidth}{786include}}\end{PQuestion}

}
\end{frame}

\begin{frame}
\only<1>{
\begin{PQuestion}{AF418}{Welche Funktion trifft für die Spule, $C_2$ und $C_3$ in der Schaltung zu?}{Tiefpass}
{Hochpass}
{Bandpass}
{Bandsperre}
{\DARCimage{1.0\linewidth}{786include}}\end{PQuestion}

}
\only<2>{
\begin{PQuestion}{AF418}{Welche Funktion trifft für die Spule, $C_2$ und $C_3$ in der Schaltung zu?}{\textbf{\textcolor{DARCgreen}{Tiefpass}}}
{Hochpass}
{Bandpass}
{Bandsperre}
{\DARCimage{1.0\linewidth}{786include}}\end{PQuestion}

}
\end{frame}

\begin{frame}
\only<1>{
\begin{PQuestion}{AF422}{Wozu dienen die mit X gekennzeichneten Spulen in der Schaltung?}{Sie verhindern die Entstehung von Oberschwingungen.}
{Sie verhindern ein Abfließen der Hochfrequenz in die Spannungsversorgung.}
{Sie dienen als Arbeitswiderstand für die Transistoren.}
{Sie transformieren die Ausgangsimpedanz der Transistoren auf \qty{50}{\ohm}.}
{\DARCimage{1.0\linewidth}{782include}}\end{PQuestion}

}
\only<2>{
\begin{PQuestion}{AF422}{Wozu dienen die mit X gekennzeichneten Spulen in der Schaltung?}{Sie verhindern die Entstehung von Oberschwingungen.}
{\textbf{\textcolor{DARCgreen}{Sie verhindern ein Abfließen der Hochfrequenz in die Spannungsversorgung.}}}
{Sie dienen als Arbeitswiderstand für die Transistoren.}
{Sie transformieren die Ausgangsimpedanz der Transistoren auf \qty{50}{\ohm}.}
{\DARCimage{1.0\linewidth}{782include}}\end{PQuestion}

}
\end{frame}

\begin{frame}
\only<1>{
\begin{PQuestion}{AF415}{Weshalb wurden jeweils $C_1$ und $C_2$, $C_3$ und $C_4$ sowie $C_5$ und $C_6$ parallel geschaltet?}{Der Kondensator geringer Kapazität dient jeweils zum Abblocken hoher Frequenzen, der Kondensator hoher Kapazität zum Abblocken niedriger Frequenzen.}
{Die Kapazität nur eines Kondensators reicht bei hohen Frequenzen nicht aus.}
{Der Kondensator mit der geringen Kapazität dient zur Siebung der niedrigen und der Kondensator mit der hohen Kapazität zur Siebung der hohen Frequenzen.}
{Zu einem Elektrolytkondensator muss immer ein keramischer Kondensator parallel geschaltet werden, weil er sonst bei hohen Frequenzen zerstört werden würde.}
{\DARCimage{1.0\linewidth}{766include}}\end{PQuestion}

}
\only<2>{
\begin{PQuestion}{AF415}{Weshalb wurden jeweils $C_1$ und $C_2$, $C_3$ und $C_4$ sowie $C_5$ und $C_6$ parallel geschaltet?}{\textbf{\textcolor{DARCgreen}{Der Kondensator geringer Kapazität dient jeweils zum Abblocken hoher Frequenzen, der Kondensator hoher Kapazität zum Abblocken niedriger Frequenzen.}}}
{Die Kapazität nur eines Kondensators reicht bei hohen Frequenzen nicht aus.}
{Der Kondensator mit der geringen Kapazität dient zur Siebung der niedrigen und der Kondensator mit der hohen Kapazität zur Siebung der hohen Frequenzen.}
{Zu einem Elektrolytkondensator muss immer ein keramischer Kondensator parallel geschaltet werden, weil er sonst bei hohen Frequenzen zerstört werden würde.}
{\DARCimage{1.0\linewidth}{766include}}\end{PQuestion}

}
\end{frame}

\begin{frame}
\only<1>{
\begin{PQuestion}{AF428}{Wie groß ist die Gesamtverstärkung des gesamten Sendezweigs ohne Berücksichtigung möglicher Kabelverluste?}{\qty{38}{\decibel}}
{\qty{48}{\decibel}}
{\qty{43}{\decibel}}
{\qty{59}{\decibel}}
{\DARCimage{1.0\linewidth}{470include}}\end{PQuestion}

}
\only<2>{
\begin{PQuestion}{AF428}{Wie groß ist die Gesamtverstärkung des gesamten Sendezweigs ohne Berücksichtigung möglicher Kabelverluste?}{\qty{38}{\decibel}}
{\textbf{\textcolor{DARCgreen}{\qty{48}{\decibel}}}}
{\qty{43}{\decibel}}
{\qty{59}{\decibel}}
{\DARCimage{1.0\linewidth}{470include}}\end{PQuestion}

}
\end{frame}

\begin{frame}
\frametitle{Lösungsweg}
\begin{itemize}
  \item gegeben: $P_1 = 0,3mW$ oder $-5dBm$
  \item gegeben: $P_2 = 20W$ oder $43dBm$
  \item gesucht: $g$
  \end{itemize}
    \pause
    $g = P_2 -- P_1 = 43dBm -- (-5dBm) = 43dBm + 5dBm = 48dB$
    \pause
    $g = 10 \cdot \log_{10}{(\frac{P_2}{P_1})}dB = 10 \cdot \log_{10}{(\frac{20W}{0,3mW})}dB \approx 48dB$



\end{frame}%ENDCONTENT


\section{Parasitäre Schwingungen}
\label{section:parasitaere_schwingungen}
\begin{frame}%STARTCONTENT

\only<1>{
\begin{QQuestion}{AJ212}{Parasitäre Schwingungen können Störungen hervorrufen. Man erkennt diese Schwingungen unter anderem daran, dass sie~...}{bei ganzzahligen Vielfachen der Betriebsfrequenz auftreten.}
{bei ungeradzahligen Vielfachen der Betriebsfrequenz auftreten.}
{bei geradzahligen Vielfachen der Betriebsfrequenz auftreten.}
{keinen festen Bezug zur Betriebsfrequenz haben.}
\end{QQuestion}

}
\only<2>{
\begin{QQuestion}{AJ212}{Parasitäre Schwingungen können Störungen hervorrufen. Man erkennt diese Schwingungen unter anderem daran, dass sie~...}{bei ganzzahligen Vielfachen der Betriebsfrequenz auftreten.}
{bei ungeradzahligen Vielfachen der Betriebsfrequenz auftreten.}
{bei geradzahligen Vielfachen der Betriebsfrequenz auftreten.}
{\textbf{\textcolor{DARCgreen}{keinen festen Bezug zur Betriebsfrequenz haben.}}}
\end{QQuestion}

}
\end{frame}

\begin{frame}
\only<1>{
\begin{QQuestion}{AJ213}{Die Ausgangsleistungsanzeige eines HF-Verstärkers zeigt beim Abstimmen geringfügige sprunghafte Schwankungen. Sie werden möglicherweise hervorgerufen durch~...}{vom Wind verursachte Bewegungen der Antenne.}
{Welligkeit auf der Stromversorgung.}
{Temperaturschwankungen im Netzteil.}
{parasitäre Schwingungen.}
\end{QQuestion}

}
\only<2>{
\begin{QQuestion}{AJ213}{Die Ausgangsleistungsanzeige eines HF-Verstärkers zeigt beim Abstimmen geringfügige sprunghafte Schwankungen. Sie werden möglicherweise hervorgerufen durch~...}{vom Wind verursachte Bewegungen der Antenne.}
{Welligkeit auf der Stromversorgung.}
{Temperaturschwankungen im Netzteil.}
{\textbf{\textcolor{DARCgreen}{parasitäre Schwingungen.}}}
\end{QQuestion}

}
\end{frame}

\begin{frame}
\only<1>{
\begin{QQuestion}{AJ217}{Wie kann man bei einem VHF-Sender mit kleiner Leistung die Entstehung parasitärer Schwingungen wirksam unterdrücken?}{Durch Anbringen eines Klappferritkerns an der Mikrofonzuleitung.}
{Durch Aufstecken einer Ferritperle auf die Emitterzuleitung des Endstufentransistors.}
{Durch Aufkleben einer Ferritperle auf das Gehäuse des Endstufentransistors.}
{Durch Anbringen eines Klappferritkerns an der Stromversorgungszuleitung.}
\end{QQuestion}

}
\only<2>{
\begin{QQuestion}{AJ217}{Wie kann man bei einem VHF-Sender mit kleiner Leistung die Entstehung parasitärer Schwingungen wirksam unterdrücken?}{Durch Anbringen eines Klappferritkerns an der Mikrofonzuleitung.}
{\textbf{\textcolor{DARCgreen}{Durch Aufstecken einer Ferritperle auf die Emitterzuleitung des Endstufentransistors.}}}
{Durch Aufkleben einer Ferritperle auf das Gehäuse des Endstufentransistors.}
{Durch Anbringen eines Klappferritkerns an der Stromversorgungszuleitung.}
\end{QQuestion}

}
\end{frame}

\begin{frame}
\only<1>{
\begin{PQuestion}{AF416}{Wozu dient der Widerstand $R$ parallel zur Trafowicklung $T_2$?}{Er dient zur Erhöhung des HF-Wirkungsgrades der Verstärkerstufe.}
{Er dient zur Anpassung der Primärwicklung an die folgende PA.}
{Er soll die Entstehung parasitärer Schwingungen verhindern.}
{Er dient zur Begrenzung des Kollektorstroms bei Übersteuerung.}
{\DARCimage{1.0\linewidth}{767include}}\end{PQuestion}

}
\only<2>{
\begin{PQuestion}{AF416}{Wozu dient der Widerstand $R$ parallel zur Trafowicklung $T_2$?}{Er dient zur Erhöhung des HF-Wirkungsgrades der Verstärkerstufe.}
{Er dient zur Anpassung der Primärwicklung an die folgende PA.}
{\textbf{\textcolor{DARCgreen}{Er soll die Entstehung parasitärer Schwingungen verhindern.}}}
{Er dient zur Begrenzung des Kollektorstroms bei Übersteuerung.}
{\DARCimage{1.0\linewidth}{767include}}\end{PQuestion}

}
\end{frame}%ENDCONTENT


\section{Messungen am Sender}
\label{section:sender_messungen}
\begin{frame}%STARTCONTENT

\only<1>{
\begin{PQuestion}{AI608}{Was stellt die folgende Schaltung dar? }{Absorptionsfrequenzmesser}
{Messkopf zur HF-Leistungsmessung}
{Antennenimpedanzmesser}
{HF-Dipmeter}
{\DARCimage{1.0\linewidth}{576include}}\end{PQuestion}

}
\only<2>{
\begin{PQuestion}{AI608}{Was stellt die folgende Schaltung dar? }{Absorptionsfrequenzmesser}
{\textbf{\textcolor{DARCgreen}{Messkopf zur HF-Leistungsmessung}}}
{Antennenimpedanzmesser}
{HF-Dipmeter}
{\DARCimage{1.0\linewidth}{576include}}\end{PQuestion}

}
\end{frame}

\begin{frame}
\only<1>{
\begin{PQuestion}{AI605}{Was stellt die folgende Schaltung dar? }{Absorptionsfrequenzmesser}
{HF-Tastkopf}
{Antennenimpedanzmesser}
{HF-Dipmeter}
{\DARCimage{1.0\linewidth}{770include}}\end{PQuestion}

}
\only<2>{
\begin{PQuestion}{AI605}{Was stellt die folgende Schaltung dar? }{Absorptionsfrequenzmesser}
{\textbf{\textcolor{DARCgreen}{HF-Tastkopf}}}
{Antennenimpedanzmesser}
{HF-Dipmeter}
{\DARCimage{1.0\linewidth}{770include}}\end{PQuestion}

}
\end{frame}

\begin{frame}
\only<1>{
\begin{PQuestion}{AI604}{Wozu dient diese Schaltung? Sie dient~...}{als Messkopf zum Abgleich von HF-Schaltungen.}
{als hochohmiger Messkopf für einen vektoriellen Netzwerkanalyzer.}
{zur Messung der Resonanzfrequenz mit einem Frequenzzähler.}
{als Gleichspannungstastkopf zur genauen Einstellung der Versorgungsspannung.}
{\DARCimage{1.0\linewidth}{770include}}\end{PQuestion}

}
\only<2>{
\begin{PQuestion}{AI604}{Wozu dient diese Schaltung? Sie dient~...}{\textbf{\textcolor{DARCgreen}{als Messkopf zum Abgleich von HF-Schaltungen.}}}
{als hochohmiger Messkopf für einen vektoriellen Netzwerkanalyzer.}
{zur Messung der Resonanzfrequenz mit einem Frequenzzähler.}
{als Gleichspannungstastkopf zur genauen Einstellung der Versorgungsspannung.}
{\DARCimage{1.0\linewidth}{770include}}\end{PQuestion}

}
\end{frame}

\begin{frame}
\only<1>{
\begin{PQuestion}{AI609}{Sie wollen mit der folgenden Messschaltung die Ausgangsleistung eines \qty{2}{\m}-Senders überprüfen, der voraussichtlich ca. \qty{15}{\W} HF-Leistung liefert. Was sollte für die Messung vor die dargestellte Messschaltung geschaltet werden?}{Dämpfungsglied \qty{20}{\decibel}, \qty{20}{\W}}
{\qty{25}{\m} langes Koaxialkabel vom Typ RG213 (MIL)}
{Stehwellenmessgerät}
{Adapter BNC-Buchse auf N-Stecker}
{\DARCimage{1.0\linewidth}{576include}}\end{PQuestion}

}
\only<2>{
\begin{PQuestion}{AI609}{Sie wollen mit der folgenden Messschaltung die Ausgangsleistung eines \qty{2}{\m}-Senders überprüfen, der voraussichtlich ca. \qty{15}{\W} HF-Leistung liefert. Was sollte für die Messung vor die dargestellte Messschaltung geschaltet werden?}{\textbf{\textcolor{DARCgreen}{Dämpfungsglied \qty{20}{\decibel}, \qty{20}{\W}}}}
{\qty{25}{\m} langes Koaxialkabel vom Typ RG213 (MIL)}
{Stehwellenmessgerät}
{Adapter BNC-Buchse auf N-Stecker}
{\DARCimage{1.0\linewidth}{576include}}\end{PQuestion}

}
\end{frame}

\begin{frame}
\only<1>{
\begin{PQuestion}{AI612}{Was muss für die genaue Messung der HF-Ausgangsleistung eines Senders mit einer solchen Schaltung berücksichtigt werden?}{Bei den Umrechnungen darf nur mit dem Effektivwert gerechnet werden.}
{$R_1$ muss genau \qty{50}{\ohm} betragen.}
{Korrekturwerte für die Schaltung, die aus einer Kalibrierung stammen.}
{Die Schaltung muss vor jeder Messung mit einem Spektrumanalysator überprüft werden.}
{\DARCimage{1.0\linewidth}{577include}}\end{PQuestion}

}
\only<2>{
\begin{PQuestion}{AI612}{Was muss für die genaue Messung der HF-Ausgangsleistung eines Senders mit einer solchen Schaltung berücksichtigt werden?}{Bei den Umrechnungen darf nur mit dem Effektivwert gerechnet werden.}
{$R_1$ muss genau \qty{50}{\ohm} betragen.}
{\textbf{\textcolor{DARCgreen}{Korrekturwerte für die Schaltung, die aus einer Kalibrierung stammen.}}}
{Die Schaltung muss vor jeder Messung mit einem Spektrumanalysator überprüft werden.}
{\DARCimage{1.0\linewidth}{577include}}\end{PQuestion}

}
\end{frame}

\begin{frame}
\only<1>{
\begin{PQuestion}{AI610}{Dem Eingang der folgenden Messschaltung wird eine HF-Leistung von \qty{1}{\W} zugeführt. D ist eine Schottkydiode mit $U_F$ = \qty{0,23}{\V}. Welche Spannung $U_A$ ist am Ausgang A zu erwarten, wenn die Messung mit einem hochohmigen Spannungsmessgerät erfolgt?}{\qty{9,8}{\V}}
{\qty{3,3}{\V}}
{\qty{7,1}{\V}}
{\qty{4,8}{\V}}
{\DARCimage{1.0\linewidth}{576include}}\end{PQuestion}

}
\only<2>{
\begin{PQuestion}{AI610}{Dem Eingang der folgenden Messschaltung wird eine HF-Leistung von \qty{1}{\W} zugeführt. D ist eine Schottkydiode mit $U_F$ = \qty{0,23}{\V}. Welche Spannung $U_A$ ist am Ausgang A zu erwarten, wenn die Messung mit einem hochohmigen Spannungsmessgerät erfolgt?}{\qty{9,8}{\V}}
{\qty{3,3}{\V}}
{\qty{7,1}{\V}}
{\textbf{\textcolor{DARCgreen}{\qty{4,8}{\V}}}}
{\DARCimage{1.0\linewidth}{576include}}\end{PQuestion}

}
\end{frame}

\begin{frame}
\frametitle{Lösungsweg}
\begin{itemize}
  \item gegeben: $P_E = 1W$
  \item gegeben: $U_F = 0,23V$
  \item gegeben: $R_V = 110Ω$, $R_T = 330Ω$
  \item gesucht: $U_A$
  \end{itemize}
    \pause
    $R = (\frac{1}{R_T + R_T} + \frac{1}{R_V} + \frac{1}{R_V})^{-1} = (\frac{1}{330Ω + 330Ω} + \frac{1}{110Ω} + \frac{1}{110Ω})^{-1} = 50,77Ω$
    \pause
    $P_E = \frac{U_{E,eff}^2}{R} \Rightarrow U_{E,eff} = \sqrt{P_E \cdot R} = \sqrt{1W \cdot 50,77Ω} = 7,125V$

$U_S = U_{E,eff} \cdot \sqrt{2} = 7,071V \cdot 1,414 = 10,07V$
    \pause
    $U_A = \frac{U_S}{2} -- U_F = \frac{10,07V}{2} -- 0,23V = 5,035V -- 0,23V = 4,805V \approx 4,8V$



\end{frame}

\begin{frame}
\only<1>{
\begin{PQuestion}{AI611}{Bei der folgenden Schaltung besteht $R_1$ aus einer Zusammenschaltung von Widerständen, die einen Gesamtwiderstand von \qty{54,1}{\ohm} hat und etwa \qty{200}{\W} aufnehmen kann. Die Diode ist eine Siliziumdiode mit $U_{\symup{F}}$ = \qty{0,7}{\V}. Am Ausgang wird mit einem digitalen Spannungsmessgerät eine Gleichspannung von \qty{14,9}{\V} gemessen. Wie groß ist etwa die HF-Leistung am Eingang der Schaltung?}{\qty{9,7}{\W}}
{\qty{37,8}{\W}}
{\qty{4,9}{\W}}
{\qty{19,4}{\W}}
{\DARCimage{1.0\linewidth}{577include}}\end{PQuestion}

}
\only<2>{
\begin{PQuestion}{AI611}{Bei der folgenden Schaltung besteht $R_1$ aus einer Zusammenschaltung von Widerständen, die einen Gesamtwiderstand von \qty{54,1}{\ohm} hat und etwa \qty{200}{\W} aufnehmen kann. Die Diode ist eine Siliziumdiode mit $U_{\symup{F}}$ = \qty{0,7}{\V}. Am Ausgang wird mit einem digitalen Spannungsmessgerät eine Gleichspannung von \qty{14,9}{\V} gemessen. Wie groß ist etwa die HF-Leistung am Eingang der Schaltung?}{\textbf{\textcolor{DARCgreen}{\qty{9,7}{\W}}}}
{\qty{37,8}{\W}}
{\qty{4,9}{\W}}
{\qty{19,4}{\W}}
{\DARCimage{1.0\linewidth}{577include}}\end{PQuestion}

}
\end{frame}

\begin{frame}
\frametitle{Lösungsweg}
\begin{itemize}
  \item gegeben: $U_A = 14,9V DC$
  \item gegeben: $U_F = 0,7V$
  \item gegeben: $R_1 = 54,1Ω$, $R_T = 330Ω$
  \item gesucht: $P_E$
  \end{itemize}
    \pause
    $R = (\frac{1}{R_T + R_T} + \frac{1}{R_1})^{-1} = (\frac{1}{330Ω + 330Ω} + \frac{1}{54,1Ω})^{-1} = 50Ω$
    \pause
    $U_S = (U_A + U_F) \cdot 2 = (14,9V + 0,7V) \cdot 2 = 31,2V$

$U_{E,eff} = \frac{U_S}{\sqrt{2}} = \frac{31,2V}{1,414} = 22,06V$
    \pause
    $P_E = \frac{U_{E,eff}^2}{R} = \frac{(22,06V)^2}{50Ω} \approx 9,7W$



\end{frame}

\begin{frame}
\only<1>{
\begin{PQuestion}{AI607}{Mit der folgenden Schaltung soll die Ausgangsleistung eines \qty{2}{\m}-FM-Handfunkgerätes gemessen werden. Die Dioden sind Schottkydioden mit $U_{\symup{F}}$~=~\qty{0,23}{\V}. Am Ausgang wird mit einem digitalen Spannungsmessgerät eine Gleichspannung von \qty{15,3}{\V} gemessen. Wie groß ist etwa die HF-Leistung am Eingang der Schaltung?}{Zirka \qty{1,2}{\W}}
{Zirka \qty{4,7}{\W}}
{Zirka \qty{600}{\mW}}
{Zirka \qty{2,4}{\W}}
{\DARCimage{1.0\linewidth}{771include}}\end{PQuestion}

}
\only<2>{
\begin{PQuestion}{AI607}{Mit der folgenden Schaltung soll die Ausgangsleistung eines \qty{2}{\m}-FM-Handfunkgerätes gemessen werden. Die Dioden sind Schottkydioden mit $U_{\symup{F}}$~=~\qty{0,23}{\V}. Am Ausgang wird mit einem digitalen Spannungsmessgerät eine Gleichspannung von \qty{15,3}{\V} gemessen. Wie groß ist etwa die HF-Leistung am Eingang der Schaltung?}{Zirka \qty{1,2}{\W}}
{Zirka \qty{4,7}{\W}}
{\textbf{\textcolor{DARCgreen}{Zirka \qty{600}{\mW}}}}
{Zirka \qty{2,4}{\W}}
{\DARCimage{1.0\linewidth}{771include}}\end{PQuestion}

}
\end{frame}

\begin{frame}
\frametitle{Lösungsweg}
\begin{itemize}
  \item gegeben: $U_A = 15,3V DC$
  \item gegeben: $U_F = 0,23V$
  \item gegeben: $R_{V1} = 56Ω$, $R_{V2} = 470Ω$
  \item gesucht: $P_E$
  \end{itemize}
    \pause
    $R = (\frac{1}{R_{V1}} + \frac{1}{R_{V2}})^{-1} = (\frac{1}{R_{56Ω}} + \frac{1}{R_{470Ω}})^{-1} = 50,04Ω$
    \pause
    $U_S = \frac{U_A}{2} + U_F = \frac{15,3V}{2} + 0,23V = 7,88V$

$U_{E,eff} = \frac{U_S}{\sqrt{2}} = \frac{7,88V}{1,414} = 5,57V$
    \pause
    $P_E = \frac{U_{E,eff}^2}{R} = \frac{{5,57V}^2}{50,04Ω} \approx 600mW$



\end{frame}

\begin{frame}
\only<1>{
\begin{PQuestion}{AI606}{Die Leistung eines \qty{2}{\metre}-Senders soll mit einer künstlichen \qty{50}{\ohm}-Antenne bestimmt werden, die über eine Anzapfung bei \qty{5}{\ohm} vom erdnahen Ende verfügt. Zur Messung an diesem Punkt wird die folgende Schaltung eingesetzt. Die Dioden sind Schottkydioden mit $U_{\symup{F}}~=$~\qty{0,23}{\V}. Am Ausgang der Schaltung wird dabei mit einem digitalen Spannungsmessgerät eine Gleichspannung von \qty{15,3}{\V} gemessen. Wie groß ist etwa die HF-Leistung des Senders?}{Zirka \qty{60}{\W}}
{Zirka \qty{480}{\W}}
{Zirka \qty{340}{\W}}
{Zirka \qty{240}{\W}}
{\DARCimage{1.0\linewidth}{770include}}\end{PQuestion}

}
\only<2>{
\begin{PQuestion}{AI606}{Die Leistung eines \qty{2}{\metre}-Senders soll mit einer künstlichen \qty{50}{\ohm}-Antenne bestimmt werden, die über eine Anzapfung bei \qty{5}{\ohm} vom erdnahen Ende verfügt. Zur Messung an diesem Punkt wird die folgende Schaltung eingesetzt. Die Dioden sind Schottkydioden mit $U_{\symup{F}}~=$~\qty{0,23}{\V}. Am Ausgang der Schaltung wird dabei mit einem digitalen Spannungsmessgerät eine Gleichspannung von \qty{15,3}{\V} gemessen. Wie groß ist etwa die HF-Leistung des Senders?}{\textbf{\textcolor{DARCgreen}{Zirka \qty{60}{\W}}}}
{Zirka \qty{480}{\W}}
{Zirka \qty{340}{\W}}
{Zirka \qty{240}{\W}}
{\DARCimage{1.0\linewidth}{770include}}\end{PQuestion}

}
\end{frame}

\begin{frame}
\frametitle{Lösungsweg}
\begin{itemize}
  \item gegeben: $U_A = 15,3V DC$
  \item gegeben: $U_F = 0,23V$
  \item gegeben: $R = 50Ω$ aus dem Messsystem
  \item gegeben: $R_A = 5Ω$ (10:1 Spannungsteiler)
  \item gesucht: $P_E$
  \end{itemize}
    \pause
    $U_S = \frac{U_A}{2} + U_F = \frac{15,3V}{2} + 0,23V = 7,88V$

$U_{E,eff} = \frac{U_S}{\sqrt{2}} = \frac{7,88V}{1,414} = 5,57V$
    \pause
    $P_E = \frac{(U_{E,eff} \cdot 10)^2}{R} = \frac{(5,57V \cdot 10)^2}{50Ω} \approx 60W$



\end{frame}

\begin{frame}
\only<1>{
\begin{PQuestion}{AI613}{Was stellt die folgende Schaltung dar? }{Antennenimpedanzmesser}
{Einfacher Peilsender}
{Feldstärkeanzeiger}
{Resonanzmessgerät}
{\DARCimage{1.0\linewidth}{496include}}\end{PQuestion}

}
\only<2>{
\begin{PQuestion}{AI613}{Was stellt die folgende Schaltung dar? }{Antennenimpedanzmesser}
{Einfacher Peilsender}
{\textbf{\textcolor{DARCgreen}{Feldstärkeanzeiger}}}
{Resonanzmessgerät}
{\DARCimage{1.0\linewidth}{496include}}\end{PQuestion}

}
\end{frame}%ENDCONTENT


\section{Dummy-Load II}
\label{section:dummy_load_2}
\begin{frame}%STARTCONTENT

\only<1>{
\begin{PQuestion}{AI601}{Die Darstellung zeigt eine aus \qty{150}{\ohm} / \qty{1}{\W}-Widerständen aufgebaute künstliche Antenne (Dummy Load). Mit dieser Kombination aus Reihen- und Parallelschaltung werden ca. \qty{50}{\ohm} erreicht. Wie viele Widerstände werden für diesen Aufbau benötigt und welche Dauerleistung verträgt diese künstliche Antenne?}{16~Widerstände, \qty{16}{\W}}
{48~Widerstände, \qty{12}{\W}}
{12~Widerstände, \qty{48}{\W}}
{48~Widerstände, \qty{48}{\W}}
{\DARCimage{1.0\linewidth}{47include}}\end{PQuestion}

}
\only<2>{
\begin{PQuestion}{AI601}{Die Darstellung zeigt eine aus \qty{150}{\ohm} / \qty{1}{\W}-Widerständen aufgebaute künstliche Antenne (Dummy Load). Mit dieser Kombination aus Reihen- und Parallelschaltung werden ca. \qty{50}{\ohm} erreicht. Wie viele Widerstände werden für diesen Aufbau benötigt und welche Dauerleistung verträgt diese künstliche Antenne?}{16~Widerstände, \qty{16}{\W}}
{48~Widerstände, \qty{12}{\W}}
{12~Widerstände, \qty{48}{\W}}
{\textbf{\textcolor{DARCgreen}{48~Widerstände, \qty{48}{\W}}}}
{\DARCimage{1.0\linewidth}{47include}}\end{PQuestion}

}
\end{frame}

\begin{frame}
\frametitle{Lösungsweg}
\begin{columns}
    \begin{column}{0.48\textwidth}
    \begin{itemize}
  \item gegeben: $R = 150Ω$
  \item gegeben: $R_S = 4\cdot 150Ω = 600Ω$
  \end{itemize}

    \end{column}
   \begin{column}{0.48\textwidth}
       \begin{itemize}
  \item gegeben: $R_{ges} = 50Ω$
  \item gegeben: $P_R = 1W$
  \item gesucht: $n$ Widerstände, $P$
  \end{itemize}

   \end{column}
\end{columns}
    \pause
    Reihen mit je 4 Widerständen:

$\frac{1}{R_{ges}} = n_S \cdot \frac{1}{R_S} \Rightarrow n_S = \frac{R_S}{R_{ges}} = \frac{600Ω}{50Ω} = 12$

$n = 4 \cdot n_S = 4 \cdot 12 = 48$
    \pause
    $P = n \cdot P_R = 48 \cdot 1W = 48W$



\end{frame}

\begin{frame}
\only<1>{
\begin{QQuestion}{AI602}{Eine künstliche Antenne (Dummy Load) verfügt über einen Messausgang, der intern an einen Spitzenwertgleichrichter angeschlossen ist. Wozu dient dieser Messausgang? Er dient~...}{zur indirekten Messung der Hochfrequenzleistung.}
{als Anschluss für einen Antennenvorverstärker.}
{als Abgriff einer ALC-Regelspannung für die Sendeendstufe.}
{zum Nachjustieren der Widerstände in der künstlichen Antenne.}
\end{QQuestion}

}
\only<2>{
\begin{QQuestion}{AI602}{Eine künstliche Antenne (Dummy Load) verfügt über einen Messausgang, der intern an einen Spitzenwertgleichrichter angeschlossen ist. Wozu dient dieser Messausgang? Er dient~...}{\textbf{\textcolor{DARCgreen}{zur indirekten Messung der Hochfrequenzleistung.}}}
{als Anschluss für einen Antennenvorverstärker.}
{als Abgriff einer ALC-Regelspannung für die Sendeendstufe.}
{zum Nachjustieren der Widerstände in der künstlichen Antenne.}
\end{QQuestion}

}
\end{frame}

\begin{frame}
\only<1>{
\begin{QQuestion}{AI603}{Eine künstliche Antenne (Dummy Load) von \qty{50}{\ohm} verfügt über eine Anzapfung bei \qty{5}{\ohm} vom erdnahen Ende. Was könnte zur ungefähren Ermittlung der Senderausgangsleistung über diesen Messpunkt eingesetzt werden?}{Stehwellenmessgerät mit Abschlusswiderstand.}
{Digitalmultimeter mit HF-Tastkopf.}
{Stehwellenmessgerät ohne Abschlusswiderstand.}
{Künstliche \qty{50}{\ohm}-Antenne mit zusätzlichem HF-Dämpfungsglied.}
\end{QQuestion}

}
\only<2>{
\begin{QQuestion}{AI603}{Eine künstliche Antenne (Dummy Load) von \qty{50}{\ohm} verfügt über eine Anzapfung bei \qty{5}{\ohm} vom erdnahen Ende. Was könnte zur ungefähren Ermittlung der Senderausgangsleistung über diesen Messpunkt eingesetzt werden?}{Stehwellenmessgerät mit Abschlusswiderstand.}
{\textbf{\textcolor{DARCgreen}{Digitalmultimeter mit HF-Tastkopf.}}}
{Stehwellenmessgerät ohne Abschlusswiderstand.}
{Künstliche \qty{50}{\ohm}-Antenne mit zusätzlichem HF-Dämpfungsglied.}
\end{QQuestion}

}
\end{frame}%ENDCONTENT


\section{Unerwünschte Aussendungen III}
\label{section:unerwuenschte_aussendungen_3}
\begin{frame}%STARTCONTENT

\only<1>{
\begin{QQuestion}{AJ211}{Wie wird vermieden, dass unerwünschte Mischprodukte aus dem Mischer in die Senderausgangsstufe gelangen?}{Das Ausgangssignal des Mischers wird über einen Hochpass ausgekoppelt.}
{Das Ausgangssignal des Mischers wird über einen Bandpass ausgekoppelt.}
{Das Ausgangssignal des Mischers wird über ein breitbandiges Dämpfungsglied ausgekoppelt.}
{Das Ausgangssignal des Mischers wird von einer linearen Klasse-A-Treiberstufe verstärkt.}
\end{QQuestion}

}
\only<2>{
\begin{QQuestion}{AJ211}{Wie wird vermieden, dass unerwünschte Mischprodukte aus dem Mischer in die Senderausgangsstufe gelangen?}{Das Ausgangssignal des Mischers wird über einen Hochpass ausgekoppelt.}
{\textbf{\textcolor{DARCgreen}{Das Ausgangssignal des Mischers wird über einen Bandpass ausgekoppelt.}}}
{Das Ausgangssignal des Mischers wird über ein breitbandiges Dämpfungsglied ausgekoppelt.}
{Das Ausgangssignal des Mischers wird von einer linearen Klasse-A-Treiberstufe verstärkt.}
\end{QQuestion}

}
\end{frame}

\begin{frame}
\only<1>{
\begin{QQuestion}{AJ209}{Welches Filter sollte hinter einen VHF-Sender geschaltet werden, um die unerwünschte Aussendung von Subharmonischen und Harmonischen auf ein Mindestmaß zu begrenzen? }{Hochpassfilter}
{Tiefpassfilter}
{Bandpass}
{Notchfilter}
\end{QQuestion}

}
\only<2>{
\begin{QQuestion}{AJ209}{Welches Filter sollte hinter einen VHF-Sender geschaltet werden, um die unerwünschte Aussendung von Subharmonischen und Harmonischen auf ein Mindestmaß zu begrenzen? }{Hochpassfilter}
{Tiefpassfilter}
{\textbf{\textcolor{DARCgreen}{Bandpass}}}
{Notchfilter}
\end{QQuestion}

}
\end{frame}

\begin{frame}
\only<1>{
\begin{question2x2}{AJ208}{Die Oberschwingungen eines Einbandsenders sollen mit einem Ausgangsfilter ünterdrückt werden. Welcher Filterkurventyp wird benötigt?}{\DARCimage{1.0\linewidth}{255include}}
{\DARCimage{1.0\linewidth}{243include}}
{\DARCimage{1.0\linewidth}{256include}}
{\DARCimage{1.0\linewidth}{257include}}
\end{question2x2}

}
\only<2>{
\begin{question2x2}{AJ208}{Die Oberschwingungen eines Einbandsenders sollen mit einem Ausgangsfilter ünterdrückt werden. Welcher Filterkurventyp wird benötigt?}{\DARCimage{1.0\linewidth}{255include}}
{\textbf{\textcolor{DARCgreen}{\DARCimage{1.0\linewidth}{243include}}}}
{\DARCimage{1.0\linewidth}{256include}}
{\DARCimage{1.0\linewidth}{257include}}
\end{question2x2}

}
\end{frame}

\begin{frame}
\only<1>{
\begin{QQuestion}{AJ204}{Die dritte Harmonische einer \qty{29,5}{\MHz}-Aussendung fällt in~...}{den FM-Rundfunkbereich.}
{den D-Netz-Mobilfunkbereich.}
{den UKW-Betriebsfunk-Bereich.}
{den \qty{2}{\m}-Amateurfunkbereich.}
\end{QQuestion}

}
\only<2>{
\begin{QQuestion}{AJ204}{Die dritte Harmonische einer \qty{29,5}{\MHz}-Aussendung fällt in~...}{\textbf{\textcolor{DARCgreen}{den FM-Rundfunkbereich.}}}
{den D-Netz-Mobilfunkbereich.}
{den UKW-Betriebsfunk-Bereich.}
{den \qty{2}{\m}-Amateurfunkbereich.}
\end{QQuestion}

}
\end{frame}

\begin{frame}
\frametitle{Lösungsweg}
\begin{itemize}
  \item gegeben: $f = 29,5MHz$
  \item gegeben: $n = 3$
  \item gegeben: Radiobereich: 88,5MHz -- 108,0MHz
  \end{itemize}
    \pause
    $f \cdot n = 29,5MHz \cdot 3 = 88,5MHz$



\end{frame}

\begin{frame}
\only<1>{
\begin{QQuestion}{AJ203}{Auf welche Frequenz müsste ein Empfänger eingestellt werden, um die dritte Oberwelle einer \qty{7,20}{\MHz}-Aussendung erkennen zu können?}{\qty{21,60}{\MHz}}
{\qty{28,80}{\MHz}}
{\qty{36,00}{\MHz}}
{\qty{14,40}{\MHz}}
\end{QQuestion}

}
\only<2>{
\begin{QQuestion}{AJ203}{Auf welche Frequenz müsste ein Empfänger eingestellt werden, um die dritte Oberwelle einer \qty{7,20}{\MHz}-Aussendung erkennen zu können?}{\qty{21,60}{\MHz}}
{\textbf{\textcolor{DARCgreen}{\qty{28,80}{\MHz}}}}
{\qty{36,00}{\MHz}}
{\qty{14,40}{\MHz}}
\end{QQuestion}

}
\end{frame}

\begin{frame}
\frametitle{Lösungsweg}
\begin{itemize}
  \item gegeben: $f = 7,20MHz$
  \item gegeben: $n = 4$
  \item gesucht: 3. Oberwelle
  \end{itemize}
    \pause
    $f \cdot n = 7,20MHz \cdot 4 = 28,80MHz$



\end{frame}

\begin{frame}
\only<1>{
\begin{PQuestion}{AJ207}{Worauf deutet die folgende Wellenform der Ausgangsspannung eines Leistungsverstärkers hin?}{Vor dem Modulator erfolgt eine Hubbegrenzung.}
{Der Verstärker wird übersteuert und erzeugt Oberschwingungen.}
{Das Ansteuersignal ist zu schwach, um den Verstärker voll auszusteuern.}
{Die Schutzdioden im Empfängerzweig begrenzen das Ausgangssignal.}
{\DARCimage{1.0\linewidth}{106include}}\end{PQuestion}

}
\only<2>{
\begin{PQuestion}{AJ207}{Worauf deutet die folgende Wellenform der Ausgangsspannung eines Leistungsverstärkers hin?}{Vor dem Modulator erfolgt eine Hubbegrenzung.}
{\textbf{\textcolor{DARCgreen}{Der Verstärker wird übersteuert und erzeugt Oberschwingungen.}}}
{Das Ansteuersignal ist zu schwach, um den Verstärker voll auszusteuern.}
{Die Schutzdioden im Empfängerzweig begrenzen das Ausgangssignal.}
{\DARCimage{1.0\linewidth}{106include}}\end{PQuestion}

}
\end{frame}

\begin{frame}
\only<1>{
\begin{QQuestion}{AJ210}{Was wird eingesetzt, um die Abstrahlung einer spezifischen Harmonischen wirkungsvoll zu begrenzen?}{Eine Gegentaktendstufe}
{Ein Sperrkreis am Senderausgang}
{Ein Hochpassfilter am Senderausgang}
{Ein Hochpassfilter am Eingang der Senderendstufe}
\end{QQuestion}

}
\only<2>{
\begin{QQuestion}{AJ210}{Was wird eingesetzt, um die Abstrahlung einer spezifischen Harmonischen wirkungsvoll zu begrenzen?}{Eine Gegentaktendstufe}
{\textbf{\textcolor{DARCgreen}{Ein Sperrkreis am Senderausgang}}}
{Ein Hochpassfilter am Senderausgang}
{Ein Hochpassfilter am Eingang der Senderendstufe}
\end{QQuestion}

}
\end{frame}

\begin{frame}
\only<1>{
\begin{QQuestion}{AJ219}{Was passiert, wenn bei einem SSB-Sender die Mikrofonverstärkung zu hoch eingestellt wurde?}{Es werden mehr Nebenprodukte der Sendefrequenz erzeugt, die als unerwünschte Ausstrahlung Störungen hervorrufen.}
{Die Gleichspannungskomponente des Ausgangssignals erhöht sich, wodurch der Wirkungsgrad des Senders abnimmt.}
{Es werden mehr Subharmonische der Sendefrequenz erzeugt, die als unerwünschte Ausstrahlung Splattern auf den benachbarten Frequenzen hervorrufen.}
{Es werden mehr Oberschwingungen der Sendefrequenz erzeugt, die als unerwünschte Ausstrahlung Splattern auf den benachbarten Frequenzen hervorrufen.}
\end{QQuestion}

}
\only<2>{
\begin{QQuestion}{AJ219}{Was passiert, wenn bei einem SSB-Sender die Mikrofonverstärkung zu hoch eingestellt wurde?}{\textbf{\textcolor{DARCgreen}{Es werden mehr Nebenprodukte der Sendefrequenz erzeugt, die als unerwünschte Ausstrahlung Störungen hervorrufen.}}}
{Die Gleichspannungskomponente des Ausgangssignals erhöht sich, wodurch der Wirkungsgrad des Senders abnimmt.}
{Es werden mehr Subharmonische der Sendefrequenz erzeugt, die als unerwünschte Ausstrahlung Splattern auf den benachbarten Frequenzen hervorrufen.}
{Es werden mehr Oberschwingungen der Sendefrequenz erzeugt, die als unerwünschte Ausstrahlung Splattern auf den benachbarten Frequenzen hervorrufen.}
\end{QQuestion}

}
\end{frame}

\begin{frame}
\only<1>{
\begin{QQuestion}{AJ222}{Durch Addition eines Störsignals zur Versorgungsspannung der Senderendstufe wird~...}{NBFM erzeugt.}
{FM erzeugt.}
{AM erzeugt.}
{PM erzeugt.}
\end{QQuestion}

}
\only<2>{
\begin{QQuestion}{AJ222}{Durch Addition eines Störsignals zur Versorgungsspannung der Senderendstufe wird~...}{NBFM erzeugt.}
{FM erzeugt.}
{\textbf{\textcolor{DARCgreen}{AM erzeugt.}}}
{PM erzeugt.}
\end{QQuestion}

}
\end{frame}

\begin{frame}
\only<1>{
\begin{QQuestion}{AJ223}{Wenn der Stromversorgung einer HF-Endstufe NF-Signale überlagert sind, kann dies eine (zusätzliche) unerwünschte Modulation der Sendefrequenz erzeugen. Um welche unerwünschte Modulation handelt es sich?}{AM}
{FM}
{NBFM}
{SSB}
\end{QQuestion}

}
\only<2>{
\begin{QQuestion}{AJ223}{Wenn der Stromversorgung einer HF-Endstufe NF-Signale überlagert sind, kann dies eine (zusätzliche) unerwünschte Modulation der Sendefrequenz erzeugen. Um welche unerwünschte Modulation handelt es sich?}{\textbf{\textcolor{DARCgreen}{AM}}}
{FM}
{NBFM}
{SSB}
\end{QQuestion}

}
\end{frame}

\begin{frame}
\only<1>{
\begin{QQuestion}{AJ224}{Was gilt beim Sendebetrieb für unerwünschte Aussendungen im Frequenzbereich zwischen \num{1,7} und \qty{35}{\MHz}? Sofern die Leistung einer unerwünschten Aussendung~...}{\qty{0,25}{\micro\W} überschreitet, sollte sie um mindestens \qty{40}{\decibel} gegenüber der maximalen PEP des Senders gedämpft werden.}
{\qty{0,25}{\micro\W} überschreitet, sollte sie um mindestens \qty{60}{\decibel} gegenüber der maximalen PEP des Senders gedämpft werden.}
{\qty{1}{\micro\W} überschreitet, sollte sie um mindestens \qty{60}{\decibel} gegenüber der maximalen PEP des Senders gedämpft werden.}
{\qty{1}{\micro\W} überschreitet, sollte sie um mindestens \qty{50}{\decibel} gegenüber der maximalen PEP des Senders gedämpft werden.}
\end{QQuestion}

}
\only<2>{
\begin{QQuestion}{AJ224}{Was gilt beim Sendebetrieb für unerwünschte Aussendungen im Frequenzbereich zwischen \num{1,7} und \qty{35}{\MHz}? Sofern die Leistung einer unerwünschten Aussendung~...}{\textbf{\textcolor{DARCgreen}{\qty{0,25}{\micro\W} überschreitet, sollte sie um mindestens \qty{40}{\decibel} gegenüber der maximalen PEP des Senders gedämpft werden.}}}
{\qty{0,25}{\micro\W} überschreitet, sollte sie um mindestens \qty{60}{\decibel} gegenüber der maximalen PEP des Senders gedämpft werden.}
{\qty{1}{\micro\W} überschreitet, sollte sie um mindestens \qty{60}{\decibel} gegenüber der maximalen PEP des Senders gedämpft werden.}
{\qty{1}{\micro\W} überschreitet, sollte sie um mindestens \qty{50}{\decibel} gegenüber der maximalen PEP des Senders gedämpft werden.}
\end{QQuestion}

}
\end{frame}

\begin{frame}
\only<1>{
\begin{QQuestion}{AJ225}{Was gilt beim Sendebetrieb für unerwünschte Aussendungen im Frequenzbereich zwischen \num{50} und \qty{1000}{\MHz}? Sofern die Leistung einer unerwünschten Aussendung~...}{\qty{0,25}{\micro\W} überschreitet, sollte sie um mindestens \qty{60}{\decibel} gegenüber der maximalen PEP des Senders gedämpft werden.}
{\qty{0,25}{\micro\W} überschreitet, sollte sie um mindestens \qty{40}{\decibel} gegenüber der maximalen PEP des Senders gedämpft werden.}
{\qty{1}{\micro\W} überschreitet, sollte sie um mindestens \qty{60}{\decibel} gegenüber der maximalen PEP des Senders gedämpft werden.}
{\qty{1}{\micro\W} überschreitet, sollte sie um mindestens \qty{50}{\decibel} gegenüber der maximalen PEP des Senders gedämpft werden.}
\end{QQuestion}

}
\only<2>{
\begin{QQuestion}{AJ225}{Was gilt beim Sendebetrieb für unerwünschte Aussendungen im Frequenzbereich zwischen \num{50} und \qty{1000}{\MHz}? Sofern die Leistung einer unerwünschten Aussendung~...}{\textbf{\textcolor{DARCgreen}{\qty{0,25}{\micro\W} überschreitet, sollte sie um mindestens \qty{60}{\decibel} gegenüber der maximalen PEP des Senders gedämpft werden.}}}
{\qty{0,25}{\micro\W} überschreitet, sollte sie um mindestens \qty{40}{\decibel} gegenüber der maximalen PEP des Senders gedämpft werden.}
{\qty{1}{\micro\W} überschreitet, sollte sie um mindestens \qty{60}{\decibel} gegenüber der maximalen PEP des Senders gedämpft werden.}
{\qty{1}{\micro\W} überschreitet, sollte sie um mindestens \qty{50}{\decibel} gegenüber der maximalen PEP des Senders gedämpft werden.}
\end{QQuestion}

}
\end{frame}%ENDCONTENT


\section{Störungen elektronischer Geräte II}
\label{section:stoerungen_elektronischer_geraete_2}
\begin{frame}%STARTCONTENT

\only<1>{
\begin{QQuestion}{AJ116}{Ein Nachbar beschwert sich über Störungen seines Fernsehempfängers, die allerdings auch bei abgezogener TV-Antenne auftreten. Die Störungen fallen zeitlich mit den Übertragungszeiten des Funkamateurs zusammen. Als erster Schritt~...}{ist der EMV-Beauftragte des RTA um Prüfung des Fernsehgeräts zu bitten.}
{ist das Fernsehgerät und der Sender von der Bundesnetzagentur zu überprüfen.}
{ist die Rückseite des Fernsehgeräts zu entfernen und das Gehäuse zu erden.}
{ist ein Netzfilter im Netzkabel des Fernsehgerätes, möglichst nahe am Gerät, vorzusehen.}
\end{QQuestion}

}
\only<2>{
\begin{QQuestion}{AJ116}{Ein Nachbar beschwert sich über Störungen seines Fernsehempfängers, die allerdings auch bei abgezogener TV-Antenne auftreten. Die Störungen fallen zeitlich mit den Übertragungszeiten des Funkamateurs zusammen. Als erster Schritt~...}{ist der EMV-Beauftragte des RTA um Prüfung des Fernsehgeräts zu bitten.}
{ist das Fernsehgerät und der Sender von der Bundesnetzagentur zu überprüfen.}
{ist die Rückseite des Fernsehgeräts zu entfernen und das Gehäuse zu erden.}
{\textbf{\textcolor{DARCgreen}{ist ein Netzfilter im Netzkabel des Fernsehgerätes, möglichst nahe am Gerät, vorzusehen.}}}
\end{QQuestion}

}
\end{frame}

\begin{frame}
\only<1>{
\begin{QQuestion}{AJ117}{Falls nachgewiesen wird, dass Störungen über das Stromversorgungsnetz in Geräte eindringen, ist wahrscheinlich~...}{die Entfernung der Erdung und Neuverlegung des Netzanschlusskabels erforderlich.}
{der Austausch des Netzteils erforderlich.}
{der Einbau eines Netzfilters erforderlich.}
{die Benachrichtigung des zuständigen Stromversorgers erforderlich.}
\end{QQuestion}

}
\only<2>{
\begin{QQuestion}{AJ117}{Falls nachgewiesen wird, dass Störungen über das Stromversorgungsnetz in Geräte eindringen, ist wahrscheinlich~...}{die Entfernung der Erdung und Neuverlegung des Netzanschlusskabels erforderlich.}
{der Austausch des Netzteils erforderlich.}
{\textbf{\textcolor{DARCgreen}{der Einbau eines Netzfilters erforderlich.}}}
{die Benachrichtigung des zuständigen Stromversorgers erforderlich.}
\end{QQuestion}

}
\end{frame}

\begin{frame}
\only<1>{
\begin{question2x2}{AJ118}{Welches der nachfolgenden Filter könnte vor einem Netzanschlusskabel eingeschleift werden, um darüber fließende HF-Ströme wirksam zu dämpfen?}{\DARCimage{1.0\linewidth}{162include}}
{\DARCimage{1.0\linewidth}{163include}}
{\DARCimage{1.0\linewidth}{164include}}
{\DARCimage{1.0\linewidth}{160include}}
\end{question2x2}

}
\only<2>{
\begin{question2x2}{AJ118}{Welches der nachfolgenden Filter könnte vor einem Netzanschlusskabel eingeschleift werden, um darüber fließende HF-Ströme wirksam zu dämpfen?}{\DARCimage{1.0\linewidth}{162include}}
{\DARCimage{1.0\linewidth}{163include}}
{\textbf{\textcolor{DARCgreen}{\DARCimage{1.0\linewidth}{164include}}}}
{\DARCimage{1.0\linewidth}{160include}}
\end{question2x2}

}
\end{frame}

\begin{frame}
\only<1>{
\begin{QQuestion}{AJ105}{Ein starkes HF-Signal gelangt unmittelbar in die ZF-Stufe des Rundfunkempfängers des Nachbarn. Dieses Phänomen wird als~...}{Direktmischung bezeichnet.}
{Direktabsorption bezeichnet.}
{Direkteinstrahlung bezeichnet.}
{HF-Durchschlag bezeichnet.}
\end{QQuestion}

}
\only<2>{
\begin{QQuestion}{AJ105}{Ein starkes HF-Signal gelangt unmittelbar in die ZF-Stufe des Rundfunkempfängers des Nachbarn. Dieses Phänomen wird als~...}{Direktmischung bezeichnet.}
{Direktabsorption bezeichnet.}
{\textbf{\textcolor{DARCgreen}{Direkteinstrahlung bezeichnet.}}}
{HF-Durchschlag bezeichnet.}
\end{QQuestion}

}
\end{frame}

\begin{frame}
\only<1>{
\begin{QQuestion}{AJ103}{Beim Betrieb eines digitalen Eigenbau-Funkempfängers ist dessen Empfang erheblich beeinträchtigt. Dies kann verbessert werden, indem die Leiterplatte~...}{in einem Kunststoffgehäuse untergebracht wird.}
{in Epoxydharz eingegossen wird.}
{über kunststoffisolierte Leitungen angeschlossen wird.}
{in einem geerdeten Metallgehäuse untergebracht wird.}
\end{QQuestion}

}
\only<2>{
\begin{QQuestion}{AJ103}{Beim Betrieb eines digitalen Eigenbau-Funkempfängers ist dessen Empfang erheblich beeinträchtigt. Dies kann verbessert werden, indem die Leiterplatte~...}{in einem Kunststoffgehäuse untergebracht wird.}
{in Epoxydharz eingegossen wird.}
{über kunststoffisolierte Leitungen angeschlossen wird.}
{\textbf{\textcolor{DARCgreen}{in einem geerdeten Metallgehäuse untergebracht wird.}}}
\end{QQuestion}

}
\end{frame}

\begin{frame}
\only<1>{
\begin{QQuestion}{AJ107}{Welche Modulationsverfahren haben das größte Potenzial, einen NF-Verstärker zu beeinflussen, der eine unzureichende Störfestigkeit aufweist?}{Frequenzumtastung (FSK) und Morsetelegrafie (CW).}
{Frequenzmodulation (FM) und Frequenzumtastung (FSK).}
{Einseitenbandmodulation (SSB) und Morsetelegrafie (CW).}
{Einseitenbandmodulation (SSB) und Frequenzmodulation (FM).}
\end{QQuestion}

}
\only<2>{
\begin{QQuestion}{AJ107}{Welche Modulationsverfahren haben das größte Potenzial, einen NF-Verstärker zu beeinflussen, der eine unzureichende Störfestigkeit aufweist?}{Frequenzumtastung (FSK) und Morsetelegrafie (CW).}
{Frequenzmodulation (FM) und Frequenzumtastung (FSK).}
{\textbf{\textcolor{DARCgreen}{Einseitenbandmodulation (SSB) und Morsetelegrafie (CW).}}}
{Einseitenbandmodulation (SSB) und Frequenzmodulation (FM).}
\end{QQuestion}

}
\end{frame}

\begin{frame}
\only<1>{
\begin{QQuestion}{AJ106}{In einem NF-Verstärker erfolgt die unerwünschte Gleichrichtung eines HF-Signals überwiegend~...}{an einem Kupferdraht.}
{an der Lautsprecherleitung.}
{an der Verbindung zweier Widerstände.}
{an einem Basis-Emitter-Übergang.}
\end{QQuestion}

}
\only<2>{
\begin{QQuestion}{AJ106}{In einem NF-Verstärker erfolgt die unerwünschte Gleichrichtung eines HF-Signals überwiegend~...}{an einem Kupferdraht.}
{an der Lautsprecherleitung.}
{an der Verbindung zweier Widerstände.}
{\textbf{\textcolor{DARCgreen}{an einem Basis-Emitter-Übergang.}}}
\end{QQuestion}

}
\end{frame}

\begin{frame}
\only<1>{
\begin{QQuestion}{AJ113}{In der Nähe eines \qty{144}{\MHz}-Senders befindet sich die passive Antenne eines DVB-T2-Fernsehempfängers. Es kommt zu einer Übersteuerung des Empfängers. Das Problem lässt sich durch den Einbau eines~...}{Tiefpassfilters bis \qty{460}{\MHz} in das Antennenzuführungskabel des Fernsehempfängers lösen.}
{Hochpassfilters ab \qty{460}{\MHz} in das Antennenzuführungskabel des Fernsehempfängers lösen.}
{Bandpassfilters für das \qty{2}{\m}-Band vor dem Tuner des Fernsehempfängers lösen.}
{\qty{460}{\MHz}-Notchfilters hinter dem Tuner des Fernsehempfängers lösen.}
\end{QQuestion}

}
\only<2>{
\begin{QQuestion}{AJ113}{In der Nähe eines \qty{144}{\MHz}-Senders befindet sich die passive Antenne eines DVB-T2-Fernsehempfängers. Es kommt zu einer Übersteuerung des Empfängers. Das Problem lässt sich durch den Einbau eines~...}{Tiefpassfilters bis \qty{460}{\MHz} in das Antennenzuführungskabel des Fernsehempfängers lösen.}
{\textbf{\textcolor{DARCgreen}{Hochpassfilters ab \qty{460}{\MHz} in das Antennenzuführungskabel des Fernsehempfängers lösen.}}}
{Bandpassfilters für das \qty{2}{\m}-Band vor dem Tuner des Fernsehempfängers lösen.}
{\qty{460}{\MHz}-Notchfilters hinter dem Tuner des Fernsehempfängers lösen.}
\end{QQuestion}

}
\end{frame}

\begin{frame}
\only<1>{
\begin{QQuestion}{AJ114}{Die Einfügedämpfung im Durchlassbereich eines passiven Hochpassfilters für ein Fernsehantennenkabel sollte~...}{mindestens \qtyrange{80}{100}{\decibel} betragen.}
{höchstens \qtyrange{10}{15}{\decibel} betragen.}
{mindestens \qtyrange{40}{60}{\decibel} betragen.}
{höchstens \qtyrange{2}{3}{\decibel} betragen.}
\end{QQuestion}

}
\only<2>{
\begin{QQuestion}{AJ114}{Die Einfügedämpfung im Durchlassbereich eines passiven Hochpassfilters für ein Fernsehantennenkabel sollte~...}{mindestens \qtyrange{80}{100}{\decibel} betragen.}
{höchstens \qtyrange{10}{15}{\decibel} betragen.}
{mindestens \qtyrange{40}{60}{\decibel} betragen.}
{\textbf{\textcolor{DARCgreen}{höchstens \qtyrange{2}{3}{\decibel} betragen.}}}
\end{QQuestion}

}
\end{frame}

\begin{frame}
\only<1>{
\begin{QQuestion}{AJ108}{Ein unselektiver TV-Antennen-Verstärker wird am wahrscheinlichsten~...}{auf Grund von Netzeinwirkungen beim Betrieb eines nahen Senders störend beeinflusst.}
{durch Übersteuerung mit dem Signal eines nahen Senders störend beeinflusst.}
{durch Einwirkungen auf die Gleichstromversorgung beim Betrieb eines nahen Senders störend beeinflusst.}
{auf Grund seiner zu niedrigen Verstärkung beim Betrieb eines nahen Senders störend beeinflusst.}
\end{QQuestion}

}
\only<2>{
\begin{QQuestion}{AJ108}{Ein unselektiver TV-Antennen-Verstärker wird am wahrscheinlichsten~...}{auf Grund von Netzeinwirkungen beim Betrieb eines nahen Senders störend beeinflusst.}
{\textbf{\textcolor{DARCgreen}{durch Übersteuerung mit dem Signal eines nahen Senders störend beeinflusst.}}}
{durch Einwirkungen auf die Gleichstromversorgung beim Betrieb eines nahen Senders störend beeinflusst.}
{auf Grund seiner zu niedrigen Verstärkung beim Betrieb eines nahen Senders störend beeinflusst.}
\end{QQuestion}

}
\end{frame}

\begin{frame}
\only<1>{
\begin{QQuestion}{AJ112}{Welche Filter sollten im Störungsfall vor die einzelnen Leitungsanschlüsse eines UKW-, DAB- und TV-Empfängers oder anderer angeschlossener Geräte eingeschleift werden, um Kurzwellensignale zu dämpfen?}{Je ein Tiefpassfilter bis \qty{40}{\MHz} unmittelbar vor dem Antennenanschluss und in das Netzkabel der gestörten Geräte.}
{Ein Hochpassfilter ab \qty{40}{\MHz} vor dem Antennenanschluss und zusätzlich je eine hochpermeable Ferritdrossel vor alle Leitungsanschlüsse der gestörten Geräte.}
{Eine Bandsperre für die entsprechenden Empfangsbereiche unmittelbar vor dem Antennenanschluss und ein Tiefpassfilter bis \qty{40}{\MHz} in das Netzkabel der gestörten Geräte.}
{Ein Bandpassfilter für \qty{30}{\MHz} mit \qty{2}{\MHz} Bandbreite unmittelbar vor dem Antennenanschluss und ein Tiefpassfilter bis \qty{30}{\MHz} in das Netzkabel der gestörten Geräte.}
\end{QQuestion}

}
\only<2>{
\begin{QQuestion}{AJ112}{Welche Filter sollten im Störungsfall vor die einzelnen Leitungsanschlüsse eines UKW-, DAB- und TV-Empfängers oder anderer angeschlossener Geräte eingeschleift werden, um Kurzwellensignale zu dämpfen?}{Je ein Tiefpassfilter bis \qty{40}{\MHz} unmittelbar vor dem Antennenanschluss und in das Netzkabel der gestörten Geräte.}
{\textbf{\textcolor{DARCgreen}{Ein Hochpassfilter ab \qty{40}{\MHz} vor dem Antennenanschluss und zusätzlich je eine hochpermeable Ferritdrossel vor alle Leitungsanschlüsse der gestörten Geräte.}}}
{Eine Bandsperre für die entsprechenden Empfangsbereiche unmittelbar vor dem Antennenanschluss und ein Tiefpassfilter bis \qty{40}{\MHz} in das Netzkabel der gestörten Geräte.}
{Ein Bandpassfilter für \qty{30}{\MHz} mit \qty{2}{\MHz} Bandbreite unmittelbar vor dem Antennenanschluss und ein Tiefpassfilter bis \qty{30}{\MHz} in das Netzkabel der gestörten Geräte.}
\end{QQuestion}

}
\end{frame}

\begin{frame}
\only<1>{
\begin{QQuestion}{AJ104}{Um die Möglichkeit unerwünschter Abstrahlungen mit Hilfe eines angepassten Antennensystems zu verringern, empfiehlt es sich~...}{die Netzspannung mit einem Bandpass für die Nutzfrequenz zu filtern.}
{mit einem hohen Stehwellenverhältnis zu arbeiten.}
{einen Antennentuner und/oder ein Filter zu verwenden.}
{nur vertikal polarisierte Antennen zu verwenden.}
\end{QQuestion}

}
\only<2>{
\begin{QQuestion}{AJ104}{Um die Möglichkeit unerwünschter Abstrahlungen mit Hilfe eines angepassten Antennensystems zu verringern, empfiehlt es sich~...}{die Netzspannung mit einem Bandpass für die Nutzfrequenz zu filtern.}
{mit einem hohen Stehwellenverhältnis zu arbeiten.}
{\textbf{\textcolor{DARCgreen}{einen Antennentuner und/oder ein Filter zu verwenden.}}}
{nur vertikal polarisierte Antennen zu verwenden.}
\end{QQuestion}

}
\end{frame}

\begin{frame}
\only<1>{
\begin{QQuestion}{AJ110}{Das Sendesignal eines VHF-Senders verursacht Empfangsstörungen in einem benachbarten DAB-Radio. Ein möglicher Grund hierfür ist~...}{eine nicht ausreichende Oberwellenunterdrückung des VHF-Senders.}
{die unterschiedliche Polarisation von VHF-Sende- und DAB-Empfangsantenne.}
{eine zu große Hubeinstellung am VHF-Sender.}
{eine Übersteuerung des Empfängereingangs des DAB-Radios.}
\end{QQuestion}

}
\only<2>{
\begin{QQuestion}{AJ110}{Das Sendesignal eines VHF-Senders verursacht Empfangsstörungen in einem benachbarten DAB-Radio. Ein möglicher Grund hierfür ist~...}{eine nicht ausreichende Oberwellenunterdrückung des VHF-Senders.}
{die unterschiedliche Polarisation von VHF-Sende- und DAB-Empfangsantenne.}
{eine zu große Hubeinstellung am VHF-Sender.}
{\textbf{\textcolor{DARCgreen}{eine Übersteuerung des Empfängereingangs des DAB-Radios.}}}
\end{QQuestion}

}
\end{frame}

\begin{frame}
\only<1>{
\begin{QQuestion}{AJ111}{Wie können sich störende Beeinflussungen in digitalen Rundfunkempfängern (DAB+) äußern?}{Der Rundfunkempfang bleibt einwandfrei, da die digitale Fehlerkorrektur alle Störungen eliminiert.}
{Die Differenz zwischen Störsignalfrequenz und der Abtastfrequenz ist im Gerätelautsprecher hörbar.}
{Die Lautstärke des Rundfunkempfangs schwankt sehr stark.}
{Der Empfänger produziert Störgeräusche und/oder schaltet stumm.}
\end{QQuestion}

}
\only<2>{
\begin{QQuestion}{AJ111}{Wie können sich störende Beeinflussungen in digitalen Rundfunkempfängern (DAB+) äußern?}{Der Rundfunkempfang bleibt einwandfrei, da die digitale Fehlerkorrektur alle Störungen eliminiert.}
{Die Differenz zwischen Störsignalfrequenz und der Abtastfrequenz ist im Gerätelautsprecher hörbar.}
{Die Lautstärke des Rundfunkempfangs schwankt sehr stark.}
{\textbf{\textcolor{DARCgreen}{Der Empfänger produziert Störgeräusche und/oder schaltet stumm.}}}
\end{QQuestion}

}
\end{frame}

\begin{frame}
\only<1>{
\begin{QQuestion}{AJ109}{Ein SSB-Sender bei \qty{432,2}{\MHz} erzeugt an einer Richtantenne, welche unmittelbar auf die DVB-T2-Fernsehantenne des Nachbarn gerichtet ist, eine effektive Strahlungsleistung von \qty{1,8}{\kW} ERP. Dies führt gegebenenfalls~...}{zu unerwünschten Reflexionen des Sendesignals.}
{zur Erzeugung von parasitären Schwingungen.}
{zur Übersteuerung der Vorstufe des Fernsehgerätes.}
{zu Störungen der IR-Fernbedienung des Fernsehgerätes.}
\end{QQuestion}

}
\only<2>{
\begin{QQuestion}{AJ109}{Ein SSB-Sender bei \qty{432,2}{\MHz} erzeugt an einer Richtantenne, welche unmittelbar auf die DVB-T2-Fernsehantenne des Nachbarn gerichtet ist, eine effektive Strahlungsleistung von \qty{1,8}{\kW} ERP. Dies führt gegebenenfalls~...}{zu unerwünschten Reflexionen des Sendesignals.}
{zur Erzeugung von parasitären Schwingungen.}
{\textbf{\textcolor{DARCgreen}{zur Übersteuerung der Vorstufe des Fernsehgerätes.}}}
{zu Störungen der IR-Fernbedienung des Fernsehgerätes.}
\end{QQuestion}

}
\end{frame}

\begin{frame}
\only<1>{
\begin{QQuestion}{AJ101}{Um die Wahrscheinlichkeit zu verringern, andere Stationen zu stören, sollte die benutzte Sendeleistung~...}{die Hälfte des maximal zulässigen Pegels betragen.}
{auf den maximal zulässigen Pegel eingestellt werden.}
{auf die für eine zufriedenstellende Kommunikation erforderlichen \qty{750}{\W} eingestellt werden.}
{auf das für eine zufriedenstellende Kommunikation erforderliche Minimum eingestellt werden.}
\end{QQuestion}

}
\only<2>{
\begin{QQuestion}{AJ101}{Um die Wahrscheinlichkeit zu verringern, andere Stationen zu stören, sollte die benutzte Sendeleistung~...}{die Hälfte des maximal zulässigen Pegels betragen.}
{auf den maximal zulässigen Pegel eingestellt werden.}
{auf die für eine zufriedenstellende Kommunikation erforderlichen \qty{750}{\W} eingestellt werden.}
{\textbf{\textcolor{DARCgreen}{auf das für eine zufriedenstellende Kommunikation erforderliche Minimum eingestellt werden.}}}
\end{QQuestion}

}
\end{frame}

\begin{frame}
\only<1>{
\begin{QQuestion}{AJ119}{Welche Art von Kondensatoren sollte zum Abblocken von HF-Spannungen vorzugsweise verwendet werden? Am besten verwendet man~...}{Keramikkondensatoren.}
{Aluminium-Elektrolytkondensatoren.}
{Tantalkondensatoren.}
{Polykarbonatkondensatoren.}
\end{QQuestion}

}
\only<2>{
\begin{QQuestion}{AJ119}{Welche Art von Kondensatoren sollte zum Abblocken von HF-Spannungen vorzugsweise verwendet werden? Am besten verwendet man~...}{\textbf{\textcolor{DARCgreen}{Keramikkondensatoren.}}}
{Aluminium-Elektrolytkondensatoren.}
{Tantalkondensatoren.}
{Polykarbonatkondensatoren.}
\end{QQuestion}

}
\end{frame}

\begin{frame}
\only<1>{
\begin{QQuestion}{AJ102}{Eine wirksame HF-Erdung sollte im genutzten Frequenzbereich~...}{induktiv gekoppelt sein.}
{über eine hohe Reaktanz verfügen.}
{über eine hohe Impedanz verfügen.}
{über eine niedrige Impedanz verfügen.}
\end{QQuestion}

}
\only<2>{
\begin{QQuestion}{AJ102}{Eine wirksame HF-Erdung sollte im genutzten Frequenzbereich~...}{induktiv gekoppelt sein.}
{über eine hohe Reaktanz verfügen.}
{über eine hohe Impedanz verfügen.}
{\textbf{\textcolor{DARCgreen}{über eine niedrige Impedanz verfügen.}}}
\end{QQuestion}

}
\end{frame}

\begin{frame}
\only<1>{
\begin{QQuestion}{AJ214}{In HF-Schaltungen können Nebenresonanzen durch die~...}{Stromversorgung hervorgerufen werden.}
{Eigenresonanz der HF-Drosseln hervorgerufen werden.}
{Sättigung der Kerne der HF-Spulen hervorgerufen werden.}
{Widerstandseigenschaft einer Drossel hervorgerufen werden.}
\end{QQuestion}

}
\only<2>{
\begin{QQuestion}{AJ214}{In HF-Schaltungen können Nebenresonanzen durch die~...}{Stromversorgung hervorgerufen werden.}
{\textbf{\textcolor{DARCgreen}{Eigenresonanz der HF-Drosseln hervorgerufen werden.}}}
{Sättigung der Kerne der HF-Spulen hervorgerufen werden.}
{Widerstandseigenschaft einer Drossel hervorgerufen werden.}
\end{QQuestion}

}
\end{frame}%ENDCONTENT


\section{Remote-Station}
\label{section:remote_station}
\begin{frame}%STARTCONTENT

\only<1>{
\begin{PQuestion}{AF701}{Sie wollen Remote-Betrieb mit dem im Blockdiagramm dargestellten Aufbau durchführen. Welche Geräte könnten Sie als Block 1 verwenden?}{Tuner oder Transceiver}
{Computer oder Bedienteil}
{Verstärker oder Netzteil}
{Verstärker oder Computer}
{\DARCimage{1.0\linewidth}{501include}}\end{PQuestion}

}
\only<2>{
\begin{PQuestion}{AF701}{Sie wollen Remote-Betrieb mit dem im Blockdiagramm dargestellten Aufbau durchführen. Welche Geräte könnten Sie als Block 1 verwenden?}{Tuner oder Transceiver}
{\textbf{\textcolor{DARCgreen}{Computer oder Bedienteil}}}
{Verstärker oder Netzteil}
{Verstärker oder Computer}
{\DARCimage{1.0\linewidth}{501include}}\end{PQuestion}

}
\end{frame}

\begin{frame}
\only<1>{
\begin{PQuestion}{AF702}{Sie wollen Remote-Betrieb mit dem im Blockdiagramm dargestellten Aufbau durchführen. Welche Geräte könnten Sie als Block 2 verwenden?}{Verstärker oder Netzteil}
{Computer oder Netzteil}
{Remote-Tuner oder Transceiver}
{Computer oder Remote-Interface}
{\DARCimage{1.0\linewidth}{501include}}\end{PQuestion}

}
\only<2>{
\begin{PQuestion}{AF702}{Sie wollen Remote-Betrieb mit dem im Blockdiagramm dargestellten Aufbau durchführen. Welche Geräte könnten Sie als Block 2 verwenden?}{Verstärker oder Netzteil}
{Computer oder Netzteil}
{Remote-Tuner oder Transceiver}
{\textbf{\textcolor{DARCgreen}{Computer oder Remote-Interface}}}
{\DARCimage{1.0\linewidth}{501include}}\end{PQuestion}

}
\end{frame}

\begin{frame}
\only<1>{
\begin{PQuestion}{AF704}{Sie führen Telefonie im Remote-Betrieb mit dem dargestellten Aufbau durch. Welche Komponente wandelt Datenpakete aus dem Netzwerk in Audio- und Steuersignale für die Aussendung um?}{Netzwerk}
{Block 1}
{Block 2}
{Block 3}
{\DARCimage{1.0\linewidth}{501include}}\end{PQuestion}

}
\only<2>{
\begin{PQuestion}{AF704}{Sie führen Telefonie im Remote-Betrieb mit dem dargestellten Aufbau durch. Welche Komponente wandelt Datenpakete aus dem Netzwerk in Audio- und Steuersignale für die Aussendung um?}{Netzwerk}
{Block 1}
{\textbf{\textcolor{DARCgreen}{Block 2}}}
{Block 3}
{\DARCimage{1.0\linewidth}{501include}}\end{PQuestion}

}
\end{frame}

\begin{frame}
\only<1>{
\begin{PQuestion}{AF703}{Sie führen Telefonie im Remote-Betrieb mit dem dargestellten Aufbau durch. Welche Komponente wandelt Audio- und Steuersignale des Operators in Datenpakete für die Übertragung im Netzwerk um?}{Netzwerk}
{Block 2}
{Block 1}
{Block 3}
{\DARCimage{1.0\linewidth}{501include}}\end{PQuestion}

}
\only<2>{
\begin{PQuestion}{AF703}{Sie führen Telefonie im Remote-Betrieb mit dem dargestellten Aufbau durch. Welche Komponente wandelt Audio- und Steuersignale des Operators in Datenpakete für die Übertragung im Netzwerk um?}{Netzwerk}
{Block 2}
{\textbf{\textcolor{DARCgreen}{Block 1}}}
{Block 3}
{\DARCimage{1.0\linewidth}{501include}}\end{PQuestion}

}
\end{frame}

\begin{frame}
\only<1>{
\begin{PQuestion}{AF705}{Sie führen Telefonie im Remote-Betrieb mit dem dargestellten Aufbau durch. Welche Komponente erzeugt den auszusendenden Hochfrequenzträger?}{Block 1}
{Block 3}
{Block 2}
{Netzwerk}
{\DARCimage{1.0\linewidth}{501include}}\end{PQuestion}

}
\only<2>{
\begin{PQuestion}{AF705}{Sie führen Telefonie im Remote-Betrieb mit dem dargestellten Aufbau durch. Welche Komponente erzeugt den auszusendenden Hochfrequenzträger?}{Block 1}
{\textbf{\textcolor{DARCgreen}{Block 3}}}
{Block 2}
{Netzwerk}
{\DARCimage{1.0\linewidth}{501include}}\end{PQuestion}

}
\end{frame}

\begin{frame}
\only<1>{
\begin{QQuestion}{AF709}{Welche technische Besonderheit bei der Nutzung einer Remote-Station wirkt sich auf den Funkbetrieb aus?}{Die Signale kommen zu früh an.}
{Die Signale kommen verzögert an.}
{Die Impedanz der Netzwerkverkabelung ist größer als \qty{50}{\ohm}.}
{Die Impedanz der Netzwerkverkabelung ist kleiner als \qty{50}{\ohm}.}
\end{QQuestion}

}
\only<2>{
\begin{QQuestion}{AF709}{Welche technische Besonderheit bei der Nutzung einer Remote-Station wirkt sich auf den Funkbetrieb aus?}{Die Signale kommen zu früh an.}
{\textbf{\textcolor{DARCgreen}{Die Signale kommen verzögert an.}}}
{Die Impedanz der Netzwerkverkabelung ist größer als \qty{50}{\ohm}.}
{Die Impedanz der Netzwerkverkabelung ist kleiner als \qty{50}{\ohm}.}
\end{QQuestion}

}
\end{frame}

\begin{frame}
\only<1>{
\begin{QQuestion}{AF708}{Wodurch kann bei Remote-Betrieb verhindert werden, dass der Sender trotz Ausfall der Verbindung zwischen Operator und Remote-Station dauerhaft auf Sendung bleibt?}{Watchdog}
{VOX-Schaltung beim Operator}
{Firewall}
{Unterbrechungsfreie Spannungsversorgung}
\end{QQuestion}

}
\only<2>{
\begin{QQuestion}{AF708}{Wodurch kann bei Remote-Betrieb verhindert werden, dass der Sender trotz Ausfall der Verbindung zwischen Operator und Remote-Station dauerhaft auf Sendung bleibt?}{\textbf{\textcolor{DARCgreen}{Watchdog}}}
{VOX-Schaltung beim Operator}
{Firewall}
{Unterbrechungsfreie Spannungsversorgung}
\end{QQuestion}

}
\end{frame}

\begin{frame}
\only<1>{
\begin{QQuestion}{AF707}{Sie führen FM-Sprechfunk über Ihre Remote-Station durch. Aufgrund einer Fehlfunktion des Transceivers reagiert dieser nicht mehr auf Steuersignale. Wie können Sie die Sendung sofort beenden?}{Herunterfahren des Internetrouters auf der Remoteseite}
{Unterbrechen des Audio-Streams, z.~B. durch Abschalten des VPNs}
{Herunterfahren des Internetrouters auf der Kontrollseite}
{Fernabschalten der Versorgungsspannung, z.~B. mittels IP-Steckdose}
\end{QQuestion}

}
\only<2>{
\begin{QQuestion}{AF707}{Sie führen FM-Sprechfunk über Ihre Remote-Station durch. Aufgrund einer Fehlfunktion des Transceivers reagiert dieser nicht mehr auf Steuersignale. Wie können Sie die Sendung sofort beenden?}{Herunterfahren des Internetrouters auf der Remoteseite}
{Unterbrechen des Audio-Streams, z.~B. durch Abschalten des VPNs}
{Herunterfahren des Internetrouters auf der Kontrollseite}
{\textbf{\textcolor{DARCgreen}{Fernabschalten der Versorgungsspannung, z.~B. mittels IP-Steckdose}}}
\end{QQuestion}

}
\end{frame}

\begin{frame}
\only<1>{
\begin{QQuestion}{AF706}{Sie nutzen Ihre weit entfernte Remote-Station. Es kommt zu problematischer Einstrahlung oder Einströmung durch ihre eigene Aussendung. Was kann dadurch beeinträchtigt werden?}{Das Mikrofon oder der Lautsprecher des Operators}
{Die Abspannung der Antennenanlage}
{Der Transceiver oder dort befindliche Komponenten für die Fernsteuerung}
{Das lokale Netzwerk des Operators}
\end{QQuestion}

}
\only<2>{
\begin{QQuestion}{AF706}{Sie nutzen Ihre weit entfernte Remote-Station. Es kommt zu problematischer Einstrahlung oder Einströmung durch ihre eigene Aussendung. Was kann dadurch beeinträchtigt werden?}{Das Mikrofon oder der Lautsprecher des Operators}
{Die Abspannung der Antennenanlage}
{\textbf{\textcolor{DARCgreen}{Der Transceiver oder dort befindliche Komponenten für die Fernsteuerung}}}
{Das lokale Netzwerk des Operators}
\end{QQuestion}

}
\end{frame}%ENDCONTENT


\title{DARC Amateurfunklehrgang Klasse A}
\author{Digitale Übertragungsverfahren}
\institute{Deutscher Amateur Radio Club e.\,V.}
\begin{frame}
\maketitle
\end{frame}

\section{Phasenumtastung (PSK)}
\label{section:psk}
\begin{frame}%STARTCONTENT

\only<1>{
\begin{question2x2}{AE401}{Welches der folgenden Diagramme zeigt einen erkennbar durch Phasenumtastung (PSK) modulierten Träger?}{\DARCimage{1.0\linewidth}{357include}}
{\DARCimage{1.0\linewidth}{356include}}
{\DARCimage{1.0\linewidth}{359include}}
{\DARCimage{1.0\linewidth}{358include}}
\end{question2x2}

}
\only<2>{
\begin{question2x2}{AE401}{Welches der folgenden Diagramme zeigt einen erkennbar durch Phasenumtastung (PSK) modulierten Träger?}{\DARCimage{1.0\linewidth}{357include}}
{\DARCimage{1.0\linewidth}{356include}}
{\textbf{\textcolor{DARCgreen}{\DARCimage{1.0\linewidth}{359include}}}}
{\DARCimage{1.0\linewidth}{358include}}
\end{question2x2}

}
\end{frame}%ENDCONTENT


\section{Symbolumschaltung und Bandbreite}
\label{section:symbolumschaltung_bandbreite}
\begin{frame}%STARTCONTENT
\begin{itemize}
  \item Als Symbol werden in der Digitaltechnik die verschiedenen Zeicheneinheiten zur Übertragung des Informationsgehaltes bezeichnet.
  \item Die Anzahl der pro Zeitspanne übertragenen Symbole ist die Symbolrate und wird in der Einheit Baud ausgedrückt.
  \item Bei jeder Umschaltung zwischen zwei Symbolen wird die Amplitude, Frequenz oder Phase eines Trägers geändert.
  \item Je schneller Amplitude, Frequenz oder Phase verändert werden, umso breitbandiger wird das erzeugte Signal.
  \end{itemize}
\end{frame}

\begin{frame}
\only<1>{
\begin{QQuestion}{AE415}{Welche Auswirkung hat eine Erhöhung der Umschaltgeschwindigkeit zwischen verschiedenen Symbolen bei digitalen Übertragungsverfahren auf die benötigte Bandbreite? Die Bandbreite~...}{bleibt gleich.}
{sinkt.}
{steigt.}
{steigt im oberen und sinkt im unteren Seitenband.}
\end{QQuestion}

}
\only<2>{
\begin{QQuestion}{AE415}{Welche Auswirkung hat eine Erhöhung der Umschaltgeschwindigkeit zwischen verschiedenen Symbolen bei digitalen Übertragungsverfahren auf die benötigte Bandbreite? Die Bandbreite~...}{bleibt gleich.}
{sinkt.}
{\textbf{\textcolor{DARCgreen}{steigt.}}}
{steigt im oberen und sinkt im unteren Seitenband.}
\end{QQuestion}

}
\end{frame}

\begin{frame}
\only<1>{
\begin{question2x2}{AE214}{Welches dieser amplitudenmodulierten Signale belegt die geringste Bandbreite?}{\DARCimage{1.0\linewidth}{599include}}
{\DARCimage{1.0\linewidth}{598include}}
{\DARCimage{1.0\linewidth}{597include}}
{\DARCimage{1.0\linewidth}{601include}}
\end{question2x2}

}
\only<2>{
\begin{question2x2}{AE214}{Welches dieser amplitudenmodulierten Signale belegt die geringste Bandbreite?}{\DARCimage{1.0\linewidth}{599include}}
{\DARCimage{1.0\linewidth}{598include}}
{\textbf{\textcolor{DARCgreen}{\DARCimage{1.0\linewidth}{597include}}}}
{\DARCimage{1.0\linewidth}{601include}}
\end{question2x2}

}
\end{frame}

\begin{frame}\begin{itemize}
  \item Von der Morsetelegrafie kennen wir bereits Tastklicks, die breitbandige Störungen darstellen.
  \item Sie entstehen, wenn beim Drücken bzw. Loslassen der Morsetaste der Träger plötzlich ein- bzw. ausgeschaltet wird.
  \end{itemize}
\end{frame}

\begin{frame}
\only<1>{
\begin{question2x2}{AJ221}{In den nachfolgenden Bildern sind mögliche Signalverläufe des Senderausgangssignals bei der CW-Tastung dargestellt. Welcher Signalverlauf führt zu den geringsten Störungen?}{\DARCimage{1.0\linewidth}{21include}}
{\DARCimage{1.0\linewidth}{20include}}
{\DARCimage{1.0\linewidth}{19include}}
{\DARCimage{1.0\linewidth}{22include}}
\end{question2x2}

}
\only<2>{
\begin{question2x2}{AJ221}{In den nachfolgenden Bildern sind mögliche Signalverläufe des Senderausgangssignals bei der CW-Tastung dargestellt. Welcher Signalverlauf führt zu den geringsten Störungen?}{\DARCimage{1.0\linewidth}{21include}}
{\DARCimage{1.0\linewidth}{20include}}
{\textbf{\textcolor{DARCgreen}{\DARCimage{1.0\linewidth}{19include}}}}
{\DARCimage{1.0\linewidth}{22include}}
\end{question2x2}

}
\end{frame}

\begin{frame}
\only<1>{
\begin{PQuestion}{AJ220}{Diese Modulationshüllkurve eines CW-Senders sollte vermieden werden, da~...}{wahrscheinlich Tastklicks erzeugt werden.}
{während der Aussetzer Probleme im Leistungsverstärker entstehen könnten.}
{die ausgesendeten Signale schwierig zu lesen sind.}
{die Stromversorgung überlastet wird.}
{\DARCimage{1.0\linewidth}{12include}}\end{PQuestion}

}
\only<2>{
\begin{PQuestion}{AJ220}{Diese Modulationshüllkurve eines CW-Senders sollte vermieden werden, da~...}{\textbf{\textcolor{DARCgreen}{wahrscheinlich Tastklicks erzeugt werden.}}}
{während der Aussetzer Probleme im Leistungsverstärker entstehen könnten.}
{die ausgesendeten Signale schwierig zu lesen sind.}
{die Stromversorgung überlastet wird.}
{\DARCimage{1.0\linewidth}{12include}}\end{PQuestion}

}
\end{frame}%ENDCONTENT


\section{Mehrwertige Verfahren}
\label{section:mehrwertige_verfahren}
\begin{frame}%STARTCONTENT
\begin{itemize}
  \item Viele digitale Modulationsverfahren verwenden mehr als zwei Symbole.
  \item So funktioniert zum Beispiel die 4-Fach-Amplitudenumtastung (4ASK) mit vier unterschiedlichen Amplituden, 25 %, 50 %, 75 %, 100 % des Maximums.
  \item So lassen sich zwei Bits zu einem Symbol zusammenfassen und gleichzeitig übertragen.
  \end{itemize}

\begin{figure}
    \DARCimage{0.85\linewidth}{701include}
    \caption{\scriptsize Quaternäre Amplitudenumtastung (Quaternary Amplitude-shift Keying)}
    \label{4ask}
\end{figure}

\end{frame}

\begin{frame}\begin{itemize}
  \item Dieses Prinzip lässt sich auf die Frequenz- und Phasenumtastung übertragen.
  \item Eine einfache Phasenumtastung (Binary Phase-Shift Keying, BPSK) verwendet nur zwei verschiedene Phasenlagen und kann daher nur ein Bit gleichzeitig senden.
  \item Die Quadraturphasenumtastung (Quadrature Phase-Shift Keying, QPSK) hingegen nutzt vier verschiedene Phasenlagen (0 °, 90 °, 180 ° und 270 °) und überträgt somit zwei Bits in jedem Schritt.
  \end{itemize}
\end{frame}

\begin{frame}
\only<1>{
\begin{QQuestion}{AE402}{Was unterscheidet BPSK- und QPSK-Modulation?}{Mit BPSK wird ein Bit pro Symbol übertragen, mit QPSK zwei Bit pro Symbol.}
{Mit QPSK wird ein Bit pro Symbol übertragen, mit BPSK zwei Bit pro Symbol.}
{Bei BPSK werden der I- und der Q-Anteil eines I/Q-Signals vertauscht, bei QPSK nicht.}
{Bei QPSK werden der I- und der Q-Anteil eines I/Q-Signals vertauscht, bei BPSK nicht.}
\end{QQuestion}

}
\only<2>{
\begin{QQuestion}{AE402}{Was unterscheidet BPSK- und QPSK-Modulation?}{\textbf{\textcolor{DARCgreen}{Mit BPSK wird ein Bit pro Symbol übertragen, mit QPSK zwei Bit pro Symbol.}}}
{Mit QPSK wird ein Bit pro Symbol übertragen, mit BPSK zwei Bit pro Symbol.}
{Bei BPSK werden der I- und der Q-Anteil eines I/Q-Signals vertauscht, bei QPSK nicht.}
{Bei QPSK werden der I- und der Q-Anteil eines I/Q-Signals vertauscht, bei BPSK nicht.}
\end{QQuestion}

}
\end{frame}

\begin{frame}\begin{itemize}
  \item Da bei Verfahren wie QPSK mehr als ein Bit pro Symbol übertragen wird, müssen wir mit den Einheiten aufpassen.
  \item Werden nur zwei Symbole verwendet und somit jedes Bit einzeln gesendet, entspricht die Symbolrate in Baud der Datenrate in Bit/s.
  \item Werden jedoch mehr Symbole verwendet und somit mehrere Bits gleichzeitig übertragen, ist die Datenrate höher als die Symbolrate.
  \end{itemize}
\end{frame}

\begin{frame}\begin{itemize}
  \item Die Formel $C = R_{ s } \cdot n$ stellt den Zusammenhang dar:
  \end{itemize}
C → Datenübertragungsrate in Bit/s

$R_{ s }$ → Symbolrate in Baud

n → Symbolgröße in Bit/Symbol

\end{frame}

\begin{frame}
\only<1>{
\begin{QQuestion}{AA104}{Welche Einheit wird üblicherweise für die Symbolrate verwendet?}{Bit pro Sekunde (Bit/s)}
{Baud (Bd)}
{Hertz (Hz)}
{Dezibel (dB)}
\end{QQuestion}

}
\only<2>{
\begin{QQuestion}{AA104}{Welche Einheit wird üblicherweise für die Symbolrate verwendet?}{Bit pro Sekunde (Bit/s)}
{\textbf{\textcolor{DARCgreen}{Baud (Bd)}}}
{Hertz (Hz)}
{Dezibel (dB)}
\end{QQuestion}

}
\end{frame}

\begin{frame}Beispiele:

\emph{RTTY}: Umschaltung zwischen zwei Symbolfrequenzen, so dass pro Symbol ein Bit (0 oder 1) übertragen werden kann.

→ Datenrate = Symbolrate

\emph{FT4}: Umschaltung zwischen vier Symbolfrequenzen, so dass pro Symbol zwei Bit (00, 01, 10 oder 11) übertragen werden können.

→ Datenrate = 2 $\cdot$ Symbolrate

\end{frame}

\begin{frame}
\only<1>{
\begin{QQuestion}{AE405}{Bei einem digitalen Übertragungsverfahren (z.~B. RTTY) wird die Frequenz eines Senders zwischen zwei Symbolfrequenzen (z.~B. \qty{14072,43}{\kHz} und \qty{14072,60}{\kHz}) umgetastet, so dass pro Symbol ein Bit (0 oder 1) übertragen werden kann. Die Symbolrate beträgt \qty{45,45}{\baud}. Welcher Datenrate entspricht das?}{\qty[per-mode=symbol]{22,725}{\bit\per\s}}
{\qty[per-mode=symbol]{90,9}{\bit\per\s}}
{\qty[per-mode=symbol]{45,45}{\bit\per\s}}
{\qty[per-mode=symbol]{181,8}{\bit\per\s}}
\end{QQuestion}

}
\only<2>{
\begin{QQuestion}{AE405}{Bei einem digitalen Übertragungsverfahren (z.~B. RTTY) wird die Frequenz eines Senders zwischen zwei Symbolfrequenzen (z.~B. \qty{14072,43}{\kHz} und \qty{14072,60}{\kHz}) umgetastet, so dass pro Symbol ein Bit (0 oder 1) übertragen werden kann. Die Symbolrate beträgt \qty{45,45}{\baud}. Welcher Datenrate entspricht das?}{\qty[per-mode=symbol]{22,725}{\bit\per\s}}
{\qty[per-mode=symbol]{90,9}{\bit\per\s}}
{\textbf{\textcolor{DARCgreen}{\qty[per-mode=symbol]{45,45}{\bit\per\s}}}}
{\qty[per-mode=symbol]{181,8}{\bit\per\s}}
\end{QQuestion}

}
\end{frame}

\begin{frame}
\frametitle{Lösungsweg}
\begin{itemize}
  \item gegeben: $R_S = 45,45Bd$
  \item gegeben: $n=1\frac{Bit}{Symbol}$
  \item gesucht: $C$
  \end{itemize}
    \pause
    $C = R_S \cdot n = 45,45Bd \cdot 1 = 45,45\frac{Bit}{s}$



\end{frame}

\begin{frame}
\only<1>{
\begin{QQuestion}{AE406}{Bei einem digitalen Übertragungsverfahren (z.~B. FT4) wird die Frequenz eines Senders zwischen vier Symbolfrequenzen (z.~B. \qty{14081,20}{\kHz}, \qty{14081,40}{\kHz}, \qty{14081,61}{\kHz} und \qty{14081,83}{\kHz}) umgetastet, so dass pro Symbol zwei Bit (00, 01, 10 oder 11) übertragen werden können. Die Symbolrate beträgt \qty{23,4}{\baud}. Welcher Datenrate entspricht das?}{\qty[per-mode=symbol]{93,6}{\bit\per\s}}
{\qty[per-mode=symbol]{11,7}{\bit\per\s}}
{\qty[per-mode=symbol]{23,4}{\bit\per\s}}
{\qty[per-mode=symbol]{46,8}{\bit\per\s}}
\end{QQuestion}

}
\only<2>{
\begin{QQuestion}{AE406}{Bei einem digitalen Übertragungsverfahren (z.~B. FT4) wird die Frequenz eines Senders zwischen vier Symbolfrequenzen (z.~B. \qty{14081,20}{\kHz}, \qty{14081,40}{\kHz}, \qty{14081,61}{\kHz} und \qty{14081,83}{\kHz}) umgetastet, so dass pro Symbol zwei Bit (00, 01, 10 oder 11) übertragen werden können. Die Symbolrate beträgt \qty{23,4}{\baud}. Welcher Datenrate entspricht das?}{\qty[per-mode=symbol]{93,6}{\bit\per\s}}
{\qty[per-mode=symbol]{11,7}{\bit\per\s}}
{\qty[per-mode=symbol]{23,4}{\bit\per\s}}
{\textbf{\textcolor{DARCgreen}{\qty[per-mode=symbol]{46,8}{\bit\per\s}}}}
\end{QQuestion}

}
\end{frame}

\begin{frame}
\frametitle{Lösungsweg}
\begin{itemize}
  \item gegeben: $R_S = 23,4Bd$
  \item gegeben: $n=2\frac{Bit}{Symbol}$
  \item gesucht: $C$
  \end{itemize}
    \pause
    $C = R_S \cdot n = 23,4 \cdot 2 = 46,8\frac{Bit}{s}$



\end{frame}%ENDCONTENT


\section{Quadraturamplitudenmodulation  (QAM)}
\label{section:qam}
\begin{frame}%STARTCONTENT
\begin{itemize}
  \item Es scheint zunächst nahe zu liegen, die Anzahl der Symbole möglichst groß zu wählen, damit pro Symbol möglichst viele Informationen übertragen werden können.
  \item Doch dann muss ein Empfänger z.B. zwischen vielen unterschiedlichen Amplituden unterscheiden können. Somit wird das Verfahren anfälliger für Störungen.
  \end{itemize}
\end{frame}

\begin{frame}\begin{itemize}
  \item Trick: Anstelle der Änderung nur eines Parameters (z.B. der Amplitude) werden pro Symbol zwei Parameter verändert, nämlich die Amplitude und die Phase.
  \item Ein Symbol entspricht dann einer Kombination einer bestimmten Amplitude mit einer bestimmten Phasenlage.
  \end{itemize}

\begin{figure}
    \DARCimage{0.85\linewidth}{702include}
    \caption{\scriptsize Signalverlauf eines 8QAM-Signals, je Symbol mit Amplitude (0,5 bzw. 1), Phasenlage und 3-stelliger Bitfolge}
    \label{8qam}
\end{figure}

\end{frame}

\begin{frame}
\only<1>{
\begin{QQuestion}{AE403}{Wie werden Informationen bei der Quadraturamplitudenmodulation (QAM) mittels eines Trägers übertragen? Durch~...}{richtungsabhängige Änderung der Frequenz}
{nichtlineare Änderung der Amplitude}
{separate Änderung des elektrischen und magnetischen Feldwellenanteils}
{Änderung der Amplitude und der Phase}
\end{QQuestion}

}
\only<2>{
\begin{QQuestion}{AE403}{Wie werden Informationen bei der Quadraturamplitudenmodulation (QAM) mittels eines Trägers übertragen? Durch~...}{richtungsabhängige Änderung der Frequenz}
{nichtlineare Änderung der Amplitude}
{separate Änderung des elektrischen und magnetischen Feldwellenanteils}
{\textbf{\textcolor{DARCgreen}{Änderung der Amplitude und der Phase}}}
\end{QQuestion}

}
\end{frame}%ENDCONTENT


\section{Orthogonales Frequenzmultiplexverfahren (OFDM)}
\label{section:ofdm}
\begin{frame}%STARTCONTENT
\begin{itemize}
  \item Es ist auch möglich, einen Datenstrom auf mehrere Träger zu verteilen, die auf unterschiedlichen, jedoch nahegelegenen Frequenzen liegen.
  \item Bei der orthogonalen Frequenzmodulation (Orthogonal Frequency-Division Multiplexing, OFDM) werden die einzelnen Träger in einem Abstand platziert, wo ein gegenseitiges Stören untereinander (ein sogenanntes „Übersprechen“) vermieden wird.
  \end{itemize}

\begin{figure}
    \DARCimage{0.85\linewidth}{704include}
    \caption{\scriptsize Frequenzspektrum eines einfachen OFDM-Signals}
    \label{ofdm}
\end{figure}

\end{frame}

\begin{frame}\begin{itemize}
  \item Ein Vorteil dieses Vorgehens liegt darin, dass schmalbandige Störungen nur einen oder wenige Träger stören.
  \item Im Zusammenspiel mit Fehlerkorrekturverfahren mit redundanter Datenübertragung, die wir später kennenlernen werden, ist es so möglich, trotz schmalbandiger Störungen eine fehlerfreie Übertragung zu erreichen.
  \end{itemize}
\end{frame}

\begin{frame}
\only<1>{
\begin{QQuestion}{AE421}{Orthogonale Frequenzmultiplexverfahren (OFDM) mit redundanter Übertragung sind besonders unempfindlich gegen~...}{schmalbandige Störungen, da es einen Träger mit hoher Bandbreite verwendet.}
{schmalbandige Störungen, da das Gesamtsignal aus mehreren Einzelträgern besteht.}
{breitbandige Störungen, da das Gesamtsignal aus mehreren Einzelträgern besteht.}
{breitbandige Störungen, da es einen Träger mit hoher Bandbreite verwendet.}
\end{QQuestion}

}
\only<2>{
\begin{QQuestion}{AE421}{Orthogonale Frequenzmultiplexverfahren (OFDM) mit redundanter Übertragung sind besonders unempfindlich gegen~...}{schmalbandige Störungen, da es einen Träger mit hoher Bandbreite verwendet.}
{\textbf{\textcolor{DARCgreen}{schmalbandige Störungen, da das Gesamtsignal aus mehreren Einzelträgern besteht.}}}
{breitbandige Störungen, da das Gesamtsignal aus mehreren Einzelträgern besteht.}
{breitbandige Störungen, da es einen Träger mit hoher Bandbreite verwendet.}
\end{QQuestion}

}
\end{frame}

\begin{frame}\begin{itemize}
  \item Ein weiterer Vorteil ergibt sich aus der geringeren Symbolrate jedes einzelnen Trägers.
  \item Durch die geringere Symbolrate ist die Dauer eines jeden Symbols länger.
  \item Im Falle zeitlicher Verschiebungen aufgrund von Mehrwegeausbreitung ist der Anteil der Überlagerung zwischen den Signalen entsprechend geringer.
  \end{itemize}
\end{frame}

\begin{frame}
\only<1>{
\begin{QQuestion}{AE422}{Bei welcher Art von Kanalstörung sind Orthogonale Frequenzmultiplexverfahren (OFDM) mit redundanter Übertragung besonders vorteilhaft?}{Überreichweiten anderer OFDM-Sender}
{Impulse durch Gewitter}
{Breitbandiges Rauschen}
{Mehrwegeausbreitung}
\end{QQuestion}

}
\only<2>{
\begin{QQuestion}{AE422}{Bei welcher Art von Kanalstörung sind Orthogonale Frequenzmultiplexverfahren (OFDM) mit redundanter Übertragung besonders vorteilhaft?}{Überreichweiten anderer OFDM-Sender}
{Impulse durch Gewitter}
{Breitbandiges Rauschen}
{\textbf{\textcolor{DARCgreen}{Mehrwegeausbreitung}}}
\end{QQuestion}

}
\end{frame}%ENDCONTENT


\section{Shannon-Hartley-Gesetz}
\label{section:shannon_hartley_gesetzt}
\begin{frame}%STARTCONTENT
\begin{itemize}
  \item Welche Datenübertragungsrate erreichbar ist, hängt von der nutzbaren Bandbreite und dem Signal-Rauschverhältnis ab.
  \item Aus diesen beiden Größen kann mit dem Shannon-Hartley-Gesetz die theoretisch maximal erreichbare Datenübertragungsrate für einen Übertragungskanal berechnet werden.
  \item Ein leicht zu merkender Wert stellt sich bei einem Signal-Rausch-Verhältnis von 0 dB ein.
  \item Hier entspricht die Bandbreite in Hertz genau der maximal erreichbaren Datenrate in Bit/s.
  \end{itemize}
\end{frame}

\begin{frame}
\only<1>{
\begin{QQuestion}{AE416}{Welche Aussage trifft auf das Shannon-Hartley-Gesetz zu? Das Gesetz~...}{bestimmt die maximale Bandbreite, die durch eine Übertragung mit einer bestimmten Datenübertragungsrate theoretisch belegt werden kann.}
{besagt, dass unabhängig von der Art der vorherrschenden Störungen eines Übertragungskanals theoretisch eine unbegrenzte Datenübertragungsrate erzielt werden kann.}
{bestimmt für einen Übertragungskanal gegebener Bandbreite die höchste theoretisch erzielbare Datenübertragungsrate in Abhängigkeit vom Signal-Rausch-Verhältnis.}
{besagt, dass theoretisch eine unendliche Abtastrate erforderlich ist, um ein bandbegrenztes Signal fehlerfrei zu rekonstruieren.}
\end{QQuestion}

}
\only<2>{
\begin{QQuestion}{AE416}{Welche Aussage trifft auf das Shannon-Hartley-Gesetz zu? Das Gesetz~...}{bestimmt die maximale Bandbreite, die durch eine Übertragung mit einer bestimmten Datenübertragungsrate theoretisch belegt werden kann.}
{besagt, dass unabhängig von der Art der vorherrschenden Störungen eines Übertragungskanals theoretisch eine unbegrenzte Datenübertragungsrate erzielt werden kann.}
{\textbf{\textcolor{DARCgreen}{bestimmt für einen Übertragungskanal gegebener Bandbreite die höchste theoretisch erzielbare Datenübertragungsrate in Abhängigkeit vom Signal-Rausch-Verhältnis.}}}
{besagt, dass theoretisch eine unendliche Abtastrate erforderlich ist, um ein bandbegrenztes Signal fehlerfrei zu rekonstruieren.}
\end{QQuestion}

}
\end{frame}

\begin{frame}\begin{itemize}
  \item Schlechtere Signal-Rausch-Verhältnisse ermöglichen entsprechend weniger Datenrate, bessere Signal-Rausch-Verhältnisse größere Datenraten.
  \item Da die Rechnungen dazu recht komplex sind, wurden die Prüfungsfragen so gestaltet, dass man das Ergebnis leicht abschätzen kann.
  \item Im Folgenden gibt es Beispiele mit 0 db, -20 db und (+)30 db.
  \end{itemize}
\end{frame}

\begin{frame}Beispiel 1:

\begin{itemize}
  \item Ein Übertragungskanal mit einer Bandbreite von 2,7 kHz wird durch additives weißes Gaußsches Rauschen (AWGN) gestört. * Das Signal-Rausch-Verhältnis (SNR) beträgt 0 dB.
  \item Welche Bitrate kann nach dem Shannon-Hartley-Gesetz etwa maximal fehlerfrei übertragen werden?
  \end{itemize}
Durch ein SNR von 0db entspricht die Bandbreite in Hertz genau der maximal erreichbaren Datenrate in Bit/s, also 2,7 kbit/s.

\end{frame}

\begin{frame}
\only<1>{
\begin{QQuestion}{AE417}{Ein Übertragungskanal mit einer Bandbreite von \qty{2,7}{\kHz} wird durch additives weißes Gaußsches Rauschen (AWGN) gestört. Das Signal-Rausch-Verhältnis (SNR) beträgt \qty{0}{\decibel}. Welche Bitrate kann nach dem Shannon-Hartley-Gesetz etwa maximal fehlerfrei übertragen werden?}{ca.~\qty{39}{\bit}/s}
{\qty{0}{\bit}/s (Übertragung nicht möglich)}
{ca.~\qty{2,7}{\bit}/s}
{ca.~\qty[per-mode=symbol]{2,7}{\kilo\bit\per\s}}
\end{QQuestion}

}
\only<2>{
\begin{QQuestion}{AE417}{Ein Übertragungskanal mit einer Bandbreite von \qty{2,7}{\kHz} wird durch additives weißes Gaußsches Rauschen (AWGN) gestört. Das Signal-Rausch-Verhältnis (SNR) beträgt \qty{0}{\decibel}. Welche Bitrate kann nach dem Shannon-Hartley-Gesetz etwa maximal fehlerfrei übertragen werden?}{ca.~\qty{39}{\bit}/s}
{\qty{0}{\bit}/s (Übertragung nicht möglich)}
{ca.~\qty{2,7}{\bit}/s}
{\textbf{\textcolor{DARCgreen}{ca.~\qty[per-mode=symbol]{2,7}{\kilo\bit\per\s}}}}
\end{QQuestion}

}
\end{frame}

\begin{frame}
\only<1>{
\begin{QQuestion}{AE418}{Ein Übertragungskanal mit einer Bandbreite von \qty{10}{\MHz} wird durch additives weißes Gaußsches Rauschen (AWGN) gestört. Das Signal-Rausch-Verhältnis (SNR) beträgt \qty{0}{\decibel}. Welche Bitrate kann nach dem Shannon-Hartley-Gesetz etwa maximal fehlerfrei übertragen werden?}{ca.~\qty[per-mode=symbol]{10}{\mega\bit\per\s}}
{ca.~\qty[per-mode=symbol]{7}{\mega\bit\per\s}}
{ca.~\qty[per-mode=symbol]{8}{\mega\bit\per\s}}
{ca.~\qty[per-mode=symbol]{100}{\mega\bit\per\s}}
\end{QQuestion}

}
\only<2>{
\begin{QQuestion}{AE418}{Ein Übertragungskanal mit einer Bandbreite von \qty{10}{\MHz} wird durch additives weißes Gaußsches Rauschen (AWGN) gestört. Das Signal-Rausch-Verhältnis (SNR) beträgt \qty{0}{\decibel}. Welche Bitrate kann nach dem Shannon-Hartley-Gesetz etwa maximal fehlerfrei übertragen werden?}{\textbf{\textcolor{DARCgreen}{ca.~\qty[per-mode=symbol]{10}{\mega\bit\per\s}}}}
{ca.~\qty[per-mode=symbol]{7}{\mega\bit\per\s}}
{ca.~\qty[per-mode=symbol]{8}{\mega\bit\per\s}}
{ca.~\qty[per-mode=symbol]{100}{\mega\bit\per\s}}
\end{QQuestion}

}
\end{frame}

\begin{frame}Beispiel 2:

\begin{itemize}
  \item Ein Übertragungskanal mit einer Bandbreite von 2,7 kHz wird durch additives weißes Gaußsches Rauschen (AWGN) gestört. * Das Signal-Rausch-Verhältnis (SNR) beträgt -20 dB.
  \item Welche Bitrate kann nach dem Shannon-Hartley-Gesetz etwa maximal fehlerfrei übertragen werden?
  \end{itemize}
Durch ein SNR von -20db muss die maximal erreichbare Datenrate kleiner als 2,7 kbit/s sein. Es kann nur \qty{39}{\bit}/s richtig sein.

\end{frame}

\begin{frame}
\only<1>{
\begin{QQuestion}{AE420}{Ein Übertragungskanal mit einer Bandbreite von \qty{2,7}{\kHz} wird durch additives weißes Gaußsches Rauschen (AWGN) gestört. Das Signal-Rausch-Verhältnis (SNR) beträgt \qty{-20}{\decibel}. Welche Bitrate kann nach dem Shannon-Hartley-Gesetz etwa maximal fehlerfrei übertragen werden?}{ca.~\qty{39}{\bit}/s}
{\qty{0}{\bit}/s (Übertragung nicht möglich)}
{ca.~\qty[per-mode=symbol]{2,7}{\kilo\bit\per\s}}
{ca.~\qty[per-mode=symbol]{5,4}{\kilo\bit\per\s}}
\end{QQuestion}

}
\only<2>{
\begin{QQuestion}{AE420}{Ein Übertragungskanal mit einer Bandbreite von \qty{2,7}{\kHz} wird durch additives weißes Gaußsches Rauschen (AWGN) gestört. Das Signal-Rausch-Verhältnis (SNR) beträgt \qty{-20}{\decibel}. Welche Bitrate kann nach dem Shannon-Hartley-Gesetz etwa maximal fehlerfrei übertragen werden?}{\textbf{\textcolor{DARCgreen}{ca.~\qty{39}{\bit}/s}}}
{\qty{0}{\bit}/s (Übertragung nicht möglich)}
{ca.~\qty[per-mode=symbol]{2,7}{\kilo\bit\per\s}}
{ca.~\qty[per-mode=symbol]{5,4}{\kilo\bit\per\s}}
\end{QQuestion}

}
\end{frame}

\begin{frame}Beispiel 3:

\begin{itemize}
  \item Ein Übertragungskanal mit einer Bandbreite von 10 MHz wird durch additives weißes Gaußsches Rauschen (AWGN) gestört. * Das Signal-Rausch-Verhältnis (SNR) beträgt 30 dB.
  \item Welche Bitrate kann nach dem Shannon-Hartley-Gesetz etwa maximal fehlerfrei übertragen werden?
  \end{itemize}
Durch ein SNR von 30db muss die maximal erreichbare Datenrate größer 10 Mbit/s sein. Es kann nur 100 Mbit/s richtig sein.

\end{frame}

\begin{frame}
\only<1>{
\begin{QQuestion}{AE419}{Ein Übertragungskanal mit einer Bandbreite von \qty{10}{\MHz} wird durch additives weißes Gaußsches Rauschen (AWGN) gestört. Das Signal-Rausch-Verhältnis (SNR) beträgt \qty{30}{\decibel}. Welche Bitrate kann nach dem Shannon-Hartley-Gesetz etwa maximal fehlerfrei übertragen werden?}{ca.~\qty[per-mode=symbol]{100}{\mega\bit\per\s}}
{ca.~\qty[per-mode=symbol]{10}{\mega\bit\per\s}}
{ca.~\qty[per-mode=symbol]{7}{\mega\bit\per\s}}
{ca.~\qty[per-mode=symbol]{8}{\mega\bit\per\s}}
\end{QQuestion}

}
\only<2>{
\begin{QQuestion}{AE419}{Ein Übertragungskanal mit einer Bandbreite von \qty{10}{\MHz} wird durch additives weißes Gaußsches Rauschen (AWGN) gestört. Das Signal-Rausch-Verhältnis (SNR) beträgt \qty{30}{\decibel}. Welche Bitrate kann nach dem Shannon-Hartley-Gesetz etwa maximal fehlerfrei übertragen werden?}{\textbf{\textcolor{DARCgreen}{ca.~\qty[per-mode=symbol]{100}{\mega\bit\per\s}}}}
{ca.~\qty[per-mode=symbol]{10}{\mega\bit\per\s}}
{ca.~\qty[per-mode=symbol]{7}{\mega\bit\per\s}}
{ca.~\qty[per-mode=symbol]{8}{\mega\bit\per\s}}
\end{QQuestion}

}
\end{frame}

\begin{frame}\end{frame}%ENDCONTENT


\section{Quellencodierung}
\label{section:quellencodierung}
\begin{frame}%STARTCONTENT
\begin{itemize}
  \item Bei der digitalen Übertragung möchte man das Frequenzspektrum möglichst effizient nutzen.
  \item Dies erreicht man durch die Kompression der Nutzdaten, die sogenannte Quellencodierung.
  \item Dabei werden Redundanzen (z. B. Wiederholungen) oder Irrelevanzen (weniger wichtige Informationsteile) aus dem Datenstrom entfernt.
  \end{itemize}

\begin{figure}
    \DARCimage{0.85\linewidth}{675include}
    \caption{\scriptsize Quellencodierer}
    \label{quellencodierer}
\end{figure}

\end{frame}

\begin{frame}
\only<1>{
\begin{QQuestion}{AE408}{Wodurch kann die Datenmenge einer zu übertragenden Nachricht reduziert werden?}{Kanalcodierung}
{Quellencodierung}
{Synchronisation}
{Mehrfachzugriff}
\end{QQuestion}

}
\only<2>{
\begin{QQuestion}{AE408}{Wodurch kann die Datenmenge einer zu übertragenden Nachricht reduziert werden?}{Kanalcodierung}
{\textbf{\textcolor{DARCgreen}{Quellencodierung}}}
{Synchronisation}
{Mehrfachzugriff}
\end{QQuestion}

}
\end{frame}%ENDCONTENT


\section{Kanalcodierung}
\label{section:kanalcodierung}
\begin{frame}%STARTCONTENT
\begin{itemize}
  \item Die Abbildung zeigt einen Sender und einen Empfänger, welche über einen Kanal miteinander verbunden sind.
  \item Durch atmosphärische Einflüsse oder Aussendungen anderer Stationen kann es zu Störungen auf dem Kanal kommen, welche zu Fehlern bei der Übertragung führen.
  \end{itemize}

\begin{figure}
    \DARCimage{0.85\linewidth}{674include}
    \caption{\scriptsize Kanal}
    \label{kanal}
\end{figure}

\end{frame}

\begin{frame}Die Kanalcodierung fügt der zu übertragenden Information gezielt Redundanz hinzu, beispielsweise Wiederholungen oder Prüfsummen.

\end{frame}

\begin{frame}Wir unterscheiden zwei Arten der Kanalcodierung:

\begin{itemize}
  \item Fehlererkennung: Man kann erkennen, dass bei der Übertragung ein Fehler aufgetreten ist, und dann z. B. eine erneute Übertragung anfordern.
  \item Vorwärtsfehlerkorrektur: Fehler, die bei der Übertragung entstehen, werden mit Hilfe der Redundanz beim Empfänger korrigiert.
  \end{itemize}

\begin{figure}
    \DARCimage{0.85\linewidth}{676include}
    \caption{\scriptsize Kanalcodierer}
    \label{kanalcodierer}
\end{figure}

\end{frame}

\begin{frame}
\only<1>{
\begin{QQuestion}{AE409}{Was wird unter Kanalcodierung verstanden?}{Verschlüsselung des Kanals zum Schutz gegen unbefugtes Abhören}
{Kompression von Daten vor der Übertragung zur Reduktion der Datenmenge}
{Hinzufügen von Redundanz vor der Übertragung zum Schutz vor Übertragungsfehlern}
{Zuordnung von Frequenzen zu Sende- bzw. Empfangskanälen zur häufigen Verwendung}
\end{QQuestion}

}
\only<2>{
\begin{QQuestion}{AE409}{Was wird unter Kanalcodierung verstanden?}{Verschlüsselung des Kanals zum Schutz gegen unbefugtes Abhören}
{Kompression von Daten vor der Übertragung zur Reduktion der Datenmenge}
{\textbf{\textcolor{DARCgreen}{Hinzufügen von Redundanz vor der Übertragung zum Schutz vor Übertragungsfehlern}}}
{Zuordnung von Frequenzen zu Sende- bzw. Empfangskanälen zur häufigen Verwendung}
\end{QQuestion}

}
\end{frame}%ENDCONTENT


\section{Fehlererkennung}
\label{section:fehlererkennung}
\begin{frame}%STARTCONTENT

\only<1>{
\begin{QQuestion}{AE411}{Eine digitale Übertragung wird durch ein einzelnes Prüfbit (Parity Bit) abgesichert. Der Empfänger stellt bei der Paritätsprüfung einen Übertragungsfehler fest. Wie viele Bits einschließlich des Prüfbits wurden fehlerhaft übertragen?}{Eine gerade Anzahl Bits}
{Eine ungerade Anzahl Bits}
{Mindestens zwei Bits}
{Maximal zwei Bits}
\end{QQuestion}

}
\only<2>{
\begin{QQuestion}{AE411}{Eine digitale Übertragung wird durch ein einzelnes Prüfbit (Parity Bit) abgesichert. Der Empfänger stellt bei der Paritätsprüfung einen Übertragungsfehler fest. Wie viele Bits einschließlich des Prüfbits wurden fehlerhaft übertragen?}{Eine gerade Anzahl Bits}
{\textbf{\textcolor{DARCgreen}{Eine ungerade Anzahl Bits}}}
{Mindestens zwei Bits}
{Maximal zwei Bits}
\end{QQuestion}

}
\end{frame}

\begin{frame}
\only<1>{
\begin{QQuestion}{AE412}{Eine digitale Übertragung wird durch ein einzelnes Prüfbit (Parity Bit) abgesichert. Der Empfänger stellt bei der Paritätsprüfung \underline{keinen} Übertragungsfehler fest. Was sagt dies über die Fehlerfreiheit der übertragenen Nutzdaten und des Prüfbits aus?}{Die Übertragung war fehlerfrei oder es ist eine gerade Anzahl an Bitfehlern aufgetreten.}
{Die Übertragung war fehlerfrei oder es ist eine ungerade Anzahl an Bitfehlern aufgetreten.}
{Die Übertragung war fehlerfrei.}
{Die Nutzdaten wurden fehlerfrei, das Prüfbit jedoch fehlerhaft übertragen.}
\end{QQuestion}

}
\only<2>{
\begin{QQuestion}{AE412}{Eine digitale Übertragung wird durch ein einzelnes Prüfbit (Parity Bit) abgesichert. Der Empfänger stellt bei der Paritätsprüfung \underline{keinen} Übertragungsfehler fest. Was sagt dies über die Fehlerfreiheit der übertragenen Nutzdaten und des Prüfbits aus?}{\textbf{\textcolor{DARCgreen}{Die Übertragung war fehlerfrei oder es ist eine gerade Anzahl an Bitfehlern aufgetreten.}}}
{Die Übertragung war fehlerfrei oder es ist eine ungerade Anzahl an Bitfehlern aufgetreten.}
{Die Übertragung war fehlerfrei.}
{Die Nutzdaten wurden fehlerfrei, das Prüfbit jedoch fehlerhaft übertragen.}
\end{QQuestion}

}
\end{frame}

\begin{frame}
\only<1>{
\begin{QQuestion}{AE410}{Was wird unter zyklischer Redundanzprüfung (CRC) verstanden?}{Wiederholte (zyklisch redundante) Prüfung der Amateurfunkanlage auf Fehler. }
{Die fortlaufende Prüfung eines zu übertragenden Datenstroms auf Redundanz.}
{Umlaufende (zyklische) Überwachung einer Frequenz durch mehrere Stationen.}
{Ein Prüfsummenverfahren zur Fehlererkennung in Datenblöcken variabler Länge.}
\end{QQuestion}

}
\only<2>{
\begin{QQuestion}{AE410}{Was wird unter zyklischer Redundanzprüfung (CRC) verstanden?}{Wiederholte (zyklisch redundante) Prüfung der Amateurfunkanlage auf Fehler. }
{Die fortlaufende Prüfung eines zu übertragenden Datenstroms auf Redundanz.}
{Umlaufende (zyklische) Überwachung einer Frequenz durch mehrere Stationen.}
{\textbf{\textcolor{DARCgreen}{Ein Prüfsummenverfahren zur Fehlererkennung in Datenblöcken variabler Länge.}}}
\end{QQuestion}

}
\end{frame}%ENDCONTENT


\section{Fehlerkorrektur}
\label{section:fehlerkorrektur}
\begin{frame}%STARTCONTENT

\only<1>{
\begin{QQuestion}{AE413}{Sie verwenden ein Datenübertragungsverfahren \underline{ohne} Vorwärtsfehlerkorrektur. Wodurch können Datenpakete trotz Prüfsummenfehlern korrigiert werden?}{I/Q-Verfahren}
{Wiederholte Prüfung}
{Duplizieren der Prüfsumme}
{Erneute Übertragung}
\end{QQuestion}

}
\only<2>{
\begin{QQuestion}{AE413}{Sie verwenden ein Datenübertragungsverfahren \underline{ohne} Vorwärtsfehlerkorrektur. Wodurch können Datenpakete trotz Prüfsummenfehlern korrigiert werden?}{I/Q-Verfahren}
{Wiederholte Prüfung}
{Duplizieren der Prüfsumme}
{\textbf{\textcolor{DARCgreen}{Erneute Übertragung}}}
\end{QQuestion}

}
\end{frame}

\begin{frame}
\only<1>{
\begin{QQuestion}{AE414}{Was ist die Voraussetzung für Vorwärtsfehlerkorrektur (FEC)?}{Automatische Anpassung der Sendeleistung}
{Kompression vor der Übertragung}
{Erneute Übertragung fehlerhafter Daten}
{Übertragung redundanter Informationen}
\end{QQuestion}

}
\only<2>{
\begin{QQuestion}{AE414}{Was ist die Voraussetzung für Vorwärtsfehlerkorrektur (FEC)?}{Automatische Anpassung der Sendeleistung}
{Kompression vor der Übertragung}
{Erneute Übertragung fehlerhafter Daten}
{\textbf{\textcolor{DARCgreen}{Übertragung redundanter Informationen}}}
\end{QQuestion}

}
\end{frame}%ENDCONTENT


\section{Mapping}
\label{section:mapping}
\begin{frame}%STARTCONTENT
\end{frame}%ENDCONTENT


\section{Sende- und Empfangsketten}
\label{section:sende_empfangsketten}
\begin{frame}%STARTCONTENT

\only<1>{
\begin{PQuestion}{AF626}{Welcher der nachfolgenden Blöcke vervollständigt den dargestellten, stark vereinfachten Sendezweig eines Funkgeräts für digitalen Sprechfunk korrekt?}{\DARCimage{1.0\linewidth}{645include}}
{\DARCimage{1.0\linewidth}{643include}}
{\DARCimage{1.0\linewidth}{644include}}
{\DARCimage{1.0\linewidth}{642include}}
{\DARCimage{1.0\linewidth}{641include}}\end{PQuestion}

}
\only<2>{
\begin{PQuestion}{AF626}{Welcher der nachfolgenden Blöcke vervollständigt den dargestellten, stark vereinfachten Sendezweig eines Funkgeräts für digitalen Sprechfunk korrekt?}{\DARCimage{1.0\linewidth}{645include}}
{\DARCimage{1.0\linewidth}{643include}}
{\DARCimage{1.0\linewidth}{644include}}
{\textbf{\textcolor{DARCgreen}{\DARCimage{1.0\linewidth}{642include}}}}
{\DARCimage{1.0\linewidth}{641include}}\end{PQuestion}

}
\end{frame}

\begin{frame}
\only<1>{
\begin{PQuestion}{AF628}{Welcher der nachfolgenden Blöcke vervollständigt den dargestellten, stark vereinfachten Empfangszweig eines Funkgeräts für digitalen Sprechfunk korrekt?}{\DARCimage{1.0\linewidth}{639include}}
{\DARCimage{1.0\linewidth}{637include}}
{\DARCimage{1.0\linewidth}{638include}}
{\DARCimage{1.0\linewidth}{636include}}
{\DARCimage{1.0\linewidth}{635include}}\end{PQuestion}

}
\only<2>{
\begin{PQuestion}{AF628}{Welcher der nachfolgenden Blöcke vervollständigt den dargestellten, stark vereinfachten Empfangszweig eines Funkgeräts für digitalen Sprechfunk korrekt?}{\DARCimage{1.0\linewidth}{639include}}
{\DARCimage{1.0\linewidth}{637include}}
{\DARCimage{1.0\linewidth}{638include}}
{\textbf{\textcolor{DARCgreen}{\DARCimage{1.0\linewidth}{636include}}}}
{\DARCimage{1.0\linewidth}{635include}}\end{PQuestion}

}
\end{frame}

\begin{frame}
\only<1>{
\begin{PQuestion}{AF629}{Welcher der nachfolgenden Blöcke vervollständigt den dargestellten, stark vereinfachten Empfangszweig für digitales Amateurfunkfernsehen (DATV) korrekt?}{\DARCimage{1.0\linewidth}{637include}}
{\DARCimage{1.0\linewidth}{636include}}
{\DARCimage{1.0\linewidth}{638include}}
{\DARCimage{1.0\linewidth}{639include}}
{\DARCimage{1.0\linewidth}{652include}}\end{PQuestion}

}
\only<2>{
\begin{PQuestion}{AF629}{Welcher der nachfolgenden Blöcke vervollständigt den dargestellten, stark vereinfachten Empfangszweig für digitales Amateurfunkfernsehen (DATV) korrekt?}{\DARCimage{1.0\linewidth}{637include}}
{\textbf{\textcolor{DARCgreen}{\DARCimage{1.0\linewidth}{636include}}}}
{\DARCimage{1.0\linewidth}{638include}}
{\DARCimage{1.0\linewidth}{639include}}
{\DARCimage{1.0\linewidth}{652include}}\end{PQuestion}

}
\end{frame}

\begin{frame}
\only<1>{
\begin{PQuestion}{AF627}{Welcher der nachfolgenden Blöcke vervollständigt den dargestellten, stark vereinfachten Sendezweig für digitales Amateurfunkfernsehen (DATV) korrekt?}{\DARCimage{1.0\linewidth}{642include}}
{\DARCimage{1.0\linewidth}{643include}}
{\DARCimage{1.0\linewidth}{644include}}
{\DARCimage{1.0\linewidth}{645include}}
{\DARCimage{1.0\linewidth}{653include}}\end{PQuestion}

}
\only<2>{
\begin{PQuestion}{AF627}{Welcher der nachfolgenden Blöcke vervollständigt den dargestellten, stark vereinfachten Sendezweig für digitales Amateurfunkfernsehen (DATV) korrekt?}{\textbf{\textcolor{DARCgreen}{\DARCimage{1.0\linewidth}{642include}}}}
{\DARCimage{1.0\linewidth}{643include}}
{\DARCimage{1.0\linewidth}{644include}}
{\DARCimage{1.0\linewidth}{645include}}
{\DARCimage{1.0\linewidth}{653include}}\end{PQuestion}

}
\end{frame}%ENDCONTENT


\section{Synchronization}
\label{section:synchronisation}
\begin{frame}%STARTCONTENT

\only<1>{
\begin{QQuestion}{AE407}{Was versteht man bei der Übertragung von Daten unter Synchronisation?}{Asynchrone Frequenzwechsel, bei denen der Empfänger den Sender sucht.}
{Automatischer Abgleich von Datenbeständen von zwei oder mehr Stationen.}
{Herstellung der zeitlichen Übereinstimmung zwischen Sender und Empfänger.}
{Anpassung der Sendeleistung synchron zu den Ausbreitungsbedingungen.}
\end{QQuestion}

}
\only<2>{
\begin{QQuestion}{AE407}{Was versteht man bei der Übertragung von Daten unter Synchronisation?}{Asynchrone Frequenzwechsel, bei denen der Empfänger den Sender sucht.}
{Automatischer Abgleich von Datenbeständen von zwei oder mehr Stationen.}
{\textbf{\textcolor{DARCgreen}{Herstellung der zeitlichen Übereinstimmung zwischen Sender und Empfänger.}}}
{Anpassung der Sendeleistung synchron zu den Ausbreitungsbedingungen.}
\end{QQuestion}

}
\end{frame}%ENDCONTENT


\title{DARC Amateurfunklehrgang Klasse A}
\author{Digitale Signalverarbeitung}
\institute{Deutscher Amateur Radio Club e.\,V.}
\begin{frame}
\maketitle
\end{frame}

\section{Sampling und Quantisierung}
\label{section:sampling_quantisierung}
\begin{frame}%STARTCONTENT
\end{frame}

\begin{frame}
\only<1>{
\begin{question2x2}{AF601}{Welche der folgenden Abbildungen symbolisiert ein zeitkontinuierliches und wertkontinuierliches Signal am besten?}{\DARCimage{1.0\linewidth}{411include}}
{\DARCimage{1.0\linewidth}{409include}}
{\DARCimage{1.0\linewidth}{410include}}
{\DARCimage{1.0\linewidth}{408include}}
\end{question2x2}

}
\only<2>{
\begin{question2x2}{AF601}{Welche der folgenden Abbildungen symbolisiert ein zeitkontinuierliches und wertkontinuierliches Signal am besten?}{\DARCimage{1.0\linewidth}{411include}}
{\DARCimage{1.0\linewidth}{409include}}
{\DARCimage{1.0\linewidth}{410include}}
{\textbf{\textcolor{DARCgreen}{\DARCimage{1.0\linewidth}{408include}}}}
\end{question2x2}

}
\end{frame}

\begin{frame}
\only<1>{
\begin{question2x2}{AF603}{Welche der folgenden Abbildungen symbolisiert ein zeitdiskretes und wertkontinuierliches Signal am besten?}{\DARCimage{1.0\linewidth}{408include}}
{\DARCimage{1.0\linewidth}{409include}}
{\DARCimage{1.0\linewidth}{410include}}
{\DARCimage{1.0\linewidth}{411include}}
\end{question2x2}

}
\only<2>{
\begin{question2x2}{AF603}{Welche der folgenden Abbildungen symbolisiert ein zeitdiskretes und wertkontinuierliches Signal am besten?}{\DARCimage{1.0\linewidth}{408include}}
{\textbf{\textcolor{DARCgreen}{\DARCimage{1.0\linewidth}{409include}}}}
{\DARCimage{1.0\linewidth}{410include}}
{\DARCimage{1.0\linewidth}{411include}}
\end{question2x2}

}
\end{frame}

\begin{frame}
\only<1>{
\begin{question2x2}{AF602}{Welche der folgenden Abbildungen symbolisiert ein zeitkontinuierliches und wertdiskretes Signal am besten?}{\DARCimage{1.0\linewidth}{409include}}
{\DARCimage{1.0\linewidth}{408include}}
{\DARCimage{1.0\linewidth}{410include}}
{\DARCimage{1.0\linewidth}{411include}}
\end{question2x2}

}
\only<2>{
\begin{question2x2}{AF602}{Welche der folgenden Abbildungen symbolisiert ein zeitkontinuierliches und wertdiskretes Signal am besten?}{\DARCimage{1.0\linewidth}{409include}}
{\DARCimage{1.0\linewidth}{408include}}
{\textbf{\textcolor{DARCgreen}{\DARCimage{1.0\linewidth}{410include}}}}
{\DARCimage{1.0\linewidth}{411include}}
\end{question2x2}

}
\end{frame}

\begin{frame}
\only<1>{
\begin{question2x2}{AF604}{Welche der folgenden Abbildungen symbolisiert ein zeitdiskretes und wertdiskretes Signal am besten?}{\DARCimage{1.0\linewidth}{411include}}
{\DARCimage{1.0\linewidth}{408include}}
{\DARCimage{1.0\linewidth}{409include}}
{\DARCimage{1.0\linewidth}{410include}}
\end{question2x2}

}
\only<2>{
\begin{question2x2}{AF604}{Welche der folgenden Abbildungen symbolisiert ein zeitdiskretes und wertdiskretes Signal am besten?}{\textbf{\textcolor{DARCgreen}{\DARCimage{1.0\linewidth}{411include}}}}
{\DARCimage{1.0\linewidth}{408include}}
{\DARCimage{1.0\linewidth}{409include}}
{\DARCimage{1.0\linewidth}{410include}}
\end{question2x2}

}
\end{frame}%ENDCONTENT


\section{Sampling}
\label{section:sampling}
\begin{frame}%STARTCONTENT

\only<1>{
\begin{QQuestion}{AF606}{Wie wird die Umwandlung eines zeitkontinuierlichen in ein zeitdiskretes Signal bezeichnet?}{Quantisierung}
{Sampling}
{Codierung}
{Zeitmultiplexing}
\end{QQuestion}

}
\only<2>{
\begin{QQuestion}{AF606}{Wie wird die Umwandlung eines zeitkontinuierlichen in ein zeitdiskretes Signal bezeichnet?}{Quantisierung}
{\textbf{\textcolor{DARCgreen}{Sampling}}}
{Codierung}
{Zeitmultiplexing}
\end{QQuestion}

}
\end{frame}

\begin{frame}
\only<1>{
\begin{QQuestion}{AF615}{Wie ist die Abtastrate (Samplingrate) eines A/D-Umsetzers definiert?}{Abtastungen mal Zeit}
{Abtastungen je Zeiteinheit}
{Abtastungen je Hertz}
{Abtastungen mal Samples}
\end{QQuestion}

}
\only<2>{
\begin{QQuestion}{AF615}{Wie ist die Abtastrate (Samplingrate) eines A/D-Umsetzers definiert?}{Abtastungen mal Zeit}
{\textbf{\textcolor{DARCgreen}{Abtastungen je Zeiteinheit}}}
{Abtastungen je Hertz}
{Abtastungen mal Samples}
\end{QQuestion}

}
\end{frame}%ENDCONTENT


\section{Abtasttheorem}
\label{section:abtasttheorem}
\begin{frame}%STARTCONTENT

\only<1>{
\begin{QQuestion}{AF616}{Welche Aussage trifft auf das Abtasttheorem zu? Das Theorem~...}{bestimmt die für eine fehlerfreie Rekonstruktion eines Signals theoretisch notwendige minimale Abtastrate.}
{besagt, dass theoretisch eine unendliche Abtastrate erforderlich ist, um ein bandbegrenztes Signal fehlerfrei zu rekonstruieren.}
{bestimmt die maximale Bandbreite, die durch eine Übertragung mit einer bestimmten Datenübertragungsrate theoretisch belegt werden kann.}
{besagt, dass unabhängig von der Art der vorherrschenden Störungen eines Übertragungskanals theoretisch eine unbegrenzte Datenübertragungsrate erzielt werden kann.}
\end{QQuestion}

}
\only<2>{
\begin{QQuestion}{AF616}{Welche Aussage trifft auf das Abtasttheorem zu? Das Theorem~...}{\textbf{\textcolor{DARCgreen}{bestimmt die für eine fehlerfreie Rekonstruktion eines Signals theoretisch notwendige minimale Abtastrate.}}}
{besagt, dass theoretisch eine unendliche Abtastrate erforderlich ist, um ein bandbegrenztes Signal fehlerfrei zu rekonstruieren.}
{bestimmt die maximale Bandbreite, die durch eine Übertragung mit einer bestimmten Datenübertragungsrate theoretisch belegt werden kann.}
{besagt, dass unabhängig von der Art der vorherrschenden Störungen eines Übertragungskanals theoretisch eine unbegrenzte Datenübertragungsrate erzielt werden kann.}
\end{QQuestion}

}
\end{frame}

\begin{frame}
\only<1>{
\begin{QQuestion}{AF618}{Ein analoges Signal mit einer Bandbreite von $f_{\symup{max}}$ soll digital verarbeitet werden. Welche der folgenden Abtastraten ist die kleinste, die Alias-Effekte vermeidet?}{knapp über $2 \cdot f_{\symup{max}}$}
{knapp über $f_{\symup{max}}$}
{knapp unter $\dfrac{f_{\mathrm{max}}}{2}$}
{knapp unter $f_{\symup{max}}$}
\end{QQuestion}

}
\only<2>{
\begin{QQuestion}{AF618}{Ein analoges Signal mit einer Bandbreite von $f_{\symup{max}}$ soll digital verarbeitet werden. Welche der folgenden Abtastraten ist die kleinste, die Alias-Effekte vermeidet?}{\textbf{\textcolor{DARCgreen}{knapp über $2 \cdot f_{\symup{max}}$}}}
{knapp über $f_{\symup{max}}$}
{knapp unter $\dfrac{f_{\mathrm{max}}}{2}$}
{knapp unter $f_{\symup{max}}$}
\end{QQuestion}

}
\end{frame}

\begin{frame}
\only<1>{
\begin{QQuestion}{AF619}{Ein analoges Sprachsignal mit \qty{4}{\kHz} Bandbreite soll digital verarbeitet werden. Welche der folgenden Abtastraten ist die kleinste, die Alias-Effekte vermeidet?}{9600 Samples/s}
{4800 Samples/s}
{4000 Samples/s}
{2400 Samples/s}
\end{QQuestion}

}
\only<2>{
\begin{QQuestion}{AF619}{Ein analoges Sprachsignal mit \qty{4}{\kHz} Bandbreite soll digital verarbeitet werden. Welche der folgenden Abtastraten ist die kleinste, die Alias-Effekte vermeidet?}{\textbf{\textcolor{DARCgreen}{9600 Samples/s}}}
{4800 Samples/s}
{4000 Samples/s}
{2400 Samples/s}
\end{QQuestion}

}
\end{frame}%ENDCONTENT


\section{Quantisierung}
\label{section:quantisierung}
\begin{frame}%STARTCONTENT

\only<1>{
\begin{QQuestion}{AF605}{Wie wird die Umwandlung eines wertkontinuierlichen in ein wertdiskretes Signal bezeichnet?}{Codierung}
{Sampling}
{Quantisierung}
{Raummultiplexing}
\end{QQuestion}

}
\only<2>{
\begin{QQuestion}{AF605}{Wie wird die Umwandlung eines wertkontinuierlichen in ein wertdiskretes Signal bezeichnet?}{Codierung}
{Sampling}
{\textbf{\textcolor{DARCgreen}{Quantisierung}}}
{Raummultiplexing}
\end{QQuestion}

}
\end{frame}%ENDCONTENT


\section{Analog-Digital-Umsetzer (ADC)}
\label{section:analog_digital_umsetzer}
\begin{frame}%STARTCONTENT

\only<1>{
\begin{PQuestion}{AF620}{Welche Funktionen haben die einzelnen Blöcke im dargestellten Blockschaltbild eines digitalen Direktempfängers?}{1:~Abtastratengenerator, 2:~Antialiasing-Filter, 3:~Analog-Digital-Umsetzer}
{1:~Analog-Digital-Umsetzer, 2:~Antialiasing-Filter, 3:~Abtastratengenerator}
{1:~Analog-Digital-Umsetzer, 2:~Abtastratengenerator, 3:~Antialiasing-Filter}
{1:~Antialiasing-Filter, 2:~Abtastratengenerator, 3:~Analog-Digital-Umsetzer}
{\DARCimage{1.0\linewidth}{430include}}\end{PQuestion}

}
\only<2>{
\begin{PQuestion}{AF620}{Welche Funktionen haben die einzelnen Blöcke im dargestellten Blockschaltbild eines digitalen Direktempfängers?}{1:~Abtastratengenerator, 2:~Antialiasing-Filter, 3:~Analog-Digital-Umsetzer}
{1:~Analog-Digital-Umsetzer, 2:~Antialiasing-Filter, 3:~Abtastratengenerator}
{1:~Analog-Digital-Umsetzer, 2:~Abtastratengenerator, 3:~Antialiasing-Filter}
{\textbf{\textcolor{DARCgreen}{1:~Antialiasing-Filter, 2:~Abtastratengenerator, 3:~Analog-Digital-Umsetzer}}}
{\DARCimage{1.0\linewidth}{430include}}\end{PQuestion}

}
\end{frame}

\begin{frame}
\only<1>{
\begin{QQuestion}{AF607}{Warum kommt es in einem A/D-Umsetzer zu Quantisierungsfehlern?}{Die Bandbreite des Eingangssignals ist begrenzt.}
{Es steht nur eine begrenzte Anzahl diskreter Werte zur Verfügung.}
{Es können nur ganzzahlige Frequenzen verwendet werden.}
{Es können nur Werte zwischen 0 und 1 genutzt werden.}
\end{QQuestion}

}
\only<2>{
\begin{QQuestion}{AF607}{Warum kommt es in einem A/D-Umsetzer zu Quantisierungsfehlern?}{Die Bandbreite des Eingangssignals ist begrenzt.}
{\textbf{\textcolor{DARCgreen}{Es steht nur eine begrenzte Anzahl diskreter Werte zur Verfügung.}}}
{Es können nur ganzzahlige Frequenzen verwendet werden.}
{Es können nur Werte zwischen 0 und 1 genutzt werden.}
\end{QQuestion}

}
\end{frame}

\begin{frame}
\only<1>{
\begin{QQuestion}{AF608}{Wie viele Bereiche von Eingangswerten, z.\,B. Spannungen, kann ein A/D-Umsetzer mit \qty{8}{\bit} Auflösung maximal trennen?}{256}
{8}
{64}
{1024}
\end{QQuestion}

}
\only<2>{
\begin{QQuestion}{AF608}{Wie viele Bereiche von Eingangswerten, z.\,B. Spannungen, kann ein A/D-Umsetzer mit \qty{8}{\bit} Auflösung maximal trennen?}{\textbf{\textcolor{DARCgreen}{256}}}
{8}
{64}
{1024}
\end{QQuestion}

}
\end{frame}

\begin{frame}
\only<1>{
\begin{QQuestion}{AF621}{Bei einer Abtastung mit einem A/D-Umsetzer mit \qty{24}{\bit} Auflösung wird ein Oszillator mit starkem Taktzittern (Jitter) eingesetzt. Welche Auswirkung wird das Zittern haben?}{Aufgrund der großen Auflösung bleibt die Schwankung ohne Auswirkung.}
{Das Abschirmblech des A/D-Umsetzers wird durch Vibration störende Geräusche erzeugen.}
{Es entsteht zusätzliches Rauschen im Abtastergebnis.}
{Das Abtastergebnis wird verbessert (Dithering).}
\end{QQuestion}

}
\only<2>{
\begin{QQuestion}{AF621}{Bei einer Abtastung mit einem A/D-Umsetzer mit \qty{24}{\bit} Auflösung wird ein Oszillator mit starkem Taktzittern (Jitter) eingesetzt. Welche Auswirkung wird das Zittern haben?}{Aufgrund der großen Auflösung bleibt die Schwankung ohne Auswirkung.}
{Das Abschirmblech des A/D-Umsetzers wird durch Vibration störende Geräusche erzeugen.}
{\textbf{\textcolor{DARCgreen}{Es entsteht zusätzliches Rauschen im Abtastergebnis.}}}
{Das Abtastergebnis wird verbessert (Dithering).}
\end{QQuestion}

}
\end{frame}%ENDCONTENT


\section{Digital-Analog-Umsetzer (DAC)}
\label{section:digital_analog_umsetzer}
\begin{frame}%STARTCONTENT

\only<1>{
\begin{QQuestion}{AF609}{Wie viele verschiedene Ausgangswerte, z.\,B. Spannungen, kann ein idealer D/A-Umsetzer mit \qty{10}{\bit} Auflösung erzeugen?}{256}
{10}
{100}
{1024}
\end{QQuestion}

}
\only<2>{
\begin{QQuestion}{AF609}{Wie viele verschiedene Ausgangswerte, z.\,B. Spannungen, kann ein idealer D/A-Umsetzer mit \qty{10}{\bit} Auflösung erzeugen?}{256}
{10}
{100}
{\textbf{\textcolor{DARCgreen}{1024}}}
\end{QQuestion}

}
\end{frame}

\begin{frame}
\only<1>{
\begin{QQuestion}{AF611}{Wie groß ist die Schrittweite zwischen den Spannungsstufen eines linear arbeitenden D/A-Umsetzers mit \qty{10}{\bit} Auflösung und einem Wertebereich von \qtyrange{0}{1}{\V}?}{ca. \qty{10}{\mV}}
{ca. \qty{1}{\mV}}
{ca. \qty{0,1}{\V}}
{ca. \qty{1}{\V}}
\end{QQuestion}

}
\only<2>{
\begin{QQuestion}{AF611}{Wie groß ist die Schrittweite zwischen den Spannungsstufen eines linear arbeitenden D/A-Umsetzers mit \qty{10}{\bit} Auflösung und einem Wertebereich von \qtyrange{0}{1}{\V}?}{ca. \qty{10}{\mV}}
{\textbf{\textcolor{DARCgreen}{ca. \qty{1}{\mV}}}}
{ca. \qty{0,1}{\V}}
{ca. \qty{1}{\V}}
\end{QQuestion}

}
\end{frame}

\begin{frame}
\only<1>{
\begin{QQuestion}{AF610}{Wie groß ist die Schrittweite zwischen den Spannungsstufen eines linear arbeitenden D/A-Umsetzers mit \qty{8}{\bit} Auflösung und einem Wertebereich von \qtyrange{0}{1}{\V}?}{ca. \qty{4}{\mV}}
{ca. \qty{1}{\mV}}
{ca. \qty{2}{\mV}}
{ca. \qty{8}{\mV}}
\end{QQuestion}

}
\only<2>{
\begin{QQuestion}{AF610}{Wie groß ist die Schrittweite zwischen den Spannungsstufen eines linear arbeitenden D/A-Umsetzers mit \qty{8}{\bit} Auflösung und einem Wertebereich von \qtyrange{0}{1}{\V}?}{\textbf{\textcolor{DARCgreen}{ca. \qty{4}{\mV}}}}
{ca. \qty{1}{\mV}}
{ca. \qty{2}{\mV}}
{ca. \qty{8}{\mV}}
\end{QQuestion}

}
\end{frame}%ENDCONTENT


\section{Anwendung von ADC und DAC}
\label{section:anwendung_dac_adc}
\begin{frame}%STARTCONTENT

\only<1>{
\begin{question2x2}{AF613}{Eine Sinusschwingung mit einem Spitzenwert von \qty{1,5}{\V} wird in einen A/D-Umsetzer eingegeben, dessen Ausgang direkt mit einem D/A-Umsetzer verbunden ist. Beide Umsetzer arbeiten linear mit einer Auflösung von \qty{12}{\bit} und einem Wertebereich von \qty{-2}{\V} bis \qty{2}{\V}. Welches Signal ist am Ausgang des D/A-Umsetzers zu erwarten?}{\DARCimage{1.0\linewidth}{299include}}
{\DARCimage{1.0\linewidth}{297include}}
{\DARCimage{1.0\linewidth}{295include}}
{\DARCimage{1.0\linewidth}{300include}}
\end{question2x2}

}
\only<2>{
\begin{question2x2}{AF613}{Eine Sinusschwingung mit einem Spitzenwert von \qty{1,5}{\V} wird in einen A/D-Umsetzer eingegeben, dessen Ausgang direkt mit einem D/A-Umsetzer verbunden ist. Beide Umsetzer arbeiten linear mit einer Auflösung von \qty{12}{\bit} und einem Wertebereich von \qty{-2}{\V} bis \qty{2}{\V}. Welches Signal ist am Ausgang des D/A-Umsetzers zu erwarten?}{\DARCimage{1.0\linewidth}{299include}}
{\DARCimage{1.0\linewidth}{297include}}
{\textbf{\textcolor{DARCgreen}{\DARCimage{1.0\linewidth}{295include}}}}
{\DARCimage{1.0\linewidth}{300include}}
\end{question2x2}

}
\end{frame}

\begin{frame}
\only<1>{
\begin{question2x2}{AF612}{Eine Sinusschwingung mit einem Spitzenwert von \qty{1,5}{\V} wird in einen A/D-Umsetzer eingegeben, dessen Ausgang direkt mit einem D/A-Umsetzer verbunden ist. Beide Umsetzer arbeiten linear mit einer Auflösung von \qty{4}{\bit} und einem Wertebereich von \qty{-2}{\V} bis \qty{2}{\V}. Welches Signal ist am Ausgang des D/A-Umsetzers zu erwarten?}{\DARCimage{1.0\linewidth}{295include}}
{\DARCimage{1.0\linewidth}{297include}}
{\DARCimage{1.0\linewidth}{299include}}
{\DARCimage{1.0\linewidth}{298include}}
\end{question2x2}

}
\only<2>{
\begin{question2x2}{AF612}{Eine Sinusschwingung mit einem Spitzenwert von \qty{1,5}{\V} wird in einen A/D-Umsetzer eingegeben, dessen Ausgang direkt mit einem D/A-Umsetzer verbunden ist. Beide Umsetzer arbeiten linear mit einer Auflösung von \qty{4}{\bit} und einem Wertebereich von \qty{-2}{\V} bis \qty{2}{\V}. Welches Signal ist am Ausgang des D/A-Umsetzers zu erwarten?}{\DARCimage{1.0\linewidth}{295include}}
{\textbf{\textcolor{DARCgreen}{\DARCimage{1.0\linewidth}{297include}}}}
{\DARCimage{1.0\linewidth}{299include}}
{\DARCimage{1.0\linewidth}{298include}}
\end{question2x2}

}
\end{frame}

\begin{frame}
\only<1>{
\begin{question2x2}{AF614}{Eine Sinusschwingung mit einem Spitzenwert von \qty{1,5}{\V} wird in einen A/D-Umsetzer eingegeben, dessen Ausgang direkt mit einem D/A-Umsetzer verbunden ist. Beide Umsetzer arbeiten linear mit einer Auflösung von \qty{12}{\bit} und einem Wertebereich von \qty{-1}{\V} bis \qty{1}{\V}. Welches Signal ist am Ausgang des D/A-Umsetzers zu erwarten?}{\DARCimage{1.0\linewidth}{298include}}
{\DARCimage{1.0\linewidth}{296include}}
{\DARCimage{1.0\linewidth}{295include}}
{\DARCimage{1.0\linewidth}{297include}}
\end{question2x2}

}
\only<2>{
\begin{question2x2}{AF614}{Eine Sinusschwingung mit einem Spitzenwert von \qty{1,5}{\V} wird in einen A/D-Umsetzer eingegeben, dessen Ausgang direkt mit einem D/A-Umsetzer verbunden ist. Beide Umsetzer arbeiten linear mit einer Auflösung von \qty{12}{\bit} und einem Wertebereich von \qty{-1}{\V} bis \qty{1}{\V}. Welches Signal ist am Ausgang des D/A-Umsetzers zu erwarten?}{\DARCimage{1.0\linewidth}{298include}}
{\textbf{\textcolor{DARCgreen}{\DARCimage{1.0\linewidth}{296include}}}}
{\DARCimage{1.0\linewidth}{295include}}
{\DARCimage{1.0\linewidth}{297include}}
\end{question2x2}

}
\end{frame}%ENDCONTENT


\section{Anti-Alias-Filter}
\label{section:anti_alias_filter}
\begin{frame}%STARTCONTENT

\only<1>{
\begin{QQuestion}{AF622}{Welcher Filtertyp ist geeignet, um Alias-Effekte zu vermeiden, und wo ist das Filter zu platzieren?}{Tiefpassfilter vor dem A/D-Umsetzer}
{Hochpassfilter vor dem A/D-Umsetzer}
{Tiefpassfilter nach dem D/A-Umsetzer}
{Hochpassfilter nach dem D/A-Umsetzer}
\end{QQuestion}

}
\only<2>{
\begin{QQuestion}{AF622}{Welcher Filtertyp ist geeignet, um Alias-Effekte zu vermeiden, und wo ist das Filter zu platzieren?}{\textbf{\textcolor{DARCgreen}{Tiefpassfilter vor dem A/D-Umsetzer}}}
{Hochpassfilter vor dem A/D-Umsetzer}
{Tiefpassfilter nach dem D/A-Umsetzer}
{Hochpassfilter nach dem D/A-Umsetzer}
\end{QQuestion}

}
\end{frame}

\begin{frame}
\only<1>{
\begin{question2x2}{AF623}{Sie wollen ein Sprachsignal mit einer Abtastrate von $f_{\symup{A}}$ = 8000 Samples je Sekunde digitalisieren. Vor dem A/D-Umsetzer soll ein Anti-Alias-Filter eingesetzt werden. Welcher Amplitudengang ist für das Filter am besten geeignet?}{\DARCimage{1.0\linewidth}{403include}}
{\DARCimage{1.0\linewidth}{401include}}
{\DARCimage{1.0\linewidth}{402include}}
{\DARCimage{1.0\linewidth}{400include}}
\end{question2x2}

}
\only<2>{
\begin{question2x2}{AF623}{Sie wollen ein Sprachsignal mit einer Abtastrate von $f_{\symup{A}}$ = 8000 Samples je Sekunde digitalisieren. Vor dem A/D-Umsetzer soll ein Anti-Alias-Filter eingesetzt werden. Welcher Amplitudengang ist für das Filter am besten geeignet?}{\DARCimage{1.0\linewidth}{403include}}
{\DARCimage{1.0\linewidth}{401include}}
{\DARCimage{1.0\linewidth}{402include}}
{\textbf{\textcolor{DARCgreen}{\DARCimage{1.0\linewidth}{400include}}}}
\end{question2x2}

}
\end{frame}%ENDCONTENT


\section{Rekonstruktionsfilter}
\label{section:rekonstruktionsfilter}
\begin{frame}%STARTCONTENT

\only<1>{
\begin{QQuestion}{AF624}{Welcher Filtertyp ist als Rekonstruktionsfilter geeignet und wo ist das Filter zu platzieren?}{Tiefpassfilter nach dem D/A-Umsetzer}
{Hochpassfilter nach dem D/A-Umsetzer}
{Tiefpassfilter vor dem A/D-Umsetzer}
{Hochpassfilter vor dem A/D-Umsetzer}
\end{QQuestion}

}
\only<2>{
\begin{QQuestion}{AF624}{Welcher Filtertyp ist als Rekonstruktionsfilter geeignet und wo ist das Filter zu platzieren?}{\textbf{\textcolor{DARCgreen}{Tiefpassfilter nach dem D/A-Umsetzer}}}
{Hochpassfilter nach dem D/A-Umsetzer}
{Tiefpassfilter vor dem A/D-Umsetzer}
{Hochpassfilter vor dem A/D-Umsetzer}
\end{QQuestion}

}
\end{frame}

\begin{frame}
\only<1>{
\begin{question2x2}{AF625}{Sie wollen ein Sprachsignal mit einer Abtastrate von $f_{\symup{A}}$ = 8000 Samples je Sekunde rekonstruieren. Nach dem D/A-Umsetzer soll ein Rekonstruktionsfilter eingesetzt werden. Welcher Amplitudengang ist für das Filter am besten geeignet?}{\DARCimage{1.0\linewidth}{401include}}
{\DARCimage{1.0\linewidth}{655include}}
{\DARCimage{1.0\linewidth}{402include}}
{\DARCimage{1.0\linewidth}{403include}}
\end{question2x2}

}
\only<2>{
\begin{question2x2}{AF625}{Sie wollen ein Sprachsignal mit einer Abtastrate von $f_{\symup{A}}$ = 8000 Samples je Sekunde rekonstruieren. Nach dem D/A-Umsetzer soll ein Rekonstruktionsfilter eingesetzt werden. Welcher Amplitudengang ist für das Filter am besten geeignet?}{\DARCimage{1.0\linewidth}{401include}}
{\textbf{\textcolor{DARCgreen}{\DARCimage{1.0\linewidth}{655include}}}}
{\DARCimage{1.0\linewidth}{402include}}
{\DARCimage{1.0\linewidth}{403include}}
\end{question2x2}

}
\end{frame}%ENDCONTENT


\section{Fourier-Transformation}
\label{section:fourier_transformation}
\begin{frame}%STARTCONTENT

\only<1>{
\begin{QQuestion}{AF630}{Wozu dient die diskrete Fouriertransformation mittels FFT? Es ist eine schnelle mathematische Methode zur Umwandlung~...}{eines diskreten Widerstandswertes in eine Impedanz.}
{eines zeitdiskreten Signals in ein analoges Signal.}
{eines zeitdiskreten Signals in ein Frequenzspektrum.}
{eines Widerstandswertes in einen diskreten Leitwert.}
\end{QQuestion}

}
\only<2>{
\begin{QQuestion}{AF630}{Wozu dient die diskrete Fouriertransformation mittels FFT? Es ist eine schnelle mathematische Methode zur Umwandlung~...}{eines diskreten Widerstandswertes in eine Impedanz.}
{eines zeitdiskreten Signals in ein analoges Signal.}
{\textbf{\textcolor{DARCgreen}{eines zeitdiskreten Signals in ein Frequenzspektrum.}}}
{eines Widerstandswertes in einen diskreten Leitwert.}
\end{QQuestion}

}
\end{frame}

\begin{frame}
\only<1>{
\begin{PQuestion}{AB404}{Welches Frequenzspektrum passt zu folgendem sinusförmigen Signal?}{\DARCimage{1.0\linewidth}{392include}}
{\DARCimage{1.0\linewidth}{391include}}
{\DARCimage{1.0\linewidth}{390include}}
{\DARCimage{1.0\linewidth}{393include}}
{\DARCimage{1.0\linewidth}{386include}}\end{PQuestion}

}
\only<2>{
\begin{PQuestion}{AB404}{Welches Frequenzspektrum passt zu folgendem sinusförmigen Signal?}{\DARCimage{1.0\linewidth}{392include}}
{\DARCimage{1.0\linewidth}{391include}}
{\textbf{\textcolor{DARCgreen}{\DARCimage{1.0\linewidth}{390include}}}}
{\DARCimage{1.0\linewidth}{393include}}
{\DARCimage{1.0\linewidth}{386include}}\end{PQuestion}

}
\end{frame}

\begin{frame}
\only<1>{
\begin{PQuestion}{AB405}{Welches Frequenzspektrum passt zu folgendem periodischen Signal?}{\DARCimage{1.0\linewidth}{391include}}
{\DARCimage{1.0\linewidth}{390include}}
{\DARCimage{1.0\linewidth}{392include}}
{\DARCimage{1.0\linewidth}{393include}}
{\DARCimage{1.0\linewidth}{387include}}\end{PQuestion}

}
\only<2>{
\begin{PQuestion}{AB405}{Welches Frequenzspektrum passt zu folgendem periodischen Signal?}{\textbf{\textcolor{DARCgreen}{\DARCimage{1.0\linewidth}{391include}}}}
{\DARCimage{1.0\linewidth}{390include}}
{\DARCimage{1.0\linewidth}{392include}}
{\DARCimage{1.0\linewidth}{393include}}
{\DARCimage{1.0\linewidth}{387include}}\end{PQuestion}

}
\end{frame}

\begin{frame}
\only<1>{
\begin{PQuestion}{AB406}{Welches Signal passt zu folgendem Frequenzspektrum?}{\DARCimage{0.75\linewidth}{386include}}
{\DARCimage{0.75\linewidth}{387include}}
{\DARCimage{0.75\linewidth}{388include}}
{\DARCimage{0.75\linewidth}{389include}}
{\DARCimage{0.75\linewidth}{390include}}\end{PQuestion}

}
\only<2>{
\begin{PQuestion}{AB406}{Welches Signal passt zu folgendem Frequenzspektrum?}{\textbf{\textcolor{DARCgreen}{\DARCimage{0.75\linewidth}{386include}}}}
{\DARCimage{0.75\linewidth}{387include}}
{\DARCimage{0.75\linewidth}{388include}}
{\DARCimage{0.75\linewidth}{389include}}
{\DARCimage{0.75\linewidth}{390include}}\end{PQuestion}

}
\end{frame}

\begin{frame}
\only<1>{
\begin{PQuestion}{AB407}{Welches Signal passt zu folgendem Frequenzspektrum?}{\DARCimage{0.75\linewidth}{632include}}
{\DARCimage{0.75\linewidth}{386include}}
{\DARCimage{0.75\linewidth}{631include}}
{\DARCimage{0.75\linewidth}{387include}}
{\DARCimage{0.75\linewidth}{391include}}\end{PQuestion}

}
\only<2>{
\begin{PQuestion}{AB407}{Welches Signal passt zu folgendem Frequenzspektrum?}{\DARCimage{0.75\linewidth}{632include}}
{\DARCimage{0.75\linewidth}{386include}}
{\DARCimage{0.75\linewidth}{631include}}
{\textbf{\textcolor{DARCgreen}{\DARCimage{0.75\linewidth}{387include}}}}
{\DARCimage{0.75\linewidth}{391include}}\end{PQuestion}

}
\end{frame}%ENDCONTENT


\section{Digitale Filter}
\label{section:digitale_filter}
\begin{frame}%STARTCONTENT

\only<1>{
\begin{QQuestion}{AF631}{Welche der folgenden Aussagen zu digitalen Filtern ist richtig? Digitale Filter können~...}{nicht in Software realisiert werden.}
{ohne Latenz realisiert werden.}
{nicht in Hardware realisiert werden.}
{als FIR- oder IIR-Filter realisiert werden.}
\end{QQuestion}

}
\only<2>{
\begin{QQuestion}{AF631}{Welche der folgenden Aussagen zu digitalen Filtern ist richtig? Digitale Filter können~...}{nicht in Software realisiert werden.}
{ohne Latenz realisiert werden.}
{nicht in Hardware realisiert werden.}
{\textbf{\textcolor{DARCgreen}{als FIR- oder IIR-Filter realisiert werden.}}}
\end{QQuestion}

}
\end{frame}%ENDCONTENT


\section{I/Q-Verfahren}
\label{section:iq_verfahren}
\begin{frame}%STARTCONTENT

\only<1>{
\begin{QQuestion}{AE404}{Wie wird Quadraturamplitudenmodulation (QAM) üblicherweise erzeugt? Durch~...}{separate Änderung der Amplitude des elektrischen und magnetischen Feldwellenanteils}
{nichtlineare Änderung der Amplitude (Quadratfunktion bzw. Quadratwurzel)}
{Änderung der Amplituden und Addition zweier um \qty{90}{\degree} phasenverschobener Träger}
{richtungsabhängige Änderung der Frequenz (bzw. richtungsinvariante Änderung der Amplitude)}
\end{QQuestion}

}
\only<2>{
\begin{QQuestion}{AE404}{Wie wird Quadraturamplitudenmodulation (QAM) üblicherweise erzeugt? Durch~...}{separate Änderung der Amplitude des elektrischen und magnetischen Feldwellenanteils}
{nichtlineare Änderung der Amplitude (Quadratfunktion bzw. Quadratwurzel)}
{\textbf{\textcolor{DARCgreen}{Änderung der Amplituden und Addition zweier um \qty{90}{\degree} phasenverschobener Träger}}}
{richtungsabhängige Änderung der Frequenz (bzw. richtungsinvariante Änderung der Amplitude)}
\end{QQuestion}

}
\end{frame}

\begin{frame}
\only<1>{
\begin{PQuestion}{AF632}{Wie groß muss die Phasenverschiebung $\varphi$ in der dargestellten Modulatorschaltung sein, damit eine korrekte Quadraturmodulation vorliegt?}{\qty{180}{\degree}}
{\qty{90}{\degree}}
{\qty{0}{\degree}}
{\qty{45}{\degree}}
{\DARCimage{1.0\linewidth}{196include}}\end{PQuestion}

}
\only<2>{
\begin{PQuestion}{AF632}{Wie groß muss die Phasenverschiebung $\varphi$ in der dargestellten Modulatorschaltung sein, damit eine korrekte Quadraturmodulation vorliegt?}{\qty{180}{\degree}}
{\textbf{\textcolor{DARCgreen}{\qty{90}{\degree}}}}
{\qty{0}{\degree}}
{\qty{45}{\degree}}
{\DARCimage{1.0\linewidth}{196include}}\end{PQuestion}

}
\end{frame}

\begin{frame}
\only<1>{
\begin{QQuestion}{AF633}{Was bildet der I- bzw. der Q-Anteil eines I/Q-Signals ab?}{Den Wechselstrom (I) in Abhängigkeit der Güte (Q) eines Schwingkreises bei seiner Resonanzfrequenz}
{Die phasengleichen (I) bzw. die um \qty{90}{\degree} phasenverschobenen (Q) Anteile eines Signals in Bezug auf eine Referenzschwingung}
{Den Stromanteil (I) und den Blindleistungsanteil (Q) eines Signals}
{Die erste (I) bzw. die vierte (Q) Harmonische in Bezug auf ein normiertes Rechtecksignal}
\end{QQuestion}

}
\only<2>{
\begin{QQuestion}{AF633}{Was bildet der I- bzw. der Q-Anteil eines I/Q-Signals ab?}{Den Wechselstrom (I) in Abhängigkeit der Güte (Q) eines Schwingkreises bei seiner Resonanzfrequenz}
{\textbf{\textcolor{DARCgreen}{Die phasengleichen (I) bzw. die um \qty{90}{\degree} phasenverschobenen (Q) Anteile eines Signals in Bezug auf eine Referenzschwingung}}}
{Den Stromanteil (I) und den Blindleistungsanteil (Q) eines Signals}
{Die erste (I) bzw. die vierte (Q) Harmonische in Bezug auf ein normiertes Rechtecksignal}
\end{QQuestion}

}
\end{frame}

\begin{frame}
\only<1>{
\begin{QQuestion}{AF634}{Welchen Frequenzbereich (z.~B. in Bezug auf eine Mitten- oder Trägerfrequenz) kann ein digitaler Datenstrom entsprechend dem Abtasttheorem maximal eindeutig abbilden, der aus einem I- und einem Q-Anteil mit einer Abtastrate von \underline{jeweils} 48000 Samples pro Sekunde besteht? Den Bereich zwischen~...}{\qty{-24}{\kHz} und \qty{+24}{\kHz}.}
{\qty{-48}{\kHz} und \qty{+48}{\kHz}.}
{\qty{0}{\Hz} und \qty{96}{\kHz}.}
{\qty{0}{\Hz} und \qty{6}{\kHz}.}
\end{QQuestion}

}
\only<2>{
\begin{QQuestion}{AF634}{Welchen Frequenzbereich (z.~B. in Bezug auf eine Mitten- oder Trägerfrequenz) kann ein digitaler Datenstrom entsprechend dem Abtasttheorem maximal eindeutig abbilden, der aus einem I- und einem Q-Anteil mit einer Abtastrate von \underline{jeweils} 48000 Samples pro Sekunde besteht? Den Bereich zwischen~...}{\textbf{\textcolor{DARCgreen}{\qty{-24}{\kHz} und \qty{+24}{\kHz}.}}}
{\qty{-48}{\kHz} und \qty{+48}{\kHz}.}
{\qty{0}{\Hz} und \qty{96}{\kHz}.}
{\qty{0}{\Hz} und \qty{6}{\kHz}.}
\end{QQuestion}

}
\end{frame}

\begin{frame}
\only<1>{
\begin{QQuestion}{AF635}{Welchen Frequenzbereich (z.~B. in Bezug auf eine Mitten- oder Trägerfrequenz) kann ein digitaler Datenstrom entsprechend dem Abtasttheorem maximal eindeutig abbilden, der aus einem I- und einem Q-Anteil mit einer Abtastrate von \underline{jeweils} 96000 Samples pro Sekunde besteht? Den Bereich zwischen~...}{\qty{-48}{\kHz} und \qty{+48}{\kHz}.}
{\qty{-24}{\kHz} und \qty{+24}{\kHz}.}
{\qty{0}{\Hz} und \qty{192}{\kHz}.}
{\qty{0}{\Hz} und \qty{9,6}{\kHz}.}
\end{QQuestion}

}
\only<2>{
\begin{QQuestion}{AF635}{Welchen Frequenzbereich (z.~B. in Bezug auf eine Mitten- oder Trägerfrequenz) kann ein digitaler Datenstrom entsprechend dem Abtasttheorem maximal eindeutig abbilden, der aus einem I- und einem Q-Anteil mit einer Abtastrate von \underline{jeweils} 96000 Samples pro Sekunde besteht? Den Bereich zwischen~...}{\textbf{\textcolor{DARCgreen}{\qty{-48}{\kHz} und \qty{+48}{\kHz}.}}}
{\qty{-24}{\kHz} und \qty{+24}{\kHz}.}
{\qty{0}{\Hz} und \qty{192}{\kHz}.}
{\qty{0}{\Hz} und \qty{9,6}{\kHz}.}
\end{QQuestion}

}
\end{frame}

\begin{frame}
\only<1>{
\begin{QQuestion}{AF636}{Welchen Frequenzbereich (z.~B. in Bezug auf eine Mitten- oder Trägerfrequenz) kann ein digitaler Datenstrom entsprechend dem Abtasttheorem maximal eindeutig abbilden, der aus einem I- und einem Q-Anteil mit einer Abtastrate von \underline{jeweils} 10 Millionen Samples pro Sekunde besteht? Den Bereich zwischen~...}{\qty{-5}{\MHz} und \qty{+5}{\MHz}.}
{\qty{-10}{\MHz} und \qty{+10}{\MHz}.}
{\qty{0}{\Hz} und \qty{512}{\kHz}.}
{\qty{0}{\Hz} und \qty{1024}{\kHz}.}
\end{QQuestion}

}
\only<2>{
\begin{QQuestion}{AF636}{Welchen Frequenzbereich (z.~B. in Bezug auf eine Mitten- oder Trägerfrequenz) kann ein digitaler Datenstrom entsprechend dem Abtasttheorem maximal eindeutig abbilden, der aus einem I- und einem Q-Anteil mit einer Abtastrate von \underline{jeweils} 10 Millionen Samples pro Sekunde besteht? Den Bereich zwischen~...}{\textbf{\textcolor{DARCgreen}{\qty{-5}{\MHz} und \qty{+5}{\MHz}.}}}
{\qty{-10}{\MHz} und \qty{+10}{\MHz}.}
{\qty{0}{\Hz} und \qty{512}{\kHz}.}
{\qty{0}{\Hz} und \qty{1024}{\kHz}.}
\end{QQuestion}

}
\end{frame}%ENDCONTENT


\section{Latenz}
\label{section:latenz}
\begin{frame}%STARTCONTENT

\only<1>{
\begin{QQuestion}{AF637}{Was wird in der digitalen Signalverarbeitung unter Latenz verstanden und in welcher Einheit kann sie angegeben werden?}{Schwankung der Amplitude eines Signals in Volt pro Sekunde}
{Geschwindigkeit eines Signals in Metern pro Sekunde}
{Schwankung der Frequenz eines Signals in Hertz pro Sekunde}
{Laufzeit bzw. Verzögerung eines Signals in Sekunden}
\end{QQuestion}

}
\only<2>{
\begin{QQuestion}{AF637}{Was wird in der digitalen Signalverarbeitung unter Latenz verstanden und in welcher Einheit kann sie angegeben werden?}{Schwankung der Amplitude eines Signals in Volt pro Sekunde}
{Geschwindigkeit eines Signals in Metern pro Sekunde}
{Schwankung der Frequenz eines Signals in Hertz pro Sekunde}
{\textbf{\textcolor{DARCgreen}{Laufzeit bzw. Verzögerung eines Signals in Sekunden}}}
\end{QQuestion}

}
\end{frame}%ENDCONTENT


\title{DARC Amateurfunklehrgang Klasse A}
\author{Antennen und Übertragungsleitungen}
\institute{Deutscher Amateur Radio Club e.\,V.}
\begin{frame}
\maketitle
\end{frame}

\section{Polarisation III}
\label{section:polarisation_3}
\begin{frame}%STARTCONTENT

\only<1>{
\begin{QQuestion}{AG201}{Mit welcher Polarisation wird auf den Kurzwellenbändern meistens gesendet?}{Es wird nur mit horizontaler Polarisation gesendet.}
{Es wird meistens mit horizontaler oder zirkularer Polarisation gesendet.}
{Es wird meistens mit vertikaler oder zirkularer Polarisation gesendet.}
{Es wird meistens mit horizontaler oder vertikaler Polarisation gesendet.}
\end{QQuestion}

}
\only<2>{
\begin{QQuestion}{AG201}{Mit welcher Polarisation wird auf den Kurzwellenbändern meistens gesendet?}{Es wird nur mit horizontaler Polarisation gesendet.}
{Es wird meistens mit horizontaler oder zirkularer Polarisation gesendet.}
{Es wird meistens mit vertikaler oder zirkularer Polarisation gesendet.}
{\textbf{\textcolor{DARCgreen}{Es wird meistens mit horizontaler oder vertikaler Polarisation gesendet.}}}
\end{QQuestion}

}
\end{frame}%ENDCONTENT


\section{Antennenformen III}
\label{section:antennenformen_3}
\begin{frame}%STARTCONTENT

\only<1>{
\begin{PQuestion}{AG419}{Was ist beim Aufbau des dargestellten Drahtantennensystems zu beachten? Die Drahtlänge des Strahlers sollte~...}{gleich 5/8~$\lambda$ der benutzten Frequenz sein oder einem Vielfachen davon entsprechen.}
{gleich 1/2~$\lambda$ der benutzten Frequenz sein oder einem Vielfachen davon entsprechen.}
{genau 1/4~$\lambda$ der benutzten Frequenzen sein.}
{genau 3/8~$\lambda$ der benutzten Frequenzen sein.}
{\DARCimage{1.0\linewidth}{310include}}\end{PQuestion}

}
\only<2>{
\begin{PQuestion}{AG419}{Was ist beim Aufbau des dargestellten Drahtantennensystems zu beachten? Die Drahtlänge des Strahlers sollte~...}{gleich 5/8~$\lambda$ der benutzten Frequenz sein oder einem Vielfachen davon entsprechen.}
{\textbf{\textcolor{DARCgreen}{gleich 1/2~$\lambda$ der benutzten Frequenz sein oder einem Vielfachen davon entsprechen.}}}
{genau 1/4~$\lambda$ der benutzten Frequenzen sein.}
{genau 3/8~$\lambda$ der benutzten Frequenzen sein.}
{\DARCimage{1.0\linewidth}{310include}}\end{PQuestion}

}
\end{frame}

\begin{frame}
\only<1>{
\begin{PQuestion}{AG123}{Wie wird die dargestellte Antenne bezeichnet (MWS~=~Mantelwellensperre)?}{endgespeiste, magnetische Multibandantenne}
{Windomantenne}
{W3DZZ}
{endgespeiste Multibandantenne}
{\DARCimage{1.0\linewidth}{315include}}\end{PQuestion}

}
\only<2>{
\begin{PQuestion}{AG123}{Wie wird die dargestellte Antenne bezeichnet (MWS~=~Mantelwellensperre)?}{endgespeiste, magnetische Multibandantenne}
{Windomantenne}
{W3DZZ}
{\textbf{\textcolor{DARCgreen}{endgespeiste Multibandantenne}}}
{\DARCimage{1.0\linewidth}{315include}}\end{PQuestion}

}
\end{frame}

\begin{frame}
\only<1>{
\begin{PQuestion}{AG124}{Wie wird die in der nachfolgenden Skizze dargestellte Antenne bezeichnet (MWS~=~Mantelwellensperre)? Es handelt sich um eine~...}{elektrisch verkürzte Windomantenne}
{mit magnetischem Balun aufgebaute Multibandantenne}
{endgespeiste Multibandantenne mit einem Trap}
{endgespeiste, resonante Multibandantenne}
{\DARCimage{1.0\linewidth}{260include}}\end{PQuestion}

}
\only<2>{
\begin{PQuestion}{AG124}{Wie wird die in der nachfolgenden Skizze dargestellte Antenne bezeichnet (MWS~=~Mantelwellensperre)? Es handelt sich um eine~...}{elektrisch verkürzte Windomantenne}
{mit magnetischem Balun aufgebaute Multibandantenne}
{endgespeiste Multibandantenne mit einem Trap}
{\textbf{\textcolor{DARCgreen}{endgespeiste, resonante Multibandantenne}}}
{\DARCimage{1.0\linewidth}{260include}}\end{PQuestion}

}
\end{frame}

\begin{frame}
\only<1>{
\begin{PQuestion}{AG120}{Wie wird die folgende Antenne in der Amateurfunkliteratur bezeichnet?  }{Fuchs-Antenne}
{Windom-Antenne}
{Zeppelin-Antenne}
{Marconi-Antenne}
{\DARCimage{1.0\linewidth}{314include}}\end{PQuestion}

}
\only<2>{
\begin{PQuestion}{AG120}{Wie wird die folgende Antenne in der Amateurfunkliteratur bezeichnet?  }{Fuchs-Antenne}
{Windom-Antenne}
{\textbf{\textcolor{DARCgreen}{Zeppelin-Antenne}}}
{Marconi-Antenne}
{\DARCimage{1.0\linewidth}{314include}}\end{PQuestion}

}
\end{frame}

\begin{frame}
\only<1>{
\begin{PQuestion}{AG117}{Wie wird die folgende Antenne in der Amateurfunkliteratur üblicherweise bezeichnet?}{Dreieck-Antenne}
{Delta-Loop (Ganzwellenschleife)}
{Koaxial-Stub-Antenne}
{koaxial gespeiste Dreilinien-Antenne}
{\DARCimage{1.0\linewidth}{311include}}\end{PQuestion}

}
\only<2>{
\begin{PQuestion}{AG117}{Wie wird die folgende Antenne in der Amateurfunkliteratur üblicherweise bezeichnet?}{Dreieck-Antenne}
{\textbf{\textcolor{DARCgreen}{Delta-Loop (Ganzwellenschleife)}}}
{Koaxial-Stub-Antenne}
{koaxial gespeiste Dreilinien-Antenne}
{\DARCimage{1.0\linewidth}{311include}}\end{PQuestion}

}
\end{frame}

\begin{frame}
\only<1>{
\begin{QQuestion}{AG119}{Bei einer Quad-Antenne beträgt die elektrische Länge jeder Seite~...}{eine ganze Wellenlänge.}
{die Hälfte der Wellenlänge.}
{dreiviertel der Wellenlänge.}
{ein Viertel der Wellenlänge.}
\end{QQuestion}

}
\only<2>{
\begin{QQuestion}{AG119}{Bei einer Quad-Antenne beträgt die elektrische Länge jeder Seite~...}{eine ganze Wellenlänge.}
{die Hälfte der Wellenlänge.}
{dreiviertel der Wellenlänge.}
{\textbf{\textcolor{DARCgreen}{ein Viertel der Wellenlänge.}}}
\end{QQuestion}

}
\end{frame}

\begin{frame}
\only<1>{
\begin{PQuestion}{AG121}{Wie wird die folgende Antenne in der Amateurfunkliteratur bezeichnet?  }{Windom-Antenne}
{G5RV-Antenne}
{Fuchs-Antenne}
{Zeppelin-Antenne}
{\DARCimage{1.0\linewidth}{313include}}\end{PQuestion}

}
\only<2>{
\begin{PQuestion}{AG121}{Wie wird die folgende Antenne in der Amateurfunkliteratur bezeichnet?  }{Windom-Antenne}
{\textbf{\textcolor{DARCgreen}{G5RV-Antenne}}}
{Fuchs-Antenne}
{Zeppelin-Antenne}
{\DARCimage{1.0\linewidth}{313include}}\end{PQuestion}

}
\end{frame}

\begin{frame}
\only<1>{
\begin{PQuestion}{AG122}{Wie wird die folgende Antenne in der Amateurfunkliteratur bezeichnet?  }{Fuchs-Antenne}
{Windom-Antenne}
{Zeppelin-Antenne}
{Marconi-Antenne}
{\DARCimage{1.0\linewidth}{309include}}\end{PQuestion}

}
\only<2>{
\begin{PQuestion}{AG122}{Wie wird die folgende Antenne in der Amateurfunkliteratur bezeichnet?  }{Fuchs-Antenne}
{\textbf{\textcolor{DARCgreen}{Windom-Antenne}}}
{Zeppelin-Antenne}
{Marconi-Antenne}
{\DARCimage{1.0\linewidth}{309include}}\end{PQuestion}

}
\end{frame}

\begin{frame}
\only<1>{
\begin{QQuestion}{AG223}{Bei welcher Länge erreicht eine Vertikalantenne für den Kurzwellenbereich über einer Erdoberfläche mittlerer Leitfähigkeit eine möglichst flache Abstrahlung?}{5/8$~\lambda$}
{$\lambda$/4}
{$\lambda$/2}
{3/4$~\lambda$}
\end{QQuestion}

}
\only<2>{
\begin{QQuestion}{AG223}{Bei welcher Länge erreicht eine Vertikalantenne für den Kurzwellenbereich über einer Erdoberfläche mittlerer Leitfähigkeit eine möglichst flache Abstrahlung?}{\textbf{\textcolor{DARCgreen}{5/8$~\lambda$}}}
{$\lambda$/4}
{$\lambda$/2}
{3/4$~\lambda$}
\end{QQuestion}

}
\end{frame}%ENDCONTENT


\section{Verkürzungsfaktor II}
\label{section:verkuerzungsfaktor_2}
\begin{frame}%STARTCONTENT

\only<1>{
\begin{QQuestion}{AG202}{Warum muss eine Antenne mechanisch etwas kürzer als der theoretisch errechnete Wert sein?}{Weil sich durch die mechanische Verkürzung die elektromagnetischen Wellen leichter von der Antenne ablösen. Dadurch steigt der Wirkungsgrad.}
{Weil sich diese Antenne nicht im idealen freien Raum befindet und weil die Antennenelemente nicht die Idealform des Kugelstrahlers besitzen. Kapazitive Einflüsse der Umgebung und die Abweichung von der idealen Kugelform verlängern die Antenne elektrisch.}
{Weil sich diese Antenne nicht im idealen freien Raum befindet und weil sie nicht unendlich dünn ist. Kapazitive Einflüsse der Umgebung und der Durchmesser des Strahlers verlängern die Antenne elektrisch.}
{Weil sich durch die mechanische Verkürzung der Verlustwiderstand eines Antennenstabes verringert. Dadurch steigt der Wirkungsgrad.}
\end{QQuestion}

}
\only<2>{
\begin{QQuestion}{AG202}{Warum muss eine Antenne mechanisch etwas kürzer als der theoretisch errechnete Wert sein?}{Weil sich durch die mechanische Verkürzung die elektromagnetischen Wellen leichter von der Antenne ablösen. Dadurch steigt der Wirkungsgrad.}
{Weil sich diese Antenne nicht im idealen freien Raum befindet und weil die Antennenelemente nicht die Idealform des Kugelstrahlers besitzen. Kapazitive Einflüsse der Umgebung und die Abweichung von der idealen Kugelform verlängern die Antenne elektrisch.}
{\textbf{\textcolor{DARCgreen}{Weil sich diese Antenne nicht im idealen freien Raum befindet und weil sie nicht unendlich dünn ist. Kapazitive Einflüsse der Umgebung und der Durchmesser des Strahlers verlängern die Antenne elektrisch.}}}
{Weil sich durch die mechanische Verkürzung der Verlustwiderstand eines Antennenstabes verringert. Dadurch steigt der Wirkungsgrad.}
\end{QQuestion}

}
\end{frame}

\begin{frame}
\only<1>{
\begin{QQuestion}{AG313}{Der Verkürzungsfaktor einer luftisolierten Paralleldrahtleitung ist~...}{unbestimmt.}
{0{,}1.}
{0{,}66.}
{ungefähr~1.}
\end{QQuestion}

}
\only<2>{
\begin{QQuestion}{AG313}{Der Verkürzungsfaktor einer luftisolierten Paralleldrahtleitung ist~...}{unbestimmt.}
{0{,}1.}
{0{,}66.}
{\textbf{\textcolor{DARCgreen}{ungefähr~1.}}}
\end{QQuestion}

}
\end{frame}

\begin{frame}
\only<1>{
\begin{QQuestion}{AG315}{Der Verkürzungsfaktor eines Koaxialkabels mit einem Dielektrikum aus massivem Polyethylen beträgt ungefähr~...}{1{,}0.}
{0{,}1.}
{0{,}8.}
{0{,}66.}
\end{QQuestion}

}
\only<2>{
\begin{QQuestion}{AG315}{Der Verkürzungsfaktor eines Koaxialkabels mit einem Dielektrikum aus massivem Polyethylen beträgt ungefähr~...}{1{,}0.}
{0{,}1.}
{0{,}8.}
{\textbf{\textcolor{DARCgreen}{0{,}66.}}}
\end{QQuestion}

}
\end{frame}

\begin{frame}
\only<1>{
\begin{QQuestion}{AG101}{Eine $\lambda$/2-Dipol-Antenne soll für \qty{14,2}{\MHz} aus Draht gefertigt werden. Es soll mit einem Verkürzungsfaktor von \num{0,95} gerechnet werden. Wie lang müssen die beiden Drähte der Dipol-Antenne jeweils sein?}{Je \qty{10,03}{\m}}
{Je \qty{10,56}{\m}}
{Je \qty{5,02}{\m}}
{Je \qty{5,28}{\m}}
\end{QQuestion}

}
\only<2>{
\begin{QQuestion}{AG101}{Eine $\lambda$/2-Dipol-Antenne soll für \qty{14,2}{\MHz} aus Draht gefertigt werden. Es soll mit einem Verkürzungsfaktor von \num{0,95} gerechnet werden. Wie lang müssen die beiden Drähte der Dipol-Antenne jeweils sein?}{Je \qty{10,03}{\m}}
{Je \qty{10,56}{\m}}
{\textbf{\textcolor{DARCgreen}{Je \qty{5,02}{\m}}}}
{Je \qty{5,28}{\m}}
\end{QQuestion}

}
\end{frame}

\begin{frame}
\frametitle{Lösungsweg}
\begin{itemize}
  \item gegeben: $f = 14,2MHz$
  \item gegeben: $k_v = 0,95$
  \item gegeben: $\frac{\lambda}{2}$-Dipol
  \item gesucht: $l_G$
  \end{itemize}
    \pause
    $l_E = \frac{1}{2} \cdot \frac{\lambda}{2} = \frac{1}{4} \cdot \frac{c}{f} = \frac{1}{4} \cdot \frac{3\cdot 10^8\frac{m}{s}}{14,2MHz} = \frac{1}{4} \cdot 21,13m = 5,28m$
    \pause
    $k_v = \frac{l_G}{l_E} \Rightarrow l_G = k_v \cdot l_E = 0,95 \cdot 5,28m = 5,02m$



\end{frame}

\begin{frame}
\only<1>{
\begin{QQuestion}{AG102}{Eine $\lambda$/2-Dipol-Antenne soll für \qty{7,1}{\MHz} aus Draht gefertigt werden. Wie lang müssen die beiden Drähte der Dipol-Antenne jeweils sein? Es soll hier mit einem Verkürzungsfaktor von \num{0,95} gerechnet werden.}{Je \qty{21,13}{\m}}
{Je \qty{10,56}{\m}}
{Je \qty{20,07}{\m}}
{Je \qty{10,04}{\m}}
\end{QQuestion}

}
\only<2>{
\begin{QQuestion}{AG102}{Eine $\lambda$/2-Dipol-Antenne soll für \qty{7,1}{\MHz} aus Draht gefertigt werden. Wie lang müssen die beiden Drähte der Dipol-Antenne jeweils sein? Es soll hier mit einem Verkürzungsfaktor von \num{0,95} gerechnet werden.}{Je \qty{21,13}{\m}}
{Je \qty{10,56}{\m}}
{Je \qty{20,07}{\m}}
{\textbf{\textcolor{DARCgreen}{Je \qty{10,04}{\m}}}}
\end{QQuestion}

}
\end{frame}

\begin{frame}
\frametitle{Lösungsweg}
\begin{itemize}
  \item gegeben: $f = 7,1MHz$
  \item gegeben: $k_v = 0,95$
  \item gegeben: $\frac{\lambda}{2}$-Dipol
  \item gesucht: $l_G$
  \end{itemize}
    \pause
    $l_E = \frac{1}{2} \cdot \frac{\lambda}{2} = \frac{1}{4} \cdot \frac{c}{f} = \frac{1}{4} \cdot \frac{3\cdot 10^8\frac{m}{s}}{7,1MHz} = \frac{1}{4} \cdot 42,25m = 10,56m$
    \pause
    $k_v = \frac{l_G}{l_E} \Rightarrow l_G = k_v \cdot l_E = 0,95 \cdot 10,56m = 10,04m$



\end{frame}

\begin{frame}
\only<1>{
\begin{QQuestion}{AG103}{Ein Drahtdipol hat eine Gesamtlänge von \qty{20}{\m}. Für welche Frequenz ist der Dipol in Resonanz, wenn mit einem Verkürzungsfaktor von \num{0,95} gerechnet wird?}{\qty{7,125}{\MHz}}
{\qty{6,768}{\MHz}}
{\qty{7,500}{\MHz}}
{\qty{7,000}{\MHz}}
\end{QQuestion}

}
\only<2>{
\begin{QQuestion}{AG103}{Ein Drahtdipol hat eine Gesamtlänge von \qty{20}{\m}. Für welche Frequenz ist der Dipol in Resonanz, wenn mit einem Verkürzungsfaktor von \num{0,95} gerechnet wird?}{\textbf{\textcolor{DARCgreen}{\qty{7,125}{\MHz}}}}
{\qty{6,768}{\MHz}}
{\qty{7,500}{\MHz}}
{\qty{7,000}{\MHz}}
\end{QQuestion}

}
\end{frame}

\begin{frame}
\frametitle{Lösungsweg}
\begin{itemize}
  \item gegeben: $l_G = 20m$
  \item gegeben: $k_v = 0,95$
  \item gegeben: Dipol
  \item gesucht: $f$
  \end{itemize}
    \pause
    $k_v = \frac{l_G}{l_E} \Rightarrow l_E = \frac{l_G}{k_v} = \frac{20m}{0,95} = 21,05m$
    \pause
    $l_E = \frac{\lambda}{2} = \frac{1}{2} \cdot \frac{c}{f} \Rightarrow f = \frac{1}{2} \cdot {c}{l_E} = 7,125MHz$



\end{frame}

\begin{frame}
\only<1>{
\begin{QQuestion}{AG104}{Eine $\lambda$/4-Groundplane-Antenne mit vier Radials soll für \qty{7,1}{\MHz} aus Drähten gefertigt werden. Für Strahlerelement und Radials kann mit einem Verkürzungsfaktor von \num{0,95} gerechnet werden. Wie lang müssen Strahlerelement und Radials jeweils sein?}{Strahlerelement:~\qty{20,06}{\m}, Radials:~je~\qty{20,06}{\m}}
{Strahlerelement:~\qty{21,13}{\m}, Radials:~je~\qty{21,13}{\m}}
{Strahlerelement:~\qty{10,56}{\m}, Radials:~je~\qty{10,56}{\m}}
{Strahlerelement:~\qty{10,04}{\m}, Radials:~je~\qty{10,04}{\m}}
\end{QQuestion}

}
\only<2>{
\begin{QQuestion}{AG104}{Eine $\lambda$/4-Groundplane-Antenne mit vier Radials soll für \qty{7,1}{\MHz} aus Drähten gefertigt werden. Für Strahlerelement und Radials kann mit einem Verkürzungsfaktor von \num{0,95} gerechnet werden. Wie lang müssen Strahlerelement und Radials jeweils sein?}{Strahlerelement:~\qty{20,06}{\m}, Radials:~je~\qty{20,06}{\m}}
{Strahlerelement:~\qty{21,13}{\m}, Radials:~je~\qty{21,13}{\m}}
{Strahlerelement:~\qty{10,56}{\m}, Radials:~je~\qty{10,56}{\m}}
{\textbf{\textcolor{DARCgreen}{Strahlerelement:~\qty{10,04}{\m}, Radials:~je~\qty{10,04}{\m}}}}
\end{QQuestion}

}
\end{frame}

\begin{frame}
\frametitle{Lösungsweg}
\begin{itemize}
  \item gegeben: $f = 7,1MHz$
  \item gegeben: $k_v = 0,95$
  \item gegeben: $\frac{\lambda}{4}$-Groundplane
  \item gesucht: $l_G$
  \end{itemize}
    \pause
    $l_E = \frac{\lambda}{4} = \frac{1}{4} \cdot \frac{c}{f} = \frac{1}{4} \cdot \frac{3\cdot 10^8\frac{m}{s}}{7,1MHz} = \frac{1}{4} \cdot 42,25m = 10,56m$
    \pause
    $k_v = \frac{l_G}{l_E} \Rightarrow l_G = k_v \cdot l_E = 0,95 \cdot 10,56m = 10,04m$



\end{frame}

\begin{frame}
\only<1>{
\begin{QQuestion}{AG105}{Eine 5/8-$\lambda$-Vertikalantenne soll für \qty{14,2}{\MHz} aus Draht hergestellt werden. Es soll mit einem Verkürzungsfaktor von \num{0,97} gerechnet werden. Wie lang muss der Draht insgesamt sein?}{\qty{10,03}{\m}}
{\qty{13,20}{\m}}
{\qty{12,80}{\m}}
{\qty{13,61}{\m}}
\end{QQuestion}

}
\only<2>{
\begin{QQuestion}{AG105}{Eine 5/8-$\lambda$-Vertikalantenne soll für \qty{14,2}{\MHz} aus Draht hergestellt werden. Es soll mit einem Verkürzungsfaktor von \num{0,97} gerechnet werden. Wie lang muss der Draht insgesamt sein?}{\qty{10,03}{\m}}
{\qty{13,20}{\m}}
{\textbf{\textcolor{DARCgreen}{\qty{12,80}{\m}}}}
{\qty{13,61}{\m}}
\end{QQuestion}

}
\end{frame}

\begin{frame}
\frametitle{Lösungsweg}
\begin{itemize}
  \item gegeben: $f = 14,2MHz$
  \item gegeben: $k_v = 0,97$
  \item gegeben: $\frac{5}{8}\lambda$-Vertikalantenne
  \item gesucht: $l_G$
  \end{itemize}
    \pause
    $l_E = \frac{5}{8}\lambda = \frac{5}{8} \cdot \frac{c}{f} = \frac{5}{8} \cdot \frac{3\cdot 10^8\frac{m}{s}}{14,2MHz} = \frac{5}{8} \cdot 21,13 = 13,20m$
    \pause
    $k_v = \frac{l_G}{l_E} \Rightarrow l_G = k_v \cdot l_E = 0,97 \cdot 13,20m = 12,80m$



\end{frame}

\begin{frame}
\only<1>{
\begin{QQuestion}{AG118}{Eine Delta-Loop-Antenne mit einer vollen Wellenlänge soll für \qty{7,1}{\MHz} aus Draht hergestellt werden. Es soll mit einem Korrekturfaktor von \num{1,02} gerechnet werden. Wie lang muss der Draht insgesamt sein?}{\qty{21,12}{\m}}
{\qty{42,25}{\m}}
{\qty{21,55}{\m}}
{\qty{43,10}{\m}}
\end{QQuestion}

}
\only<2>{
\begin{QQuestion}{AG118}{Eine Delta-Loop-Antenne mit einer vollen Wellenlänge soll für \qty{7,1}{\MHz} aus Draht hergestellt werden. Es soll mit einem Korrekturfaktor von \num{1,02} gerechnet werden. Wie lang muss der Draht insgesamt sein?}{\qty{21,12}{\m}}
{\qty{42,25}{\m}}
{\qty{21,55}{\m}}
{\textbf{\textcolor{DARCgreen}{\qty{43,10}{\m}}}}
\end{QQuestion}

}
\end{frame}

\begin{frame}
\frametitle{Lösungsweg}
\begin{itemize}
  \item gegeben: $f = 7,1MHz$
  \item gegeben: $k_v = 1,02$
  \item gegeben: Delta-Loop
  \item gesucht: $l_G$
  \end{itemize}
    \pause
    $l_E = \lambda = \frac{c}{f} = \frac{3\cdot 10^8\frac{m}{s}}{7,1MHz} = 42,23m$
    \pause
    $k_v = \frac{l_G}{l_E} \Rightarrow l_G = k_v \cdot l_E = 1,02 \cdot 42,23m = 43,10m$



\end{frame}

\begin{frame}
\only<1>{
\begin{QQuestion}{AG316}{Wie lang ist ein Koaxialkabel, das für eine ganze Wellenlänge bei \qty{145}{\MHz} zugeschnitten wurde, wenn der Verkürzungsfaktor 0,66 beträgt?}{\qty{0,68}{\m}}
{\qty{2,07}{\m}}
{\qty{1,37}{\m}}
{\qty{2,72}{\m}}
\end{QQuestion}

}
\only<2>{
\begin{QQuestion}{AG316}{Wie lang ist ein Koaxialkabel, das für eine ganze Wellenlänge bei \qty{145}{\MHz} zugeschnitten wurde, wenn der Verkürzungsfaktor 0,66 beträgt?}{\qty{0,68}{\m}}
{\qty{2,07}{\m}}
{\textbf{\textcolor{DARCgreen}{\qty{1,37}{\m}}}}
{\qty{2,72}{\m}}
\end{QQuestion}

}
\end{frame}

\begin{frame}
\frametitle{Lösungsweg}
\begin{itemize}
  \item gegeben: $f = 145MHz$
  \item gegeben: $k_v = 0,66$
  \item gesucht: $l_G$
  \end{itemize}
    \pause
    $l_E = \lambda = \frac{c}{f} = \frac{3\cdot 10^8\frac{m}{s}}{144MHz} = 2,08m$
    \pause
    $k_v = \frac{l_G}{l_E} \Rightarrow l_G = k_v \cdot l_E = 0,66 \cdot 2,08m = 1,37m$



\end{frame}%ENDCONTENT


\section{Fußpunktimpedanz II}
\label{section:fusspunktimpedanz_2}
\begin{frame}%STARTCONTENT

\only<1>{
\begin{QQuestion}{AG211}{Welchen Eingangs- bzw. Fußpunktwiderstand hat ein $\lambda$/2-Dipol in ungefähr einer Wellenlänge Höhe über dem Boden bei seiner Grundfrequenz?}{ca.~65~bis~\qty{75}{\ohm}}
{ca.~\qty{30}{\ohm}}
{ca.~\qty{120}{\ohm}}
{ca.~240~bis~\qty{300}{\ohm}}
\end{QQuestion}

}
\only<2>{
\begin{QQuestion}{AG211}{Welchen Eingangs- bzw. Fußpunktwiderstand hat ein $\lambda$/2-Dipol in ungefähr einer Wellenlänge Höhe über dem Boden bei seiner Grundfrequenz?}{\textbf{\textcolor{DARCgreen}{ca.~65~bis~\qty{75}{\ohm}}}}
{ca.~\qty{30}{\ohm}}
{ca.~\qty{120}{\ohm}}
{ca.~240~bis~\qty{300}{\ohm}}
\end{QQuestion}

}
\end{frame}

\begin{frame}
\only<1>{
\begin{QQuestion}{AG209}{Der Fusspunktwiderstand eines mittengespeisten $\lambda$/2-Dipols zeigt sich bei dessen Resonanzfrequenzen~...}{abwechselnd als kapazitiver oder induktiver Blindwiderstand.}
{im Wesentlichen als kapazitiver Blindwiderstand.}
{im Wesentlichen als induktiver Blindwiderstand.}
{im Wesentlichen als Wirkwiderstand.}
\end{QQuestion}

}
\only<2>{
\begin{QQuestion}{AG209}{Der Fusspunktwiderstand eines mittengespeisten $\lambda$/2-Dipols zeigt sich bei dessen Resonanzfrequenzen~...}{abwechselnd als kapazitiver oder induktiver Blindwiderstand.}
{im Wesentlichen als kapazitiver Blindwiderstand.}
{im Wesentlichen als induktiver Blindwiderstand.}
{\textbf{\textcolor{DARCgreen}{im Wesentlichen als Wirkwiderstand.}}}
\end{QQuestion}

}
\end{frame}

\begin{frame}
\only<1>{
\begin{QQuestion}{AG210}{Welche Fußpunktimpedanz hat ein $\lambda$/2-Dipol unterhalb und oberhalb seiner Grundfrequenz?}{Unterhalb der Grundfrequenz ist die Impedanz kapazitiv, oberhalb induktiv.}
{Unterhalb der Grundfrequenz ist die Impedanz induktiv, oberhalb kapazitiv.}
{Unterhalb der Grundfrequenz ist die Impedanz niedriger, oberhalb höher.}
{Unterhalb der Grundfrequenz ist die Impedanz höher, oberhalb niedriger.}
\end{QQuestion}

}
\only<2>{
\begin{QQuestion}{AG210}{Welche Fußpunktimpedanz hat ein $\lambda$/2-Dipol unterhalb und oberhalb seiner Grundfrequenz?}{\textbf{\textcolor{DARCgreen}{Unterhalb der Grundfrequenz ist die Impedanz kapazitiv, oberhalb induktiv.}}}
{Unterhalb der Grundfrequenz ist die Impedanz induktiv, oberhalb kapazitiv.}
{Unterhalb der Grundfrequenz ist die Impedanz niedriger, oberhalb höher.}
{Unterhalb der Grundfrequenz ist die Impedanz höher, oberhalb niedriger.}
\end{QQuestion}

}
\end{frame}%ENDCONTENT


\section{Elektrische Verlängerung und Verkürzung}
\label{section:elektrische_verlaengerung_verkuerzung}
\begin{frame}%STARTCONTENT

\only<1>{
\begin{PQuestion}{AG106}{Wozu dient die Spule in dieser Antenne?}{Verringerung der Ausbreitungsgeschwindigkeit}
{Elektrische Verkürzung des Strahlers}
{Erhöhung der Ausbreitungsgeschwindigkeit}
{Elektrische Verlängerung des Strahlers}
{\DARCimage{1.0\linewidth}{650include}}\end{PQuestion}

}
\only<2>{
\begin{PQuestion}{AG106}{Wozu dient die Spule in dieser Antenne?}{Verringerung der Ausbreitungsgeschwindigkeit}
{Elektrische Verkürzung des Strahlers}
{Erhöhung der Ausbreitungsgeschwindigkeit}
{\textbf{\textcolor{DARCgreen}{Elektrische Verlängerung des Strahlers}}}
{\DARCimage{1.0\linewidth}{650include}}\end{PQuestion}

}
\end{frame}

\begin{frame}
\only<1>{
\begin{PQuestion}{AG107}{Wozu dient der Kondensator in dieser Antenne?}{Verringerung der Ausbreitungsgeschwindigkeit}
{Elektrische Verlängerung des Strahlers}
{Erhöhung der Ausbreitungsgeschwindigkeit}
{Elektrische Verkürzung des Strahlers}
{\DARCimage{1.0\linewidth}{563include}}\end{PQuestion}

}
\only<2>{
\begin{PQuestion}{AG107}{Wozu dient der Kondensator in dieser Antenne?}{Verringerung der Ausbreitungsgeschwindigkeit}
{Elektrische Verlängerung des Strahlers}
{Erhöhung der Ausbreitungsgeschwindigkeit}
{\textbf{\textcolor{DARCgreen}{Elektrische Verkürzung des Strahlers}}}
{\DARCimage{1.0\linewidth}{563include}}\end{PQuestion}

}
\end{frame}

\begin{frame}
\only<1>{
\begin{QQuestion}{AG108}{Was sollte in jeden Schenkel einer symmetrischen, zweimal \qty{15}{\m} langen Dipol-Antenne eingefügt werden, damit die Antenne im Bereich um \qty{3,6}{\MHz} resonant wird?}{Eine Spule}
{Ein Parallelkreis mit einer Resonanzfrequenz von \qty{3,6}{\MHz}}
{Ein Kondensator}
{Ein RC-Glied}
\end{QQuestion}

}
\only<2>{
\begin{QQuestion}{AG108}{Was sollte in jeden Schenkel einer symmetrischen, zweimal \qty{15}{\m} langen Dipol-Antenne eingefügt werden, damit die Antenne im Bereich um \qty{3,6}{\MHz} resonant wird?}{\textbf{\textcolor{DARCgreen}{Eine Spule}}}
{Ein Parallelkreis mit einer Resonanzfrequenz von \qty{3,6}{\MHz}}
{Ein Kondensator}
{Ein RC-Glied}
\end{QQuestion}

}
\end{frame}%ENDCONTENT


\section{Near Vertical Incidence Skywave (NVIS)}
\label{section:nvis}
\begin{frame}%STARTCONTENT

\only<1>{
\begin{QQuestion}{AG125}{Welche Antennen sind für NVIS-Ausbreitung (Near Vertical Incident Skywave), wie sie für Notfunk-Verbindungen im KW-Bereich benutzt werden, gut geeignet?}{Eine Vertikalantenne einer Gesamtlänge zwischen \num{0,5} und \num{0,625} (5/8) Wellenlängen über gutem Radialnetz.}
{Mit Drähten aufgebauter horizontaler Faltdipol in möglichst genau \num{0,8} Wellenlängen Höhe über Grund.}
{Als \glqq Inverted-V\grqq{} aufgespannte Drähte mit einem Speisepunkt in mindestens einer Wellenlänge Höhe über Grund.}
{Horizontal aufgespannte Drähte in einer Höhe von höchstens \num{0,25} Wellenlängen über Grund.}
\end{QQuestion}

}
\only<2>{
\begin{QQuestion}{AG125}{Welche Antennen sind für NVIS-Ausbreitung (Near Vertical Incident Skywave), wie sie für Notfunk-Verbindungen im KW-Bereich benutzt werden, gut geeignet?}{Eine Vertikalantenne einer Gesamtlänge zwischen \num{0,5} und \num{0,625} (5/8) Wellenlängen über gutem Radialnetz.}
{Mit Drähten aufgebauter horizontaler Faltdipol in möglichst genau \num{0,8} Wellenlängen Höhe über Grund.}
{Als \glqq Inverted-V\grqq{} aufgespannte Drähte mit einem Speisepunkt in mindestens einer Wellenlänge Höhe über Grund.}
{\textbf{\textcolor{DARCgreen}{Horizontal aufgespannte Drähte in einer Höhe von höchstens \num{0,25} Wellenlängen über Grund.}}}
\end{QQuestion}

}
\end{frame}

\begin{frame}
\only<1>{
\begin{QQuestion}{AG224}{Welche Eigenschaften besitzt eine in geringer Höhe aufgebaute, auf Kurzwelle betriebene NVIS-Antenne (Near Vertical Incident Skywave)?}{Sie vergrößert durch ihre flache Abstrahlung den Bereich der Bodenwelle.}
{Sie ermöglicht durch annähernd senkrechte Abstrahlung eine Raumwellenausbreitung ohne tote Zone um den Sendeort herum.}
{Ihre senkrechte Abstrahlung bringt die D-Region zum Verschwinden, so dass die Tagesdämpfung über dem Sendeort lokal aufgehoben wird.}
{Sie erzeugt mit ihrer Reflexion am nahen Erdboden eine zirkular polarisierte Abstrahlung, die Fading reduziert.}
\end{QQuestion}

}
\only<2>{
\begin{QQuestion}{AG224}{Welche Eigenschaften besitzt eine in geringer Höhe aufgebaute, auf Kurzwelle betriebene NVIS-Antenne (Near Vertical Incident Skywave)?}{Sie vergrößert durch ihre flache Abstrahlung den Bereich der Bodenwelle.}
{\textbf{\textcolor{DARCgreen}{Sie ermöglicht durch annähernd senkrechte Abstrahlung eine Raumwellenausbreitung ohne tote Zone um den Sendeort herum.}}}
{Ihre senkrechte Abstrahlung bringt die D-Region zum Verschwinden, so dass die Tagesdämpfung über dem Sendeort lokal aufgehoben wird.}
{Sie erzeugt mit ihrer Reflexion am nahen Erdboden eine zirkular polarisierte Abstrahlung, die Fading reduziert.}
\end{QQuestion}

}
\end{frame}%ENDCONTENT


\section{Traps}
\label{section:traps}
\begin{frame}%STARTCONTENT

\only<1>{
\begin{PQuestion}{AG109}{Welche Antennenart ist hier dargestellt?  }{Sperrkreis-Dipol}
{Einband-Dipol mit Oberwellenfilter}
{Dipol mit Gleichwellenfilter}
{Saugkreis-Dipol}
{\DARCimage{1.0\linewidth}{234include}}\end{PQuestion}

}
\only<2>{
\begin{PQuestion}{AG109}{Welche Antennenart ist hier dargestellt?  }{\textbf{\textcolor{DARCgreen}{Sperrkreis-Dipol}}}
{Einband-Dipol mit Oberwellenfilter}
{Dipol mit Gleichwellenfilter}
{Saugkreis-Dipol}
{\DARCimage{1.0\linewidth}{234include}}\end{PQuestion}

}
\end{frame}

\begin{frame}
\only<1>{
\begin{QQuestion}{AG110}{Ein Parallelresonanzkreis (Trap) in jeder Dipolhälfte~...}{erhöht die effiziente Nutzung des jeweiligen Frequenzbereichs.}
{erlaubt eine Nutzung der Antenne für mindestens zwei Frequenzbereiche.}
{beschränkt die Nutzbarkeit der Antenne auf einen Frequenzbereich.}
{ermöglicht die Unterdrückung der Harmonischen.}
\end{QQuestion}

}
\only<2>{
\begin{QQuestion}{AG110}{Ein Parallelresonanzkreis (Trap) in jeder Dipolhälfte~...}{erhöht die effiziente Nutzung des jeweiligen Frequenzbereichs.}
{\textbf{\textcolor{DARCgreen}{erlaubt eine Nutzung der Antenne für mindestens zwei Frequenzbereiche.}}}
{beschränkt die Nutzbarkeit der Antenne auf einen Frequenzbereich.}
{ermöglicht die Unterdrückung der Harmonischen.}
\end{QQuestion}

}
\end{frame}

\begin{frame}
\only<1>{
\begin{PQuestion}{AG113}{Wenn man diese Mehrband-Antenne auf \qty{14}{\MHz} erregt, dann wirken die LC-Resonanzkreise~...}{als Vergrößerung des Strahlungswiderstands der Antenne.}
{als Sperrkreise für die Erregerfrequenz.}
{als induktive Verlängerung des Strahlers.}
{als kapazitive Verkürzung des Strahlers.}
{\DARCimage{1.0\linewidth}{235include}}\end{PQuestion}

}
\only<2>{
\begin{PQuestion}{AG113}{Wenn man diese Mehrband-Antenne auf \qty{14}{\MHz} erregt, dann wirken die LC-Resonanzkreise~...}{als Vergrößerung des Strahlungswiderstands der Antenne.}
{als Sperrkreise für die Erregerfrequenz.}
{als induktive Verlängerung des Strahlers.}
{\textbf{\textcolor{DARCgreen}{als kapazitive Verkürzung des Strahlers.}}}
{\DARCimage{1.0\linewidth}{235include}}\end{PQuestion}

}
\end{frame}

\begin{frame}
\only<1>{
\begin{PQuestion}{AG112}{Wenn man diese Mehrband-Antenne auf \qty{7}{\MHz} erregt, dann wirken die LC-Resonanzkreise~...}{als Vergrößerung des Strahlungswiderstands der Antenne.}
{als induktive Verlängerung des Strahlers.}
{als kapazitive Verkürzung des Strahlers.}
{als Sperrkreise für die Erregerfrequenz.}
{\DARCimage{1.0\linewidth}{235include}}\end{PQuestion}

}
\only<2>{
\begin{PQuestion}{AG112}{Wenn man diese Mehrband-Antenne auf \qty{7}{\MHz} erregt, dann wirken die LC-Resonanzkreise~...}{als Vergrößerung des Strahlungswiderstands der Antenne.}
{als induktive Verlängerung des Strahlers.}
{als kapazitive Verkürzung des Strahlers.}
{\textbf{\textcolor{DARCgreen}{als Sperrkreise für die Erregerfrequenz.}}}
{\DARCimage{1.0\linewidth}{235include}}\end{PQuestion}

}
\end{frame}

\begin{frame}
\only<1>{
\begin{PQuestion}{AG116}{Sie wollen eine Zweibandantenne für \qty{160}{\m} und \qty{80}{\m} selbst bauen. Welche der folgenden Antworten enthält die richtige Drahtlänge $l$ zwischen den Traps und die richtige Resonanzfrequenz $f_{\symup{res}}$ der Schwingkreise?}{$l$ beträgt zirka \qty{80}{\m}, $f_{\symup{res}}$ liegt bei zirka \qty{3,65}{\MHz}.}
{$l$ beträgt zirka \qty{40}{\m}, $f_{\symup{res}}$ liegt bei zirka \qty{3,65}{\MHz}.}
{$l$ beträgt zirka \qty{40}{\m}, $f_{\symup{res}}$ liegt bei zirka \qty{1,85}{\MHz}.}
{$l$ beträgt zirka \qty{80}{\m}, $f_{\symup{res}}$ liegt bei zirka \qty{1,85}{\MHz}.}
{\DARCimage{1.0\linewidth}{236include}}\end{PQuestion}

}
\only<2>{
\begin{PQuestion}{AG116}{Sie wollen eine Zweibandantenne für \qty{160}{\m} und \qty{80}{\m} selbst bauen. Welche der folgenden Antworten enthält die richtige Drahtlänge $l$ zwischen den Traps und die richtige Resonanzfrequenz $f_{\symup{res}}$ der Schwingkreise?}{$l$ beträgt zirka \qty{80}{\m}, $f_{\symup{res}}$ liegt bei zirka \qty{3,65}{\MHz}.}
{\textbf{\textcolor{DARCgreen}{$l$ beträgt zirka \qty{40}{\m}, $f_{\symup{res}}$ liegt bei zirka \qty{3,65}{\MHz}.}}}
{$l$ beträgt zirka \qty{40}{\m}, $f_{\symup{res}}$ liegt bei zirka \qty{1,85}{\MHz}.}
{$l$ beträgt zirka \qty{80}{\m}, $f_{\symup{res}}$ liegt bei zirka \qty{1,85}{\MHz}.}
{\DARCimage{1.0\linewidth}{236include}}\end{PQuestion}

}
\end{frame}

\begin{frame}
\only<1>{
\begin{PQuestion}{AG111}{Wenn man diese Mehrband-Antenne auf \qty{3,5}{\MHz} erregt, dann wirken die LC-Resonanzkreise~...}{als Vergrößerung des Strahlungswiderstands der Antenne.}
{als Sperrkreise für die Erregerfrequenz.}
{als kapazitive Verkürzung des Strahlers.}
{als induktive Verlängerung des Strahlers.}
{\DARCimage{1.0\linewidth}{235include}}\end{PQuestion}

}
\only<2>{
\begin{PQuestion}{AG111}{Wenn man diese Mehrband-Antenne auf \qty{3,5}{\MHz} erregt, dann wirken die LC-Resonanzkreise~...}{als Vergrößerung des Strahlungswiderstands der Antenne.}
{als Sperrkreise für die Erregerfrequenz.}
{als kapazitive Verkürzung des Strahlers.}
{\textbf{\textcolor{DARCgreen}{als induktive Verlängerung des Strahlers.}}}
{\DARCimage{1.0\linewidth}{235include}}\end{PQuestion}

}
\end{frame}

\begin{frame}
\only<1>{
\begin{PQuestion}{AG115}{Das folgende Bild stellt einen Dreiband-Dipol für die Frequenzbänder \num{20}, \num{15} und \qty{10}{\m} dar. Die mit b gekennzeichneten Schwingkreise sind abgestimmt auf:}{\qty{21,2}{\MHz}}
{\qty{10,1}{\MHz}}
{\qty{14,2}{\MHz}}
{\qty{29,0}{\MHz}}
{\DARCimage{1.0\linewidth}{237include}}\end{PQuestion}

}
\only<2>{
\begin{PQuestion}{AG115}{Das folgende Bild stellt einen Dreiband-Dipol für die Frequenzbänder \num{20}, \num{15} und \qty{10}{\m} dar. Die mit b gekennzeichneten Schwingkreise sind abgestimmt auf:}{\qty{21,2}{\MHz}}
{\qty{10,1}{\MHz}}
{\qty{14,2}{\MHz}}
{\textbf{\textcolor{DARCgreen}{\qty{29,0}{\MHz}}}}
{\DARCimage{1.0\linewidth}{237include}}\end{PQuestion}

}
\end{frame}

\begin{frame}
\only<1>{
\begin{PQuestion}{AG114}{Das folgende Bild stellt einen Dreiband-Dipol für die Frequenzbänder \num{20}, \num{15} und \qty{10}{\m} dar. Die mit a gekennzeichneten Schwingkreise sind abgestimmt auf:}{\qty{10,1}{\MHz}}
{\qty{21,2}{\MHz}}
{\qty{14,2}{\MHz}}
{\qty{29,0}{\MHz}}
{\DARCimage{1.0\linewidth}{237include}}\end{PQuestion}

}
\only<2>{
\begin{PQuestion}{AG114}{Das folgende Bild stellt einen Dreiband-Dipol für die Frequenzbänder \num{20}, \num{15} und \qty{10}{\m} dar. Die mit a gekennzeichneten Schwingkreise sind abgestimmt auf:}{\qty{10,1}{\MHz}}
{\textbf{\textcolor{DARCgreen}{\qty{21,2}{\MHz}}}}
{\qty{14,2}{\MHz}}
{\qty{29,0}{\MHz}}
{\DARCimage{1.0\linewidth}{237include}}\end{PQuestion}

}
\end{frame}%ENDCONTENT


\section{Yagi-Uda-Antenne III}
\label{section:yagi_uda_3}
\begin{frame}%STARTCONTENT

\only<1>{
\begin{QQuestion}{AG212}{Die Impedanz des Strahlers eines Kurzwellenbeams richtet sich auch nach~...}{den Abständen zwischen Reflektor, Strahler und den Direktoren.}
{dem Strahlungswiderstand des Reflektors.}
{dem Widerstand des Zuführungskabels.}
{den Ausbreitungsbedingungen.}
\end{QQuestion}

}
\only<2>{
\begin{QQuestion}{AG212}{Die Impedanz des Strahlers eines Kurzwellenbeams richtet sich auch nach~...}{\textbf{\textcolor{DARCgreen}{den Abständen zwischen Reflektor, Strahler und den Direktoren.}}}
{dem Strahlungswiderstand des Reflektors.}
{dem Widerstand des Zuführungskabels.}
{den Ausbreitungsbedingungen.}
\end{QQuestion}

}
\end{frame}

\begin{frame}
\only<1>{
\begin{QQuestion}{AG222}{Worin unterscheidet sich eine Yagi-Uda-Antenne mit 11 Elementen von einer mit 3 Elementen? Bei der Antenne mit 11 Elementen ist~...}{der Strahlungswiderstand erhöht.}
{der Öffnungswinkel erhöht.}
{der Öffnungswinkel verringert.}
{das Vor-Rück-Verhältnis verringert.}
\end{QQuestion}

}
\only<2>{
\begin{QQuestion}{AG222}{Worin unterscheidet sich eine Yagi-Uda-Antenne mit 11 Elementen von einer mit 3 Elementen? Bei der Antenne mit 11 Elementen ist~...}{der Strahlungswiderstand erhöht.}
{der Öffnungswinkel erhöht.}
{\textbf{\textcolor{DARCgreen}{der Öffnungswinkel verringert.}}}
{das Vor-Rück-Verhältnis verringert.}
\end{QQuestion}

}
\end{frame}

\begin{frame}
\only<1>{
\begin{QQuestion}{AG126}{Für die Erzeugung von zirkularer Polarisation mit Yagi-Uda-Antennen wird eine horizontale und eine dazu um \qty{90}{\degree} um die Strahlungsachse gedrehte Yagi-Uda-Antenne zusammengeschaltet. Was ist dabei zu beachten, damit tatsächlich zirkulare Polarisation entsteht?}{Bei einer der Antennen muss die Welle um $\lambda$/4 verzögert werden. Dies kann entweder durch eine zusätzlich eingefügte Viertelwellen-Verzögerungsleitung oder durch mechanische \glqq Verschiebung\grqq{} beider Yagi-Uda-Antennen um $\lambda$/4 gegeneinander hergestellt werden.}
{Bei einer der Antennen muss die Welle um $\lambda$/2 verzögert werden. Dies kann entweder durch eine zusätzlich eingefügte $\lambda$/2-Verzögerungsleitung oder durch mechanische \glqq Verschiebung\grqq{} beider Yagi-Uda-Antennen um $\lambda$/2 gegeneinander hergestellt werden.}
{Die Zusammenschaltung der Antennen muss über eine Halbwellen-Lecherleitung erfolgen. Zur Anpassung an den Wellenwiderstand muss zwischen der Speiseleitung und den Antennen noch ein $\lambda$/4-Transformationsstück eingefügt werden.}
{Die kreuzförmig angeordneten Elemente der beiden Antennen sind um \qty{45}{\degree} zu verdrehen, so dass in der Draufsicht ein liegendes Kreuz gebildet wird. Die Antennen werden über Leitungsstücke gleicher Länge parallel geschaltet. Die Anpassung erfolgt mit einem Symmetrierglied.}
\end{QQuestion}

}
\only<2>{
\begin{QQuestion}{AG126}{Für die Erzeugung von zirkularer Polarisation mit Yagi-Uda-Antennen wird eine horizontale und eine dazu um \qty{90}{\degree} um die Strahlungsachse gedrehte Yagi-Uda-Antenne zusammengeschaltet. Was ist dabei zu beachten, damit tatsächlich zirkulare Polarisation entsteht?}{\textbf{\textcolor{DARCgreen}{Bei einer der Antennen muss die Welle um $\lambda$/4 verzögert werden. Dies kann entweder durch eine zusätzlich eingefügte Viertelwellen-Verzögerungsleitung oder durch mechanische \glqq Verschiebung\grqq{} beider Yagi-Uda-Antennen um $\lambda$/4 gegeneinander hergestellt werden.}}}
{Bei einer der Antennen muss die Welle um $\lambda$/2 verzögert werden. Dies kann entweder durch eine zusätzlich eingefügte $\lambda$/2-Verzögerungsleitung oder durch mechanische \glqq Verschiebung\grqq{} beider Yagi-Uda-Antennen um $\lambda$/2 gegeneinander hergestellt werden.}
{Die Zusammenschaltung der Antennen muss über eine Halbwellen-Lecherleitung erfolgen. Zur Anpassung an den Wellenwiderstand muss zwischen der Speiseleitung und den Antennen noch ein $\lambda$/4-Transformationsstück eingefügt werden.}
{Die kreuzförmig angeordneten Elemente der beiden Antennen sind um \qty{45}{\degree} zu verdrehen, so dass in der Draufsicht ein liegendes Kreuz gebildet wird. Die Antennen werden über Leitungsstücke gleicher Länge parallel geschaltet. Die Anpassung erfolgt mit einem Symmetrierglied.}
\end{QQuestion}

}
\end{frame}%ENDCONTENT


\section{Parabolspiegel II}
\label{section:parbolspiegel_2}
\begin{frame}%STARTCONTENT

\only<1>{
\begin{QQuestion}{AG225}{Welche Antennentypen kommen üblicherweise als Erregerantennen (Feed) in Parabolspiegeln für den Mikrowellenbereich zum Einsatz?}{Dipol, Helix, Hornantenne}
{Collinear, Helix, isotroper Strahler}
{Groundplane, Hornantenne, Ringdipol}
{Helix, Hornantenne, Sperrkreisdipol}
\end{QQuestion}

}
\only<2>{
\begin{QQuestion}{AG225}{Welche Antennentypen kommen üblicherweise als Erregerantennen (Feed) in Parabolspiegeln für den Mikrowellenbereich zum Einsatz?}{\textbf{\textcolor{DARCgreen}{Dipol, Helix, Hornantenne}}}
{Collinear, Helix, isotroper Strahler}
{Groundplane, Hornantenne, Ringdipol}
{Helix, Hornantenne, Sperrkreisdipol}
\end{QQuestion}

}
\end{frame}

\begin{frame}
\only<1>{
\begin{QQuestion}{AG226}{Wie hoch ist der Gewinn eines Parabolspiegels mit einem Durchmesser von \qty{30}{\cm} und mit einem Wirkungsgrad von $\eta_{\symup{eff}}$~=~1 bei einer Arbeitsfrequenz von \qty{5,7}{\GHz}?}{\qty{25,1}{\dBi}}
{\qty{12,5}{\dBi}}
{\qty{28,1}{\dBi}}
{\qty{16,8}{\dBi}}
\end{QQuestion}

}
\only<2>{
\begin{QQuestion}{AG226}{Wie hoch ist der Gewinn eines Parabolspiegels mit einem Durchmesser von \qty{30}{\cm} und mit einem Wirkungsgrad von $\eta_{\symup{eff}}$~=~1 bei einer Arbeitsfrequenz von \qty{5,7}{\GHz}?}{\textbf{\textcolor{DARCgreen}{\qty{25,1}{\dBi}}}}
{\qty{12,5}{\dBi}}
{\qty{28,1}{\dBi}}
{\qty{16,8}{\dBi}}
\end{QQuestion}

}
\end{frame}

\begin{frame}
\frametitle{Lösungsweg}
\begin{itemize}
  \item gegeben: $d = 30cm$
  \item gegeben: $\eta_{eff} = 1$
  \item gegeben: $f = 5,7GHz$
  \item gesucht: $g_i$
  \end{itemize}
    \pause
    $\lambda = \frac{c}{f} = \frac{3\cdot 10^8\frac{m}{s}}{5,7GHz} = 0,053m$
    \pause
    $g_i = 10 \cdot \log_{10}{(\frac{\pi \cdot d}{\lambda})^2} \cdot \eta dB = 10 \cdot \log_{10}{(\frac{\pi \cdot 0,3m}{0,053m})^2} \cdot 1dB = 25,1dBi$



\end{frame}

\begin{frame}
\only<1>{
\begin{QQuestion}{AG227}{Wie hoch ist der Gewinn eines Parabolspiegels mit einem Durchmesser von \qty{80}{\cm} und mit einem Wirkungsgrad von $\eta_{\symup{eff}}$~=~1 bei einer Arbeitsfrequenz von \qty{5,7}{\GHz}?}{\qty{21,8}{\dBi}}
{\qty{16,8}{\dBi}}
{\qty{36,4}{\dBi}}
{\qty{33,6}{\dBi}}
\end{QQuestion}

}
\only<2>{
\begin{QQuestion}{AG227}{Wie hoch ist der Gewinn eines Parabolspiegels mit einem Durchmesser von \qty{80}{\cm} und mit einem Wirkungsgrad von $\eta_{\symup{eff}}$~=~1 bei einer Arbeitsfrequenz von \qty{5,7}{\GHz}?}{\qty{21,8}{\dBi}}
{\qty{16,8}{\dBi}}
{\qty{36,4}{\dBi}}
{\textbf{\textcolor{DARCgreen}{\qty{33,6}{\dBi}}}}
\end{QQuestion}

}
\end{frame}

\begin{frame}
\frametitle{Lösungsweg}
\begin{itemize}
  \item gegeben: $d = 80cm$
  \item gegeben: $\eta_{eff} = 1$
  \item gegeben: $f = 5,7GHz$
  \item gesucht: $g_i$
  \end{itemize}
    \pause
    $\lambda = \frac{c}{f} = \frac{3\cdot 10^8\frac{m}{s}}{5,7GHz} = 0,053m$
    \pause
    $g_i = 10 \cdot \log_{10}{(\frac{\pi \cdot d}{\lambda})^2} \cdot \eta dB = 10 \cdot \log_{10}{(\frac{\pi \cdot 0,8m}{0,053m})^2} \cdot 1dB = 33,6dBi$



\end{frame}

\begin{frame}
\only<1>{
\begin{QQuestion}{AG228}{Wie hoch ist der Gewinn eines Parabolspiegels mit einem Durchmesser von \qty{80}{\cm} und mit einem Wirkungsgrad von $\eta_{\symup{eff}}$~=~1 bei einer Arbeitsfrequenz von \qty{10,4}{\GHz}?}{\qty{19,4}{\dBi}}
{\qty{38,8}{\dBi}}
{\qty{42,4}{\dBi}}
{\qty{25,2}{\dBi}}
\end{QQuestion}

}
\only<2>{
\begin{QQuestion}{AG228}{Wie hoch ist der Gewinn eines Parabolspiegels mit einem Durchmesser von \qty{80}{\cm} und mit einem Wirkungsgrad von $\eta_{\symup{eff}}$~=~1 bei einer Arbeitsfrequenz von \qty{10,4}{\GHz}?}{\qty{19,4}{\dBi}}
{\textbf{\textcolor{DARCgreen}{\qty{38,8}{\dBi}}}}
{\qty{42,4}{\dBi}}
{\qty{25,2}{\dBi}}
\end{QQuestion}

}
\end{frame}

\begin{frame}
\frametitle{Lösungsweg}
\begin{itemize}
  \item gegeben: $d = 80cm$
  \item gegeben: $\eta_{eff} = 1$
  \item gegeben: $f = 10,4GHz$
  \item gesucht: $g_i$
  \end{itemize}
    \pause
    $\lambda = \frac{c}{f} = \frac{3\cdot 10^8\frac{m}{s}}{10,4GHz} = 0,029m$
    \pause
    $g_i = 10 \cdot \log_{10}{(\frac{\pi \cdot d}{\lambda})^2} \cdot \eta dB = 10 \cdot \log_{10}{(\frac{\pi \cdot 0,8m}{0,029m})^2} \cdot 1dB = 38,8dBi$



\end{frame}

\begin{frame}
\only<1>{
\begin{QQuestion}{AG229}{Wie hoch ist der Gewinn eines Parabolspiegels mit einem Durchmesser von \qty{120}{\cm} und mit einem Wirkungsgrad von $\eta_{\symup{eff}}$~=~1 bei einer Arbeitsfrequenz von \qty{10,4}{\GHz}?}{\qty{21,2}{\dBi}}
{\qty{42,3}{\dBi}}
{\qty{25,9}{\dBi}}
{\qty{50,5}{\dBi}}
\end{QQuestion}

}
\only<2>{
\begin{QQuestion}{AG229}{Wie hoch ist der Gewinn eines Parabolspiegels mit einem Durchmesser von \qty{120}{\cm} und mit einem Wirkungsgrad von $\eta_{\symup{eff}}$~=~1 bei einer Arbeitsfrequenz von \qty{10,4}{\GHz}?}{\qty{21,2}{\dBi}}
{\textbf{\textcolor{DARCgreen}{\qty{42,3}{\dBi}}}}
{\qty{25,9}{\dBi}}
{\qty{50,5}{\dBi}}
\end{QQuestion}

}
\end{frame}

\begin{frame}
\frametitle{Lösungsweg}
\begin{itemize}
  \item gegeben: $d = 120cm$
  \item gegeben: $\eta_{eff} = 1$
  \item gegeben: $f = 10,4GHz$
  \item gesucht: $g_i$
  \end{itemize}
    \pause
    $\lambda = \frac{c}{f} = \frac{3\cdot 10^8\frac{m}{s}}{10,4GHz} = 0,029m$
    \pause
    $g_i = 10 \cdot \log_{10}{(\frac{\pi \cdot d}{\lambda})^2} \cdot \eta dB = 10 \cdot \log_{10}{(\frac{\pi \cdot 1,2m}{0,029m})^2} \cdot 1dB = 42,3dBi$



\end{frame}%ENDCONTENT


\section{Offset-Spiegel}
\label{section:offsetspiegel}
\begin{frame}%STARTCONTENT

\only<1>{
\begin{QQuestion}{AG127}{Welchen Vorteil bietet im Mikrowellenbereich ein Offsetspiegel gegenüber einem rotationssymmetrischen Parabolspiegel?}{Die Auswahl an möglichen Erregerantennentypen ist größer.}
{Keinen, da beide Typen nach dem gleichen Funktionsprinzip arbeiten.}
{Die Erregerantenne sitzt außerhalb des Strahlenganges und verursacht keine Abschattungen.}
{Offsetspiegel erzeugen unabhängig von der Erregerantenne grundsätzlich eine  zirkulare Polarisation.}
\end{QQuestion}

}
\only<2>{
\begin{QQuestion}{AG127}{Welchen Vorteil bietet im Mikrowellenbereich ein Offsetspiegel gegenüber einem rotationssymmetrischen Parabolspiegel?}{Die Auswahl an möglichen Erregerantennentypen ist größer.}
{Keinen, da beide Typen nach dem gleichen Funktionsprinzip arbeiten.}
{\textbf{\textcolor{DARCgreen}{Die Erregerantenne sitzt außerhalb des Strahlenganges und verursacht keine Abschattungen.}}}
{Offsetspiegel erzeugen unabhängig von der Erregerantenne grundsätzlich eine  zirkulare Polarisation.}
\end{QQuestion}

}
\end{frame}%ENDCONTENT


\section{Vor-/Rückverhältnis}
\label{section:vor_rueck_verhaeltnis}
\begin{frame}%STARTCONTENT

\only<1>{
\begin{PQuestion}{AG214}{Das folgende Bild zeigt die Strahlungscharakteristik eines Dipols und einer Richtantenne. Das Vor-/Rück-Verhältnis der Richtantenne ist definiert als das Verhältnis~...}{von $P_{\symup{D}}$ zu $P_{\symup{R}}$.}
{von $P_{\symup{V}}$ zu $P_{\symup{R}}$.}
{von $P_{\symup{V}}$ zu $P_{\symup{D}}$.}
{von $0{,}7 \cdot P_{\symup{V}}$ zu $0{,}7 \cdot P_{\symup{D}}$.}
{\DARCimage{1.0\linewidth}{264include}}\end{PQuestion}

}
\only<2>{
\begin{PQuestion}{AG214}{Das folgende Bild zeigt die Strahlungscharakteristik eines Dipols und einer Richtantenne. Das Vor-/Rück-Verhältnis der Richtantenne ist definiert als das Verhältnis~...}{von $P_{\symup{D}}$ zu $P_{\symup{R}}$.}
{\textbf{\textcolor{DARCgreen}{von $P_{\symup{V}}$ zu $P_{\symup{R}}$.}}}
{von $P_{\symup{V}}$ zu $P_{\symup{D}}$.}
{von $0{,}7 \cdot P_{\symup{V}}$ zu $0{,}7 \cdot P_{\symup{D}}$.}
{\DARCimage{1.0\linewidth}{264include}}\end{PQuestion}

}
\end{frame}

\begin{frame}
\only<1>{
\begin{PQuestion}{AG213}{Das folgende Bild zeigt die Strahlungscharakteristik eines Dipols und einer Richtantenne. Der Antennengewinn der Richtantenne über dem Dipol ist definiert als das Verhältnis~...}{von $P_{\symup{V}}$ zu $P_{\symup{R}}$.}
{von $P_{\symup{D}}$ zu $P_{\symup{R}}$.}
{von $P_{\symup{V}}$ zu $P_{\symup{D}}$.}
{von $0{,}7 \cdot P_{\symup{V}}$ zu $0{,}7 \cdot P_{\symup{R}}$.}
{\DARCimage{1.0\linewidth}{264include}}\end{PQuestion}

}
\only<2>{
\begin{PQuestion}{AG213}{Das folgende Bild zeigt die Strahlungscharakteristik eines Dipols und einer Richtantenne. Der Antennengewinn der Richtantenne über dem Dipol ist definiert als das Verhältnis~...}{von $P_{\symup{V}}$ zu $P_{\symup{R}}$.}
{von $P_{\symup{D}}$ zu $P_{\symup{R}}$.}
{\textbf{\textcolor{DARCgreen}{von $P_{\symup{V}}$ zu $P_{\symup{D}}$.}}}
{von $0{,}7 \cdot P_{\symup{V}}$ zu $0{,}7 \cdot P_{\symup{R}}$.}
{\DARCimage{1.0\linewidth}{264include}}\end{PQuestion}

}
\end{frame}

\begin{frame}
\only<1>{
\begin{PQuestion}{AG217}{Bei einer Yagi-Uda-Antenne mit dem folgenden Strahlungsdiagramm beträgt die ERP in Richtung a \qty{0,6}{\W} und in Richtung b \qty{15}{\W}. Welches Vor-Rück-Verhältnis hat die Antenne?}{\qty{2,8}{\decibel}}
{\qty{27,9}{\decibel}}
{\qty{14}{\decibel}}
{\qty{25}{\decibel}}
{\DARCimage{1.0\linewidth}{263include}}\end{PQuestion}

}
\only<2>{
\begin{PQuestion}{AG217}{Bei einer Yagi-Uda-Antenne mit dem folgenden Strahlungsdiagramm beträgt die ERP in Richtung a \qty{0,6}{\W} und in Richtung b \qty{15}{\W}. Welches Vor-Rück-Verhältnis hat die Antenne?}{\qty{2,8}{\decibel}}
{\qty{27,9}{\decibel}}
{\textbf{\textcolor{DARCgreen}{\qty{14}{\decibel}}}}
{\qty{25}{\decibel}}
{\DARCimage{1.0\linewidth}{263include}}\end{PQuestion}

}
\end{frame}

\begin{frame}
\frametitle{Lösungsweg}
\begin{itemize}
  \item gegeben: $P_R = 0,6W$
  \item gegeben: $P_V = 15W$
  \item gesucht: $\frac{Vor}{Rück}$
  \end{itemize}
    \pause
    $\frac{Vor}{Rück} = 10 \cdot \log_{10}{(\frac{P_V}{P_R})} dB = 10 \cdot \log_{10}{(\frac{15W}{0,6W})} dB = 14dB$



\end{frame}

\begin{frame}
\only<1>{
\begin{QQuestion}{AG215}{Eine Richtantenne mit einem Gewinn von \qty{10}{\decibel} über dem Halbwellendipol und einem Vor-Rück-Verhältnis von \qty{20}{\decibel} wird mit \qty{100}{\W} Sendeleistung direkt gespeist. Welche ERP strahlt die Antenne entgegengesetzt zur Senderichtung ab?}{\qty{100}{\W}}
{\qty{10}{\W}}
{\qty{0,1}{\W}}
{\qty{1}{\W}}
\end{QQuestion}

}
\only<2>{
\begin{QQuestion}{AG215}{Eine Richtantenne mit einem Gewinn von \qty{10}{\decibel} über dem Halbwellendipol und einem Vor-Rück-Verhältnis von \qty{20}{\decibel} wird mit \qty{100}{\W} Sendeleistung direkt gespeist. Welche ERP strahlt die Antenne entgegengesetzt zur Senderichtung ab?}{\qty{100}{\W}}
{\textbf{\textcolor{DARCgreen}{\qty{10}{\W}}}}
{\qty{0,1}{\W}}
{\qty{1}{\W}}
\end{QQuestion}

}
\end{frame}

\begin{frame}
\frametitle{Lösungsweg}
\begin{itemize}
  \item gegeben: $g_D= 10dB$
  \item gegeben: $\frac{Vor}{Rück} = 20dB$
  \item gegeben: $P_S = 100W$
  \item gesucht: $P_R$
  \end{itemize}
    \pause
    $P_V = P_{ERP} = P_S \cdot 10^{\frac{g_d}{10dB}} = 100W \cdot 10^{\frac{10dB}{10dB}} = 1000W$
    \pause
    $20dB = 10 \cdot \log_{10}{(\frac{P_V}{P_R})} dB \Rightarrow \frac{P_V}{P_R} = 10^{\frac{20dB}{10}} = 100 \Rightarrow P_R = \frac{P_V}{100} = \frac{1000W}{100} = 10W$



\end{frame}

\begin{frame}
\only<1>{
\begin{QQuestion}{AG216}{Eine Richtantenne mit einem Gewinn von \qty{15}{\decibel} über dem Halbwellendipol und einem Vor-Rück-Verhältnis von \qty{25}{\decibel} wird mit \qty{6}{\W} Sendeleistung direkt gespeist. Welche ERP strahlt die Antenne entgegengesetzt zur Senderichtung ab?}{\qty{60}{\W}}
{\qty{0,019}{\W}}
{\qty{0,19}{\W}}
{\qty{0,6}{\W}}
\end{QQuestion}

}
\only<2>{
\begin{QQuestion}{AG216}{Eine Richtantenne mit einem Gewinn von \qty{15}{\decibel} über dem Halbwellendipol und einem Vor-Rück-Verhältnis von \qty{25}{\decibel} wird mit \qty{6}{\W} Sendeleistung direkt gespeist. Welche ERP strahlt die Antenne entgegengesetzt zur Senderichtung ab?}{\qty{60}{\W}}
{\qty{0,019}{\W}}
{\qty{0,19}{\W}}
{\textbf{\textcolor{DARCgreen}{\qty{0,6}{\W}}}}
\end{QQuestion}

}
\end{frame}

\begin{frame}
\frametitle{Lösungsweg}
\begin{itemize}
  \item gegeben: $g_D= 15dB$
  \item gegeben: $\frac{Vor}{Rück} = 25dB$
  \item gegeben: $P_S = 6W$
  \item gesucht: $P_R$
  \end{itemize}
    \pause
    $P_V = P_{ERP} = P_S \cdot 10^{\frac{g_d}{10dB}} = 6W \cdot 10^{\frac{15dB}{10dB}} = 189,7W$
    \pause
    $25dB = 10 \cdot \log_{10}{(\frac{P_V}{P_R})} dB \Rightarrow \frac{P_V}{P_R} = 10^{\frac{25dB}{10}} = 316,2 \Rightarrow P_R = \frac{P_V}{316,2} = \frac{189,7W}{316,2} = 0,6W$



\end{frame}

\begin{frame}
\only<1>{
\begin{QQuestion}{AG218}{Mit einem Feldstärkemessgerät wurden Vergleichsmessungen zwischen Beam und Dipol durchgeführt. In einem Abstand von \qty{32}{\m} wurden folgende Feldstärken gemessen: Beam vorwärts: \qty{300}{\micro\V}/m, Beam rückwärts: \qty{20}{\micro\V}/m, Halbwellendipol in Hauptstrahlrichtung: \qty{128}{\micro\V}/m. Welcher Gewinn und welches Vor-Rück-Verhältnis ergibt sich daraus für den Beam?}{Gewinn: 7,4~dBd, Vor-Rück-Verhältnis: \qty{15}{\decibel}}
{Gewinn: 3,7~dBd, Vor-Rück-Verhältnis: \qty{11,7}{\decibel}}
{Gewinn: 9,4~dBd, Vor-Rück-Verhältnis: \qty{23,5}{\decibel}}
{Gewinn: 7,4~dBd, Vor-Rück-Verhältnis: \qty{23,5}{\decibel}}
\end{QQuestion}

}
\only<2>{
\begin{QQuestion}{AG218}{Mit einem Feldstärkemessgerät wurden Vergleichsmessungen zwischen Beam und Dipol durchgeführt. In einem Abstand von \qty{32}{\m} wurden folgende Feldstärken gemessen: Beam vorwärts: \qty{300}{\micro\V}/m, Beam rückwärts: \qty{20}{\micro\V}/m, Halbwellendipol in Hauptstrahlrichtung: \qty{128}{\micro\V}/m. Welcher Gewinn und welches Vor-Rück-Verhältnis ergibt sich daraus für den Beam?}{Gewinn: 7,4~dBd, Vor-Rück-Verhältnis: \qty{15}{\decibel}}
{Gewinn: 3,7~dBd, Vor-Rück-Verhältnis: \qty{11,7}{\decibel}}
{Gewinn: 9,4~dBd, Vor-Rück-Verhältnis: \qty{23,5}{\decibel}}
{\textbf{\textcolor{DARCgreen}{Gewinn: 7,4~dBd, Vor-Rück-Verhältnis: \qty{23,5}{\decibel}}}}
\end{QQuestion}

}
\end{frame}

\begin{frame}
\frametitle{Lösungsweg}
\begin{itemize}
  \item gegeben: $U_V = 300µV/m$
  \item gegeben: $U_R = 20µV/m$
  \item gegeben: $U_D = 128µV/m$
  \item gesucht: $g_D$, $\frac{Vor}{Rück}$
  \end{itemize}
    \pause
    $g_D = 20 \cdot \log_{10}{(\frac{U_V}{U_D})} dB = 20 \cdot \log_{10}{(\frac{300µV/m}{128µV/m})} = 7,4dB$
    \pause
    $\frac{Vor}{Rück} = 20 \cdot \log_{10}{(\frac{U_V}{U_R})} dB = 20 \cdot \log_{10}{(\frac{300µV/m}{20µV/m})} = 23,5dB$



\end{frame}%ENDCONTENT


\section{Halbwertsbreite}
\label{section:halbwertsbreite}
\begin{frame}%STARTCONTENT

\only<1>{
\begin{QQuestion}{AG219}{Die Halbwertsbreite einer Antenne ist der Winkelbereich, innerhalb dessen~...}{die Feldstärke auf nicht weniger als den 0,707-fachen Wert der maximalen Feldstärke absinkt.}
{die Feldstärke auf nicht weniger als die Hälfte der maximalen Feldstärke absinkt.}
{die Strahlungsdichte auf nicht weniger als den $\dfrac{1}{\sqrt{2}}$-fachen Wert der maximalen Strahlungsdichte absinkt.}
{die abgestrahlte Leistung auf nicht weniger als den $\dfrac{1}{\sqrt{2}}$-fachen Wert des Leistungsmaximums absinkt.}
\end{QQuestion}

}
\only<2>{
\begin{QQuestion}{AG219}{Die Halbwertsbreite einer Antenne ist der Winkelbereich, innerhalb dessen~...}{\textbf{\textcolor{DARCgreen}{die Feldstärke auf nicht weniger als den 0,707-fachen Wert der maximalen Feldstärke absinkt.}}}
{die Feldstärke auf nicht weniger als die Hälfte der maximalen Feldstärke absinkt.}
{die Strahlungsdichte auf nicht weniger als den $\dfrac{1}{\sqrt{2}}$-fachen Wert der maximalen Strahlungsdichte absinkt.}
{die abgestrahlte Leistung auf nicht weniger als den $\dfrac{1}{\sqrt{2}}$-fachen Wert des Leistungsmaximums absinkt.}
\end{QQuestion}

}
\end{frame}

\begin{frame}
\only<1>{
\begin{PQuestion}{AG220}{In dem folgenden Richtdiagramm sind auf der Skala der relativen Feldstärke $\frac{E}{{E}_{\symup{max}}}$ die Punkte~a bis~d markiert. Durch welchen der Punkte a bis d ziehen Sie den Kreisbogen, um die Halbwertsbreite der Antenne an den Schnittpunkten des Kreises mit der Richtkeule ablesen zu können?}{Durch den Punkt~a.}
{Durch den Punkt~b.}
{Durch den Punkt~d.}
{Durch den Punkt~c.}
{\DARCimage{1.0\linewidth}{266include}}\end{PQuestion}

}
\only<2>{
\begin{PQuestion}{AG220}{In dem folgenden Richtdiagramm sind auf der Skala der relativen Feldstärke $\frac{E}{{E}_{\symup{max}}}$ die Punkte~a bis~d markiert. Durch welchen der Punkte a bis d ziehen Sie den Kreisbogen, um die Halbwertsbreite der Antenne an den Schnittpunkten des Kreises mit der Richtkeule ablesen zu können?}{Durch den Punkt~a.}
{Durch den Punkt~b.}
{Durch den Punkt~d.}
{\textbf{\textcolor{DARCgreen}{Durch den Punkt~c.}}}
{\DARCimage{1.0\linewidth}{266include}}\end{PQuestion}

}
\end{frame}

\begin{frame}
\only<1>{
\begin{PQuestion}{AG221}{Die folgende Skizze zeigt das Horizontaldiagramm der relativen Feldstärke einer Yagi-Uda-Antenne. Wie groß ist im vorliegenden Fall die Halbwertsbreite (Öffnungswinkel)?}{Etwa \qty{27}{\degree}}
{Etwa \qty{34}{\degree}}
{Etwa \qty{69}{\degree}}
{Etwa \qty{55}{\degree}}
{\DARCimage{1.0\linewidth}{266include}}\end{PQuestion}

}
\only<2>{
\begin{PQuestion}{AG221}{Die folgende Skizze zeigt das Horizontaldiagramm der relativen Feldstärke einer Yagi-Uda-Antenne. Wie groß ist im vorliegenden Fall die Halbwertsbreite (Öffnungswinkel)?}{Etwa \qty{27}{\degree}}
{Etwa \qty{34}{\degree}}
{Etwa \qty{69}{\degree}}
{\textbf{\textcolor{DARCgreen}{Etwa \qty{55}{\degree}}}}
{\DARCimage{1.0\linewidth}{266include}}\end{PQuestion}

}
\end{frame}%ENDCONTENT


\section{Strom- und Spannungsspeisung II}
\label{section:strom_spannung_speisung_2}
\begin{frame}%STARTCONTENT

\only<1>{
\begin{QQuestion}{AG207}{Ein mittengespeister $\lambda$/2-Dipol ist bei seiner Grundfrequenz und deren ungeradzahligen Vielfachen~...}{stromgespeist, in Serienresonanz und am Eingang niederohmig.}
{spannungsgespeist, in Parallelresonanz und am Eingang hochohmig.}
{strom- und spannungsgespeist und weist einen rein kapazitiven Eingangswiderstand auf.}
{strom- und spannungsgespeist und weist einen rein induktiven Eingangswiderstand auf.}
\end{QQuestion}

}
\only<2>{
\begin{QQuestion}{AG207}{Ein mittengespeister $\lambda$/2-Dipol ist bei seiner Grundfrequenz und deren ungeradzahligen Vielfachen~...}{\textbf{\textcolor{DARCgreen}{stromgespeist, in Serienresonanz und am Eingang niederohmig.}}}
{spannungsgespeist, in Parallelresonanz und am Eingang hochohmig.}
{strom- und spannungsgespeist und weist einen rein kapazitiven Eingangswiderstand auf.}
{strom- und spannungsgespeist und weist einen rein induktiven Eingangswiderstand auf.}
\end{QQuestion}

}
\end{frame}

\begin{frame}
\only<1>{
\begin{QQuestion}{AG208}{Ein mittengespeister $\lambda$/2-Dipol ist bei geradzahligen Vielfachen seiner Grundfrequenz~...}{strom- und spannungsgespeist und weist einen rein kapazitiven Eingangswiderstand auf.}
{stromgespeist, in Serienresonanz und am Eingang niederohmig.}
{spannungsgespeist, in Parallelresonanz und am Eingang hochohmig.}
{strom- und spannungsgespeist und weist einen rein induktiven Eingangswiderstand auf.}
\end{QQuestion}

}
\only<2>{
\begin{QQuestion}{AG208}{Ein mittengespeister $\lambda$/2-Dipol ist bei geradzahligen Vielfachen seiner Grundfrequenz~...}{strom- und spannungsgespeist und weist einen rein kapazitiven Eingangswiderstand auf.}
{stromgespeist, in Serienresonanz und am Eingang niederohmig.}
{\textbf{\textcolor{DARCgreen}{spannungsgespeist, in Parallelresonanz und am Eingang hochohmig.}}}
{strom- und spannungsgespeist und weist einen rein induktiven Eingangswiderstand auf.}
\end{QQuestion}

}
\end{frame}%ENDCONTENT


\section{Frequenzabhängige Stromverteilung}
\label{section:frequenzabhaengige_stromverteilung}
\begin{frame}%STARTCONTENT

\only<1>{
\begin{PQuestion}{AG203}{Das folgende Bild zeigt die Stromverteilungen a~bis~d auf einem Dipol, der auf verschiedenen Resonanzfrequenzen erregt werden kann. Für welche Erregerfrequenz gilt die Stromkurve nach~a?}{Sie gilt für eine Erregung auf \qty{3,5}{\MHz}.}
{Sie gilt für eine Erregung auf \qty{14}{\MHz}.}
{Sie gilt für eine Erregung auf \qty{7}{\MHz}.}
{Sie gilt für eine Erregung auf \qty{28}{\MHz}.}
{\DARCimage{1.0\linewidth}{34include}}\end{PQuestion}

}
\only<2>{
\begin{PQuestion}{AG203}{Das folgende Bild zeigt die Stromverteilungen a~bis~d auf einem Dipol, der auf verschiedenen Resonanzfrequenzen erregt werden kann. Für welche Erregerfrequenz gilt die Stromkurve nach~a?}{Sie gilt für eine Erregung auf \qty{3,5}{\MHz}.}
{Sie gilt für eine Erregung auf \qty{14}{\MHz}.}
{Sie gilt für eine Erregung auf \qty{7}{\MHz}.}
{\textbf{\textcolor{DARCgreen}{Sie gilt für eine Erregung auf \qty{28}{\MHz}.}}}
{\DARCimage{1.0\linewidth}{34include}}\end{PQuestion}

}
\end{frame}

\begin{frame}
\only<1>{
\begin{PQuestion}{AG204}{Das folgende Bild zeigt die Stromverteilungen~a bis~d auf einem Dipol, der auf verschiedenen Resonanzfrequenzen erregt werden kann. Für welche Erregerfrequenz gilt die Stromkurve nach~b?}{Sie gilt für eine Erregung auf \qty{28}{\MHz}.}
{Sie gilt für eine Erregung auf \qty{14}{\MHz}.}
{Sie gilt für eine Erregung auf \qty{7}{\MHz}.}
{Sie gilt für eine Erregung auf \qty{3,5}{\MHz}.}
{\DARCimage{1.0\linewidth}{34include}}\end{PQuestion}

}
\only<2>{
\begin{PQuestion}{AG204}{Das folgende Bild zeigt die Stromverteilungen~a bis~d auf einem Dipol, der auf verschiedenen Resonanzfrequenzen erregt werden kann. Für welche Erregerfrequenz gilt die Stromkurve nach~b?}{Sie gilt für eine Erregung auf \qty{28}{\MHz}.}
{\textbf{\textcolor{DARCgreen}{Sie gilt für eine Erregung auf \qty{14}{\MHz}.}}}
{Sie gilt für eine Erregung auf \qty{7}{\MHz}.}
{Sie gilt für eine Erregung auf \qty{3,5}{\MHz}.}
{\DARCimage{1.0\linewidth}{34include}}\end{PQuestion}

}
\end{frame}

\begin{frame}
\only<1>{
\begin{PQuestion}{AG205}{Das folgende Bild zeigt die Stromverteilungen a bis d auf einem Dipol, der auf verschiedenen Resonanzfrequenzen erregt werden kann. Für welche Erregerfrequenz gilt die Stromkurve nach c?}{Sie gilt für eine Erregung auf \qty{28}{\MHz}.}
{Sie gilt für eine Erregung auf \qty{7}{\MHz}.}
{Sie gilt für eine Erregung auf \qty{14}{\MHz}.}
{Sie gilt für eine Erregung auf \qty{3,5}{\MHz}.}
{\DARCimage{1.0\linewidth}{34include}}\end{PQuestion}

}
\only<2>{
\begin{PQuestion}{AG205}{Das folgende Bild zeigt die Stromverteilungen a bis d auf einem Dipol, der auf verschiedenen Resonanzfrequenzen erregt werden kann. Für welche Erregerfrequenz gilt die Stromkurve nach c?}{Sie gilt für eine Erregung auf \qty{28}{\MHz}.}
{\textbf{\textcolor{DARCgreen}{Sie gilt für eine Erregung auf \qty{7}{\MHz}.}}}
{Sie gilt für eine Erregung auf \qty{14}{\MHz}.}
{Sie gilt für eine Erregung auf \qty{3,5}{\MHz}.}
{\DARCimage{1.0\linewidth}{34include}}\end{PQuestion}

}
\end{frame}

\begin{frame}
\only<1>{
\begin{PQuestion}{AG206}{Das folgende Bild zeigt die Stromverteilungen~a bis~d auf einem Dipol, der auf verschiedenen Resonanzfrequenzen erregt werden kann. Für welche Erregerfrequenz gilt die Stromkurve nach~d?}{Sie gilt für eine Erregung auf \qty{14}{\MHz}.}
{Sie gilt für eine Erregung auf \qty{28}{\MHz}.}
{Sie gilt für eine Erregung auf \qty{7}{\MHz}.}
{Sie gilt für eine Erregung auf \qty{3,5}{\MHz}.}
{\DARCimage{1.0\linewidth}{34include}}\end{PQuestion}

}
\only<2>{
\begin{PQuestion}{AG206}{Das folgende Bild zeigt die Stromverteilungen~a bis~d auf einem Dipol, der auf verschiedenen Resonanzfrequenzen erregt werden kann. Für welche Erregerfrequenz gilt die Stromkurve nach~d?}{Sie gilt für eine Erregung auf \qty{14}{\MHz}.}
{Sie gilt für eine Erregung auf \qty{28}{\MHz}.}
{Sie gilt für eine Erregung auf \qty{7}{\MHz}.}
{\textbf{\textcolor{DARCgreen}{Sie gilt für eine Erregung auf \qty{3,5}{\MHz}.}}}
{\DARCimage{1.0\linewidth}{34include}}\end{PQuestion}

}
\end{frame}%ENDCONTENT


\section{Übertragungsleitungen III}
\label{section:uebertragungsleitungen_3}
\begin{frame}%STARTCONTENT

\only<1>{
\begin{QQuestion}{AG312}{Bei einer symmetrischen Zweidrahtleitung ohne Gleichtaktanteil~...}{sind Spannung gegenüber Erde und Strom in beiden Leitern gleich groß und an jeder Stelle gleichphasig.}
{liegt einer der beiden Leiter auf Erdpotential.}
{gibt es keine Strom- und Spannungsverteilung auf der Leitung.}
{sind Spannung gegenüber Erde und Strom in beiden Leitern gleich groß und an jeder Stelle gegenphasig.}
\end{QQuestion}

}
\only<2>{
\begin{QQuestion}{AG312}{Bei einer symmetrischen Zweidrahtleitung ohne Gleichtaktanteil~...}{sind Spannung gegenüber Erde und Strom in beiden Leitern gleich groß und an jeder Stelle gleichphasig.}
{liegt einer der beiden Leiter auf Erdpotential.}
{gibt es keine Strom- und Spannungsverteilung auf der Leitung.}
{\textbf{\textcolor{DARCgreen}{sind Spannung gegenüber Erde und Strom in beiden Leitern gleich groß und an jeder Stelle gegenphasig.}}}
\end{QQuestion}

}
\end{frame}

\begin{frame}
\only<1>{
\begin{QQuestion}{AG301}{Um bei hohen Sendeleistungen auf den Kurzwellenbändern die Störwahrscheinlichkeit auf ein Mindestmaß zu begrenzen, sollte die für die Sendeantenne verwendete Speiseleitung innerhalb von Gebäuden~...}{kein ganzzahliges Vielfaches von $\lambda$/4 lang sein.}
{möglichst $\lambda$/4 lang sein.}
{geschirmt sein.}
{an keiner Stelle geerdet sein.}
\end{QQuestion}

}
\only<2>{
\begin{QQuestion}{AG301}{Um bei hohen Sendeleistungen auf den Kurzwellenbändern die Störwahrscheinlichkeit auf ein Mindestmaß zu begrenzen, sollte die für die Sendeantenne verwendete Speiseleitung innerhalb von Gebäuden~...}{kein ganzzahliges Vielfaches von $\lambda$/4 lang sein.}
{möglichst $\lambda$/4 lang sein.}
{\textbf{\textcolor{DARCgreen}{geschirmt sein.}}}
{an keiner Stelle geerdet sein.}
\end{QQuestion}

}
\end{frame}

\begin{frame}
\only<1>{
\begin{QQuestion}{AG303}{Welche Parameter beschreiben charakteristische Hochfrequenzeigenschaften eines Koaxialkabels?}{Biegeradius, Kabeldämpfung, Leitermaterial.}
{Wellenwiderstand, Kabeldämpfung, Verkürzungsfaktor.}
{Verkürzungsfaktor, Kabeldämpfung, Kabelfarbe.}
{Rückflußdämpfung, Dielektrizitätskonstante, Kabeldämpfung.}
\end{QQuestion}

}
\only<2>{
\begin{QQuestion}{AG303}{Welche Parameter beschreiben charakteristische Hochfrequenzeigenschaften eines Koaxialkabels?}{Biegeradius, Kabeldämpfung, Leitermaterial.}
{\textbf{\textcolor{DARCgreen}{Wellenwiderstand, Kabeldämpfung, Verkürzungsfaktor.}}}
{Verkürzungsfaktor, Kabeldämpfung, Kabelfarbe.}
{Rückflußdämpfung, Dielektrizitätskonstante, Kabeldämpfung.}
\end{QQuestion}

}
\end{frame}

\begin{frame}
\only<1>{
\begin{QQuestion}{AG314}{Die Ausbreitungsgeschwindigkeit in einem Koaxialkabel~...}{ist unbegrenzt.}
{ist höher als im Freiraum.}
{entspricht der Geschwindigkeit im Freiraum.}
{ist geringer als im Freiraum.}
\end{QQuestion}

}
\only<2>{
\begin{QQuestion}{AG314}{Die Ausbreitungsgeschwindigkeit in einem Koaxialkabel~...}{ist unbegrenzt.}
{ist höher als im Freiraum.}
{entspricht der Geschwindigkeit im Freiraum.}
{\textbf{\textcolor{DARCgreen}{ist geringer als im Freiraum.}}}
\end{QQuestion}

}
\end{frame}

\begin{frame}
\only<1>{
\begin{QQuestion}{AG302}{Welche Materialien werden für die Dielektriken gebräuchlicher Koaxkabel üblicherweise verwendet?}{PE-Schaum, Polystyrol, PTFE (Teflon).}
{Pertinax, Voll-PE, PE-Schaum.}
{PTFE (Teflon), Voll-PE, PE-Schaum.}
{Voll-PE, PE-Schaum, Epoxyd.}
\end{QQuestion}

}
\only<2>{
\begin{QQuestion}{AG302}{Welche Materialien werden für die Dielektriken gebräuchlicher Koaxkabel üblicherweise verwendet?}{PE-Schaum, Polystyrol, PTFE (Teflon).}
{Pertinax, Voll-PE, PE-Schaum.}
{\textbf{\textcolor{DARCgreen}{PTFE (Teflon), Voll-PE, PE-Schaum.}}}
{Voll-PE, PE-Schaum, Epoxyd.}
\end{QQuestion}

}
\end{frame}

\begin{frame}
\only<1>{
\begin{QQuestion}{AG317}{Welche mechanische Länge hat ein elektrisch $\lambda$/4 langes Koaxkabel mit Vollpolyethylenisolierung bei \qty{145}{\MHz}?}{\qty{34,2}{\cm}}
{\qty{51,7}{\cm}}
{\qty{103}{\cm}}
{\qty{17,1}{\cm}}
\end{QQuestion}

}
\only<2>{
\begin{QQuestion}{AG317}{Welche mechanische Länge hat ein elektrisch $\lambda$/4 langes Koaxkabel mit Vollpolyethylenisolierung bei \qty{145}{\MHz}?}{\textbf{\textcolor{DARCgreen}{\qty{34,2}{\cm}}}}
{\qty{51,7}{\cm}}
{\qty{103}{\cm}}
{\qty{17,1}{\cm}}
\end{QQuestion}

}
\end{frame}%ENDCONTENT


\section{Wellenwiderstand}
\label{section:wellenwiderstand}
\begin{frame}%STARTCONTENT

\only<1>{
\begin{QQuestion}{AG305}{Eine offene Paralleldrahtleitung ist aus Draht mit einem Durchmesser d~=~\qty{2}{\mm} gefertigt. Der Abstand der parallelen Leiter beträgt a~=~\qty{20}{\cm}. Wie groß ist der Wellenwiderstand $Z_0$ der Leitung?}{ca. \qty{2,8}{\kohm}}
{ca. \qty{276}{\ohm}}
{ca. \qty{635}{\ohm}}
{ca. \qty{820}{\ohm}}
\end{QQuestion}

}
\only<2>{
\begin{QQuestion}{AG305}{Eine offene Paralleldrahtleitung ist aus Draht mit einem Durchmesser d~=~\qty{2}{\mm} gefertigt. Der Abstand der parallelen Leiter beträgt a~=~\qty{20}{\cm}. Wie groß ist der Wellenwiderstand $Z_0$ der Leitung?}{ca. \qty{2,8}{\kohm}}
{ca. \qty{276}{\ohm}}
{\textbf{\textcolor{DARCgreen}{ca. \qty{635}{\ohm}}}}
{ca. \qty{820}{\ohm}}
\end{QQuestion}

}
\end{frame}

\begin{frame}
\frametitle{Lösungsweg}
\begin{itemize}
  \item gegeben: $d = 2mm$
  \item gegeben: $a = 20cm$
  \item gegeben: $\epsilon_\mathrm{r} \approx 1$ für Luft
  \item gesucht: $Z$
  \end{itemize}
    \pause
    $Z = \dfrac{120Ω}{\sqrt{\epsilon_\mathrm{r}}} \cdot \ln{(\dfrac{2 \cdot a}{d})} = \dfrac{120Ω}{\sqrt{1}} \cdot \ln{(\dfrac{2 \cdot 200mm}{2mm})} \approx 635Ω$



\end{frame}

\begin{frame}
\only<1>{
\begin{QQuestion}{AG306}{Ein Koaxialkabel (luftisoliert) hat einen Innendurchmesser der Abschirmung von \qty{5}{\mm}. Der Außendurchmesser des inneren Leiters beträgt \qty{1}{\mm}. Wie groß ist der Wellenwiderstand $Z_0$ des Kabels?}{ca. \qty{123}{\ohm}}
{ca. \qty{60}{\ohm}}
{ca. \qty{50}{\ohm}}
{ca. \qty{97}{\ohm}}
\end{QQuestion}

}
\only<2>{
\begin{QQuestion}{AG306}{Ein Koaxialkabel (luftisoliert) hat einen Innendurchmesser der Abschirmung von \qty{5}{\mm}. Der Außendurchmesser des inneren Leiters beträgt \qty{1}{\mm}. Wie groß ist der Wellenwiderstand $Z_0$ des Kabels?}{ca. \qty{123}{\ohm}}
{ca. \qty{60}{\ohm}}
{ca. \qty{50}{\ohm}}
{\textbf{\textcolor{DARCgreen}{ca. \qty{97}{\ohm}}}}
\end{QQuestion}

}
\end{frame}

\begin{frame}
\frametitle{Lösungsweg}
\begin{itemize}
  \item gegeben: $D = 5mm$
  \item gegeben: $d = 1mm$
  \item gegeben: $\epsilon_\mathrm{r} \approx 1$ für Luft
  \item gesucht: $Z$
  \end{itemize}
    \pause
    $Z = \dfrac{60Ω}{\sqrt{\epsilon_\mathrm{r}}} \cdot \ln{(\dfrac{D}{d})} = \dfrac{60Ω}{\sqrt{1}} \cdot \ln{(\dfrac{5mm}{1mm})} \approx 97Ω$



\end{frame}

\begin{frame}
\only<1>{
\begin{QQuestion}{AG307}{Ein Koaxialkabel hat einen Innenleiterdurchmesser von \qty{0,7}{\mm}. Die Isolierung zwischen Innenleiter und Abschirmgeflecht besteht aus Polyethylen (PE) und sie hat einen Durchmesser von \qty{4,4}{\mm}. Der Außendurchmesser des Kabels ist \qty{7,4}{\mm}. Wie hoch ist der ungefähre Wellenwiderstand des Kabels?}{ca. \qty{75}{\ohm}}
{ca. \qty{20}{\ohm}}
{ca. \qty{50}{\ohm}}
{ca. \qty{95}{\ohm}}
\end{QQuestion}

}
\only<2>{
\begin{QQuestion}{AG307}{Ein Koaxialkabel hat einen Innenleiterdurchmesser von \qty{0,7}{\mm}. Die Isolierung zwischen Innenleiter und Abschirmgeflecht besteht aus Polyethylen (PE) und sie hat einen Durchmesser von \qty{4,4}{\mm}. Der Außendurchmesser des Kabels ist \qty{7,4}{\mm}. Wie hoch ist der ungefähre Wellenwiderstand des Kabels?}{\textbf{\textcolor{DARCgreen}{ca. \qty{75}{\ohm}}}}
{ca. \qty{20}{\ohm}}
{ca. \qty{50}{\ohm}}
{ca. \qty{95}{\ohm}}
\end{QQuestion}

}
\end{frame}

\begin{frame}
\frametitle{Lösungsweg}
\begin{itemize}
  \item gegeben: $d = 0,7mm$
  \item gegeben: $D = 4,4mm$
  \item gegeben: $\epsilon_\mathrm{r} = 2,29$
  \item gesucht: $Z$
  \end{itemize}
    \pause
    $Z = \dfrac{60Ω}{\sqrt{\epsilon_\mathrm{r}}} \cdot \ln{(\dfrac{D}{d})} = \dfrac{60Ω}{\sqrt{2,29}} \cdot \ln{(\dfrac{4,4mm}{0,7mm})} \approx 75Ω$



\end{frame}

\begin{frame}
\only<1>{
\begin{QQuestion}{AG304}{Eine Übertragungsleitung wird angepasst betrieben, wenn der Widerstand, mit dem sie abgeschlossen ist,~...}{den Wert des Wellenwiderstandes der Leitung aufweist.}
{\qty{50}{\ohm} beträgt.}
{ein ohmscher Wirkwiderstand ist.}
{eine offene Leitung darstellt.}
\end{QQuestion}

}
\only<2>{
\begin{QQuestion}{AG304}{Eine Übertragungsleitung wird angepasst betrieben, wenn der Widerstand, mit dem sie abgeschlossen ist,~...}{\textbf{\textcolor{DARCgreen}{den Wert des Wellenwiderstandes der Leitung aufweist.}}}
{\qty{50}{\ohm} beträgt.}
{ein ohmscher Wirkwiderstand ist.}
{eine offene Leitung darstellt.}
\end{QQuestion}

}
\end{frame}%ENDCONTENT


\section{Kabeldämpfung II}
\label{section:kabeldaempfung_2}
\begin{frame}%STARTCONTENT

\only<1>{
\begin{QQuestion}{AG309}{Welches Koaxkabel ist nach dem zur Verfügung gestellten Kabeldämpfungsdiagramm für eine \qty{20}{\m} lange Verbindung zwischen Senderausgang und Antenne geeignet, wenn die Kabeldämpfung im \qty{13}{\cm}-Band bei \qty{2,350}{\GHz} nicht mehr als \qty{4}{\decibel} betragen soll?}{Voll-PE-Kabel mit \qty{10,3}{\mm} Durchmesser (Typ RG213).
}
{PE-Schaumkabel mit \qty{7,3}{\mm} Durchmesser.}
{PE-Schaumkabel mit \qty{12,7}{\mm} Durchmesser.}
{PE-Schaumkabel mit \qty{10,3}{\mm} Durchmesser.}
\end{QQuestion}

}
\only<2>{
\begin{QQuestion}{AG309}{Welches Koaxkabel ist nach dem zur Verfügung gestellten Kabeldämpfungsdiagramm für eine \qty{20}{\m} lange Verbindung zwischen Senderausgang und Antenne geeignet, wenn die Kabeldämpfung im \qty{13}{\cm}-Band bei \qty{2,350}{\GHz} nicht mehr als \qty{4}{\decibel} betragen soll?}{Voll-PE-Kabel mit \qty{10,3}{\mm} Durchmesser (Typ RG213).
}
{PE-Schaumkabel mit \qty{7,3}{\mm} Durchmesser.}
{\textbf{\textcolor{DARCgreen}{PE-Schaumkabel mit \qty{12,7}{\mm} Durchmesser.}}}
{PE-Schaumkabel mit \qty{10,3}{\mm} Durchmesser.}
\end{QQuestion}

}
\end{frame}

\begin{frame}
\only<1>{
\begin{QQuestion}{AG310}{Zur Verbindung Ihres \qty{5,700}{\GHz}-Senders (\qty{6}{\cm}-Band) mit dem Feed eines Parabolspiegels benötigen Sie ein \qty{8}{\m} langes und möglichst dünnes Koaxialkabel, das nicht mehr als \qty{3}{\decibel} Dämpfung haben soll. Welches der Koaxialkabel aus dem Kabeldämpfungsdiagramm erfüllt diese Anforderung?}{PE-Schaumkabel mit \qty{7,3}{\mm} Durchmesser.}
{PE-Schaumkabel mit \qty{12,7}{\mm} Durchmesser.}
{PE-Schaumkabel mit Massivschirm und \qty{16,4}{\mm} Durchmesser.}
{PE-Schaumkabel mit \qty{10,3}{\mm} Durchmesser.}
\end{QQuestion}

}
\only<2>{
\begin{QQuestion}{AG310}{Zur Verbindung Ihres \qty{5,700}{\GHz}-Senders (\qty{6}{\cm}-Band) mit dem Feed eines Parabolspiegels benötigen Sie ein \qty{8}{\m} langes und möglichst dünnes Koaxialkabel, das nicht mehr als \qty{3}{\decibel} Dämpfung haben soll. Welches der Koaxialkabel aus dem Kabeldämpfungsdiagramm erfüllt diese Anforderung?}{PE-Schaumkabel mit \qty{7,3}{\mm} Durchmesser.}
{\textbf{\textcolor{DARCgreen}{PE-Schaumkabel mit \qty{12,7}{\mm} Durchmesser.}}}
{PE-Schaumkabel mit Massivschirm und \qty{16,4}{\mm} Durchmesser.}
{PE-Schaumkabel mit \qty{10,3}{\mm} Durchmesser.}
\end{QQuestion}

}
\end{frame}

\begin{frame}
\only<1>{
\begin{QQuestion}{AG308}{Welcher Typ Koaxialkabel ist laut zur Verfügung gestelltem Kabeldämpfungsdiagramm für eine \qty{60}{\m} lange Verbindung zwischen Senderausgang und einem Multiband-Kurzwellenbeam für die Bänder \qty{20}{\m}, \qty{15}{\m} und \qty{10}{\m} geeignet, wenn die Kabeldämpfung bei \qty{29}{\MHz} nicht mehr als \qty{2}{\decibel} betragen soll?}{Voll-PE-Kabel mit \qty{4,95}{\mm} Durchmesser (Typ RG58).}
{PE-Schaumkabel mit \qty{10,3}{\mm} Durchmesser.}
{Voll-PE-Kabel mit \qty{10,3}{\mm} Durchmesser (Typ RG213).}
{PE-Schaumkabel mit \qty{7,3}{\mm} Durchmesser.}
\end{QQuestion}

}
\only<2>{
\begin{QQuestion}{AG308}{Welcher Typ Koaxialkabel ist laut zur Verfügung gestelltem Kabeldämpfungsdiagramm für eine \qty{60}{\m} lange Verbindung zwischen Senderausgang und einem Multiband-Kurzwellenbeam für die Bänder \qty{20}{\m}, \qty{15}{\m} und \qty{10}{\m} geeignet, wenn die Kabeldämpfung bei \qty{29}{\MHz} nicht mehr als \qty{2}{\decibel} betragen soll?}{Voll-PE-Kabel mit \qty{4,95}{\mm} Durchmesser (Typ RG58).}
{\textbf{\textcolor{DARCgreen}{PE-Schaumkabel mit \qty{10,3}{\mm} Durchmesser.}}}
{Voll-PE-Kabel mit \qty{10,3}{\mm} Durchmesser (Typ RG213).}
{PE-Schaumkabel mit \qty{7,3}{\mm} Durchmesser.}
\end{QQuestion}

}
\end{frame}

\begin{frame}
\only<1>{
\begin{QQuestion}{AG311}{Welche der folgenden Leitungen weist bei gleichem Leiterquerschnitt im Kurzwellenbereich den geringsten Verlust auf?}{Zweidrahtleitung mit großem Abstand und schmalen Stegen.}
{Zweidrahtleitung mit großem Abstand und breiten Stegen.}
{Zweidrahtleitung mit geringem Abstand und Kunststoffumhüllung.}
{Verdrillte Zweidrahtleitung mit Kunststoffumhüllung.}
\end{QQuestion}

}
\only<2>{
\begin{QQuestion}{AG311}{Welche der folgenden Leitungen weist bei gleichem Leiterquerschnitt im Kurzwellenbereich den geringsten Verlust auf?}{\textbf{\textcolor{DARCgreen}{Zweidrahtleitung mit großem Abstand und schmalen Stegen.}}}
{Zweidrahtleitung mit großem Abstand und breiten Stegen.}
{Zweidrahtleitung mit geringem Abstand und Kunststoffumhüllung.}
{Verdrillte Zweidrahtleitung mit Kunststoffumhüllung.}
\end{QQuestion}

}
\end{frame}%ENDCONTENT


\section{Skineffekt}
\label{section:skineffekt}
\begin{frame}%STARTCONTENT

\only<1>{
\begin{QQuestion}{AG318}{Wie bezeichnet man den Effekt, dass sich mit steigender Frequenz der Elektronenstrom mehr und mehr zur Oberfläche eines Leiters hin verlagert, so dass sich mit steigender Frequenz der ohmsche Verlustwiderstand des Leiters erhöht?}{Als Dunning-Kruger-Effekt}
{Als Mögel-Dellinger-Effekt}
{Als Doppler-Effekt}
{Als Skin-Effekt}
\end{QQuestion}

}
\only<2>{
\begin{QQuestion}{AG318}{Wie bezeichnet man den Effekt, dass sich mit steigender Frequenz der Elektronenstrom mehr und mehr zur Oberfläche eines Leiters hin verlagert, so dass sich mit steigender Frequenz der ohmsche Verlustwiderstand des Leiters erhöht?}{Als Dunning-Kruger-Effekt}
{Als Mögel-Dellinger-Effekt}
{Als Doppler-Effekt}
{\textbf{\textcolor{DARCgreen}{Als Skin-Effekt}}}
\end{QQuestion}

}
\end{frame}

\begin{frame}
\only<1>{
\begin{QQuestion}{AG319}{Welche Folgen hat der Skin-Effekt bei steigender Frequenz? Der stromdurchflossene Querschnitt des Leiters~...}{sinkt und dadurch sinkt der effektive Widerstand des Leiters.}
{steigt und dadurch sinkt der effektive Widerstand des Leiters.}
{sinkt und dadurch steigt der effektive Widerstand des Leiters.}
{steigt und dadurch steigt der effektive Widerstand des Leiters.}
\end{QQuestion}

}
\only<2>{
\begin{QQuestion}{AG319}{Welche Folgen hat der Skin-Effekt bei steigender Frequenz? Der stromdurchflossene Querschnitt des Leiters~...}{sinkt und dadurch sinkt der effektive Widerstand des Leiters.}
{steigt und dadurch sinkt der effektive Widerstand des Leiters.}
{\textbf{\textcolor{DARCgreen}{sinkt und dadurch steigt der effektive Widerstand des Leiters.}}}
{steigt und dadurch steigt der effektive Widerstand des Leiters.}
\end{QQuestion}

}
\end{frame}%ENDCONTENT


\section{Stehwellenverhältnis (SWR) III}
\label{section:swr_3}
\begin{frame}%STARTCONTENT

\only<1>{
\begin{QQuestion}{AG405}{Ein Kabel mit einem Wellenwiderstand von \qty{75}{\ohm} und vernachlässigbarer Dämpfung wird zur Speisung einer Faltdipol-Antenne verwendet. Welches SWR kann man auf der Leitung erwarten?}{ca. \num{1,5} bis \num{2}}
{\num{0,3}}
{ca. \num{3,2} bis \num{4}}
{\num{5,7}}
\end{QQuestion}

}
\only<2>{
\begin{QQuestion}{AG405}{Ein Kabel mit einem Wellenwiderstand von \qty{75}{\ohm} und vernachlässigbarer Dämpfung wird zur Speisung einer Faltdipol-Antenne verwendet. Welches SWR kann man auf der Leitung erwarten?}{ca. \num{1,5} bis \num{2}}
{\num{0,3}}
{\textbf{\textcolor{DARCgreen}{ca. \num{3,2} bis \num{4}}}}
{\num{5,7}}
\end{QQuestion}

}
\end{frame}

\begin{frame}
\frametitle{Lösungsweg}
\begin{itemize}
  \item gegeben: $Z = 75Ω$
  \item gegeben: $R_2 \approx 300Ω$ Widerstand Faltdipol
  \item gesucht: $s$
  \end{itemize}
    \pause
    $s = \frac{R_2}{Z} = \frac{300Ω}{75Ω} = 4$



\end{frame}

\begin{frame}
\only<1>{
\begin{QQuestion}{AG402}{Am Eingang einer angepassten HF-Übertragungsleitung werden \qty{100}{\W} HF-Leistung eingespeist. Die Dämpfung der Leitung beträgt \qty{3}{\decibel}. Welche Leistung wird bei Leerlauf oder Kurzschluss am Leitungsende reflektiert?}{\qty{25}{\W}}
{\qty{50}{\W}}
{\qty{50}{\W} bei Leerlauf und \qty{0}{\W} bei Kurzschluss}
{\qty{0}{\W} bei Leerlauf und \qty{50}{\W} bei Kurzschluss}
\end{QQuestion}

}
\only<2>{
\begin{QQuestion}{AG402}{Am Eingang einer angepassten HF-Übertragungsleitung werden \qty{100}{\W} HF-Leistung eingespeist. Die Dämpfung der Leitung beträgt \qty{3}{\decibel}. Welche Leistung wird bei Leerlauf oder Kurzschluss am Leitungsende reflektiert?}{\qty{25}{\W}}
{\textbf{\textcolor{DARCgreen}{\qty{50}{\W}}}}
{\qty{50}{\W} bei Leerlauf und \qty{0}{\W} bei Kurzschluss}
{\qty{0}{\W} bei Leerlauf und \qty{50}{\W} bei Kurzschluss}
\end{QQuestion}

}
\end{frame}

\begin{frame}
\only<1>{
\begin{QQuestion}{AG403}{In den Eingang einer Antennenleitung mit einer Dämpfung von \qty{3}{\decibel} werden \qty{10}{\W} HF-Leistung eingespeist. Mit der am Leitungsende angeschlossenen Antenne misst man am Leitungseingang ein SWR von 3. Mit einer künstlichen \qty{50}{\ohm}-Antenne am Leitungsende beträgt das SWR am Leitungseingang etwa 1. Was lässt sich aus diesen Messergebnissen schließen?}{Die Antenne ist fehlerhaft. Sie strahlt so gut wie keine HF-Leistung ab.}
{Die Antennenleitung ist fehlerhaft, an der Antenne kommt so gut wie keine HF-Leistung an.}
{Die Antennenanlage ist in Ordnung. Es werden etwa \qty{5}{\W} HF-Leistung abgestrahlt.}
{Die Antennenanlage ist in Ordnung. Es werden etwa \qty{3,75}{\W} HF-Leistung abgestrahlt.}
\end{QQuestion}

}
\only<2>{
\begin{QQuestion}{AG403}{In den Eingang einer Antennenleitung mit einer Dämpfung von \qty{3}{\decibel} werden \qty{10}{\W} HF-Leistung eingespeist. Mit der am Leitungsende angeschlossenen Antenne misst man am Leitungseingang ein SWR von 3. Mit einer künstlichen \qty{50}{\ohm}-Antenne am Leitungsende beträgt das SWR am Leitungseingang etwa 1. Was lässt sich aus diesen Messergebnissen schließen?}{\textbf{\textcolor{DARCgreen}{Die Antenne ist fehlerhaft. Sie strahlt so gut wie keine HF-Leistung ab.}}}
{Die Antennenleitung ist fehlerhaft, an der Antenne kommt so gut wie keine HF-Leistung an.}
{Die Antennenanlage ist in Ordnung. Es werden etwa \qty{5}{\W} HF-Leistung abgestrahlt.}
{Die Antennenanlage ist in Ordnung. Es werden etwa \qty{3,75}{\W} HF-Leistung abgestrahlt.}
\end{QQuestion}

}
\end{frame}

\begin{frame}
\only<1>{
\begin{QQuestion}{AG404}{Am Eingang einer Antennenleitung mit einer Dämpfung von \qty{5}{\decibel} werden \qty{10}{\W} HF-Leistung eingespeist. Mit der am Leitungsende angeschlossenen Antenne misst man am Leitungseingang ein SWR von 1. Welches SWR ist am Leitungseingang zu erwarten, wenn die Antenne abgeklemmt wird?}{Ein SWR von ca. 3,6}
{Ein SWR von ca. 1,92}
{Ein SWR von ca. 0, da sich vorlaufende und rücklaufende Leistung gegenseitig auslöschen}
{Ein SWR, das gegen unendlich geht, da am Ende der Leitung die gesamte HF-Leistung reflektiert wird}
\end{QQuestion}

}
\only<2>{
\begin{QQuestion}{AG404}{Am Eingang einer Antennenleitung mit einer Dämpfung von \qty{5}{\decibel} werden \qty{10}{\W} HF-Leistung eingespeist. Mit der am Leitungsende angeschlossenen Antenne misst man am Leitungseingang ein SWR von 1. Welches SWR ist am Leitungseingang zu erwarten, wenn die Antenne abgeklemmt wird?}{Ein SWR von ca. 3,6}
{\textbf{\textcolor{DARCgreen}{Ein SWR von ca. 1,92}}}
{Ein SWR von ca. 0, da sich vorlaufende und rücklaufende Leistung gegenseitig auslöschen}
{Ein SWR, das gegen unendlich geht, da am Ende der Leitung die gesamte HF-Leistung reflektiert wird}
\end{QQuestion}

}
\end{frame}

\begin{frame}
\frametitle{Lösungsweg}
\begin{itemize}
  \item gegeben: $P_V = 5W$
  \item gegeben: $a = 5dB$
  \item gesucht: $s$
  \end{itemize}
    \pause
    Dämpfung auf gesamtes Kabel für Hin- und Rückweg: 10dB

$P_R = 10dB \cdot P_V = \frac{5W}{10} = 0,5W$
    \pause
    $s = \frac{\sqrt{P_\mathrm{v}}+\sqrt{P_\mathrm{r}}}{\sqrt{P_\mathrm{v}}-\sqrt{P_\mathrm{r}}} = \frac{\sqrt{5W}+\sqrt{0,5W}}{\sqrt{5W}-\sqrt{0,5W}} = 1,92$



\end{frame}%ENDCONTENT


\section{Stehwellenmessgerät (SWR-Meter) II}
\label{section:swr_meter_2}
\begin{frame}%STARTCONTENT

\only<1>{
\begin{QQuestion}{AI401}{Ein Stehwellenmessgerät misst und vergleicht bei einer HF-Leitung im Sendebetrieb~...}{die Maximalleistung $P_{\symup{max}}$ am Richtkoppler und die Minimalspannung $U_{\symup{min}}$ auf der Leitung.}
{mittels der eingebauten Richtkoppler die vorhandenen Impedanzen in Vor- und Rückrichtung der Leitung.}
{den Phasenwinkel zwischen vorlaufender und rücklaufender Leistung am eingebauten Abschlusswiderstand der Richtkoppler.}
{die Ausgangsspannungen zweier in die Leitung eingeschleifter Richtkoppler, die in gegensätzlicher Richtung betrieben werden.  }
\end{QQuestion}

}
\only<2>{
\begin{QQuestion}{AI401}{Ein Stehwellenmessgerät misst und vergleicht bei einer HF-Leitung im Sendebetrieb~...}{die Maximalleistung $P_{\symup{max}}$ am Richtkoppler und die Minimalspannung $U_{\symup{min}}$ auf der Leitung.}
{mittels der eingebauten Richtkoppler die vorhandenen Impedanzen in Vor- und Rückrichtung der Leitung.}
{den Phasenwinkel zwischen vorlaufender und rücklaufender Leistung am eingebauten Abschlusswiderstand der Richtkoppler.}
{\textbf{\textcolor{DARCgreen}{die Ausgangsspannungen zweier in die Leitung eingeschleifter Richtkoppler, die in gegensätzlicher Richtung betrieben werden.  }}}
\end{QQuestion}

}
\end{frame}

\begin{frame}
\only<1>{
\begin{PQuestion}{AI402}{Bei dieser Schaltung handelt es sich um~...}{einen Absolutleistungsmesser.}
{ein Impedanzmessgerät.}
{ein Stehwellenmessgerät.}
{einen Absorptionsfrequenzmesser.}
{\DARCimage{1.0\linewidth}{499include}}\end{PQuestion}

}
\only<2>{
\begin{PQuestion}{AI402}{Bei dieser Schaltung handelt es sich um~...}{einen Absolutleistungsmesser.}
{ein Impedanzmessgerät.}
{\textbf{\textcolor{DARCgreen}{ein Stehwellenmessgerät.}}}
{einen Absorptionsfrequenzmesser.}
{\DARCimage{1.0\linewidth}{499include}}\end{PQuestion}

}
\end{frame}

\begin{frame}
\only<1>{
\begin{QQuestion}{AI403}{Zur Überprüfung eines Stehwellenmessgerätes wird dessen Ausgang mit einem HF-geeigneten \qty{150}{\ohm}-Lastwiderstand abgeschlossen. Welches Stehwellenverhältnis muss das Messgerät anzeigen, wenn die Impedanz von Messgerät und Sender \qty{50}{\ohm} beträgt?}{\num{2,5}}
{\num{3}}
{\num{3,33}}
{\num{2}}
\end{QQuestion}

}
\only<2>{
\begin{QQuestion}{AI403}{Zur Überprüfung eines Stehwellenmessgerätes wird dessen Ausgang mit einem HF-geeigneten \qty{150}{\ohm}-Lastwiderstand abgeschlossen. Welches Stehwellenverhältnis muss das Messgerät anzeigen, wenn die Impedanz von Messgerät und Sender \qty{50}{\ohm} beträgt?}{\num{2,5}}
{\textbf{\textcolor{DARCgreen}{\num{3}}}}
{\num{3,33}}
{\num{2}}
\end{QQuestion}

}
\end{frame}

\begin{frame}
\frametitle{Lösungsweg}
\begin{itemize}
  \item gegeben: $R_2 = 150Ω$
  \item gegeben: $Z = 50Ω$
  \item gesucht: $s$
  \end{itemize}
    \pause
    $s = \frac{R_2}{Z} = \frac{150Ω}{50Ω} = 3$



\end{frame}%ENDCONTENT


\section{Vektorieller Netzwerkanalysator (VNA) II}
\label{section:vna_2}
\begin{frame}%STARTCONTENT

\only<1>{
\begin{QQuestion}{AI201}{Wie funktioniert ein vektorieller Netzwerkanalysator (VNA)? Ein HF-Generator erzeugt ein~...}{frequenzstabiles HF-Signal, mit dem  z.~B. ein Filter oder eine Antenne beaufschlagt wird. Aus der durch das Messobjekt entstehenden Fehlanpassung werden Dämpfungsverlauf oder Antennengewinn ermittelt.}
{frequenzstabiles HF-Signal, mit dem z.~B. ein Filter oder eine Antenne beaufschlagt wird. Die durch das angeschlossene Messobjekt erzeugten Strom- und Spannungsbäuche werden als Verläufe von z.~B. Impedanz und Phasenwinkel, Wirk- und Blindanteil oder dem Stehwellenverhältnis grafisch dargestellt.}
{frequenzveränderliches HF-Signal, mit dem z.~B. ein Filter oder eine Antenne beaufschlagt wird. Aus den durch das Messobjekt entstehenden Spannungseinbrüchen wird der Scheinwiderstand des Messobjektes ermittelt.}
{frequenzveränderliches HF-Signal, mit dem z.~B. ein Filter oder eine Antenne beaufschlagt wird. Die durch das angeschlossene Messobjekt veränderten Amplituden und Phasen des HF-Signals werden als Verläufe von z.~B. Impedanz und Phasenwinkel, Wirk- und Blindanteil oder dem Stehwellenverhältnis grafisch dargestellt.}
\end{QQuestion}

}
\only<2>{
\begin{QQuestion}{AI201}{Wie funktioniert ein vektorieller Netzwerkanalysator (VNA)? Ein HF-Generator erzeugt ein~...}{frequenzstabiles HF-Signal, mit dem  z.~B. ein Filter oder eine Antenne beaufschlagt wird. Aus der durch das Messobjekt entstehenden Fehlanpassung werden Dämpfungsverlauf oder Antennengewinn ermittelt.}
{frequenzstabiles HF-Signal, mit dem z.~B. ein Filter oder eine Antenne beaufschlagt wird. Die durch das angeschlossene Messobjekt erzeugten Strom- und Spannungsbäuche werden als Verläufe von z.~B. Impedanz und Phasenwinkel, Wirk- und Blindanteil oder dem Stehwellenverhältnis grafisch dargestellt.}
{frequenzveränderliches HF-Signal, mit dem z.~B. ein Filter oder eine Antenne beaufschlagt wird. Aus den durch das Messobjekt entstehenden Spannungseinbrüchen wird der Scheinwiderstand des Messobjektes ermittelt.}
{\textbf{\textcolor{DARCgreen}{frequenzveränderliches HF-Signal, mit dem z.~B. ein Filter oder eine Antenne beaufschlagt wird. Die durch das angeschlossene Messobjekt veränderten Amplituden und Phasen des HF-Signals werden als Verläufe von z.~B. Impedanz und Phasenwinkel, Wirk- und Blindanteil oder dem Stehwellenverhältnis grafisch dargestellt.}}}
\end{QQuestion}

}
\end{frame}

\begin{frame}
\only<1>{
\begin{QQuestion}{AI202}{Welches dieser Messgeräte ist für die Ermittlung der Resonanzfrequenz eines Traps, der für einen Dipol genutzt werden soll, am besten geeignet?}{Ein Frequenzmessgerät}
{Eine SWR-Messbrücke}
{Ein vektorieller Netzwerk Analysator}
{Ein Resonanzwellenmesser}
\end{QQuestion}

}
\only<2>{
\begin{QQuestion}{AI202}{Welches dieser Messgeräte ist für die Ermittlung der Resonanzfrequenz eines Traps, der für einen Dipol genutzt werden soll, am besten geeignet?}{Ein Frequenzmessgerät}
{Eine SWR-Messbrücke}
{\textbf{\textcolor{DARCgreen}{Ein vektorieller Netzwerk Analysator}}}
{Ein Resonanzwellenmesser}
\end{QQuestion}

}
\end{frame}

\begin{frame}
\only<1>{
\begin{QQuestion}{AI203}{Die Resonanzfrequenz eines abgestimmten HF-Kreises kann mit einem~...}{vektoriellen Netzwerkanalysator (VNA) überprüft werden.}
{Gleichspannungsmessgerät überprüft werden.}
{digitalen Frequenzmessgerät überprüft werden.}
{Ohmmeter überprüft werden.}
\end{QQuestion}

}
\only<2>{
\begin{QQuestion}{AI203}{Die Resonanzfrequenz eines abgestimmten HF-Kreises kann mit einem~...}{\textbf{\textcolor{DARCgreen}{vektoriellen Netzwerkanalysator (VNA) überprüft werden.}}}
{Gleichspannungsmessgerät überprüft werden.}
{digitalen Frequenzmessgerät überprüft werden.}
{Ohmmeter überprüft werden.}
\end{QQuestion}

}
\end{frame}

\begin{frame}
\only<1>{
\begin{QQuestion}{AI204}{Sie haben einen vektoriellen Netzwerkanalysator (VNA) an den Speisepunkt ihrer Kurzwellenantenne angeschlossen. Das Gerät zeigt R = \qty{54}{\ohm} und jX = \qty{-12}{\ohm} an. Was bedeutet das Messergebnis?}{Der ohmsche Anteil der Antennenimpedanz beträgt \qty{54}{\ohm}, der Blindanteil beträgt \qty{12}{\ohm} und ist induktiv.}
{Die Impedanz der Antenne beträgt \qty{66}{\ohm}. Es entsteht eine große induktive Fehlanpassung.}
{Der ohmsche Widerstand der Antennenimpedanz beträgt \qty{54}{\ohm}, der Blindanteil beträgt \qty{12}{\ohm} und ist kapazitiv.}
{Die Antenne ist wegen ihres großen Blindwiderstandes nur zum Empfang, nicht jedoch zum Senden geeignet.}
\end{QQuestion}

}
\only<2>{
\begin{QQuestion}{AI204}{Sie haben einen vektoriellen Netzwerkanalysator (VNA) an den Speisepunkt ihrer Kurzwellenantenne angeschlossen. Das Gerät zeigt R = \qty{54}{\ohm} und jX = \qty{-12}{\ohm} an. Was bedeutet das Messergebnis?}{Der ohmsche Anteil der Antennenimpedanz beträgt \qty{54}{\ohm}, der Blindanteil beträgt \qty{12}{\ohm} und ist induktiv.}
{Die Impedanz der Antenne beträgt \qty{66}{\ohm}. Es entsteht eine große induktive Fehlanpassung.}
{\textbf{\textcolor{DARCgreen}{Der ohmsche Widerstand der Antennenimpedanz beträgt \qty{54}{\ohm}, der Blindanteil beträgt \qty{12}{\ohm} und ist kapazitiv.}}}
{Die Antenne ist wegen ihres großen Blindwiderstandes nur zum Empfang, nicht jedoch zum Senden geeignet.}
\end{QQuestion}

}
\end{frame}

\begin{frame}
\only<1>{
\begin{QQuestion}{AI205}{Sie haben einen vektoriellen Netzwerkanalysator (VNA), der auf den VHF-Bereich eingestellt ist, an den Speisepunkt ihrer VHF-Antenne angeschlossen. Das Gerät zeigt R~=~\qty{50}{\ohm}~und~jX~=~\qty{0}{\ohm} an. Was erkennen Sie aus diesen Werten?}{Der fehlende Blindanteil~(jX) deutet darauf hin, dass die Antenne defekt ist.}
{Die Antenne ist für den Betrieb an einen VHF-Sender mit \qty{50}{\ohm} Ausgangsimpedanz gut angepasst.}
{Die Antenne ist für den Betrieb an einem Sender mit \qty{50}{\ohm} Ausgangsimpedanz schlecht angepasst, da der erforderliche Blindanteil~(jX) von \qty{50}{\ohm} fehlt.}
{Die Antenne ist wegen des fehlenden Blindwiderstandanteils nur zum Empfang, nicht jedoch zum Senden geeignet.}
\end{QQuestion}

}
\only<2>{
\begin{QQuestion}{AI205}{Sie haben einen vektoriellen Netzwerkanalysator (VNA), der auf den VHF-Bereich eingestellt ist, an den Speisepunkt ihrer VHF-Antenne angeschlossen. Das Gerät zeigt R~=~\qty{50}{\ohm}~und~jX~=~\qty{0}{\ohm} an. Was erkennen Sie aus diesen Werten?}{Der fehlende Blindanteil~(jX) deutet darauf hin, dass die Antenne defekt ist.}
{\textbf{\textcolor{DARCgreen}{Die Antenne ist für den Betrieb an einen VHF-Sender mit \qty{50}{\ohm} Ausgangsimpedanz gut angepasst.}}}
{Die Antenne ist für den Betrieb an einem Sender mit \qty{50}{\ohm} Ausgangsimpedanz schlecht angepasst, da der erforderliche Blindanteil~(jX) von \qty{50}{\ohm} fehlt.}
{Die Antenne ist wegen des fehlenden Blindwiderstandanteils nur zum Empfang, nicht jedoch zum Senden geeignet.}
\end{QQuestion}

}
\end{frame}

\begin{frame}
\only<1>{
\begin{QQuestion}{AI206}{Sie haben einen vektoriellen Netzwerkanalysator (VNA) an den Speisepunkt Ihrer Kurzwellenantenne angeschlossen. Das Gerät zeigt R = \qty{54}{\ohm} und jX = \qty{+12}{\ohm} an. Was bedeutet das Messergebnis?}{Der ohmsche Anteil der Antennenimpedanz beträgt \qty{54}{\ohm}, der Blindanteil beträgt \qty{12}{\ohm} und ist induktiv.}
{Die Impedanz der Antenne beträgt \qty{66}{\ohm}. Es entsteht eine große induktive Fehlanpassung.}
{Der ohmsche Widerstand der Antennenimpedanz beträgt \qty{54}{\ohm}, der Blindanteil beträgt \qty{12}{\ohm} und ist kapazitiv.}
{Die Antenne ist wegen ihres großen Blindwiderstandes nur zum Empfang, nicht jedoch zum Senden geeignet.}
\end{QQuestion}

}
\only<2>{
\begin{QQuestion}{AI206}{Sie haben einen vektoriellen Netzwerkanalysator (VNA) an den Speisepunkt Ihrer Kurzwellenantenne angeschlossen. Das Gerät zeigt R = \qty{54}{\ohm} und jX = \qty{+12}{\ohm} an. Was bedeutet das Messergebnis?}{\textbf{\textcolor{DARCgreen}{Der ohmsche Anteil der Antennenimpedanz beträgt \qty{54}{\ohm}, der Blindanteil beträgt \qty{12}{\ohm} und ist induktiv.}}}
{Die Impedanz der Antenne beträgt \qty{66}{\ohm}. Es entsteht eine große induktive Fehlanpassung.}
{Der ohmsche Widerstand der Antennenimpedanz beträgt \qty{54}{\ohm}, der Blindanteil beträgt \qty{12}{\ohm} und ist kapazitiv.}
{Die Antenne ist wegen ihres großen Blindwiderstandes nur zum Empfang, nicht jedoch zum Senden geeignet.}
\end{QQuestion}

}
\end{frame}

\begin{frame}
\only<1>{
\begin{PQuestion}{AI207}{Sie haben einen vektoriellen Netzwerkanalysator (VNA) an einen selbstgebauten Halbwellendipol angeschlossen und messen den dargestellten Resonanzverlauf. Was müssen Sie tun, um diese Antenne auf das \qty{80}{\metre}-Band abzustimmen? }{Sie fügen in beide Strahlerhälften jeweils einen \qty{50}{\ohm}~Widerstand ein}
{Sie verlängern beide Enden gleichmäßig.}
{Sie fügen in beide Strahlerhälften jeweils eine Induktivität ein.}
{Sie verkürzen beide Enden gleichmäßig.}
{\DARCimage{1.0\linewidth}{526include}}\end{PQuestion}

}
\only<2>{
\begin{PQuestion}{AI207}{Sie haben einen vektoriellen Netzwerkanalysator (VNA) an einen selbstgebauten Halbwellendipol angeschlossen und messen den dargestellten Resonanzverlauf. Was müssen Sie tun, um diese Antenne auf das \qty{80}{\metre}-Band abzustimmen? }{Sie fügen in beide Strahlerhälften jeweils einen \qty{50}{\ohm}~Widerstand ein}
{Sie verlängern beide Enden gleichmäßig.}
{Sie fügen in beide Strahlerhälften jeweils eine Induktivität ein.}
{\textbf{\textcolor{DARCgreen}{Sie verkürzen beide Enden gleichmäßig.}}}
{\DARCimage{1.0\linewidth}{526include}}\end{PQuestion}

}
\end{frame}

\begin{frame}
\only<1>{
\begin{PQuestion}{AI208}{Sie haben einen vektoriellen Netzwerkanalysator (VNA) an einen selbstgebauten Halbwellendipol angeschlossen und messen den dargestellten Resonanzverlauf. Was müssen Sie tun, um diese Antenne auf das \qty{80}{\metre}-Band abzustimmen? }{Sie verlängern beide Drahtenden gleichmäßig.}
{Sie verkürzen beide Drahtenden gleichmäßig.}
{Sie fügen in beide Strahlerhälften jeweils eine Kapazität ein.}
{Sie fügen eine Mantelwellensperre ein.}
{\DARCimage{1.0\linewidth}{527include}}\end{PQuestion}

}
\only<2>{
\begin{PQuestion}{AI208}{Sie haben einen vektoriellen Netzwerkanalysator (VNA) an einen selbstgebauten Halbwellendipol angeschlossen und messen den dargestellten Resonanzverlauf. Was müssen Sie tun, um diese Antenne auf das \qty{80}{\metre}-Band abzustimmen? }{\textbf{\textcolor{DARCgreen}{Sie verlängern beide Drahtenden gleichmäßig.}}}
{Sie verkürzen beide Drahtenden gleichmäßig.}
{Sie fügen in beide Strahlerhälften jeweils eine Kapazität ein.}
{Sie fügen eine Mantelwellensperre ein.}
{\DARCimage{1.0\linewidth}{527include}}\end{PQuestion}

}
\end{frame}%ENDCONTENT


\section{Phasenverschiebung in Übertragungsleitungen}
\label{section:leitung_phasenverschiebung}
\begin{frame}%STARTCONTENT

\only<1>{
\begin{PQuestion}{AG407}{Welche Phasenverschiebung erhält ein HF-Signal von a nach b, wenn die elektrische Länge der abgebildeten Koaxialleitung $\lambda$/4 beträgt?}{\qty{180}{\degree}}
{\qty{90}{\degree}}
{$\dfrac{\pi}{4}$}
{Null}
{\DARCimage{0.75\linewidth}{442include}}\end{PQuestion}

}
\only<2>{
\begin{PQuestion}{AG407}{Welche Phasenverschiebung erhält ein HF-Signal von a nach b, wenn die elektrische Länge der abgebildeten Koaxialleitung $\lambda$/4 beträgt?}{\qty{180}{\degree}}
{\textbf{\textcolor{DARCgreen}{\qty{90}{\degree}}}}
{$\dfrac{\pi}{4}$}
{Null}
{\DARCimage{0.75\linewidth}{442include}}\end{PQuestion}

}
\end{frame}

\begin{frame}
\only<1>{
\begin{PQuestion}{AG408}{Welche Phasenverschiebung erhält ein HF-Signal von a nach b, wenn die elektrische Länge der abgebildeten Koaxialleitung gleich der Wellenlänge ist?  }{\qty{90}{\degree}}
{\qty{180}{\degree}}
{$\dfrac{\pi^2}{4}$}
{\qty{0}{\degree}}
{\DARCimage{0.75\linewidth}{442include}}\end{PQuestion}

}
\only<2>{
\begin{PQuestion}{AG408}{Welche Phasenverschiebung erhält ein HF-Signal von a nach b, wenn die elektrische Länge der abgebildeten Koaxialleitung gleich der Wellenlänge ist?  }{\qty{90}{\degree}}
{\qty{180}{\degree}}
{$\dfrac{\pi^2}{4}$}
{\textbf{\textcolor{DARCgreen}{\qty{0}{\degree}}}}
{\DARCimage{0.75\linewidth}{442include}}\end{PQuestion}

}
\end{frame}%ENDCONTENT


\section{Impedanztransformation}
\label{section:impedanztransformation}
\begin{frame}%STARTCONTENT

\only<1>{
\begin{QQuestion}{AG412}{Eine Halbwellen-Übertragungsleitung ist an einem Ende mit \qty{50}{\ohm} abgeschlossen. Wie groß ist die Eingangsimpedanz am anderen Ende dieser Leitung?}{\qty{200}{\ohm}}
{\qty{25}{\ohm}}
{\qty{100}{\ohm}}
{\qty{50}{\ohm}}
\end{QQuestion}

}
\only<2>{
\begin{QQuestion}{AG412}{Eine Halbwellen-Übertragungsleitung ist an einem Ende mit \qty{50}{\ohm} abgeschlossen. Wie groß ist die Eingangsimpedanz am anderen Ende dieser Leitung?}{\qty{200}{\ohm}}
{\qty{25}{\ohm}}
{\qty{100}{\ohm}}
{\textbf{\textcolor{DARCgreen}{\qty{50}{\ohm}}}}
\end{QQuestion}

}
\end{frame}

\begin{frame}
\only<1>{
\begin{QQuestion}{AG416}{Ein Halbwellendipol hat bei seiner Resonanzfrequenz am Einspeisepunkt eine Impedanz von \qty{70}{\ohm}. Er wird über ein $\lambda$/2-langes \qty{300}{\ohm}-Flachbandkabel gespeist. Wie groß ist die Impedanz am Eingang der Speiseleitung?}{\qty{300}{\ohm}.}
{\qty{185}{\ohm}.}
{\qty{70}{\ohm}.}
{\qty{370}{\ohm}.}
\end{QQuestion}

}
\only<2>{
\begin{QQuestion}{AG416}{Ein Halbwellendipol hat bei seiner Resonanzfrequenz am Einspeisepunkt eine Impedanz von \qty{70}{\ohm}. Er wird über ein $\lambda$/2-langes \qty{300}{\ohm}-Flachbandkabel gespeist. Wie groß ist die Impedanz am Eingang der Speiseleitung?}{\qty{300}{\ohm}.}
{\qty{185}{\ohm}.}
{\textbf{\textcolor{DARCgreen}{\qty{70}{\ohm}.}}}
{\qty{370}{\ohm}.}
\end{QQuestion}

}
\end{frame}

\begin{frame}
\only<1>{
\begin{PQuestion}{AG413}{Einem Halbwellendipol wird die Sendeleistung über eine abgestimmte $\lambda$/2-Speiseleitung zugeführt. Wie hoch ist die Impedanz $Z_1$ am Einspeisepunkt des Dipols? Und wie hoch ist die Impedanz $Z_2$ am Anfang der Speiseleitung?}{$Z_1$ ist hochohmig und $Z_2$ niederohmig.}
{$Z_1$ und $Z_2$ sind hochohmig.}
{$Z_1$ ist niederohmig und $Z_2$ hochohmig.}
{$Z_1$ und $Z_2$ sind niederohmig.}
{\DARCimage{1.0\linewidth}{312include}}\end{PQuestion}

}
\only<2>{
\begin{PQuestion}{AG413}{Einem Halbwellendipol wird die Sendeleistung über eine abgestimmte $\lambda$/2-Speiseleitung zugeführt. Wie hoch ist die Impedanz $Z_1$ am Einspeisepunkt des Dipols? Und wie hoch ist die Impedanz $Z_2$ am Anfang der Speiseleitung?}{$Z_1$ ist hochohmig und $Z_2$ niederohmig.}
{$Z_1$ und $Z_2$ sind hochohmig.}
{$Z_1$ ist niederohmig und $Z_2$ hochohmig.}
{\textbf{\textcolor{DARCgreen}{$Z_1$ und $Z_2$ sind niederohmig.}}}
{\DARCimage{1.0\linewidth}{312include}}\end{PQuestion}

}
\end{frame}

\begin{frame}
\only<1>{
\begin{PQuestion}{AG414}{Einem Ganzwellendipol wird die Sendeleistung über eine abgestimmte $\lambda$/2-Speiseleitung zugeführt. Wie hoch ist die Impedanz $Z_1$ am Einspeisepunkt des Dipols und wie hoch ist die Impedanz $Z_2$ am Anfang der Speiseleitung?}{$Z_1$ ist niederohmig und $Z_2$ hochohmig.}
{$Z_1$ und $Z_2$ sind hochohmig.}
{$Z_1$ und $Z_2$ sind niederohmig.}
{$Z_1$ ist hochohmig und $Z_2$ niederohmig.}
{\DARCimage{1.0\linewidth}{312include}}\end{PQuestion}

}
\only<2>{
\begin{PQuestion}{AG414}{Einem Ganzwellendipol wird die Sendeleistung über eine abgestimmte $\lambda$/2-Speiseleitung zugeführt. Wie hoch ist die Impedanz $Z_1$ am Einspeisepunkt des Dipols und wie hoch ist die Impedanz $Z_2$ am Anfang der Speiseleitung?}{$Z_1$ ist niederohmig und $Z_2$ hochohmig.}
{\textbf{\textcolor{DARCgreen}{$Z_1$ und $Z_2$ sind hochohmig.}}}
{$Z_1$ und $Z_2$ sind niederohmig.}
{$Z_1$ ist hochohmig und $Z_2$ niederohmig.}
{\DARCimage{1.0\linewidth}{312include}}\end{PQuestion}

}
\end{frame}

\begin{frame}
\only<1>{
\begin{PQuestion}{AG415}{Einem Ganzwellendipol wird die Sendeleistung über eine abgestimmte $\lambda$/4-Speiseleitung zugeführt. Wie hoch ist die Impedanz $Z_1$ am Einspeisepunkt des Dipols und wie hoch ist die Impedanz $Z_2$ am Anfang der Speiseleitung?}{$Z_1$ ist hochohmig und $Z_2$ niederohmig.}
{$Z_1$ und $Z_2$ sind hochohmig.}
{$Z_1$ und $Z_2$ sind niederohmig.}
{$Z_1$ ist niederohmig und $Z_2$ hochohmig.}
{\DARCimage{1.0\linewidth}{312include}}\end{PQuestion}

}
\only<2>{
\begin{PQuestion}{AG415}{Einem Ganzwellendipol wird die Sendeleistung über eine abgestimmte $\lambda$/4-Speiseleitung zugeführt. Wie hoch ist die Impedanz $Z_1$ am Einspeisepunkt des Dipols und wie hoch ist die Impedanz $Z_2$ am Anfang der Speiseleitung?}{\textbf{\textcolor{DARCgreen}{$Z_1$ ist hochohmig und $Z_2$ niederohmig.}}}
{$Z_1$ und $Z_2$ sind hochohmig.}
{$Z_1$ und $Z_2$ sind niederohmig.}
{$Z_1$ ist niederohmig und $Z_2$ hochohmig.}
{\DARCimage{1.0\linewidth}{312include}}\end{PQuestion}

}
\end{frame}

\begin{frame}
\only<1>{
\begin{QQuestion}{AG417}{Ein Dipol mit einem Fußpunktwiderstand von \qty{60}{\ohm} soll über eine $\lambda$/4-Transformationsleitung mit einem \qty{240}{\ohm}-Flachbandkabel gespeist werden. Welchen Wellenwiderstand muss die Transformationsleitung haben?}{\qty{232}{\ohm}}
{\qty{150}{\ohm}}
{\qty{120}{\ohm}}
{\qty{300}{\ohm}}
\end{QQuestion}

}
\only<2>{
\begin{QQuestion}{AG417}{Ein Dipol mit einem Fußpunktwiderstand von \qty{60}{\ohm} soll über eine $\lambda$/4-Transformationsleitung mit einem \qty{240}{\ohm}-Flachbandkabel gespeist werden. Welchen Wellenwiderstand muss die Transformationsleitung haben?}{\qty{232}{\ohm}}
{\qty{150}{\ohm}}
{\textbf{\textcolor{DARCgreen}{\qty{120}{\ohm}}}}
{\qty{300}{\ohm}}
\end{QQuestion}

}
\end{frame}

\begin{frame}
\frametitle{Lösungsweg}
\begin{itemize}
  \item gegeben: $Z_A = 60Ω$
  \item gegeben: $Z_E = 240Ω$
  \item gesucht: $Z$
  \end{itemize}
    \pause
    $Z = \sqrt{Z_E \cdot Z_A} = \sqrt{240Ω \cdot 60Ω} = 120Ω$



\end{frame}

\begin{frame}
\only<1>{
\begin{QQuestion}{AG418}{Ein Faltdipol mit einem Fußpunktwiderstand von \qty{240}{\ohm} soll mit einer Hühnerleiter gespeist werden, deren Wellenwiderstand \qty{600}{\ohm} beträgt. Zur Anpassung soll ein $\lambda$/4 langes Stück Hühnerleiter mit einem anderen Wellenwiderstand verwendet werden. Welchen Wellenwiderstand muss die Transformationsleitung haben?}{\qty{380}{\ohm}}
{\qty{420}{\ohm}}
{\qty{840}{\ohm}}
{\qty{240}{\ohm}}
\end{QQuestion}

}
\only<2>{
\begin{QQuestion}{AG418}{Ein Faltdipol mit einem Fußpunktwiderstand von \qty{240}{\ohm} soll mit einer Hühnerleiter gespeist werden, deren Wellenwiderstand \qty{600}{\ohm} beträgt. Zur Anpassung soll ein $\lambda$/4 langes Stück Hühnerleiter mit einem anderen Wellenwiderstand verwendet werden. Welchen Wellenwiderstand muss die Transformationsleitung haben?}{\textbf{\textcolor{DARCgreen}{\qty{380}{\ohm}}}}
{\qty{420}{\ohm}}
{\qty{840}{\ohm}}
{\qty{240}{\ohm}}
\end{QQuestion}

}
\end{frame}

\begin{frame}
\frametitle{Lösungsweg}
\begin{itemize}
  \item gegeben: $Z_A = 240Ω$
  \item gegeben: $Z_E = 600Ω$
  \item gesucht: $Z$
  \end{itemize}
    \pause
    $Z = \sqrt{Z_E \cdot Z_A} = \sqrt{600Ω \cdot 240Ω} = 380Ω$



\end{frame}

\begin{frame}
\only<1>{
\begin{PQuestion}{AG406}{Worum handelt es sich bei dieser Schaltung? Es handelt sich um~...}{einen abstimmbaren Sperrkreis zur Entkopplung der Antenne  vom Sender.}
{einen regelbaren Bandpass mit veränderbarer Bandbreite zur Kompensation der Auskoppelverluste.}
{ein Pi-Filter zur Impedanztransformation und Verbesserung der Unterdrückung von Oberwellen.}
{einen Saugkreis, der die zweite Harmonische unterdrückt und so den Wirkungsgrad der Verstärkerstufe erhöht.}
{\DARCimage{0.75\linewidth}{425include}}\end{PQuestion}

}
\only<2>{
\begin{PQuestion}{AG406}{Worum handelt es sich bei dieser Schaltung? Es handelt sich um~...}{einen abstimmbaren Sperrkreis zur Entkopplung der Antenne  vom Sender.}
{einen regelbaren Bandpass mit veränderbarer Bandbreite zur Kompensation der Auskoppelverluste.}
{\textbf{\textcolor{DARCgreen}{ein Pi-Filter zur Impedanztransformation und Verbesserung der Unterdrückung von Oberwellen.}}}
{einen Saugkreis, der die zweite Harmonische unterdrückt und so den Wirkungsgrad der Verstärkerstufe erhöht.}
{\DARCimage{0.75\linewidth}{425include}}\end{PQuestion}

}
\end{frame}%ENDCONTENT


\section{Lecherleitung}
\label{section:lecherleitung}
\begin{frame}%STARTCONTENT

\only<1>{
\begin{QQuestion}{AG320}{Eine Lecherleitung besteht aus zwei parallelen Leitern. Wovon ist ihre Resonanzfrequenz wesentlich abhängig? Sie ist abhängig~...}{vom Wellenwiderstand der beiden parallelen Leiter.}
{vom verwendeten Balun.}
{vom SWR auf der Leitung.}
{von der Leitungslänge.}
\end{QQuestion}

}
\only<2>{
\begin{QQuestion}{AG320}{Eine Lecherleitung besteht aus zwei parallelen Leitern. Wovon ist ihre Resonanzfrequenz wesentlich abhängig? Sie ist abhängig~...}{vom Wellenwiderstand der beiden parallelen Leiter.}
{vom verwendeten Balun.}
{vom SWR auf der Leitung.}
{\textbf{\textcolor{DARCgreen}{von der Leitungslänge.}}}
\end{QQuestion}

}
\end{frame}

\begin{frame}
\only<1>{
\begin{QQuestion}{AG411}{Eine Viertelwellen-Übertragungsleitung ist an einem Ende offen. Die Impedanz am anderen Ende~...}{ist nahezu unendlich hochohmig.}
{ist gleich dem Wellenwiderstand.}
{beträgt das Dreifache des Wellenwiderstandes.}
{beträgt nahezu null Ohm.}
\end{QQuestion}

}
\only<2>{
\begin{QQuestion}{AG411}{Eine Viertelwellen-Übertragungsleitung ist an einem Ende offen. Die Impedanz am anderen Ende~...}{ist nahezu unendlich hochohmig.}
{ist gleich dem Wellenwiderstand.}
{beträgt das Dreifache des Wellenwiderstandes.}
{\textbf{\textcolor{DARCgreen}{beträgt nahezu null Ohm.}}}
\end{QQuestion}

}
\end{frame}

\begin{frame}
\only<1>{
\begin{PQuestion}{AG410}{Wie groß ist die Impedanz am Punkt X, wenn die elektrische Länge der abgebildeten Koaxialleitung $\lambda$/4 beträgt?}{\qty{50}{\ohm}}
{Sehr hochohmig}
{Annähernd \qty{0}{\ohm}}
{Ungefähr \qty{100}{\ohm}}
{\DARCimage{0.5\linewidth}{446include}}\end{PQuestion}

}
\only<2>{
\begin{PQuestion}{AG410}{Wie groß ist die Impedanz am Punkt X, wenn die elektrische Länge der abgebildeten Koaxialleitung $\lambda$/4 beträgt?}{\qty{50}{\ohm}}
{Sehr hochohmig}
{\textbf{\textcolor{DARCgreen}{Annähernd \qty{0}{\ohm}}}}
{Ungefähr \qty{100}{\ohm}}
{\DARCimage{0.5\linewidth}{446include}}\end{PQuestion}

}
\end{frame}

\begin{frame}
\only<1>{
\begin{PQuestion}{AG409}{Wie groß ist die Impedanz am Punkt~a, wenn die elektrische Länge der abgebildeten Koaxialleitung $\lambda$/4 beträgt?}{Ungefähr~\qty{100}{\ohm}}
{Annähernd~\qty{0}{\ohm}}
{\qty{50}{\ohm}}
{Sehr hochohmig}
{\DARCimage{0.5\linewidth}{445include}}\end{PQuestion}

}
\only<2>{
\begin{PQuestion}{AG409}{Wie groß ist die Impedanz am Punkt~a, wenn die elektrische Länge der abgebildeten Koaxialleitung $\lambda$/4 beträgt?}{Ungefähr~\qty{100}{\ohm}}
{Annähernd~\qty{0}{\ohm}}
{\qty{50}{\ohm}}
{\textbf{\textcolor{DARCgreen}{Sehr hochohmig}}}
{\DARCimage{0.5\linewidth}{445include}}\end{PQuestion}

}
\end{frame}%ENDCONTENT


\section{Mantelwellen II}
\label{section:mantelwellen_2}
\begin{frame}%STARTCONTENT

\only<1>{
\begin{QQuestion}{AG425}{Wann liegen Mantelwellen auf einem Koaxialkabel vor? Wenn~...}{vor- und rücklaufende Leistung nicht identisch sind.}
{der Schirm geerdet ist.}
{Stehwellen vorhanden sind.}
{Gleichtaktanteile vorhanden sind.}
\end{QQuestion}

}
\only<2>{
\begin{QQuestion}{AG425}{Wann liegen Mantelwellen auf einem Koaxialkabel vor? Wenn~...}{vor- und rücklaufende Leistung nicht identisch sind.}
{der Schirm geerdet ist.}
{Stehwellen vorhanden sind.}
{\textbf{\textcolor{DARCgreen}{Gleichtaktanteile vorhanden sind.}}}
\end{QQuestion}

}
\end{frame}

\begin{frame}
\only<1>{
\begin{QQuestion}{AG426}{Wie wirkt eine stromkompensierte Drossel (z.~B. Koaxialkabel um einen Ferritkern gewickelt) Mantelwellen entgegen? Sie wirkt~...}{hochohmig für Oberschwingungen und niederohmig für Grundschwingungen.}
{hochohmig für Gleichtaktanteile und niederohmig für Gegentaktanteile.}
{hochohmig für alle Ströme im Außenleiter und niederohmig für alle Ströme im Innenleiter.}
{hochohmig für Wechselströme des Innenleiters und niederohmig für Gleichströme des Außenleiters.}
\end{QQuestion}

}
\only<2>{
\begin{QQuestion}{AG426}{Wie wirkt eine stromkompensierte Drossel (z.~B. Koaxialkabel um einen Ferritkern gewickelt) Mantelwellen entgegen? Sie wirkt~...}{hochohmig für Oberschwingungen und niederohmig für Grundschwingungen.}
{\textbf{\textcolor{DARCgreen}{hochohmig für Gleichtaktanteile und niederohmig für Gegentaktanteile.}}}
{hochohmig für alle Ströme im Außenleiter und niederohmig für alle Ströme im Innenleiter.}
{hochohmig für Wechselströme des Innenleiters und niederohmig für Gleichströme des Außenleiters.}
\end{QQuestion}

}
\end{frame}

\begin{frame}
\only<1>{
\begin{QQuestion}{AJ115}{Zur Verhinderung von Rundfunk-Empfangsstörungen (z.~B. UKW, DAB, DVB-T), die durch Mantelwellen hervorgerufen werden, ist anstelle einer Mantelwellendrossel alternativ~...}{der Einbau einer seriellen Drosselspule in den Innenleiter der Empfangsantennenleitung möglich.}
{der Einbau eines Tiefpassfilters nach dem Senderausgang möglich.}
{der Einbau eines Bandpassfilters nach dem Senderausgang möglich.}
{der Einbau eines HF-Trenntrafos in die Empfangsantennenleitung möglich.}
\end{QQuestion}

}
\only<2>{
\begin{QQuestion}{AJ115}{Zur Verhinderung von Rundfunk-Empfangsstörungen (z.~B. UKW, DAB, DVB-T), die durch Mantelwellen hervorgerufen werden, ist anstelle einer Mantelwellendrossel alternativ~...}{der Einbau einer seriellen Drosselspule in den Innenleiter der Empfangsantennenleitung möglich.}
{der Einbau eines Tiefpassfilters nach dem Senderausgang möglich.}
{der Einbau eines Bandpassfilters nach dem Senderausgang möglich.}
{\textbf{\textcolor{DARCgreen}{der Einbau eines HF-Trenntrafos in die Empfangsantennenleitung möglich.}}}
\end{QQuestion}

}
\end{frame}

\begin{frame}
\only<1>{
\begin{QQuestion}{AG427}{Wodurch können Mantelwellen auf Koaxialkabeln verursacht werden?}{Durch symmetrische Antennen, schlechte Erdung asymmetrischer Antennen oder Einkopplung in den Koax-Schirm}
{Durch Asymmetrie der Spannungsversorgung oder durch Dielektrika der Speiseleitung, die einen hohen Widerstand aufweisen}
{Durch Stehwellen in Koaxialkabeln mit geflochtenem Mantel, deren Länge ein Vielfaches von $\lambda$/2 betragen}
{Durch Oberwellen auf Speiseleitungen, deren Länge ein Vielfaches von $\lambda$/4 oder 5/8 $\lambda$ betragen}
\end{QQuestion}

}
\only<2>{
\begin{QQuestion}{AG427}{Wodurch können Mantelwellen auf Koaxialkabeln verursacht werden?}{\textbf{\textcolor{DARCgreen}{Durch symmetrische Antennen, schlechte Erdung asymmetrischer Antennen oder Einkopplung in den Koax-Schirm}}}
{Durch Asymmetrie der Spannungsversorgung oder durch Dielektrika der Speiseleitung, die einen hohen Widerstand aufweisen}
{Durch Stehwellen in Koaxialkabeln mit geflochtenem Mantel, deren Länge ein Vielfaches von $\lambda$/2 betragen}
{Durch Oberwellen auf Speiseleitungen, deren Länge ein Vielfaches von $\lambda$/4 oder 5/8 $\lambda$ betragen}
\end{QQuestion}

}
\end{frame}

\begin{frame}
\only<1>{
\begin{PQuestion}{AG421}{Für welche Antennenimpedanz ist der folgende Balun-Transformator aus zweimal acht Windungen ausgelegt?}{\qty{400}{\ohm}}
{\qty{50}{\ohm}}
{\qty{100}{\ohm}}
{\qty{200}{\ohm}}
{\DARCimage{1.0\linewidth}{447include}}\end{PQuestion}

}
\only<2>{
\begin{PQuestion}{AG421}{Für welche Antennenimpedanz ist der folgende Balun-Transformator aus zweimal acht Windungen ausgelegt?}{\qty{400}{\ohm}}
{\qty{50}{\ohm}}
{\qty{100}{\ohm}}
{\textbf{\textcolor{DARCgreen}{\qty{200}{\ohm}}}}
{\DARCimage{1.0\linewidth}{447include}}\end{PQuestion}

}
\end{frame}

\begin{frame}
\only<1>{
\begin{PQuestion}{AG422}{Dargestellt ist ein HF-Übertrager (Balun). An den Anschlüssen a und b wird ein Faltdipol mit \qty{200}{\ohm} Impedanz angeschlossen. Welche Impedanz misst man zwischen den Anschlüssen a und m?}{\qty{100}{\ohm}}
{\qty{0}{\ohm}}
{\qty{50}{\ohm}}
{\qty{200}{\ohm}}
{\DARCimage{1.0\linewidth}{448include}}\end{PQuestion}

}
\only<2>{
\begin{PQuestion}{AG422}{Dargestellt ist ein HF-Übertrager (Balun). An den Anschlüssen a und b wird ein Faltdipol mit \qty{200}{\ohm} Impedanz angeschlossen. Welche Impedanz misst man zwischen den Anschlüssen a und m?}{\qty{100}{\ohm}}
{\qty{0}{\ohm}}
{\textbf{\textcolor{DARCgreen}{\qty{50}{\ohm}}}}
{\qty{200}{\ohm}}
{\DARCimage{1.0\linewidth}{448include}}\end{PQuestion}

}
\end{frame}

\begin{frame}
\only<1>{
\begin{PQuestion}{AG428}{Die Darstellung zeigt die bei Ankopplung eines Koaxialkabels an eine Antenne auftretenden Ströme. Wie kann man den als $I_3$ bezeichneten, unerwünschten Mantelstrom reduzieren?}{Einfügen eines Oberwellenfilters oder bei unsymmetrischen Störeinflüssen auch eines Spannungs-Baluns}
{Einfügen einer Gleichtaktdrossel oder bei symmetrischen Antennen auch eines Spannungs-Baluns}
{Auftrennen des Koax-Schirms vom Arm 2 der dargestellten Antenne (direkt an oder kurz vor der Antenne)}
{Herstellung einer direkten Verbindung zwischen dem Arm 1 der Antenne mit einer guten HF-Erde}
{\DARCimage{1.0\linewidth}{633include}}\end{PQuestion}

}
\only<2>{
\begin{PQuestion}{AG428}{Die Darstellung zeigt die bei Ankopplung eines Koaxialkabels an eine Antenne auftretenden Ströme. Wie kann man den als $I_3$ bezeichneten, unerwünschten Mantelstrom reduzieren?}{Einfügen eines Oberwellenfilters oder bei unsymmetrischen Störeinflüssen auch eines Spannungs-Baluns}
{\textbf{\textcolor{DARCgreen}{Einfügen einer Gleichtaktdrossel oder bei symmetrischen Antennen auch eines Spannungs-Baluns}}}
{Auftrennen des Koax-Schirms vom Arm 2 der dargestellten Antenne (direkt an oder kurz vor der Antenne)}
{Herstellung einer direkten Verbindung zwischen dem Arm 1 der Antenne mit einer guten HF-Erde}
{\DARCimage{1.0\linewidth}{633include}}\end{PQuestion}

}
\end{frame}

\begin{frame}
\only<1>{
\begin{QQuestion}{AG429}{Wodurch können Mantelwellen im Falle einer koax-gespeisten symmetrischen Antenne auftreten, obwohl ein Spannungs-Balun verwendet wird?}{Fehlanpassung durch Impedanztransformation des Baluns (z.\,B. 4:1-Spartransformator) sowie Stehwellen in der Zuleitung}
{Ungleichmäßige Belastung der Antenne durch Störeinflüsse der Umgebung (z.\,B. Bäume oder Gebäude) sowie Einkopplung in den Koax-Schirm}
{Dämpfung der Abstrahlung durch als Oberwellenfilter wirkenden Balun (z.\,B. 1:1-Transformator) sowie Einkopplung in den Koax-Schirm}
{Erhitzung des Ringkerns durch unzureichende Abschirmung (z.\,B. Kunststoffgehäuse) des Baluns sowie Stehwellen in der Zuleitung}
\end{QQuestion}

}
\only<2>{
\begin{QQuestion}{AG429}{Wodurch können Mantelwellen im Falle einer koax-gespeisten symmetrischen Antenne auftreten, obwohl ein Spannungs-Balun verwendet wird?}{Fehlanpassung durch Impedanztransformation des Baluns (z.\,B. 4:1-Spartransformator) sowie Stehwellen in der Zuleitung}
{\textbf{\textcolor{DARCgreen}{Ungleichmäßige Belastung der Antenne durch Störeinflüsse der Umgebung (z.\,B. Bäume oder Gebäude) sowie Einkopplung in den Koax-Schirm}}}
{Dämpfung der Abstrahlung durch als Oberwellenfilter wirkenden Balun (z.\,B. 1:1-Transformator) sowie Einkopplung in den Koax-Schirm}
{Erhitzung des Ringkerns durch unzureichende Abschirmung (z.\,B. Kunststoffgehäuse) des Baluns sowie Stehwellen in der Zuleitung}
\end{QQuestion}

}
\end{frame}%ENDCONTENT


\section{Umwegleitung}
\label{section:umwegleitung}
\begin{frame}%STARTCONTENT

\only<1>{
\begin{QQuestion}{AG420}{Ein Dipol soll mit einem Koaxkabel gleicher Impedanz gespeist werden. Dabei erreicht man einen Symmetriereffekt zum Beispiel durch~...}{Parallelschalten eines am freien Ende offenen $\lambda$/4 langen Leitungsstücks (Stub) am Speisepunkt der Antenne.}
{die Einfügung von Sperrkreisen (Traps) in den Dipol.}
{Symmetrierglieder wie Umwegleitung oder Balun.}
{Parallelschalten eines am freien Ende kurzgeschlossenen $\lambda$/2 langen Leitungsstücks (Stub) am Speisepunkt der Antenne.}
\end{QQuestion}

}
\only<2>{
\begin{QQuestion}{AG420}{Ein Dipol soll mit einem Koaxkabel gleicher Impedanz gespeist werden. Dabei erreicht man einen Symmetriereffekt zum Beispiel durch~...}{Parallelschalten eines am freien Ende offenen $\lambda$/4 langen Leitungsstücks (Stub) am Speisepunkt der Antenne.}
{die Einfügung von Sperrkreisen (Traps) in den Dipol.}
{\textbf{\textcolor{DARCgreen}{Symmetrierglieder wie Umwegleitung oder Balun.}}}
{Parallelschalten eines am freien Ende kurzgeschlossenen $\lambda$/2 langen Leitungsstücks (Stub) am Speisepunkt der Antenne.}
\end{QQuestion}

}
\end{frame}

\begin{frame}
\only<1>{
\begin{PQuestion}{AG423}{Was zeigt diese Darstellung?}{Sie zeigt einen symmetrischen \qty{60}{\ohm}-Schleifendipol mit einem koaxialen Leitungskreis, der als Sperrfilter zur Unterdrückung von unerwünschten Aussendungen eingesetzt ist.}
{Sie zeigt einen symmetrischen \qty{60}{\ohm}-Schleifendipol mit Koaxialkabel-Balun. Durch die Anordnung wird die symmetrische Antenne an ein unsymmetrisches \qty{60}{\ohm}-Antennenkabel angepasst.}
{Sie zeigt einen $\lambda$/2-Dipol mit symmetrierender $\lambda$/2-Umwegleitung. Durch die Anordnung wird der Fußpunktwiderstand der symmetrischen Antenne von \qty{120}{\ohm} an ein unsymmetrisches \qty{60}{\ohm}-Antennenkabel angepasst.}
{Sie zeigt einen $\lambda$/2-Faltdipol mit $\lambda$/2-Umwegleitung. Durch die Anordnung wird der Fußpunktwiderstand der symmetrischen Antenne von \qty{240}{\ohm} an ein unsymmetrisches \qty{60}{\ohm}-Antennenkabel angepasst.}
{\DARCimage{1.0\linewidth}{562include}}\end{PQuestion}

}
\only<2>{
\begin{PQuestion}{AG423}{Was zeigt diese Darstellung?}{Sie zeigt einen symmetrischen \qty{60}{\ohm}-Schleifendipol mit einem koaxialen Leitungskreis, der als Sperrfilter zur Unterdrückung von unerwünschten Aussendungen eingesetzt ist.}
{Sie zeigt einen symmetrischen \qty{60}{\ohm}-Schleifendipol mit Koaxialkabel-Balun. Durch die Anordnung wird die symmetrische Antenne an ein unsymmetrisches \qty{60}{\ohm}-Antennenkabel angepasst.}
{Sie zeigt einen $\lambda$/2-Dipol mit symmetrierender $\lambda$/2-Umwegleitung. Durch die Anordnung wird der Fußpunktwiderstand der symmetrischen Antenne von \qty{120}{\ohm} an ein unsymmetrisches \qty{60}{\ohm}-Antennenkabel angepasst.}
{\textbf{\textcolor{DARCgreen}{Sie zeigt einen $\lambda$/2-Faltdipol mit $\lambda$/2-Umwegleitung. Durch die Anordnung wird der Fußpunktwiderstand der symmetrischen Antenne von \qty{240}{\ohm} an ein unsymmetrisches \qty{60}{\ohm}-Antennenkabel angepasst.}}}
{\DARCimage{1.0\linewidth}{562include}}\end{PQuestion}

}
\end{frame}

\begin{frame}
\only<1>{
\begin{PQuestion}{AG424}{Zur Anpassung von Antennen werden häufig Umwegleitungen verwendet. Wie arbeitet die folgende Schaltung?}{Der $\lambda$/2-Faltdipol hat eine Impedanz von \qty{240}{\ohm}. Durch die $\lambda$/2-Umwegleitung erfolgt eine Widerstandstransformation von 4:1 mit Phasendrehung um \qty{360}{\degree}, womit an der Seite der Antennenleitung eine Ausgangsimpedanz von \qty{60}{\ohm} erreicht wird.}
{Der $\lambda$/2-Faltdipol hat an jedem seiner Anschlüsse eine Impedanz von \qty{120}{\ohm} gegen Erde. Durch die $\lambda$/2-Umwegleitung erfolgt eine 1:1-Widerstandstransformation mit Phasendrehung um \qty{180}{\degree}. An der Seite der Antennenleitung erfolgt eine phasenrichtige Parallelschaltung von 2~mal \qty{120}{\ohm} gegen Erde, womit eine Ausgangsimpedanz von \qty{60}{\ohm} erreicht wird.
}
{Der $\lambda$/2-Dipol hat eine Impedanz von \qty{60}{\ohm}. Durch die $\lambda$/2-Umwegleitung erfolgt eine Widerstandstransformation von 1:2 mit Phasendrehung um \qty{180}{\degree}. An der Seite der Antennenleitung erfolgt eine phasenrichtige Parallelschaltung von 2~mal \qty{120}{\ohm} gegen Erde, womit eine Ausgangsimpedanz von \qty{60}{\ohm} erreicht wird.}
{Der $\lambda$/2-Dipol hat eine Impedanz von \qty{240}{\ohm}. Durch die $\lambda$/2-Umwegleitung erfolgt eine Widerstandstransformation von 4:1 mit Phasendrehung um \qty{360}{\degree}, womit an der Seite der Antennenleitung eine Ausgangsimpedanz von \qty{60}{\ohm} erreicht wird.}
{\DARCimage{1.0\linewidth}{562include}}\end{PQuestion}

}
\only<2>{
\begin{PQuestion}{AG424}{Zur Anpassung von Antennen werden häufig Umwegleitungen verwendet. Wie arbeitet die folgende Schaltung?}{Der $\lambda$/2-Faltdipol hat eine Impedanz von \qty{240}{\ohm}. Durch die $\lambda$/2-Umwegleitung erfolgt eine Widerstandstransformation von 4:1 mit Phasendrehung um \qty{360}{\degree}, womit an der Seite der Antennenleitung eine Ausgangsimpedanz von \qty{60}{\ohm} erreicht wird.}
{\textbf{\textcolor{DARCgreen}{Der $\lambda$/2-Faltdipol hat an jedem seiner Anschlüsse eine Impedanz von \qty{120}{\ohm} gegen Erde. Durch die $\lambda$/2-Umwegleitung erfolgt eine 1:1-Widerstandstransformation mit Phasendrehung um \qty{180}{\degree}. An der Seite der Antennenleitung erfolgt eine phasenrichtige Parallelschaltung von 2~mal \qty{120}{\ohm} gegen Erde, womit eine Ausgangsimpedanz von \qty{60}{\ohm} erreicht wird.
}}}
{Der $\lambda$/2-Dipol hat eine Impedanz von \qty{60}{\ohm}. Durch die $\lambda$/2-Umwegleitung erfolgt eine Widerstandstransformation von 1:2 mit Phasendrehung um \qty{180}{\degree}. An der Seite der Antennenleitung erfolgt eine phasenrichtige Parallelschaltung von 2~mal \qty{120}{\ohm} gegen Erde, womit eine Ausgangsimpedanz von \qty{60}{\ohm} erreicht wird.}
{Der $\lambda$/2-Dipol hat eine Impedanz von \qty{240}{\ohm}. Durch die $\lambda$/2-Umwegleitung erfolgt eine Widerstandstransformation von 4:1 mit Phasendrehung um \qty{360}{\degree}, womit an der Seite der Antennenleitung eine Ausgangsimpedanz von \qty{60}{\ohm} erreicht wird.}
{\DARCimage{1.0\linewidth}{562include}}\end{PQuestion}

}
\end{frame}%ENDCONTENT


\title{DARC Amateurfunklehrgang Klasse A}
\author{Personenschutzabstand}
\institute{Deutscher Amateur Radio Club e.\,V.}
\begin{frame}
\maketitle
\end{frame}

\section{Effektive Strahlungsleistung (ERP) II}
\label{section:effektive_strahlungsleistung_erp_2}
\begin{frame}%STARTCONTENT

\only<1>{
\begin{QQuestion}{AG501}{Die äquivalente (effektive) Strahlungsleistung (ERP) ist~...}{das Produkt aus der Leistung, die unmittelbar der Antenne zugeführt wird, und ihrem Gewinnfaktor in einer Richtung, bezogen auf den isotropen Strahler.}
{das Produkt aus der Leistung, die unmittelbar der Antenne zugeführt wird, und ihrem Gewinnfaktor in einer Richtung, bezogen auf den Halbwellendipol.}
{die durchschnittliche Leistung, die ein Sender unter normalen Betriebsbedingungen während einer Periode der Hochfrequenzschwingung bei der höchsten Spitze der Modulationshüllkurve der Antennenspeiseleitung zuführt.}
{die durchschnittliche Leistung, die ein Sender unter normalen Betriebsbedingungen an die Antennenspeiseleitung während eines Zeitintervalls abgibt, das im Verhältnis zur Periode der tiefsten Modulationsfrequenz ausreichend lang ist.}
\end{QQuestion}

}
\only<2>{
\begin{QQuestion}{AG501}{Die äquivalente (effektive) Strahlungsleistung (ERP) ist~...}{das Produkt aus der Leistung, die unmittelbar der Antenne zugeführt wird, und ihrem Gewinnfaktor in einer Richtung, bezogen auf den isotropen Strahler.}
{\textbf{\textcolor{DARCgreen}{das Produkt aus der Leistung, die unmittelbar der Antenne zugeführt wird, und ihrem Gewinnfaktor in einer Richtung, bezogen auf den Halbwellendipol.}}}
{die durchschnittliche Leistung, die ein Sender unter normalen Betriebsbedingungen während einer Periode der Hochfrequenzschwingung bei der höchsten Spitze der Modulationshüllkurve der Antennenspeiseleitung zuführt.}
{die durchschnittliche Leistung, die ein Sender unter normalen Betriebsbedingungen an die Antennenspeiseleitung während eines Zeitintervalls abgibt, das im Verhältnis zur Periode der tiefsten Modulationsfrequenz ausreichend lang ist.}
\end{QQuestion}

}
\end{frame}

\begin{frame}
\only<1>{
\begin{QQuestion}{AG502}{Nach welcher der Antworten kann die ERP (Effective Radiated Power) berechnet werden?}{$P_{\symup{ERP}} = (P_{\symup{Sender}} - P_{\symup{Verluste}}) + G_{\symup{Antenne}}$ bezogen auf einen Halbwellendipol}
{$P_{\symup{ERP}} = (P_{\symup{Sender}} \cdot P_{\symup{Verluste}}) \cdot G_{\symup{Antenne}}$ bezogen auf einen isotropen Strahler}
{$P_{\symup{ERP}} = (P_{\symup{Sender}} - P_{\symup{Verluste}}) \cdot G_{\symup{Antenne}}$ bezogen auf einen Halbwellendipol}
{$P_{\symup{ERP}} = (P_{\symup{Sender}} + P_{\symup{Verluste}}) + G_{\symup{Antenne}}$ bezogen auf einen isotropen Strahler}
\end{QQuestion}

}
\only<2>{
\begin{QQuestion}{AG502}{Nach welcher der Antworten kann die ERP (Effective Radiated Power) berechnet werden?}{$P_{\symup{ERP}} = (P_{\symup{Sender}} - P_{\symup{Verluste}}) + G_{\symup{Antenne}}$ bezogen auf einen Halbwellendipol}
{$P_{\symup{ERP}} = (P_{\symup{Sender}} \cdot P_{\symup{Verluste}}) \cdot G_{\symup{Antenne}}$ bezogen auf einen isotropen Strahler}
{\textbf{\textcolor{DARCgreen}{$P_{\symup{ERP}} = (P_{\symup{Sender}} - P_{\symup{Verluste}}) \cdot G_{\symup{Antenne}}$ bezogen auf einen Halbwellendipol}}}
{$P_{\symup{ERP}} = (P_{\symup{Sender}} + P_{\symup{Verluste}}) + G_{\symup{Antenne}}$ bezogen auf einen isotropen Strahler}
\end{QQuestion}

}
\end{frame}

\begin{frame}
\only<1>{
\begin{QQuestion}{AG503}{Ein Sender für das \qty{630}{\m}-Band mit \qty{50}{\W} Ausgangsleistung ist mittels eines kurzen Koaxialkabels an eine Antenne mit 20~dBd Verlust angeschlossen. Welche ERP wird von der Antenne abgestrahlt?}{\qty{2,5}{\W}}
{\qty{5,0}{\W}}
{\qty{0,5}{\W}}
{\qty{50}{\W}}
\end{QQuestion}

}
\only<2>{
\begin{QQuestion}{AG503}{Ein Sender für das \qty{630}{\m}-Band mit \qty{50}{\W} Ausgangsleistung ist mittels eines kurzen Koaxialkabels an eine Antenne mit 20~dBd Verlust angeschlossen. Welche ERP wird von der Antenne abgestrahlt?}{\qty{2,5}{\W}}
{\qty{5,0}{\W}}
{\textbf{\textcolor{DARCgreen}{\qty{0,5}{\W}}}}
{\qty{50}{\W}}
\end{QQuestion}

}
\end{frame}

\begin{frame}
\frametitle{Lösungsweg}
\begin{itemize}
  \item gegeben: $P_S = 50W$
  \item gegeben: $a \approx 0W$
  \item gegeben: $g_d = -20dBd$
  \item gesucht: $P_{\textrm{ERP}}$
  \end{itemize}
    \pause
    $P_{\textrm{ERP}} = P_S \cdot 10^{\frac{g_d -- a}{10dB}} = 50W \cdot 10^{\frac{-20dBd -- 0W}{10dB}} = 50W \cdot 10^{-2} = 0,5W$



\end{frame}%ENDCONTENT


\section{Personenschutzabstand III}
\label{section:personenschutzabstand_3}
\begin{frame}%STARTCONTENT

\only<1>{
\begin{QQuestion}{AK102}{Durch welche Größe sind Beträge der elektrischen und magnetischen Feldstärke eines elektromagnetischen Feldes im Fernfeld miteinander verknüpft?}{Durch die Aufbauhöhe der Antenne}
{Durch den Wellenwiderstand im jeweiligen Medium }
{Durch die Ausbreitungsbedingungen in der Ionosphäre}
{Durch die Polarisationsrichtung der verwendeten Antenne}
\end{QQuestion}

}
\only<2>{
\begin{QQuestion}{AK102}{Durch welche Größe sind Beträge der elektrischen und magnetischen Feldstärke eines elektromagnetischen Feldes im Fernfeld miteinander verknüpft?}{Durch die Aufbauhöhe der Antenne}
{\textbf{\textcolor{DARCgreen}{Durch den Wellenwiderstand im jeweiligen Medium }}}
{Durch die Ausbreitungsbedingungen in der Ionosphäre}
{Durch die Polarisationsrichtung der verwendeten Antenne}
\end{QQuestion}

}
\end{frame}

\begin{frame}
\only<1>{
\begin{QQuestion}{AK104}{Wie errechnen Sie die Leistung am Einspeisepunkt der Antenne (Antenneneingangsleistung) bei bekannter Senderausgangsleistung?}{Die Antenneneingangsleistung ist der Spitzenwert der Senderausgangsleistung, also: $P_{\symup{Ant}}~=~\sqrt{2\cdot P_{\symup{Sender}}}$}
{Antenneneingangsleistung und Senderausgangsleistung sind gleich, da die Kabelverluste bei Amateurfunkstationen vernachlässigbar klein sind, d. h. es gilt: $P_{\symup{Ant}}~=~P_{\symup{Sender}}$}
{Sie ermitteln die Verluste zwischen Senderausgang und Antenneneingang und berechnen aus dieser Dämpfung einen Dämpfungsfaktor$~D$; die Antenneneingangsleistung ist dann: $P_{\symup{Ant}}~=~D\cdot P_{\symup{Sender}}$}
{Die Antenneneingangsleistung ist der Spitzen-Spitzen-Wert der Senderausgangsleistung, also: $P_{\symup{Ant}}~=~2\cdot\sqrt{2\cdot P_{\symup{Sender}}}$}
\end{QQuestion}

}
\only<2>{
\begin{QQuestion}{AK104}{Wie errechnen Sie die Leistung am Einspeisepunkt der Antenne (Antenneneingangsleistung) bei bekannter Senderausgangsleistung?}{Die Antenneneingangsleistung ist der Spitzenwert der Senderausgangsleistung, also: $P_{\symup{Ant}}~=~\sqrt{2\cdot P_{\symup{Sender}}}$}
{Antenneneingangsleistung und Senderausgangsleistung sind gleich, da die Kabelverluste bei Amateurfunkstationen vernachlässigbar klein sind, d. h. es gilt: $P_{\symup{Ant}}~=~P_{\symup{Sender}}$}
{\textbf{\textcolor{DARCgreen}{Sie ermitteln die Verluste zwischen Senderausgang und Antenneneingang und berechnen aus dieser Dämpfung einen Dämpfungsfaktor$~D$; die Antenneneingangsleistung ist dann: $P_{\symup{Ant}}~=~D\cdot P_{\symup{Sender}}$}}}
{Die Antenneneingangsleistung ist der Spitzen-Spitzen-Wert der Senderausgangsleistung, also: $P_{\symup{Ant}}~=~2\cdot\sqrt{2\cdot P_{\symup{Sender}}}$}
\end{QQuestion}

}
\end{frame}

\begin{frame}
\only<1>{
\begin{QQuestion}{AK115}{Eine Amateurfunkstelle sendet in FM mit einer äquivalenten Strahlungsleistung (ERP) von \qty{100}{\W}. Wie groß ist die Feldstärke im freien Raum in einer Entfernung von \qty{100}{\m}?}{\qty{0,7}{\V}/m}
{\qty{0,5}{\V}/m}
{\qty{0,43}{\V}/m}
{\qty{0,55}{\V}/m}
\end{QQuestion}

}
\only<2>{
\begin{QQuestion}{AK115}{Eine Amateurfunkstelle sendet in FM mit einer äquivalenten Strahlungsleistung (ERP) von \qty{100}{\W}. Wie groß ist die Feldstärke im freien Raum in einer Entfernung von \qty{100}{\m}?}{\textbf{\textcolor{DARCgreen}{\qty{0,7}{\V}/m}}}
{\qty{0,5}{\V}/m}
{\qty{0,43}{\V}/m}
{\qty{0,55}{\V}/m}
\end{QQuestion}

}
\end{frame}

\begin{frame}
\frametitle{Lösungsweg}
\begin{itemize}
  \item gegeben: $P_{ERP} = 100W$
  \item gegeben: $d = 100m$
  \item gesucht: $E$
  \end{itemize}
    \pause
    $P_{EIRP} = P_{ERP} \cdot 1,64 = 100W \cdot 1,64 = 164W$
    \pause
    $E = \frac{\sqrt{30Ω \cdot P_{EIRP}}}{d} = \frac{\sqrt{30Ω \cdot 164W}}{100m} = 0,7\frac{V}{m}$



\end{frame}

\begin{frame}
\only<1>{
\begin{QQuestion}{AK114}{Eine vertikale Dipol-Antenne wird mit \qty{10}{\W} Sendeleistung im \qty{70}{\cm}-Band direkt gespeist. Welche elektrische Feldstärke ergibt sich bei Freiraumausbreitung in \qty{10}{\m} Entfernung in etwa?}{\qty{1,7}{\V}/m}
{\qty{8,9}{\V}/m}
{\qty{0,4}{\V}/m}
{\qty{2,2}{\V}/m}
\end{QQuestion}

}
\only<2>{
\begin{QQuestion}{AK114}{Eine vertikale Dipol-Antenne wird mit \qty{10}{\W} Sendeleistung im \qty{70}{\cm}-Band direkt gespeist. Welche elektrische Feldstärke ergibt sich bei Freiraumausbreitung in \qty{10}{\m} Entfernung in etwa?}{\qty{1,7}{\V}/m}
{\qty{8,9}{\V}/m}
{\qty{0,4}{\V}/m}
{\textbf{\textcolor{DARCgreen}{\qty{2,2}{\V}/m}}}
\end{QQuestion}

}
\end{frame}

\begin{frame}
\frametitle{Lösungsweg}
\begin{itemize}
  \item gegeben: $P_{ERP} = 10W$
  \item gegeben: $d = 10m$
  \item gesucht: $E$
  \end{itemize}
    \pause
    $P_{EIRP} = P_{ERP} \cdot 1,64 = 10W \cdot 1,64 = 16,4W$
    \pause
    $E = \frac{\sqrt{30Ω \cdot P_{EIRP}}}{d} = \frac{\sqrt{30Ω \cdot 16,4W}}{10m} = 2,2\frac{V}{m}$



\end{frame}

\begin{frame}
\only<1>{
\begin{QQuestion}{AK113}{Eine Yagi-Uda-Antenne mit \qty{12,15}{\dBi} Antennengewinn wird mit \qty{250}{\W} Sendeleistung im \qty{2}{\m}-Band direkt gespeist. Welche elektrische Feldstärke ergibt sich bei Freiraumausbreitung in \qty{30}{\m} Entfernung in etwa?}{\qty{9,1}{\V}/m}
{\qty{11,7}{\V}/m}
{\qty{15,0}{\V}/m}
{\qty{10,1}{\V}/m}
\end{QQuestion}

}
\only<2>{
\begin{QQuestion}{AK113}{Eine Yagi-Uda-Antenne mit \qty{12,15}{\dBi} Antennengewinn wird mit \qty{250}{\W} Sendeleistung im \qty{2}{\m}-Band direkt gespeist. Welche elektrische Feldstärke ergibt sich bei Freiraumausbreitung in \qty{30}{\m} Entfernung in etwa?}{\qty{9,1}{\V}/m}
{\textbf{\textcolor{DARCgreen}{\qty{11,7}{\V}/m}}}
{\qty{15,0}{\V}/m}
{\qty{10,1}{\V}/m}
\end{QQuestion}

}
\end{frame}

\begin{frame}
\frametitle{Lösungsweg}
\begin{itemize}
  \item gegeben: $g_i = 12,15dBi$
  \item gegeben: $P_A = 250W$
  \item gegeben: $d = 30m$
  \item gesucht: $E$
  \end{itemize}
    \pause
    $G_i = 10^{\frac{g_i}{10dB}} = 10^{\frac{12,15dBi}{10dB}} = 16,4$
    \pause
    $E = \frac{\sqrt{30Ω \cdot P_A \cdot G_i}}{d} = \frac{\sqrt{30Ω \cdot 250W \cdot 16,4}}{30m} = \frac{350V}{30m} \approx 11,7\frac{V}{m}$



\end{frame}

\begin{frame}
\only<1>{
\begin{QQuestion}{AK107}{Sie betreiben eine Amateurfunkstelle auf dem \qty{2}{\m}-Band im Modulationsverfahren FM mit einer Rundstrahlantenne mit \qty{6}{\decibel} Gewinn bezogen auf einen Dipol. Wie hoch darf die maximale Ausgangsleistung Ihres Senders unter Vernachlässigung der Kabeldämpfung sein, wenn der Grenzwert für den Personenschutz \qty{28}{\V\per\m} und der zur Verfügung stehende Sicherheitsabstand \qty{5}{\m} beträgt?}{ca. \qty{75}{\W}}
{ca. \qty{100}{\W}}
{ca. \qty{160}{\W}}
{ca. \qty{265}{\W}}
\end{QQuestion}

}
\only<2>{
\begin{QQuestion}{AK107}{Sie betreiben eine Amateurfunkstelle auf dem \qty{2}{\m}-Band im Modulationsverfahren FM mit einer Rundstrahlantenne mit \qty{6}{\decibel} Gewinn bezogen auf einen Dipol. Wie hoch darf die maximale Ausgangsleistung Ihres Senders unter Vernachlässigung der Kabeldämpfung sein, wenn der Grenzwert für den Personenschutz \qty{28}{\V\per\m} und der zur Verfügung stehende Sicherheitsabstand \qty{5}{\m} beträgt?}{ca. \qty{75}{\W}}
{\textbf{\textcolor{DARCgreen}{ca. \qty{100}{\W}}}}
{ca. \qty{160}{\W}}
{ca. \qty{265}{\W}}
\end{QQuestion}

}
\end{frame}

\begin{frame}
\frametitle{Lösungsweg}
\begin{itemize}
  \item gegeben: $g_d = 6dBd$
  \item gegeben: $E = 28\frac{V}{m}$
  \item gegeben: $d = 5m$
  \item gesucht: $P_S$
  \end{itemize}
    \pause
    $E = \frac{\sqrt{30Ω \cdot P_{EIRP}}}{d} \Rightarrow P_{EIRP} = \frac{(E \cdot d)^2}{30Ω} = \frac{(28\frac{V}{m} \cdot 5m)^2}{30Ω} = 653W$
    \pause
    $P_{EIRP} = P_S \cdot 10^{\frac{g_d -- a + 2,15dB}{10dB}} \Rightarrow P_S = \frac{P_{EIRP}}{10^{\frac{g_d -- a + 2,15dB}{10dB}}} = \frac{653W}{10^{\frac{6dBd -- 0 + 2,15dB}{10dB}}} = \frac{653W}{6,53} \approx 100W$



\end{frame}%ENDCONTENT


\section{Näherungsformel II}
\label{section:naeherungsformel_2}
\begin{frame}%STARTCONTENT

\only<1>{
\begin{QQuestion}{AK106}{Sie möchten den Personenschutz-Sicherheitsabstand für die Antenne Ihrer Amateurfunkstelle für das \qty{10}{\m}-Band und das Übertragungsverfahren RTTY berechnen. Der Grenzwert im Fall des Personenschutzes beträgt \qty{28}{\V}/m. Sie betreiben einen Dipol, der von einem Sender mit einer Leistung von \qty{100}{\W} über ein Koaxialkabel gespeist wird. Die Kabeldämpfung sei vernachlässigbar. Wie groß muss der Sicherheitsabstand sein?}{\qty{2,50}{\m}}
{\qty{1,96}{\m}}
{\qty{5,01}{\m}}
{\qty{13,7}{\m}}
\end{QQuestion}

}
\only<2>{
\begin{QQuestion}{AK106}{Sie möchten den Personenschutz-Sicherheitsabstand für die Antenne Ihrer Amateurfunkstelle für das \qty{10}{\m}-Band und das Übertragungsverfahren RTTY berechnen. Der Grenzwert im Fall des Personenschutzes beträgt \qty{28}{\V}/m. Sie betreiben einen Dipol, der von einem Sender mit einer Leistung von \qty{100}{\W} über ein Koaxialkabel gespeist wird. Die Kabeldämpfung sei vernachlässigbar. Wie groß muss der Sicherheitsabstand sein?}{\textbf{\textcolor{DARCgreen}{\qty{2,50}{\m}}}}
{\qty{1,96}{\m}}
{\qty{5,01}{\m}}
{\qty{13,7}{\m}}
\end{QQuestion}

}
\end{frame}

\begin{frame}
\frametitle{Lösungsweg}
\begin{itemize}
  \item gegeben: $E = 28\frac{V}{m}$
  \item gegeben: $P_S = P_A = 100W$
  \item gegeben: $G_i = 1,64$
  \item gesucht: $d$
  \end{itemize}
    \pause
    $E = \frac{\sqrt{30Ω \cdot P_A \cdot G_i}}{d} \Rightarrow d = \frac{\sqrt{30Ω \cdot P_A \cdot G_i}}{E} = \frac{\sqrt{30Ω \cdot 100W \cdot 1,64}}{28\frac{V}{m}} = 2,5m$



\end{frame}

\begin{frame}
\only<1>{
\begin{QQuestion}{AK108}{Sie möchten den Personenschutz-Sicherheitsabstand für die Antenne Ihrer Amateurfunkstelle für das \qty{20}{\m}-Band und das Übertragungsverfahren RTTY berechnen. Der Grenzwert im Fall des Personenschutzes beträgt \qty{28}{\V}/m. Sie betreiben einen Dipol, der von einem Sender mit einer Leistung von \qty{300}{\W} über ein Koaxialkabel gespeist wird. Die Kabeldämpfung beträgt \qty{0,5}{\decibel}. Wie groß ist der Sicherheitsabstand?}{\qty{4,97}{\m}}
{\qty{4,10}{\m}}
{\qty{3,20}{\m}}
{\qty{2,39}{\m}}
\end{QQuestion}

}
\only<2>{
\begin{QQuestion}{AK108}{Sie möchten den Personenschutz-Sicherheitsabstand für die Antenne Ihrer Amateurfunkstelle für das \qty{20}{\m}-Band und das Übertragungsverfahren RTTY berechnen. Der Grenzwert im Fall des Personenschutzes beträgt \qty{28}{\V}/m. Sie betreiben einen Dipol, der von einem Sender mit einer Leistung von \qty{300}{\W} über ein Koaxialkabel gespeist wird. Die Kabeldämpfung beträgt \qty{0,5}{\decibel}. Wie groß ist der Sicherheitsabstand?}{\qty{4,97}{\m}}
{\textbf{\textcolor{DARCgreen}{\qty{4,10}{\m}}}}
{\qty{3,20}{\m}}
{\qty{2,39}{\m}}
\end{QQuestion}

}
\end{frame}

\begin{frame}
\frametitle{Lösungsweg}
\begin{itemize}
  \item gegeben: $E = 28\frac{V}{m}$
  \item gegeben: $P_S = 300W$
  \item gegeben: $a = 0,5dB$
  \item gegeben: $g_d = 0dBd$
  \item gesucht: $d$
  \end{itemize}
    \pause
    $P_{EIRP} = P_S \cdot 10^{\frac{g_d -a + 2,15dB}{10dB}} = 300W \cdot 10^{\frac{0dBd -- 0,5dB + 2,15dB}{10dB}} = 438,7W$
    \pause
    $E = \frac{\sqrt{30Ω \cdot P_{EIRP}}}{d} \Rightarrow d = \frac{\sqrt{30Ω \cdot P_{EIRP}}}{E} = \frac{\sqrt{30Ω \cdot 438,7W}}{28\frac{V}{m}} = 4,10m$



\end{frame}

\begin{frame}
\only<1>{
\begin{QQuestion}{AK109}{Sie möchten den Personenschutz-Sicherheitsabstand für die Antenne Ihrer Amateurfunkstelle für das \qty{20}{\m}-Band und das Übertragungsverfahren RTTY berechnen. Der Grenzwert im Fall des Personenschutzes beträgt \qty{28}{\V}/m. Sie betreiben einen Dipol, der von einem Sender mit einer Leistung von \qty{700}{\W} über ein Koaxialkabel gespeist wird. Die Kabeldämpfung beträgt \qty{0,5}{\decibel}. Wie groß ist der Sicherheitsabstand?}{\qty{6,26}{\m}}
{\qty{7,36}{\m}}
{\qty{4,87}{\m}}
{\qty{5,62}{\m}}
\end{QQuestion}

}
\only<2>{
\begin{QQuestion}{AK109}{Sie möchten den Personenschutz-Sicherheitsabstand für die Antenne Ihrer Amateurfunkstelle für das \qty{20}{\m}-Band und das Übertragungsverfahren RTTY berechnen. Der Grenzwert im Fall des Personenschutzes beträgt \qty{28}{\V}/m. Sie betreiben einen Dipol, der von einem Sender mit einer Leistung von \qty{700}{\W} über ein Koaxialkabel gespeist wird. Die Kabeldämpfung beträgt \qty{0,5}{\decibel}. Wie groß ist der Sicherheitsabstand?}{\textbf{\textcolor{DARCgreen}{\qty{6,26}{\m}}}}
{\qty{7,36}{\m}}
{\qty{4,87}{\m}}
{\qty{5,62}{\m}}
\end{QQuestion}

}
\end{frame}

\begin{frame}
\frametitle{Lösungsweg}
\begin{itemize}
  \item gegeben: $E = 28\frac{V}{m}$
  \item gegeben: $P_S = 700W$
  \item gegeben: $a = 0,5dB$
  \item gegeben: $g_d = 0dBd$
  \item gesucht: $d$
  \end{itemize}
    \pause
    $P_{EIRP} = P_S \cdot 10^{\frac{g_d -a + 2,15dB}{10dB}} = 700W \cdot 10^{\frac{0dBd -- 0,5dB + 2,15dB}{10dB}} = 1023,5W$
    \pause
    $E = \frac{\sqrt{30Ω \cdot P_{EIRP}}}{d} \Rightarrow d = \frac{\sqrt{30Ω \cdot P_{EIRP}}}{E} = \frac{\sqrt{30Ω \cdot 1023,5W}}{28\frac{V}{m}} = 6,26m$



\end{frame}

\begin{frame}
\only<1>{
\begin{QQuestion}{AK110}{Sie möchten den Personenschutz-Sicherheitsabstand für die Antenne Ihrer Amateurfunkstelle in Hauptstrahlrichtung für das \qty{2}{\m}-Band und die Modulationsverfahren FM berechnen. Der Grenzwert im Fall des Personenschutzes beträgt \qty{28}{\V}/m. Sie betreiben eine Yagi-Uda-Antenne mit einem Gewinn von $11,5~$dBd. Die Antenne wird von einem Sender mit einer Leistung von \qty{75}{\W} über ein Koaxialkabel gespeist. Die Kabeldämpfung beträgt \qty{1,5}{\decibel}. Wie groß muss der Sicherheitsabstand sein?}{\qty{6,86}{\m}}
{\qty{5,35}{\m}}
{\qty{2,17}{\m}}
{\qty{22,09}{\m}}
\end{QQuestion}

}
\only<2>{
\begin{QQuestion}{AK110}{Sie möchten den Personenschutz-Sicherheitsabstand für die Antenne Ihrer Amateurfunkstelle in Hauptstrahlrichtung für das \qty{2}{\m}-Band und die Modulationsverfahren FM berechnen. Der Grenzwert im Fall des Personenschutzes beträgt \qty{28}{\V}/m. Sie betreiben eine Yagi-Uda-Antenne mit einem Gewinn von $11,5~$dBd. Die Antenne wird von einem Sender mit einer Leistung von \qty{75}{\W} über ein Koaxialkabel gespeist. Die Kabeldämpfung beträgt \qty{1,5}{\decibel}. Wie groß muss der Sicherheitsabstand sein?}{\textbf{\textcolor{DARCgreen}{\qty{6,86}{\m}}}}
{\qty{5,35}{\m}}
{\qty{2,17}{\m}}
{\qty{22,09}{\m}}
\end{QQuestion}

}
\end{frame}

\begin{frame}
\frametitle{Lösungsweg}
\begin{itemize}
  \item gegeben: $E = 28\frac{V}{m}$
  \item gegeben: $P_S = 75W$
  \item gegeben: $a = 1,5dB$
  \item gegeben: $g_d = 11,5dBd$
  \item gesucht: $d$
  \end{itemize}
    \pause
    $P_{EIRP} = P_S \cdot 10^{\frac{g_d -a + 2,15dB}{10dB}} = 75W \cdot 10^{\frac{11,5dBd -- 1,5dB + 2,15dB}{10dB}} = 1230,4W$
    \pause
    $E = \frac{\sqrt{30Ω \cdot P_{EIRP}}}{d} \Rightarrow d = \frac{\sqrt{30Ω \cdot P_{EIRP}}}{E} = \frac{\sqrt{30Ω \cdot 1230,4W}}{28\frac{V}{m}} = 6,86m$



\end{frame}

\begin{frame}
\only<1>{
\begin{QQuestion}{AK111}{Sie möchten den Personenschutz-Sicherheitsabstand für die Antenne Ihrer Amateurfunkstelle für das \qty{2}{\m}-Band und das Modulationsverfahren FM berechnen. Der Grenzwert im Fall des Personenschutzes beträgt \qty{28}{\V}/m. Sie betreiben eine Yagi-Uda-Antenne mit einem Gewinn von 10,5 dBd. Die Antenne wird von einem Sender mit einer Leistung von \qty{100}{\W} über ein Koaxialkabel gespeist. Die Kabeldämpfung beträgt \qty{1,5}{\decibel}. Wie groß ist der Sicherheitsabstand?}{\qty{7,1}{\m}}
{\qty{6,6}{\m}}
{\qty{8,4}{\m}}
{\qty{5,6}{\m}}
\end{QQuestion}

}
\only<2>{
\begin{QQuestion}{AK111}{Sie möchten den Personenschutz-Sicherheitsabstand für die Antenne Ihrer Amateurfunkstelle für das \qty{2}{\m}-Band und das Modulationsverfahren FM berechnen. Der Grenzwert im Fall des Personenschutzes beträgt \qty{28}{\V}/m. Sie betreiben eine Yagi-Uda-Antenne mit einem Gewinn von 10,5 dBd. Die Antenne wird von einem Sender mit einer Leistung von \qty{100}{\W} über ein Koaxialkabel gespeist. Die Kabeldämpfung beträgt \qty{1,5}{\decibel}. Wie groß ist der Sicherheitsabstand?}{\textbf{\textcolor{DARCgreen}{\qty{7,1}{\m}}}}
{\qty{6,6}{\m}}
{\qty{8,4}{\m}}
{\qty{5,6}{\m}}
\end{QQuestion}

}
\end{frame}

\begin{frame}
\frametitle{Lösungsweg}
\begin{itemize}
  \item gegeben: $E = 28\frac{V}{m}$
  \item gegeben: $P_S = 100W$
  \item gegeben: $a = 1,5dB$
  \item gegeben: $g_d = 10,5dBd$
  \item gesucht: $d$
  \end{itemize}
    \pause
    $P_{EIRP} = P_S \cdot 10^{\frac{g_d -a + 2,15dB}{10dB}} = 100W \cdot 10^{\frac{10,5dBd -- 1,5dB + 2,15dB}{10dB}} = 1303,2W$
    \pause
    $E = \frac{\sqrt{30Ω \cdot P_{EIRP}}}{d} \Rightarrow d = \frac{\sqrt{30Ω \cdot P_{EIRP}}}{E} = \frac{\sqrt{30Ω \cdot 1303,2W}}{28\frac{V}{m}} = 7,1m$



\end{frame}

\begin{frame}
\only<1>{
\begin{QQuestion}{AK112}{Sie möchten den Personenschutz-Sicherheitsabstand für das \qty{13}{\cm}-Band und das Modulationsverfahren FM berechnen. Der Grenzwert im Fall des Personenschutzes beträgt \qty{61}{\V}/m. Sie betreiben einen Parabolspiegel mit einem Gewinn von 18~dBd. Die Antenne wird von einem Sender mit einer Leistung von \qty{40}{\W} über ein PE-Schaum-Massivschirm-Kabel mit einer Dämpfung von \qty{2}{\decibel} gespeist. Wie groß muss der Personenschutz-Sicherheitsabstand in Hauptstrahlrichtung sein?}{\qty{14,5}{\m}}
{\qty{5,8}{\m}}
{\qty{4,6}{\m}}
{\qty{3,6}{\m}}
\end{QQuestion}

}
\only<2>{
\begin{QQuestion}{AK112}{Sie möchten den Personenschutz-Sicherheitsabstand für das \qty{13}{\cm}-Band und das Modulationsverfahren FM berechnen. Der Grenzwert im Fall des Personenschutzes beträgt \qty{61}{\V}/m. Sie betreiben einen Parabolspiegel mit einem Gewinn von 18~dBd. Die Antenne wird von einem Sender mit einer Leistung von \qty{40}{\W} über ein PE-Schaum-Massivschirm-Kabel mit einer Dämpfung von \qty{2}{\decibel} gespeist. Wie groß muss der Personenschutz-Sicherheitsabstand in Hauptstrahlrichtung sein?}{\qty{14,5}{\m}}
{\qty{5,8}{\m}}
{\textbf{\textcolor{DARCgreen}{\qty{4,6}{\m}}}}
{\qty{3,6}{\m}}
\end{QQuestion}

}
\end{frame}

\begin{frame}
\frametitle{Lösungsweg}
\begin{itemize}
  \item gegeben: $E = 61\frac{V}{m}$
  \item gegeben: $P_S = 40W$
  \item gegeben: $a = 2dB$
  \item gegeben: $g_d = 18dBd$
  \item gesucht: $d$
  \end{itemize}
    \pause
    $P_{EIRP} = P_S \cdot 10^{\frac{g_d -a + 2,15dB}{10dB}} = 40W \cdot 10^{\frac{18dBd -- 2dB + 2,15dB}{10dB}} = 2612,5W$
    \pause
    $E = \frac{\sqrt{30Ω \cdot P_{EIRP}}}{d} \Rightarrow d = \frac{\sqrt{30Ω \cdot P_{EIRP}}}{E} = \frac{\sqrt{30Ω \cdot 2612,5W}}{61\frac{V}{m}} = 4,6m$



\end{frame}%ENDCONTENT


\section{Nahfeld}
\label{section:nahfeld}
\begin{frame}%STARTCONTENT

\only<1>{
\begin{QQuestion}{AK101}{Warum ist im Nahfeld einer Strahlungsquelle keine einfache Umrechnung zwischen den Feldgrößen~E und~H und damit auch keine vereinfachte Berechnung des Schutzabstandes möglich?}{Weil die elektrische und die magnetische Feldstärke im Nahfeld nicht senkrecht zur Ausbreitungsrichtung stehen und auf Grund des Einflusses der Erdoberfläche eine Phasendifferenz von größer \qty{180}{\degree} aufweisen.}
{Weil die elektrische und die magnetische Feldstärke im Nahfeld immer senkrecht aufeinander stehen und eine Phasendifferenz von \qty{90}{\degree} aufweisen.}
{Weil die elektrische und die magnetische Feldstärke im Nahfeld keine konstante Phasenbeziehung zueinander aufweisen.}
{Weil die elektrische und die magnetische Feldstärke im Nahfeld nicht exakt senkrecht aufeinander stehen und sich durch die nicht ideale Leitfähigkeit des Erdbodens am Sendeort der Feldwellenwiderstand des freien Raumes verändert.}
\end{QQuestion}

}
\only<2>{
\begin{QQuestion}{AK101}{Warum ist im Nahfeld einer Strahlungsquelle keine einfache Umrechnung zwischen den Feldgrößen~E und~H und damit auch keine vereinfachte Berechnung des Schutzabstandes möglich?}{Weil die elektrische und die magnetische Feldstärke im Nahfeld nicht senkrecht zur Ausbreitungsrichtung stehen und auf Grund des Einflusses der Erdoberfläche eine Phasendifferenz von größer \qty{180}{\degree} aufweisen.}
{Weil die elektrische und die magnetische Feldstärke im Nahfeld immer senkrecht aufeinander stehen und eine Phasendifferenz von \qty{90}{\degree} aufweisen.}
{\textbf{\textcolor{DARCgreen}{Weil die elektrische und die magnetische Feldstärke im Nahfeld keine konstante Phasenbeziehung zueinander aufweisen.}}}
{Weil die elektrische und die magnetische Feldstärke im Nahfeld nicht exakt senkrecht aufeinander stehen und sich durch die nicht ideale Leitfähigkeit des Erdbodens am Sendeort der Feldwellenwiderstand des freien Raumes verändert.}
\end{QQuestion}

}
\end{frame}

\begin{frame}
\only<1>{
\begin{QQuestion}{AK103}{In welchem Fall hat die folgende Formel zur Berechnung des Sicherheitsabstandes Gültigkeit und was sollten Sie tun, wenn die Gültigkeit nicht mehr sichergestellt ist? $d = \frac{\sqrt{\qty{30}{\ohm}\cdot P_{\symup{EIRP}}}}{E}$}{Die Formel gilt nur für Abstände $d > \frac{\lambda}{2\cdot\pi}$ bei horizontal polarisierten Antennen.
Bei kleineren Abständen und immer bei vertikal polarisierten Antennen muss der Sicherheitsabstand durch zum Beispiel Feldstärkemessungen oder Nahfeldberechnungen (Simulationen) ermittelt werden.}
{Im Bereich von Amateurfunkstellen ist der Unterschied zwischen Nah- und Fernfeld so gering, dass obige Formel, die eigentlich nur im Fernfeld gilt, trotzdem für alle Raumbereiche verwendet werden kann.}
{Die Formel gilt nur für Abstände $d > \frac{\lambda}{2\cdot\pi}$ bei den meisten Antennenformen (z.~B. Dipol-Antennen). Für Antennen, die z.~B. geometrisch klein im Verhältnis zur Wellenlänge sind und/oder in kürzerem Abstand zur Antenne muss der Sicherheitsabstand zum Beispiel durch Feldstärkemessungen oder Nahfeldberechnungen (Simulationen) ermittelt werden.}
{Die Formel gilt nur für Abstände $d > \frac{\lambda}{2\cdot\pi}$ bei vertikal polarisierten Antennen.
Bei kleineren Abständen und immer bei horizontal polarisierten Antennen muss der Sicherheitsabstand durch zum Beispiel Feldstärkemessungen oder Nahfeldberechnungen (Simulationen) ermittelt werden.}
\end{QQuestion}

}
\only<2>{
\begin{QQuestion}{AK103}{In welchem Fall hat die folgende Formel zur Berechnung des Sicherheitsabstandes Gültigkeit und was sollten Sie tun, wenn die Gültigkeit nicht mehr sichergestellt ist? $d = \frac{\sqrt{\qty{30}{\ohm}\cdot P_{\symup{EIRP}}}}{E}$}{Die Formel gilt nur für Abstände $d > \frac{\lambda}{2\cdot\pi}$ bei horizontal polarisierten Antennen.
Bei kleineren Abständen und immer bei vertikal polarisierten Antennen muss der Sicherheitsabstand durch zum Beispiel Feldstärkemessungen oder Nahfeldberechnungen (Simulationen) ermittelt werden.}
{Im Bereich von Amateurfunkstellen ist der Unterschied zwischen Nah- und Fernfeld so gering, dass obige Formel, die eigentlich nur im Fernfeld gilt, trotzdem für alle Raumbereiche verwendet werden kann.}
{\textbf{\textcolor{DARCgreen}{Die Formel gilt nur für Abstände $d > \frac{\lambda}{2\cdot\pi}$ bei den meisten Antennenformen (z.~B. Dipol-Antennen). Für Antennen, die z.~B. geometrisch klein im Verhältnis zur Wellenlänge sind und/oder in kürzerem Abstand zur Antenne muss der Sicherheitsabstand zum Beispiel durch Feldstärkemessungen oder Nahfeldberechnungen (Simulationen) ermittelt werden.}}}
{Die Formel gilt nur für Abstände $d > \frac{\lambda}{2\cdot\pi}$ bei vertikal polarisierten Antennen.
Bei kleineren Abständen und immer bei horizontal polarisierten Antennen muss der Sicherheitsabstand durch zum Beispiel Feldstärkemessungen oder Nahfeldberechnungen (Simulationen) ermittelt werden.}
\end{QQuestion}

}
\end{frame}%ENDCONTENT


\section{Fernfeld}
\label{section:fernfeld}
\begin{frame}%STARTCONTENT
\end{frame}%ENDCONTENT


\section{Personenschutz bei Richtantennen}
\label{section:personenschutzabstand_richtantennen}
\begin{frame}%STARTCONTENT

\only<1>{
\begin{QQuestion}{AK105}{An der Spitze Ihres Antennenmastes befindet sich eine Yagi-Uda-Antenne, deren Sicherheitsabstand in Hauptstrahlrichtung \qty{20}{\m} beträgt. Schräg unterhalb dieser Antenne befindet sich ein nicht kontrollierbarer Bereich. Sie ermitteln einen kritischen Winkel von \qty{40}{\degree}. Das vertikale Strahlungsdiagramm der Antenne weist bei diesem Winkel eine Dämpfung von \qty{6}{\decibel} auf. Auf welchen Wert verringert sich dann rechnerisch der Sicherheitsabstand bei \qty{40}{\degree}?}{Er verringert sich auf \qty{10}{\m}.}
{Er verringert sich auf \qty{3,33}{\m}.}
{Er verringert sich auf \qty{5,02}{\m}.}
{Er verringert sich nicht.}
\end{QQuestion}

}
\only<2>{
\begin{QQuestion}{AK105}{An der Spitze Ihres Antennenmastes befindet sich eine Yagi-Uda-Antenne, deren Sicherheitsabstand in Hauptstrahlrichtung \qty{20}{\m} beträgt. Schräg unterhalb dieser Antenne befindet sich ein nicht kontrollierbarer Bereich. Sie ermitteln einen kritischen Winkel von \qty{40}{\degree}. Das vertikale Strahlungsdiagramm der Antenne weist bei diesem Winkel eine Dämpfung von \qty{6}{\decibel} auf. Auf welchen Wert verringert sich dann rechnerisch der Sicherheitsabstand bei \qty{40}{\degree}?}{\textbf{\textcolor{DARCgreen}{Er verringert sich auf \qty{10}{\m}.}}}
{Er verringert sich auf \qty{3,33}{\m}.}
{Er verringert sich auf \qty{5,02}{\m}.}
{Er verringert sich nicht.}
\end{QQuestion}

}
\end{frame}

\begin{frame}
\frametitle{Lösungsweg}
\end{frame}%ENDCONTENT


\title{DARC Amateurfunklehrgang Klasse A}
\author{Sicherheit}
\institute{Deutscher Amateur Radio Club e.\,V.}
\begin{frame}
\maketitle
\end{frame}

\section{Öffnen elektrischer Geräte II}
\label{section:elektrische_geaete_oeffnen_2}
\begin{frame}%STARTCONTENT

\only<1>{
\begin{QQuestion}{AK201}{Bei der Fehlersuche in einer defekten Senderendstufe sollte vor Beginn von Reparaturarbeiten aus Sicherheitsgründen das Gerät vom Netz getrennt werden und die Netzteilkondensatoren~...}{durch Kurzschluss über ein Strommessgerät sicher entladen werden.}
{über einen sehr niederohmigen Widerstand (<~\qty{1}{\ohm}~/~\qty{0,5}{\W}) sofort vollständig entladen werden.}
{über einen hochohmigen Widerstand mit ausreichender Leistung dauerhaft entladen werden.}
{erst nach Ablauf einer Wartezeit von ca. zwei Minuten berührt werden.}
\end{QQuestion}

}
\only<2>{
\begin{QQuestion}{AK201}{Bei der Fehlersuche in einer defekten Senderendstufe sollte vor Beginn von Reparaturarbeiten aus Sicherheitsgründen das Gerät vom Netz getrennt werden und die Netzteilkondensatoren~...}{durch Kurzschluss über ein Strommessgerät sicher entladen werden.}
{über einen sehr niederohmigen Widerstand (<~\qty{1}{\ohm}~/~\qty{0,5}{\W}) sofort vollständig entladen werden.}
{\textbf{\textcolor{DARCgreen}{über einen hochohmigen Widerstand mit ausreichender Leistung dauerhaft entladen werden.}}}
{erst nach Ablauf einer Wartezeit von ca. zwei Minuten berührt werden.}
\end{QQuestion}

}
\end{frame}%ENDCONTENT


\section{Schutzerdung und Potentialausgleich II}
\label{section:schutzerdung_2}
\begin{frame}%STARTCONTENT

\only<1>{
\begin{QQuestion}{AK202}{Warum ist eine möglichst niederohmige Verbindung aller Potentialausgleichsanschlüsse der Geräte einer Amateurfunkstelle anzustreben?}{Zur Vermeidung von Geräteschäden bei Überspannungen}
{Zur Begrenzung von Kurzschlussströmen bei Gerätefehlern}
{Zum Schutz von Personen}
{Zur Symmetrierung bei paralleldrahtgespeisten Antennen}
\end{QQuestion}

}
\only<2>{
\begin{QQuestion}{AK202}{Warum ist eine möglichst niederohmige Verbindung aller Potentialausgleichsanschlüsse der Geräte einer Amateurfunkstelle anzustreben?}{Zur Vermeidung von Geräteschäden bei Überspannungen}
{Zur Begrenzung von Kurzschlussströmen bei Gerätefehlern}
{\textbf{\textcolor{DARCgreen}{Zum Schutz von Personen}}}
{Zur Symmetrierung bei paralleldrahtgespeisten Antennen}
\end{QQuestion}

}
\end{frame}

\begin{frame}
\only<1>{
\begin{QQuestion}{AK203}{Ihr \qty{400}{\W}-Kurzwellensender ist über eine separate Erdungsleitung mit dem Potentialausgleich Ihres Hauses verbunden. Im Sendebetrieb stellen Sie fest, dass auf bestimmten Bändern das Gehäuse des Senders \glqq heiß\grqq{} ist, d. h. Hochfrequenzspannung merklicher Amplitude auf dem Gerätegehäuse liegt. Was kann die Ursache hierfür sein?}{Die Länge der Erdleitung entspricht annähernd einem Viertel der Wellenlänge der
Sendefrequenz oder einem ungeraden Vielfachen davon.}
{Die verwendete Kupfer-Erdleitung ist nicht versilbert und somit zur guten Ableitung von Hochfrequenz nicht geeignet.}
{Die Länge der Erdleitung entspricht annähernd einer halben Wellenlänge der Sendefrequenz oder Vielfachen davon.}
{Für die verwendete Erdleitung wurde ein massiver Leiter anstatt einer für Hochfrequenz besser geeigneten mehradrigen Litze verwendet.}
\end{QQuestion}

}
\only<2>{
\begin{QQuestion}{AK203}{Ihr \qty{400}{\W}-Kurzwellensender ist über eine separate Erdungsleitung mit dem Potentialausgleich Ihres Hauses verbunden. Im Sendebetrieb stellen Sie fest, dass auf bestimmten Bändern das Gehäuse des Senders \glqq heiß\grqq{} ist, d. h. Hochfrequenzspannung merklicher Amplitude auf dem Gerätegehäuse liegt. Was kann die Ursache hierfür sein?}{\textbf{\textcolor{DARCgreen}{Die Länge der Erdleitung entspricht annähernd einem Viertel der Wellenlänge der
Sendefrequenz oder einem ungeraden Vielfachen davon.}}}
{Die verwendete Kupfer-Erdleitung ist nicht versilbert und somit zur guten Ableitung von Hochfrequenz nicht geeignet.}
{Die Länge der Erdleitung entspricht annähernd einer halben Wellenlänge der Sendefrequenz oder Vielfachen davon.}
{Für die verwendete Erdleitung wurde ein massiver Leiter anstatt einer für Hochfrequenz besser geeigneten mehradrigen Litze verwendet.}
\end{QQuestion}

}
\end{frame}%ENDCONTENT


\section{Berühren von Antennen II}
\label{section:antennen_beruehrung_2}
\begin{frame}%STARTCONTENT

\only<1>{
\begin{QQuestion}{AK204}{Ab welchen Sendeleistungen kann an Sendeantennen Verletzungsgefahr durch hochfrequente Spannungen bestehen?}{Auf Kurzwelle ab 100 Watt, auf VHF/UHF ab 50 Watt}
{Bereits bei geringen Sendeleistungen von wenigen Watt}
{Bei Sendeleistungen höher 100 Watt}
{Bei Sendeleistungen höher 500 Watt}
\end{QQuestion}

}
\only<2>{
\begin{QQuestion}{AK204}{Ab welchen Sendeleistungen kann an Sendeantennen Verletzungsgefahr durch hochfrequente Spannungen bestehen?}{Auf Kurzwelle ab 100 Watt, auf VHF/UHF ab 50 Watt}
{\textbf{\textcolor{DARCgreen}{Bereits bei geringen Sendeleistungen von wenigen Watt}}}
{Bei Sendeleistungen höher 100 Watt}
{Bei Sendeleistungen höher 500 Watt}
\end{QQuestion}

}
\end{frame}%ENDCONTENT

\end{document}