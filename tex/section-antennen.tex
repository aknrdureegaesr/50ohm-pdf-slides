
\section{Antennen}
\label{section:antennen}
\begin{frame}%STARTCONTENT

\begin{figure}
    \DARCimage{0.85\linewidth}{657include}
    \caption{\scriptsize Schematische Darstellung einer Amateurfunkstation mit Funkgerät, Speiseleitung und Antenne}
    \label{n_trx_kabel_und_antenne}
\end{figure}
\begin{columns}
    \begin{column}{0.48\textwidth}
    \begin{itemize}
  \item Gibt elektrische Schwingungen als Funkwellen ab
  \item Funkwellen breiten sich in der Ferne aus
  \end{itemize}

    \end{column}
   \begin{column}{0.48\textwidth}
       \begin{itemize}
  \item Nimmt beim Empfang Funkwellen auf
  \item Leitet sie als elektrische Schwingungen über das Antennenkabel zum Funkgerät
  \end{itemize}

   \end{column}
\end{columns}

\end{frame}

\begin{frame}
\only<1>{
\begin{PQuestion}{NG101}{Welches Bauteil wird durch das Schaltzeichen symbolisiert?}{Diode}
{Erde}
{Antenne}
{Transistor}
{\DARCimage{0.1\linewidth}{543include}}\end{PQuestion}

}
\only<2>{
\begin{PQuestion}{NG101}{Welches Bauteil wird durch das Schaltzeichen symbolisiert?}{Diode}
{Erde}
{\textbf{\textcolor{DARCgreen}{Antenne}}}
{Transistor}
{\DARCimage{0.1\linewidth}{543include}}\end{PQuestion}

}
\end{frame}%ENDCONTENT
