
\section{Innenwiderstand}
\label{section:innenwiderstand}
\begin{frame}%STARTCONTENT

\only<1>{
\begin{QQuestion}{AB201}{Welche Eigenschaften sollten Strom- und Spannungsquellen nach Möglichkeit aufweisen?}{Stromquellen sollten einen möglichst hohen Innenwiderstand und Spannungsquellen einen möglichst niedrigen Innenwiderstand haben.}
{Strom- und Spannungsquellen sollten einen möglichst niedrigen Innenwiderstand haben.}
{Strom- und Spannungsquellen sollten einen möglichst hohen Innenwiderstand haben.}
{Stromquellen sollten einen möglichst niedrigen Innenwiderstand und Spannungsquellen einen möglichst hohen Innenwiderstand haben.}
\end{QQuestion}

}
\only<2>{
\begin{QQuestion}{AB201}{Welche Eigenschaften sollten Strom- und Spannungsquellen nach Möglichkeit aufweisen?}{\textbf{\textcolor{DARCgreen}{Stromquellen sollten einen möglichst hohen Innenwiderstand und Spannungsquellen einen möglichst niedrigen Innenwiderstand haben.}}}
{Strom- und Spannungsquellen sollten einen möglichst niedrigen Innenwiderstand haben.}
{Strom- und Spannungsquellen sollten einen möglichst hohen Innenwiderstand haben.}
{Stromquellen sollten einen möglichst niedrigen Innenwiderstand und Spannungsquellen einen möglichst hohen Innenwiderstand haben.}
\end{QQuestion}

}
\end{frame}

\begin{frame}
\only<1>{
\begin{QQuestion}{AG401}{Welche Lastimpedanz ist für eine Leistungsanpassung erforderlich, wenn die Signalquelle eine Ausgangsimpedanz von \qty{50}{\ohm} hat? }{\qty{50}{\ohm}}
{1/\qty{50}{\ohm}}
{\qty{100}{\ohm}}
{\qty{200}{\ohm}}
\end{QQuestion}

}
\only<2>{
\begin{QQuestion}{AG401}{Welche Lastimpedanz ist für eine Leistungsanpassung erforderlich, wenn die Signalquelle eine Ausgangsimpedanz von \qty{50}{\ohm} hat? }{\textbf{\textcolor{DARCgreen}{\qty{50}{\ohm}}}}
{1/\qty{50}{\ohm}}
{\qty{100}{\ohm}}
{\qty{200}{\ohm}}
\end{QQuestion}

}
\end{frame}

\begin{frame}
\only<1>{
\begin{QQuestion}{AB202}{In welchem Zusammenhang müssen der Innenwiderstand $R_\textrm{i}$ einer Strom- oder Spannungsquelle und ein direkt daran angeschlossener Lastwiderstand $R_\textrm{L}$ stehen, damit Leistungsanpassung vorliegt?}{$R_\textrm{L} \ll R_\textrm{i}$}
{$R_\textrm{L} \gg R_\textrm{i}$}
{$R_\textrm{L} = R_\textrm{i}$}
{$R_\textrm{L} = \dfrac{1}{R_\textrm{i}}$}
\end{QQuestion}

}
\only<2>{
\begin{QQuestion}{AB202}{In welchem Zusammenhang müssen der Innenwiderstand $R_\textrm{i}$ einer Strom- oder Spannungsquelle und ein direkt daran angeschlossener Lastwiderstand $R_\textrm{L}$ stehen, damit Leistungsanpassung vorliegt?}{$R_\textrm{L} \ll R_\textrm{i}$}
{$R_\textrm{L} \gg R_\textrm{i}$}
{\textbf{\textcolor{DARCgreen}{$R_\textrm{L} = R_\textrm{i}$}}}
{$R_\textrm{L} = \dfrac{1}{R_\textrm{i}}$}
\end{QQuestion}

}
\end{frame}

\begin{frame}
\only<1>{
\begin{QQuestion}{AB203}{In welchem Zusammenhang müssen der Innenwiderstand $R_{\symup{i}}$ einer Spannungsquelle und ein direkt daran angeschlossener Lastwiderstand $R_{\symup{L}}$ stehen, damit Spannungsanpassung vorliegt?}{$R_{\symup{L}} \gg R_{\symup{i}}$}
{$R_{\symup{L}} \ll R_{\symup{i}}$}
{$R_{\symup{L}} = R_{\symup{i}}$}
{$R_{\symup{L}} = \frac{1}{R_{\symup{i}}}$}
\end{QQuestion}

}
\only<2>{
\begin{QQuestion}{AB203}{In welchem Zusammenhang müssen der Innenwiderstand $R_{\symup{i}}$ einer Spannungsquelle und ein direkt daran angeschlossener Lastwiderstand $R_{\symup{L}}$ stehen, damit Spannungsanpassung vorliegt?}{\textbf{\textcolor{DARCgreen}{$R_{\symup{L}} \gg R_{\symup{i}}$}}}
{$R_{\symup{L}} \ll R_{\symup{i}}$}
{$R_{\symup{L}} = R_{\symup{i}}$}
{$R_{\symup{L}} = \frac{1}{R_{\symup{i}}}$}
\end{QQuestion}

}
\end{frame}

\begin{frame}
\only<1>{
\begin{QQuestion}{AB204}{In welchem Zusammenhang müssen der Innenwiderstand $R_\textrm{i}$ einer Stromquelle und ein direkt daran angeschlossener Lastwiderstand $R_\textrm{L}$ stehen, damit Stromanpassung vorliegt?}{$R_{\textrm{L}} \ll R_{\textrm{i}}$}
{$R_{\textrm{L}} \gg R_{\textrm{i}}$}
{$R_{\textrm{L}} = R_{\textrm{i}}$}
{$R_{\textrm{L}} = \dfrac{1}{R_{\textrm{i}}}$}
\end{QQuestion}

}
\only<2>{
\begin{QQuestion}{AB204}{In welchem Zusammenhang müssen der Innenwiderstand $R_\textrm{i}$ einer Stromquelle und ein direkt daran angeschlossener Lastwiderstand $R_\textrm{L}$ stehen, damit Stromanpassung vorliegt?}{\textbf{\textcolor{DARCgreen}{$R_{\textrm{L}} \ll R_{\textrm{i}}$}}}
{$R_{\textrm{L}} \gg R_{\textrm{i}}$}
{$R_{\textrm{L}} = R_{\textrm{i}}$}
{$R_{\textrm{L}} = \dfrac{1}{R_{\textrm{i}}}$}
\end{QQuestion}

}
\end{frame}

\begin{frame}
\only<1>{
\begin{QQuestion}{AB207}{Die Leerlaufspannung einer Gleichspannungsquelle beträgt \qty{13,5}{\V}. Wenn die Spannungsquelle einen Strom von \qty{2}{\A} abgibt, sinkt die Klemmenspannung auf \qty{13}{\V}. Wie groß ist der Innenwiderstand der Spannungsquelle?}{\qty{4}{\ohm}}
{\qty{6,75}{\ohm}}
{\qty{0,25}{\ohm}}
{\qty{1}{\ohm}}
\end{QQuestion}

}
\only<2>{
\begin{QQuestion}{AB207}{Die Leerlaufspannung einer Gleichspannungsquelle beträgt \qty{13,5}{\V}. Wenn die Spannungsquelle einen Strom von \qty{2}{\A} abgibt, sinkt die Klemmenspannung auf \qty{13}{\V}. Wie groß ist der Innenwiderstand der Spannungsquelle?}{\qty{4}{\ohm}}
{\qty{6,75}{\ohm}}
{\textbf{\textcolor{DARCgreen}{\qty{0,25}{\ohm}}}}
{\qty{1}{\ohm}}
\end{QQuestion}

}
\end{frame}

\begin{frame}
\frametitle{Lösungsweg}
\begin{itemize}
  \item gegeben: $U_0 = 13,5V$
  \item gegeben: $U_{Kl} = 13V$
  \item gegeben: $I = 2A$
  \item gesucht: $R_i$
  \end{itemize}
    \pause
    $R_i = \frac{U_i}{I} = \frac{U_0-U_{Kl}}{I} = \frac{13,5V-13V}{2A} = 0,25Ω$



\end{frame}

\begin{frame}
\only<1>{
\begin{QQuestion}{AB208}{Die Leerlaufspannung einer Gleichspannungsquelle beträgt \qty{13,8}{\V}. Wenn die Spannungsquelle einen Strom von \qty{20}{\A} abgibt, bleibt die Klemmenspannung auf \qty{13,6}{\V}. Wie groß ist der Innenwiderstand der Spannungsquelle?}{\qty{20}{\m}$\Omega$}
{\qty{10}{\m}$\Omega$}
{\qty{0,2}{\ohm}}
{\qty{0,1}{\ohm}}
\end{QQuestion}

}
\only<2>{
\begin{QQuestion}{AB208}{Die Leerlaufspannung einer Gleichspannungsquelle beträgt \qty{13,8}{\V}. Wenn die Spannungsquelle einen Strom von \qty{20}{\A} abgibt, bleibt die Klemmenspannung auf \qty{13,6}{\V}. Wie groß ist der Innenwiderstand der Spannungsquelle?}{\qty{20}{\m}$\Omega$}
{\textbf{\textcolor{DARCgreen}{\qty{10}{\m}$\Omega$}}}
{\qty{0,2}{\ohm}}
{\qty{0,1}{\ohm}}
\end{QQuestion}

}
\end{frame}

\begin{frame}
\frametitle{Lösungsweg}
\begin{itemize}
  \item gegeben: $U_0 = 13,8V$
  \item gegeben: $U_{Kl} = 13,6V$
  \item gegeben: $I = 20A$
  \item gesucht: $R_i$
  \end{itemize}
    \pause
    $R_i = \frac{U_i}{I} = \frac{U_0-U_{Kl}}{I} = \frac{13,8V-13,6V}{20A} = 10mΩ$



\end{frame}

\begin{frame}
\only<1>{
\begin{QQuestion}{AB206}{Die Leerlaufspannung einer Gleichspannungsquelle beträgt \qty{13,5}{\V}. Wenn die Spannungsquelle einen Strom von \qty{0,9}{\A} abgibt, sinkt die Klemmenspannung auf \qty{12,4}{\V}. Wie groß ist der Innenwiderstand der Spannungsquelle?}{\qty{0,99}{\ohm}}
{\qty{0,82}{\ohm}}
{\qty{1,22}{\ohm}}
{\qty{15,0}{\ohm}}
\end{QQuestion}

}
\only<2>{
\begin{QQuestion}{AB206}{Die Leerlaufspannung einer Gleichspannungsquelle beträgt \qty{13,5}{\V}. Wenn die Spannungsquelle einen Strom von \qty{0,9}{\A} abgibt, sinkt die Klemmenspannung auf \qty{12,4}{\V}. Wie groß ist der Innenwiderstand der Spannungsquelle?}{\qty{0,99}{\ohm}}
{\qty{0,82}{\ohm}}
{\textbf{\textcolor{DARCgreen}{\qty{1,22}{\ohm}}}}
{\qty{15,0}{\ohm}}
\end{QQuestion}

}
\end{frame}

\begin{frame}
\frametitle{Lösungsweg}
\begin{itemize}
  \item gegeben: $U_0 = 13,5V$
  \item gegeben: $U_{Kl} = 12,4V$
  \item gegeben: $I = 0,9A$
  \item gesucht: $R_i$
  \end{itemize}
    \pause
    $R_i = \frac{U_i}{I} = \frac{U_0-U_{Kl}}{I} = \frac{13,5V-12,4V}{0,9A} = 1,22Ω$



\end{frame}

\begin{frame}
\only<1>{
\begin{QQuestion}{AB205}{Die Leerlaufspannung einer Spannungsquelle beträgt \qty{5,0}{\V}. Schließt man einen Belastungswiderstand mit \qty{1,2}{\ohm} an, so geht die Klemmenspannung der Spannungsquelle auf \qty{4,8}{\V} zurück. Wie hoch ist der Innenwiderstand der Spannungsquelle?}{\qty{8,2}{\ohm}}
{\qty{0,05}{\ohm}}
{\qty{0,17}{\ohm}}
{\qty{0,25}{\ohm}}
\end{QQuestion}

}
\only<2>{
\begin{QQuestion}{AB205}{Die Leerlaufspannung einer Spannungsquelle beträgt \qty{5,0}{\V}. Schließt man einen Belastungswiderstand mit \qty{1,2}{\ohm} an, so geht die Klemmenspannung der Spannungsquelle auf \qty{4,8}{\V} zurück. Wie hoch ist der Innenwiderstand der Spannungsquelle?}{\qty{8,2}{\ohm}}
{\textbf{\textcolor{DARCgreen}{\qty{0,05}{\ohm}}}}
{\qty{0,17}{\ohm}}
{\qty{0,25}{\ohm}}
\end{QQuestion}

}
\end{frame}

\begin{frame}
\frametitle{Lösungsweg}
\begin{itemize}
  \item gegeben: $U_0 = 5,0V$
  \item gegeben: $U_{Kl} = 4,8V$
  \item gegeben: $R_L = 1,2Ω$
  \item gesucht: $R_i$
  \end{itemize}
    \pause
    $I = \frac{U_{Kl}}{R_L} = \frac{4,8V}{1,2Ω} = 4A$
    \pause
    $R_i = \frac{U_i}{I} = \frac{U_0-U_{Kl}}{I} = \frac{5,0V-4,8V}{4A} = 0,05Ω$



\end{frame}%ENDCONTENT
