% Author: Dr. Matthias Jung, DL9MJ
% Year: 2020
% TF208
\documentclass[convert=false]{standalone}
\input{../common/settings.tex}

\begin{document}

\begin{circuitikz}
    %\tikzstyle{help lines}=[blue!50];
    %\draw[style=help lines] (0,0) grid (10,4);
    \draw (0,3) 
        to [short, o-] ++(1.5,0)
        to [bandpass, >, label=Filter] ++(2,0)
        to [amp, box, >, label=HF] ++(2,0);
    \draw (6.5,3) node[mixer, box] (mix) {} ++(1,0)
        to [bandpass, >, label=Filter] ++(2,0)
        to [short, -o] ++(0.5,0);
    \draw (mix.south) node[inputarrow, rotate=90] {} to [short] ++(0,-1.5);
    \draw (mix.west) node[inputarrow] {} to [short] ++(-0.5,0);
    \draw (mix.east) to [short] ++(0.5,0);
    \draw (9.5,2.15) node[] {28-30\,MHz};

    \draw (2,1) to [twoport, label={38.667 MHz~~}] ++(1,0)
        to [twoportsplit, >, label={~~Vervielfacher}, t1=1, t2=3] ++(3,0)
        to [short] ++(0.5,0);
    \draw(2.3,1.2) node[] {G};

    \def\x{0.08}
    \draw[] (2.55,1.3) sin ++(\x, \x)
                       cos ++(\x,-\x)
                       sin ++(\x,-\x)
                       cos ++(\x, \x);

    \draw[] (2.55,1.15)sin ++(\x, \x)
                       cos ++(\x,-\x)
                       sin ++(\x,-\x)
                       cos ++(\x, \x);

    \draw[] (2.55,1.0) sin ++(\x, \x)
                       cos ++(\x,-\x)
                       sin ++(\x,-\x)
                       cos ++(\x, \x);

    \draw (2.2,1-0.40) -- ++(0.6,0);
    \draw (2.2,1-0.35) rectangle ++ (0.6,0.1);
    \draw (2.2,1-0.20) -- ++(0.6,0);
\end{circuitikz}
\end{document}