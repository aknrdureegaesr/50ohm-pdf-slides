% Author: Dr. Matthias Jung, DL9MJ
% Year: 2022
\documentclass[convert=false]{standalone}
\input{../common/settings.tex}

\begin{document}

\begin{circuitikz}
    %\tikzstyle{help lines}=[blue!50];
    %\draw[style=help lines] (0,-2) grid (18,8);
    \draw (-0.2,0) circle (0.2);
    \draw[thick]  (-0.4,0.2) -- (-0.4,-0.2);
    \draw(0,0) 
        to [adc, >] ++ (3,0)
        to [twoportsplit, >,     l ={\footnotesize Quellencodierer}] ++(1,0)
        to [twoportsplit, >,     l_={\footnotesize Kanalcodierer}] ++(3,0)
        to [twoportsplit, l ={\footnotesize Mapper}] ++(1,0) coordinate(c1);

    \draw(c1) ++(0, 0.25) -| ++(0.5, 0.5)
        to [short] ++(0.5,0)
        to [dac, >] ++(1,0)
        -| ++(0.5, -0.5)
        to [short] ++(0.5,0) node[inputarrow] {} ++(0,-0.25) coordinate(c2);
    \draw(c1) ++(0,-0.25) -| ++(0.5,-0.5)
        to [short] ++(0.5,0)
        to [dac, >] ++(1,0)
        -| ++(0.5, +0.5)
        to [short] ++(0.5,0) node[inputarrow] {};
    \draw(c2) ++(0.5,0) node[mixer, box, label={IQ}] (mixx) {} ++(0.5,0);

    %G
    \draw(mixx.south) node[inputarrow, rotate=90] {} to [short] ++(0,-1.0);
    \draw(3+8,-2) to [twoport] ++(1,0);
    \draw(3.3+8,1.2-3) node[] {G};
    \def\x{0.08}
    \draw[] (7.55+4,1.3-3) sin ++(\x, \x)
                       cos ++(\x,-\x)
                       sin ++(\x,-\x)
                       cos ++(\x, \x);
    \draw[] (7.55+4,1.15-3)sin ++(\x, \x)
                       cos ++(\x,-\x)
                       sin ++(\x,-\x)
                       cos ++(\x, \x);
    \draw[] (7.55+4,1.0-3) sin ++(\x, \x)
                       cos ++(\x,-\x)
                       sin ++(\x,-\x)
                       cos ++(\x, \x);
    \draw(7.2+4,0.60-3) -- ++(0.6,0);
    \draw(7.2+4,0.65-3) rectangle ++ (0.6,0.1);
    \draw(7.2+4,0.80-3) -- ++(0.6,0);

    \draw (mixx.east) 
        to [amp, box, >] ++ (3,0)
        to [bandpass, >] ++(1,0)
        to [short] ++(1,0)
        node[antenna]{};

    %% Decoration:
    \draw (3.35,+0.25) node(){\tiny\texttt{0101}};
    \draw (3.65,-0.25) node(){\texttt{01}};
    %
    \draw (5.35,+0.25) node(){\texttt{01}};
    \draw (5.65,-0.25) node(){\tiny\texttt{01+01}};
    %
    \draw[thick] (7+0.3+0.4,0.4-0.4) -- ++(0,-0.4);
    \draw[thick] (7+0.1+0.4,0.2-0.4) -- ++(0.4,0);
    \filldraw    (7+0.2+0.4,0.3-0.4) circle (1pt);
    \filldraw    (7+0.4+0.4,0.3-0.4) circle (1pt);
    \filldraw    (7+0.2+0.4,0.1-0.4) circle (1pt);
    \filldraw    (7+0.4+0.4,0.1-0.4) circle (1pt);
    \draw        (7.35,+0.25) node(){\texttt{01}};

    % Dashed Block:
    \draw[thick, gray, dashed] (2.25,-1.25) rectangle ++(6.0,2.5);
    \draw[gray] (5.25,1.5) node(){FPGA oder Software};
\end{circuitikz}
\end{document}
