
\section{Recht zum Selbstbau}
\label{section:recht_zum_selbstbau}
\begin{frame}%STARTCONTENT
\begin{itemize}
  \item Sender und Sendeanlagen benötigen normalerweise eine behördliche Zulassung
  \item Funkamateure sind davon ausgenommen
  \item Sie sind berechtigt, \emph{im Handel erhältliche, selbstgefertigte oder auf Amateurfunkfrequenzen umgebaute Sendeanlagen} zu betreiben
  \end{itemize}

\end{frame}

\begin{frame}
\only<1>{
\begin{QQuestion}{VC109}{Welches Recht haben Funkamateure in Bezug auf den Betrieb von Sendeanlagen? Ein Funkamateur~...}{muss die einschlägigen Bestimmungen der BNetzA zur elektrischen Sicherheit nicht beachten.}
{benötigt in keinem Fall eine Standortbescheinigung der BNetzA für seine Amateurfunkstelle.}
{ist berechtigt, im Handel erhältliche, selbst gefertigte oder auf Amateurfunkfrequenzen umgebaute Sendeanlagen zu betreiben.}
{darf mit seiner Amateurfunkstelle jederzeit Nachrichten für und an Dritte übermitteln, die nicht den Amateurfunkdienst betreffen.}
\end{QQuestion}

}
\only<2>{
\begin{QQuestion}{VC109}{Welches Recht haben Funkamateure in Bezug auf den Betrieb von Sendeanlagen? Ein Funkamateur~...}{muss die einschlägigen Bestimmungen der BNetzA zur elektrischen Sicherheit nicht beachten.}
{benötigt in keinem Fall eine Standortbescheinigung der BNetzA für seine Amateurfunkstelle.}
{\textbf{\textcolor{DARCgreen}{ist berechtigt, im Handel erhältliche, selbst gefertigte oder auf Amateurfunkfrequenzen umgebaute Sendeanlagen zu betreiben.}}}
{darf mit seiner Amateurfunkstelle jederzeit Nachrichten für und an Dritte übermitteln, die nicht den Amateurfunkdienst betreffen.}
\end{QQuestion}

}
\end{frame}

\begin{frame}
\frametitle{Bauteile}
\begin{itemize}
  \item Zum Selbstbau werden verschiede elektronische Bauteile benötigt
  \item Diese weisen unterschiedliche Eigenschaften auf
  \item In Klasse~N gibt es nur wenige, einfache Schaltungen $\rightarrow$ mehr in Klasse~E und A
  \item Kenntnisse der Symbole und Bezeichnungen reichen
  \end{itemize}

\end{frame}

\begin{frame}
\frametitle{Anforderungen an Funkgeräte}
\begin{itemize}
  \item Alle \emph{im Handel erhältlichen, seriengefertigten Funkanlagen} müssen die \emph{grundlegenden Anforderungen und Bestimmungen des Funkanlagengesetzes (FuAG)} einhalten
  \item EU-Konformitätserklärung (\emph{CE-Kennzeichnung}) vor in Verkehr bringen erstellen
  \item Nur dann dürfen vom Markt bereitgestellte Anlagen in Betrieb genommen werden
  \end{itemize}

\end{frame}

\begin{frame}
\only<1>{
\begin{QQuestion}{VE401}{Welches Gesetz regelt unter anderem das Inverkehrbringen, den freien Warenverkehr und die Inbetriebnahme von auf dem Markt bereitgestellten Amateurfunkanlagen?}{Das Gesetz über die elektromagnetische Verträglichkeit von Betriebsmitteln (EMVG)}
{Die Amateurfunkverordnung (AfuV)}
{Das Funkanlagengesetz (FuAG)}
{Für solche Amateurfunkgeräte gibt es keine Regelung.}
\end{QQuestion}

}
\only<2>{
\begin{QQuestion}{VE401}{Welches Gesetz regelt unter anderem das Inverkehrbringen, den freien Warenverkehr und die Inbetriebnahme von auf dem Markt bereitgestellten Amateurfunkanlagen?}{Das Gesetz über die elektromagnetische Verträglichkeit von Betriebsmitteln (EMVG)}
{Die Amateurfunkverordnung (AfuV)}
{\textbf{\textcolor{DARCgreen}{Das Funkanlagengesetz (FuAG)}}}
{Für solche Amateurfunkgeräte gibt es keine Regelung.}
\end{QQuestion}

}
\end{frame}

\begin{frame}
\only<1>{
\begin{QQuestion}{VE402}{Welche Geräte fallen in den Anwendungsbereich des Funkanlagengesetzes (FuAG)?}{Selbstgebaute Amateurfunkanlagen}
{Auf dem Markt bereitgestellte Amateurfunkanlagen}
{Kommerziell hergestellte Funkanlagen, die zu Amateurfunkzwecken umgebaut wurden}
{Bausätze für Amateurfunkanlagen}
\end{QQuestion}

}
\only<2>{
\begin{QQuestion}{VE402}{Welche Geräte fallen in den Anwendungsbereich des Funkanlagengesetzes (FuAG)?}{Selbstgebaute Amateurfunkanlagen}
{\textbf{\textcolor{DARCgreen}{Auf dem Markt bereitgestellte Amateurfunkanlagen}}}
{Kommerziell hergestellte Funkanlagen, die zu Amateurfunkzwecken umgebaut wurden}
{Bausätze für Amateurfunkanlagen}
\end{QQuestion}

}
\end{frame}

\begin{frame}
\only<1>{
\begin{QQuestion}{VE403}{Welche grundlegenden Anforderungen werden nach dem Funkanlagengesetz (FuAG) an Amateurfunkgeräte gestellt?}{Die Funkgeräte müssen eine nationale Zulassungskennzeichnung nach Vorgabe der BNetzA tragen.}
{Seriengefertigte Geräte müssen die grundlegenden Anforderungen nach dem Funkanlagengesetz (FuAG) einhalten und eine CE-Kennzeichnung tragen.}
{Selbstgebaute Funkgeräte müssen die grundlegenden Anforderungen nach dem Funkanlagengesetz (FuAG) einhalten und eine CE-Kennzeichnung tragen.}
{Seriengefertigte Amateurfunkgeräte unterliegen nicht dem Funkanlagengesetz (FuAG).}
\end{QQuestion}

}
\only<2>{
\begin{QQuestion}{VE403}{Welche grundlegenden Anforderungen werden nach dem Funkanlagengesetz (FuAG) an Amateurfunkgeräte gestellt?}{Die Funkgeräte müssen eine nationale Zulassungskennzeichnung nach Vorgabe der BNetzA tragen.}
{\textbf{\textcolor{DARCgreen}{Seriengefertigte Geräte müssen die grundlegenden Anforderungen nach dem Funkanlagengesetz (FuAG) einhalten und eine CE-Kennzeichnung tragen.}}}
{Selbstgebaute Funkgeräte müssen die grundlegenden Anforderungen nach dem Funkanlagengesetz (FuAG) einhalten und eine CE-Kennzeichnung tragen.}
{Seriengefertigte Amateurfunkgeräte unterliegen nicht dem Funkanlagengesetz (FuAG).}
\end{QQuestion}

}
\end{frame}

\begin{frame}
\only<1>{
\begin{QQuestion}{VE404}{Welche Vorschriften müssen im Handel erhältliche Empfangsfunkanlagen einhalten, die dem Amateurfunk zugewiesene Frequenzen empfangen können?}{Grundlegende Anforderungen an Amateurfunkempfänger sind in der Amateurfunkverordnung geregelt.}
{Amateurfunkempfänger brauchen grundsätzlich keinerlei Bestimmungen einzuhalten.}
{Es sind die Bestimmungen des Funkanlagengesetzes (FuAG) einzuhalten.}
{Amateurfunkempfänger dürfen ausschließlich von Funkamateuren betrieben werden; darüber hinaus gibt es keine weiteren Vorschriften.}
\end{QQuestion}

}
\only<2>{
\begin{QQuestion}{VE404}{Welche Vorschriften müssen im Handel erhältliche Empfangsfunkanlagen einhalten, die dem Amateurfunk zugewiesene Frequenzen empfangen können?}{Grundlegende Anforderungen an Amateurfunkempfänger sind in der Amateurfunkverordnung geregelt.}
{Amateurfunkempfänger brauchen grundsätzlich keinerlei Bestimmungen einzuhalten.}
{\textbf{\textcolor{DARCgreen}{Es sind die Bestimmungen des Funkanlagengesetzes (FuAG) einzuhalten.}}}
{Amateurfunkempfänger dürfen ausschließlich von Funkamateuren betrieben werden; darüber hinaus gibt es keine weiteren Vorschriften.}
\end{QQuestion}

}
\end{frame}

\begin{frame}
\frametitle{Selbstbau}
\begin{itemize}
  \item Ausnahme: von Funkamateuren \emph{selbst gebaute und umgebaute} Funkanlagen
  \item Müssen nicht die Anforderungen des Funkanlagengesetzes erfüllen
  \item Müssen keine CE-Kennzeichnung tragen
  \end{itemize}
\end{frame}

\begin{frame}
\only<1>{
\begin{QQuestion}{VE405}{Wird für von Funkamateuren zusammengebaute Funkanlagen der Nachweis auf Einhaltung der technischen Vorschriften nach den Bestimmungen des Funkanlagengesetzes (FuAG) verlangt?}{Solche Amateurfunkanlagen sind nach den Funkanlagengesetzes (FuAG) nicht zulässig.}
{Solche Amateurfunkanlagen müssen den Anforderungen des Funkanlagengesetzes (FuAG) genügen.}
{Solche Amateurfunkanlagen müssen nicht den Anforderungen des Funkanlagengesetzes (FuAG) genügen.}
{Solche Amateurfunkanlagen müssen der BNetzA zur Prüfung vorgestellt werden.}
\end{QQuestion}

}
\only<2>{
\begin{QQuestion}{VE405}{Wird für von Funkamateuren zusammengebaute Funkanlagen der Nachweis auf Einhaltung der technischen Vorschriften nach den Bestimmungen des Funkanlagengesetzes (FuAG) verlangt?}{Solche Amateurfunkanlagen sind nach den Funkanlagengesetzes (FuAG) nicht zulässig.}
{Solche Amateurfunkanlagen müssen den Anforderungen des Funkanlagengesetzes (FuAG) genügen.}
{\textbf{\textcolor{DARCgreen}{Solche Amateurfunkanlagen müssen nicht den Anforderungen des Funkanlagengesetzes (FuAG) genügen.}}}
{Solche Amateurfunkanlagen müssen der BNetzA zur Prüfung vorgestellt werden.}
\end{QQuestion}

}
\end{frame}%ENDCONTENT
