
\section{Fehlerkorrektur}
\label{section:fehlerkorrektur}
\begin{frame}%STARTCONTENT

\only<1>{
\begin{QQuestion}{AE413}{Sie verwenden ein Datenübertragungsverfahren \underline{ohne} Vorwärtsfehlerkorrektur. Wodurch können Datenpakete trotz Prüfsummenfehlern korrigiert werden?}{I/Q-Verfahren}
{Wiederholte Prüfung}
{Duplizieren der Prüfsumme}
{Erneute Übertragung}
\end{QQuestion}

}
\only<2>{
\begin{QQuestion}{AE413}{Sie verwenden ein Datenübertragungsverfahren \underline{ohne} Vorwärtsfehlerkorrektur. Wodurch können Datenpakete trotz Prüfsummenfehlern korrigiert werden?}{I/Q-Verfahren}
{Wiederholte Prüfung}
{Duplizieren der Prüfsumme}
{\textbf{\textcolor{DARCgreen}{Erneute Übertragung}}}
\end{QQuestion}

}
\end{frame}

\begin{frame}
\only<1>{
\begin{QQuestion}{AE414}{Was ist die Voraussetzung für Vorwärtsfehlerkorrektur (FEC)?}{Automatische Anpassung der Sendeleistung}
{Kompression vor der Übertragung}
{Erneute Übertragung fehlerhafter Daten}
{Übertragung redundanter Informationen}
\end{QQuestion}

}
\only<2>{
\begin{QQuestion}{AE414}{Was ist die Voraussetzung für Vorwärtsfehlerkorrektur (FEC)?}{Automatische Anpassung der Sendeleistung}
{Kompression vor der Übertragung}
{Erneute Übertragung fehlerhafter Daten}
{\textbf{\textcolor{DARCgreen}{Übertragung redundanter Informationen}}}
\end{QQuestion}

}
\end{frame}%ENDCONTENT
