
\section{Orthogonales Frequenzmultiplexverfahren (OFDM)}
\label{section:ofdm}
\begin{frame}%STARTCONTENT
\begin{itemize}
  \item Es ist auch möglich, einen Datenstrom auf mehrere Träger zu verteilen, die auf unterschiedlichen, jedoch nahegelegenen Frequenzen liegen.
  \item Bei der orthogonalen Frequenzmodulation (Orthogonal Frequency-Division Multiplexing, OFDM) werden die einzelnen Träger in einem Abstand platziert, wo ein gegenseitiges Stören untereinander (ein sogenanntes „Übersprechen“) vermieden wird.
  \end{itemize}

\begin{figure}
    \DARCimage{0.85\linewidth}{704include}
    \caption{\scriptsize Frequenzspektrum eines einfachen OFDM-Signals}
    \label{ofdm}
\end{figure}

\end{frame}

\begin{frame}\begin{itemize}
  \item Ein Vorteil dieses Vorgehens liegt darin, dass schmalbandige Störungen nur einen oder wenige Träger stören.
  \item Im Zusammenspiel mit Fehlerkorrekturverfahren mit redundanter Datenübertragung, die wir später kennenlernen werden, ist es so möglich, trotz schmalbandiger Störungen eine fehlerfreie Übertragung zu erreichen.
  \end{itemize}
\end{frame}

\begin{frame}
\only<1>{
\begin{QQuestion}{AE421}{Orthogonale Frequenzmultiplexverfahren (OFDM) mit redundanter Übertragung sind besonders unempfindlich gegen~...}{schmalbandige Störungen, da es einen Träger mit hoher Bandbreite verwendet.}
{schmalbandige Störungen, da das Gesamtsignal aus mehreren Einzelträgern besteht.}
{breitbandige Störungen, da das Gesamtsignal aus mehreren Einzelträgern besteht.}
{breitbandige Störungen, da es einen Träger mit hoher Bandbreite verwendet.}
\end{QQuestion}

}
\only<2>{
\begin{QQuestion}{AE421}{Orthogonale Frequenzmultiplexverfahren (OFDM) mit redundanter Übertragung sind besonders unempfindlich gegen~...}{schmalbandige Störungen, da es einen Träger mit hoher Bandbreite verwendet.}
{\textbf{\textcolor{DARCgreen}{schmalbandige Störungen, da das Gesamtsignal aus mehreren Einzelträgern besteht.}}}
{breitbandige Störungen, da das Gesamtsignal aus mehreren Einzelträgern besteht.}
{breitbandige Störungen, da es einen Träger mit hoher Bandbreite verwendet.}
\end{QQuestion}

}
\end{frame}

\begin{frame}\begin{itemize}
  \item Ein weiterer Vorteil ergibt sich aus der geringeren Symbolrate jedes einzelnen Trägers.
  \item Durch die geringere Symbolrate ist die Dauer eines jeden Symbols länger.
  \item Im Falle zeitlicher Verschiebungen aufgrund von Mehrwegeausbreitung ist der Anteil der Überlagerung zwischen den Signalen entsprechend geringer.
  \end{itemize}
\end{frame}

\begin{frame}
\only<1>{
\begin{QQuestion}{AE422}{Bei welcher Art von Kanalstörung sind Orthogonale Frequenzmultiplexverfahren (OFDM) mit redundanter Übertragung besonders vorteilhaft?}{Überreichweiten anderer OFDM-Sender}
{Impulse durch Gewitter}
{Breitbandiges Rauschen}
{Mehrwegeausbreitung}
\end{QQuestion}

}
\only<2>{
\begin{QQuestion}{AE422}{Bei welcher Art von Kanalstörung sind Orthogonale Frequenzmultiplexverfahren (OFDM) mit redundanter Übertragung besonders vorteilhaft?}{Überreichweiten anderer OFDM-Sender}
{Impulse durch Gewitter}
{Breitbandiges Rauschen}
{\textbf{\textcolor{DARCgreen}{Mehrwegeausbreitung}}}
\end{QQuestion}

}
\end{frame}%ENDCONTENT
