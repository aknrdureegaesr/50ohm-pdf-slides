
\section{Endgespeiste Antennen (End-Fed)}
\label{section:endgespeiste_antennen}
\begin{frame}%STARTCONTENT

\begin{figure}
    \DARCimage{0.85\linewidth}{615include}
    \caption{\scriptsize Schaltbild einer endgespeisten Antenne}
    \label{n_antennenformen_schaltbild_endfed}
\end{figure}

\begin{itemize}
  \item Statt in der Mitte das Antennenkabel an einem Ende des Dipols anschließen
  \item Häufige Bauform: Endgespeister Halbwellendipol
  \item Ist der Draht einer endgespeisten Antenne länger als die Wellenlänge: Langdraht-Antenne
  \end{itemize}
\end{frame}

\begin{frame}
\only<1>{
\begin{PQuestion}{NG107}{Wie wird die dargestellte Antenne bezeichnet?}{Endgespeiste Antenne}
{Groundplane-Antenne}
{Dipol-Antenne}
{Yagi-Uda-Antenne}
{\DARCimage{1.0\linewidth}{615include}}\end{PQuestion}

}
\only<2>{
\begin{PQuestion}{NG107}{Wie wird die dargestellte Antenne bezeichnet?}{\textbf{\textcolor{DARCgreen}{Endgespeiste Antenne}}}
{Groundplane-Antenne}
{Dipol-Antenne}
{Yagi-Uda-Antenne}
{\DARCimage{1.0\linewidth}{615include}}\end{PQuestion}

}
\end{frame}

\begin{frame}
\only<1>{
\begin{QQuestion}{NG109}{Welche Antennenform wird von Funkamateuren in der Regel nur im Kurzwellenbereich und \underline{nicht} im VHF/UHF-Bereich verwendet?}{Langdraht-Antenne}
{Yagi-Uda-Antenne}
{Quad-Antenne}
{Groundplane-Antenne}
\end{QQuestion}

}
\only<2>{
\begin{QQuestion}{NG109}{Welche Antennenform wird von Funkamateuren in der Regel nur im Kurzwellenbereich und \underline{nicht} im VHF/UHF-Bereich verwendet?}{\textbf{\textcolor{DARCgreen}{Langdraht-Antenne}}}
{Yagi-Uda-Antenne}
{Quad-Antenne}
{Groundplane-Antenne}
\end{QQuestion}

}
\end{frame}%ENDCONTENT
