
\section{Dummy-Load II}
\label{section:dummy_load_2}
\begin{frame}%STARTCONTENT

\only<1>{
\begin{PQuestion}{AI601}{Die Darstellung zeigt eine aus \qty{150}{\ohm} / \qty{1}{\W}-Widerständen aufgebaute künstliche Antenne (Dummy Load). Mit dieser Kombination aus Reihen- und Parallelschaltung werden ca. \qty{50}{\ohm} erreicht. Wie viele Widerstände werden für diesen Aufbau benötigt und welche Dauerleistung verträgt diese künstliche Antenne?}{16~Widerstände, \qty{16}{\W}}
{48~Widerstände, \qty{12}{\W}}
{12~Widerstände, \qty{48}{\W}}
{48~Widerstände, \qty{48}{\W}}
{\DARCimage{1.0\linewidth}{47include}}\end{PQuestion}

}
\only<2>{
\begin{PQuestion}{AI601}{Die Darstellung zeigt eine aus \qty{150}{\ohm} / \qty{1}{\W}-Widerständen aufgebaute künstliche Antenne (Dummy Load). Mit dieser Kombination aus Reihen- und Parallelschaltung werden ca. \qty{50}{\ohm} erreicht. Wie viele Widerstände werden für diesen Aufbau benötigt und welche Dauerleistung verträgt diese künstliche Antenne?}{16~Widerstände, \qty{16}{\W}}
{48~Widerstände, \qty{12}{\W}}
{12~Widerstände, \qty{48}{\W}}
{\textbf{\textcolor{DARCgreen}{48~Widerstände, \qty{48}{\W}}}}
{\DARCimage{1.0\linewidth}{47include}}\end{PQuestion}

}
\end{frame}

\begin{frame}
\frametitle{Lösungsweg}
\begin{columns}
    \begin{column}{0.48\textwidth}
    \begin{itemize}
  \item gegeben: $R = 150Ω$
  \item gegeben: $R_S = 4\cdot 150Ω = 600Ω$
  \end{itemize}

    \end{column}
   \begin{column}{0.48\textwidth}
       \begin{itemize}
  \item gegeben: $R_{ges} = 50Ω$
  \item gegeben: $P_R = 1W$
  \item gesucht: $n$ Widerstände, $P$
  \end{itemize}

   \end{column}
\end{columns}
    \pause
    Reihen mit je 4 Widerständen:

$\frac{1}{R_{ges}} = n_S \cdot \frac{1}{R_S} \Rightarrow n_S = \frac{R_S}{R_{ges}} = \frac{600Ω}{50Ω} = 12$

$n = 4 \cdot n_S = 4 \cdot 12 = 48$
    \pause
    $P = n \cdot P_R = 48 \cdot 1W = 48W$



\end{frame}

\begin{frame}
\only<1>{
\begin{QQuestion}{AI602}{Eine künstliche Antenne (Dummy Load) verfügt über einen Messausgang, der intern an einen Spitzenwertgleichrichter angeschlossen ist. Wozu dient dieser Messausgang? Er dient~...}{zur indirekten Messung der Hochfrequenzleistung.}
{als Anschluss für einen Antennenvorverstärker.}
{als Abgriff einer ALC-Regelspannung für die Sendeendstufe.}
{zum Nachjustieren der Widerstände in der künstlichen Antenne.}
\end{QQuestion}

}
\only<2>{
\begin{QQuestion}{AI602}{Eine künstliche Antenne (Dummy Load) verfügt über einen Messausgang, der intern an einen Spitzenwertgleichrichter angeschlossen ist. Wozu dient dieser Messausgang? Er dient~...}{\textbf{\textcolor{DARCgreen}{zur indirekten Messung der Hochfrequenzleistung.}}}
{als Anschluss für einen Antennenvorverstärker.}
{als Abgriff einer ALC-Regelspannung für die Sendeendstufe.}
{zum Nachjustieren der Widerstände in der künstlichen Antenne.}
\end{QQuestion}

}
\end{frame}

\begin{frame}
\only<1>{
\begin{QQuestion}{AI603}{Eine künstliche Antenne (Dummy Load) von \qty{50}{\ohm} verfügt über eine Anzapfung bei \qty{5}{\ohm} vom erdnahen Ende. Was könnte zur ungefähren Ermittlung der Senderausgangsleistung über diesen Messpunkt eingesetzt werden?}{Stehwellenmessgerät mit Abschlusswiderstand.}
{Digitalmultimeter mit HF-Tastkopf.}
{Stehwellenmessgerät ohne Abschlusswiderstand.}
{Künstliche \qty{50}{\ohm}-Antenne mit zusätzlichem HF-Dämpfungsglied.}
\end{QQuestion}

}
\only<2>{
\begin{QQuestion}{AI603}{Eine künstliche Antenne (Dummy Load) von \qty{50}{\ohm} verfügt über eine Anzapfung bei \qty{5}{\ohm} vom erdnahen Ende. Was könnte zur ungefähren Ermittlung der Senderausgangsleistung über diesen Messpunkt eingesetzt werden?}{Stehwellenmessgerät mit Abschlusswiderstand.}
{\textbf{\textcolor{DARCgreen}{Digitalmultimeter mit HF-Tastkopf.}}}
{Stehwellenmessgerät ohne Abschlusswiderstand.}
{Künstliche \qty{50}{\ohm}-Antenne mit zusätzlichem HF-Dämpfungsglied.}
\end{QQuestion}

}
\end{frame}%ENDCONTENT
