
\section{Schutzerdung und Potentialausgleich I}
\label{section:schutzerdung_1}
\begin{frame}%STARTCONTENT

\begin{columns}
    \begin{column}{0.48\textwidth}
    Problem:

\begin{itemize}
  \item Leitfähige Gegenstände können unerwünschte Potentiale (Spannungen) aufweisen
  \item Z.B. elektrische Aufladung, Blitz oder Fehler im Gerät
  \end{itemize}

    \end{column}
   \begin{column}{0.48\textwidth}
       Maßnahmen:

\begin{itemize}
  \item Elektrisch leitende Teile miteinander verbinden
  \item Bei Koaxialkabeln die Schirme miteinander verbinden und an der Haupterdungsschiene anschließen
  \end{itemize}

   \end{column}
\end{columns}

\end{frame}

\begin{frame}
\only<1>{
\begin{QQuestion}{EK208}{Welche Maßnahmen müssen zum Personenschutz bei Koaxialkabeln zur Verhinderung von Spannungsunterschieden ergriffen werden?}{Die Schirme aller Koaxialkabel von Antennen müssen miteinander und mit der Haupterdungsschiene verbunden werden.}
{Für alle Koaxialkabel von Antennen sind Überspannungsableiter vorzusehen.}
{Neben der Erdung des Antennenmastes sind keine weiteren Maßnahmen erforderlich.}
{Die Koaxialkabel müssen ein Schirmungsmaß von mindestens 40 dB aufweisen.}
\end{QQuestion}

}
\only<2>{
\begin{QQuestion}{EK208}{Welche Maßnahmen müssen zum Personenschutz bei Koaxialkabeln zur Verhinderung von Spannungsunterschieden ergriffen werden?}{\textbf{\textcolor{DARCgreen}{Die Schirme aller Koaxialkabel von Antennen müssen miteinander und mit der Haupterdungsschiene verbunden werden.}}}
{Für alle Koaxialkabel von Antennen sind Überspannungsableiter vorzusehen.}
{Neben der Erdung des Antennenmastes sind keine weiteren Maßnahmen erforderlich.}
{Die Koaxialkabel müssen ein Schirmungsmaß von mindestens 40 dB aufweisen.}
\end{QQuestion}

}
\end{frame}%ENDCONTENT
