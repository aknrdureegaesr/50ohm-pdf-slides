
\section{Mehrwertige Verfahren}
\label{section:mehrwertige_verfahren}
\begin{frame}%STARTCONTENT
\begin{itemize}
  \item Viele digitale Modulationsverfahren verwenden mehr als zwei Symbole.
  \item So funktioniert zum Beispiel die 4-Fach-Amplitudenumtastung (4ASK) mit vier unterschiedlichen Amplituden, 25 %, 50 %, 75 %, 100 % des Maximums.
  \item So lassen sich zwei Bits zu einem Symbol zusammenfassen und gleichzeitig übertragen.
  \end{itemize}

\begin{figure}
    \DARCimage{0.85\linewidth}{701include}
    \caption{\scriptsize Quaternäre Amplitudenumtastung (Quaternary Amplitude-shift Keying)}
    \label{4ask}
\end{figure}

\end{frame}

\begin{frame}\begin{itemize}
  \item Dieses Prinzip lässt sich auf die Frequenz- und Phasenumtastung übertragen.
  \item Eine einfache Phasenumtastung (Binary Phase-Shift Keying, BPSK) verwendet nur zwei verschiedene Phasenlagen und kann daher nur ein Bit gleichzeitig senden.
  \item Die Quadraturphasenumtastung (Quadrature Phase-Shift Keying, QPSK) hingegen nutzt vier verschiedene Phasenlagen (0 °, 90 °, 180 ° und 270 °) und überträgt somit zwei Bits in jedem Schritt.
  \end{itemize}
\end{frame}

\begin{frame}
\only<1>{
\begin{QQuestion}{AE402}{Was unterscheidet BPSK- und QPSK-Modulation?}{Mit BPSK wird ein Bit pro Symbol übertragen, mit QPSK zwei Bit pro Symbol.}
{Mit QPSK wird ein Bit pro Symbol übertragen, mit BPSK zwei Bit pro Symbol.}
{Bei BPSK werden der I- und der Q-Anteil eines I/Q-Signals vertauscht, bei QPSK nicht.}
{Bei QPSK werden der I- und der Q-Anteil eines I/Q-Signals vertauscht, bei BPSK nicht.}
\end{QQuestion}

}
\only<2>{
\begin{QQuestion}{AE402}{Was unterscheidet BPSK- und QPSK-Modulation?}{\textbf{\textcolor{DARCgreen}{Mit BPSK wird ein Bit pro Symbol übertragen, mit QPSK zwei Bit pro Symbol.}}}
{Mit QPSK wird ein Bit pro Symbol übertragen, mit BPSK zwei Bit pro Symbol.}
{Bei BPSK werden der I- und der Q-Anteil eines I/Q-Signals vertauscht, bei QPSK nicht.}
{Bei QPSK werden der I- und der Q-Anteil eines I/Q-Signals vertauscht, bei BPSK nicht.}
\end{QQuestion}

}
\end{frame}

\begin{frame}\begin{itemize}
  \item Da bei Verfahren wie QPSK mehr als ein Bit pro Symbol übertragen wird, müssen wir mit den Einheiten aufpassen.
  \item Werden nur zwei Symbole verwendet und somit jedes Bit einzeln gesendet, entspricht die Symbolrate in Baud der Datenrate in Bit/s.
  \item Werden jedoch mehr Symbole verwendet und somit mehrere Bits gleichzeitig übertragen, ist die Datenrate höher als die Symbolrate.
  \end{itemize}
\end{frame}

\begin{frame}\begin{itemize}
  \item Die Formel $C = R_{ s } \cdot n$ stellt den Zusammenhang dar:
  \end{itemize}
C → Datenübertragungsrate in Bit/s

$R_{ s }$ → Symbolrate in Baud

n → Symbolgröße in Bit/Symbol

\end{frame}

\begin{frame}
\only<1>{
\begin{QQuestion}{AA104}{Welche Einheit wird üblicherweise für die Symbolrate verwendet?}{Bit pro Sekunde (Bit/s)}
{Baud (Bd)}
{Hertz (Hz)}
{Dezibel (dB)}
\end{QQuestion}

}
\only<2>{
\begin{QQuestion}{AA104}{Welche Einheit wird üblicherweise für die Symbolrate verwendet?}{Bit pro Sekunde (Bit/s)}
{\textbf{\textcolor{DARCgreen}{Baud (Bd)}}}
{Hertz (Hz)}
{Dezibel (dB)}
\end{QQuestion}

}
\end{frame}

\begin{frame}Beispiele:

\emph{RTTY}: Umschaltung zwischen zwei Symbolfrequenzen, so dass pro Symbol ein Bit (0 oder 1) übertragen werden kann.

→ Datenrate = Symbolrate

\emph{FT4}: Umschaltung zwischen vier Symbolfrequenzen, so dass pro Symbol zwei Bit (00, 01, 10 oder 11) übertragen werden können.

→ Datenrate = 2 $\cdot$ Symbolrate

\end{frame}

\begin{frame}
\only<1>{
\begin{QQuestion}{AE405}{Bei einem digitalen Übertragungsverfahren (z.~B. RTTY) wird die Frequenz eines Senders zwischen zwei Symbolfrequenzen (z.~B. \qty{14072,43}{\kHz} und \qty{14072,60}{\kHz}) umgetastet, so dass pro Symbol ein Bit (0 oder 1) übertragen werden kann. Die Symbolrate beträgt \qty{45,45}{\baud}. Welcher Datenrate entspricht das?}{\qty[per-mode=symbol]{22,725}{\bit\per\s}}
{\qty[per-mode=symbol]{90,9}{\bit\per\s}}
{\qty[per-mode=symbol]{45,45}{\bit\per\s}}
{\qty[per-mode=symbol]{181,8}{\bit\per\s}}
\end{QQuestion}

}
\only<2>{
\begin{QQuestion}{AE405}{Bei einem digitalen Übertragungsverfahren (z.~B. RTTY) wird die Frequenz eines Senders zwischen zwei Symbolfrequenzen (z.~B. \qty{14072,43}{\kHz} und \qty{14072,60}{\kHz}) umgetastet, so dass pro Symbol ein Bit (0 oder 1) übertragen werden kann. Die Symbolrate beträgt \qty{45,45}{\baud}. Welcher Datenrate entspricht das?}{\qty[per-mode=symbol]{22,725}{\bit\per\s}}
{\qty[per-mode=symbol]{90,9}{\bit\per\s}}
{\textbf{\textcolor{DARCgreen}{\qty[per-mode=symbol]{45,45}{\bit\per\s}}}}
{\qty[per-mode=symbol]{181,8}{\bit\per\s}}
\end{QQuestion}

}
\end{frame}

\begin{frame}
\frametitle{Lösungsweg}
\begin{itemize}
  \item gegeben: $R_S = 45,45Bd$
  \item gegeben: $n=1\frac{Bit}{Symbol}$
  \item gesucht: $C$
  \end{itemize}
    \pause
    $C = R_S \cdot n = 45,45Bd \cdot 1 = 45,45\frac{Bit}{s}$



\end{frame}

\begin{frame}
\only<1>{
\begin{QQuestion}{AE406}{Bei einem digitalen Übertragungsverfahren (z.~B. FT4) wird die Frequenz eines Senders zwischen vier Symbolfrequenzen (z.~B. \qty{14081,20}{\kHz}, \qty{14081,40}{\kHz}, \qty{14081,61}{\kHz} und \qty{14081,83}{\kHz}) umgetastet, so dass pro Symbol zwei Bit (00, 01, 10 oder 11) übertragen werden können. Die Symbolrate beträgt \qty{23,4}{\baud}. Welcher Datenrate entspricht das?}{\qty[per-mode=symbol]{93,6}{\bit\per\s}}
{\qty[per-mode=symbol]{11,7}{\bit\per\s}}
{\qty[per-mode=symbol]{23,4}{\bit\per\s}}
{\qty[per-mode=symbol]{46,8}{\bit\per\s}}
\end{QQuestion}

}
\only<2>{
\begin{QQuestion}{AE406}{Bei einem digitalen Übertragungsverfahren (z.~B. FT4) wird die Frequenz eines Senders zwischen vier Symbolfrequenzen (z.~B. \qty{14081,20}{\kHz}, \qty{14081,40}{\kHz}, \qty{14081,61}{\kHz} und \qty{14081,83}{\kHz}) umgetastet, so dass pro Symbol zwei Bit (00, 01, 10 oder 11) übertragen werden können. Die Symbolrate beträgt \qty{23,4}{\baud}. Welcher Datenrate entspricht das?}{\qty[per-mode=symbol]{93,6}{\bit\per\s}}
{\qty[per-mode=symbol]{11,7}{\bit\per\s}}
{\qty[per-mode=symbol]{23,4}{\bit\per\s}}
{\textbf{\textcolor{DARCgreen}{\qty[per-mode=symbol]{46,8}{\bit\per\s}}}}
\end{QQuestion}

}
\end{frame}

\begin{frame}
\frametitle{Lösungsweg}
\begin{itemize}
  \item gegeben: $R_S = 23,4Bd$
  \item gegeben: $n=2\frac{Bit}{Symbol}$
  \item gesucht: $C$
  \end{itemize}
    \pause
    $C = R_S \cdot n = 23,4 \cdot 2 = 46,8\frac{Bit}{s}$



\end{frame}%ENDCONTENT
