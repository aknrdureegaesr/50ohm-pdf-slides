
\section{Gefahren durch elektrischen Strom}
\label{section:gefahren}
\begin{frame}%STARTCONTENT
\begin{itemize}
  \item Stromschlag vermeiden!
  \item An anerkannte Regeln der Technik halten
  \item Vom Verband der Elektrotechnik Elektronik und Informationstechnik e.V. (VDE)
  \item Schutz von Menschen, Tieren und Sachen
  \end{itemize}
\end{frame}

\begin{frame}
\only<1>{
\begin{QQuestion}{VE601}{Wie ist die Stromversorgung von Eigenbaugeräten elektrotechnisch sicher aufzubauen?}{Es gelten die Vorschriften der örtlichen Stromversorger.}
{Es gelten keine besonderen Vorschriften, da ein Funkamateur eine sachkundige Person ist.}
{Nach den anerkannte Regeln der Technik, wie sie z. B. in den VDE-Normen festgelegt sind.}
{Sie ist nach den CEPT-Empfehlungen aufzubauen.}
\end{QQuestion}

}
\only<2>{
\begin{QQuestion}{VE601}{Wie ist die Stromversorgung von Eigenbaugeräten elektrotechnisch sicher aufzubauen?}{Es gelten die Vorschriften der örtlichen Stromversorger.}
{Es gelten keine besonderen Vorschriften, da ein Funkamateur eine sachkundige Person ist.}
{\textbf{\textcolor{DARCgreen}{Nach den anerkannte Regeln der Technik, wie sie z. B. in den VDE-Normen festgelegt sind.}}}
{Sie ist nach den CEPT-Empfehlungen aufzubauen.}
\end{QQuestion}

}
\end{frame}

\begin{frame}
\frametitle{Gefährliche Spannung}
\begin{itemize}
  \item Wechselspannung (AC) über 50 V
  \item Gleichspannung (DC) über 120 V
  \item Darunter kommt es zu keinen lebensbedrohlichen Beeinträchtigungen des menschlichen Körpers
  \end{itemize}

\end{frame}

\begin{frame}
\only<1>{
\begin{QQuestion}{NK301}{Ab welcher Höhe kann das Berühren elektrischer Wechselspannung (AC) und elektrischer Gleichspannung (DC) für den erwachsenen Menschen lebensgefährlich sein?}{\qty{50}{\V} (AC), \qty{120}{\V} (DC)}
{\qty{20}{\V} (AC), \qty{60}{\V} (DC)}
{\qty{100}{\V} (AC), \qty{140}{\V} (DC)}
{\qty{75}{\V} (AC), \qty{150}{\V} (DC)}
\end{QQuestion}

}
\only<2>{
\begin{QQuestion}{NK301}{Ab welcher Höhe kann das Berühren elektrischer Wechselspannung (AC) und elektrischer Gleichspannung (DC) für den erwachsenen Menschen lebensgefährlich sein?}{\textbf{\textcolor{DARCgreen}{\qty{50}{\V} (AC), \qty{120}{\V} (DC)}}}
{\qty{20}{\V} (AC), \qty{60}{\V} (DC)}
{\qty{100}{\V} (AC), \qty{140}{\V} (DC)}
{\qty{75}{\V} (AC), \qty{150}{\V} (DC)}
\end{QQuestion}

}
\end{frame}

\begin{frame}
\frametitle{Stromunfälle}
\begin{columns}
    \begin{column}{0.48\textwidth}
    \begin{itemize}
  \item Abhängig von Stromstärke und Dauer des Stromflusses
  \item Weg durch den Körper
  \item Ab 30 mA lebensgefährliche Schäden
  \end{itemize}

    \end{column}
   \begin{column}{0.48\textwidth}
       
\begin{figure}
    \DARCimage{0.85\linewidth}{681include}
    \caption{\scriptsize Stromschlag / Körperdurchströmung}
    \label{n_fehlerstrom}
\end{figure}


   \end{column}
\end{columns}

\end{frame}

\begin{frame}
\frametitle{Auswirkungen auf den Körper}
\begin{itemize}
  \item \emph{Herzrhythmusstörungen}, Herzkammerflimmern oder Herzstillstand, inbesondere bei einem Stromweg im Brustbereich
  \item \emph{Verbrennungen}, meist an den Ein- und Austrittstellen des elektrischen Stroms
  \item \emph{Verkrampfen der Muskulatur}
  \item \emph{Sekundärunfälle} wie einen Sturz, verursacht durch den hervorgerufenden Schreck oder eine Muskelverkrampfung
  \item Zusätzlich \emph{(Stör-)Lichtbogen} mit hellem Leuchten über die Luft möglich
  \end{itemize}
\end{frame}

\begin{frame}
\frametitle{Gefahr beim Öffnen von Geräten}
\begin{itemize}
  \item Kondensatoren können hohe Spannungen speichern
  \item Es können in abgeschalteten Geräten noch gefährliche Spannungen anliegen
  \item Beim Öffnen von Geräten erfahrenen Funkamateur oder Elektrofachkraft zu Hilfe holen
  \end{itemize}
\end{frame}

\begin{frame}
\only<1>{
\begin{QQuestion}{NK303}{Welche gefährlichen Folgen kann eine Körperdurchströmung mit elektrischem Strom verursachen?}{Verätzungen, Muskelentzündungen, Herzklopfen}
{Verbrennungen, Muskelverkrampfungen, Herzrhythmusstörungen}
{Verbrühungen, Muskelkater, Atembeschwerden}
{Verkochungen, Muskelzucken, Herzasthma}
\end{QQuestion}

}
\only<2>{
\begin{QQuestion}{NK303}{Welche gefährlichen Folgen kann eine Körperdurchströmung mit elektrischem Strom verursachen?}{Verätzungen, Muskelentzündungen, Herzklopfen}
{\textbf{\textcolor{DARCgreen}{Verbrennungen, Muskelverkrampfungen, Herzrhythmusstörungen}}}
{Verbrühungen, Muskelkater, Atembeschwerden}
{Verkochungen, Muskelzucken, Herzasthma}
\end{QQuestion}

}
\end{frame}

\begin{frame}
\only<1>{
\begin{QQuestion}{NK302}{Die größten Gefährdungen durch elektrischen Strom sind insbesondere~...}{Lichtblitze, Stromspitzen, Folgeschäden durch Ohnmacht}
{Stromschlag, Kurzschluss, Auslösen von Sicherungen}
{Stromunfälle, Spannungsabfälle, Unfälle durch Erschrecken}
{elektrische Körperdurchströmung, Störlichtbogen, Sekundärunfälle}
\end{QQuestion}

}
\only<2>{
\begin{QQuestion}{NK302}{Die größten Gefährdungen durch elektrischen Strom sind insbesondere~...}{Lichtblitze, Stromspitzen, Folgeschäden durch Ohnmacht}
{Stromschlag, Kurzschluss, Auslösen von Sicherungen}
{Stromunfälle, Spannungsabfälle, Unfälle durch Erschrecken}
{\textbf{\textcolor{DARCgreen}{elektrische Körperdurchströmung, Störlichtbogen, Sekundärunfälle}}}
\end{QQuestion}

}
\end{frame}

\begin{frame}
\frametitle{Erste Hilfe}
\begin{itemize}
  \item In den ersten Minuten entscheidend für die Schwere der Unfallfolgen
  \item Unbedingt Arzt aufsuchen
  \item Herzrhythmusstörungen und Herzkammerflimmern können Stunden nach dem Unfall auftreten
  \end{itemize}

\end{frame}

\begin{frame}
\only<1>{
\begin{QQuestion}{NK304}{Welche Maßnahme ist nach einem Elektrounfall mit Körperdurchströmung (Stromschlag) zu ergreifen?}{Es ist ein Arzt aufzusuchen, da Herzrhythmusstörungen und Herzkammerflimmern auch noch viele Stunden nach einem Stromschlag auftreten können.}
{Personen, die einen Stromschlag erlitten haben, sind unverzüglich in eine stabile Seitenlage zu bringen.}
{Sofern sich die verunfallte Person gut fühlt, sind keine Maßnahmen erforderlich.}
{Bei Stromschlag mit Wechselstrom (AC) ist ein Arzt aufzusuchen, bei Stromschlag mit Gleichstrom (DC) ist kein Arzt erforderlich.}
\end{QQuestion}

}
\only<2>{
\begin{QQuestion}{NK304}{Welche Maßnahme ist nach einem Elektrounfall mit Körperdurchströmung (Stromschlag) zu ergreifen?}{\textbf{\textcolor{DARCgreen}{Es ist ein Arzt aufzusuchen, da Herzrhythmusstörungen und Herzkammerflimmern auch noch viele Stunden nach einem Stromschlag auftreten können.}}}
{Personen, die einen Stromschlag erlitten haben, sind unverzüglich in eine stabile Seitenlage zu bringen.}
{Sofern sich die verunfallte Person gut fühlt, sind keine Maßnahmen erforderlich.}
{Bei Stromschlag mit Wechselstrom (AC) ist ein Arzt aufzusuchen, bei Stromschlag mit Gleichstrom (DC) ist kein Arzt erforderlich.}
\end{QQuestion}

}
\end{frame}

\begin{frame}
\frametitle{5 Sicherheitsregeln in der Elektrotechnik}
\begin{enumerate}
  \item[1] \emph{Freischalten}, z.\,B. Gerät ausschalten
  \item[2] \emph{Gegen Wiedereinschalten sichern}, z.\,B.  Stecker ziehen
  \item[3] \emph{Spannungsfreiheit feststellen}, z.\,B. mit einem Multimeter messen
  \item[4] \emph{Erden und Kurzschließen}, z.\,B. das Gehäuse und Zuleitungen erden
  \item[5] \emph{Benachbarte, unter Spannung stehende Teile abdecken oder abschranken} (findet bei einzelnen Geräten meist keine Anwendung)
  \end{enumerate}
\end{frame}%ENDCONTENT
