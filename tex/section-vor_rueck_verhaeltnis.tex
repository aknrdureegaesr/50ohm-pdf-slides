
\section{Vor-/Rückverhältnis}
\label{section:vor_rueck_verhaeltnis}
\begin{frame}%STARTCONTENT

\only<1>{
\begin{PQuestion}{AG214}{Das folgende Bild zeigt die Strahlungscharakteristik eines Dipols und einer Richtantenne. Das Vor-/Rück-Verhältnis der Richtantenne ist definiert als das Verhältnis~...}{von $P_{\symup{D}}$ zu $P_{\symup{R}}$.}
{von $P_{\symup{V}}$ zu $P_{\symup{R}}$.}
{von $P_{\symup{V}}$ zu $P_{\symup{D}}$.}
{von $0{,}7 \cdot P_{\symup{V}}$ zu $0{,}7 \cdot P_{\symup{D}}$.}
{\DARCimage{1.0\linewidth}{264include}}\end{PQuestion}

}
\only<2>{
\begin{PQuestion}{AG214}{Das folgende Bild zeigt die Strahlungscharakteristik eines Dipols und einer Richtantenne. Das Vor-/Rück-Verhältnis der Richtantenne ist definiert als das Verhältnis~...}{von $P_{\symup{D}}$ zu $P_{\symup{R}}$.}
{\textbf{\textcolor{DARCgreen}{von $P_{\symup{V}}$ zu $P_{\symup{R}}$.}}}
{von $P_{\symup{V}}$ zu $P_{\symup{D}}$.}
{von $0{,}7 \cdot P_{\symup{V}}$ zu $0{,}7 \cdot P_{\symup{D}}$.}
{\DARCimage{1.0\linewidth}{264include}}\end{PQuestion}

}
\end{frame}

\begin{frame}
\only<1>{
\begin{PQuestion}{AG213}{Das folgende Bild zeigt die Strahlungscharakteristik eines Dipols und einer Richtantenne. Der Antennengewinn der Richtantenne über dem Dipol ist definiert als das Verhältnis~...}{von $P_{\symup{V}}$ zu $P_{\symup{R}}$.}
{von $P_{\symup{D}}$ zu $P_{\symup{R}}$.}
{von $P_{\symup{V}}$ zu $P_{\symup{D}}$.}
{von $0{,}7 \cdot P_{\symup{V}}$ zu $0{,}7 \cdot P_{\symup{R}}$.}
{\DARCimage{1.0\linewidth}{264include}}\end{PQuestion}

}
\only<2>{
\begin{PQuestion}{AG213}{Das folgende Bild zeigt die Strahlungscharakteristik eines Dipols und einer Richtantenne. Der Antennengewinn der Richtantenne über dem Dipol ist definiert als das Verhältnis~...}{von $P_{\symup{V}}$ zu $P_{\symup{R}}$.}
{von $P_{\symup{D}}$ zu $P_{\symup{R}}$.}
{\textbf{\textcolor{DARCgreen}{von $P_{\symup{V}}$ zu $P_{\symup{D}}$.}}}
{von $0{,}7 \cdot P_{\symup{V}}$ zu $0{,}7 \cdot P_{\symup{R}}$.}
{\DARCimage{1.0\linewidth}{264include}}\end{PQuestion}

}
\end{frame}

\begin{frame}
\only<1>{
\begin{PQuestion}{AG217}{Bei einer Yagi-Uda-Antenne mit dem folgenden Strahlungsdiagramm beträgt die ERP in Richtung a \qty{0,6}{\W} und in Richtung b \qty{15}{\W}. Welches Vor-Rück-Verhältnis hat die Antenne?}{\qty{2,8}{\decibel}}
{\qty{27,9}{\decibel}}
{\qty{14}{\decibel}}
{\qty{25}{\decibel}}
{\DARCimage{1.0\linewidth}{263include}}\end{PQuestion}

}
\only<2>{
\begin{PQuestion}{AG217}{Bei einer Yagi-Uda-Antenne mit dem folgenden Strahlungsdiagramm beträgt die ERP in Richtung a \qty{0,6}{\W} und in Richtung b \qty{15}{\W}. Welches Vor-Rück-Verhältnis hat die Antenne?}{\qty{2,8}{\decibel}}
{\qty{27,9}{\decibel}}
{\textbf{\textcolor{DARCgreen}{\qty{14}{\decibel}}}}
{\qty{25}{\decibel}}
{\DARCimage{1.0\linewidth}{263include}}\end{PQuestion}

}
\end{frame}

\begin{frame}
\frametitle{Lösungsweg}
\begin{itemize}
  \item gegeben: $P_R = 0,6W$
  \item gegeben: $P_V = 15W$
  \item gesucht: $\frac{Vor}{Rück}$
  \end{itemize}
    \pause
    $\frac{Vor}{Rück} = 10 \cdot \log_{10}{(\frac{P_V}{P_R})} dB = 10 \cdot \log_{10}{(\frac{15W}{0,6W})} dB = 14dB$



\end{frame}

\begin{frame}
\only<1>{
\begin{QQuestion}{AG215}{Eine Richtantenne mit einem Gewinn von \qty{10}{\decibel} über dem Halbwellendipol und einem Vor-Rück-Verhältnis von \qty{20}{\decibel} wird mit \qty{100}{\W} Sendeleistung direkt gespeist. Welche ERP strahlt die Antenne entgegengesetzt zur Senderichtung ab?}{\qty{100}{\W}}
{\qty{10}{\W}}
{\qty{0,1}{\W}}
{\qty{1}{\W}}
\end{QQuestion}

}
\only<2>{
\begin{QQuestion}{AG215}{Eine Richtantenne mit einem Gewinn von \qty{10}{\decibel} über dem Halbwellendipol und einem Vor-Rück-Verhältnis von \qty{20}{\decibel} wird mit \qty{100}{\W} Sendeleistung direkt gespeist. Welche ERP strahlt die Antenne entgegengesetzt zur Senderichtung ab?}{\qty{100}{\W}}
{\textbf{\textcolor{DARCgreen}{\qty{10}{\W}}}}
{\qty{0,1}{\W}}
{\qty{1}{\W}}
\end{QQuestion}

}
\end{frame}

\begin{frame}
\frametitle{Lösungsweg}
\begin{itemize}
  \item gegeben: $g_D= 10dB$
  \item gegeben: $\frac{Vor}{Rück} = 20dB$
  \item gegeben: $P_S = 100W$
  \item gesucht: $P_R$
  \end{itemize}
    \pause
    $P_V = P_{ERP} = P_S \cdot 10^{\frac{g_d}{10dB}} = 100W \cdot 10^{\frac{10dB}{10dB}} = 1000W$
    \pause
    $20dB = 10 \cdot \log_{10}{(\frac{P_V}{P_R})} dB \Rightarrow \frac{P_V}{P_R} = 10^{\frac{20dB}{10}} = 100 \Rightarrow P_R = \frac{P_V}{100} = \frac{1000W}{100} = 10W$



\end{frame}

\begin{frame}
\only<1>{
\begin{QQuestion}{AG216}{Eine Richtantenne mit einem Gewinn von \qty{15}{\decibel} über dem Halbwellendipol und einem Vor-Rück-Verhältnis von \qty{25}{\decibel} wird mit \qty{6}{\W} Sendeleistung direkt gespeist. Welche ERP strahlt die Antenne entgegengesetzt zur Senderichtung ab?}{\qty{60}{\W}}
{\qty{0,019}{\W}}
{\qty{0,19}{\W}}
{\qty{0,6}{\W}}
\end{QQuestion}

}
\only<2>{
\begin{QQuestion}{AG216}{Eine Richtantenne mit einem Gewinn von \qty{15}{\decibel} über dem Halbwellendipol und einem Vor-Rück-Verhältnis von \qty{25}{\decibel} wird mit \qty{6}{\W} Sendeleistung direkt gespeist. Welche ERP strahlt die Antenne entgegengesetzt zur Senderichtung ab?}{\qty{60}{\W}}
{\qty{0,019}{\W}}
{\qty{0,19}{\W}}
{\textbf{\textcolor{DARCgreen}{\qty{0,6}{\W}}}}
\end{QQuestion}

}
\end{frame}

\begin{frame}
\frametitle{Lösungsweg}
\begin{itemize}
  \item gegeben: $g_D= 15dB$
  \item gegeben: $\frac{Vor}{Rück} = 25dB$
  \item gegeben: $P_S = 6W$
  \item gesucht: $P_R$
  \end{itemize}
    \pause
    $P_V = P_{ERP} = P_S \cdot 10^{\frac{g_d}{10dB}} = 6W \cdot 10^{\frac{15dB}{10dB}} = 189,7W$
    \pause
    $25dB = 10 \cdot \log_{10}{(\frac{P_V}{P_R})} dB \Rightarrow \frac{P_V}{P_R} = 10^{\frac{25dB}{10}} = 316,2 \Rightarrow P_R = \frac{P_V}{316,2} = \frac{189,7W}{316,2} = 0,6W$



\end{frame}

\begin{frame}
\only<1>{
\begin{QQuestion}{AG218}{Mit einem Feldstärkemessgerät wurden Vergleichsmessungen zwischen Beam und Dipol durchgeführt. In einem Abstand von \qty{32}{\m} wurden folgende Feldstärken gemessen: Beam vorwärts: \qty{300}{\micro\V}/m, Beam rückwärts: \qty{20}{\micro\V}/m, Halbwellendipol in Hauptstrahlrichtung: \qty{128}{\micro\V}/m. Welcher Gewinn und welches Vor-Rück-Verhältnis ergibt sich daraus für den Beam?}{Gewinn: 7,4~dBd, Vor-Rück-Verhältnis: \qty{15}{\decibel}}
{Gewinn: 3,7~dBd, Vor-Rück-Verhältnis: \qty{11,7}{\decibel}}
{Gewinn: 9,4~dBd, Vor-Rück-Verhältnis: \qty{23,5}{\decibel}}
{Gewinn: 7,4~dBd, Vor-Rück-Verhältnis: \qty{23,5}{\decibel}}
\end{QQuestion}

}
\only<2>{
\begin{QQuestion}{AG218}{Mit einem Feldstärkemessgerät wurden Vergleichsmessungen zwischen Beam und Dipol durchgeführt. In einem Abstand von \qty{32}{\m} wurden folgende Feldstärken gemessen: Beam vorwärts: \qty{300}{\micro\V}/m, Beam rückwärts: \qty{20}{\micro\V}/m, Halbwellendipol in Hauptstrahlrichtung: \qty{128}{\micro\V}/m. Welcher Gewinn und welches Vor-Rück-Verhältnis ergibt sich daraus für den Beam?}{Gewinn: 7,4~dBd, Vor-Rück-Verhältnis: \qty{15}{\decibel}}
{Gewinn: 3,7~dBd, Vor-Rück-Verhältnis: \qty{11,7}{\decibel}}
{Gewinn: 9,4~dBd, Vor-Rück-Verhältnis: \qty{23,5}{\decibel}}
{\textbf{\textcolor{DARCgreen}{Gewinn: 7,4~dBd, Vor-Rück-Verhältnis: \qty{23,5}{\decibel}}}}
\end{QQuestion}

}
\end{frame}

\begin{frame}
\frametitle{Lösungsweg}
\begin{itemize}
  \item gegeben: $U_V = 300µV/m$
  \item gegeben: $U_R = 20µV/m$
  \item gegeben: $U_D = 128µV/m$
  \item gesucht: $g_D$, $\frac{Vor}{Rück}$
  \end{itemize}
    \pause
    $g_D = 20 \cdot \log_{10}{(\frac{U_V}{U_D})} dB = 20 \cdot \log_{10}{(\frac{300µV/m}{128µV/m})} = 7,4dB$
    \pause
    $\frac{Vor}{Rück} = 20 \cdot \log_{10}{(\frac{U_V}{U_R})} dB = 20 \cdot \log_{10}{(\frac{300µV/m}{20µV/m})} = 23,5dB$



\end{frame}%ENDCONTENT
