
\section{Anwendung von ADC und DAC}
\label{section:anwendung_dac_adc}
\begin{frame}%STARTCONTENT

\only<1>{
\begin{question2x2}{AF613}{Eine Sinusschwingung mit einem Spitzenwert von \qty{1,5}{\V} wird in einen A/D-Umsetzer eingegeben, dessen Ausgang direkt mit einem D/A-Umsetzer verbunden ist. Beide Umsetzer arbeiten linear mit einer Auflösung von \qty{12}{\bit} und einem Wertebereich von \qty{-2}{\V} bis \qty{2}{\V}. Welches Signal ist am Ausgang des D/A-Umsetzers zu erwarten?}{\DARCimage{1.0\linewidth}{299include}}
{\DARCimage{1.0\linewidth}{297include}}
{\DARCimage{1.0\linewidth}{295include}}
{\DARCimage{1.0\linewidth}{300include}}
\end{question2x2}

}
\only<2>{
\begin{question2x2}{AF613}{Eine Sinusschwingung mit einem Spitzenwert von \qty{1,5}{\V} wird in einen A/D-Umsetzer eingegeben, dessen Ausgang direkt mit einem D/A-Umsetzer verbunden ist. Beide Umsetzer arbeiten linear mit einer Auflösung von \qty{12}{\bit} und einem Wertebereich von \qty{-2}{\V} bis \qty{2}{\V}. Welches Signal ist am Ausgang des D/A-Umsetzers zu erwarten?}{\DARCimage{1.0\linewidth}{299include}}
{\DARCimage{1.0\linewidth}{297include}}
{\textbf{\textcolor{DARCgreen}{\DARCimage{1.0\linewidth}{295include}}}}
{\DARCimage{1.0\linewidth}{300include}}
\end{question2x2}

}
\end{frame}

\begin{frame}
\only<1>{
\begin{question2x2}{AF612}{Eine Sinusschwingung mit einem Spitzenwert von \qty{1,5}{\V} wird in einen A/D-Umsetzer eingegeben, dessen Ausgang direkt mit einem D/A-Umsetzer verbunden ist. Beide Umsetzer arbeiten linear mit einer Auflösung von \qty{4}{\bit} und einem Wertebereich von \qty{-2}{\V} bis \qty{2}{\V}. Welches Signal ist am Ausgang des D/A-Umsetzers zu erwarten?}{\DARCimage{1.0\linewidth}{295include}}
{\DARCimage{1.0\linewidth}{297include}}
{\DARCimage{1.0\linewidth}{299include}}
{\DARCimage{1.0\linewidth}{298include}}
\end{question2x2}

}
\only<2>{
\begin{question2x2}{AF612}{Eine Sinusschwingung mit einem Spitzenwert von \qty{1,5}{\V} wird in einen A/D-Umsetzer eingegeben, dessen Ausgang direkt mit einem D/A-Umsetzer verbunden ist. Beide Umsetzer arbeiten linear mit einer Auflösung von \qty{4}{\bit} und einem Wertebereich von \qty{-2}{\V} bis \qty{2}{\V}. Welches Signal ist am Ausgang des D/A-Umsetzers zu erwarten?}{\DARCimage{1.0\linewidth}{295include}}
{\textbf{\textcolor{DARCgreen}{\DARCimage{1.0\linewidth}{297include}}}}
{\DARCimage{1.0\linewidth}{299include}}
{\DARCimage{1.0\linewidth}{298include}}
\end{question2x2}

}
\end{frame}

\begin{frame}
\only<1>{
\begin{question2x2}{AF614}{Eine Sinusschwingung mit einem Spitzenwert von \qty{1,5}{\V} wird in einen A/D-Umsetzer eingegeben, dessen Ausgang direkt mit einem D/A-Umsetzer verbunden ist. Beide Umsetzer arbeiten linear mit einer Auflösung von \qty{12}{\bit} und einem Wertebereich von \qty{-1}{\V} bis \qty{1}{\V}. Welches Signal ist am Ausgang des D/A-Umsetzers zu erwarten?}{\DARCimage{1.0\linewidth}{298include}}
{\DARCimage{1.0\linewidth}{296include}}
{\DARCimage{1.0\linewidth}{295include}}
{\DARCimage{1.0\linewidth}{297include}}
\end{question2x2}

}
\only<2>{
\begin{question2x2}{AF614}{Eine Sinusschwingung mit einem Spitzenwert von \qty{1,5}{\V} wird in einen A/D-Umsetzer eingegeben, dessen Ausgang direkt mit einem D/A-Umsetzer verbunden ist. Beide Umsetzer arbeiten linear mit einer Auflösung von \qty{12}{\bit} und einem Wertebereich von \qty{-1}{\V} bis \qty{1}{\V}. Welches Signal ist am Ausgang des D/A-Umsetzers zu erwarten?}{\DARCimage{1.0\linewidth}{298include}}
{\textbf{\textcolor{DARCgreen}{\DARCimage{1.0\linewidth}{296include}}}}
{\DARCimage{1.0\linewidth}{295include}}
{\DARCimage{1.0\linewidth}{297include}}
\end{question2x2}

}
\end{frame}%ENDCONTENT
