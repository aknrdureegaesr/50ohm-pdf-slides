
\section{Logbuch}
\label{section:logbuch}
\begin{frame}%STARTCONTENT
\begin{itemize}
  \item Mit einem Logbuch hat man die Möglichkeit, seine Funkaktivitäten zu dokumentieren
  \item Die Führung eines Logbuchs ist freiwillig
  \end{itemize}

\end{frame}

\begin{frame}Folgende Daten werden üblicherweise in einem Logbuch für jede Funkverbindung festgehalten:

\begin{itemize}
  \item Rufzeichen der Gegenstation
  \item Frequenz oder Band
  \item Datum und Uhrzeit
  \item Übertragungsverfahren (z.\,B. SSB, FT8, RTTY,~...)
  \item Vergebener und erhaltener Rapport
  \item Verwendete Sendeleistung
  \item Bemerkungen (z.B. Name des QSO-Partners oder eigene verwendete Station)
  \end{itemize}
\end{frame}

\begin{frame}
\only<1>{
\begin{QQuestion}{BG101}{Was verstehen Funkamateure unter einem Logbuch?}{Es ist die Dokumentation aller Geräte und Antennen des Funkamateurs.}
{Es ist das Stationstagebuch, das jeder Funkamateur führen muss.}
{Es ist das Stationstagebuch, das ein Funkamateur freiwillig führt oder in besonderen Fällen führen muss.}
{Es ist die Dokumentation über die Einhaltung der Sicherheitsabstände bezüglich des Personenschutzes.}
\end{QQuestion}

}
\only<2>{
\begin{QQuestion}{BG101}{Was verstehen Funkamateure unter einem Logbuch?}{Es ist die Dokumentation aller Geräte und Antennen des Funkamateurs.}
{Es ist das Stationstagebuch, das jeder Funkamateur führen muss.}
{\textbf{\textcolor{DARCgreen}{Es ist das Stationstagebuch, das ein Funkamateur freiwillig führt oder in besonderen Fällen führen muss.}}}
{Es ist die Dokumentation über die Einhaltung der Sicherheitsabstände bezüglich des Personenschutzes.}
\end{QQuestion}

}
\end{frame}

\begin{frame}
\frametitle{Verpflichtende Logbuchführung}
Es kann aber auch dazu kommen, dass man eine Aufforderung der Bundesnetzagentur erhält, die einen verpflichtet, ein Logbuch zu führen.

\begin{itemize}
  \item Zur Untersuchung von Störungsursachen
  \item Zur Klärung frequenztechnischer Fragen
  \end{itemize}
\end{frame}

\begin{frame}
\only<1>{
\begin{QQuestion}{VD109}{Wann muss der Funkamateur Angaben über den Betrieb seiner Amateurfunkstelle schriftlich festhalten, z. B. als Logbuch?}{Bei Funkbetrieb auf der Kurzwelle}
{Auf Verlangen der Bundesnetzagentur}
{Bei internationalem Funkbetrieb}
{In den ersten 12 Monaten nach der Zulassung}
\end{QQuestion}

}
\only<2>{
\begin{QQuestion}{VD109}{Wann muss der Funkamateur Angaben über den Betrieb seiner Amateurfunkstelle schriftlich festhalten, z. B. als Logbuch?}{Bei Funkbetrieb auf der Kurzwelle}
{\textbf{\textcolor{DARCgreen}{Auf Verlangen der Bundesnetzagentur}}}
{Bei internationalem Funkbetrieb}
{In den ersten 12 Monaten nach der Zulassung}
\end{QQuestion}

}
\end{frame}

\begin{frame}
\only<1>{
\begin{QQuestion}{VD108}{Zu welchen Zwecken kann die Bundesnetzagentur schriftliche Nachweise über den Funkbetrieb verlangen?}{Zur Überprüfung der Qualifikation des Funkamateurs und des Inhalts seiner Aussendungen}
{Zur Untersuchung von Störungsursachen oder zur Klärung frequenztechnischer Fragen}
{Als Nachweis für die Einhaltung von Grenzwerten nach dem Bundes-Immissionsschutzgesetz (BImSchG)}
{Als Nachweis zur Abrechnung der Frequenznutzungsbeiträge}
\end{QQuestion}

}
\only<2>{
\begin{QQuestion}{VD108}{Zu welchen Zwecken kann die Bundesnetzagentur schriftliche Nachweise über den Funkbetrieb verlangen?}{Zur Überprüfung der Qualifikation des Funkamateurs und des Inhalts seiner Aussendungen}
{\textbf{\textcolor{DARCgreen}{Zur Untersuchung von Störungsursachen oder zur Klärung frequenztechnischer Fragen}}}
{Als Nachweis für die Einhaltung von Grenzwerten nach dem Bundes-Immissionsschutzgesetz (BImSchG)}
{Als Nachweis zur Abrechnung der Frequenznutzungsbeiträge}
\end{QQuestion}

}
\end{frame}

\begin{frame}
\frametitle{Aufbewahrung des Logbuchs}
\begin{itemize}
  \item Bei angeordneter Logbuchführung
  \item Über eine bestimmte Zeit einsehbar
  \item Bei elektronischem Logbuch ist Transfer mit ADIF-Dateiformat möglich
  \item Bei Transfer von Papier auf elektronisch muss das Papierformat weiter aufbewahrt werden
  \end{itemize}
\end{frame}

\begin{frame}
\only<1>{
\begin{QQuestion}{BG102}{Was ist bei der Erstellung eines \glqq Computer-Logbuchs\grqq{} bei angeordneter Logbuchführung zu beachten?}{Die Logdatei muss auch mit einem Textverarbeitungsprogramm gelesen werden können.}
{Die Daten müssen, wie auch beim Papierlogbuch, über eine bestimmte Zeit einsehbar sein.}
{Es muss zusätzlich ein herkömmliches Papierlogbuch geführt werden.}
{Es muss jederzeit ein Ausdruck des Logbuches vorhanden sein.}
\end{QQuestion}

}
\only<2>{
\begin{QQuestion}{BG102}{Was ist bei der Erstellung eines \glqq Computer-Logbuchs\grqq{} bei angeordneter Logbuchführung zu beachten?}{Die Logdatei muss auch mit einem Textverarbeitungsprogramm gelesen werden können.}
{\textbf{\textcolor{DARCgreen}{Die Daten müssen, wie auch beim Papierlogbuch, über eine bestimmte Zeit einsehbar sein.}}}
{Es muss zusätzlich ein herkömmliches Papierlogbuch geführt werden.}
{Es muss jederzeit ein Ausdruck des Logbuches vorhanden sein.}
\end{QQuestion}

}
\end{frame}

\begin{frame}
\only<1>{
\begin{QQuestion}{BG103}{Was ist bei angeordneter Logbuchführung bei einem Wechsel der \glqq Logbuchsoftware\grqq{} zu berücksichtigen?}{Es sollte auf eine Software für ein \qty{64}{\bit}-System gewechselt werden.}
{Die Logbuchdaten müssen verfügbar bleiben, um die Betriebsdaten bei eventuellen späteren Überprüfungen einsehen zu können.}
{Die alte Software muss auf jeden Fall gelöscht werden, um Kollisionen bei den Datenformaten zu vermeiden.}
{Es sollte ein Logbuchprogramm genutzt werden, welches ermöglicht, die Daten in der Cloud zu speichern.}
\end{QQuestion}

}
\only<2>{
\begin{QQuestion}{BG103}{Was ist bei angeordneter Logbuchführung bei einem Wechsel der \glqq Logbuchsoftware\grqq{} zu berücksichtigen?}{Es sollte auf eine Software für ein \qty{64}{\bit}-System gewechselt werden.}
{\textbf{\textcolor{DARCgreen}{Die Logbuchdaten müssen verfügbar bleiben, um die Betriebsdaten bei eventuellen späteren Überprüfungen einsehen zu können.}}}
{Die alte Software muss auf jeden Fall gelöscht werden, um Kollisionen bei den Datenformaten zu vermeiden.}
{Es sollte ein Logbuchprogramm genutzt werden, welches ermöglicht, die Daten in der Cloud zu speichern.}
\end{QQuestion}

}
\end{frame}

\begin{frame}
\frametitle{Koordinierte Weltzeit}
\begin{itemize}
  \item Zeiten in UTC (Universal Time Coordinated) führen
  \item Uhrzeiten über unterschiedliche Zeitzonen müssen nicht umgerechnet werden
  \item Keine Probleme mit Sommer-/Winterzeit
  \item Berlin zu Mitteleuropäischer Zeit (MEZ): UTC+1
  \item Berlin zu Mitteleuropäischer Sommerzeit (MESZ): UTC+2
  \end{itemize}
\end{frame}%ENDCONTENT
