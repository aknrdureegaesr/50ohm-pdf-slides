
\section{Anzeige ortsfester Amateurfunkanlagen}
\label{section:anzeige_ortsfester_amateurfunkanlagen}
\begin{frame}%STARTCONTENT
Für ortsfeste Amateurfunkstellen muss das Nachweisverfahren nur dann durchgeführt werden, wenn die Sendeanlage eine Strahlungsleistung von \qty{10}{\watt} EIRP oder höher erreicht.

\end{frame}

\begin{frame}
\only<1>{
\begin{QQuestion}{VE508}{Wer muss seine Amateurfunkstelle bei der BNetzA gemäß der Verordnung über das Nachweisverfahren zur Begrenzung elektromagnetischer Felder (BEMFV) anzeigen?}{Alle Funkamateure der Zeugnisklasse A}
{Alle Funkamateure, die Portabel- bzw. Mobilbetrieb durchführen}
{Alle Funkamateure, die ortsfeste Amateurfunkstellen mit einer Leistung ab \qty{10}{\W} EIRP betreiben}
{Alle Funkamateure, die auf der Kurzwelle aktiv sind}
\end{QQuestion}

}
\only<2>{
\begin{QQuestion}{VE508}{Wer muss seine Amateurfunkstelle bei der BNetzA gemäß der Verordnung über das Nachweisverfahren zur Begrenzung elektromagnetischer Felder (BEMFV) anzeigen?}{Alle Funkamateure der Zeugnisklasse A}
{Alle Funkamateure, die Portabel- bzw. Mobilbetrieb durchführen}
{\textbf{\textcolor{DARCgreen}{Alle Funkamateure, die ortsfeste Amateurfunkstellen mit einer Leistung ab \qty{10}{\W} EIRP betreiben}}}
{Alle Funkamateure, die auf der Kurzwelle aktiv sind}
\end{QQuestion}

}
\end{frame}

\begin{frame}
\only<1>{
\begin{QQuestion}{VE507}{Für welche Amateurfunkstellen muss der Schutz von Personen in elektromagnetischen Feldern vom Funkamateur dokumentiert werden?}{Für alle ortsfesten Amateurfunkstellen ab einer äquivalenten isotropen Strahlungsleistung von \qty{10}{\W} EIRP}
{Für alle Amateurfunkstellen}
{Für alle ortsfesten Amateurfunkstellen}
{Für alle Amateurfunkstellen ab einer äquivalenten Strahlungsleistung von \qty{10}{\W} EIRP}
\end{QQuestion}

}
\only<2>{
\begin{QQuestion}{VE507}{Für welche Amateurfunkstellen muss der Schutz von Personen in elektromagnetischen Feldern vom Funkamateur dokumentiert werden?}{\textbf{\textcolor{DARCgreen}{Für alle ortsfesten Amateurfunkstellen ab einer äquivalenten isotropen Strahlungsleistung von \qty{10}{\W} EIRP}}}
{Für alle Amateurfunkstellen}
{Für alle ortsfesten Amateurfunkstellen}
{Für alle Amateurfunkstellen ab einer äquivalenten Strahlungsleistung von \qty{10}{\W} EIRP}
\end{QQuestion}

}
\end{frame}

\begin{frame}
\frametitle{Anzeige bei BNetzA}
\begin{itemize}
  \item vor der Aufnahme des Betriebs der ortsfesten Amateurfunkanlage
  \item bei zuständiger Außenstelle der BNetzA
  \end{itemize}
\end{frame}

\begin{frame}
\only<1>{
\begin{QQuestion}{VE509}{Bei welcher Stelle und zu welchem Zeitpunkt ist die Anzeige gemäß der Verordnung über das Nachweisverfahren zur Begrenzung elektromagnetischer Felder (BEMFV) für eine ortsfeste Amateurfunkanlage mit einer EIRP ab \qty{10}{\W} einzureichen?}{Die Anzeige ist spätestens drei Monate nach Betriebsaufnahme bei der zuständigen Außenstelle der BNetzA einzureichen.}
{Die Anzeige ist vor Aufnahme des Betriebs der Amateurfunkanlage bei der zuständigen Außenstelle der BNetzA einzureichen.}
{Wenn die Anzeige den tatsächlichen Gegebenheiten nicht mehr entspricht, ist dieses einer beliebigen Außenstelle der BNetzA mitzuteilen.}
{Die Anzeige ist bei einer beliebigen Außenstelle der BNetzA vor Aufnahme des Betriebs der Amateurfunkanlage einzureichen.}
\end{QQuestion}

}
\only<2>{
\begin{QQuestion}{VE509}{Bei welcher Stelle und zu welchem Zeitpunkt ist die Anzeige gemäß der Verordnung über das Nachweisverfahren zur Begrenzung elektromagnetischer Felder (BEMFV) für eine ortsfeste Amateurfunkanlage mit einer EIRP ab \qty{10}{\W} einzureichen?}{Die Anzeige ist spätestens drei Monate nach Betriebsaufnahme bei der zuständigen Außenstelle der BNetzA einzureichen.}
{\textbf{\textcolor{DARCgreen}{Die Anzeige ist vor Aufnahme des Betriebs der Amateurfunkanlage bei der zuständigen Außenstelle der BNetzA einzureichen.}}}
{Wenn die Anzeige den tatsächlichen Gegebenheiten nicht mehr entspricht, ist dieses einer beliebigen Außenstelle der BNetzA mitzuteilen.}
{Die Anzeige ist bei einer beliebigen Außenstelle der BNetzA vor Aufnahme des Betriebs der Amateurfunkanlage einzureichen.}
\end{QQuestion}

}
\end{frame}

\begin{frame}
\frametitle{Inhalt der Anzeige}
Nachvollziehbare zeichnerische Darstellung mit

\begin{itemize}
  \item Standortbezogener Sicherheitsabstand
  \item Vom Betreiber kontrollierbarer Bereich
  \end{itemize}
\end{frame}

\begin{frame}
\frametitle{Zusätzlich zur Anzeige}
An der Funkstation liegend und auf Verlangen der BNetzA vorzulegen:

\begin{itemize}
  \item Einhaltung der Anforderungen
  \item ggf. Antennendiagramme
  \item Lageplan
  \item Bauzeichnung oder Skizze mit Bemaßung
  \item Konfiguration der Funkanlage
  \end{itemize}
\end{frame}

\begin{frame}
\only<1>{
\begin{QQuestion}{VE512}{Welche Unterlagen sind ergänzend zur Anzeige gemäß der Verordnung über das Nachweisverfahren zur Begrenzung elektromagnetischer Felder (BEMFV) einer ortsfesten Amateurfunkanlage bei der zuständigen Außenstelle der BNetzA einzureichen?}{Es ist ein Blockschaltbild der Amateurfunkstelle beizufügen.}
{Der Anzeige sind Antennendiagramme, Lageplan, Bauzeichnung oder Skizze mit Bemaßung beizufügen.}
{Es sind keine weiteren Unterlagen beizufügen.}
{Der Anzeige ist eine nachvollziehbare zeichnerische Darstellung des standortbezogenen Sicherheitsabstands und des vom Betreiber kontrollierbaren Bereichs beizufügen.}
\end{QQuestion}

}
\only<2>{
\begin{QQuestion}{VE512}{Welche Unterlagen sind ergänzend zur Anzeige gemäß der Verordnung über das Nachweisverfahren zur Begrenzung elektromagnetischer Felder (BEMFV) einer ortsfesten Amateurfunkanlage bei der zuständigen Außenstelle der BNetzA einzureichen?}{Es ist ein Blockschaltbild der Amateurfunkstelle beizufügen.}
{Der Anzeige sind Antennendiagramme, Lageplan, Bauzeichnung oder Skizze mit Bemaßung beizufügen.}
{Es sind keine weiteren Unterlagen beizufügen.}
{\textbf{\textcolor{DARCgreen}{Der Anzeige ist eine nachvollziehbare zeichnerische Darstellung des standortbezogenen Sicherheitsabstands und des vom Betreiber kontrollierbaren Bereichs beizufügen.}}}
\end{QQuestion}

}
\end{frame}

\begin{frame}
\only<1>{
\begin{QQuestion}{VE513}{Welche Unterlagen hat der Funkamateur ergänzend zur Anzeige einer ortsfesten Amateurfunkanlage gemäß der Verordnung über das Nachweisverfahren zur Begrenzung elektromagnetischer Felder (BEMFV) ab dem Zeitpunkt der Inbetriebnahme bereitzuhalten und der Bundesnetzagentur nach Aufforderung vorzulegen?}{Eine nachvollziehbare Dokumentation über die Einhaltung der Anforderungen, gegebenenfalls Antennendiagramme, einen Lageplan, eine Bauzeichnung oder Skizze mit Bemaßung und die Konfiguration der Funkanlage}
{Das Anzeigeformblatt mit den Daten der ortsfesten Amateurfunkanlage und eine maßstäbliche Skizze des standortbezogenen Sicherheitsabstandes und des kontrollierbaren Bereiches}
{Die Zulassung zur Teilnahme am Amateurfunkdienst, die Datenblätter aller Amateurfunkgeräte und das Logbuch, denn sie müssen jederzeit für eine mögliche Kontrolle durch die Bundesnetzagentur verfügbar sein}
{Eine Fotodokumentation der Amateurfunkanlage einschließlich der Antennen sowie eine formlose Aufstellung aller Messwerte nebst Antennendiagrammen}
\end{QQuestion}

}
\only<2>{
\begin{QQuestion}{VE513}{Welche Unterlagen hat der Funkamateur ergänzend zur Anzeige einer ortsfesten Amateurfunkanlage gemäß der Verordnung über das Nachweisverfahren zur Begrenzung elektromagnetischer Felder (BEMFV) ab dem Zeitpunkt der Inbetriebnahme bereitzuhalten und der Bundesnetzagentur nach Aufforderung vorzulegen?}{\textbf{\textcolor{DARCgreen}{Eine nachvollziehbare Dokumentation über die Einhaltung der Anforderungen, gegebenenfalls Antennendiagramme, einen Lageplan, eine Bauzeichnung oder Skizze mit Bemaßung und die Konfiguration der Funkanlage}}}
{Das Anzeigeformblatt mit den Daten der ortsfesten Amateurfunkanlage und eine maßstäbliche Skizze des standortbezogenen Sicherheitsabstandes und des kontrollierbaren Bereiches}
{Die Zulassung zur Teilnahme am Amateurfunkdienst, die Datenblätter aller Amateurfunkgeräte und das Logbuch, denn sie müssen jederzeit für eine mögliche Kontrolle durch die Bundesnetzagentur verfügbar sein}
{Eine Fotodokumentation der Amateurfunkanlage einschließlich der Antennen sowie eine formlose Aufstellung aller Messwerte nebst Antennendiagrammen}
\end{QQuestion}

}
\end{frame}

\begin{frame}
\only<1>{
\begin{QQuestion}{VD107}{In welchem Fall hat ein Funkamateur der Bundesnetzagentur gemäß Amateurfunkverordnung (AFuV) technische Unterlagen über seine Sendeanlage vorzulegen?}{Auf Anforderung der Bundesnetzagentur}
{Bei jeder technischen Änderung an der Sendeanlage}
{Unverzüglich nach Erhalt der Amateurfunkzulassung}
{Bei Sendeleistungen größer als \qty{750}{\W}}
\end{QQuestion}

}
\only<2>{
\begin{QQuestion}{VD107}{In welchem Fall hat ein Funkamateur der Bundesnetzagentur gemäß Amateurfunkverordnung (AFuV) technische Unterlagen über seine Sendeanlage vorzulegen?}{\textbf{\textcolor{DARCgreen}{Auf Anforderung der Bundesnetzagentur}}}
{Bei jeder technischen Änderung an der Sendeanlage}
{Unverzüglich nach Erhalt der Amateurfunkzulassung}
{Bei Sendeleistungen größer als \qty{750}{\W}}
\end{QQuestion}

}
\end{frame}

\begin{frame}
\frametitle{Änderungen}
\begin{itemize}
  \item Fortlaufend prüfen, ob die Anlage gleich zu der in der Anzeige ist
  \item Bei wesentlichen Änderungen erneute Anzeige durchführen
  \end{itemize}
\end{frame}

\begin{frame}
\only<1>{
\begin{QQuestion}{VE514}{Was hat ein Funkamateur zu beachten, nachdem er seine ortsfeste Amateurfunkstelle bei der Bundesnetzagentur gemäß BEMFV angezeigt hat?}{Nachdem die ortsfeste Amateurfunkstelle in Betrieb genommen wurde, ist die Dokumentation über die Einhaltung der Anforderungen mit allen erforderlichen Unterlagen der zuständigen Außenstelle der Bundesnetzagentur vorzulegen.}
{Mit der Anzeige seiner ortsfesten Amateurfunkstelle ist ein Funkamateur seinen Verpflichtungen zum Schutz von Personen in elektromagnetischen Feldern nachgekommen und muss diesbezüglich nichts weiter beachten.}
{Das Anzeigeverfahren ist jedes Jahr erneut durchzuführen, um die Aktualität zu gewährleisten.}
{Er hat eine Dokumentation über die Einhaltung der Anforderungen mit allen erforderlichen Unterlagen bereitzuhalten und fortlaufend zu prüfen, ob die Bedingungen, unter denen die Anzeige durchgeführt wurde, noch zutreffend sind. Bei wesentlichen Änderungen ist die Amateurfunkstelle erneut anzuzeigen.}
\end{QQuestion}

}
\only<2>{
\begin{QQuestion}{VE514}{Was hat ein Funkamateur zu beachten, nachdem er seine ortsfeste Amateurfunkstelle bei der Bundesnetzagentur gemäß BEMFV angezeigt hat?}{Nachdem die ortsfeste Amateurfunkstelle in Betrieb genommen wurde, ist die Dokumentation über die Einhaltung der Anforderungen mit allen erforderlichen Unterlagen der zuständigen Außenstelle der Bundesnetzagentur vorzulegen.}
{Mit der Anzeige seiner ortsfesten Amateurfunkstelle ist ein Funkamateur seinen Verpflichtungen zum Schutz von Personen in elektromagnetischen Feldern nachgekommen und muss diesbezüglich nichts weiter beachten.}
{Das Anzeigeverfahren ist jedes Jahr erneut durchzuführen, um die Aktualität zu gewährleisten.}
{\textbf{\textcolor{DARCgreen}{Er hat eine Dokumentation über die Einhaltung der Anforderungen mit allen erforderlichen Unterlagen bereitzuhalten und fortlaufend zu prüfen, ob die Bedingungen, unter denen die Anzeige durchgeführt wurde, noch zutreffend sind. Bei wesentlichen Änderungen ist die Amateurfunkstelle erneut anzuzeigen.}}}
\end{QQuestion}

}
\end{frame}

\begin{frame}
\only<1>{
\begin{QQuestion}{VE510}{Wann ist erneut eine Anzeige einer ortsfesten Amateurfunkanlage gemäß der Verordnung über das Nachweisverfahren zur Begrenzung elektromagnetischer Felder (BEMFV) bei der zuständigen Stelle der BNetzA einzureichen?}{Die Anzeige ist jährlich zu aktualisieren. Wurden keine Änderungen an der Amateurfunkanlage vorgenommen, reicht eine formlose Mitteilung.}
{Wenn die bestehende Anzeige nicht mehr den tatsächlichen Gegebenheiten entspricht, ist vom Betreiber das Anzeigeverfahren erneut durchzuführen.}
{Bei einem Wechsel der nationalen Zeugnisklasse}
{Nach Aufforderung der zuständigen Stelle der BNetzA}
\end{QQuestion}

}
\only<2>{
\begin{QQuestion}{VE510}{Wann ist erneut eine Anzeige einer ortsfesten Amateurfunkanlage gemäß der Verordnung über das Nachweisverfahren zur Begrenzung elektromagnetischer Felder (BEMFV) bei der zuständigen Stelle der BNetzA einzureichen?}{Die Anzeige ist jährlich zu aktualisieren. Wurden keine Änderungen an der Amateurfunkanlage vorgenommen, reicht eine formlose Mitteilung.}
{\textbf{\textcolor{DARCgreen}{Wenn die bestehende Anzeige nicht mehr den tatsächlichen Gegebenheiten entspricht, ist vom Betreiber das Anzeigeverfahren erneut durchzuführen.}}}
{Bei einem Wechsel der nationalen Zeugnisklasse}
{Nach Aufforderung der zuständigen Stelle der BNetzA}
\end{QQuestion}

}
\end{frame}

\begin{frame}
\frametitle{Nachweisverfahren}
\begin{columns}
    \begin{column}{0.48\textwidth}
    \begin{itemize}
  \item Berechnung des Personen-Sicherheitsabstands
  \item Während des Sendebetriebs dürfen keine unbefugten Personen in diesem Bereich sein
  \item Ist erfüllt, wenn dieses im kontrollierbaren Bereich stattfindet, z.B. eigenes Grundstück
  \end{itemize}

    \end{column}
   \begin{column}{0.48\textwidth}
       Hilfsmittel:

\begin{itemize}
  \item Software \enquote{Watt Wächter (\textcolor{DARCblue}{\faLink~\href{https://50ohm.de/ww}{50ohm.de/ww}})}
  \item vereinfachtes Bewertungsverfahren
  \item Feldstärkemessung
  \item Fernfeldberechnung
  \item Nahfeldberechnung
  \end{itemize}

   \end{column}
\end{columns}

\end{frame}

\begin{frame}
\only<1>{
\begin{QQuestion}{VE506}{Was muss ein Funkamateur zum Schutz von Personen bei dem Betrieb von ortsfesten Amateurfunkanlagen gemäß der Verordnung über das Nachweisverfahren zur Begrenzung elektromagnetischer Felder (BEMFV) vornehmen?}{Er kann bei einer Leistung von bis zu \qty{100}{\W} PEP den standardisierten Sicherheitsabstand von \qty{25}{\m} annehmen.}
{Er kann bei einer Leistung von bis zu \qty{100}{\W} PEP den standardisierten Sicherheitsabstand von \qty{10}{\m} annehmen.}
{Er hat den zur Einhaltung der Grenzwerte erforderlichen Sicherheitsabstand einer Funkanlage mit EIRP von 10 W oder mehr rechnerisch oder messtechnisch zu ermitteln und in nachvollziehbarer Form zu dokumentieren.}
{Er hat den zur Einhaltung der Grenzwerte erforderlichen Sicherheitsabstand durch ein zertifiziertes Messlabor ermitteln zu lassen.}
\end{QQuestion}

}
\only<2>{
\begin{QQuestion}{VE506}{Was muss ein Funkamateur zum Schutz von Personen bei dem Betrieb von ortsfesten Amateurfunkanlagen gemäß der Verordnung über das Nachweisverfahren zur Begrenzung elektromagnetischer Felder (BEMFV) vornehmen?}{Er kann bei einer Leistung von bis zu \qty{100}{\W} PEP den standardisierten Sicherheitsabstand von \qty{25}{\m} annehmen.}
{Er kann bei einer Leistung von bis zu \qty{100}{\W} PEP den standardisierten Sicherheitsabstand von \qty{10}{\m} annehmen.}
{\textbf{\textcolor{DARCgreen}{Er hat den zur Einhaltung der Grenzwerte erforderlichen Sicherheitsabstand einer Funkanlage mit EIRP von 10 W oder mehr rechnerisch oder messtechnisch zu ermitteln und in nachvollziehbarer Form zu dokumentieren.}}}
{Er hat den zur Einhaltung der Grenzwerte erforderlichen Sicherheitsabstand durch ein zertifiziertes Messlabor ermitteln zu lassen.}
\end{QQuestion}

}
\end{frame}

\begin{frame}
\only<1>{
\begin{QQuestion}{VE515}{Welche Verfahren können Funkamateure nutzen, um den Nachweis zur Begrenzung von elektromagnetischen Feldern zu erstellen?}{Funkamateure sind ausdrücklich vom Nachweis zur Begrenzung von elektromagnetischen Feldern ausgenommen.}
{Funkamateure müssen eine zertifizierte Firma mit dem Nachweis zur Begrenzung von elektromagnetischen Feldern beauftragen.}
{Funkamateure können aufgrund ihrer Fachkenntnisse die Einhaltung der elektromagnetische Grenzwerte abschätzen.}
{Das Bewertungsverfahren mit der Anwendung \glqq Watt Wächter\grqq{}, das vereinfachte Bewertungsverfahren, Feldstärkemessung, Fernfeldberechnung und Nahfeldberechnung}
\end{QQuestion}

}
\only<2>{
\begin{QQuestion}{VE515}{Welche Verfahren können Funkamateure nutzen, um den Nachweis zur Begrenzung von elektromagnetischen Feldern zu erstellen?}{Funkamateure sind ausdrücklich vom Nachweis zur Begrenzung von elektromagnetischen Feldern ausgenommen.}
{Funkamateure müssen eine zertifizierte Firma mit dem Nachweis zur Begrenzung von elektromagnetischen Feldern beauftragen.}
{Funkamateure können aufgrund ihrer Fachkenntnisse die Einhaltung der elektromagnetische Grenzwerte abschätzen.}
{\textbf{\textcolor{DARCgreen}{Das Bewertungsverfahren mit der Anwendung \glqq Watt Wächter\grqq{}, das vereinfachte Bewertungsverfahren, Feldstärkemessung, Fernfeldberechnung und Nahfeldberechnung}}}
\end{QQuestion}

}
\end{frame}

\begin{frame}
\frametitle{Mehrere Aussendungen gleichzeitig}
\begin{itemize}
  \item Es können mehrere Funkamateure gleichzeitig an einer Anlage auf verschiedenen Frequenzen senden
  \item In der Regel über verschiedene Antennen
  \item Alle Antennen zusammen müssen für den Personenschutzabstand berücksichtigt werden
  \end{itemize}

\end{frame}

\begin{frame}
\only<1>{
\begin{QQuestion}{VE516}{Welche Aussendungen von Amateurfunkanlagen müssen bei der Ermittlung des standortbezogenen Sicherheitsabstandes berücksichtigt werden?}{Alle Aussendungen der ortsfesten Amateurfunkstelle, die ein Funkamateur zeitgleich durchzuführen beabsichtigt}
{Ausschließlich Aussendungen von ortsfest betriebenen Amateurfunkstellen mit einer Strahlungsleistung (EIRP) größer \qty{10}{\W}}
{Nur die Aussendungen der maximalen Sendeleistung, die die Amateurfunkanlage erbringen kann}
{Alle Aussendungen mit einer Strahlungsleistung (EIRP) größer \qty{10}{\W}, auch Aussendungen im Mobilbetrieb.}
\end{QQuestion}

}
\only<2>{
\begin{QQuestion}{VE516}{Welche Aussendungen von Amateurfunkanlagen müssen bei der Ermittlung des standortbezogenen Sicherheitsabstandes berücksichtigt werden?}{\textbf{\textcolor{DARCgreen}{Alle Aussendungen der ortsfesten Amateurfunkstelle, die ein Funkamateur zeitgleich durchzuführen beabsichtigt}}}
{Ausschließlich Aussendungen von ortsfest betriebenen Amateurfunkstellen mit einer Strahlungsleistung (EIRP) größer \qty{10}{\W}}
{Nur die Aussendungen der maximalen Sendeleistung, die die Amateurfunkanlage erbringen kann}
{Alle Aussendungen mit einer Strahlungsleistung (EIRP) größer \qty{10}{\W}, auch Aussendungen im Mobilbetrieb.}
\end{QQuestion}

}
\end{frame}

\begin{frame}
\only<1>{
\begin{QQuestion}{VE517}{Sie wollen eine Amateurfunkstelle mit mehreren Sendeantennen betreiben und die Personenschutz-Sicherheitsabstände ermitteln. Dabei ergibt sich, dass der Sicherheitsabstand mehrerer Antennen überlappt. Was müssen Sie nun beachten?}{Für die gesamte Antennenanlage gilt der Sicherheitsabstand der Antenne mit der größten Strahlungsleistung.}
{Die betroffenen Antennen sind gemeinsam zu betrachten, sofern mit ihnen gleichzeitig gesendet werden soll.}
{Es ist sicherzustellen, dass der Sendebetrieb zu jedem Zeitpunkt auf eine der Antennen beschränkt wird.}
{Die Sicherheitsabstände sind mit der Anzahl der Sendeantennen als Sicherheitsfaktor zu multiplizieren.}
\end{QQuestion}

}
\only<2>{
\begin{QQuestion}{VE517}{Sie wollen eine Amateurfunkstelle mit mehreren Sendeantennen betreiben und die Personenschutz-Sicherheitsabstände ermitteln. Dabei ergibt sich, dass der Sicherheitsabstand mehrerer Antennen überlappt. Was müssen Sie nun beachten?}{Für die gesamte Antennenanlage gilt der Sicherheitsabstand der Antenne mit der größten Strahlungsleistung.}
{\textbf{\textcolor{DARCgreen}{Die betroffenen Antennen sind gemeinsam zu betrachten, sofern mit ihnen gleichzeitig gesendet werden soll.}}}
{Es ist sicherzustellen, dass der Sendebetrieb zu jedem Zeitpunkt auf eine der Antennen beschränkt wird.}
{Die Sicherheitsabstände sind mit der Anzahl der Sendeantennen als Sicherheitsfaktor zu multiplizieren.}
\end{QQuestion}

}
\end{frame}%ENDCONTENT
