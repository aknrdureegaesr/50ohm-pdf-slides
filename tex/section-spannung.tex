
\section{Elektrische Spannung}
\label{section:spannung}
\begin{frame}%STARTCONTENT
Wiederholung vom Anfang des Kurses:

\begin{itemize}
  \item Nach Trennung von positiven und negativen Ladungen versuchen diese wieder zusammenzukommen
  \item Es liegt eine elektrische Spannung vor
  \item Die Einheit ist Volt, abgekürzt V
  \end{itemize}
\end{frame}

\begin{frame}
\only<1>{
\begin{QQuestion}{NA201}{Welche Einheit wird üblicherweise für die elektrische Spannung verwendet?}{Ohm ($\Omega$)}
{Ampere (A)}
{Volt (V)}
{Amperestunden (Ah)}
\end{QQuestion}

}
\only<2>{
\begin{QQuestion}{NA201}{Welche Einheit wird üblicherweise für die elektrische Spannung verwendet?}{Ohm ($\Omega$)}
{Ampere (A)}
{\textbf{\textcolor{DARCgreen}{Volt (V)}}}
{Amperestunden (Ah)}
\end{QQuestion}

}
\end{frame}

\begin{frame}
\frametitle{Kleine Spannungen}
\begin{columns}
    \begin{column}{0.48\textwidth}
    \begin{itemize}
  \item Empfängereingang: 10 µV
  \item Mikrofon: 200 mV
  \item Batterie: 1,5 V oder 9 V
  \end{itemize}

    \end{column}
   \begin{column}{0.48\textwidth}
       \begin{table}
\begin{DARCtabular}{lrl}
     Bezeichnung  & Abkürzung  & Wert   \\
     1 Mikrovolt  & \qty{1}{\micro\volt}  & \qty{0,000001}{\volt}   \\
     1 Millivolt  & \qty{1}{\milli\volt}  & \qty{0,001}{\volt}   \\
     1 Volt  & \qty{1}{\volt}  & \qty{1}{\volt}   \\
\end{DARCtabular}
\caption{Kurzschreibweisen für kleine Spannungen}
\label{spannung_einheitenvorzeichen}
\end{table}

   \end{column}
\end{columns}

\end{frame}

\begin{frame}
\frametitle{Große Spannungen}
\begin{columns}
    \begin{column}{0.48\textwidth}
    \begin{itemize}
  \item Steckdose: 230 V
  \item Elektrostatisch aufgeladene Antenne: 1,5 kV
  \item Höchstspannungsleitung: 380 kV
  \end{itemize}

    \end{column}
   \begin{column}{0.48\textwidth}
       \begin{table}
\begin{DARCtabular}{lrr}
     Bezeichnung  & Abk.  & Wert   \\
     1 Kilovolt  & \qty{1}{\kilo\volt}  & \qty{1000}{\volt}   \\
     1 Megavolt  & \qty{1}{\mega\volt}  & \qty{1000000}{\volt}   \\
     1 Gigavolt  & \qty{1}{\giga\volt}  & \qty{1000000000}{\volt}   \\
\end{DARCtabular}
\caption{Kurzschreibweise für große Spannungen}
\label{n_frequenz_einheitenvorzeichen}
\end{table}

   \end{column}
\end{columns}

\end{frame}

\begin{frame}
\only<1>{
\begin{QQuestion}{NA208}{\qty{4,2}{\V} entspricht ...}{\qty{4,200}{\micro\volt}}
{\qty{4200}{\mV}}
{\qty{4200}{\kV}}
{\qty{4200}{\mega\V}}
\end{QQuestion}

}
\only<2>{
\begin{QQuestion}{NA208}{\qty{4,2}{\V} entspricht ...}{\qty{4,200}{\micro\volt}}
{\textbf{\textcolor{DARCgreen}{\qty{4200}{\mV}}}}
{\qty{4200}{\kV}}
{\qty{4200}{\mega\V}}
\end{QQuestion}

}
\end{frame}%ENDCONTENT
