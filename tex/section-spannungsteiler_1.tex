
\section{Spannungsteiler I}
\label{section:spannungsteiler_1}
\begin{frame}%STARTCONTENT

\begin{columns}
    \begin{column}{0.48\textwidth}
    \begin{itemize}
  \item Eine Reihenschaltung von Widerständen nennt man auch Spannungsteiler, weil die Spannungen sich an den Widerständen aufteilen.
  \item Je größer der Widerstand, desto größer die Spannung, die an ihm abfällt.
  \end{itemize}

    \end{column}
   \begin{column}{0.48\textwidth}
       
\begin{figure}
    \DARCimage{0.85\linewidth}{819include}
    \caption{\scriptsize Spannungsteiler}
    \label{e_spannungsteiler}
\end{figure}


   \end{column}
\end{columns}

\end{frame}

\begin{frame}
\begin{columns}
    \begin{column}{0.48\textwidth}
    \begin{itemize}
  \item Das kann man mathematisch in folgender Formel ausdrücken (Formelsammlung):
  \end{itemize}
$\dfrac{ U_{ 1 } }{ U_{ 2 } } = \dfrac{ R_{ 1 } }{ R_{ 2 } }$


    \end{column}
   \begin{column}{0.48\textwidth}
       
\begin{figure}
    \DARCimage{0.85\linewidth}{819include}
    \caption{\scriptsize Spannungsteiler}
    \label{e_spannungsteiler}
\end{figure}


   \end{column}
\end{columns}

\end{frame}

\begin{frame}Wie geht man an die Aufgaben ran?

\begin{itemize}
  \item Beispiele:
  \item Wenn $R_{ 1 }$ drei mal so groß wie $R_{ 2 }$ ist, ist $U_{ 1 }$ drei mal so groß wie $U_{ 2 }$.
  \item Wenn $R_{ 1 }$ $\frac{ 1 }{ 3 }$ so groß wie $R_{ 2 }$ ist, ist $U_{ 1 }$ $\frac{ 1 }{ 3 }$ so groß wie $U_{ 2 }$.
  \end{itemize}
Schauen wir uns dazu zwei Aufgaben an.

\end{frame}

\begin{frame}
\only<1>{
\begin{PQuestion}{ED101}{Wie teilt sich die Spannung an zwei in Reihe geschalteten Widerständen auf, wenn $R_1$~=~5-mal so groß ist wie $R_2$?}{$U_1 = 5\cdot U_2$}
{$U_1 = \frac{U_2}{5}$}
{$U_1 = 6\cdot U_2$}
{$U_1 = \frac{U_2}{6}$}
{\DARCimage{0.5\linewidth}{397include}}\end{PQuestion}

}
\only<2>{
\begin{PQuestion}{ED101}{Wie teilt sich die Spannung an zwei in Reihe geschalteten Widerständen auf, wenn $R_1$~=~5-mal so groß ist wie $R_2$?}{\textbf{\textcolor{DARCgreen}{$U_1 = 5\cdot U_2$}}}
{$U_1 = \frac{U_2}{5}$}
{$U_1 = 6\cdot U_2$}
{$U_1 = \frac{U_2}{6}$}
{\DARCimage{0.5\linewidth}{397include}}\end{PQuestion}

}
\end{frame}

\begin{frame}
\only<1>{
\begin{PQuestion}{ED102}{Wie teilt sich die Spannung an zwei in Reihe geschalteten Widerständen auf, wenn $R_1~=~\frac{1}{6}$ von $R_2$ ist? }{$U_1 = 6\cdot U_2$}
{$U_1 = \frac{U_2}{6}$}
{$U_1 = \frac{U_2}{5}$}
{$U_1 = 5\cdot U_2$}
{\DARCimage{0.5\linewidth}{397include}}\end{PQuestion}

}
\only<2>{
\begin{PQuestion}{ED102}{Wie teilt sich die Spannung an zwei in Reihe geschalteten Widerständen auf, wenn $R_1~=~\frac{1}{6}$ von $R_2$ ist? }{$U_1 = 6\cdot U_2$}
{\textbf{\textcolor{DARCgreen}{$U_1 = \frac{U_2}{6}$}}}
{$U_1 = \frac{U_2}{5}$}
{$U_1 = 5\cdot U_2$}
{\DARCimage{0.5\linewidth}{397include}}\end{PQuestion}

}
\end{frame}

\begin{frame}
\begin{columns}
    \begin{column}{0.48\textwidth}
    \begin{itemize}
  \item Die Summe der Spannnungsabfälle ist gleich der Spannung, die aus der Spannungsquelle herauskommt.
  \item Das kann man mathematisch in folgender Formel ausdrücken (Formelsammlung):
  \end{itemize}
$U_{ G } = U_{ 1 } + U_{ 2 }$


    \end{column}
   \begin{column}{0.48\textwidth}
       
\begin{figure}
    \DARCimage{0.85\linewidth}{819include}
    \caption{\scriptsize Spannungsteiler}
    \label{E 63. Spannungsteiler}
\end{figure}


   \end{column}
\end{columns}

\end{frame}

\begin{frame}
\begin{columns}
    \begin{column}{0.48\textwidth}
    \begin{itemize}
  \item Hat man eine Gesamtspannung und muss $U_{ 2 }$ berechnen, können wir ebenfalls auf eine Formel aus der Formelsammlung zurückgreifen:
  \end{itemize}
$\dfrac{ U_{ 2 } }{ U_{ G } } = \dfrac{ R_{ 2 } }{ R_{ 1 } + R_{ 2 } }$


    \end{column}
   \begin{column}{0.48\textwidth}
       
\begin{figure}
    \DARCimage{0.85\linewidth}{819include}
    \caption{\scriptsize Spannungsteiler}
    \label{E 63. Spannungsteiler}
\end{figure}


   \end{column}
\end{columns}

\end{frame}

\begin{frame}
\begin{columns}
    \begin{column}{0.48\textwidth}
    \begin{itemize}
  \item Diese muss man noch zu $U_{ 2 }$ umstellen, indem man auf beiden Seiten mit $U_{ G }$ multipliziert, dann erhält man:
  \end{itemize}
$U_{ 2 } = \dfrac{ R_{ 2 } }{ R_{ 1 } + R_{ 2 } } \cdot U_{ G }$

\begin{itemize}
  \item Damit kann man sich dann auch an die nächste Aufgabe heranwagen.
  \end{itemize}

    \end{column}
   \begin{column}{0.48\textwidth}
       
\begin{figure}
    \DARCimage{0.85\linewidth}{819include}
    \caption{\scriptsize Spannungsteiler}
    \label{E 63. Spannungsteiler}
\end{figure}


   \end{column}
\end{columns}

\end{frame}

\begin{frame}
\only<1>{
\begin{PQuestion}{ED103}{Die Gesamtspannung $U$ an folgendem Spannungsteiler beträgt \qty{9}{\V}. Die Widerstände haben die Werte $R_1$~=~\qty{10}{\kohm} und $R_2$~=~\qty{20}{\kohm}. Wie groß ist die Teilspannung $U_2$?}{\qty{4,5}{\V}}
{\qty{3,0}{\V}}
{\qty{6,0}{\V}}
{\qty{7,5}{\V}}
{\DARCimage{0.5\linewidth}{458include}}\end{PQuestion}

}
\only<2>{
\begin{PQuestion}{ED103}{Die Gesamtspannung $U$ an folgendem Spannungsteiler beträgt \qty{9}{\V}. Die Widerstände haben die Werte $R_1$~=~\qty{10}{\kohm} und $R_2$~=~\qty{20}{\kohm}. Wie groß ist die Teilspannung $U_2$?}{\qty{4,5}{\V}}
{\qty{3,0}{\V}}
{\textbf{\textcolor{DARCgreen}{\qty{6,0}{\V}}}}
{\qty{7,5}{\V}}
{\DARCimage{0.5\linewidth}{458include}}\end{PQuestion}

}
\end{frame}%ENDCONTENT
