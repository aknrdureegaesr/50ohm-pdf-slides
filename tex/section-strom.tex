
\section{Elektrischer Strom}
\label{section:strom}
\begin{frame}%STARTCONTENT

\frametitle{Stromkreis}
\begin{itemize}
  \item Beim Anschluss eines elektrischen Verbrauchers an die Pole einer Spannungsquelle, fangen die Ladungen an sich zu bewegen
  \item Das ist ein \emph{geschlossener Stromkreis}
  \item Je nach Spannung und Verbraucher fließt mehr oder weniger Strom
  \item Die \emph{elektrische Stromstärke} wird in Ampere (A) gemessen
  \end{itemize}
\end{frame}

\begin{frame}
\only<1>{
\begin{QQuestion}{NA202}{Welche Einheit wird üblicherweise für die elektrische Stromstärke verwendet?}{Amperestunden (Ah)}
{Volt (V)}
{Ohm ($\Omega$)}
{Ampere (A)}
\end{QQuestion}

}
\only<2>{
\begin{QQuestion}{NA202}{Welche Einheit wird üblicherweise für die elektrische Stromstärke verwendet?}{Amperestunden (Ah)}
{Volt (V)}
{Ohm ($\Omega$)}
{\textbf{\textcolor{DARCgreen}{Ampere (A)}}}
\end{QQuestion}

}
\end{frame}

\begin{frame}
\frametitle{Beispiele für Stromstärke}
\begin{table}
\begin{DARCtabular}{lrcl}
     Verbraucher  & & &  \\
     Leuchtdiode (LED)  & \qty{5}{\milli\ampere}  & =  & \qty{0,005}{\ampere}   \\
     Transceiver im Empfangsbetrieb  & \qty{900}{\milli\ampere}  & =  & \qty{0,9}{\ampere}   \\
     Transceiver im Sendebetrieb  & \qty{21}{\ampere}  & =  & \qty{21}{\ampere}   \\
\end{DARCtabular}
\caption{Beispiele für Ströme}
\label{strom_beispiele}
\end{table}
\end{frame}

\begin{frame}
\only<1>{
\begin{QQuestion}{NA209}{\qty{42}{\mA} entspricht~...}{\qty{0,42}{\A}.}
{\qty{0,0042}{\A}.}
{\qty{4,2}{\A}.}
{\qty{0,042}{\A}.}
\end{QQuestion}

}
\only<2>{
\begin{QQuestion}{NA209}{\qty{42}{\mA} entspricht~...}{\qty{0,42}{\A}.}
{\qty{0,0042}{\A}.}
{\qty{4,2}{\A}.}
{\textbf{\textcolor{DARCgreen}{\qty{0,042}{\A}.}}}
\end{QQuestion}

}
\end{frame}%ENDCONTENT
