
\section{Wellenausbreitung}
\label{section:wellenausbreitung}
\begin{frame}%STARTCONTENT

\begin{columns}
    \begin{column}{0.48\textwidth}
    Je nach Frequenz breitet sich eine Funkwelle anders über unseren Planeten aus.


\begin{figure}
    \DARCimage{0.85\linewidth}{731include}
    \caption{\scriptsize Ionosphäre, Troposhäre und Sporadic-E}
    \label{n_ionosphäre}
\end{figure}


    \end{column}
   \begin{column}{0.48\textwidth}
       \begin{itemize}
  \item Der Funkhorizont, der etwas weiter geht als der sichtbare Horizont (VHF, UHF und höher)
  \item Überreichweiten durch Wetterereignisse in der Troposphäre (VHF, UHF und höher)
  \item Besondere Überreichweiten durch Sporadic-E (VHF, UHF)
  \item Die Raumwelle durch Brechung an der Ionosphäre (Kurzwelle)
  \end{itemize}

   \end{column}
\end{columns}

\end{frame}%ENDCONTENT
