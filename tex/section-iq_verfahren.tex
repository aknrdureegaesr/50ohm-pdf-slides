
\section{I/Q-Verfahren}
\label{section:iq_verfahren}
\begin{frame}%STARTCONTENT

\only<1>{
\begin{QQuestion}{AE404}{Wie wird Quadraturamplitudenmodulation (QAM) üblicherweise erzeugt? Durch~...}{separate Änderung der Amplitude des elektrischen und magnetischen Feldwellenanteils}
{nichtlineare Änderung der Amplitude (Quadratfunktion bzw. Quadratwurzel)}
{Änderung der Amplituden und Addition zweier um \qty{90}{\degree} phasenverschobener Träger}
{richtungsabhängige Änderung der Frequenz (bzw. richtungsinvariante Änderung der Amplitude)}
\end{QQuestion}

}
\only<2>{
\begin{QQuestion}{AE404}{Wie wird Quadraturamplitudenmodulation (QAM) üblicherweise erzeugt? Durch~...}{separate Änderung der Amplitude des elektrischen und magnetischen Feldwellenanteils}
{nichtlineare Änderung der Amplitude (Quadratfunktion bzw. Quadratwurzel)}
{\textbf{\textcolor{DARCgreen}{Änderung der Amplituden und Addition zweier um \qty{90}{\degree} phasenverschobener Träger}}}
{richtungsabhängige Änderung der Frequenz (bzw. richtungsinvariante Änderung der Amplitude)}
\end{QQuestion}

}
\end{frame}

\begin{frame}
\only<1>{
\begin{PQuestion}{AF632}{Wie groß muss die Phasenverschiebung $\varphi$ in der dargestellten Modulatorschaltung sein, damit eine korrekte Quadraturmodulation vorliegt?}{\qty{180}{\degree}}
{\qty{90}{\degree}}
{\qty{0}{\degree}}
{\qty{45}{\degree}}
{\DARCimage{1.0\linewidth}{196include}}\end{PQuestion}

}
\only<2>{
\begin{PQuestion}{AF632}{Wie groß muss die Phasenverschiebung $\varphi$ in der dargestellten Modulatorschaltung sein, damit eine korrekte Quadraturmodulation vorliegt?}{\qty{180}{\degree}}
{\textbf{\textcolor{DARCgreen}{\qty{90}{\degree}}}}
{\qty{0}{\degree}}
{\qty{45}{\degree}}
{\DARCimage{1.0\linewidth}{196include}}\end{PQuestion}

}
\end{frame}

\begin{frame}
\only<1>{
\begin{QQuestion}{AF633}{Was bildet der I- bzw. der Q-Anteil eines I/Q-Signals ab?}{Den Wechselstrom (I) in Abhängigkeit der Güte (Q) eines Schwingkreises bei seiner Resonanzfrequenz}
{Die phasengleichen (I) bzw. die um \qty{90}{\degree} phasenverschobenen (Q) Anteile eines Signals in Bezug auf eine Referenzschwingung}
{Den Stromanteil (I) und den Blindleistungsanteil (Q) eines Signals}
{Die erste (I) bzw. die vierte (Q) Harmonische in Bezug auf ein normiertes Rechtecksignal}
\end{QQuestion}

}
\only<2>{
\begin{QQuestion}{AF633}{Was bildet der I- bzw. der Q-Anteil eines I/Q-Signals ab?}{Den Wechselstrom (I) in Abhängigkeit der Güte (Q) eines Schwingkreises bei seiner Resonanzfrequenz}
{\textbf{\textcolor{DARCgreen}{Die phasengleichen (I) bzw. die um \qty{90}{\degree} phasenverschobenen (Q) Anteile eines Signals in Bezug auf eine Referenzschwingung}}}
{Den Stromanteil (I) und den Blindleistungsanteil (Q) eines Signals}
{Die erste (I) bzw. die vierte (Q) Harmonische in Bezug auf ein normiertes Rechtecksignal}
\end{QQuestion}

}
\end{frame}

\begin{frame}
\only<1>{
\begin{QQuestion}{AF634}{Welchen Frequenzbereich (z.~B. in Bezug auf eine Mitten- oder Trägerfrequenz) kann ein digitaler Datenstrom entsprechend dem Abtasttheorem maximal eindeutig abbilden, der aus einem I- und einem Q-Anteil mit einer Abtastrate von \underline{jeweils} 48000 Samples pro Sekunde besteht? Den Bereich zwischen~...}{\qty{-24}{\kHz} und \qty{+24}{\kHz}.}
{\qty{-48}{\kHz} und \qty{+48}{\kHz}.}
{\qty{0}{\Hz} und \qty{96}{\kHz}.}
{\qty{0}{\Hz} und \qty{6}{\kHz}.}
\end{QQuestion}

}
\only<2>{
\begin{QQuestion}{AF634}{Welchen Frequenzbereich (z.~B. in Bezug auf eine Mitten- oder Trägerfrequenz) kann ein digitaler Datenstrom entsprechend dem Abtasttheorem maximal eindeutig abbilden, der aus einem I- und einem Q-Anteil mit einer Abtastrate von \underline{jeweils} 48000 Samples pro Sekunde besteht? Den Bereich zwischen~...}{\textbf{\textcolor{DARCgreen}{\qty{-24}{\kHz} und \qty{+24}{\kHz}.}}}
{\qty{-48}{\kHz} und \qty{+48}{\kHz}.}
{\qty{0}{\Hz} und \qty{96}{\kHz}.}
{\qty{0}{\Hz} und \qty{6}{\kHz}.}
\end{QQuestion}

}
\end{frame}

\begin{frame}
\only<1>{
\begin{QQuestion}{AF635}{Welchen Frequenzbereich (z.~B. in Bezug auf eine Mitten- oder Trägerfrequenz) kann ein digitaler Datenstrom entsprechend dem Abtasttheorem maximal eindeutig abbilden, der aus einem I- und einem Q-Anteil mit einer Abtastrate von \underline{jeweils} 96000 Samples pro Sekunde besteht? Den Bereich zwischen~...}{\qty{-48}{\kHz} und \qty{+48}{\kHz}.}
{\qty{-24}{\kHz} und \qty{+24}{\kHz}.}
{\qty{0}{\Hz} und \qty{192}{\kHz}.}
{\qty{0}{\Hz} und \qty{9,6}{\kHz}.}
\end{QQuestion}

}
\only<2>{
\begin{QQuestion}{AF635}{Welchen Frequenzbereich (z.~B. in Bezug auf eine Mitten- oder Trägerfrequenz) kann ein digitaler Datenstrom entsprechend dem Abtasttheorem maximal eindeutig abbilden, der aus einem I- und einem Q-Anteil mit einer Abtastrate von \underline{jeweils} 96000 Samples pro Sekunde besteht? Den Bereich zwischen~...}{\textbf{\textcolor{DARCgreen}{\qty{-48}{\kHz} und \qty{+48}{\kHz}.}}}
{\qty{-24}{\kHz} und \qty{+24}{\kHz}.}
{\qty{0}{\Hz} und \qty{192}{\kHz}.}
{\qty{0}{\Hz} und \qty{9,6}{\kHz}.}
\end{QQuestion}

}
\end{frame}

\begin{frame}
\only<1>{
\begin{QQuestion}{AF636}{Welchen Frequenzbereich (z.~B. in Bezug auf eine Mitten- oder Trägerfrequenz) kann ein digitaler Datenstrom entsprechend dem Abtasttheorem maximal eindeutig abbilden, der aus einem I- und einem Q-Anteil mit einer Abtastrate von \underline{jeweils} 10 Millionen Samples pro Sekunde besteht? Den Bereich zwischen~...}{\qty{-5}{\MHz} und \qty{+5}{\MHz}.}
{\qty{-10}{\MHz} und \qty{+10}{\MHz}.}
{\qty{0}{\Hz} und \qty{512}{\kHz}.}
{\qty{0}{\Hz} und \qty{1024}{\kHz}.}
\end{QQuestion}

}
\only<2>{
\begin{QQuestion}{AF636}{Welchen Frequenzbereich (z.~B. in Bezug auf eine Mitten- oder Trägerfrequenz) kann ein digitaler Datenstrom entsprechend dem Abtasttheorem maximal eindeutig abbilden, der aus einem I- und einem Q-Anteil mit einer Abtastrate von \underline{jeweils} 10 Millionen Samples pro Sekunde besteht? Den Bereich zwischen~...}{\textbf{\textcolor{DARCgreen}{\qty{-5}{\MHz} und \qty{+5}{\MHz}.}}}
{\qty{-10}{\MHz} und \qty{+10}{\MHz}.}
{\qty{0}{\Hz} und \qty{512}{\kHz}.}
{\qty{0}{\Hz} und \qty{1024}{\kHz}.}
\end{QQuestion}

}
\end{frame}%ENDCONTENT
