
\section{Parasitäre Schwingungen}
\label{section:parasitaere_schwingungen}
\begin{frame}%STARTCONTENT

\only<1>{
\begin{QQuestion}{AJ212}{Parasitäre Schwingungen können Störungen hervorrufen. Man erkennt diese Schwingungen unter anderem daran, dass sie~...}{bei ganzzahligen Vielfachen der Betriebsfrequenz auftreten.}
{bei ungeradzahligen Vielfachen der Betriebsfrequenz auftreten.}
{bei geradzahligen Vielfachen der Betriebsfrequenz auftreten.}
{keinen festen Bezug zur Betriebsfrequenz haben.}
\end{QQuestion}

}
\only<2>{
\begin{QQuestion}{AJ212}{Parasitäre Schwingungen können Störungen hervorrufen. Man erkennt diese Schwingungen unter anderem daran, dass sie~...}{bei ganzzahligen Vielfachen der Betriebsfrequenz auftreten.}
{bei ungeradzahligen Vielfachen der Betriebsfrequenz auftreten.}
{bei geradzahligen Vielfachen der Betriebsfrequenz auftreten.}
{\textbf{\textcolor{DARCgreen}{keinen festen Bezug zur Betriebsfrequenz haben.}}}
\end{QQuestion}

}
\end{frame}

\begin{frame}
\only<1>{
\begin{QQuestion}{AJ213}{Die Ausgangsleistungsanzeige eines HF-Verstärkers zeigt beim Abstimmen geringfügige sprunghafte Schwankungen. Sie werden möglicherweise hervorgerufen durch~...}{vom Wind verursachte Bewegungen der Antenne.}
{Welligkeit auf der Stromversorgung.}
{Temperaturschwankungen im Netzteil.}
{parasitäre Schwingungen.}
\end{QQuestion}

}
\only<2>{
\begin{QQuestion}{AJ213}{Die Ausgangsleistungsanzeige eines HF-Verstärkers zeigt beim Abstimmen geringfügige sprunghafte Schwankungen. Sie werden möglicherweise hervorgerufen durch~...}{vom Wind verursachte Bewegungen der Antenne.}
{Welligkeit auf der Stromversorgung.}
{Temperaturschwankungen im Netzteil.}
{\textbf{\textcolor{DARCgreen}{parasitäre Schwingungen.}}}
\end{QQuestion}

}
\end{frame}

\begin{frame}
\only<1>{
\begin{QQuestion}{AJ217}{Wie kann man bei einem VHF-Sender mit kleiner Leistung die Entstehung parasitärer Schwingungen wirksam unterdrücken?}{Durch Anbringen eines Klappferritkerns an der Mikrofonzuleitung.}
{Durch Aufstecken einer Ferritperle auf die Emitterzuleitung des Endstufentransistors.}
{Durch Aufkleben einer Ferritperle auf das Gehäuse des Endstufentransistors.}
{Durch Anbringen eines Klappferritkerns an der Stromversorgungszuleitung.}
\end{QQuestion}

}
\only<2>{
\begin{QQuestion}{AJ217}{Wie kann man bei einem VHF-Sender mit kleiner Leistung die Entstehung parasitärer Schwingungen wirksam unterdrücken?}{Durch Anbringen eines Klappferritkerns an der Mikrofonzuleitung.}
{\textbf{\textcolor{DARCgreen}{Durch Aufstecken einer Ferritperle auf die Emitterzuleitung des Endstufentransistors.}}}
{Durch Aufkleben einer Ferritperle auf das Gehäuse des Endstufentransistors.}
{Durch Anbringen eines Klappferritkerns an der Stromversorgungszuleitung.}
\end{QQuestion}

}
\end{frame}

\begin{frame}
\only<1>{
\begin{PQuestion}{AF416}{Wozu dient der Widerstand $R$ parallel zur Trafowicklung $T_2$?}{Er dient zur Erhöhung des HF-Wirkungsgrades der Verstärkerstufe.}
{Er dient zur Anpassung der Primärwicklung an die folgende PA.}
{Er soll die Entstehung parasitärer Schwingungen verhindern.}
{Er dient zur Begrenzung des Kollektorstroms bei Übersteuerung.}
{\DARCimage{1.0\linewidth}{767include}}\end{PQuestion}

}
\only<2>{
\begin{PQuestion}{AF416}{Wozu dient der Widerstand $R$ parallel zur Trafowicklung $T_2$?}{Er dient zur Erhöhung des HF-Wirkungsgrades der Verstärkerstufe.}
{Er dient zur Anpassung der Primärwicklung an die folgende PA.}
{\textbf{\textcolor{DARCgreen}{Er soll die Entstehung parasitärer Schwingungen verhindern.}}}
{Er dient zur Begrenzung des Kollektorstroms bei Übersteuerung.}
{\DARCimage{1.0\linewidth}{767include}}\end{PQuestion}

}
\end{frame}%ENDCONTENT
