
\section{Effektive Strahlungsleistung (ERP) II}
\label{section:effektive_strahlungsleistung_erp_2}
\begin{frame}%STARTCONTENT

\only<1>{
\begin{QQuestion}{AG501}{Die äquivalente (effektive) Strahlungsleistung (ERP) ist~...}{das Produkt aus der Leistung, die unmittelbar der Antenne zugeführt wird, und ihrem Gewinnfaktor in einer Richtung, bezogen auf den isotropen Strahler.}
{das Produkt aus der Leistung, die unmittelbar der Antenne zugeführt wird, und ihrem Gewinnfaktor in einer Richtung, bezogen auf den Halbwellendipol.}
{die durchschnittliche Leistung, die ein Sender unter normalen Betriebsbedingungen während einer Periode der Hochfrequenzschwingung bei der höchsten Spitze der Modulationshüllkurve der Antennenspeiseleitung zuführt.}
{die durchschnittliche Leistung, die ein Sender unter normalen Betriebsbedingungen an die Antennenspeiseleitung während eines Zeitintervalls abgibt, das im Verhältnis zur Periode der tiefsten Modulationsfrequenz ausreichend lang ist.}
\end{QQuestion}

}
\only<2>{
\begin{QQuestion}{AG501}{Die äquivalente (effektive) Strahlungsleistung (ERP) ist~...}{das Produkt aus der Leistung, die unmittelbar der Antenne zugeführt wird, und ihrem Gewinnfaktor in einer Richtung, bezogen auf den isotropen Strahler.}
{\textbf{\textcolor{DARCgreen}{das Produkt aus der Leistung, die unmittelbar der Antenne zugeführt wird, und ihrem Gewinnfaktor in einer Richtung, bezogen auf den Halbwellendipol.}}}
{die durchschnittliche Leistung, die ein Sender unter normalen Betriebsbedingungen während einer Periode der Hochfrequenzschwingung bei der höchsten Spitze der Modulationshüllkurve der Antennenspeiseleitung zuführt.}
{die durchschnittliche Leistung, die ein Sender unter normalen Betriebsbedingungen an die Antennenspeiseleitung während eines Zeitintervalls abgibt, das im Verhältnis zur Periode der tiefsten Modulationsfrequenz ausreichend lang ist.}
\end{QQuestion}

}
\end{frame}

\begin{frame}
\only<1>{
\begin{QQuestion}{AG502}{Nach welcher der Antworten kann die ERP (Effective Radiated Power) berechnet werden?}{$P_{\symup{ERP}} = (P_{\symup{Sender}} - P_{\symup{Verluste}}) + G_{\symup{Antenne}}$ bezogen auf einen Halbwellendipol}
{$P_{\symup{ERP}} = (P_{\symup{Sender}} \cdot P_{\symup{Verluste}}) \cdot G_{\symup{Antenne}}$ bezogen auf einen isotropen Strahler}
{$P_{\symup{ERP}} = (P_{\symup{Sender}} - P_{\symup{Verluste}}) \cdot G_{\symup{Antenne}}$ bezogen auf einen Halbwellendipol}
{$P_{\symup{ERP}} = (P_{\symup{Sender}} + P_{\symup{Verluste}}) + G_{\symup{Antenne}}$ bezogen auf einen isotropen Strahler}
\end{QQuestion}

}
\only<2>{
\begin{QQuestion}{AG502}{Nach welcher der Antworten kann die ERP (Effective Radiated Power) berechnet werden?}{$P_{\symup{ERP}} = (P_{\symup{Sender}} - P_{\symup{Verluste}}) + G_{\symup{Antenne}}$ bezogen auf einen Halbwellendipol}
{$P_{\symup{ERP}} = (P_{\symup{Sender}} \cdot P_{\symup{Verluste}}) \cdot G_{\symup{Antenne}}$ bezogen auf einen isotropen Strahler}
{\textbf{\textcolor{DARCgreen}{$P_{\symup{ERP}} = (P_{\symup{Sender}} - P_{\symup{Verluste}}) \cdot G_{\symup{Antenne}}$ bezogen auf einen Halbwellendipol}}}
{$P_{\symup{ERP}} = (P_{\symup{Sender}} + P_{\symup{Verluste}}) + G_{\symup{Antenne}}$ bezogen auf einen isotropen Strahler}
\end{QQuestion}

}
\end{frame}

\begin{frame}
\only<1>{
\begin{QQuestion}{AG503}{Ein Sender für das \qty{630}{\m}-Band mit \qty{50}{\W} Ausgangsleistung ist mittels eines kurzen Koaxialkabels an eine Antenne mit 20~dBd Verlust angeschlossen. Welche ERP wird von der Antenne abgestrahlt?}{\qty{2,5}{\W}}
{\qty{5,0}{\W}}
{\qty{0,5}{\W}}
{\qty{50}{\W}}
\end{QQuestion}

}
\only<2>{
\begin{QQuestion}{AG503}{Ein Sender für das \qty{630}{\m}-Band mit \qty{50}{\W} Ausgangsleistung ist mittels eines kurzen Koaxialkabels an eine Antenne mit 20~dBd Verlust angeschlossen. Welche ERP wird von der Antenne abgestrahlt?}{\qty{2,5}{\W}}
{\qty{5,0}{\W}}
{\textbf{\textcolor{DARCgreen}{\qty{0,5}{\W}}}}
{\qty{50}{\W}}
\end{QQuestion}

}
\end{frame}

\begin{frame}
\frametitle{Lösungsweg}
\begin{itemize}
  \item gegeben: $P_S = 50W$
  \item gegeben: $a \approx 0W$
  \item gegeben: $g_d = -20dBd$
  \item gesucht: $P_{\textrm{ERP}}$
  \end{itemize}
    \pause
    $P_{\textrm{ERP}} = P_S \cdot 10^{\frac{g_d -- a}{10dB}} = 50W \cdot 10^{\frac{-20dBd -- 0W}{10dB}} = 50W \cdot 10^{-2} = 0,5W$



\end{frame}%ENDCONTENT
