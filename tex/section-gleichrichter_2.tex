
\section{Gleichrichter II}
\label{section:gleichrichter_2}
\begin{frame}%STARTCONTENT

\only<1>{
\begin{PQuestion}{AD302}{Berechnen Sie für diese Schaltung die Leerlaufspannung an den Klemmen A - B.}{Zirka \qty{42}{\V}}
{Zirka \qty{15}{\V}}
{Zirka \qty{30}{\V}}
{Zirka \qty{21}{\V}}
{\DARCimage{1.0\linewidth}{198include}}\end{PQuestion}

}
\only<2>{
\begin{PQuestion}{AD302}{Berechnen Sie für diese Schaltung die Leerlaufspannung an den Klemmen A - B.}{Zirka \qty{42}{\V}}
{Zirka \qty{15}{\V}}
{Zirka \qty{30}{\V}}
{\textbf{\textcolor{DARCgreen}{Zirka \qty{21}{\V}}}}
{\DARCimage{1.0\linewidth}{198include}}\end{PQuestion}

}
\end{frame}

\begin{frame}
\frametitle{Lösungsweg}
\begin{itemize}
  \item gegeben: $U_{eff} = 15V$
  \item gesucht: $\hat{U}$
  \end{itemize}
    \pause
    $\hat{U} = U_{eff} \cdot \sqrt{2} = 15V \cdot 1,41 = 21,21V$



\end{frame}

\begin{frame}
\only<1>{
\begin{PQuestion}{AD303}{Welche Spannungsfestigkeit des Kondensators sollte mindestens gewählt werden, wenn das Transformationsverhältnis 20:1 beträgt und ein Sicherheitsaufschlag auf die Spannungsfestigkeit von \qty{50}{\percent} berücksichtigt werden soll?}{\qty{35}{\V}}
{\qty{16}{\V} }
{\qty{25}{\V} }
{\qty{10}{\V}}
{\DARCimage{1.0\linewidth}{29include}}\end{PQuestion}

}
\only<2>{
\begin{PQuestion}{AD303}{Welche Spannungsfestigkeit des Kondensators sollte mindestens gewählt werden, wenn das Transformationsverhältnis 20:1 beträgt und ein Sicherheitsaufschlag auf die Spannungsfestigkeit von \qty{50}{\percent} berücksichtigt werden soll?}{\qty{35}{\V}}
{\qty{16}{\V} }
{\textbf{\textcolor{DARCgreen}{\qty{25}{\V} }}}
{\qty{10}{\V}}
{\DARCimage{1.0\linewidth}{29include}}\end{PQuestion}

}
\end{frame}

\begin{frame}
\frametitle{Lösungsweg}
\begin{itemize}
  \item gegeben: $U_P = 230V$
  \item gegeben: $\"{u} = 20:1$
  \item gesucht: $\hat{U} + 50\%$
  \end{itemize}
    \pause
    $ü = \frac{U_P}{U_S} \Rightarrow U_S = \frac{U_P}{ü} = \frac{230V}{20} = 11,5V$
    \pause
    $\hat{U} = U_S \cdot \sqrt{2} = 11,5V \cdot 1,41 \approx 16,26V$
    \pause
    $\hat{U} + 50\% \approx 25V$



\end{frame}

\begin{frame}
\only<1>{
\begin{PQuestion}{AD304}{Bei einem Transformationsverhältnis von 5:1 sollte die Spannungsfestigkeit der Diode (max. Spannung plus \qty{20}{\percent} Sicherheitsaufschlag) in dieser Schaltung nicht weniger als~...}{\qty{90}{\V} betragen.}
{\qty{78}{\V} betragen.}
{\qty{156}{\V} betragen.}
{\qty{130}{\V} betragen.}
{\DARCimage{1.0\linewidth}{29include}}\end{PQuestion}

}
\only<2>{
\begin{PQuestion}{AD304}{Bei einem Transformationsverhältnis von 5:1 sollte die Spannungsfestigkeit der Diode (max. Spannung plus \qty{20}{\percent} Sicherheitsaufschlag) in dieser Schaltung nicht weniger als~...}{\qty{90}{\V} betragen.}
{\qty{78}{\V} betragen.}
{\textbf{\textcolor{DARCgreen}{\qty{156}{\V} betragen.}}}
{\qty{130}{\V} betragen.}
{\DARCimage{1.0\linewidth}{29include}}\end{PQuestion}

}
\end{frame}

\begin{frame}
\frametitle{Lösungsweg}
\begin{itemize}
  \item gegeben: $U_P = 230V$
  \item gegeben: $\"{u} = 5:1$
  \item gesucht: $U_{SS} + 20\%$
  \end{itemize}
    \pause
    $ü = \frac{U_P}{U_S} \Rightarrow U_S = \frac{U_P}{ü} = \frac{230V}{5} = 46V$
    \pause
    $\hat{U} = U_S \cdot \sqrt{2} = 46V \cdot 1,41 \approx 65,05V$
    \pause
    $U_{SS} + 20\% = 2 \cdot \hat{U} + 20\% \approx 156V$



\end{frame}%ENDCONTENT
