
\section{Einseitenbandmodulation (SSB) II}
\label{section:ssb_2}
\begin{frame}%STARTCONTENT

\frametitle{Bandbreite}
\begin{columns}
    \begin{column}{0.48\textwidth}
    \begin{itemize}
  \item Im Gegensatz zu AM wird weniger als die halbe Bandbreite verwendet
  \item Maximal \qty{2,7}{\kilo\hertz}
  \item Entspricht dem NF-Signal
  \end{itemize}

    \end{column}
   \begin{column}{0.48\textwidth}
       
\begin{figure}
    \DARCimage{0.85\linewidth}{743include}
    \caption{\scriptsize Bandbreite von AM, USB und LSB}
    \label{e_bandbreite_am_ssb}
\end{figure}


   \end{column}
\end{columns}

\end{frame}

\begin{frame}
\only<1>{
\begin{QQuestion}{EE201}{Wie unterscheidet sich SSB von AM in Bezug auf die Bandbreite?}{SSB beansprucht etwa 1/4 Bandbreite der Modulationsart AM.}
{SSB beansprucht etwas mehr als die halbe Bandbreite der Modulationsart AM.}
{SSB beansprucht weniger als die halbe Bandbreite der Modulationsart AM.}
{SSB und AM lassen keinen Vergleich zu, da sie grundverschieden erzeugt werden.}
\end{QQuestion}

}
\only<2>{
\begin{QQuestion}{EE201}{Wie unterscheidet sich SSB von AM in Bezug auf die Bandbreite?}{SSB beansprucht etwa 1/4 Bandbreite der Modulationsart AM.}
{SSB beansprucht etwas mehr als die halbe Bandbreite der Modulationsart AM.}
{\textbf{\textcolor{DARCgreen}{SSB beansprucht weniger als die halbe Bandbreite der Modulationsart AM.}}}
{SSB und AM lassen keinen Vergleich zu, da sie grundverschieden erzeugt werden.}
\end{QQuestion}

}
\end{frame}

\begin{frame}
\only<1>{
\begin{QQuestion}{EE202}{Wie groß ist in etwa die HF-Bandbreite, die für die Übertragung eines SSB-Signals erforderlich ist?}{Sie entspricht der Hälfte der Bandbreite des NF-Signals.}
{Sie entspricht der Bandbreite des NF-Signals.}
{Sie entspricht der doppelten Bandbreite des NF-Signals.}
{Sie ist Null, weil bei SSB-Modulation der HF-Träger unterdrückt wird.}
\end{QQuestion}

}
\only<2>{
\begin{QQuestion}{EE202}{Wie groß ist in etwa die HF-Bandbreite, die für die Übertragung eines SSB-Signals erforderlich ist?}{Sie entspricht der Hälfte der Bandbreite des NF-Signals.}
{\textbf{\textcolor{DARCgreen}{Sie entspricht der Bandbreite des NF-Signals.}}}
{Sie entspricht der doppelten Bandbreite des NF-Signals.}
{Sie ist Null, weil bei SSB-Modulation der HF-Träger unterdrückt wird.}
\end{QQuestion}

}
\end{frame}

\begin{frame}
\only<1>{
\begin{QQuestion}{EJ210}{Um Störungen auf benachbarten Frequenzen zu minimieren, sollte die Übertragungsbandbreite bei SSB~...}{höchstens \qty{1,8}{\kHz} betragen.}
{höchstens \qty{2,7}{\kHz} betragen.}
{höchstens \qty{3,1}{\kHz} betragen.}
{höchstens \qty{15,0}{\kHz} betragen.}
\end{QQuestion}

}
\only<2>{
\begin{QQuestion}{EJ210}{Um Störungen auf benachbarten Frequenzen zu minimieren, sollte die Übertragungsbandbreite bei SSB~...}{höchstens \qty{1,8}{\kHz} betragen.}
{\textbf{\textcolor{DARCgreen}{höchstens \qty{2,7}{\kHz} betragen.}}}
{höchstens \qty{3,1}{\kHz} betragen.}
{höchstens \qty{15,0}{\kHz} betragen.}
\end{QQuestion}

}
\end{frame}

\begin{frame}
\frametitle{Modulation}
\begin{columns}
    \begin{column}{0.48\textwidth}
    \begin{itemize}
  \item Durch Mischung und Filterung
  \item Mit der Vorauswahl von USB und LSB wird die Trägerfrequenz gewählt
  \item Durch den Mischer entstehen zwei Frequenzen
  \item Im Bandfilter wird nur eine Frequenz durchgelassen
  \end{itemize}

    \end{column}
   \begin{column}{0.48\textwidth}
       
\begin{figure}
    \DARCimage{0.85\linewidth}{500include}
    \caption{\scriptsize Blockschaltbild zur Modulation von SSB mit der Filtermethode}
    \label{e_ssb_modulation}
\end{figure}


   \end{column}
\end{columns}

\end{frame}

\begin{frame}
\begin{columns}
    \begin{column}{0.48\textwidth}
    \begin{itemize}
  \item Der Trick ist hier, dass das Bandfilter nur eine Resonanzfrequenz hat
  \item Durch die Verschiebung der Trägerfrequenz im Oszillator wird dann das gewünschte Seitenband durchgelassen
  \end{itemize}

    \end{column}
   \begin{column}{0.48\textwidth}
       
\begin{figure}
    \DARCimage{0.85\linewidth}{500include}
    \caption{\scriptsize Blockschaltbild zur Modulation von SSB mit der Filtermethode}
    \label{e_ssb_modulation}
\end{figure}


   \end{column}
\end{columns}

\end{frame}

\begin{frame}
\begin{columns}
    \begin{column}{0.48\textwidth}
    Beispiel LSB:

\begin{itemize}
  \item Mikrofon: \qty{300}{\hertz} -- \qty{3}{\kilo\hertz}
  \item LSB-Oszillator: \qty{9001,5}{\kilo\hertz}
  \item DSB-Signal:<br/> a) 8998,5 -- 9001,2<br/> b) 9001,8 -- 9004,5 
  \item Filter: \qty{9000}{\kilo\hertz}  $\pm$  \qty{1,5}{\kilo\hertz}
  \item SSB-Signal:<br/> \qtyrange{8998,5}{9001,2}{\kilo\hertz}
  \end{itemize}

    \end{column}
   \begin{column}{0.48\textwidth}
       
\begin{figure}
    \DARCimage{0.85\linewidth}{831include}
    \caption{\scriptsize Frequenzen mit der Filtermethode bei LSB}
    \label{e_ssb_modulation_lsb}
\end{figure}


   \end{column}
\end{columns}

\end{frame}

\begin{frame}
\begin{columns}
    \begin{column}{0.48\textwidth}
    Beispiel USB:

\begin{itemize}
  \item Mikrofon: \qty{300}{\hertz} -- \qty{3}{\kilo\hertz}
  \item USB-Oszillator: \qty{8998,5}{\kilo\hertz}
  \item DSB-Signal:<br/> a) \qtyrange{8995,5}{8998,2}{\kilo\hertz}<br/> b) \qtyrange{8998,8}{9001,5}{\kilo\hertz}
  \item Filter: \qty{9000}{\kilo\hertz}  $\pm$  \qty{1,5}{\kilo\hertz}
  \item SSB-Signal:<br/> \qtyrange{8998,8}{9001,5}{\kilo\hertz}
  \end{itemize}

    \end{column}
   \begin{column}{0.48\textwidth}
       
\begin{figure}
    \DARCimage{0.85\linewidth}{832include}
    \caption{\scriptsize Frequenzen mit der Filtermethode bei USB}
    \label{e_ssb_modulation_usb}
\end{figure}


   \end{column}
\end{columns}

\end{frame}

\begin{frame}
\only<1>{
\begin{QQuestion}{EE203}{Ein Träger von \qty{21,250}{\MHz} wird mit der NF-Frequenz von \qty{1}{\kHz} in SSB (USB) moduliert. Welche Frequenz tritt im ideal modulierten HF-Signal auf?}{\qty{21,260}{\MHz}}
{\qty{21,250}{\MHz}}
{\qty{21,249}{\MHz}}
{\qty{21,251}{\MHz}}
\end{QQuestion}

}
\only<2>{
\begin{QQuestion}{EE203}{Ein Träger von \qty{21,250}{\MHz} wird mit der NF-Frequenz von \qty{1}{\kHz} in SSB (USB) moduliert. Welche Frequenz tritt im ideal modulierten HF-Signal auf?}{\qty{21,260}{\MHz}}
{\qty{21,250}{\MHz}}
{\qty{21,249}{\MHz}}
{\textbf{\textcolor{DARCgreen}{\qty{21,251}{\MHz}}}}
\end{QQuestion}

}
\end{frame}

\begin{frame}
\only<1>{
\begin{QQuestion}{EE204}{Ein Träger von \qty{3,65}{\MHz} wird mit der NF-Frequenz von \qty{2}{\kHz} in SSB (LSB) moduliert. Welche Frequenz/Frequenzen treten im modulierten HF-Signal hauptsächlich auf?}{\qty{3,648}{\MHz} und \qty{3,650}{\MHz}}
{\qty{3,648}{\MHz}}
{\qty{3,652}{\MHz}}
{\qty{3,648}{\MHz} und \qty{3,652}{\MHz}}
\end{QQuestion}

}
\only<2>{
\begin{QQuestion}{EE204}{Ein Träger von \qty{3,65}{\MHz} wird mit der NF-Frequenz von \qty{2}{\kHz} in SSB (LSB) moduliert. Welche Frequenz/Frequenzen treten im modulierten HF-Signal hauptsächlich auf?}{\qty{3,648}{\MHz} und \qty{3,650}{\MHz}}
{\textbf{\textcolor{DARCgreen}{\qty{3,648}{\MHz}}}}
{\qty{3,652}{\MHz}}
{\qty{3,648}{\MHz} und \qty{3,652}{\MHz}}
\end{QQuestion}

}
\end{frame}

\begin{frame}
\frametitle{NF-Signal}
\begin{columns}
    \begin{column}{0.48\textwidth}
    \begin{itemize}
  \item Für Sprache reicht zwischen 300 und \qty{3000}{\hertz}
  \item Entspricht \qty{2,7}{\kilo\hertz}
  \item Es werden auch kleinere Filter, z.B. \qty{2,4}{\kilo\hertz} verwendet
  \item An vielen TRX lassen sich die Filter einstellen
  \end{itemize}

    \end{column}
   \begin{column}{0.48\textwidth}
       \begin{itemize}
  \item Wird ein NF-Signal mit größerer Bandbreite verwendet, steigt die HF-Bandbreite
  \item Sollte vermieden werden, um benachbarte Signale nicht zu stören
  \item Auf maximale Bandbreite im Bandplan achten
  \end{itemize}

   \end{column}
\end{columns}

\end{frame}

\begin{frame}
\only<1>{
\begin{QQuestion}{EJ211}{Um etwaige Funkstörungen auf Nachbarfrequenzen zu begrenzen, sollte bei SSB-Telefonie die höchste zu übertragende NF-Frequenz~...}{unter \qty{3}{\kHz} liegen.}
{unter \qty{1}{\kHz} liegen.}
{unter \qty{5}{\kHz} liegen.}
{unter \qty{10}{\kHz} liegen.}
\end{QQuestion}

}
\only<2>{
\begin{QQuestion}{EJ211}{Um etwaige Funkstörungen auf Nachbarfrequenzen zu begrenzen, sollte bei SSB-Telefonie die höchste zu übertragende NF-Frequenz~...}{\textbf{\textcolor{DARCgreen}{unter \qty{3}{\kHz} liegen.}}}
{unter \qty{1}{\kHz} liegen.}
{unter \qty{5}{\kHz} liegen.}
{unter \qty{10}{\kHz} liegen.}
\end{QQuestion}

}

\end{frame}

\begin{frame}
\only<1>{
\begin{QQuestion}{EF310}{Welche Bandbreite sollte das nachgeschaltete Filter zur Unterdrückung eines Seitenbandes bei der Erzeugung eines SSB-Telefoniesignals haben?}{\qty{455}{\kHz} }
{\qty{800}{\Hz} }
{\qty{2,4}{\kHz} }
{\qty{10,7}{\MHz} }
\end{QQuestion}

}
\only<2>{
\begin{QQuestion}{EF310}{Welche Bandbreite sollte das nachgeschaltete Filter zur Unterdrückung eines Seitenbandes bei der Erzeugung eines SSB-Telefoniesignals haben?}{\qty{455}{\kHz} }
{\qty{800}{\Hz} }
{\textbf{\textcolor{DARCgreen}{\qty{2,4}{\kHz} }}}
{\qty{10,7}{\MHz} }
\end{QQuestion}

}
\end{frame}

\begin{frame}
\only<1>{
\begin{QQuestion}{EE207}{Wie groß ist die Bandbreite von CW im Vergleich zu einem Sprachsignal in SSB oder AM?}{In beiden Fällen weist CW eine kleinere Bandbreite auf.}
{In beiden Fällen weist CW eine größere Bandbreite auf.}
{Die Bandbreite von CW ist kleiner als bei SSB, jedoch größer als bei AM.}
{Die Bandbreite von CW ist größer als bei SSB, jedoch kleiner als bei AM.}
\end{QQuestion}

}
\only<2>{
\begin{QQuestion}{EE207}{Wie groß ist die Bandbreite von CW im Vergleich zu einem Sprachsignal in SSB oder AM?}{\textbf{\textcolor{DARCgreen}{In beiden Fällen weist CW eine kleinere Bandbreite auf.}}}
{In beiden Fällen weist CW eine größere Bandbreite auf.}
{Die Bandbreite von CW ist kleiner als bei SSB, jedoch größer als bei AM.}
{Die Bandbreite von CW ist größer als bei SSB, jedoch kleiner als bei AM.}
\end{QQuestion}

}
\end{frame}

\begin{frame}
\frametitle{Mikrofonverstärkung}
\begin{itemize}
  \item Mit der NF-Leistung wird die Leistung der HF gesteuert
  \item Zu leises Mikrofon bewirkt weniger Ausgangleistung am Sender
  \item Eine zu starke Mikrofonverstärkung kann Störungen bei Stationen auf dicht benachbarten Frequenzen verursachen
  \end{itemize}
\end{frame}

\begin{frame}
\only<1>{
\begin{QQuestion}{EE206}{Was bewirkt eine zu geringe Mikrofonverstärkung bei einem SSB-Transceiver?}{Störungen bei Stationen, die auf dicht benachbarten Frequenzen arbeiten}
{Störungen von Stationen, die auf einem anderen Frequenzband arbeiten}
{geringe Bandbreite}
{geringe Ausgangsleistung}
\end{QQuestion}

}
\only<2>{
\begin{QQuestion}{EE206}{Was bewirkt eine zu geringe Mikrofonverstärkung bei einem SSB-Transceiver?}{Störungen bei Stationen, die auf dicht benachbarten Frequenzen arbeiten}
{Störungen von Stationen, die auf einem anderen Frequenzband arbeiten}
{geringe Bandbreite}
{\textbf{\textcolor{DARCgreen}{geringe Ausgangsleistung}}}
\end{QQuestion}

}
\end{frame}

\begin{frame}
\only<1>{
\begin{QQuestion}{EE205}{Welche der aufgeführten Maßnahmen verringert die Ausgangsleistung eines SSB-Senders?}{Lauter ins Mikrofon sprechen }
{Verringern der NF-Amplitude}
{Verringern der Squelcheinstellung }
{Erhöhen der NF-Bandbreite}
\end{QQuestion}

}
\only<2>{
\begin{QQuestion}{EE205}{Welche der aufgeführten Maßnahmen verringert die Ausgangsleistung eines SSB-Senders?}{Lauter ins Mikrofon sprechen }
{\textbf{\textcolor{DARCgreen}{Verringern der NF-Amplitude}}}
{Verringern der Squelcheinstellung }
{Erhöhen der NF-Bandbreite}
\end{QQuestion}

}
\end{frame}

\begin{frame}
\only<1>{
\begin{QQuestion}{EJ215}{Was bewirkt in der Regel eine zu hohe Mikrofonverstärkung bei einem SSB-Transceiver?}{Störungen bei Stationen, die auf dicht benachbarten Frequenzen arbeiten}
{Störungen von Stationen, die auf einem anderen Frequenzband arbeiten}
{Störungen der Stromversorgung des Transceivers}
{Störungen von anderen elektronischen Geräten}
\end{QQuestion}

}
\only<2>{
\begin{QQuestion}{EJ215}{Was bewirkt in der Regel eine zu hohe Mikrofonverstärkung bei einem SSB-Transceiver?}{\textbf{\textcolor{DARCgreen}{Störungen bei Stationen, die auf dicht benachbarten Frequenzen arbeiten}}}
{Störungen von Stationen, die auf einem anderen Frequenzband arbeiten}
{Störungen der Stromversorgung des Transceivers}
{Störungen von anderen elektronischen Geräten}
\end{QQuestion}

}
\end{frame}%ENDCONTENT
