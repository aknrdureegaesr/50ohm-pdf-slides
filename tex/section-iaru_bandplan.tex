
\section{IARU-Bandpläne}
\label{section:iaru_bandplan}
\begin{frame}%STARTCONTENT

\frametitle{International Amateur Radio Union}
\begin{itemize}
  \item Zusammenschluss nationaler Amateurfunkverbände
  \item \emph{Weltweit geordnetes Nebeneinander der verschiedenen Betriebsarten auf den Amateurfunkbändern}
  \item Geben einen IARU-Bandplan heraus
  \item Funkamateure sollen diesen einhalten
  \end{itemize}

\end{frame}

\begin{frame}
\only<1>{
\begin{QQuestion}{BC201}{Wie verbindlich sind die Bandpläne der IARU?}{Sie müssen in Regionen mit hoher Dichte von Amateurfunkstellen eingehalten werden.}
{Sie sind eine Empfehlung. Ihre Einhaltung soll allen Funkamateuren zugute kommen.}
{Sie sind für unbesetzte und automatisch arbeitende Amateurfunkstellen amtlich vorgeschrieben.}
{Sie müssen von jedem Funkamateur bei internationalem Funkverkehr angewendet werden.}
\end{QQuestion}

}
\only<2>{
\begin{QQuestion}{BC201}{Wie verbindlich sind die Bandpläne der IARU?}{Sie müssen in Regionen mit hoher Dichte von Amateurfunkstellen eingehalten werden.}
{\textbf{\textcolor{DARCgreen}{Sie sind eine Empfehlung. Ihre Einhaltung soll allen Funkamateuren zugute kommen.}}}
{Sie sind für unbesetzte und automatisch arbeitende Amateurfunkstellen amtlich vorgeschrieben.}
{Sie müssen von jedem Funkamateur bei internationalem Funkverkehr angewendet werden.}
\end{QQuestion}

}
\end{frame}

\begin{frame}\begin{itemize}
  \item Die Bandpläne behandeln auch die Frequenzbereiche für verschiedene Übertragungsarten
  \item Für Morsetelegrafie (CW) ist der empfohlene Bereich am Bandanfang
  \end{itemize}
\end{frame}

\begin{frame}
\only<1>{
\begin{QQuestion}{BC204}{In welchem Bereich der Amateurfunkbänder empfiehlt der IARU-Bandplan üblicherweise die Nutzung von Morsetelegrafie?}{Unterhalb von 10 MHz am Bandanfang, oberhalb von 10 MHz am Bandende}
{Am Bandende}
{In der Bandmitte}
{Am Bandanfang}
\end{QQuestion}

}
\only<2>{
\begin{QQuestion}{BC204}{In welchem Bereich der Amateurfunkbänder empfiehlt der IARU-Bandplan üblicherweise die Nutzung von Morsetelegrafie?}{Unterhalb von 10 MHz am Bandanfang, oberhalb von 10 MHz am Bandende}
{Am Bandende}
{In der Bandmitte}
{\textbf{\textcolor{DARCgreen}{Am Bandanfang}}}
\end{QQuestion}

}
\end{frame}%ENDCONTENT
