
\section{Leiter und Nichtleiter}
\label{section:leiter_nichtleiter}
\begin{frame}%STARTCONTENT
Materialien lassen sich in drei Gruppen einteilen:

\begin{enumerate}
  \item[1] Leiter
  \item[2] Nichtleiter
  \item[3] Halbleiter
  \end{enumerate}

\end{frame}

\begin{frame}
\frametitle{Leiter}
\begin{itemize}
  \item Leiten elektrischen Strom
  \item Sind meistens aus Metall
  \item Manche können Strom besser leiten als andere
  \end{itemize}

\end{frame}

\begin{frame}
\frametitle{Leiter, sortiert von besonders gut zu weniger gut leitend}
\begin{table}
\begin{DARCtabular}{cl}
     1  & \emph{Silber}   \\
     2  & \emph{Kupfer}   \\
     3  & Gold   \\
     4  & Aluminium   \\
     5  & Wolfram   \\
     6  & Zink   \\
     7  & \emph{Zinn}   \\
\end{DARCtabular}
\caption{Einige leitende Materialien, sortiert von den besonders gut leitenden zu den weniger gut leitenden Materialen}
\label{leiter}
\end{table}

\end{frame}

\begin{frame}
\only<1>{
\begin{QQuestion}{NB101}{Welches der genannten Metalle hat bei Raumtemperatur die höchste elektrische Leitfähigkeit?}{Kupfer}
{Aluminium}
{Wolfram}
{Zink}
\end{QQuestion}

}
\only<2>{
\begin{QQuestion}{NB101}{Welches der genannten Metalle hat bei Raumtemperatur die höchste elektrische Leitfähigkeit?}{\textbf{\textcolor{DARCgreen}{Kupfer}}}
{Aluminium}
{Wolfram}
{Zink}
\end{QQuestion}

}
\end{frame}

\begin{frame}
\only<1>{
\begin{QQuestion}{NB102}{Welches der genannten Metalle hat bei Raumtemperatur die höchste elektrische Leitfähigkeit?}{Gold}
{Kupfer}
{Silber}
{Zinn}
\end{QQuestion}

}
\only<2>{
\begin{QQuestion}{NB102}{Welches der genannten Metalle hat bei Raumtemperatur die höchste elektrische Leitfähigkeit?}{Gold}
{Kupfer}
{\textbf{\textcolor{DARCgreen}{Silber}}}
{Zinn}
\end{QQuestion}

}
\end{frame}

\begin{frame}
\only<1>{
\begin{QQuestion}{NB103}{Welches der genannten Metalle hat bei Raumtemperatur die schlechteste elektrische Leitfähigkeit?}{Aluminium}
{Kupfer}
{Gold}
{Zinn}
\end{QQuestion}

}
\only<2>{
\begin{QQuestion}{NB103}{Welches der genannten Metalle hat bei Raumtemperatur die schlechteste elektrische Leitfähigkeit?}{Aluminium}
{Kupfer}
{Gold}
{\textbf{\textcolor{DARCgreen}{Zinn}}}
\end{QQuestion}

}
\end{frame}

\begin{frame}
\frametitle{Nichtleiter}
\begin{itemize}
  \item Leiten keinen elektrischen Strom
  \item Auch \emph{Isolatoren} genannt
  \end{itemize}
\end{frame}

\begin{frame}
\frametitle{Isolatoren}
\begin{table}
\begin{DARCtabular}{ll}
     Bezeichnung  & Abkürzung   \\
     \emph{Porzellan}  &   \\
     \emph{Polyethylen}  & \emph{PE}   \\
     \emph{Polystyrol}  & \emph{PS}   \\
     Kork  &   \\
     Polyvinylchlorid  & PVC   \\
     Polytetrafluorethylen  & PTFE   \\
\end{DARCtabular}
\caption{Einige nicht-leitende Materialien}
\label{nichtleiter}
\end{table}

\end{frame}

\begin{frame}
\only<1>{
\begin{QQuestion}{NB104}{Die Materialien welcher Gruppe sind bei Raumtemperatur alle Nichtleiter (Isolatoren)?}{Porzellan, Polyethylen (PE), Polystyrol (PS)}
{Polytetrafluorethylen (PTFE), Polyvinylchlorid (PVC), Wolfram}
{Polystyrol (PS), Messing, Kork}
{Porzellan, Polyethylen (PE), Bronze}
\end{QQuestion}

}
\only<2>{
\begin{QQuestion}{NB104}{Die Materialien welcher Gruppe sind bei Raumtemperatur alle Nichtleiter (Isolatoren)?}{\textbf{\textcolor{DARCgreen}{Porzellan, Polyethylen (PE), Polystyrol (PS)}}}
{Polytetrafluorethylen (PTFE), Polyvinylchlorid (PVC), Wolfram}
{Polystyrol (PS), Messing, Kork}
{Porzellan, Polyethylen (PE), Bronze}
\end{QQuestion}

}
\end{frame}%ENDCONTENT
