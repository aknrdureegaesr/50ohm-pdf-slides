
\section{Verkürzungsfaktor I}
\label{section:verkuerzungsfaktor_1}
\begin{frame}%STARTCONTENT

\begin{columns}
    \begin{column}{0.48\textwidth}
    Wellenausbreitung in Luft und Vakuum:

$\lambda = \dfrac{c}{f}$


    \end{column}
   \begin{column}{0.48\textwidth}
       \begin{itemize}
  \item Leitungen und Antennendrähte benötigen einen Korrekturfaktor
  \item Den \emph{Verkürzungsfaktor} $k_\mathrm{v}$
  \item In etwa \qty{95}{\percent} zur Vakuumausbreitung
  \item $\lambda_\mathrm{Leitung} = k_\mathrm{v} \cdot \dfrac{c}{f}$
  \end{itemize}

   \end{column}
\end{columns}

\end{frame}

\begin{frame}
\only<1>{
\begin{QQuestion}{EG201}{Der Verkürzungsfaktor ist~...}{das Verhältnis des Leiterwiderstandes zum Fußpunktwiderstand der Antenne.}
{das Verhältnis von Durchmesser zur Länge eines Leiters.}
{das Verhältnis der Ausbreitungsgeschwindigkeit entlang einer Leitung zur Ausbreitungsgeschwindigkeit im Vakuum.}
{die Wurzel aus dem Verhältnis von Induktivität zur Kapazität einer Leitung.}
\end{QQuestion}

}
\only<2>{
\begin{QQuestion}{EG201}{Der Verkürzungsfaktor ist~...}{das Verhältnis des Leiterwiderstandes zum Fußpunktwiderstand der Antenne.}
{das Verhältnis von Durchmesser zur Länge eines Leiters.}
{\textbf{\textcolor{DARCgreen}{das Verhältnis der Ausbreitungsgeschwindigkeit entlang einer Leitung zur Ausbreitungsgeschwindigkeit im Vakuum.}}}
{die Wurzel aus dem Verhältnis von Induktivität zur Kapazität einer Leitung.}
\end{QQuestion}

}
\end{frame}

\begin{frame}\begin{itemize}
  \item Korrekturfaktor hängt von Drahtdurchmesser, Isolierung und Umgebungseinflüssen ab
  \item Bei Drahtantennen sind diese für Resonanz um ca. \qty{5}{\percent} zu kürzen
  \end{itemize}
\end{frame}

\begin{frame}
\only<1>{
\begin{QQuestion}{EG202}{Welcher Prozentsatz entspricht dem Verkürzungsfaktor (Korrekturfaktor), der üblicherweise für die Berechnung der Länge einer Drahtantenne verwendet wird?}{\qty{75}{\percent}}
{\qty{95}{\percent}}
{\qty{66}{\percent}}
{\qty{100}{\percent}}
\end{QQuestion}

}
\only<2>{
\begin{QQuestion}{EG202}{Welcher Prozentsatz entspricht dem Verkürzungsfaktor (Korrekturfaktor), der üblicherweise für die Berechnung der Länge einer Drahtantenne verwendet wird?}{\qty{75}{\percent}}
{\textbf{\textcolor{DARCgreen}{\qty{95}{\percent}}}}
{\qty{66}{\percent}}
{\qty{100}{\percent}}
\end{QQuestion}

}
\end{frame}%ENDCONTENT
