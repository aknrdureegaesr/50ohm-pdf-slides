
\section{Frequenzmodulation (FM) III}
\label{section:fm_3}
\begin{frame}%STARTCONTENT

\only<1>{
\begin{QQuestion}{AE301}{Wie beeinflusst die Frequenz eines sinusförmigen Modulationssignals den HF-Träger bei Frequenzmodulation?}{Wie weit sich die Trägerfrequenz ändert.}
{Wie schnell sich die Trägeramplitude ändert.}
{In welcher Häufigkeit sich der HF-Träger ändert.
}
{Wie weit sich die Trägeramplitude ändert.}
\end{QQuestion}

}
\only<2>{
\begin{QQuestion}{AE301}{Wie beeinflusst die Frequenz eines sinusförmigen Modulationssignals den HF-Träger bei Frequenzmodulation?}{Wie weit sich die Trägerfrequenz ändert.}
{Wie schnell sich die Trägeramplitude ändert.}
{\textbf{\textcolor{DARCgreen}{In welcher Häufigkeit sich der HF-Träger ändert.
}}}
{Wie weit sich die Trägeramplitude ändert.}
\end{QQuestion}

}
\end{frame}

\begin{frame}
\only<1>{
\begin{QQuestion}{AE302}{Welches der nachfolgenden Übertragungsverfahren weist die geringste Störanfälligkeit gegenüber Impulsstörungen durch Funkenbildung in Elektromotoren auf?}{AM-Sprechfunk, weil hier die wichtige Information in den Amplituden der beiden Seitenbänder enthalten ist.}
{CW-Morsetelegrafie, weil hier die wichtige Information in der Amplitude von zwei Seitenbändern liegt.}
{SSB-Sprechfunk, weil hier die wichtige Information in der Amplitude eines Seitenbandes enthalten ist.}
{FM-Sprechfunk, weil hier die wichtige Information nicht in der Amplitude enthalten ist.}
\end{QQuestion}

}
\only<2>{
\begin{QQuestion}{AE302}{Welches der nachfolgenden Übertragungsverfahren weist die geringste Störanfälligkeit gegenüber Impulsstörungen durch Funkenbildung in Elektromotoren auf?}{AM-Sprechfunk, weil hier die wichtige Information in den Amplituden der beiden Seitenbänder enthalten ist.}
{CW-Morsetelegrafie, weil hier die wichtige Information in der Amplitude von zwei Seitenbändern liegt.}
{SSB-Sprechfunk, weil hier die wichtige Information in der Amplitude eines Seitenbandes enthalten ist.}
{\textbf{\textcolor{DARCgreen}{FM-Sprechfunk, weil hier die wichtige Information nicht in der Amplitude enthalten ist.}}}
\end{QQuestion}

}
\end{frame}

\begin{frame}
\only<1>{
\begin{QQuestion}{AE303}{Eine Quarzoszillator-Schaltung mit Kapazitätsdiode ermöglicht es~...}{Zweiseitenbandmodulation zu erzeugen.}
{Frequenzmodulation zu erzeugen.}
{Einseitenbandmodulation zu erzeugen.}
{Amplitudenmodulation zu erzeugen.}
\end{QQuestion}

}
\only<2>{
\begin{QQuestion}{AE303}{Eine Quarzoszillator-Schaltung mit Kapazitätsdiode ermöglicht es~...}{Zweiseitenbandmodulation zu erzeugen.}
{\textbf{\textcolor{DARCgreen}{Frequenzmodulation zu erzeugen.}}}
{Einseitenbandmodulation zu erzeugen.}
{Amplitudenmodulation zu erzeugen.}
\end{QQuestion}

}
\end{frame}

\begin{frame}
\only<1>{
\begin{QQuestion}{AE304}{Eine zu hohe Modulationsfrequenz eines FM-Senders führt dazu,~...}{dass die Sendeendstufe übersteuert wird.}
{dass die HF-Bandbreite zu groß wird.}
{dass Verzerrungen auf Grund unerwünschter Unterdrückung der Trägerfrequenz auftreten.}
{dass Verzerrungen auf Grund gegenseitiger Auslöschung der Seitenbänder auftreten.}
\end{QQuestion}

}
\only<2>{
\begin{QQuestion}{AE304}{Eine zu hohe Modulationsfrequenz eines FM-Senders führt dazu,~...}{dass die Sendeendstufe übersteuert wird.}
{\textbf{\textcolor{DARCgreen}{dass die HF-Bandbreite zu groß wird.}}}
{dass Verzerrungen auf Grund unerwünschter Unterdrückung der Trägerfrequenz auftreten.}
{dass Verzerrungen auf Grund gegenseitiger Auslöschung der Seitenbänder auftreten.}
\end{QQuestion}

}
\end{frame}

\begin{frame}
\only<1>{
\begin{QQuestion}{AE305}{Was bewirkt die Erhöhung des Hubes eines frequenzmodulierten Senders im Empfänger?}{Eine geringere Lautstärke}
{Eine größere Sprachkomprimierung}
{Eine Verringerung des Signal-Rausch-Abstandes}
{Eine größere Lautstärke}
\end{QQuestion}

}
\only<2>{
\begin{QQuestion}{AE305}{Was bewirkt die Erhöhung des Hubes eines frequenzmodulierten Senders im Empfänger?}{Eine geringere Lautstärke}
{Eine größere Sprachkomprimierung}
{Eine Verringerung des Signal-Rausch-Abstandes}
{\textbf{\textcolor{DARCgreen}{Eine größere Lautstärke}}}
\end{QQuestion}

}
\end{frame}

\begin{frame}
\only<1>{
\begin{QQuestion}{AE306}{Eine FM-Telefonie-Aussendung mit zu großem Hub führt möglicherweise~...}{zu unerwünschter Begrenzung des Trägerfrequenzsignals.}
{zur Verminderung der Ausgangsleistung.}
{zu Nachbarkanalstörungen.}
{zur Auslöschung der Seitenbänder.}
\end{QQuestion}

}
\only<2>{
\begin{QQuestion}{AE306}{Eine FM-Telefonie-Aussendung mit zu großem Hub führt möglicherweise~...}{zu unerwünschter Begrenzung des Trägerfrequenzsignals.}
{zur Verminderung der Ausgangsleistung.}
{\textbf{\textcolor{DARCgreen}{zu Nachbarkanalstörungen.}}}
{zur Auslöschung der Seitenbänder.}
\end{QQuestion}

}
\end{frame}

\begin{frame}
\only<1>{
\begin{QQuestion}{AE307}{Zu starke Ansteuerung des Modulators führt bei Frequenzmodulation zur~...}{Überlastung des Netzteils.}
{Übersteuerung der HF-Endstufe.}
{Verzerrung des HF-Sendesignals.}
{Erhöhung der HF-Bandbreite.}
\end{QQuestion}

}
\only<2>{
\begin{QQuestion}{AE307}{Zu starke Ansteuerung des Modulators führt bei Frequenzmodulation zur~...}{Überlastung des Netzteils.}
{Übersteuerung der HF-Endstufe.}
{Verzerrung des HF-Sendesignals.}
{\textbf{\textcolor{DARCgreen}{Erhöhung der HF-Bandbreite.}}}
\end{QQuestion}

}
\end{frame}

\begin{frame}
\only<1>{
\begin{QQuestion}{AE308}{Wie groß ist die Bandbreite eines FM-Signals bei einer Modulationsfrequenz von \qty{2,7}{\kHz} und einem Hub von \qty{2,5}{\kHz} nach der Carson-Formel?}{\qty{5,5}{\kHz}}
{\qty{12,5}{\kHz}}
{\qty{10,4}{\kHz}}
{\qty{2,5}{\kHz}}
\end{QQuestion}

}
\only<2>{
\begin{QQuestion}{AE308}{Wie groß ist die Bandbreite eines FM-Signals bei einer Modulationsfrequenz von \qty{2,7}{\kHz} und einem Hub von \qty{2,5}{\kHz} nach der Carson-Formel?}{\qty{5,5}{\kHz}}
{\qty{12,5}{\kHz}}
{\textbf{\textcolor{DARCgreen}{\qty{10,4}{\kHz}}}}
{\qty{2,5}{\kHz}}
\end{QQuestion}

}
\end{frame}

\begin{frame}
\frametitle{Lösungsweg}
\begin{itemize}
  \item gegeben: $f_{mod max} = 2,7kHz$
  \item gegeben: $\Delta f_T = 2,5kHz$
  \item gesucht: $B$
  \end{itemize}
    \pause
    $B \approx 2 \cdot (\Delta f_T + f_{mod max}) = 2 \cdot (2,5kHz + 2,7kHz) = 10,4kHz$



\end{frame}

\begin{frame}
\only<1>{
\begin{QQuestion}{AE309}{Ein Träger von \qty{145}{\MHz} wird mit der NF-Frequenz von \qty{2}{\kHz} und einem Hub von \qty{1,8}{\kHz} frequenzmoduliert. Welche Bandbreite hat das modulierte Signal ungefähr? Die Bandbreite beträgt ungefähr~...}{\qty{12}{\kHz}}
{\qty{3,8}{\kHz}}
{\qty{5,8}{\kHz}}
{\qty{7,6}{\kHz}}
\end{QQuestion}

}
\only<2>{
\begin{QQuestion}{AE309}{Ein Träger von \qty{145}{\MHz} wird mit der NF-Frequenz von \qty{2}{\kHz} und einem Hub von \qty{1,8}{\kHz} frequenzmoduliert. Welche Bandbreite hat das modulierte Signal ungefähr? Die Bandbreite beträgt ungefähr~...}{\qty{12}{\kHz}}
{\qty{3,8}{\kHz}}
{\qty{5,8}{\kHz}}
{\textbf{\textcolor{DARCgreen}{\qty{7,6}{\kHz}}}}
\end{QQuestion}

}
\end{frame}

\begin{frame}
\frametitle{Lösungsweg}
\begin{itemize}
  \item gegeben: $f_{mod max} = 2kHz$
  \item gegeben: $\Delta f_T = 1,8kHz$
  \item gesucht: $B$
  \end{itemize}
    \pause
    $B \approx 2 \cdot (\Delta f_T + f_{mod max}) = 2 \cdot (1,8kHz + 2kHz) = 7,6kHz$



\end{frame}

\begin{frame}
\only<1>{
\begin{QQuestion}{AE310}{Der typische Spitzenhub eines NBFM-Signals im \qty{12,5}{\kHz} Kanalraster beträgt~...}{\qty{25}{\kHz}.}
{\qty{2,5}{\kHz}.}
{\qty{6,25}{\kHz}.}
{\qty{12,5}{\kHz}.}
\end{QQuestion}

}
\only<2>{
\begin{QQuestion}{AE310}{Der typische Spitzenhub eines NBFM-Signals im \qty{12,5}{\kHz} Kanalraster beträgt~...}{\qty{25}{\kHz}.}
{\textbf{\textcolor{DARCgreen}{\qty{2,5}{\kHz}.}}}
{\qty{6,25}{\kHz}.}
{\qty{12,5}{\kHz}.}
\end{QQuestion}

}
\end{frame}

\begin{frame}
\only<1>{
\begin{QQuestion}{AE311}{Die Bandbreite eines FM-Signals soll \qty{10}{\kHz} nicht überschreiten. Der Hub beträgt \qty{2,5}{\kHz}. Wie groß ist dabei die höchste Modulationsfrequenz?}{\qty{3}{\kHz}}
{\qty{1,5}{\kHz}}
{\qty{2,5}{\kHz}}
{\qty{2}{\kHz}}
\end{QQuestion}

}
\only<2>{
\begin{QQuestion}{AE311}{Die Bandbreite eines FM-Signals soll \qty{10}{\kHz} nicht überschreiten. Der Hub beträgt \qty{2,5}{\kHz}. Wie groß ist dabei die höchste Modulationsfrequenz?}{\qty{3}{\kHz}}
{\qty{1,5}{\kHz}}
{\textbf{\textcolor{DARCgreen}{\qty{2,5}{\kHz}}}}
{\qty{2}{\kHz}}
\end{QQuestion}

}
\end{frame}

\begin{frame}
\frametitle{Lösungsweg}
\begin{itemize}
  \item gegeben: $B = 10kHz$
  \item gegeben: $\Delta f_T = 2,5kHz$
  \item gesucht: $f_{mod max}$
  \end{itemize}
    \pause
    $B \approx 2 \cdot (\Delta f_T + f_{mod max}) \Rightarrow f_{mod max} = \frac{B}{2} -- \Delta f_T = \frac{10kHz}{2} -- 2,5kHz = 2,5kHz$



\end{frame}

\begin{frame}
\only<1>{
\begin{QQuestion}{AE312}{Die Bandbreite eines FM-Senders soll \qty{10}{\kHz} nicht überschreiten. Wie hoch darf der Frequenzhub bei einer Modulationsfrequenz von \qty{2,7}{\kHz} maximal sein?}{\qty{4,6}{\kHz}}
{\qty{7,7}{\kHz}}
{\qty{2,3}{\kHz}}
{\qty{12,7}{\kHz}}
\end{QQuestion}

}
\only<2>{
\begin{QQuestion}{AE312}{Die Bandbreite eines FM-Senders soll \qty{10}{\kHz} nicht überschreiten. Wie hoch darf der Frequenzhub bei einer Modulationsfrequenz von \qty{2,7}{\kHz} maximal sein?}{\qty{4,6}{\kHz}}
{\qty{7,7}{\kHz}}
{\textbf{\textcolor{DARCgreen}{\qty{2,3}{\kHz}}}}
{\qty{12,7}{\kHz}}
\end{QQuestion}

}
\end{frame}

\begin{frame}
\frametitle{Lösungsweg}
\begin{itemize}
  \item gegeben: $B = 10kHz$
  \item gegeben: $f_{mod max} = 2,7kHz$
  \item gesucht: $\Delta f_T$
  \end{itemize}
    \pause
    $B \approx 2 \cdot (\Delta f_T + f_{mod max}) \Rightarrow \Delta f_T = \frac{B}{2} -- f_{mod max} = \frac{10kHz}{2} -- 2,7kHz = 2,3kHz$



\end{frame}%ENDCONTENT
