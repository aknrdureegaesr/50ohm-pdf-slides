
\section{Vorverstärker und Dämpfungsglied}
\label{section:vorverstaerker_daempfungsglied}
\begin{frame}%STARTCONTENT

\frametitle{Dämpfungsglied}
\begin{columns}
    \begin{column}{0.48\textwidth}
    \begin{itemize}
  \item Kurzwellenempfänger können durch starke Signale übersteuern
  \item Insbesondere im Empfangsbereich und 1. Mischer
  \item Verzerrrte und unverständliche Wiedergabe der Signale
  \end{itemize}

    \end{column}
   \begin{column}{0.48\textwidth}
       \begin{itemize}
  \item Zuschalten von Abschwächer (\emph{Attenuator}) im TRX
  \item Dämpfen der Eingangssignale um einen vorgegebenen Wert
  \end{itemize}

   \end{column}
\end{columns}

\end{frame}

\begin{frame}
\only<1>{
\begin{QQuestion}{EF217}{Welche Baugruppe vermindert die Übersteuerung eines Empfängereingangs?}{ZF-Filter}
{Dämpfungsglied}
{Rauschsperre}
{Oszillator}
\end{QQuestion}

}
\only<2>{
\begin{QQuestion}{EF217}{Welche Baugruppe vermindert die Übersteuerung eines Empfängereingangs?}{ZF-Filter}
{\textbf{\textcolor{DARCgreen}{Dämpfungsglied}}}
{Rauschsperre}
{Oszillator}
\end{QQuestion}

}
\end{frame}

\begin{frame}
\frametitle{Vorverstärker}
\begin{columns}
    \begin{column}{0.48\textwidth}
    \begin{itemize}
  \item Hohe Signale (UHF und höher) werden durch Antennenleitung abgeschwächt
  \item \emph{Vorverstärker} direkt an Empfangsantenne montieren
  \end{itemize}

    \end{column}
   \begin{column}{0.48\textwidth}
       \begin{itemize}
  \item Kabelverluste werden ausgeglichen
  \item Abschaltbar im Sendefall
  \item Deaktivierbar bei starken lokalen Signalen
  \end{itemize}

   \end{column}
\end{columns}

\end{frame}

\begin{frame}
\only<1>{
\begin{QQuestion}{EF218}{An welcher Stelle einer Amateurfunkanlage sollte ein UHF-Vorverstärker eingefügt werden?}{Zwischen Stehwellenmessgerät und Empfängereingang}
{Möglichst unmittelbar vor dem Empfängereingang}
{Zwischen Senderausgang und Antennenkabel}
{Möglichst direkt an der UHF-Antenne}
\end{QQuestion}

}
\only<2>{
\begin{QQuestion}{EF218}{An welcher Stelle einer Amateurfunkanlage sollte ein UHF-Vorverstärker eingefügt werden?}{Zwischen Stehwellenmessgerät und Empfängereingang}
{Möglichst unmittelbar vor dem Empfängereingang}
{Zwischen Senderausgang und Antennenkabel}
{\textbf{\textcolor{DARCgreen}{Möglichst direkt an der UHF-Antenne}}}
\end{QQuestion}

}
\end{frame}%ENDCONTENT
