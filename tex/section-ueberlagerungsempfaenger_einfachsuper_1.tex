
\section{Überlagerungsempfänger (Einfachsuper) I}
\label{section:ueberlagerungsempfaenger_einfachsuper_1}
\begin{frame}%STARTCONTENT

\begin{columns}
    \begin{column}{0.48\textwidth}
    \begin{itemize}
  \item Mischprozess mit einer Oszillatorfrequenz
  \item Konstante \emph{Zwischenfrequenz}
  \item Bessere \emph{Selektivität} des Empfangssignals
  \item Filter müssen nicht veränderlich sein
  \item Höhere \emph{Trennschärfe}
  \end{itemize}

    \end{column}
   \begin{column}{0.48\textwidth}
       
\begin{figure}
    \DARCimage{0.85\linewidth}{803include}
    \caption{\scriptsize Überlagerungsempfänger (Einfachsuper); die notwendigen Filter sind der Übersichtlichkeit halber in den Verstärkerstufen enthalten}
    \label{ueberlagerungsempfaenger_einfachsuper}
\end{figure}


   \end{column}
\end{columns}

\end{frame}

\begin{frame}
\only<1>{
\begin{QQuestion}{EF102}{Welchen Vorteil bietet ein Überlagerungsempfänger gegenüber einem Geradeaus-Empfänger?}{Höhere Bandbreiten}
{Bessere Trennschärfe}
{Geringere Anforderungen an die VFO-Stabilität}
{Wesentlich einfachere Konstruktion}
\end{QQuestion}

}
\only<2>{
\begin{QQuestion}{EF102}{Welchen Vorteil bietet ein Überlagerungsempfänger gegenüber einem Geradeaus-Empfänger?}{Höhere Bandbreiten}
{\textbf{\textcolor{DARCgreen}{Bessere Trennschärfe}}}
{Geringere Anforderungen an die VFO-Stabilität}
{Wesentlich einfachere Konstruktion}
\end{QQuestion}

}
\end{frame}

\begin{frame}
\frametitle{Zwischenfrequenz}
\begin{itemize}
  \item Eine Frequenz, auf die für die weitere Verarbeitung gemischt wird
  \item Im einfachsten Fall die NF $\rightarrow$ Direktüberlagerungsempfänger
  \item Bei NF muss die Oszillatorfrequenz nahe der Empfangsfrequenz sein
  \item Klasse~A behandelt Mehrfachsuper-Empfänger mit mehreren Zwischenfrequenzen
  \end{itemize}
\end{frame}

\begin{frame}
\only<1>{
\begin{QQuestion}{EF208}{Wo liegt bei einem Direktüberlagerungsempfänger üblicherweise die Oszillatorfrequenz für den Mischer?}{Sie liegt bei der Zwischenfrequenz. }
{Sie liegt sehr weit über der Empfangsfrequenz.}
{Sie liegt sehr viel tiefer als die Empfangsfrequenz.}
{Sie liegt in nächster Nähe zur Empfangsfrequenz.}
\end{QQuestion}

}
\only<2>{
\begin{QQuestion}{EF208}{Wo liegt bei einem Direktüberlagerungsempfänger üblicherweise die Oszillatorfrequenz für den Mischer?}{Sie liegt bei der Zwischenfrequenz. }
{Sie liegt sehr weit über der Empfangsfrequenz.}
{Sie liegt sehr viel tiefer als die Empfangsfrequenz.}
{\textbf{\textcolor{DARCgreen}{Sie liegt in nächster Nähe zur Empfangsfrequenz.}}}
\end{QQuestion}

}
\end{frame}%ENDCONTENT
