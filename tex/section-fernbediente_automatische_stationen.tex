
\section{Fernbediente und automatische Stationen}
\label{section:fernbediente_automatische_stationen}
\begin{frame}%STARTCONTENT

\frametitle{Normalerweise}
\begin{itemize}
  \item Funkamateur muss die Station besetzt betreiben
  \item Aussendungen dürfen nur unter Aufsicht erfolgen
  \item Direkt an der Sendeanlage oder mittelbar via Remote-Station
  \end{itemize}
    \pause
    Ausnahme: Fernbediente und automatische Stationen



\end{frame}

\begin{frame}
\frametitle{Relaisfunkstelle}
\begin{itemize}
  \item Ermöglicht Funverbindungen zwischen Funkamateuren, die sich nicht direkt erreichen können
  \item Sendet alles, was sie auf einer Frequenz empfängt, auf einer anderen wieder aus
  \end{itemize}

\end{frame}

\begin{frame}
\frametitle{Bake}
\begin{itemize}
  \item Sendet nur immer das gleiche
  \item In regelmäßigen Abständen
  \item Oftmals nur das Rufzeichen
  \item Zur Untersuchung der Ausbreitungsbedingungen
  \end{itemize}
\end{frame}

\begin{frame}
\only<1>{
\begin{QQuestion}{VD501}{Was ist notwendig, damit ein Funkamateur eine Amateurfunkstelle als Relaisfunkstelle oder Funkbake betreiben darf?}{Es ist eine Zulassung der höchsten Amateurfunkklasse erforderlich.}
{Es bedarf einer Rufzeichenzuteilung für den Betrieb einer fernbedienten oder automatisch arbeitenden Amateurfunkstelle.}
{Für den Betrieb einer Relaisfunkstelle oder Funkbake ist der mindestens 2-jährige Besitz einer gültigen Amateurfunkzulassung erforderlich.}
{Es sind keine besonderen Bedingungen zu erfüllen.}
\end{QQuestion}

}
\only<2>{
\begin{QQuestion}{VD501}{Was ist notwendig, damit ein Funkamateur eine Amateurfunkstelle als Relaisfunkstelle oder Funkbake betreiben darf?}{Es ist eine Zulassung der höchsten Amateurfunkklasse erforderlich.}
{\textbf{\textcolor{DARCgreen}{Es bedarf einer Rufzeichenzuteilung für den Betrieb einer fernbedienten oder automatisch arbeitenden Amateurfunkstelle.}}}
{Für den Betrieb einer Relaisfunkstelle oder Funkbake ist der mindestens 2-jährige Besitz einer gültigen Amateurfunkzulassung erforderlich.}
{Es sind keine besonderen Bedingungen zu erfüllen.}
\end{QQuestion}

}
\end{frame}

\begin{frame}
\only<1>{
\begin{QQuestion}{VD502}{Unter welchen Voraussetzungen darf ein Funkamateur eine Amateurfunkstelle als Relaisfunkstelle betreiben?}{Wenn die Relaisfunkstelle keine große Reichweite hat}
{Wenn er für die Relaisfunkstelle eine Rufzeichenzuteilung besitzt und die darin festgelegten Rahmenbedingungen einhält}
{Wenn er mindestens 20 Unterschriften als Beweis der Notwendigkeit vorlegen kann und die Rahmenbedingungen für Relaisfunkstellen einhält}
{Wenn er die technischen Einrichtungen dafür selbst instand halten kann}
\end{QQuestion}

}
\only<2>{
\begin{QQuestion}{VD502}{Unter welchen Voraussetzungen darf ein Funkamateur eine Amateurfunkstelle als Relaisfunkstelle betreiben?}{Wenn die Relaisfunkstelle keine große Reichweite hat}
{\textbf{\textcolor{DARCgreen}{Wenn er für die Relaisfunkstelle eine Rufzeichenzuteilung besitzt und die darin festgelegten Rahmenbedingungen einhält}}}
{Wenn er mindestens 20 Unterschriften als Beweis der Notwendigkeit vorlegen kann und die Rahmenbedingungen für Relaisfunkstellen einhält}
{Wenn er die technischen Einrichtungen dafür selbst instand halten kann}
\end{QQuestion}

}
\end{frame}%ENDCONTENT
