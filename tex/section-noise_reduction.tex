
\section{Rauschunterdrückung}
\label{section:noise_reduction}
\begin{frame}%STARTCONTENT

\begin{columns}
    \begin{column}{0.48\textwidth}
    \begin{itemize}
  \item Empfangssignal gestört durch Rauschen oder Impulse
  \item Schwaches Signal mit Rauschanteilen
  \item Zündfunken, Schaltnetzteile, Maschinen etc.
  \end{itemize}

    \end{column}
   \begin{column}{0.48\textwidth}
       \begin{itemize}
  \item Noise Reduction (\emph{NR})
  \item $\rightarrow$ Aktive Differenzierung von Nutzsignal und Rauschen
  \item Noise Blanker (\emph{NB})
  \item $\rightarrow$ Tastet impulsförmige Störungen aus
  \end{itemize}

   \end{column}
\end{columns}

\end{frame}

\begin{frame}
\only<1>{
\begin{QQuestion}{EF213}{Welche Aufgabe hat das Rauschunterdrückungsverfahren (Noise Reduction) in einem Empfänger?}{Verringerung des Rauschanteils in der Versorgungsspannung}
{Verringerung des Rauschanteils im Signal}
{Verringerung der Umgebungsgeräusche im Kopfhörer}
{Verringerung des Dynamikbereichs im ZF-Signal}
\end{QQuestion}

}
\only<2>{
\begin{QQuestion}{EF213}{Welche Aufgabe hat das Rauschunterdrückungsverfahren (Noise Reduction) in einem Empfänger?}{Verringerung des Rauschanteils in der Versorgungsspannung}
{\textbf{\textcolor{DARCgreen}{Verringerung des Rauschanteils im Signal}}}
{Verringerung der Umgebungsgeräusche im Kopfhörer}
{Verringerung des Dynamikbereichs im ZF-Signal}
\end{QQuestion}

}
\end{frame}

\begin{frame}
\only<1>{
\begin{QQuestion}{EF214}{Welche Baugruppe könnte in einem Empfänger gegebenenfalls dazu verwendet werden, impulsförmige Störungen auszublenden?}{Passband Tuning}
{Notch Filter}
{Noise Blanker}
{Automatic Gain Control}
\end{QQuestion}

}
\only<2>{
\begin{QQuestion}{EF214}{Welche Baugruppe könnte in einem Empfänger gegebenenfalls dazu verwendet werden, impulsförmige Störungen auszublenden?}{Passband Tuning}
{Notch Filter}
{\textbf{\textcolor{DARCgreen}{Noise Blanker}}}
{Automatic Gain Control}
\end{QQuestion}

}
\end{frame}%ENDCONTENT
