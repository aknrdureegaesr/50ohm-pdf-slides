
\section{Kollektorschaltung}
\label{section:kollektorschaltung}
\begin{frame}%STARTCONTENT

\only<1>{
\begin{PQuestion}{AD401}{Bei dieser Schaltung handelt es sich um~...}{einen Oszillator in Kollektorschaltung.}
{einen Verstärker in Emitterschaltung.}
{einen Verstärker in Kollektorschaltung.}
{einen Oszillator in Emitterschaltung.}
{\DARCimage{1.0\linewidth}{140include}}\end{PQuestion}

}
\only<2>{
\begin{PQuestion}{AD401}{Bei dieser Schaltung handelt es sich um~...}{einen Oszillator in Kollektorschaltung.}
{einen Verstärker in Emitterschaltung.}
{\textbf{\textcolor{DARCgreen}{einen Verstärker in Kollektorschaltung.}}}
{einen Oszillator in Emitterschaltung.}
{\DARCimage{1.0\linewidth}{140include}}\end{PQuestion}

}
\end{frame}

\begin{frame}
\only<1>{
\begin{QQuestion}{AD405}{Welche Phasenverschiebung tritt zwischen den sinusförmigen Ein- und Ausgangsspannungen eines Transistorverstärkers in Kollektorschaltung auf?}{\qty{0}{\degree}}
{\qty{90}{\degree}}
{\qty{180}{\degree}}
{\qty{270}{\degree}}
\end{QQuestion}

}
\only<2>{
\begin{QQuestion}{AD405}{Welche Phasenverschiebung tritt zwischen den sinusförmigen Ein- und Ausgangsspannungen eines Transistorverstärkers in Kollektorschaltung auf?}{\textbf{\textcolor{DARCgreen}{\qty{0}{\degree}}}}
{\qty{90}{\degree}}
{\qty{180}{\degree}}
{\qty{270}{\degree}}
\end{QQuestion}

}
\end{frame}

\begin{frame}
\only<1>{
\begin{PQuestion}{AD402}{Was lässt sich über die Wechselspannungsverstärkung $v_U$ und die Phasenverschiebung $\varphi$ zwischen Ausgangs- und Eingangsspannung dieser Schaltung aussagen?}{$v_U$ ist groß (z.~B. 100~... 300) und $\varphi = \qty{0}{\degree}$.}
{$v_U$ ist klein (z.~B. 0,9~... 0,98) und $\varphi = \qty{0}{\degree}$.}
{$v_U$ ist klein (z.~B. 0,9~... 0,98) und $\varphi = \qty{180}{\degree}$.}
{$v_U$ ist groß (z.~B. 100~... 300) und $\varphi = \qty{180}{\degree}$.}
{\DARCimage{1.0\linewidth}{140include}}\end{PQuestion}

}
\only<2>{
\begin{PQuestion}{AD402}{Was lässt sich über die Wechselspannungsverstärkung $v_U$ und die Phasenverschiebung $\varphi$ zwischen Ausgangs- und Eingangsspannung dieser Schaltung aussagen?}{$v_U$ ist groß (z.~B. 100~... 300) und $\varphi = \qty{0}{\degree}$.}
{\textbf{\textcolor{DARCgreen}{$v_U$ ist klein (z.~B. 0,9~... 0,98) und $\varphi = \qty{0}{\degree}$.}}}
{$v_U$ ist klein (z.~B. 0,9~... 0,98) und $\varphi = \qty{180}{\degree}$.}
{$v_U$ ist groß (z.~B. 100~... 300) und $\varphi = \qty{180}{\degree}$.}
{\DARCimage{1.0\linewidth}{140include}}\end{PQuestion}

}
\end{frame}

\begin{frame}
\only<1>{
\begin{PQuestion}{AD403}{Die Ausgangsimpedanz dieser Schaltung ist~...}{in etwa gleich der Eingangsimpedanz und hochohmig.}
{in etwa gleich der Eingangsimpedanz und niederohmig.}
{sehr hoch im Vergleich zur Eingangsimpedanz.}
{sehr niedrig im Vergleich zur Eingangsimpedanz.}
{\DARCimage{1.0\linewidth}{140include}}\end{PQuestion}

}
\only<2>{
\begin{PQuestion}{AD403}{Die Ausgangsimpedanz dieser Schaltung ist~...}{in etwa gleich der Eingangsimpedanz und hochohmig.}
{in etwa gleich der Eingangsimpedanz und niederohmig.}
{sehr hoch im Vergleich zur Eingangsimpedanz.}
{\textbf{\textcolor{DARCgreen}{sehr niedrig im Vergleich zur Eingangsimpedanz.}}}
{\DARCimage{1.0\linewidth}{140include}}\end{PQuestion}

}
\end{frame}

\begin{frame}
\only<1>{
\begin{PQuestion}{AD404}{Diese Schaltung kann unter anderem als~...}{Pufferstufe zwischen Oszillator und Last verwendet werden.}
{Spannungsverstärker mit hoher Verstärkung verwendet werden.}
{Phasenumkehrstufe verwendet werden.}
{Frequenzvervielfacher verwendet werden.}
{\DARCimage{1.0\linewidth}{140include}}\end{PQuestion}

}
\only<2>{
\begin{PQuestion}{AD404}{Diese Schaltung kann unter anderem als~...}{\textbf{\textcolor{DARCgreen}{Pufferstufe zwischen Oszillator und Last verwendet werden.}}}
{Spannungsverstärker mit hoher Verstärkung verwendet werden.}
{Phasenumkehrstufe verwendet werden.}
{Frequenzvervielfacher verwendet werden.}
{\DARCimage{1.0\linewidth}{140include}}\end{PQuestion}

}
\end{frame}%ENDCONTENT
