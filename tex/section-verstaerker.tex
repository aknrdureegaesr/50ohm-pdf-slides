
\section{Verstärker}
\label{section:verstaerker}
\begin{frame}%STARTCONTENT
\begin{itemize}
  \item Mittels Transistoren lassen sich, abhängig von der Art der Schaltung, alle Arten von Signalen (Digital, NF oder HF) verstärken.
  \item Dabei ist die Ausgangsleistung gegenüber der Eingangsleistung größer.
  \item Es ist eine Spannungsquelle notwendig.
  \item Wir haben im Kapitel Transistor schon gesehen, wie das funktioniert.
  \end{itemize}
\end{frame}

\begin{frame}
\only<1>{
\begin{QQuestion}{ED401}{Was versteht man in der Elektronik unter Leistungsverstärkung?}{Die Ausgangsleistung ist gegenüber der Eingangsleistung größer und dazu ist eine Spannungsquelle notwendig.}
{Die Ausgangsleistung ist gegenüber der Eingangsleistung größer, obwohl keine  Spannungsquelle notwendig ist.}
{Die Ausgangsleistung ist gleich der Eingangsleistung, obwohl keine Spannungsquelle notwendig ist.}
{Die Ausgangsleistung ist gleich der Eingangsleistung, da eine Spannungsquelle notwendig ist.}
\end{QQuestion}

}
\only<2>{
\begin{QQuestion}{ED401}{Was versteht man in der Elektronik unter Leistungsverstärkung?}{\textbf{\textcolor{DARCgreen}{Die Ausgangsleistung ist gegenüber der Eingangsleistung größer und dazu ist eine Spannungsquelle notwendig.}}}
{Die Ausgangsleistung ist gegenüber der Eingangsleistung größer, obwohl keine  Spannungsquelle notwendig ist.}
{Die Ausgangsleistung ist gleich der Eingangsleistung, obwohl keine Spannungsquelle notwendig ist.}
{Die Ausgangsleistung ist gleich der Eingangsleistung, da eine Spannungsquelle notwendig ist.}
\end{QQuestion}

}
\end{frame}

\begin{frame}
\only<1>{
\begin{QQuestion}{ED403}{Für welchen Zweck werden HF-Leistungsverstärker eingesetzt?}{Filterung des Sendesignals}
{Modulation des Sendesignals}
{Mischung des Sendesignals}
{Anhebung des Sendesignals}
\end{QQuestion}

}
\only<2>{
\begin{QQuestion}{ED403}{Für welchen Zweck werden HF-Leistungsverstärker eingesetzt?}{Filterung des Sendesignals}
{Modulation des Sendesignals}
{Mischung des Sendesignals}
{\textbf{\textcolor{DARCgreen}{Anhebung des Sendesignals}}}
\end{QQuestion}

}
\end{frame}

\begin{frame}
\begin{columns}
    \begin{column}{0.48\textwidth}
    \begin{itemize}
  \item \emph{Linearität} bedeutet: Eine Verdoppelung des Eingangssignals muss zu einer Verdoppelung des Ausgangssignals führen
  \item Linearitätsabweichungen sind unerwünscht, weil sie zu Frequenzen führen, die im Originalsignal nicht vorhanden sind.
  \item Sie werden im NF-Bereich als \emph{Verzerrungen} wahrgenommen und im HF-Bereich als \emph{Oberwellen}
  \end{itemize}

    \end{column}
   \begin{column}{0.48\textwidth}
       
\begin{figure}
    \DARCimage{0.85\linewidth}{828include}
    \caption{\scriptsize Das Eingangssignal wird verstärkt. Bei Begrenzung durch fehlende Linearität wird das Ausgangssignal verformt.}
    \label{e_verstaerker_linearitaet}
\end{figure}


   \end{column}
\end{columns}

\end{frame}

\begin{frame}
\only<1>{
\begin{QQuestion}{EF403}{Wie ist die Ausgangsstufe eines SSB-Senders aufgebaut?  }{Als linearer Verstärker}
{Als Begrenzerverstärker}
{Als nichtlinearer Verstärker}
{Als Vervielfacher}
\end{QQuestion}

}
\only<2>{
\begin{QQuestion}{EF403}{Wie ist die Ausgangsstufe eines SSB-Senders aufgebaut?  }{\textbf{\textcolor{DARCgreen}{Als linearer Verstärker}}}
{Als Begrenzerverstärker}
{Als nichtlinearer Verstärker}
{Als Vervielfacher}
\end{QQuestion}

}
\end{frame}

\begin{frame}
\begin{columns}
    \begin{column}{0.48\textwidth}
    \begin{itemize}
  \item NF-Verstärker finden im Amateurfunk zum Beispiel bei der Anhebung des Signals für eine Ausgabe im Lautsprecher Anwendung
  \end{itemize}

    \end{column}
   \begin{column}{0.48\textwidth}
       
\begin{figure}
    \DARCimage{0.85\linewidth}{763include}
    \caption{\scriptsize Schaltbild eines NF-Verstärkers}
    \label{e_nf_verstaerker}
\end{figure}


   \end{column}
\end{columns}

\end{frame}

\begin{frame}
\only<1>{
\begin{PQuestion}{ED402}{Worum handelt es sich bei dieser Schaltung?}{Tongenerator}
{ZF-Verstärker}
{HF-Verstärker}
{NF-Verstärker}
{\DARCimage{1.0\linewidth}{763include}}\end{PQuestion}

}
\only<2>{
\begin{PQuestion}{ED402}{Worum handelt es sich bei dieser Schaltung?}{Tongenerator}
{ZF-Verstärker}
{HF-Verstärker}
{\textbf{\textcolor{DARCgreen}{NF-Verstärker}}}
{\DARCimage{1.0\linewidth}{763include}}\end{PQuestion}

}
\end{frame}

\begin{frame}
\begin{columns}
    \begin{column}{0.48\textwidth}
    \begin{itemize}
  \item Auch bei der Verstärkung des Mikrofonsignals findet man Verstärker
  \item Verstärkung im Bereich von ca. 300 bis 3.\qty{000}{\hertz}
  \item Die Bandbreite liegt bei \qty{2,7}{\kilo\hertz} oder darunter
  \end{itemize}

    \end{column}
   \begin{column}{0.48\textwidth}
       
\begin{figure}
    \DARCimage{0.85\linewidth}{246include}
    \caption{\scriptsize Typischer Frequenzgang für einen Amateurfunk Mikrofonverstärker}
    \label{e_frequenzgang_mikrofonverstaerker}
\end{figure}


   \end{column}
\end{columns}

\end{frame}

\begin{frame}
\only<1>{
\begin{question2x2}{EF307}{Welcher Frequenzgang ist am besten für den Mikrofonverstärker eines Sprechfunkgeräts geeignet?}{\DARCimage{0.75\linewidth}{249include}}
{\DARCimage{0.75\linewidth}{247include}}
{\DARCimage{0.75\linewidth}{248include}}
{\DARCimage{0.75\linewidth}{246include}}
\end{question2x2}

}
\only<2>{
\begin{question2x2}{EF307}{Welcher Frequenzgang ist am besten für den Mikrofonverstärker eines Sprechfunkgeräts geeignet?}{\DARCimage{0.75\linewidth}{249include}}
{\DARCimage{0.75\linewidth}{247include}}
{\DARCimage{0.75\linewidth}{248include}}
{\textbf{\textcolor{DARCgreen}{\DARCimage{0.75\linewidth}{246include}}}}
\end{question2x2}

}
\end{frame}

\begin{frame}
\only<1>{
\begin{PQuestion}{EF308}{Über welche Bandbreite sollte der in der Blockschaltung dargestellte NF-Verstärker für eine gute Sprachverständlichkeit mindestens verfügen?}{ca. \qty{1,0}{\kHz}}
{ca. \qty{6,0}{\kHz}}
{ca. \qty{2,5}{\kHz}}
{ca. \qty{12,5}{\kHz}}
{\DARCimage{1.0\linewidth}{500include}}\end{PQuestion}

}
\only<2>{
\begin{PQuestion}{EF308}{Über welche Bandbreite sollte der in der Blockschaltung dargestellte NF-Verstärker für eine gute Sprachverständlichkeit mindestens verfügen?}{ca. \qty{1,0}{\kHz}}
{ca. \qty{6,0}{\kHz}}
{\textbf{\textcolor{DARCgreen}{ca. \qty{2,5}{\kHz}}}}
{ca. \qty{12,5}{\kHz}}
{\DARCimage{1.0\linewidth}{500include}}\end{PQuestion}

}

\end{frame}

\begin{frame}\begin{itemize}
  \item Die Stromzufuhr eines Senders sollte neben Stabilität auch einen guten Schutz gegen HF-Einstrahlung haben
  \item Damit verhindert man das Einströmen von Hochfrequenz in das Stromnetz
  \end{itemize}
\begin{itemize}
  \item Mehr dazu im Abschnitt Unerwünschte Ausstrahlungen im Kapitel Sender
  \end{itemize}
\end{frame}

\begin{frame}
\only<1>{
\begin{QQuestion}{EF405}{Wie sollte die Stromzufuhr in einem Sender beschaffen sein?}{Sie sollte mit möglichst wenig Kapazität gegen Masse ausgelegt werden. }
{Sie sollte möglichst hochohmig sein.}
{Sie sollte über das Leistungsverstärkergehäuse geführt werden.}
{Sie sollte gegen HF-Einstrahlung gut entkoppelt sein.}
\end{QQuestion}

}
\only<2>{
\begin{QQuestion}{EF405}{Wie sollte die Stromzufuhr in einem Sender beschaffen sein?}{Sie sollte mit möglichst wenig Kapazität gegen Masse ausgelegt werden. }
{Sie sollte möglichst hochohmig sein.}
{Sie sollte über das Leistungsverstärkergehäuse geführt werden.}
{\textbf{\textcolor{DARCgreen}{Sie sollte gegen HF-Einstrahlung gut entkoppelt sein.}}}
\end{QQuestion}

}
\end{frame}%ENDCONTENT
