
\section{Troposphäre III}
\label{section:troposphaere_3}
\begin{frame}%STARTCONTENT

\frametitle{Foliensatz in Arbeit}
2024-04-28: Die Inhalte werden noch aufbereitet.

Derzeit sind in diesem Abschnitt nur die Fragen sortiert enthalten.

Für das Selbststudium verweisen wir aktuell auf den Abschnitt Wellenausbreitung im DARC Online Lehrgang (\textcolor{DARCblue}{\faLink~\href{https://www.darc.de/der-club/referate/ajw/lehrgang-te/e09/}{www.darc.de/der-club/referate/ajw/lehrgang-te/e09/}}) für die Prüfung bis Juni 2024. Bis auf die Fragen hat sich an der Thematik nichts geändert. Das Thema war bisher Stoff der Klasse~E und wurde mit der neuen Prüfungsordnung auf alle drei Klassen aufgeteilt.

\end{frame}

\begin{frame}
\frametitle{Lage der Ionosphären-Schichten}

\begin{figure}
    \DARCimage{0.85\linewidth}{731include}
    \caption{\scriptsize Für den Amateurfunk relevante Schichten in der Atmosphäre}
    \label{e_atmosphaeren_schichten}
\end{figure}

\end{frame}

\begin{frame}
\only<1>{
\begin{QQuestion}{AH105}{In welcher Höhe befindet sich die für die Fernausbreitung (DX) wichtige F1-Region während der Tagesstunden? Sie befindet sich in ungefähr~...}{\qtyrange{200}{450}{\km} Höhe.}
{\qtyrange{90}{130}{\km} Höhe.}
{\qtyrange{50}{90}{\km} Höhe.}
{\qtyrange{130}{200}{\km} Höhe.}
\end{QQuestion}

}
\only<2>{
\begin{QQuestion}{AH105}{In welcher Höhe befindet sich die für die Fernausbreitung (DX) wichtige F1-Region während der Tagesstunden? Sie befindet sich in ungefähr~...}{\qtyrange{200}{450}{\km} Höhe.}
{\qtyrange{90}{130}{\km} Höhe.}
{\qtyrange{50}{90}{\km} Höhe.}
{\textbf{\textcolor{DARCgreen}{\qtyrange{130}{200}{\km} Höhe.}}}
\end{QQuestion}

}
\end{frame}

\begin{frame}
\only<1>{
\begin{QQuestion}{AH106}{In welcher Höhe befindet sich die für die Fernausbreitung (DX) wichtige F2-Region während der Tagesstunden an einem Sommertag? Sie befindet sich in ungefähr~...}{\qtyrange{130}{200}{\km} Höhe.}
{\qtyrange{250}{450}{\km} Höhe.}
{\qtyrange{90}{130}{\km} Höhe.}
{\qtyrange{50}{90}{\km} Höhe.}
\end{QQuestion}

}
\only<2>{
\begin{QQuestion}{AH106}{In welcher Höhe befindet sich die für die Fernausbreitung (DX) wichtige F2-Region während der Tagesstunden an einem Sommertag? Sie befindet sich in ungefähr~...}{\qtyrange{130}{200}{\km} Höhe.}
{\textbf{\textcolor{DARCgreen}{\qtyrange{250}{450}{\km} Höhe.}}}
{\qtyrange{90}{130}{\km} Höhe.}
{\qtyrange{50}{90}{\km} Höhe.}
\end{QQuestion}

}
\end{frame}

\begin{frame}
\frametitle{Effekte der Ionisierung}
\end{frame}

\begin{frame}
\only<1>{
\begin{QQuestion}{AH101}{Welcher Effekt sorgt hauptsächlich dafür, dass ionosphärische Regionen Funkwellen zur Erde ablenken können?}{Die von der Sonne ausgehende Infrarotstrahlung ionisiert~-~je nach Strahlungsintensität~-~die Moleküle in den verschiedenen Regionen.}
{Die von der Sonne ausgehende UV-Strahlung ionisiert~-~je nach Strahlungsintensität~-~die Moleküle in den verschiedenen Regionen.}
{Die von der Sonne ausgehende UV-Strahlung aktiviert~-~je nach Strahlungsintensität~-~die Sauerstoffatome in den verschiedenen Regionen.}
{Die von der Sonne ausgehende Infrarotstrahlung aktiviert~-~je nach Strahlungsintensität~-~die Sauerstoffatome in den verschiedenen Regionen.}
\end{QQuestion}

}
\only<2>{
\begin{QQuestion}{AH101}{Welcher Effekt sorgt hauptsächlich dafür, dass ionosphärische Regionen Funkwellen zur Erde ablenken können?}{Die von der Sonne ausgehende Infrarotstrahlung ionisiert~-~je nach Strahlungsintensität~-~die Moleküle in den verschiedenen Regionen.}
{\textbf{\textcolor{DARCgreen}{Die von der Sonne ausgehende UV-Strahlung ionisiert~-~je nach Strahlungsintensität~-~die Moleküle in den verschiedenen Regionen.}}}
{Die von der Sonne ausgehende UV-Strahlung aktiviert~-~je nach Strahlungsintensität~-~die Sauerstoffatome in den verschiedenen Regionen.}
{Die von der Sonne ausgehende Infrarotstrahlung aktiviert~-~je nach Strahlungsintensität~-~die Sauerstoffatome in den verschiedenen Regionen.}
\end{QQuestion}

}
\end{frame}

\begin{frame}
\only<1>{
\begin{QQuestion}{AH108}{Zu welcher Jahres- und Tageszeit hat die F2-Region ihre größte Höhe? Sie hat ihre größte Höhe~...}{im Sommer um Mitternacht.}
{im Sommer zur Mittagszeit.}
{im Frühjahr und Herbst zur Dämmerungszeit.}
{im Winter zur Mittagszeit.}
\end{QQuestion}

}
\only<2>{
\begin{QQuestion}{AH108}{Zu welcher Jahres- und Tageszeit hat die F2-Region ihre größte Höhe? Sie hat ihre größte Höhe~...}{im Sommer um Mitternacht.}
{\textbf{\textcolor{DARCgreen}{im Sommer zur Mittagszeit.}}}
{im Frühjahr und Herbst zur Dämmerungszeit.}
{im Winter zur Mittagszeit.}
\end{QQuestion}

}
\end{frame}

\begin{frame}
\only<1>{
\begin{QQuestion}{AH221}{Massiv erhöhte UV- und Röntgenstrahlung, wie sie vor allem durch starke Sonneneruptionen hervorgerufen wird, beeinflusst in der Ionosphäre vor allem~...}{die F2-Region, die dann so stark ionisiert wird, dass fast die gesamte KW-Ausstrahlung reflektiert wird.}
{die D-Region, die die Kurzwellen-Signale dann so massiv dämpft, dass keine Ausbreitung über die Raumwelle mehr möglich ist.}
{die E-Region, die dann für die höheren Frequenzen durchlässiger wird und durch Refraktion (Brechung) in der F2-Region für gute Ausbreitungsbedingungen sorgt.}
{die F1-Region, die durch Absorption der höheren Frequenzen die Refraktion (Brechung) an der F2-Region behindert.}
\end{QQuestion}

}
\only<2>{
\begin{QQuestion}{AH221}{Massiv erhöhte UV- und Röntgenstrahlung, wie sie vor allem durch starke Sonneneruptionen hervorgerufen wird, beeinflusst in der Ionosphäre vor allem~...}{die F2-Region, die dann so stark ionisiert wird, dass fast die gesamte KW-Ausstrahlung reflektiert wird.}
{\textbf{\textcolor{DARCgreen}{die D-Region, die die Kurzwellen-Signale dann so massiv dämpft, dass keine Ausbreitung über die Raumwelle mehr möglich ist.}}}
{die E-Region, die dann für die höheren Frequenzen durchlässiger wird und durch Refraktion (Brechung) in der F2-Region für gute Ausbreitungsbedingungen sorgt.}
{die F1-Region, die durch Absorption der höheren Frequenzen die Refraktion (Brechung) an der F2-Region behindert.}
\end{QQuestion}

}
\end{frame}

\begin{frame}
\frametitle{Polarisation}
\end{frame}

\begin{frame}
\only<1>{
\begin{QQuestion}{AH219}{Wie wird die Polarisation einer elektromagnetischen Welle bei der Ausbreitung über die Raumwelle beeinflusst?}{Die Polarisation der ausgesendeten Wellen wird in der Ionosphäre stets um \qty{90}{\degree} gedreht.}
{Die Polarisation der ausgesendeten Wellen bleibt bei der Refraktion (Brechung) in der Ionosphäre stets unverändert.}
{Die Polarisation der ausgesendeten Wellen wird bei der Refraktion (Brechung) in der Ionosphäre stets verändert.}
{Die Polarisation der ausgesendeten Wellen wird bei jedem Sprung (Hop) in der Ionosphäre um \qty{90}{\degree} gedreht.}
\end{QQuestion}

}
\only<2>{
\begin{QQuestion}{AH219}{Wie wird die Polarisation einer elektromagnetischen Welle bei der Ausbreitung über die Raumwelle beeinflusst?}{Die Polarisation der ausgesendeten Wellen wird in der Ionosphäre stets um \qty{90}{\degree} gedreht.}
{Die Polarisation der ausgesendeten Wellen bleibt bei der Refraktion (Brechung) in der Ionosphäre stets unverändert.}
{\textbf{\textcolor{DARCgreen}{Die Polarisation der ausgesendeten Wellen wird bei der Refraktion (Brechung) in der Ionosphäre stets verändert.}}}
{Die Polarisation der ausgesendeten Wellen wird bei jedem Sprung (Hop) in der Ionosphäre um \qty{90}{\degree} gedreht.}
\end{QQuestion}

}
\end{frame}

\begin{frame}
\frametitle{Funkwellen-Ausbreitung}
\end{frame}

\begin{frame}
\only<1>{
\begin{QQuestion}{AH201}{Welches der nachstehend aufgeführten Bänder ist für KW-Verbindungen zwischen Hamburg und München um die Mittagszeit herum üblicherweise gut geeignet?}{\qty{40}{\m}-Band}
{\qty{160}{\m}-Band}
{\qty{80}{\m}-Band}
{\qty{15}{\m}-Band}
\end{QQuestion}

}
\only<2>{
\begin{QQuestion}{AH201}{Welches der nachstehend aufgeführten Bänder ist für KW-Verbindungen zwischen Hamburg und München um die Mittagszeit herum üblicherweise gut geeignet?}{\textbf{\textcolor{DARCgreen}{\qty{40}{\m}-Band}}}
{\qty{160}{\m}-Band}
{\qty{80}{\m}-Band}
{\qty{15}{\m}-Band}
\end{QQuestion}

}
\end{frame}

\begin{frame}
\only<1>{
\begin{QQuestion}{AH203}{Welche der folgenden Frequenzbänder können in den Nachtstunden am ehesten für weltweite Funkverbindungen genutzt werden?}{\qty{160}{\m}, \qty{80}{\m} und \qty{40}{\m}}
{\qty{40}{\m}, \qty{20}{\m} und \qty{15}{\m}}
{\qty{40}{\m}, \qty{17}{\m} und \qty{6}{\m}}
{ \qty{30}{\m}, \qty{12}{\m} und \qty{10}{\m}}
\end{QQuestion}

}
\only<2>{
\begin{QQuestion}{AH203}{Welche der folgenden Frequenzbänder können in den Nachtstunden am ehesten für weltweite Funkverbindungen genutzt werden?}{\textbf{\textcolor{DARCgreen}{\qty{160}{\m}, \qty{80}{\m} und \qty{40}{\m}}}}
{\qty{40}{\m}, \qty{20}{\m} und \qty{15}{\m}}
{\qty{40}{\m}, \qty{17}{\m} und \qty{6}{\m}}
{ \qty{30}{\m}, \qty{12}{\m} und \qty{10}{\m}}
\end{QQuestion}

}
\end{frame}

\begin{frame}
\only<1>{
\begin{QQuestion}{AH107}{Für die DX-Kurzwellenausbreitung über die Raumwelle ist die F1-Region~...}{erwünscht, weil sie durch zusätzliche Reflexion die Wirkung der F2-Region verstärken kann.}
{meist unerwünscht, weil sie durch Abdeckung die Ausbreitung durch Refraktion (Brechung) an der F2-Region verhindern kann.}
{nicht von großer Bedeutung, weil sie vor allem für die höheren Frequenzen durchlässig ist.}
{von großer Bedeutung, weil sie die Dämpfung in der E-Region senkt und damit die Sprungdistanz vergrößert.}
\end{QQuestion}

}
\only<2>{
\begin{QQuestion}{AH107}{Für die DX-Kurzwellenausbreitung über die Raumwelle ist die F1-Region~...}{erwünscht, weil sie durch zusätzliche Reflexion die Wirkung der F2-Region verstärken kann.}
{\textbf{\textcolor{DARCgreen}{meist unerwünscht, weil sie durch Abdeckung die Ausbreitung durch Refraktion (Brechung) an der F2-Region verhindern kann.}}}
{nicht von großer Bedeutung, weil sie vor allem für die höheren Frequenzen durchlässig ist.}
{von großer Bedeutung, weil sie die Dämpfung in der E-Region senkt und damit die Sprungdistanz vergrößert.}
\end{QQuestion}

}
\end{frame}

\begin{frame}
\frametitle{Troposphärische Ausbreitung}
\end{frame}

\begin{frame}
\only<1>{
\begin{QQuestion}{AH309}{Überhorizontverbindungen im VHF/UHF-Bereich kommen unter anderem zustande durch~...}{Polarisationsdrehungen in der Troposphäre an Gewitterfronten.}
{Reflexion der Wellen in der Troposphäre durch das Auftreten sporadischer D-Regionen.}
{Polarisationsdrehungen in der Troposphäre bei hoch liegender Bewölkung.}
{troposphärische Duct-Übertragung beim Auftreten von Inversionsschichten.}
\end{QQuestion}

}
\only<2>{
\begin{QQuestion}{AH309}{Überhorizontverbindungen im VHF/UHF-Bereich kommen unter anderem zustande durch~...}{Polarisationsdrehungen in der Troposphäre an Gewitterfronten.}
{Reflexion der Wellen in der Troposphäre durch das Auftreten sporadischer D-Regionen.}
{Polarisationsdrehungen in der Troposphäre bei hoch liegender Bewölkung.}
{\textbf{\textcolor{DARCgreen}{troposphärische Duct-Übertragung beim Auftreten von Inversionsschichten.}}}
\end{QQuestion}

}
\end{frame}%ENDCONTENT
