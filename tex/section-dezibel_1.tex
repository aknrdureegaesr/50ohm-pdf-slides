
\section{Dezibel I}
\label{section:dezibel_1}
\begin{frame}%STARTCONTENT

\frametitle{Dezibel einfach erklärt}
\begin{table}
\begin{DARCtabular}{lr}
    Was  &Leistung in mW   \\
     effektive Leistung EME-Station  & 100 000 000   \\
     Standard Transceiver  & 100 000   \\
     Kleine Handfunke  & 1 000   \\
     Lautsprechersignal (Zimmerlautstärke)  & 100   \\
     Kopfhörersignal  & 1   \\
     Lautes KW-Signal  & 0,000 001   \\
     Leises KW-Signal (Antenneneingang RX)  & 0,000 000 000 001   \\
\end{DARCtabular}
\caption{Leistungen in mW}
\label{e_dezibel_leistungen_mw}
\end{table}
Wer mit diesen Zahlen umgeht, fängt automatisch an, die Nullen zu zählen.

\end{frame}

\begin{frame}Wir zählen die Nullen (und nennen das Ergebnis \enquote{Bel})

\begin{table}
\begin{DARCtabular}{lrr}
    Was  &Leistung in mW  &Bel   \\
     effektive Leistung EME-Station  & 100 000 000  & 8   \\
     Standard Transceiver  & 100 000  & 5   \\
     Kleine Handfunke  & 1 000  & 3   \\
     Lautsprechersignal (Zimmerlautstärke)  & 100  & 2   \\
     Kopfhörersignal  & 1  & 0   \\
     Lautes KW-Signal  & 0,000 001  & -6   \\
     Leises KW-Signal (Antenneneingang RX)  & 0,000 000 000 001  & -12   \\
\end{DARCtabular}
\caption{Leistungen in mW und Bel}
\label{e_dezibel_leistungen_bel}
\end{table}

\end{frame}

\begin{frame}dBm = Dezibel bezogen auf mW

\begin{table}
\begin{DARCtabular}{lrrr}
    Was  &Leistung in mW  &Bel  &dBm   \\
     effektive Leistung EME-Station  & 100 000 000  & 8  & 80   \\
     Standard Transceiver  & 100 000  & 5  & 50   \\
     Kleine Handfunke  & 1 000  & 3  & 30   \\
     Lautsprechersignal (Zimmerlautstärke)  & 100  & 2  & 20   \\
     Kopfhörersignal  & 1  & 0  & 0   \\
     Lautes KW-Signal  & 0,000 001  & -6  & -60   \\
     Leises KW-Signal (Antenneneingang RX)  & 0,000 000 000 001  & -12  & -120   \\
\end{DARCtabular}
\caption{Leistungen in mW und Bel}
\label{e_dezibel_leistungen_bel}
\end{table}

\end{frame}

\begin{frame}
\frametitle{Leistungsverstärkung}
\emph{Empfänger}

\begin{itemize}
  \item Eingangssignal: 0,000 000 000 \qty{001}{\milli\watt}
  \item Ausgangssignal: \qty{100}{\milli\watt}
  \item Benötigte Verstärkung: 100 000 000 000 000
  \end{itemize}
\emph{Sender}

\begin{itemize}
  \item Frequenzerzeugende Stufe (Oszillator): \qty{10}{\milli\watt}
  \item Ausgangssignal: 100 \qty{000}{\milli\watt}
  \item Benötigte Verstärkung: 10 000
  \end{itemize}
\end{frame}

\begin{frame}
\frametitle{Leistungsverstärkung mit dB}
\emph{Empfänger}

\begin{itemize}
  \item Eingangssignal: 0,000 000 000 \qty{001}{\milli\watt} = -\qty{120}{\dBm}
  \item Ausgangssignal: \qty{100}{\milli\watt} = \qty{20}{\dBm}
  \item Benötigte Verstärkung: 100 000 000 000 000 = \qty{140}{\dB}
  \end{itemize}
\emph{Sender}

\begin{itemize}
  \item Frequenzerzeugende Stufe (Oszillator): \qty{10}{\milli\watt} = \qty{10}{\dBm}
  \item Ausgangssignal: 100 \qty{000}{\milli\watt} = \qty{50}{\dBm}
  \item Benötigte Verstärkung: 10 000 = \qty{40}{\dB}
  \end{itemize}

\end{frame}

\begin{frame}
\frametitle{Wichtige Leistungsfaktoren}
\begin{table}
\begin{DARCtabular}{cc}
    dB  & $\approx$  Leistungsfaktor   \\
     0  & 1   \\
     1,5  &$\sqrt{2} = 1,41$  \\
     2,15  & 1,64   \\
     3  & 2   \\
     5  &$\sqrt{10} = 3,16$  \\
     6  & 4   \\
     10  & 10   \\
     20  & 100   \\
\end{DARCtabular}
\caption{Wichtige Leistungsfaktoren in dB}
\label{e_dezibel_leistungsfaktoren}
\end{table}

\end{frame}

\begin{frame}
\frametitle{Berechnung mit Taschenrechner}
Ältere Modelle

\begin{itemize}
  \item Faktor-Wert $\rightarrow$ \emph{log}-Taste $\rightarrow$ $\cdot$10 $\rightarrow$ dB
  \item dB-Wert $\rightarrow$  $\div$ 10 $\rightarrow$ \emph{10<sup>x</sup>}-Taste $\rightarrow$ Faktor
  \end{itemize}
Neuere Modelle

\begin{itemize}
  \item \emph{log}-Taste $\rightarrow$ Faktor-Wert $\rightarrow$ \emph{)}-Taste $\rightarrow$ $\cdot$10 $\rightarrow$ \emph{=}-Taste $\rightarrow$ dB
  \item \emph{10<sup>x</sup>}-Taste $\rightarrow$ dB-Wert $\rightarrow$  $\div$ 10 $\rightarrow$ \emph{=}-Taste $\rightarrow$ Faktor
  \end{itemize}
\end{frame}

\begin{frame}
\only<1>{
\begin{QQuestion}{EA107}{Um wie viel Dezibel verändert sich der Leistungspegel, wenn die Leistung verdoppelt wird?}{\qty{3}{\decibel}}
{\qty{6}{\decibel}}
{\qty{1,5}{\decibel}}
{\qty{12}{\decibel}}
\end{QQuestion}

}
\only<2>{
\begin{QQuestion}{EA107}{Um wie viel Dezibel verändert sich der Leistungspegel, wenn die Leistung verdoppelt wird?}{\textbf{\textcolor{DARCgreen}{\qty{3}{\decibel}}}}
{\qty{6}{\decibel}}
{\qty{1,5}{\decibel}}
{\qty{12}{\decibel}}
\end{QQuestion}

}
\end{frame}%ENDCONTENT
