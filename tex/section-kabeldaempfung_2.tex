
\section{Kabeldämpfung II}
\label{section:kabeldaempfung_2}
\begin{frame}%STARTCONTENT

\only<1>{
\begin{QQuestion}{AG309}{Welches Koaxkabel ist nach dem zur Verfügung gestellten Kabeldämpfungsdiagramm für eine \qty{20}{\m} lange Verbindung zwischen Senderausgang und Antenne geeignet, wenn die Kabeldämpfung im \qty{13}{\cm}-Band bei \qty{2,350}{\GHz} nicht mehr als \qty{4}{\decibel} betragen soll?}{Voll-PE-Kabel mit \qty{10,3}{\mm} Durchmesser (Typ RG213).
}
{PE-Schaumkabel mit \qty{7,3}{\mm} Durchmesser.}
{PE-Schaumkabel mit \qty{12,7}{\mm} Durchmesser.}
{PE-Schaumkabel mit \qty{10,3}{\mm} Durchmesser.}
\end{QQuestion}

}
\only<2>{
\begin{QQuestion}{AG309}{Welches Koaxkabel ist nach dem zur Verfügung gestellten Kabeldämpfungsdiagramm für eine \qty{20}{\m} lange Verbindung zwischen Senderausgang und Antenne geeignet, wenn die Kabeldämpfung im \qty{13}{\cm}-Band bei \qty{2,350}{\GHz} nicht mehr als \qty{4}{\decibel} betragen soll?}{Voll-PE-Kabel mit \qty{10,3}{\mm} Durchmesser (Typ RG213).
}
{PE-Schaumkabel mit \qty{7,3}{\mm} Durchmesser.}
{\textbf{\textcolor{DARCgreen}{PE-Schaumkabel mit \qty{12,7}{\mm} Durchmesser.}}}
{PE-Schaumkabel mit \qty{10,3}{\mm} Durchmesser.}
\end{QQuestion}

}
\end{frame}

\begin{frame}
\only<1>{
\begin{QQuestion}{AG310}{Zur Verbindung Ihres \qty{5,700}{\GHz}-Senders (\qty{6}{\cm}-Band) mit dem Feed eines Parabolspiegels benötigen Sie ein \qty{8}{\m} langes und möglichst dünnes Koaxialkabel, das nicht mehr als \qty{3}{\decibel} Dämpfung haben soll. Welches der Koaxialkabel aus dem Kabeldämpfungsdiagramm erfüllt diese Anforderung?}{PE-Schaumkabel mit \qty{7,3}{\mm} Durchmesser.}
{PE-Schaumkabel mit \qty{12,7}{\mm} Durchmesser.}
{PE-Schaumkabel mit Massivschirm und \qty{16,4}{\mm} Durchmesser.}
{PE-Schaumkabel mit \qty{10,3}{\mm} Durchmesser.}
\end{QQuestion}

}
\only<2>{
\begin{QQuestion}{AG310}{Zur Verbindung Ihres \qty{5,700}{\GHz}-Senders (\qty{6}{\cm}-Band) mit dem Feed eines Parabolspiegels benötigen Sie ein \qty{8}{\m} langes und möglichst dünnes Koaxialkabel, das nicht mehr als \qty{3}{\decibel} Dämpfung haben soll. Welches der Koaxialkabel aus dem Kabeldämpfungsdiagramm erfüllt diese Anforderung?}{PE-Schaumkabel mit \qty{7,3}{\mm} Durchmesser.}
{\textbf{\textcolor{DARCgreen}{PE-Schaumkabel mit \qty{12,7}{\mm} Durchmesser.}}}
{PE-Schaumkabel mit Massivschirm und \qty{16,4}{\mm} Durchmesser.}
{PE-Schaumkabel mit \qty{10,3}{\mm} Durchmesser.}
\end{QQuestion}

}
\end{frame}

\begin{frame}
\only<1>{
\begin{QQuestion}{AG308}{Welcher Typ Koaxialkabel ist laut zur Verfügung gestelltem Kabeldämpfungsdiagramm für eine \qty{60}{\m} lange Verbindung zwischen Senderausgang und einem Multiband-Kurzwellenbeam für die Bänder \qty{20}{\m}, \qty{15}{\m} und \qty{10}{\m} geeignet, wenn die Kabeldämpfung bei \qty{29}{\MHz} nicht mehr als \qty{2}{\decibel} betragen soll?}{Voll-PE-Kabel mit \qty{4,95}{\mm} Durchmesser (Typ RG58).}
{PE-Schaumkabel mit \qty{10,3}{\mm} Durchmesser.}
{Voll-PE-Kabel mit \qty{10,3}{\mm} Durchmesser (Typ RG213).}
{PE-Schaumkabel mit \qty{7,3}{\mm} Durchmesser.}
\end{QQuestion}

}
\only<2>{
\begin{QQuestion}{AG308}{Welcher Typ Koaxialkabel ist laut zur Verfügung gestelltem Kabeldämpfungsdiagramm für eine \qty{60}{\m} lange Verbindung zwischen Senderausgang und einem Multiband-Kurzwellenbeam für die Bänder \qty{20}{\m}, \qty{15}{\m} und \qty{10}{\m} geeignet, wenn die Kabeldämpfung bei \qty{29}{\MHz} nicht mehr als \qty{2}{\decibel} betragen soll?}{Voll-PE-Kabel mit \qty{4,95}{\mm} Durchmesser (Typ RG58).}
{\textbf{\textcolor{DARCgreen}{PE-Schaumkabel mit \qty{10,3}{\mm} Durchmesser.}}}
{Voll-PE-Kabel mit \qty{10,3}{\mm} Durchmesser (Typ RG213).}
{PE-Schaumkabel mit \qty{7,3}{\mm} Durchmesser.}
\end{QQuestion}

}
\end{frame}

\begin{frame}
\only<1>{
\begin{QQuestion}{AG311}{Welche der folgenden Leitungen weist bei gleichem Leiterquerschnitt im Kurzwellenbereich den geringsten Verlust auf?}{Zweidrahtleitung mit großem Abstand und schmalen Stegen.}
{Zweidrahtleitung mit großem Abstand und breiten Stegen.}
{Zweidrahtleitung mit geringem Abstand und Kunststoffumhüllung.}
{Verdrillte Zweidrahtleitung mit Kunststoffumhüllung.}
\end{QQuestion}

}
\only<2>{
\begin{QQuestion}{AG311}{Welche der folgenden Leitungen weist bei gleichem Leiterquerschnitt im Kurzwellenbereich den geringsten Verlust auf?}{\textbf{\textcolor{DARCgreen}{Zweidrahtleitung mit großem Abstand und schmalen Stegen.}}}
{Zweidrahtleitung mit großem Abstand und breiten Stegen.}
{Zweidrahtleitung mit geringem Abstand und Kunststoffumhüllung.}
{Verdrillte Zweidrahtleitung mit Kunststoffumhüllung.}
\end{QQuestion}

}
\end{frame}%ENDCONTENT
