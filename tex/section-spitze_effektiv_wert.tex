
\section{Spitzen- und Effektivwert}
\label{section:spitze_effektiv_wert}
\begin{frame}%STARTCONTENT

\frametitle{Spitzenwert}
\begin{columns}
    \begin{column}{0.48\textwidth}
    \begin{itemize}
  \item Der Spitzenwert einer Sinusschwingung entspricht der Amplitude
  \item Von Nulllinie bis höchstem Wert
  \item Spitzen-Spitzen-Wert von niedrigstem bis höchstem Wert
  \end{itemize}

    \end{column}
   \begin{column}{0.48\textwidth}
       
\begin{figure}
    \DARCimage{0.85\linewidth}{834include}
    \caption{\scriptsize Peridoendauer, Spitzenspannung, Effektivspannung und Spitzen-Spitzen-Spannung}
    \label{e_spitze_effektiv_wert_bezeichnungen_sinus}
\end{figure}


   \end{column}
\end{columns}

\end{frame}

\begin{frame}Spitzen-Spitzen-Wert bei sinusförmigen Spannungen

$U_{SS} = 2\cdot \^{U}$

\end{frame}

\begin{frame}
\only<1>{
\begin{PQuestion}{EB407}{Wie groß ist der Spitzen-Spitzen-Wert ($U_{\symup{ss}}$) der in der Abbildung dargestellten Spannung?}{\qty{10}{\V}}
{\qty{20}{\V}}
{\qty{40}{\V}}
{\qty{4}{\V}}
{\DARCimage{1.0\linewidth}{52include}}\end{PQuestion}

}
\only<2>{
\begin{PQuestion}{EB407}{Wie groß ist der Spitzen-Spitzen-Wert ($U_{\symup{ss}}$) der in der Abbildung dargestellten Spannung?}{\qty{10}{\V}}
{\qty{20}{\V}}
{\textbf{\textcolor{DARCgreen}{\qty{40}{\V}}}}
{\qty{4}{\V}}
{\DARCimage{1.0\linewidth}{52include}}\end{PQuestion}

}
\end{frame}

\begin{frame}
\only<1>{
\begin{PQuestion}{EB406}{Wie groß ist der Spitzen-Spitzen-Wert der in diesem Schirmbild dargestellten Spannung?}{\qty{12}{\V}}
{\qty{6}{\V}}
{\qty{8,5}{\V}}
{\qty{2}{\V}}
{\DARCimage{1.0\linewidth}{53include}}\end{PQuestion}

}
\only<2>{
\begin{PQuestion}{EB406}{Wie groß ist der Spitzen-Spitzen-Wert der in diesem Schirmbild dargestellten Spannung?}{\textbf{\textcolor{DARCgreen}{\qty{12}{\V}}}}
{\qty{6}{\V}}
{\qty{8,5}{\V}}
{\qty{2}{\V}}
{\DARCimage{1.0\linewidth}{53include}}\end{PQuestion}

}
\end{frame}

\begin{frame}
\frametitle{Effektivwert}
Bei einer Wechselspannung der Wert, der in einem Widerstand zu einer vergleichsweisen Gleichspannung in Leistung umgesetzt wird


\begin{figure}
    \DARCimage{0.85\linewidth}{725include}
    \caption{\scriptsize Effektivwert und Spitzenwert der Spannung im Haushalt}
    \label{e_effektivwert_230v}
\end{figure}

\end{frame}

\begin{frame}Bei Spannungen (ohne Herleitung)

$\^{U} = U_{eff}\cdot \sqrt{2}$

\end{frame}

\begin{frame}
\only<1>{
\begin{PQuestion}{EB405}{Welche der im folgenden Diagramm eingezeichneten Gleichspannungen setzen an einem Wirkwiderstand etwa die gleiche Leistung um wie die dargestellte sinusförmige Wechselspannung?}{\qty{0}{\V}}
{\qty{1}{\V} und \qty{-1}{\V}}
{\qty{0,5}{\V} und \qty{-0,5}{\V}}
{\qty{0,7}{\V} und \qty{-0,7}{\V}}
{\DARCimage{1.0\linewidth}{206include}}\end{PQuestion}

}
\only<2>{
\begin{PQuestion}{EB405}{Welche der im folgenden Diagramm eingezeichneten Gleichspannungen setzen an einem Wirkwiderstand etwa die gleiche Leistung um wie die dargestellte sinusförmige Wechselspannung?}{\qty{0}{\V}}
{\qty{1}{\V} und \qty{-1}{\V}}
{\qty{0,5}{\V} und \qty{-0,5}{\V}}
{\textbf{\textcolor{DARCgreen}{\qty{0,7}{\V} und \qty{-0,7}{\V}}}}
{\DARCimage{1.0\linewidth}{206include}}\end{PQuestion}

}

\end{frame}

\begin{frame}
\frametitle{Lösungsweg}
$\^{U} = U_{eff}\cdot \sqrt{2}$

$U_{eff} = \dfrac{\^{U}}{\sqrt{2}}$

$U_{eff} = \dfrac{1V}{1,41} \approx 0,7V$

\end{frame}

\begin{frame}
\only<1>{
\begin{QQuestion}{EB404}{Eine sinusförmige Wechselspannung hat einen Spitzenwert von \qty{12}{\V}. Wie groß ist in etwa der Effektivwert der Wechselspannung?}{\qty{24}{\V}}
{\qty{6,0}{\V}}
{\qty{17}{\V}}
{\qty{8,5}{\V}}
\end{QQuestion}

}
\only<2>{
\begin{QQuestion}{EB404}{Eine sinusförmige Wechselspannung hat einen Spitzenwert von \qty{12}{\V}. Wie groß ist in etwa der Effektivwert der Wechselspannung?}{\qty{24}{\V}}
{\qty{6,0}{\V}}
{\qty{17}{\V}}
{\textbf{\textcolor{DARCgreen}{\qty{8,5}{\V}}}}
\end{QQuestion}

}

\end{frame}

\begin{frame}
\frametitle{Lösungsweg}
$\^{U} = U_{eff}\cdot \sqrt{2}$

$U_{eff} = \dfrac{\^{U}}{\sqrt{2}}$

$U_{eff} = \dfrac{12V}{1,41} \approx 8,5V$

\end{frame}

\begin{frame}
\only<1>{
\begin{QQuestion}{EB403}{Ein sinusförmiges Signal hat einen Effektivwert von \qty{12}{\V}. Wie groß ist in etwa der Spitzen-Spitzen-Wert?}{\qty{17}{\V}}
{\qty{24}{\V}}
{\qty{34}{\V}}
{\qty{8,5}{\V}}
\end{QQuestion}

}
\only<2>{
\begin{QQuestion}{EB403}{Ein sinusförmiges Signal hat einen Effektivwert von \qty{12}{\V}. Wie groß ist in etwa der Spitzen-Spitzen-Wert?}{\qty{17}{\V}}
{\qty{24}{\V}}
{\textbf{\textcolor{DARCgreen}{\qty{34}{\V}}}}
{\qty{8,5}{\V}}
\end{QQuestion}

}
\end{frame}

\begin{frame}
\frametitle{Lösungsweg}
$\^{U} = U_{eff}\cdot \sqrt{2}$

$\^{U} = 12V\cdot 1,41 \approx 17V$

$U_{SS} = 2\cdot \^{U}$

$U_{SS} = 2\cdot 17V = 34V$

\end{frame}

\begin{frame}
\only<1>{
\begin{QQuestion}{EB401}{Der Spitzenwert an einer häuslichen, einphasigen \qty{230}{\V}-Stromversorgung beträgt~...}{\qty{163}{\V}.}
{\qty{325}{\V}.}
{\qty{460}{\V}.}
{\qty{650}{\V}.}
\end{QQuestion}

}
\only<2>{
\begin{QQuestion}{EB401}{Der Spitzenwert an einer häuslichen, einphasigen \qty{230}{\V}-Stromversorgung beträgt~...}{\qty{163}{\V}.}
{\textbf{\textcolor{DARCgreen}{\qty{325}{\V}.}}}
{\qty{460}{\V}.}
{\qty{650}{\V}.}
\end{QQuestion}

}

\end{frame}

\begin{frame}
\frametitle{Lösungsweg}
$\^{U} = U_{eff}\cdot \sqrt{2}$

$\^{U} = 230V\cdot 1,41 \approx 325V$

\end{frame}

\begin{frame}
\only<1>{
\begin{QQuestion}{EB402}{Der Spitze-Spitze-Wert der häuslichen \qty{230}{\V}-Spannungsversorgung beträgt~...}{\qty{651}{\V}.}
{\qty{163}{\V}.}
{\qty{325}{\V}.}
{\qty{460}{\V}.}
\end{QQuestion}

}
\only<2>{
\begin{QQuestion}{EB402}{Der Spitze-Spitze-Wert der häuslichen \qty{230}{\V}-Spannungsversorgung beträgt~...}{\textbf{\textcolor{DARCgreen}{\qty{651}{\V}.}}}
{\qty{163}{\V}.}
{\qty{325}{\V}.}
{\qty{460}{\V}.}
\end{QQuestion}

}

\end{frame}%ENDCONTENT
