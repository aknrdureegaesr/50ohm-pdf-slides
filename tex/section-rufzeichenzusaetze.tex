
\section{Rufzeichenzusätze}
\label{section:rufzeichenzusaetze}
\begin{frame}%STARTCONTENT
Beim Funken von einem Standort anders als dem in der Zulassungsurkunde angegebenen Heimatstandort, kann ein Rufzeichenzusatz verwendet werden.

\end{frame}

\begin{frame}\begin{table}
\begin{DARCtabular}{lll}
     Zusatz  & Gesprochen  & Bedeutung   \\
     am  & aeronautisch mobil  & An Bord eines Luftfahrzeugst, das sich im Flug befindet   \\
     mm  & maritim mobil  & An Bord eines Schiffs auf See   \\
     m  & mobil  & Von einem Landfahrzeug oder einem Schiff auf Binnengewässern aus   \\
     p  & portabel  & Zu Fuß unterwegs oder vorübergehend ortsfest   \\
     R  & Remote  & Remote-Betrieb   \\
     T  & Trainee  & Ausbildungsfunk   \\
\end{DARCtabular}
\caption{Mögliche Rufzeichenzusätze}
\label{n_rufzeichenzusaetze}
\end{table}

\end{frame}

\begin{frame}\begin{itemize}
  \item Geschrieben mit \enquote{/}
  \item Gesprochen direkt im Anschluss an das Rufzeichen oder mit \enquote{Stroke}
  \end{itemize}
    \pause
    \begin{table}
\begin{DARCtabular}{lX}
     Schreibweise  & Aussprache   \\
     DL1FLO/m  & Delta Lima Eins Foxtrott Lima Oskar (Stroke) Mobil   \\
     DM4EAX/p  & Delta Mike Vier Echo Alpha X-Ray (Stroke) Portabel   \\
     DL1ASN/mm  & Delta Lima Eins Alpha Sierra November (Stroke) Maritim Mobil   \\
     DG2RON/am  & Delta Golf Zwei Romeo Oscar November (Stroke) Aeronautisch Mobil   \\
\end{DARCtabular}
\caption{Sprechweise von Rufzeichenzusätzen, "Stroke" ist optional und kann weggelassen werden}
\label{n_rufzeichenzusaetze_sprechweise}
\end{table}


\end{frame}

\begin{frame}
\frametitle{Aeronautisch Mobil}
\begin{itemize}
  \item An Bord eines Luftfahrzeugs (Flugzeug, Heißluftballon, Zeppelin, o.ä.)
  \item Muss sich komplett in der Luft befinden
  \item Keine Verbindung zum Boden
  \item Betrieb muss vom Luftfahrzeugführer erlaubt sein, jedoch nicht von der BNetzA genehmigt werden
  \end{itemize}
\end{frame}

\begin{frame}
\only<1>{
\begin{QQuestion}{BD201}{Was bedeutet der Rufzeichenzusatz \glqq /am\grqq{}? Die Amateurfunkstelle~...}{verwendet Amplitudenmodulation.}
{wird an Bord eines Luftfahrzeugs betrieben.}
{wird an Bord eines Wasserfahrzeugs betrieben.}
{arbeitet mit geringer Leistung.}
\end{QQuestion}

}
\only<2>{
\begin{QQuestion}{BD201}{Was bedeutet der Rufzeichenzusatz \glqq /am\grqq{}? Die Amateurfunkstelle~...}{verwendet Amplitudenmodulation.}
{\textbf{\textcolor{DARCgreen}{wird an Bord eines Luftfahrzeugs betrieben.}}}
{wird an Bord eines Wasserfahrzeugs betrieben.}
{arbeitet mit geringer Leistung.}
\end{QQuestion}

}
\end{frame}

\begin{frame}
\only<1>{
\begin{QQuestion}{BD202}{Welche Bedeutung hat das Rufzeichen VE8ZZ/am?}{Es handelt sich um eine kanadische Amateurfunkstelle, die vorübergehend in den Vereinigten Staaten von Amerika betrieben wird.}
{Es handelt sich um eine Amateurfunkstelle mit einem kanadischen Rufzeichen, die in einem Luftfahrzeug betrieben wird.}
{Es handelt sich um eine kanadische Amateurfunkstelle, die in der Modulationsart AM betrieben wird.}
{Es handelt sich um eine automatisch arbeitende Pactor-Amateurfunkstelle mit angeschlossener Mailbox in Kanada.}
\end{QQuestion}

}
\only<2>{
\begin{QQuestion}{BD202}{Welche Bedeutung hat das Rufzeichen VE8ZZ/am?}{Es handelt sich um eine kanadische Amateurfunkstelle, die vorübergehend in den Vereinigten Staaten von Amerika betrieben wird.}
{\textbf{\textcolor{DARCgreen}{Es handelt sich um eine Amateurfunkstelle mit einem kanadischen Rufzeichen, die in einem Luftfahrzeug betrieben wird.}}}
{Es handelt sich um eine kanadische Amateurfunkstelle, die in der Modulationsart AM betrieben wird.}
{Es handelt sich um eine automatisch arbeitende Pactor-Amateurfunkstelle mit angeschlossener Mailbox in Kanada.}
\end{QQuestion}

}
\end{frame}

\begin{frame}
\only<1>{
\begin{QQuestion}{VE705}{Welche Voraussetzung muss erfüllt sein, damit Sie Amateurfunk an Bord eines Luftfahrzeugs betreiben dürfen?}{Genehmigung der Bundesnetzagentur für aeronautischen Funkbetrieb}
{Zustimmung des verantwortlichen Luftfahrzeugführers}
{Verwendung einer fest installierten Funkstelle des mobilen Flugfunkdienstes}
{Nutzung von Frequenzen, die dem mobilen Flugfunkdienst zugewiesen sind}
\end{QQuestion}

}
\only<2>{
\begin{QQuestion}{VE705}{Welche Voraussetzung muss erfüllt sein, damit Sie Amateurfunk an Bord eines Luftfahrzeugs betreiben dürfen?}{Genehmigung der Bundesnetzagentur für aeronautischen Funkbetrieb}
{\textbf{\textcolor{DARCgreen}{Zustimmung des verantwortlichen Luftfahrzeugführers}}}
{Verwendung einer fest installierten Funkstelle des mobilen Flugfunkdienstes}
{Nutzung von Frequenzen, die dem mobilen Flugfunkdienst zugewiesen sind}
\end{QQuestion}

}
\end{frame}

\begin{frame}
\frametitle{Maritim Mobil}
\begin{itemize}
  \item An Bord eines Wasserfahrzeugs (Motorboot, Segelyacht, o.ä.)
  \item Außerhalb der 12-Meilen-Zone
  \item Auf Flüssen, Seen oder ähnlichen Binnengewässern darf \enquote{/m} (mobil) verwendet werden
  \item Betrieb muss vom Schiffsführer erlaubt sein, jedoch nicht von der BNetzA genehmigt werden
  \end{itemize}
\end{frame}

\begin{frame}
\only<1>{
\begin{QQuestion}{BD205}{Was ist aus dem Rufzeichen DC4LW/mm hinsichtlich des Betriebsortes zu erkennen? Die deutsche Amateurfunkstelle DC4LW~...}{wird von einem Schiff aus betrieben, das sich auf einem Binnengewässer befindet.}
{wird an Bord eines Wasserfahrzeugs betrieben, das sich auf See befindet.}
{möchte mit anderen Funkamateuren in Kontakt treten, die ihre Funkstelle zur Zeit auch \glqq maritim mobil\grqq{} betreiben.}
{wird an Bord eines Schiffes als eine mobile Station des See- und Binnenschifffahrtsfunks betrieben.}
\end{QQuestion}

}
\only<2>{
\begin{QQuestion}{BD205}{Was ist aus dem Rufzeichen DC4LW/mm hinsichtlich des Betriebsortes zu erkennen? Die deutsche Amateurfunkstelle DC4LW~...}{wird von einem Schiff aus betrieben, das sich auf einem Binnengewässer befindet.}
{\textbf{\textcolor{DARCgreen}{wird an Bord eines Wasserfahrzeugs betrieben, das sich auf See befindet.}}}
{möchte mit anderen Funkamateuren in Kontakt treten, die ihre Funkstelle zur Zeit auch \glqq maritim mobil\grqq{} betreiben.}
{wird an Bord eines Schiffes als eine mobile Station des See- und Binnenschifffahrtsfunks betrieben.}
\end{QQuestion}

}
\end{frame}

\begin{frame}
\only<1>{
\begin{QQuestion}{VE706}{Darf eine Amateurfunkstelle auch an Bord eines Schiffes, welches sich in internationalen Gewässern befindet, betrieben werden?}{Ja, mit der Zustimmung des Schiffsführers}
{Ja, mit der Zustimmung eines beliebigen Crewmitglieds}
{Ja, mit einer Genehmigung der BNetzA}
{Ja, mit einer Genehmigung des Bundesamtes für Seeschifffahrt und Hydrographie}
\end{QQuestion}

}
\only<2>{
\begin{QQuestion}{VE706}{Darf eine Amateurfunkstelle auch an Bord eines Schiffes, welches sich in internationalen Gewässern befindet, betrieben werden?}{\textbf{\textcolor{DARCgreen}{Ja, mit der Zustimmung des Schiffsführers}}}
{Ja, mit der Zustimmung eines beliebigen Crewmitglieds}
{Ja, mit einer Genehmigung der BNetzA}
{Ja, mit einer Genehmigung des Bundesamtes für Seeschifffahrt und Hydrographie}
\end{QQuestion}

}
\end{frame}

\begin{frame}
\only<1>{
\begin{QQuestion}{VD115}{Ist für den Betrieb einer Amateurfunkstelle in einem Wasser- oder Luftfahrzeug eine Sondergenehmigung der Bundesnetzagentur erforderlich?}{Es ist keine Sondergenehmigung  erforderlich.}
{Wenn der Funkamateur auch Inhaber eines Flugfunk- oder Seefunkzeugnisses ist, benötigt er keine Sondergenehmigung.}
{Es ist in jedem Fall eine Sondergenehmigung erforderlich.}
{Bei Strahlungsleistungen von über \qty{10}{\W} EIRP ist eine Sondergenehmigung erforderlich.}
\end{QQuestion}

}
\only<2>{
\begin{QQuestion}{VD115}{Ist für den Betrieb einer Amateurfunkstelle in einem Wasser- oder Luftfahrzeug eine Sondergenehmigung der Bundesnetzagentur erforderlich?}{\textbf{\textcolor{DARCgreen}{Es ist keine Sondergenehmigung  erforderlich.}}}
{Wenn der Funkamateur auch Inhaber eines Flugfunk- oder Seefunkzeugnisses ist, benötigt er keine Sondergenehmigung.}
{Es ist in jedem Fall eine Sondergenehmigung erforderlich.}
{Bei Strahlungsleistungen von über \qty{10}{\W} EIRP ist eine Sondergenehmigung erforderlich.}
\end{QQuestion}

}
\end{frame}

\begin{frame}
\frametitle{Mobil}
\begin{itemize}
  \item In einem Landfahrzeug wie Auto oder Zug
  \item Oder an Bord eines Schiffs auf Binnengewässern
  \end{itemize}
\end{frame}

\begin{frame}
\only<1>{
\begin{QQuestion}{BD203}{Ein Rufzeichen mit dem Zusatz \glqq /m\grqq{} kann bei einer Amateurfunkstelle bedeuten, dass sie~...}{vorübergehend ortsfest betrieben wird oder tragbar ist.}
{mit minimaler Leistung sendet.}
{beweglich ist und sich in einem Landfahrzeug befindet.}
{an Bord eines Wasserfahrzeugs betrieben wird, das sich auf See befindet.}
\end{QQuestion}

}
\only<2>{
\begin{QQuestion}{BD203}{Ein Rufzeichen mit dem Zusatz \glqq /m\grqq{} kann bei einer Amateurfunkstelle bedeuten, dass sie~...}{vorübergehend ortsfest betrieben wird oder tragbar ist.}
{mit minimaler Leistung sendet.}
{\textbf{\textcolor{DARCgreen}{beweglich ist und sich in einem Landfahrzeug befindet.}}}
{an Bord eines Wasserfahrzeugs betrieben wird, das sich auf See befindet.}
\end{QQuestion}

}
\end{frame}

\begin{frame}
\only<1>{
\begin{QQuestion}{BD204}{Ein Rufzeichen mit dem Zusatz \glqq /m\grqq{} kann bei einer Amateurfunkstelle bedeuten, dass sie~...}{an Bord eines Wasserfahrzeugs betrieben wird, das sich auf See befindet.}
{mit minimaler Leistung sendet.}
{vorübergehend ortsfest betrieben wird oder tragbar ist.}
{sich an Bord eines Wasserfahrzeugs auf Binnengewässern befindet.}
\end{QQuestion}

}
\only<2>{
\begin{QQuestion}{BD204}{Ein Rufzeichen mit dem Zusatz \glqq /m\grqq{} kann bei einer Amateurfunkstelle bedeuten, dass sie~...}{an Bord eines Wasserfahrzeugs betrieben wird, das sich auf See befindet.}
{mit minimaler Leistung sendet.}
{vorübergehend ortsfest betrieben wird oder tragbar ist.}
{\textbf{\textcolor{DARCgreen}{sich an Bord eines Wasserfahrzeugs auf Binnengewässern befindet.}}}
\end{QQuestion}

}
\end{frame}

\begin{frame}
\frametitle{Portabel}
\begin{itemize}
  \item Station vorübergehend an einem Standort, der nicht auf der Zuteilungsurkunde eingetragen ist
  \item z.B. in der Natur
  \item Auch bei Bewegung (zu Fuß) mit tragbarem Funkgerät
  \end{itemize}
\end{frame}

\begin{frame}
\only<1>{
\begin{QQuestion}{BD206}{Was bedeutet der Rufzeichenzusatz \glqq /p\grqq{}? Es bedeutet, dass die Amateurfunkstelle~...}{sich an Bord eines Wasserfahrzeugs auf See befindet.}
{sich in einem Landfahrzeug in Bewegung befindet.}
{vorübergehend exterritorial betrieben wird.}
{vorübergehend ortsfest betrieben wird oder tragbar ist.}
\end{QQuestion}

}
\only<2>{
\begin{QQuestion}{BD206}{Was bedeutet der Rufzeichenzusatz \glqq /p\grqq{}? Es bedeutet, dass die Amateurfunkstelle~...}{sich an Bord eines Wasserfahrzeugs auf See befindet.}
{sich in einem Landfahrzeug in Bewegung befindet.}
{vorübergehend exterritorial betrieben wird.}
{\textbf{\textcolor{DARCgreen}{vorübergehend ortsfest betrieben wird oder tragbar ist.}}}
\end{QQuestion}

}
\end{frame}

\begin{frame}
\only<1>{
\begin{QQuestion}{BD207}{Muss beim Betrieb einer tragbaren oder vorübergehend ortsfest betriebenen Amateurfunkstelle in Deutschland dem Rufzeichen der Zusatz \glqq /p\grqq{} hinzugefügt werden?}{Nein, es sei denn, es handelt sich um eine ausländische Station.}
{Ja, damit die BNetzA erkennen kann, dass die Amateurfunkstelle nicht am gemeldeten Standort betrieben wird.}
{Ja, weil dies durch die internationalen Regelungen in den Radio Regulations (RR) so vorgegeben ist.}
{Nein, er kann zur weiteren Information verwendet werden.}
\end{QQuestion}

}
\only<2>{
\begin{QQuestion}{BD207}{Muss beim Betrieb einer tragbaren oder vorübergehend ortsfest betriebenen Amateurfunkstelle in Deutschland dem Rufzeichen der Zusatz \glqq /p\grqq{} hinzugefügt werden?}{Nein, es sei denn, es handelt sich um eine ausländische Station.}
{Ja, damit die BNetzA erkennen kann, dass die Amateurfunkstelle nicht am gemeldeten Standort betrieben wird.}
{Ja, weil dies durch die internationalen Regelungen in den Radio Regulations (RR) so vorgegeben ist.}
{\textbf{\textcolor{DARCgreen}{Nein, er kann zur weiteren Information verwendet werden.}}}
\end{QQuestion}

}
\end{frame}

\begin{frame}
\frametitle{Remote}
\begin{itemize}
  \item Betrieb an einer Remote-Station
  \item Optional \enquote{/R} bzw. \enquote{/Remote}
  \end{itemize}
\end{frame}

\begin{frame}
\only<1>{
\begin{QQuestion}{BD208}{Welcher Rufzeichenzusatz kann verwendet werden, um \glqq Remote-Betrieb\grqq{} zu kennzeichnen?}{/RB bzw. /Remotebetrieb}
{/R bzw. /Remote}
{/FB bzw. /Fernbedient}
{/F bzw. /Fern}
\end{QQuestion}

}
\only<2>{
\begin{QQuestion}{BD208}{Welcher Rufzeichenzusatz kann verwendet werden, um \glqq Remote-Betrieb\grqq{} zu kennzeichnen?}{/RB bzw. /Remotebetrieb}
{\textbf{\textcolor{DARCgreen}{/R bzw. /Remote}}}
{/FB bzw. /Fernbedient}
{/F bzw. /Fern}
\end{QQuestion}

}
\end{frame}

\begin{frame}
\frametitle{Trainee}
\begin{itemize}
  \item Bei Ausbildungsfunkbetrieb ist \enquote{/T} bzw. \enquote{/Trainee} verpflichtend
  \item Alle anderen Zusätze sind freiwillig und können weggelassen werden
  \end{itemize}
\end{frame}

\begin{frame}
\only<1>{
\begin{QQuestion}{BD209}{Der Funkamateur mit dem Rufzeichen DL1PZ möchte Ausbildungsfunkbetrieb im Sprechfunk durchführen. Welches Rufzeichen darf der Auszubildende verwenden?}{DL1PZ/Trainee}
{DL1PZ/Ausbildung}
{Ausbildung/DL1PZ}
{Trainee/DL1PZ}
\end{QQuestion}

}
\only<2>{
\begin{QQuestion}{BD209}{Der Funkamateur mit dem Rufzeichen DL1PZ möchte Ausbildungsfunkbetrieb im Sprechfunk durchführen. Welches Rufzeichen darf der Auszubildende verwenden?}{\textbf{\textcolor{DARCgreen}{DL1PZ/Trainee}}}
{DL1PZ/Ausbildung}
{Ausbildung/DL1PZ}
{Trainee/DL1PZ}
\end{QQuestion}

}
\end{frame}

\begin{frame}
\only<1>{
\begin{QQuestion}{VD306}{Von wem ist während des Ausbildungsfunkbetriebs der Rufzeichenzusatz \glqq /T\grqq{} bzw. \glqq /Trainee\grqq{} zu benutzen?}{Vom Verantwortlichen der Schulstation}
{Vom Ausbilder}
{Vom Auszubildenden und vom Ausbilder}
{Vom Auszubildenden}
\end{QQuestion}

}
\only<2>{
\begin{QQuestion}{VD306}{Von wem ist während des Ausbildungsfunkbetriebs der Rufzeichenzusatz \glqq /T\grqq{} bzw. \glqq /Trainee\grqq{} zu benutzen?}{Vom Verantwortlichen der Schulstation}
{Vom Ausbilder}
{Vom Auszubildenden und vom Ausbilder}
{\textbf{\textcolor{DARCgreen}{Vom Auszubildenden}}}
\end{QQuestion}

}
\end{frame}

\begin{frame}
\only<1>{
\begin{QQuestion}{BD210}{An der Klubstation DL0MOL soll Ausbildungsfunkbetrieb stattfinden. Darf der Auszubildende das Rufzeichen der Klubstation verwenden?}{Nein, es ist das persönliche Rufzeichen des Ausbilders zu verwenden.}
{Ja, wenn T/DL0MOL bzw. Trainee/DL0MOL verwendet wird.}
{Ja, wenn DL0MOL/T bzw. DL0MOL/Trainee verwendet wird.}
{Nein, an Klubstationen darf nicht ausgebildet werden.}
\end{QQuestion}

}
\only<2>{
\begin{QQuestion}{BD210}{An der Klubstation DL0MOL soll Ausbildungsfunkbetrieb stattfinden. Darf der Auszubildende das Rufzeichen der Klubstation verwenden?}{Nein, es ist das persönliche Rufzeichen des Ausbilders zu verwenden.}
{Ja, wenn T/DL0MOL bzw. Trainee/DL0MOL verwendet wird.}
{\textbf{\textcolor{DARCgreen}{Ja, wenn DL0MOL/T bzw. DL0MOL/Trainee verwendet wird.}}}
{Nein, an Klubstationen darf nicht ausgebildet werden.}
\end{QQuestion}

}
\end{frame}%ENDCONTENT
