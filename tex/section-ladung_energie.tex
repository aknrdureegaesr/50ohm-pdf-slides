
\section{Ladung und Energie}
\label{section:ladung_energie}
\begin{frame}%STARTCONTENT

\frametitle{Elektrische Ladung}
Strom über Zeit

$Q = I\cdot t$

in Amperesekunde (As)

\end{frame}

\begin{frame}
\only<1>{
\begin{QQuestion}{AA102}{Welche Einheit wird üblicherweise für die elektrische Ladung verwendet?}{Kilowatt (kW)}
{Amperesekunde (As)}
{Joule (J)}
{Ampere (A)}
\end{QQuestion}

}
\only<2>{
\begin{QQuestion}{AA102}{Welche Einheit wird üblicherweise für die elektrische Ladung verwendet?}{Kilowatt (kW)}
{\textbf{\textcolor{DARCgreen}{Amperesekunde (As)}}}
{Joule (J)}
{Ampere (A)}
\end{QQuestion}

}
\end{frame}

\begin{frame}
\frametitle{Elektrische Energie}
Leistung über Zeit

$W = P\cdot t$

in Joule (J) oder Wattstunden (Wh)

\end{frame}

\begin{frame}
\only<1>{
\begin{QQuestion}{AA103}{Welche Einheit wird üblicherweise für die Energie verwendet?}{Volt (V) bzw. Watt pro Ampere (W/A)}
{Joule (J) bzw. Wattstunden (Wh)}
{Watt (W) bzw. Joule pro Stunde (J/h)}
{Watt (W) bzw. Voltampere (VA)}
\end{QQuestion}

}
\only<2>{
\begin{QQuestion}{AA103}{Welche Einheit wird üblicherweise für die Energie verwendet?}{Volt (V) bzw. Watt pro Ampere (W/A)}
{\textbf{\textcolor{DARCgreen}{Joule (J) bzw. Wattstunden (Wh)}}}
{Watt (W) bzw. Joule pro Stunde (J/h)}
{Watt (W) bzw. Voltampere (VA)}
\end{QQuestion}

}
\end{frame}

\begin{frame}
\only<1>{
\begin{QQuestion}{AB502}{Eine Stromversorgung nimmt bei einer Spannung von \qty{230}{\V} einen Strom von \qty{0,63}{\A} auf. Wieviel Energie wird bei einer Betriebsdauer von 7 Stunden umgesetzt?}{\qty{1,01}{\kWh}}
{\qty{0,14}{\kWh}}
{\qty{2,56}{\kWh}}
{\qty{20,7}{\kWh}}
\end{QQuestion}

}
\only<2>{
\begin{QQuestion}{AB502}{Eine Stromversorgung nimmt bei einer Spannung von \qty{230}{\V} einen Strom von \qty{0,63}{\A} auf. Wieviel Energie wird bei einer Betriebsdauer von 7 Stunden umgesetzt?}{\textbf{\textcolor{DARCgreen}{\qty{1,01}{\kWh}}}}
{\qty{0,14}{\kWh}}
{\qty{2,56}{\kWh}}
{\qty{20,7}{\kWh}}
\end{QQuestion}

}
\end{frame}

\begin{frame}
\frametitle{Lösungweg}
\begin{itemize}
  \item gegeben: $U = 230V$
  \item gegeben: $I = 0,63A$
  \item gegeben: $t = 7h$
  \item gesucht: $W$
  \end{itemize}
    \pause
    \begin{equation} \nonumber W = P\cdot t = U\cdot I\cdot t = 230V\cdot 0,63A\cdot 7h = 1,01kWh \end{equation}



\end{frame}

\begin{frame}
\only<1>{
\begin{PQuestion}{AB503}{Wie viel Energie wird vom Widerstand innerhalb einer Stunde in Wärme umgewandelt?}{\qty{0,5}{\W\hour} bzw. \qty{1800}{\J}}
{\qty{2}{\W\hour} bzw. \qty{7200}{\J}}
{\qty{0,1}{\W\hour} bzw. \qty{360}{\J}}
{\qty{1}{\W\hour} bzw. \qty{3600}{\J}}
{\DARCimage{1.0\linewidth}{556include}}\end{PQuestion}

}
\only<2>{
\begin{PQuestion}{AB503}{Wie viel Energie wird vom Widerstand innerhalb einer Stunde in Wärme umgewandelt?}{\qty{0,5}{\W\hour} bzw. \qty{1800}{\J}}
{\qty{2}{\W\hour} bzw. \qty{7200}{\J}}
{\qty{0,1}{\W\hour} bzw. \qty{360}{\J}}
{\textbf{\textcolor{DARCgreen}{\qty{1}{\W\hour} bzw. \qty{3600}{\J}}}}
{\DARCimage{1.0\linewidth}{556include}}\end{PQuestion}

}
\end{frame}

\begin{frame}
\frametitle{Lösungsweg}
\begin{itemize}
  \item gegeben: $U = 10V$
  \item gegeben: $R = 100\Omega$
  \item gegeben: $t = 1h$
  \item gesucht: $W$
  \end{itemize}
    \pause
    \begin{equation} \nonumber W = P\cdot t = \frac{U^2}{R} \cdot t = \frac{(10V)^2}{100\Omega}\cdot 1h = 1Wh \end{equation}



\end{frame}%ENDCONTENT
