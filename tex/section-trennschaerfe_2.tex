
\section{Trennschärfe II}
\label{section:trennschaerfe_2}
\begin{frame}%STARTCONTENT

\only<1>{
\begin{QQuestion}{AF208}{Welches der folgenden Bandpassfilter verfügt bei jeweils gleicher Mittenfrequenz am ehesten über die geringste Bandbreite und höchste Flankensteilheit?}{RC-Filter}
{LC-Filter}
{Keramikfilter}
{Quarzfilter}
\end{QQuestion}

}
\only<2>{
\begin{QQuestion}{AF208}{Welches der folgenden Bandpassfilter verfügt bei jeweils gleicher Mittenfrequenz am ehesten über die geringste Bandbreite und höchste Flankensteilheit?}{RC-Filter}
{LC-Filter}
{Keramikfilter}
{\textbf{\textcolor{DARCgreen}{Quarzfilter}}}
\end{QQuestion}

}
\end{frame}

\begin{frame}
\only<1>{
\begin{QQuestion}{AF206}{Welche ungefähren Werte sollte die Bandbreite der ZF-Verstärker eines Amateurfunkempfängers für folgende Übertragungsverfahren aufweisen: SSB-Sprechfunk, RTTY (Shift \qty{170}{\Hz}), FM-Sprechfunk?}{SSB:~\qty{2,7}{\kHz}; RTTY:~\qty{340}{\Hz}; FM:~\qty{3,6}{\kHz}}
{SSB:~\qty{6}{\kHz}; RTTY:~\qty{1,5}{\kHz}; FM:~\qty{12}{\kHz}}
{SSB:~\qty{2,7}{\kHz}; RTTY:~\qty{500}{\Hz}; FM:~\qty{12}{\kHz}}
{SSB:~\qty{3,6}{\kHz}; RTTY:~\qty{170}{\Hz}; FM:~\qty{25}{\kHz}}
\end{QQuestion}

}
\only<2>{
\begin{QQuestion}{AF206}{Welche ungefähren Werte sollte die Bandbreite der ZF-Verstärker eines Amateurfunkempfängers für folgende Übertragungsverfahren aufweisen: SSB-Sprechfunk, RTTY (Shift \qty{170}{\Hz}), FM-Sprechfunk?}{SSB:~\qty{2,7}{\kHz}; RTTY:~\qty{340}{\Hz}; FM:~\qty{3,6}{\kHz}}
{SSB:~\qty{6}{\kHz}; RTTY:~\qty{1,5}{\kHz}; FM:~\qty{12}{\kHz}}
{\textbf{\textcolor{DARCgreen}{SSB:~\qty{2,7}{\kHz}; RTTY:~\qty{500}{\Hz}; FM:~\qty{12}{\kHz}}}}
{SSB:~\qty{3,6}{\kHz}; RTTY:~\qty{170}{\Hz}; FM:~\qty{25}{\kHz}}
\end{QQuestion}

}
\end{frame}

\begin{frame}
\only<1>{
\begin{QQuestion}{AF205}{Welche Baugruppe eines Empfängers bestimmt die Trennschärfe?}{Der Oszillatorschwingkreis in der Mischstufe}
{Das Oberwellenfilter im ZF-Verstärker}
{Die Filter im ZF-Verstärker}
{Die PLL-Frequenzaufbereitung}
\end{QQuestion}

}
\only<2>{
\begin{QQuestion}{AF205}{Welche Baugruppe eines Empfängers bestimmt die Trennschärfe?}{Der Oszillatorschwingkreis in der Mischstufe}
{Das Oberwellenfilter im ZF-Verstärker}
{\textbf{\textcolor{DARCgreen}{Die Filter im ZF-Verstärker}}}
{Die PLL-Frequenzaufbereitung}
\end{QQuestion}

}
\end{frame}

\begin{frame}
\only<1>{
\begin{PQuestion}{AF207}{Für welche Signale ist die untenstehende Durchlasskurve eines Empfängerfilters geeignet?}{FM-Signale}
{AM-Signale}
{OFDM-Signale}
{SSB-Signale}
{\DARCimage{1.0\linewidth}{38include}}\end{PQuestion}

}
\only<2>{
\begin{PQuestion}{AF207}{Für welche Signale ist die untenstehende Durchlasskurve eines Empfängerfilters geeignet?}{FM-Signale}
{AM-Signale}
{OFDM-Signale}
{\textbf{\textcolor{DARCgreen}{SSB-Signale}}}
{\DARCimage{1.0\linewidth}{38include}}\end{PQuestion}

}
\end{frame}%ENDCONTENT
