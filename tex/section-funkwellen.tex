
\section{Funkwellen}
\label{section:funkwellen}
\begin{frame}%STARTCONTENT

\frametitle{Antenne}
\begin{itemize}
  \item Eine elektrische Schwingung an einer Antenne wird als Funkwelle abgestrahlt
  \item Funkwellen sind elektromagnetische Wellen
  \item Sie breiten sich mit Lichtgeschwindigkeit aus
  \item Lichtgeschwindigkeit im Freiraum: etwa 300.000 Kilometer pro Sekunde
  \end{itemize}
\end{frame}

\begin{frame}
\only<1>{
\begin{QQuestion}{NB301}{Die Ausbreitungsgeschwindigkeit elektromagnetischer Wellen beträgt im Freiraum etwa~...}{\qty{3000000}{\km}/s.}
{\qty{300000}{\km}/s.}
{\qty{30000}{\km}/s.}
{\qty{3000}{\km}/s.}
\end{QQuestion}

}
\only<2>{
\begin{QQuestion}{NB301}{Die Ausbreitungsgeschwindigkeit elektromagnetischer Wellen beträgt im Freiraum etwa~...}{\qty{3000000}{\km}/s.}
{\textbf{\textcolor{DARCgreen}{\qty{300000}{\km}/s.}}}
{\qty{30000}{\km}/s.}
{\qty{3000}{\km}/s.}
\end{QQuestion}

}
\end{frame}

\begin{frame}
\frametitle{Funkwellen}
\begin{itemize}
  \item Bestehen aus Wellenbergen und Wellentälern
  \item Stellen die Stärke des Funksignals dar
  \item Das entspricht der \emph{Feldstärke}
  \end{itemize}

\end{frame}

\begin{frame}
\only<1>{
\begin{PQuestion}{NB402}{Was ist in der dargestellten Momentaufnahme einer Welle mit 1 markiert?}{Amplitude}
{Frequenz}
{Periode}
{Wellenlänge}
{\DARCimage{1.0\linewidth}{628include}}\end{PQuestion}

}
\only<2>{
\begin{PQuestion}{NB402}{Was ist in der dargestellten Momentaufnahme einer Welle mit 1 markiert?}{\textbf{\textcolor{DARCgreen}{Amplitude}}}
{Frequenz}
{Periode}
{Wellenlänge}
{\DARCimage{1.0\linewidth}{628include}}\end{PQuestion}

}
\end{frame}%ENDCONTENT
