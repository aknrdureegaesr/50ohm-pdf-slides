
\section{Transistor II}
\label{section:transistor_2}
\begin{frame}%STARTCONTENT

\frametitle{Bipolarer Transistor}
\end{frame}

\begin{frame}
\only<1>{
\begin{QQuestion}{AC501}{Ein bipolarer Transistor ist~...}{thermisch gesteuert.}
{spannungsgesteuert.}
{stromgesteuert.}
{feldgesteuert.}
\end{QQuestion}

}
\only<2>{
\begin{QQuestion}{AC501}{Ein bipolarer Transistor ist~...}{thermisch gesteuert.}
{spannungsgesteuert.}
{\textbf{\textcolor{DARCgreen}{stromgesteuert.}}}
{feldgesteuert.}
\end{QQuestion}

}
\end{frame}

\begin{frame}
\only<1>{
\begin{QQuestion}{AC503}{Mit welchem Anschluss ist der p-dotierte Bereich eines NPN-Transistors verbunden?}{Gehäuse}
{Kollektor}
{Emitter}
{Basis}
\end{QQuestion}

}
\only<2>{
\begin{QQuestion}{AC503}{Mit welchem Anschluss ist der p-dotierte Bereich eines NPN-Transistors verbunden?}{Gehäuse}
{Kollektor}
{Emitter}
{\textbf{\textcolor{DARCgreen}{Basis}}}
\end{QQuestion}

}
\end{frame}

\begin{frame}
\only<1>{
\begin{QQuestion}{AC504}{Mit welchem Anschluss ist der n-dotierte Bereich eines PNP-Transistors verbunden?}{Kollektor}
{Emitter}
{Basis}
{Gehäuse}
\end{QQuestion}

}
\only<2>{
\begin{QQuestion}{AC504}{Mit welchem Anschluss ist der n-dotierte Bereich eines PNP-Transistors verbunden?}{Kollektor}
{Emitter}
{\textbf{\textcolor{DARCgreen}{Basis}}}
{Gehäuse}
\end{QQuestion}

}
\end{frame}

\begin{frame}
\only<1>{
\begin{QQuestion}{AC505}{Bei einem bipolaren Transistor in leitendem Zustand befindet sich der Basis-Emitter-PN-Übergang~...}{in Sperrrichtung.}
{im Leerlauf.}
{im Kurzschluss.}
{in Durchlassrichtung.}
\end{QQuestion}

}
\only<2>{
\begin{QQuestion}{AC505}{Bei einem bipolaren Transistor in leitendem Zustand befindet sich der Basis-Emitter-PN-Übergang~...}{in Sperrrichtung.}
{im Leerlauf.}
{im Kurzschluss.}
{\textbf{\textcolor{DARCgreen}{in Durchlassrichtung.}}}
\end{QQuestion}

}
\end{frame}

\begin{frame}
\frametitle{Rechnungen}
\end{frame}

\begin{frame}
\only<1>{
\begin{PQuestion}{AC515}{Die Betriebsspannung beträgt \qty{12}{\V}, der Kollektorstrom soll \qty{5}{\mA} betragen, die Gleichstromverstärkung des Transistors beträgt 298. Berechnen Sie den Vorwiderstand $R_1$.}{ca. \qty{680}{\kohm}}
{ca. \qty{715}{\kohm}}
{ca. \qty{68}{\kohm}}
{ca. \qty{2,3}{\kohm}}
{\DARCimage{1.0\linewidth}{360include}}\end{PQuestion}

}
\only<2>{
\begin{PQuestion}{AC515}{Die Betriebsspannung beträgt \qty{12}{\V}, der Kollektorstrom soll \qty{5}{\mA} betragen, die Gleichstromverstärkung des Transistors beträgt 298. Berechnen Sie den Vorwiderstand $R_1$.}{\textbf{\textcolor{DARCgreen}{ca. \qty{680}{\kohm}}}}
{ca. \qty{715}{\kohm}}
{ca. \qty{68}{\kohm}}
{ca. \qty{2,3}{\kohm}}
{\DARCimage{1.0\linewidth}{360include}}\end{PQuestion}

}
\end{frame}

\begin{frame}
\frametitle{Lösungsweg}
\begin{itemize}
  \item gegeben: $U = 12V$
  \item gegeben: $I_{\textrm{C}} = 5mA$
  \item gegeben: $B = 298$
  \item gegeben: $U_{\textrm{BE}} = 0,6V$
  \item gesucht: $R_1$
  \end{itemize}
    \pause
    $B = \frac{I_{\textrm{C}}}{I_{\textrm{B}}} \Rightarrow I_{\textrm{B}} = \frac{I_{\textrm{C}}}{B} = \frac{5mA}{298} = 16,779\mu A$
    \pause
    $R_1 = \frac{U-U_{\textrm{BE}}}{I_{\textrm{B}}} = \frac{12V -- 0,6V}{16,779\mu A} \approx 680k\Omega$



\end{frame}

\begin{frame}
\only<1>{
\begin{PQuestion}{AC518}{Die Betriebsspannung beträgt \qty{10}{\V}, der Kollektorstrom soll \qty{2}{\mA} betragen, die Gleichstromverstärkung des Transistors beträgt 200. Durch den Querwiderstand $R_2$ soll der zehnfache Basisstrom fließen. Berechnen Sie den Vorwiderstand $R_1$.}{ca. \qty{76,4}{\kohm}}
{ca. \qty{940}{\kohm}}
{ca. \qty{85,5}{\kohm}}
{ca. \qty{540}{\kohm}}
{\DARCimage{1.0\linewidth}{361include}}\end{PQuestion}

}
\only<2>{
\begin{PQuestion}{AC518}{Die Betriebsspannung beträgt \qty{10}{\V}, der Kollektorstrom soll \qty{2}{\mA} betragen, die Gleichstromverstärkung des Transistors beträgt 200. Durch den Querwiderstand $R_2$ soll der zehnfache Basisstrom fließen. Berechnen Sie den Vorwiderstand $R_1$.}{ca. \qty{76,4}{\kohm}}
{ca. \qty{940}{\kohm}}
{\textbf{\textcolor{DARCgreen}{ca. \qty{85,5}{\kohm}}}}
{ca. \qty{540}{\kohm}}
{\DARCimage{1.0\linewidth}{361include}}\end{PQuestion}

}
\end{frame}

\begin{frame}
\frametitle{Lösungsweg}
\begin{columns}
    \begin{column}{0.48\textwidth}
    \begin{itemize}
  \item gegeben: $U = 10V$
  \item gegeben: $I_{\textrm{C}} = 2mA$
  \item gegeben: $B = 200$
  \end{itemize}

    \end{column}
   \begin{column}{0.48\textwidth}
       \begin{itemize}
  \item gegeben: $U_{\textrm{R2}} = 0,6$
  \item gegeben: $I_{\textrm{R2}} = 10 \cdot I_{\textrm{B}}$
  \item gesucht: $R_1$
  \end{itemize}

   \end{column}
\end{columns}
    \pause
    $B = \frac{I_{\textrm{C}}}{I_{\textrm{B}}} \Rightarrow I_{\textrm{B}} = \frac{I_{\textrm{C}}}{B} = \frac{2mA}{200} = 10\mu A$
    \pause
    $U_{\textrm{R1}} = U -- U_{\textrm{R2}} = 10V -- 0,6V = 9,4V$
    \pause
    $I_{\textrm{R1}} = I_{\textrm{B}} + I_{\textrm{R2}} = I_{\textrm{B}} + 10 \cdot I_{\textrm{B}} = 110\mu A$
    \pause
    $R_1 = \frac{U_{\textrm{R1}}}{I_{\textrm{R1}}} = \frac{9,4V}{110\mu A} \approx 85,5k\Omega$



\end{frame}

\begin{frame}
\only<1>{
\begin{PQuestion}{AC517}{Die Betriebsspannung beträgt \qty{10}{\V}, der Kollektorstrom soll \qty{2}{\mA} betragen, die Gleichstromverstärkung des Transistors beträgt 200. Durch den Querwiderstand $R_2$ soll der zehnfache Basisstrom fließen. Am Emitterwiderstand soll \qty{1}{\V} abfallen. Berechnen Sie den Vorwiderstand $R_1$.}{ca. \qty{540}{\kohm}}
{ca. \qty{76,4}{\kohm}}
{ca. \qty{85,5}{\kohm}}
{ca. \qty{940}{\kohm}}
{\DARCimage{1.0\linewidth}{362include}}\end{PQuestion}

}
\only<2>{
\begin{PQuestion}{AC517}{Die Betriebsspannung beträgt \qty{10}{\V}, der Kollektorstrom soll \qty{2}{\mA} betragen, die Gleichstromverstärkung des Transistors beträgt 200. Durch den Querwiderstand $R_2$ soll der zehnfache Basisstrom fließen. Am Emitterwiderstand soll \qty{1}{\V} abfallen. Berechnen Sie den Vorwiderstand $R_1$.}{ca. \qty{540}{\kohm}}
{\textbf{\textcolor{DARCgreen}{ca. \qty{76,4}{\kohm}}}}
{ca. \qty{85,5}{\kohm}}
{ca. \qty{940}{\kohm}}
{\DARCimage{1.0\linewidth}{362include}}\end{PQuestion}

}
\end{frame}

\begin{frame}
\frametitle{Lösungsweg}
\begin{columns}
    \begin{column}{0.48\textwidth}
    \begin{itemize}
  \item gegeben: $U = 10V$
  \item gegeben: $I_{\textrm{C}} = 2mA$
  \item gegeben: $B = 200$
  \end{itemize}

    \end{column}
   \begin{column}{0.48\textwidth}
       \begin{itemize}
  \item gegeben: $U_{\textrm{BE}} = 0,6V$
  \item gegeben: $U_{\textrm{RE}} = 1V$
  \item gegeben: $I_{\textrm{R2}} = 10 \cdot I_{\textrm{B}}$
  \end{itemize}

   \end{column}
\end{columns}

\begin{itemize}
  \item gesucht: $R_1$
  \end{itemize}
    \pause
    $B = \frac{I_{\textrm{C}}}{I_{\textrm{B}}} \Rightarrow I_{\textrm{B}} = \frac{I_{\textrm{C}}}{B} = \frac{2mA}{200} = 10\mu A$
    \pause
    $U_{\textrm{R2}} = U_{\textrm{BE}} + U_{R_{\textrm{E}}} = 0,6V + 1V = 1,6V$
    \pause
    $U_{\textrm{R1}} = U -- U_{\textrm{R2}} = 10V -- 1,6V = 8,4V$
    \pause
    $I_{\textrm{R1}} = I_{\textrm{B}} + I_{\textrm{R2}} = I_{\textrm{B}} + 10 \cdot I_{\textrm{B}} = 110\mu A$
    \pause
    $R_1 = \frac{U_{\textrm{R1}}}{I_{\textrm{R1}}} = \frac{8,4V}{110\mu A} \approx 76,4k\Omega$



\end{frame}

\begin{frame}
\only<1>{
\begin{PQuestion}{AC516}{Warum soll bei dem gezeigten Basisspannungsteiler der Strom durch $R_2$ etwa 10-mal größer als der Basisstrom sein?}{Damit sich der Basisstrom bei Erwärmung nicht ändert.}
{Damit der Arbeitspunkt stabil bleibt.}
{Damit $R_2$ eine Stromgegenkopplung bewirkt.}
{Damit $R_2$ eine Spannungsgegenkopplung bewirkt}
{\DARCimage{1.0\linewidth}{361include}}\end{PQuestion}

}
\only<2>{
\begin{PQuestion}{AC516}{Warum soll bei dem gezeigten Basisspannungsteiler der Strom durch $R_2$ etwa 10-mal größer als der Basisstrom sein?}{Damit sich der Basisstrom bei Erwärmung nicht ändert.}
{\textbf{\textcolor{DARCgreen}{Damit der Arbeitspunkt stabil bleibt.}}}
{Damit $R_2$ eine Stromgegenkopplung bewirkt.}
{Damit $R_2$ eine Spannungsgegenkopplung bewirkt}
{\DARCimage{1.0\linewidth}{361include}}\end{PQuestion}

}
\end{frame}

\begin{frame}
\only<1>{
\begin{PQuestion}{AC519}{Was passiert, wenn der Widerstand $R_1$ durch eine fehlerhafte Lötstelle an einer Seite keinen Kontakt mehr zur Schaltung hat? Welche Beschreibung trifft zu?}{Es fließt Kurzschlussstrom. Der Transistor wird zerstört.}
{Es fließt kein Kollektorstrom mehr. Die Kollektorspannung steigt auf die Betriebsspannung an.}
{Der Kollektorstrom wird nur durch $R_{\symup{C}}$ begrenzt. Die Kollektorspannung sinkt auf zirka \qty{0,1}{\V}.}
{Der Kollektorstrom steigt stark an. Die Kollektorspannung erhöht sich.}
{\DARCimage{1.0\linewidth}{365include}}\end{PQuestion}

}
\only<2>{
\begin{PQuestion}{AC519}{Was passiert, wenn der Widerstand $R_1$ durch eine fehlerhafte Lötstelle an einer Seite keinen Kontakt mehr zur Schaltung hat? Welche Beschreibung trifft zu?}{Es fließt Kurzschlussstrom. Der Transistor wird zerstört.}
{\textbf{\textcolor{DARCgreen}{Es fließt kein Kollektorstrom mehr. Die Kollektorspannung steigt auf die Betriebsspannung an.}}}
{Der Kollektorstrom wird nur durch $R_{\symup{C}}$ begrenzt. Die Kollektorspannung sinkt auf zirka \qty{0,1}{\V}.}
{Der Kollektorstrom steigt stark an. Die Kollektorspannung erhöht sich.}
{\DARCimage{1.0\linewidth}{365include}}\end{PQuestion}

}
\end{frame}

\begin{frame}
\only<1>{
\begin{PQuestion}{AC520}{Was passiert, wenn der Widerstand $R_2$ durch eine fehlerhafte Lötstelle an einer Seite keinen Kontakt mehr zur Schaltung hat? In welcher Antwort sind beide Aussagen richtig?}{Es fließt kein Kollektorstrom mehr. Die Kollektorspannung steigt auf die Betriebsspannung an.}
{Es fließt Kurzschlussstrom. Der Transistor wird zerstört.}
{Der Kollektorstrom wird nur durch $R_{\symup{C}}$ begrenzt. Die Kollektorspannung sinkt auf zirka \qty{0,1}{\V}.}
{Der Kollektorstrom steigt stark an. Die Kollektorspannung erhöht sich.}
{\DARCimage{1.0\linewidth}{364include}}\end{PQuestion}

}
\only<2>{
\begin{PQuestion}{AC520}{Was passiert, wenn der Widerstand $R_2$ durch eine fehlerhafte Lötstelle an einer Seite keinen Kontakt mehr zur Schaltung hat? In welcher Antwort sind beide Aussagen richtig?}{Es fließt kein Kollektorstrom mehr. Die Kollektorspannung steigt auf die Betriebsspannung an.}
{Es fließt Kurzschlussstrom. Der Transistor wird zerstört.}
{\textbf{\textcolor{DARCgreen}{Der Kollektorstrom wird nur durch $R_{\symup{C}}$ begrenzt. Die Kollektorspannung sinkt auf zirka \qty{0,1}{\V}.}}}
{Der Kollektorstrom steigt stark an. Die Kollektorspannung erhöht sich.}
{\DARCimage{1.0\linewidth}{364include}}\end{PQuestion}

}
\end{frame}

\begin{frame}
\frametitle{Feldeffekttransistor}
\end{frame}

\begin{frame}
\only<1>{
\begin{QQuestion}{AC502}{Ein Feldeffekttransistor ist~...}{stromgesteuert.}
{spannungsgesteuert.}
{leistungsgesteuert.}
{optisch gesteuert.}
\end{QQuestion}

}
\only<2>{
\begin{QQuestion}{AC502}{Ein Feldeffekttransistor ist~...}{stromgesteuert.}
{\textbf{\textcolor{DARCgreen}{spannungsgesteuert.}}}
{leistungsgesteuert.}
{optisch gesteuert.}
\end{QQuestion}

}
\end{frame}

\begin{frame}
\only<1>{
\begin{PQuestion}{AC506}{Welches Bauteil wird durch das Schaltzeichen symbolisiert?}{Lautsprecher}
{Bipolartransistor}
{Diode}
{Feldeffekttransistor}
{\DARCimage{0.25\linewidth}{561include}}\end{PQuestion}

}
\only<2>{
\begin{PQuestion}{AC506}{Welches Bauteil wird durch das Schaltzeichen symbolisiert?}{Lautsprecher}
{Bipolartransistor}
{Diode}
{\textbf{\textcolor{DARCgreen}{Feldeffekttransistor}}}
{\DARCimage{0.25\linewidth}{561include}}\end{PQuestion}

}
\end{frame}

\begin{frame}
\only<1>{
\begin{PQuestion}{AC513}{Wie bezeichnet man die Anschlüsse des abgebildeten Transistors?}{1: Drain, 2: Source, 3: Gate}
{1: Anode, 2: Kollektor, 3: Gate}
{1: Anode, 2: Kathode, 3: Gate}
{1: Kollektor, 2: Emitter, 3: Basis}
{\DARCimage{0.25\linewidth}{376include}}\end{PQuestion}

}
\only<2>{
\begin{PQuestion}{AC513}{Wie bezeichnet man die Anschlüsse des abgebildeten Transistors?}{\textbf{\textcolor{DARCgreen}{1: Drain, 2: Source, 3: Gate}}}
{1: Anode, 2: Kollektor, 3: Gate}
{1: Anode, 2: Kathode, 3: Gate}
{1: Kollektor, 2: Emitter, 3: Basis}
{\DARCimage{0.25\linewidth}{376include}}\end{PQuestion}

}
\end{frame}

\begin{frame}
\only<1>{
\begin{QQuestion}{AC512}{Wie lauten die Bezeichnungen der Anschlüsse eines Feldeffekttransistors?}{Emitter, Basis, Kollektor}
{Drain, Gate, Source}
{Emitter, Drain, Source}
{Gate, Source, Kollektor}
\end{QQuestion}

}
\only<2>{
\begin{QQuestion}{AC512}{Wie lauten die Bezeichnungen der Anschlüsse eines Feldeffekttransistors?}{Emitter, Basis, Kollektor}
{\textbf{\textcolor{DARCgreen}{Drain, Gate, Source}}}
{Emitter, Drain, Source}
{Gate, Source, Kollektor}
\end{QQuestion}

}
\end{frame}

\begin{frame}
\only<1>{
\begin{QQuestion}{AC514}{Wie erfolgt die Steuerung des Stroms im Feldeffekttransistor (FET)?}{Die Gate-Source-Spannung steuert den Widerstand des Kanals zwischen Source und Drain.}
{Die Gate-Source-Spannung steuert den Gatestrom.}
{Der Gatestrom steuert den Drainstrom.}
{Der Gatestrom steuert den Widerstand des Kanals zwischen Source und Drain.}
\end{QQuestion}

}
\only<2>{
\begin{QQuestion}{AC514}{Wie erfolgt die Steuerung des Stroms im Feldeffekttransistor (FET)?}{\textbf{\textcolor{DARCgreen}{Die Gate-Source-Spannung steuert den Widerstand des Kanals zwischen Source und Drain.}}}
{Die Gate-Source-Spannung steuert den Gatestrom.}
{Der Gatestrom steuert den Drainstrom.}
{Der Gatestrom steuert den Widerstand des Kanals zwischen Source und Drain.}
\end{QQuestion}

}
\end{frame}

\begin{frame}
\frametitle{Bauarten FET}
\end{frame}

\begin{frame}
\only<1>{
\begin{PQuestion}{AC507}{Welche Bezeichnungen für die Bauelemente sind richtig?}{1: Selbstleitender N-Kanal-Sperrschicht-FET
2: Selbstleitender P-Kanal-Sperrschicht-FET}
{1: Selbstsperrender N-Kanal-Sperrschicht-FET
2: Selbstsperrender P-Kanal-Sperrschicht-FET}
{1: Selbstleitender P-Kanal-Sperrschicht-FET
2: Selbstleitender N-Kanal-Sperrschicht-FET}
{1: Selbstsperrender P-Kanal-Sperrschicht-FET
2: Selbstsperrender N-Kanal-Sperrschicht-FET}
{\DARCimage{1.0\linewidth}{271include}}\end{PQuestion}

}
\only<2>{
\begin{PQuestion}{AC507}{Welche Bezeichnungen für die Bauelemente sind richtig?}{\textbf{\textcolor{DARCgreen}{1: Selbstleitender N-Kanal-Sperrschicht-FET
2: Selbstleitender P-Kanal-Sperrschicht-FET}}}
{1: Selbstsperrender N-Kanal-Sperrschicht-FET
2: Selbstsperrender P-Kanal-Sperrschicht-FET}
{1: Selbstleitender P-Kanal-Sperrschicht-FET
2: Selbstleitender N-Kanal-Sperrschicht-FET}
{1: Selbstsperrender P-Kanal-Sperrschicht-FET
2: Selbstsperrender N-Kanal-Sperrschicht-FET}
{\DARCimage{1.0\linewidth}{271include}}\end{PQuestion}

}
\end{frame}

\begin{frame}
\only<1>{
\begin{PQuestion}{AC508}{Der folgende Transistor ist ein~...}{Selbstleitender P-Kanal-Isolierschicht-FET (MOSFET).}
{Selbstsperrender P-Kanal-Isolierschicht-FET (MOSFET).}
{Selbstleitender N-Kanal-Isolierschicht-FET (MOSFET).}
{Selbstsperrender N-Kanal-Isolierschicht-FET (MOSFET).}
{\DARCimage{0.25\linewidth}{272include}}\end{PQuestion}

}
\only<2>{
\begin{PQuestion}{AC508}{Der folgende Transistor ist ein~...}{Selbstleitender P-Kanal-Isolierschicht-FET (MOSFET).}
{Selbstsperrender P-Kanal-Isolierschicht-FET (MOSFET).}
{Selbstleitender N-Kanal-Isolierschicht-FET (MOSFET).}
{\textbf{\textcolor{DARCgreen}{Selbstsperrender N-Kanal-Isolierschicht-FET (MOSFET).}}}
{\DARCimage{0.25\linewidth}{272include}}\end{PQuestion}

}
\end{frame}

\begin{frame}
\only<1>{
\begin{question2x2}{AC509}{Welcher der folgenden Transistoren ist ein selbstsperrender N-Kanal-MOSFET?}{\DARCimage{1.0\linewidth}{275include}}
{\DARCimage{1.0\linewidth}{274include}}
{\DARCimage{1.0\linewidth}{273include}}
{\DARCimage{1.0\linewidth}{276include}}
\end{question2x2}

}
\only<2>{
\begin{question2x2}{AC509}{Welcher der folgenden Transistoren ist ein selbstsperrender N-Kanal-MOSFET?}{\DARCimage{1.0\linewidth}{275include}}
{\DARCimage{1.0\linewidth}{274include}}
{\DARCimage{1.0\linewidth}{273include}}
{\textbf{\textcolor{DARCgreen}{\DARCimage{1.0\linewidth}{276include}}}}
\end{question2x2}

}
\end{frame}

\begin{frame}
\only<1>{
\begin{question2x2}{AC510}{Welcher der folgenden Transistoren ist ein selbstleitender N-Kanal-MOSFET?}{\DARCimage{1.0\linewidth}{275include}}
{\DARCimage{1.0\linewidth}{273include}}
{\DARCimage{1.0\linewidth}{274include}}
{\DARCimage{1.0\linewidth}{276include}}
\end{question2x2}

}
\only<2>{
\begin{question2x2}{AC510}{Welcher der folgenden Transistoren ist ein selbstleitender N-Kanal-MOSFET?}{\DARCimage{1.0\linewidth}{275include}}
{\DARCimage{1.0\linewidth}{273include}}
{\textbf{\textcolor{DARCgreen}{\DARCimage{1.0\linewidth}{274include}}}}
{\DARCimage{1.0\linewidth}{276include}}
\end{question2x2}

}
\end{frame}

\begin{frame}
\only<1>{
\begin{question2x2}{AC511}{Welcher der folgenden Transistoren ist ein selbstleitender P-Kanal-MOSFET?}{\DARCimage{1.0\linewidth}{274include}}
{\DARCimage{1.0\linewidth}{273include}}
{\DARCimage{1.0\linewidth}{276include}}
{\DARCimage{1.0\linewidth}{275include}}
\end{question2x2}

}
\only<2>{
\begin{question2x2}{AC511}{Welcher der folgenden Transistoren ist ein selbstleitender P-Kanal-MOSFET?}{\DARCimage{1.0\linewidth}{274include}}
{\textbf{\textcolor{DARCgreen}{\DARCimage{1.0\linewidth}{273include}}}}
{\DARCimage{1.0\linewidth}{276include}}
{\DARCimage{1.0\linewidth}{275include}}
\end{question2x2}

}
\end{frame}

\begin{frame}
\frametitle{Rechnungen}
\end{frame}

\begin{frame}
\only<1>{
\begin{PQuestion}{AC521}{Wie groß ist die Gate-Source-Spannung in der gezeichneten Schaltung? $U_{\symup{B}} = \qty{44}{\V}$; $R_1 = 10~k\Omega$; $R_2 = 1~k\Omega$; $R_3 = 2,2~k\Omega$~...}{\qty{4}{\V}}
{\qty{8}{\V}}
{\qty{0,7}{\V} }
{\qty{4,4}{\V}}
{\DARCimage{0.5\linewidth}{345include}}\end{PQuestion}

}
\only<2>{
\begin{PQuestion}{AC521}{Wie groß ist die Gate-Source-Spannung in der gezeichneten Schaltung? $U_{\symup{B}} = \qty{44}{\V}$; $R_1 = 10~k\Omega$; $R_2 = 1~k\Omega$; $R_3 = 2,2~k\Omega$~...}{\textbf{\textcolor{DARCgreen}{\qty{4}{\V}}}}
{\qty{8}{\V}}
{\qty{0,7}{\V} }
{\qty{4,4}{\V}}
{\DARCimage{0.5\linewidth}{345include}}\end{PQuestion}

}
\end{frame}

\begin{frame}
\frametitle{Lösungsweg}
\begin{columns}
    \begin{column}{0.48\textwidth}
    \begin{itemize}
  \item gegeben: $U_{\textrm{B}} = 44V$
  \item gegeben: $R_1 = 10k\Omega$
  \item gegeben: $R_2 = 1k\Omega$
  \item gegeben: $R_3 = 2,2k\Omega$
  \item gesucht: $U_{\textrm{GS}}$
  \item Ansatz: Spannungsteiler über $R_1$ und $R_2$, mit $U_{\textrm{GS}} = U_{\textrm{R2}}$
  \end{itemize}

    \end{column}
   \begin{column}{0.48\textwidth}
       
    \pause
    \begin{equation}\begin{align}\nonumber \frac{U_{\textrm{R2}}}{U_{\textrm{B}}} &= \frac{R_2}{R_1+R_2}\\ \nonumber \Rightarrow U_{\textrm{R2}} &= \frac{R_2}{R_1+R_2} \cdot U_{\textrm{G}}\\ \nonumber &= \frac{1k\Omega}{10k\Omega+1k\Omega} \cdot 44V\\ \nonumber &= \frac{1}{11} \cdot 44V = 4V \end{align}\end{equation}




   \end{column}
\end{columns}

\end{frame}

\begin{frame}
\only<1>{
\begin{PQuestion}{AC522}{Wie groß muss $R_2$ gewählt werden, damit sich eine Spannung von \qty{2,8}{\V} zwischen Gate und Source einstellt? $U_{\symup{B}}$=\qty{44}{\V}; $R_1$=\qty{10}{\kohm}; $R_3$=\qty{2,2}{\kohm}~...}{ca. \qty{820}{\ohm}}
{ca. \qty{1405}{\ohm}}
{ca. \qty{68}{\ohm}}
{ca. \qty{680}{\ohm}}
{\DARCimage{0.5\linewidth}{345include}}\end{PQuestion}

}
\only<2>{
\begin{PQuestion}{AC522}{Wie groß muss $R_2$ gewählt werden, damit sich eine Spannung von \qty{2,8}{\V} zwischen Gate und Source einstellt? $U_{\symup{B}}$=\qty{44}{\V}; $R_1$=\qty{10}{\kohm}; $R_3$=\qty{2,2}{\kohm}~...}{ca. \qty{820}{\ohm}}
{ca. \qty{1405}{\ohm}}
{ca. \qty{68}{\ohm}}
{\textbf{\textcolor{DARCgreen}{ca. \qty{680}{\ohm}}}}
{\DARCimage{0.5\linewidth}{345include}}\end{PQuestion}

}
\end{frame}

\begin{frame}
\frametitle{Lösungsweg}
\begin{columns}
    \begin{column}{0.48\textwidth}
    \begin{itemize}
  \item gegeben: $U_{\textrm{B}} = 44V$
  \item gegeben: $R_1 = 10k\Omega$
  \item gegeben: $R_3 = 2,2k\Omega$
  \item gegeben: $U_{\textrm{GS}} = U_{\textrm{R2}} = 2,8V$
  \item gegeben: $U_{\textrm{B}} = U_{\textrm{R1}} + U_{\textrm{R2}}$
  \item gesucht: $R_2$
  \end{itemize}

    \end{column}
   \begin{column}{0.48\textwidth}
       
    \pause
    \begin{equation}\begin{align}\nonumber \frac{U_{\textrm{R1}}}{U_{\textrm{R2}}} &= \frac{R_1}{R_2}\\ \nonumber \Rightarrow R_2 &= R_1 \cdot \frac{U_{\textrm{R2}}}{U_{\textrm{R1}}}\\ \nonumber &= R_1 \cdot \frac{U_{\textrm{R2}}}{U_{\textrm{B}}-U_{\textrm{GS}}}\\ \nonumber &= 10k\Omega \cdot \frac{2,8V}{44V-2,8V}\\ \nonumber &\approx 680\Omega \end{align}\end{equation}




   \end{column}
\end{columns}

\end{frame}

\begin{frame}
\only<1>{
\begin{QQuestion}{AC523}{Welche Verlustleistung erzeugt ein Power-MOS-FET mit $R_{\symup{DSon}}$ = \qty{4}{\m\ohm} bei einem Strom von \qty{25}{\A}?}{\qty{2,5}{\W}}
{\qty{1}{\W}}
{\qty{0,1}{\W}}
{\qty{6,25}{\W}}
\end{QQuestion}

}
\only<2>{
\begin{QQuestion}{AC523}{Welche Verlustleistung erzeugt ein Power-MOS-FET mit $R_{\symup{DSon}}$ = \qty{4}{\m\ohm} bei einem Strom von \qty{25}{\A}?}{\textbf{\textcolor{DARCgreen}{\qty{2,5}{\W}}}}
{\qty{1}{\W}}
{\qty{0,1}{\W}}
{\qty{6,25}{\W}}
\end{QQuestion}

}
\end{frame}

\begin{frame}
\frametitle{Lösungsweg}
\begin{itemize}
  \item gegeben: $R_{\textrm{DSon}} = 4m\Omega$
  \item gegeben: $I = 25A$
  \item gesucht: $P$
  \end{itemize}
    \pause
    $P = I^2 \cdot R = 25^2A \cdot 4m\Omega = 2,5W$



\end{frame}

\begin{frame}
\frametitle{Freilaufdiode}
\end{frame}

\begin{frame}
\only<1>{
\begin{question2x2}{AC524}{In welcher der folgenden Schaltungen ist die Freilaufdiode richtig eingesetzt?}{\DARCimage{1.0\linewidth}{429include}}
{\DARCimage{1.0\linewidth}{427include}}
{\DARCimage{1.0\linewidth}{428include}}
{\DARCimage{1.0\linewidth}{426include}}
\end{question2x2}

}
\only<2>{
\begin{question2x2}{AC524}{In welcher der folgenden Schaltungen ist die Freilaufdiode richtig eingesetzt?}{\DARCimage{1.0\linewidth}{429include}}
{\DARCimage{1.0\linewidth}{427include}}
{\DARCimage{1.0\linewidth}{428include}}
{\textbf{\textcolor{DARCgreen}{\DARCimage{1.0\linewidth}{426include}}}}
\end{question2x2}

}
\end{frame}%ENDCONTENT
