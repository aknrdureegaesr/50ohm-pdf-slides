
\section{Frequenzmessung II}
\label{section:frequenzmessung_2}
\begin{frame}%STARTCONTENT

\only<1>{
\begin{QQuestion}{AI511}{Womit kann die Frequenzanzeige eines durchstimmbaren Empfängers möglichst genau geprüft werden?}{Mit einem LC-Oszillator hoher Schwingkreisgüte.}
{Mit einem Quarzofen- oder GPS-synchronisierten Frequenzgenerator.}
{Mit den Oberschwingungen eines konstant belasteten Schaltnetzteils.}
{Mit einem temperaturstabiliserten RC-Oszillator.}
\end{QQuestion}

}
\only<2>{
\begin{QQuestion}{AI511}{Womit kann die Frequenzanzeige eines durchstimmbaren Empfängers möglichst genau geprüft werden?}{Mit einem LC-Oszillator hoher Schwingkreisgüte.}
{\textbf{\textcolor{DARCgreen}{Mit einem Quarzofen- oder GPS-synchronisierten Frequenzgenerator.}}}
{Mit den Oberschwingungen eines konstant belasteten Schaltnetzteils.}
{Mit einem temperaturstabiliserten RC-Oszillator.}
\end{QQuestion}

}
\end{frame}

\begin{frame}
\only<1>{
\begin{QQuestion}{AI504}{Eine Frequenzmessung wird genauer, wenn bei einem Frequenzzähler~...}{ein Vorteiler mit höherem Teilverhältnis benutzt wird.}
{der Hauptoszillator temperaturstabilisiert wird.}
{die Messdauer möglichst kurz gehalten wird.}
{das Eingangssignal gleichgerichtet wird.}
\end{QQuestion}

}
\only<2>{
\begin{QQuestion}{AI504}{Eine Frequenzmessung wird genauer, wenn bei einem Frequenzzähler~...}{ein Vorteiler mit höherem Teilverhältnis benutzt wird.}
{\textbf{\textcolor{DARCgreen}{der Hauptoszillator temperaturstabilisiert wird.}}}
{die Messdauer möglichst kurz gehalten wird.}
{das Eingangssignal gleichgerichtet wird.}
\end{QQuestion}

}
\end{frame}

\begin{frame}
\only<1>{
\begin{QQuestion}{AI502}{Was kann man mit einem passenden Dämpfungsglied und einem Frequenzzähler messen?}{Die Sendefrequenz eines CW-Senders}
{Den Modulationsindex eines FM-Senders}
{Die Ausdehnung des Seitenbandes eines SSB-Senders}
{Den Frequenzhub eines FM-Senders}
\end{QQuestion}

}
\only<2>{
\begin{QQuestion}{AI502}{Was kann man mit einem passenden Dämpfungsglied und einem Frequenzzähler messen?}{\textbf{\textcolor{DARCgreen}{Die Sendefrequenz eines CW-Senders}}}
{Den Modulationsindex eines FM-Senders}
{Die Ausdehnung des Seitenbandes eines SSB-Senders}
{Den Frequenzhub eines FM-Senders}
\end{QQuestion}

}
\end{frame}

\begin{frame}
\only<1>{
\begin{QQuestion}{AI501}{Wenn die Frequenz eines Senders mit einem Frequenzzähler überprüft wird, ist~...}{der Zähler mit der Netzfrequenz zu synchronisieren.}
{ein Träger ohne Modulation zu verwenden.}
{der Zähler mit der Sendefrequenz zu synchronisieren.}
{eine analoge Modulation des Trägers zu verwenden.}
\end{QQuestion}

}
\only<2>{
\begin{QQuestion}{AI501}{Wenn die Frequenz eines Senders mit einem Frequenzzähler überprüft wird, ist~...}{der Zähler mit der Netzfrequenz zu synchronisieren.}
{\textbf{\textcolor{DARCgreen}{ein Träger ohne Modulation zu verwenden.}}}
{der Zähler mit der Sendefrequenz zu synchronisieren.}
{eine analoge Modulation des Trägers zu verwenden.}
\end{QQuestion}

}
\end{frame}

\begin{frame}
\only<1>{
\begin{QQuestion}{AI503}{Welche Konfiguration gewährleistet die höchste Genauigkeit bei der Prüfung der Trägerfrequenz eines FM-Senders?}{Absorptionsfrequenzmesser und modulierter Träger}
{Oszilloskop und unmodulierter Träger}
{Frequenzzähler und modulierter Träger}
{Frequenzzähler und unmodulierter Träger}
\end{QQuestion}

}
\only<2>{
\begin{QQuestion}{AI503}{Welche Konfiguration gewährleistet die höchste Genauigkeit bei der Prüfung der Trägerfrequenz eines FM-Senders?}{Absorptionsfrequenzmesser und modulierter Träger}
{Oszilloskop und unmodulierter Träger}
{Frequenzzähler und modulierter Träger}
{\textbf{\textcolor{DARCgreen}{Frequenzzähler und unmodulierter Träger}}}
\end{QQuestion}

}
\end{frame}

\begin{frame}
\only<1>{
\begin{QQuestion}{AI505}{Benutzt man bei einem Frequenzzähler eine Torzeit von 10 s anstelle von 1 s erhöht sich~...}{die Empfindlichkeit.}
{die Langzeitstabilität.}
{die Auflösung.}
{die Stabilität.}
\end{QQuestion}

}
\only<2>{
\begin{QQuestion}{AI505}{Benutzt man bei einem Frequenzzähler eine Torzeit von 10 s anstelle von 1 s erhöht sich~...}{die Empfindlichkeit.}
{die Langzeitstabilität.}
{\textbf{\textcolor{DARCgreen}{die Auflösung.}}}
{die Stabilität.}
\end{QQuestion}

}
\end{frame}%ENDCONTENT
