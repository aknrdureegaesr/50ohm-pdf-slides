
\section{Betriebliche Abkürzungen}
\label{section:betriebliche_abkuerzungen}
\begin{frame}%STARTCONTENT
\begin{table}
\begin{DARCtabular}{cX}
     Abkürzung  & Bedeutung   \\
     BK  & Unterbrechung der Sendung; Formlose Übergabe (\emph{B}rea\emph{k})   \\
     CQ  & Allgemeiner Anruf (vom Englischen \enquote{Seek You})   \\
     CW  & \emph{C}ontinuous \emph{W}ave (Synonym für Morsetelegraphie)   \\
     K  & Aufforderung zum Senden (\enquote{\emph{K}ommen})   \\
     PSE  & Bitte, \emph{P}lea\emph{se}   \\
     R  & Received (Empfangsbestätigung, \emph{R}oger)   \\
     RX  & Receiver (Empfänger)   \\
     TX  & Transmitter (Sender)   \\
     TRX  & Transceiver (Sendeempfänger)   \\
\end{DARCtabular}
\caption{Zusammenfassung der Abkürzungen}
\label{n_abkuerzungen}
\end{table}

\end{frame}

\begin{frame}
\only<1>{
\begin{QQuestion}{BB106}{Was bedeuten die Abkürzungen \glqq TX\grqq{}, \glqq RX\grqq{}, \glqq TRX\grqq{} in dieser Reihenfolge?}{Sender, Empfänger, Sendeempfänger }
{Sendeempfänger, Empfänger, Sender}
{Tonqualität, Lesbarkeit, Signalstärke}
{Signalstärke, Lesbarkeit, Tonqualität}
\end{QQuestion}

}
\only<2>{
\begin{QQuestion}{BB106}{Was bedeuten die Abkürzungen \glqq TX\grqq{}, \glqq RX\grqq{}, \glqq TRX\grqq{} in dieser Reihenfolge?}{\textbf{\textcolor{DARCgreen}{Sender, Empfänger, Sendeempfänger }}}
{Sendeempfänger, Empfänger, Sender}
{Tonqualität, Lesbarkeit, Signalstärke}
{Signalstärke, Lesbarkeit, Tonqualität}
\end{QQuestion}

}
\end{frame}

\begin{frame}
\only<1>{
\begin{QQuestion}{BB107}{Was bedeutet die Abkürzung \glqq CW\grqq{} im Amateurfunk?}{Contestwertung}
{Codewort}
{Calling Wide}
{Continuous Wave}
\end{QQuestion}

}
\only<2>{
\begin{QQuestion}{BB107}{Was bedeutet die Abkürzung \glqq CW\grqq{} im Amateurfunk?}{Contestwertung}
{Codewort}
{Calling Wide}
{\textbf{\textcolor{DARCgreen}{Continuous Wave}}}
\end{QQuestion}

}
\end{frame}

\begin{frame}
\only<1>{
\begin{QQuestion}{BB110}{Was bedeutet \glqq R\grqq{} am Anfang eines Durchgangs in Telegrafie?}{Repeat (wiederhole)}
{Received (empfangen)}
{Rapport (Bericht)}
{Readability (Lesbarkeit)}
\end{QQuestion}

}
\only<2>{
\begin{QQuestion}{BB110}{Was bedeutet \glqq R\grqq{} am Anfang eines Durchgangs in Telegrafie?}{Repeat (wiederhole)}
{\textbf{\textcolor{DARCgreen}{Received (empfangen)}}}
{Rapport (Bericht)}
{Readability (Lesbarkeit)}
\end{QQuestion}

}
\end{frame}

\begin{frame}
\only<1>{
\begin{QQuestion}{BB109}{Was bedeutet \glqq K\grqq{} am Ende eines Durchgangs in Telegrafie?}{Unterbrechung der Sendung}
{Aufforderung zum Senden}
{Bitte warten}
{Beendigung des Funkverkehrs}
\end{QQuestion}

}
\only<2>{
\begin{QQuestion}{BB109}{Was bedeutet \glqq K\grqq{} am Ende eines Durchgangs in Telegrafie?}{Unterbrechung der Sendung}
{\textbf{\textcolor{DARCgreen}{Aufforderung zum Senden}}}
{Bitte warten}
{Beendigung des Funkverkehrs}
\end{QQuestion}

}
\end{frame}

\begin{frame}
\only<1>{
\begin{QQuestion}{BB108}{Was bedeutet die Betriebsabkürzung \glqq BK\grqq{} in Telegrafie?}{Beendigung des Funkverkehrs; wird auch zur formlosen Begrüßung genutzt}
{Alles richtig verstanden; wird auch zur schnellen Beendigung eines Funkkontakts genutzt}
{Bitte warten; wird auch zur schnellen Anforderung eines Rapports genutzt}
{Signal zur Unterbrechung einer laufenden Sendung; wird auch zur formlosen Übergabe genutzt}
\end{QQuestion}

}
\only<2>{
\begin{QQuestion}{BB108}{Was bedeutet die Betriebsabkürzung \glqq BK\grqq{} in Telegrafie?}{Beendigung des Funkverkehrs; wird auch zur formlosen Begrüßung genutzt}
{Alles richtig verstanden; wird auch zur schnellen Beendigung eines Funkkontakts genutzt}
{Bitte warten; wird auch zur schnellen Anforderung eines Rapports genutzt}
{\textbf{\textcolor{DARCgreen}{Signal zur Unterbrechung einer laufenden Sendung; wird auch zur formlosen Übergabe genutzt}}}
\end{QQuestion}

}
\end{frame}%ENDCONTENT
