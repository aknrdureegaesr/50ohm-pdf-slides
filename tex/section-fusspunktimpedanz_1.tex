
\section{Fußpunktimpedanz I}
\label{section:fusspunktimpedanz_1}
\begin{frame}%STARTCONTENT

\frametitle{Mittengespeister Dipol}
\begin{itemize}
  \item Speiseimpedanz 73,1 Ω
  \item Im Freiraum, also bei einer Aufbauhöhe von min. einer Wellenlänge
  \item Recht nahe bei 50 Ω
  \end{itemize}
\end{frame}

\begin{frame}
\only<1>{
\begin{QQuestion}{EG207}{Die Fußpunktimpedanz eines mittengespeisten Halbwellendipols in einer Höhe von mindestens einer Wellenlänge über dem Boden beträgt ungefähr~...}{\qty{50}{\ohm}.}
{\qty{75}{\ohm}.}
{\qty{30}{\ohm}.}
{\qty{600}{\ohm}.}
\end{QQuestion}

}
\only<2>{
\begin{QQuestion}{EG207}{Die Fußpunktimpedanz eines mittengespeisten Halbwellendipols in einer Höhe von mindestens einer Wellenlänge über dem Boden beträgt ungefähr~...}{\qty{50}{\ohm}.}
{\textbf{\textcolor{DARCgreen}{\qty{75}{\ohm}.}}}
{\qty{30}{\ohm}.}
{\qty{600}{\ohm}.}
\end{QQuestion}

}
\end{frame}

\begin{frame}
\frametitle{Mittengespeister Dipol}
\begin{itemize}
  \item Bei geringerer Aufbauhöhe kommt es zu Wechselwirkungen mit dem Boden
  \item Speiseimpedanz ca. 40 Ω bis 90 Ω
  \end{itemize}
\end{frame}

\begin{frame}
\only<1>{
\begin{QQuestion}{EG208}{Der Fußpunktwiderstand in der Mitte eines Halbwellendipols beträgt je nach Aufbauhöhe ungefähr~...}{\qtyrange{100}{120}{\ohm}.}
{\qtyrange{40}{90}{\ohm}.}
{\qtyrange{120}{240}{\ohm}.}
{\qtyrange{240}{600}{\ohm}.}
\end{QQuestion}

}
\only<2>{
\begin{QQuestion}{EG208}{Der Fußpunktwiderstand in der Mitte eines Halbwellendipols beträgt je nach Aufbauhöhe ungefähr~...}{\qtyrange{100}{120}{\ohm}.}
{\textbf{\textcolor{DARCgreen}{\qtyrange{40}{90}{\ohm}.}}}
{\qtyrange{120}{240}{\ohm}.}
{\qtyrange{240}{600}{\ohm}.}
\end{QQuestion}

}
\end{frame}

\begin{frame}
\only<1>{
\begin{QQuestion}{EG209}{Welchen Eingangswiderstand hat ein gestreckter mittengespeister Halbwellendipol?}{ca. \num{40} bis \qty{90}{\ohm}}
{ca. \qty{30}{\ohm}}
{ca. \qty{120}{\ohm}}
{ca. \num{240} bis \qty{300}{\ohm}}
\end{QQuestion}

}
\only<2>{
\begin{QQuestion}{EG209}{Welchen Eingangswiderstand hat ein gestreckter mittengespeister Halbwellendipol?}{\textbf{\textcolor{DARCgreen}{ca. \num{40} bis \qty{90}{\ohm}}}}
{ca. \qty{30}{\ohm}}
{ca. \qty{120}{\ohm}}
{ca. \num{240} bis \qty{300}{\ohm}}
\end{QQuestion}

}
\end{frame}

\begin{frame}
\frametitle{Faltdipol}
\begin{itemize}
  \item Antennenabschnitte sind teilweise parallel geführt
  \item Verdoppelt die Spannung
  \item Halbiert den Strom
  \item $R = \frac{2 \cdot U}{\frac{I}{2}} = 4 \cdot \frac{U}{I}$
  \item Speiseimpedanz vervierfacht sich: ca. 240 Ω bis 300 Ω
  \end{itemize}

\end{frame}

\begin{frame}
\only<1>{
\begin{QQuestion}{EG210}{Welchen Eingangs- bzw. Fußpunktwiderstand hat ein Faltdipol?}{ca. \qtyrange{240}{300}{\ohm}}
{ca. \qtyrange{30}{60}{\ohm}}
{ca. \qty{60}{\ohm}}
{ca. \qty{120}{\ohm}}
\end{QQuestion}

}
\only<2>{
\begin{QQuestion}{EG210}{Welchen Eingangs- bzw. Fußpunktwiderstand hat ein Faltdipol?}{\textbf{\textcolor{DARCgreen}{ca. \qtyrange{240}{300}{\ohm}}}}
{ca. \qtyrange{30}{60}{\ohm}}
{ca. \qty{60}{\ohm}}
{ca. \qty{120}{\ohm}}
\end{QQuestion}

}
\end{frame}

\begin{frame}
\frametitle{Groundplane-Antenne}
\begin{itemize}
  \item Ein Dipolschenkel entfällt und wird durch eine Erde mit möglichst geringem Widerstand ersetzt
  \item Hälfte eines Dipols im Freiraum
  \item $\rightarrow$ Speisewiderstand: $\dfrac{73,1 Ω}{2} \approx 37 Ω$
  \item Radiale um \qty{45}{\degree} nach unten abwinkeln ergibt zusätzliche Abstrahlung
  \item $\rightarrow$ Speisewiderstand: 50 Ω
  \end{itemize}
\end{frame}

\begin{frame}
\only<1>{
\begin{QQuestion}{EG211}{Welchen Eingangswiderstand hat eine Groundplane-Antenne?}{ca. \qty{600}{\ohm}}
{ca. \qtyrange{60}{120}{\ohm}}
{ca. \qtyrange{30}{50}{\ohm}}
{ca. \qty{240}{\ohm}}
\end{QQuestion}

}
\only<2>{
\begin{QQuestion}{EG211}{Welchen Eingangswiderstand hat eine Groundplane-Antenne?}{ca. \qty{600}{\ohm}}
{ca. \qtyrange{60}{120}{\ohm}}
{\textbf{\textcolor{DARCgreen}{ca. \qtyrange{30}{50}{\ohm}}}}
{ca. \qty{240}{\ohm}}
\end{QQuestion}

}
\end{frame}%ENDCONTENT
