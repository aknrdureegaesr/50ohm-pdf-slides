
\section{Störungen elektronischer Geräte II}
\label{section:stoerungen_elektronischer_geraete_2}
\begin{frame}%STARTCONTENT

\only<1>{
\begin{QQuestion}{AJ116}{Ein Nachbar beschwert sich über Störungen seines Fernsehempfängers, die allerdings auch bei abgezogener TV-Antenne auftreten. Die Störungen fallen zeitlich mit den Übertragungszeiten des Funkamateurs zusammen. Als erster Schritt~...}{ist der EMV-Beauftragte des RTA um Prüfung des Fernsehgeräts zu bitten.}
{ist das Fernsehgerät und der Sender von der Bundesnetzagentur zu überprüfen.}
{ist die Rückseite des Fernsehgeräts zu entfernen und das Gehäuse zu erden.}
{ist ein Netzfilter im Netzkabel des Fernsehgerätes, möglichst nahe am Gerät, vorzusehen.}
\end{QQuestion}

}
\only<2>{
\begin{QQuestion}{AJ116}{Ein Nachbar beschwert sich über Störungen seines Fernsehempfängers, die allerdings auch bei abgezogener TV-Antenne auftreten. Die Störungen fallen zeitlich mit den Übertragungszeiten des Funkamateurs zusammen. Als erster Schritt~...}{ist der EMV-Beauftragte des RTA um Prüfung des Fernsehgeräts zu bitten.}
{ist das Fernsehgerät und der Sender von der Bundesnetzagentur zu überprüfen.}
{ist die Rückseite des Fernsehgeräts zu entfernen und das Gehäuse zu erden.}
{\textbf{\textcolor{DARCgreen}{ist ein Netzfilter im Netzkabel des Fernsehgerätes, möglichst nahe am Gerät, vorzusehen.}}}
\end{QQuestion}

}
\end{frame}

\begin{frame}
\only<1>{
\begin{QQuestion}{AJ117}{Falls nachgewiesen wird, dass Störungen über das Stromversorgungsnetz in Geräte eindringen, ist wahrscheinlich~...}{die Entfernung der Erdung und Neuverlegung des Netzanschlusskabels erforderlich.}
{der Austausch des Netzteils erforderlich.}
{der Einbau eines Netzfilters erforderlich.}
{die Benachrichtigung des zuständigen Stromversorgers erforderlich.}
\end{QQuestion}

}
\only<2>{
\begin{QQuestion}{AJ117}{Falls nachgewiesen wird, dass Störungen über das Stromversorgungsnetz in Geräte eindringen, ist wahrscheinlich~...}{die Entfernung der Erdung und Neuverlegung des Netzanschlusskabels erforderlich.}
{der Austausch des Netzteils erforderlich.}
{\textbf{\textcolor{DARCgreen}{der Einbau eines Netzfilters erforderlich.}}}
{die Benachrichtigung des zuständigen Stromversorgers erforderlich.}
\end{QQuestion}

}
\end{frame}

\begin{frame}
\only<1>{
\begin{question2x2}{AJ118}{Welches der nachfolgenden Filter könnte vor einem Netzanschlusskabel eingeschleift werden, um darüber fließende HF-Ströme wirksam zu dämpfen?}{\DARCimage{1.0\linewidth}{162include}}
{\DARCimage{1.0\linewidth}{163include}}
{\DARCimage{1.0\linewidth}{164include}}
{\DARCimage{1.0\linewidth}{160include}}
\end{question2x2}

}
\only<2>{
\begin{question2x2}{AJ118}{Welches der nachfolgenden Filter könnte vor einem Netzanschlusskabel eingeschleift werden, um darüber fließende HF-Ströme wirksam zu dämpfen?}{\DARCimage{1.0\linewidth}{162include}}
{\DARCimage{1.0\linewidth}{163include}}
{\textbf{\textcolor{DARCgreen}{\DARCimage{1.0\linewidth}{164include}}}}
{\DARCimage{1.0\linewidth}{160include}}
\end{question2x2}

}
\end{frame}

\begin{frame}
\only<1>{
\begin{QQuestion}{AJ105}{Ein starkes HF-Signal gelangt unmittelbar in die ZF-Stufe des Rundfunkempfängers des Nachbarn. Dieses Phänomen wird als~...}{Direktmischung bezeichnet.}
{Direktabsorption bezeichnet.}
{Direkteinstrahlung bezeichnet.}
{HF-Durchschlag bezeichnet.}
\end{QQuestion}

}
\only<2>{
\begin{QQuestion}{AJ105}{Ein starkes HF-Signal gelangt unmittelbar in die ZF-Stufe des Rundfunkempfängers des Nachbarn. Dieses Phänomen wird als~...}{Direktmischung bezeichnet.}
{Direktabsorption bezeichnet.}
{\textbf{\textcolor{DARCgreen}{Direkteinstrahlung bezeichnet.}}}
{HF-Durchschlag bezeichnet.}
\end{QQuestion}

}
\end{frame}

\begin{frame}
\only<1>{
\begin{QQuestion}{AJ103}{Beim Betrieb eines digitalen Eigenbau-Funkempfängers ist dessen Empfang erheblich beeinträchtigt. Dies kann verbessert werden, indem die Leiterplatte~...}{in einem Kunststoffgehäuse untergebracht wird.}
{in Epoxydharz eingegossen wird.}
{über kunststoffisolierte Leitungen angeschlossen wird.}
{in einem geerdeten Metallgehäuse untergebracht wird.}
\end{QQuestion}

}
\only<2>{
\begin{QQuestion}{AJ103}{Beim Betrieb eines digitalen Eigenbau-Funkempfängers ist dessen Empfang erheblich beeinträchtigt. Dies kann verbessert werden, indem die Leiterplatte~...}{in einem Kunststoffgehäuse untergebracht wird.}
{in Epoxydharz eingegossen wird.}
{über kunststoffisolierte Leitungen angeschlossen wird.}
{\textbf{\textcolor{DARCgreen}{in einem geerdeten Metallgehäuse untergebracht wird.}}}
\end{QQuestion}

}
\end{frame}

\begin{frame}
\only<1>{
\begin{QQuestion}{AJ107}{Welche Modulationsverfahren haben das größte Potenzial, einen NF-Verstärker zu beeinflussen, der eine unzureichende Störfestigkeit aufweist?}{Frequenzumtastung (FSK) und Morsetelegrafie (CW).}
{Frequenzmodulation (FM) und Frequenzumtastung (FSK).}
{Einseitenbandmodulation (SSB) und Morsetelegrafie (CW).}
{Einseitenbandmodulation (SSB) und Frequenzmodulation (FM).}
\end{QQuestion}

}
\only<2>{
\begin{QQuestion}{AJ107}{Welche Modulationsverfahren haben das größte Potenzial, einen NF-Verstärker zu beeinflussen, der eine unzureichende Störfestigkeit aufweist?}{Frequenzumtastung (FSK) und Morsetelegrafie (CW).}
{Frequenzmodulation (FM) und Frequenzumtastung (FSK).}
{\textbf{\textcolor{DARCgreen}{Einseitenbandmodulation (SSB) und Morsetelegrafie (CW).}}}
{Einseitenbandmodulation (SSB) und Frequenzmodulation (FM).}
\end{QQuestion}

}
\end{frame}

\begin{frame}
\only<1>{
\begin{QQuestion}{AJ106}{In einem NF-Verstärker erfolgt die unerwünschte Gleichrichtung eines HF-Signals überwiegend~...}{an einem Kupferdraht.}
{an der Lautsprecherleitung.}
{an der Verbindung zweier Widerstände.}
{an einem Basis-Emitter-Übergang.}
\end{QQuestion}

}
\only<2>{
\begin{QQuestion}{AJ106}{In einem NF-Verstärker erfolgt die unerwünschte Gleichrichtung eines HF-Signals überwiegend~...}{an einem Kupferdraht.}
{an der Lautsprecherleitung.}
{an der Verbindung zweier Widerstände.}
{\textbf{\textcolor{DARCgreen}{an einem Basis-Emitter-Übergang.}}}
\end{QQuestion}

}
\end{frame}

\begin{frame}
\only<1>{
\begin{QQuestion}{AJ113}{In der Nähe eines \qty{144}{\MHz}-Senders befindet sich die passive Antenne eines DVB-T2-Fernsehempfängers. Es kommt zu einer Übersteuerung des Empfängers. Das Problem lässt sich durch den Einbau eines~...}{Tiefpassfilters bis \qty{460}{\MHz} in das Antennenzuführungskabel des Fernsehempfängers lösen.}
{Hochpassfilters ab \qty{460}{\MHz} in das Antennenzuführungskabel des Fernsehempfängers lösen.}
{Bandpassfilters für das \qty{2}{\m}-Band vor dem Tuner des Fernsehempfängers lösen.}
{\qty{460}{\MHz}-Notchfilters hinter dem Tuner des Fernsehempfängers lösen.}
\end{QQuestion}

}
\only<2>{
\begin{QQuestion}{AJ113}{In der Nähe eines \qty{144}{\MHz}-Senders befindet sich die passive Antenne eines DVB-T2-Fernsehempfängers. Es kommt zu einer Übersteuerung des Empfängers. Das Problem lässt sich durch den Einbau eines~...}{Tiefpassfilters bis \qty{460}{\MHz} in das Antennenzuführungskabel des Fernsehempfängers lösen.}
{\textbf{\textcolor{DARCgreen}{Hochpassfilters ab \qty{460}{\MHz} in das Antennenzuführungskabel des Fernsehempfängers lösen.}}}
{Bandpassfilters für das \qty{2}{\m}-Band vor dem Tuner des Fernsehempfängers lösen.}
{\qty{460}{\MHz}-Notchfilters hinter dem Tuner des Fernsehempfängers lösen.}
\end{QQuestion}

}
\end{frame}

\begin{frame}
\only<1>{
\begin{QQuestion}{AJ114}{Die Einfügedämpfung im Durchlassbereich eines passiven Hochpassfilters für ein Fernsehantennenkabel sollte~...}{mindestens \qtyrange{80}{100}{\decibel} betragen.}
{höchstens \qtyrange{10}{15}{\decibel} betragen.}
{mindestens \qtyrange{40}{60}{\decibel} betragen.}
{höchstens \qtyrange{2}{3}{\decibel} betragen.}
\end{QQuestion}

}
\only<2>{
\begin{QQuestion}{AJ114}{Die Einfügedämpfung im Durchlassbereich eines passiven Hochpassfilters für ein Fernsehantennenkabel sollte~...}{mindestens \qtyrange{80}{100}{\decibel} betragen.}
{höchstens \qtyrange{10}{15}{\decibel} betragen.}
{mindestens \qtyrange{40}{60}{\decibel} betragen.}
{\textbf{\textcolor{DARCgreen}{höchstens \qtyrange{2}{3}{\decibel} betragen.}}}
\end{QQuestion}

}
\end{frame}

\begin{frame}
\only<1>{
\begin{QQuestion}{AJ108}{Ein unselektiver TV-Antennen-Verstärker wird am wahrscheinlichsten~...}{auf Grund von Netzeinwirkungen beim Betrieb eines nahen Senders störend beeinflusst.}
{durch Übersteuerung mit dem Signal eines nahen Senders störend beeinflusst.}
{durch Einwirkungen auf die Gleichstromversorgung beim Betrieb eines nahen Senders störend beeinflusst.}
{auf Grund seiner zu niedrigen Verstärkung beim Betrieb eines nahen Senders störend beeinflusst.}
\end{QQuestion}

}
\only<2>{
\begin{QQuestion}{AJ108}{Ein unselektiver TV-Antennen-Verstärker wird am wahrscheinlichsten~...}{auf Grund von Netzeinwirkungen beim Betrieb eines nahen Senders störend beeinflusst.}
{\textbf{\textcolor{DARCgreen}{durch Übersteuerung mit dem Signal eines nahen Senders störend beeinflusst.}}}
{durch Einwirkungen auf die Gleichstromversorgung beim Betrieb eines nahen Senders störend beeinflusst.}
{auf Grund seiner zu niedrigen Verstärkung beim Betrieb eines nahen Senders störend beeinflusst.}
\end{QQuestion}

}
\end{frame}

\begin{frame}
\only<1>{
\begin{QQuestion}{AJ112}{Welche Filter sollten im Störungsfall vor die einzelnen Leitungsanschlüsse eines UKW-, DAB- und TV-Empfängers oder anderer angeschlossener Geräte eingeschleift werden, um Kurzwellensignale zu dämpfen?}{Je ein Tiefpassfilter bis \qty{40}{\MHz} unmittelbar vor dem Antennenanschluss und in das Netzkabel der gestörten Geräte.}
{Ein Hochpassfilter ab \qty{40}{\MHz} vor dem Antennenanschluss und zusätzlich je eine hochpermeable Ferritdrossel vor alle Leitungsanschlüsse der gestörten Geräte.}
{Eine Bandsperre für die entsprechenden Empfangsbereiche unmittelbar vor dem Antennenanschluss und ein Tiefpassfilter bis \qty{40}{\MHz} in das Netzkabel der gestörten Geräte.}
{Ein Bandpassfilter für \qty{30}{\MHz} mit \qty{2}{\MHz} Bandbreite unmittelbar vor dem Antennenanschluss und ein Tiefpassfilter bis \qty{30}{\MHz} in das Netzkabel der gestörten Geräte.}
\end{QQuestion}

}
\only<2>{
\begin{QQuestion}{AJ112}{Welche Filter sollten im Störungsfall vor die einzelnen Leitungsanschlüsse eines UKW-, DAB- und TV-Empfängers oder anderer angeschlossener Geräte eingeschleift werden, um Kurzwellensignale zu dämpfen?}{Je ein Tiefpassfilter bis \qty{40}{\MHz} unmittelbar vor dem Antennenanschluss und in das Netzkabel der gestörten Geräte.}
{\textbf{\textcolor{DARCgreen}{Ein Hochpassfilter ab \qty{40}{\MHz} vor dem Antennenanschluss und zusätzlich je eine hochpermeable Ferritdrossel vor alle Leitungsanschlüsse der gestörten Geräte.}}}
{Eine Bandsperre für die entsprechenden Empfangsbereiche unmittelbar vor dem Antennenanschluss und ein Tiefpassfilter bis \qty{40}{\MHz} in das Netzkabel der gestörten Geräte.}
{Ein Bandpassfilter für \qty{30}{\MHz} mit \qty{2}{\MHz} Bandbreite unmittelbar vor dem Antennenanschluss und ein Tiefpassfilter bis \qty{30}{\MHz} in das Netzkabel der gestörten Geräte.}
\end{QQuestion}

}
\end{frame}

\begin{frame}
\only<1>{
\begin{QQuestion}{AJ104}{Um die Möglichkeit unerwünschter Abstrahlungen mit Hilfe eines angepassten Antennensystems zu verringern, empfiehlt es sich~...}{die Netzspannung mit einem Bandpass für die Nutzfrequenz zu filtern.}
{mit einem hohen Stehwellenverhältnis zu arbeiten.}
{einen Antennentuner und/oder ein Filter zu verwenden.}
{nur vertikal polarisierte Antennen zu verwenden.}
\end{QQuestion}

}
\only<2>{
\begin{QQuestion}{AJ104}{Um die Möglichkeit unerwünschter Abstrahlungen mit Hilfe eines angepassten Antennensystems zu verringern, empfiehlt es sich~...}{die Netzspannung mit einem Bandpass für die Nutzfrequenz zu filtern.}
{mit einem hohen Stehwellenverhältnis zu arbeiten.}
{\textbf{\textcolor{DARCgreen}{einen Antennentuner und/oder ein Filter zu verwenden.}}}
{nur vertikal polarisierte Antennen zu verwenden.}
\end{QQuestion}

}
\end{frame}

\begin{frame}
\only<1>{
\begin{QQuestion}{AJ110}{Das Sendesignal eines VHF-Senders verursacht Empfangsstörungen in einem benachbarten DAB-Radio. Ein möglicher Grund hierfür ist~...}{eine nicht ausreichende Oberwellenunterdrückung des VHF-Senders.}
{die unterschiedliche Polarisation von VHF-Sende- und DAB-Empfangsantenne.}
{eine zu große Hubeinstellung am VHF-Sender.}
{eine Übersteuerung des Empfängereingangs des DAB-Radios.}
\end{QQuestion}

}
\only<2>{
\begin{QQuestion}{AJ110}{Das Sendesignal eines VHF-Senders verursacht Empfangsstörungen in einem benachbarten DAB-Radio. Ein möglicher Grund hierfür ist~...}{eine nicht ausreichende Oberwellenunterdrückung des VHF-Senders.}
{die unterschiedliche Polarisation von VHF-Sende- und DAB-Empfangsantenne.}
{eine zu große Hubeinstellung am VHF-Sender.}
{\textbf{\textcolor{DARCgreen}{eine Übersteuerung des Empfängereingangs des DAB-Radios.}}}
\end{QQuestion}

}
\end{frame}

\begin{frame}
\only<1>{
\begin{QQuestion}{AJ111}{Wie können sich störende Beeinflussungen in digitalen Rundfunkempfängern (DAB+) äußern?}{Der Rundfunkempfang bleibt einwandfrei, da die digitale Fehlerkorrektur alle Störungen eliminiert.}
{Die Differenz zwischen Störsignalfrequenz und der Abtastfrequenz ist im Gerätelautsprecher hörbar.}
{Die Lautstärke des Rundfunkempfangs schwankt sehr stark.}
{Der Empfänger produziert Störgeräusche und/oder schaltet stumm.}
\end{QQuestion}

}
\only<2>{
\begin{QQuestion}{AJ111}{Wie können sich störende Beeinflussungen in digitalen Rundfunkempfängern (DAB+) äußern?}{Der Rundfunkempfang bleibt einwandfrei, da die digitale Fehlerkorrektur alle Störungen eliminiert.}
{Die Differenz zwischen Störsignalfrequenz und der Abtastfrequenz ist im Gerätelautsprecher hörbar.}
{Die Lautstärke des Rundfunkempfangs schwankt sehr stark.}
{\textbf{\textcolor{DARCgreen}{Der Empfänger produziert Störgeräusche und/oder schaltet stumm.}}}
\end{QQuestion}

}
\end{frame}

\begin{frame}
\only<1>{
\begin{QQuestion}{AJ109}{Ein SSB-Sender bei \qty{432,2}{\MHz} erzeugt an einer Richtantenne, welche unmittelbar auf die DVB-T2-Fernsehantenne des Nachbarn gerichtet ist, eine effektive Strahlungsleistung von \qty{1,8}{\kW} ERP. Dies führt gegebenenfalls~...}{zu unerwünschten Reflexionen des Sendesignals.}
{zur Erzeugung von parasitären Schwingungen.}
{zur Übersteuerung der Vorstufe des Fernsehgerätes.}
{zu Störungen der IR-Fernbedienung des Fernsehgerätes.}
\end{QQuestion}

}
\only<2>{
\begin{QQuestion}{AJ109}{Ein SSB-Sender bei \qty{432,2}{\MHz} erzeugt an einer Richtantenne, welche unmittelbar auf die DVB-T2-Fernsehantenne des Nachbarn gerichtet ist, eine effektive Strahlungsleistung von \qty{1,8}{\kW} ERP. Dies führt gegebenenfalls~...}{zu unerwünschten Reflexionen des Sendesignals.}
{zur Erzeugung von parasitären Schwingungen.}
{\textbf{\textcolor{DARCgreen}{zur Übersteuerung der Vorstufe des Fernsehgerätes.}}}
{zu Störungen der IR-Fernbedienung des Fernsehgerätes.}
\end{QQuestion}

}
\end{frame}

\begin{frame}
\only<1>{
\begin{QQuestion}{AJ101}{Um die Wahrscheinlichkeit zu verringern, andere Stationen zu stören, sollte die benutzte Sendeleistung~...}{die Hälfte des maximal zulässigen Pegels betragen.}
{auf den maximal zulässigen Pegel eingestellt werden.}
{auf die für eine zufriedenstellende Kommunikation erforderlichen \qty{750}{\W} eingestellt werden.}
{auf das für eine zufriedenstellende Kommunikation erforderliche Minimum eingestellt werden.}
\end{QQuestion}

}
\only<2>{
\begin{QQuestion}{AJ101}{Um die Wahrscheinlichkeit zu verringern, andere Stationen zu stören, sollte die benutzte Sendeleistung~...}{die Hälfte des maximal zulässigen Pegels betragen.}
{auf den maximal zulässigen Pegel eingestellt werden.}
{auf die für eine zufriedenstellende Kommunikation erforderlichen \qty{750}{\W} eingestellt werden.}
{\textbf{\textcolor{DARCgreen}{auf das für eine zufriedenstellende Kommunikation erforderliche Minimum eingestellt werden.}}}
\end{QQuestion}

}
\end{frame}

\begin{frame}
\only<1>{
\begin{QQuestion}{AJ119}{Welche Art von Kondensatoren sollte zum Abblocken von HF-Spannungen vorzugsweise verwendet werden? Am besten verwendet man~...}{Keramikkondensatoren.}
{Aluminium-Elektrolytkondensatoren.}
{Tantalkondensatoren.}
{Polykarbonatkondensatoren.}
\end{QQuestion}

}
\only<2>{
\begin{QQuestion}{AJ119}{Welche Art von Kondensatoren sollte zum Abblocken von HF-Spannungen vorzugsweise verwendet werden? Am besten verwendet man~...}{\textbf{\textcolor{DARCgreen}{Keramikkondensatoren.}}}
{Aluminium-Elektrolytkondensatoren.}
{Tantalkondensatoren.}
{Polykarbonatkondensatoren.}
\end{QQuestion}

}
\end{frame}

\begin{frame}
\only<1>{
\begin{QQuestion}{AJ102}{Eine wirksame HF-Erdung sollte im genutzten Frequenzbereich~...}{induktiv gekoppelt sein.}
{über eine hohe Reaktanz verfügen.}
{über eine hohe Impedanz verfügen.}
{über eine niedrige Impedanz verfügen.}
\end{QQuestion}

}
\only<2>{
\begin{QQuestion}{AJ102}{Eine wirksame HF-Erdung sollte im genutzten Frequenzbereich~...}{induktiv gekoppelt sein.}
{über eine hohe Reaktanz verfügen.}
{über eine hohe Impedanz verfügen.}
{\textbf{\textcolor{DARCgreen}{über eine niedrige Impedanz verfügen.}}}
\end{QQuestion}

}
\end{frame}

\begin{frame}
\only<1>{
\begin{QQuestion}{AJ214}{In HF-Schaltungen können Nebenresonanzen durch die~...}{Stromversorgung hervorgerufen werden.}
{Eigenresonanz der HF-Drosseln hervorgerufen werden.}
{Sättigung der Kerne der HF-Spulen hervorgerufen werden.}
{Widerstandseigenschaft einer Drossel hervorgerufen werden.}
\end{QQuestion}

}
\only<2>{
\begin{QQuestion}{AJ214}{In HF-Schaltungen können Nebenresonanzen durch die~...}{Stromversorgung hervorgerufen werden.}
{\textbf{\textcolor{DARCgreen}{Eigenresonanz der HF-Drosseln hervorgerufen werden.}}}
{Sättigung der Kerne der HF-Spulen hervorgerufen werden.}
{Widerstandseigenschaft einer Drossel hervorgerufen werden.}
\end{QQuestion}

}
\end{frame}%ENDCONTENT
