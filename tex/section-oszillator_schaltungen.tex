
\section{Oszillatorschaltungen}
\label{section:oszillator_schaltungen}
\begin{frame}%STARTCONTENT

\only<1>{
\begin{QQuestion}{AD613}{Welche Bedingungen müssen zur Erzeugung ungedämpfter Schwingungen in Oszillatoren erfüllt sein?}{Die Schleifenverstärkung des Signalwegs im Oszillator muss kleiner als 1 sein, und das entstehende Oszillatorsignal darf auf dem Rückkopplungsweg nicht in der Phase gedreht werden.}
{Die Grenzfrequenz des verwendeten Verstärkerelements muss mindestens der Schwingfrequenz des Oszillators entsprechen, und das entstehende Eingangssignal muss über den Rückkopplungsweg wieder gegenphasig zum Eingang zurückgeführt werden.}
{Das an einem Schaltungspunkt betrachtete Oszillatorsignal muss auf dem Signalweg im Oszillator so verstärkt und phasengedreht werden, dass es wieder gleichphasig und mit mindestens der gleichen Amplitude zum selben Punkt zurückgekoppelt wird.}
{Die Schleifenverstärkung des Signalwegs im Oszillator muss größer als 1 sein, und das Ausgangssignal muss über den Rückkopplungsweg in der Phase so gedreht werden, dass es gegenphasig zum Ausgangspunkt zurückgeführt wird.}
\end{QQuestion}

}
\only<2>{
\begin{QQuestion}{AD613}{Welche Bedingungen müssen zur Erzeugung ungedämpfter Schwingungen in Oszillatoren erfüllt sein?}{Die Schleifenverstärkung des Signalwegs im Oszillator muss kleiner als 1 sein, und das entstehende Oszillatorsignal darf auf dem Rückkopplungsweg nicht in der Phase gedreht werden.}
{Die Grenzfrequenz des verwendeten Verstärkerelements muss mindestens der Schwingfrequenz des Oszillators entsprechen, und das entstehende Eingangssignal muss über den Rückkopplungsweg wieder gegenphasig zum Eingang zurückgeführt werden.}
{\textbf{\textcolor{DARCgreen}{Das an einem Schaltungspunkt betrachtete Oszillatorsignal muss auf dem Signalweg im Oszillator so verstärkt und phasengedreht werden, dass es wieder gleichphasig und mit mindestens der gleichen Amplitude zum selben Punkt zurückgekoppelt wird.}}}
{Die Schleifenverstärkung des Signalwegs im Oszillator muss größer als 1 sein, und das Ausgangssignal muss über den Rückkopplungsweg in der Phase so gedreht werden, dass es gegenphasig zum Ausgangspunkt zurückgeführt wird.}
\end{QQuestion}

}
\end{frame}

\begin{frame}
\only<1>{
\begin{PQuestion}{AD614}{ Bei dieser Schaltung handelt es sich um~...}{einen Hochfrequenzverstärker in Emitterschaltung.}
{einen Hochfrequenzverstärker in Kollektorschaltung.}
{einen kapazitiv rückgekoppelten Dreipunkt-Oszillator.}
{einen Oberton-Oszillator in Kollektorschaltung.}
{\DARCimage{1.0\linewidth}{760include}}\end{PQuestion}

}
\only<2>{
\begin{PQuestion}{AD614}{ Bei dieser Schaltung handelt es sich um~...}{einen Hochfrequenzverstärker in Emitterschaltung.}
{einen Hochfrequenzverstärker in Kollektorschaltung.}
{\textbf{\textcolor{DARCgreen}{einen kapazitiv rückgekoppelten Dreipunkt-Oszillator.}}}
{einen Oberton-Oszillator in Kollektorschaltung.}
{\DARCimage{1.0\linewidth}{760include}}\end{PQuestion}

}
\end{frame}

\begin{frame}
\only<1>{
\begin{PQuestion}{AD616}{Welche Funktion haben die beiden Kondensatoren $C_1$ und $C_2$ in der folgenden Schaltung?}{$C_1$ kompensiert die Basis-Kollektor-Kapazität und $C_2$ die Basis-Emitter-Kapazität.}
{Sie bilden in der dargestellten Audionschaltung die notwendige Rückkopplung.}
{$C_1$ stabilisiert die Basisvorspannung und $C_2$ die Emittervorspannung.}
{Sie bilden im dargestellten LC-Oszillator einen kapazitiven Spannungsteiler zur Rückkopplung.}
{\DARCimage{1.0\linewidth}{761include}}\end{PQuestion}

}
\only<2>{
\begin{PQuestion}{AD616}{Welche Funktion haben die beiden Kondensatoren $C_1$ und $C_2$ in der folgenden Schaltung?}{$C_1$ kompensiert die Basis-Kollektor-Kapazität und $C_2$ die Basis-Emitter-Kapazität.}
{Sie bilden in der dargestellten Audionschaltung die notwendige Rückkopplung.}
{$C_1$ stabilisiert die Basisvorspannung und $C_2$ die Emittervorspannung.}
{\textbf{\textcolor{DARCgreen}{Sie bilden im dargestellten LC-Oszillator einen kapazitiven Spannungsteiler zur Rückkopplung.}}}
{\DARCimage{1.0\linewidth}{761include}}\end{PQuestion}

}
\end{frame}

\begin{frame}
\only<1>{
\begin{PQuestion}{AD617}{ Bei dieser Oszillatorschaltung handelt es sich um einen kapazitiv rückgekoppelten Quarz-Oszillator in~...}{Emitterschaltung. Der Quarz wird in Parallelresonanz betrieben.}
{Kollektorschaltung. Der Quarz schwingt auf dem dritten Oberton.}
{Kollektorschaltung. Der Quarz schwingt auf seiner Grundfrequenz.}
{Emitterschaltung. Der Quarz wird in Serienresonanz betrieben.}
{\DARCimage{0.75\linewidth}{497include}}\end{PQuestion}

}
\only<2>{
\begin{PQuestion}{AD617}{ Bei dieser Oszillatorschaltung handelt es sich um einen kapazitiv rückgekoppelten Quarz-Oszillator in~...}{Emitterschaltung. Der Quarz wird in Parallelresonanz betrieben.}
{Kollektorschaltung. Der Quarz schwingt auf dem dritten Oberton.}
{\textbf{\textcolor{DARCgreen}{Kollektorschaltung. Der Quarz schwingt auf seiner Grundfrequenz.}}}
{Emitterschaltung. Der Quarz wird in Serienresonanz betrieben.}
{\DARCimage{0.75\linewidth}{497include}}\end{PQuestion}

}
\end{frame}

\begin{frame}
\only<1>{
\begin{QQuestion}{AD610}{Wie sollte ein Oszillator im Regelfall ausgangsseitig betrieben werden?}{Er sollte an eine Pufferstufe angeschlossen sein.}
{Er sollte direkt an einen HF-Leistungsverstärker angeschlossen sein.}
{Er sollte an ein passives Hochpassfilter angeschlossen sein.}
{Er sollte an ein passives Notchfilter angeschlossen sein.}
\end{QQuestion}

}
\only<2>{
\begin{QQuestion}{AD610}{Wie sollte ein Oszillator im Regelfall ausgangsseitig betrieben werden?}{\textbf{\textcolor{DARCgreen}{Er sollte an eine Pufferstufe angeschlossen sein.}}}
{Er sollte direkt an einen HF-Leistungsverstärker angeschlossen sein.}
{Er sollte an ein passives Hochpassfilter angeschlossen sein.}
{Er sollte an ein passives Notchfilter angeschlossen sein.}
\end{QQuestion}

}
\end{frame}

\begin{frame}
\only<1>{
\begin{PQuestion}{AD615}{An welchem Punkt der Schaltung sollte die HF-Ausgangsleistung ausgekoppelt werden?}{Schaltungspunkt D}
{Schaltungspunkt A}
{Schaltungspunkt B}
{Schaltungspunkt C}
{\DARCimage{1.0\linewidth}{777include}}\end{PQuestion}

}
\only<2>{
\begin{PQuestion}{AD615}{An welchem Punkt der Schaltung sollte die HF-Ausgangsleistung ausgekoppelt werden?}{\textbf{\textcolor{DARCgreen}{Schaltungspunkt D}}}
{Schaltungspunkt A}
{Schaltungspunkt B}
{Schaltungspunkt C}
{\DARCimage{1.0\linewidth}{777include}}\end{PQuestion}

}
\end{frame}

\begin{frame}
\only<1>{
\begin{PQuestion}{AD619}{Für die Messung der Oszillatorfrequenz sollte der Tastkopf hier vorzugsweise am Punkt~...}{2 angelegt werden.}
{1 angelegt werden.}
{3 angelegt werden.}
{4 angelegt werden.}
{\DARCimage{1.0\linewidth}{498include}}\end{PQuestion}

}
\only<2>{
\begin{PQuestion}{AD619}{Für die Messung der Oszillatorfrequenz sollte der Tastkopf hier vorzugsweise am Punkt~...}{2 angelegt werden.}
{1 angelegt werden.}
{3 angelegt werden.}
{\textbf{\textcolor{DARCgreen}{4 angelegt werden.}}}
{\DARCimage{1.0\linewidth}{498include}}\end{PQuestion}

}
\end{frame}

\begin{frame}
\only<1>{
\begin{PQuestion}{AD618}{Welche Auswirkung hat die Messung der Oszillatorfrequenz mit einem Tastkopf an Punkt 3?}{Die Oszillatorfrequenz verändert sich.}
{Der Transistor wird überlastet.}
{Der Quarz wird überlastet.}
{Es gibt keine Auswirkungen.}
{\DARCimage{1.0\linewidth}{498include}}\end{PQuestion}

}
\only<2>{
\begin{PQuestion}{AD618}{Welche Auswirkung hat die Messung der Oszillatorfrequenz mit einem Tastkopf an Punkt 3?}{\textbf{\textcolor{DARCgreen}{Die Oszillatorfrequenz verändert sich.}}}
{Der Transistor wird überlastet.}
{Der Quarz wird überlastet.}
{Es gibt keine Auswirkungen.}
{\DARCimage{1.0\linewidth}{498include}}\end{PQuestion}

}
\end{frame}%ENDCONTENT
