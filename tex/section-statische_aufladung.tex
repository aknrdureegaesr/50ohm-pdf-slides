
\section{Statische Aufladung von Antennen}
\label{section:statische_aufladung}
\begin{frame}%STARTCONTENT

\begin{columns}
    \begin{column}{0.48\textwidth}
    \begin{itemize}
  \item Unerwünschte Spannungen an Antennen durch statische Aufladungen
  \item Bei ungeerdeten Drahtantennen
  \item Z.B. durch Regen und Hagel
  \item Führt zu Prasselstörungen beim Empfang
  \end{itemize}

    \end{column}
   \begin{column}{0.48\textwidth}
       \begin{itemize}
  \item Abhilfe: Einbringen von Ableitwiderständen zwischen Leitern der Antennenzuführund und der Erdung der Amateurfunkstation
  \item Hochohmig, z.B. 100 kΩ
  \item Dadurch wird die Funktion der Funkanlage nicht beeinträchtigt
  \end{itemize}

   \end{column}
\end{columns}

\end{frame}

\begin{frame}
\only<1>{
\begin{QQuestion}{EK206}{Auf welchen besonderen Sicherheitsaspekt ist speziell bei ungeerdeten Drahtantennen zu achten?}{Durch die fehlende Erdung und den Strombauch im Speisepunkt kann der Mittenisolator zu stark erhitzt werden und durchschmelzen.}
{Durch die Sendeleistung entstehen hohe Spannungen gegen Erde, die eine dickere Isolierung des Antennendrahtes erfordern.}
{Bei Sonnenstürmen entstehen elektrische Aufladungen, die hohe Spannungen erzeugen können.}
{Bereits durch Regen oder Hagel kann es zu elektrischen Aufladungen der Antenne kommen.}
\end{QQuestion}

}
\only<2>{
\begin{QQuestion}{EK206}{Auf welchen besonderen Sicherheitsaspekt ist speziell bei ungeerdeten Drahtantennen zu achten?}{Durch die fehlende Erdung und den Strombauch im Speisepunkt kann der Mittenisolator zu stark erhitzt werden und durchschmelzen.}
{Durch die Sendeleistung entstehen hohe Spannungen gegen Erde, die eine dickere Isolierung des Antennendrahtes erfordern.}
{Bei Sonnenstürmen entstehen elektrische Aufladungen, die hohe Spannungen erzeugen können.}
{\textbf{\textcolor{DARCgreen}{Bereits durch Regen oder Hagel kann es zu elektrischen Aufladungen der Antenne kommen.}}}
\end{QQuestion}

}
\end{frame}

\begin{frame}
\only<1>{
\begin{QQuestion}{EK207}{Wie lassen sich elektrostatische Aufladungen, die insbesondere bei ungeerdeten Drahtantennen auftreten können, wirkungsvoll vermeiden, ohne die Funktion der Funkanlage zu beeinträchtigen?}{Das Einschleifen eines Anpassgerätes zwischen Transceiver und Antenne neutralisiert die Aufladungen.}
{Durch niederohmige Ableitwiderstände zwischen den Anschlüssen an der Antenne und dem Erdanschluss der Amateurfunkstelle.}
{Durch hochohmige Ableitwiderstände zwischen den Anschlüssen an der Antenne und dem Erdanschluss der Amateurfunkstelle.}
{Mit Hilfe der Abblockkondensatoren in einem zwischengeschalteten Stehwellenmessgerät.}
\end{QQuestion}

}
\only<2>{
\begin{QQuestion}{EK207}{Wie lassen sich elektrostatische Aufladungen, die insbesondere bei ungeerdeten Drahtantennen auftreten können, wirkungsvoll vermeiden, ohne die Funktion der Funkanlage zu beeinträchtigen?}{Das Einschleifen eines Anpassgerätes zwischen Transceiver und Antenne neutralisiert die Aufladungen.}
{Durch niederohmige Ableitwiderstände zwischen den Anschlüssen an der Antenne und dem Erdanschluss der Amateurfunkstelle.}
{\textbf{\textcolor{DARCgreen}{Durch hochohmige Ableitwiderstände zwischen den Anschlüssen an der Antenne und dem Erdanschluss der Amateurfunkstelle.}}}
{Mit Hilfe der Abblockkondensatoren in einem zwischengeschalteten Stehwellenmessgerät.}
\end{QQuestion}

}
\end{frame}%ENDCONTENT
