
\section{Shannon-Hartley-Gesetz}
\label{section:shannon_hartley_gesetzt}
\begin{frame}%STARTCONTENT
\begin{itemize}
  \item Welche Datenübertragungsrate erreichbar ist, hängt von der nutzbaren Bandbreite und dem Signal-Rauschverhältnis ab.
  \item Aus diesen beiden Größen kann mit dem Shannon-Hartley-Gesetz die theoretisch maximal erreichbare Datenübertragungsrate für einen Übertragungskanal berechnet werden.
  \item Ein leicht zu merkender Wert stellt sich bei einem Signal-Rausch-Verhältnis von 0 dB ein.
  \item Hier entspricht die Bandbreite in Hertz genau der maximal erreichbaren Datenrate in Bit/s.
  \end{itemize}
\end{frame}

\begin{frame}
\only<1>{
\begin{QQuestion}{AE416}{Welche Aussage trifft auf das Shannon-Hartley-Gesetz zu? Das Gesetz~...}{bestimmt die maximale Bandbreite, die durch eine Übertragung mit einer bestimmten Datenübertragungsrate theoretisch belegt werden kann.}
{besagt, dass unabhängig von der Art der vorherrschenden Störungen eines Übertragungskanals theoretisch eine unbegrenzte Datenübertragungsrate erzielt werden kann.}
{bestimmt für einen Übertragungskanal gegebener Bandbreite die höchste theoretisch erzielbare Datenübertragungsrate in Abhängigkeit vom Signal-Rausch-Verhältnis.}
{besagt, dass theoretisch eine unendliche Abtastrate erforderlich ist, um ein bandbegrenztes Signal fehlerfrei zu rekonstruieren.}
\end{QQuestion}

}
\only<2>{
\begin{QQuestion}{AE416}{Welche Aussage trifft auf das Shannon-Hartley-Gesetz zu? Das Gesetz~...}{bestimmt die maximale Bandbreite, die durch eine Übertragung mit einer bestimmten Datenübertragungsrate theoretisch belegt werden kann.}
{besagt, dass unabhängig von der Art der vorherrschenden Störungen eines Übertragungskanals theoretisch eine unbegrenzte Datenübertragungsrate erzielt werden kann.}
{\textbf{\textcolor{DARCgreen}{bestimmt für einen Übertragungskanal gegebener Bandbreite die höchste theoretisch erzielbare Datenübertragungsrate in Abhängigkeit vom Signal-Rausch-Verhältnis.}}}
{besagt, dass theoretisch eine unendliche Abtastrate erforderlich ist, um ein bandbegrenztes Signal fehlerfrei zu rekonstruieren.}
\end{QQuestion}

}
\end{frame}

\begin{frame}\begin{itemize}
  \item Schlechtere Signal-Rausch-Verhältnisse ermöglichen entsprechend weniger Datenrate, bessere Signal-Rausch-Verhältnisse größere Datenraten.
  \item Da die Rechnungen dazu recht komplex sind, wurden die Prüfungsfragen so gestaltet, dass man das Ergebnis leicht abschätzen kann.
  \item Im Folgenden gibt es Beispiele mit 0 db, -20 db und (+)30 db.
  \end{itemize}
\end{frame}

\begin{frame}Beispiel 1:

\begin{itemize}
  \item Ein Übertragungskanal mit einer Bandbreite von 2,7 kHz wird durch additives weißes Gaußsches Rauschen (AWGN) gestört. * Das Signal-Rausch-Verhältnis (SNR) beträgt 0 dB.
  \item Welche Bitrate kann nach dem Shannon-Hartley-Gesetz etwa maximal fehlerfrei übertragen werden?
  \end{itemize}
Durch ein SNR von 0db entspricht die Bandbreite in Hertz genau der maximal erreichbaren Datenrate in Bit/s, also 2,7 kbit/s.

\end{frame}

\begin{frame}
\only<1>{
\begin{QQuestion}{AE417}{Ein Übertragungskanal mit einer Bandbreite von \qty{2,7}{\kHz} wird durch additives weißes Gaußsches Rauschen (AWGN) gestört. Das Signal-Rausch-Verhältnis (SNR) beträgt \qty{0}{\decibel}. Welche Bitrate kann nach dem Shannon-Hartley-Gesetz etwa maximal fehlerfrei übertragen werden?}{ca.~\qty{39}{\bit}/s}
{\qty{0}{\bit}/s (Übertragung nicht möglich)}
{ca.~\qty{2,7}{\bit}/s}
{ca.~\qty[per-mode=symbol]{2,7}{\kilo\bit\per\s}}
\end{QQuestion}

}
\only<2>{
\begin{QQuestion}{AE417}{Ein Übertragungskanal mit einer Bandbreite von \qty{2,7}{\kHz} wird durch additives weißes Gaußsches Rauschen (AWGN) gestört. Das Signal-Rausch-Verhältnis (SNR) beträgt \qty{0}{\decibel}. Welche Bitrate kann nach dem Shannon-Hartley-Gesetz etwa maximal fehlerfrei übertragen werden?}{ca.~\qty{39}{\bit}/s}
{\qty{0}{\bit}/s (Übertragung nicht möglich)}
{ca.~\qty{2,7}{\bit}/s}
{\textbf{\textcolor{DARCgreen}{ca.~\qty[per-mode=symbol]{2,7}{\kilo\bit\per\s}}}}
\end{QQuestion}

}
\end{frame}

\begin{frame}
\only<1>{
\begin{QQuestion}{AE418}{Ein Übertragungskanal mit einer Bandbreite von \qty{10}{\MHz} wird durch additives weißes Gaußsches Rauschen (AWGN) gestört. Das Signal-Rausch-Verhältnis (SNR) beträgt \qty{0}{\decibel}. Welche Bitrate kann nach dem Shannon-Hartley-Gesetz etwa maximal fehlerfrei übertragen werden?}{ca.~\qty[per-mode=symbol]{10}{\mega\bit\per\s}}
{ca.~\qty[per-mode=symbol]{7}{\mega\bit\per\s}}
{ca.~\qty[per-mode=symbol]{8}{\mega\bit\per\s}}
{ca.~\qty[per-mode=symbol]{100}{\mega\bit\per\s}}
\end{QQuestion}

}
\only<2>{
\begin{QQuestion}{AE418}{Ein Übertragungskanal mit einer Bandbreite von \qty{10}{\MHz} wird durch additives weißes Gaußsches Rauschen (AWGN) gestört. Das Signal-Rausch-Verhältnis (SNR) beträgt \qty{0}{\decibel}. Welche Bitrate kann nach dem Shannon-Hartley-Gesetz etwa maximal fehlerfrei übertragen werden?}{\textbf{\textcolor{DARCgreen}{ca.~\qty[per-mode=symbol]{10}{\mega\bit\per\s}}}}
{ca.~\qty[per-mode=symbol]{7}{\mega\bit\per\s}}
{ca.~\qty[per-mode=symbol]{8}{\mega\bit\per\s}}
{ca.~\qty[per-mode=symbol]{100}{\mega\bit\per\s}}
\end{QQuestion}

}
\end{frame}

\begin{frame}Beispiel 2:

\begin{itemize}
  \item Ein Übertragungskanal mit einer Bandbreite von 2,7 kHz wird durch additives weißes Gaußsches Rauschen (AWGN) gestört. * Das Signal-Rausch-Verhältnis (SNR) beträgt -20 dB.
  \item Welche Bitrate kann nach dem Shannon-Hartley-Gesetz etwa maximal fehlerfrei übertragen werden?
  \end{itemize}
Durch ein SNR von -20db muss die maximal erreichbare Datenrate kleiner als 2,7 kbit/s sein. Es kann nur \qty{39}{\bit}/s richtig sein.

\end{frame}

\begin{frame}
\only<1>{
\begin{QQuestion}{AE420}{Ein Übertragungskanal mit einer Bandbreite von \qty{2,7}{\kHz} wird durch additives weißes Gaußsches Rauschen (AWGN) gestört. Das Signal-Rausch-Verhältnis (SNR) beträgt \qty{-20}{\decibel}. Welche Bitrate kann nach dem Shannon-Hartley-Gesetz etwa maximal fehlerfrei übertragen werden?}{ca.~\qty{39}{\bit}/s}
{\qty{0}{\bit}/s (Übertragung nicht möglich)}
{ca.~\qty[per-mode=symbol]{2,7}{\kilo\bit\per\s}}
{ca.~\qty[per-mode=symbol]{5,4}{\kilo\bit\per\s}}
\end{QQuestion}

}
\only<2>{
\begin{QQuestion}{AE420}{Ein Übertragungskanal mit einer Bandbreite von \qty{2,7}{\kHz} wird durch additives weißes Gaußsches Rauschen (AWGN) gestört. Das Signal-Rausch-Verhältnis (SNR) beträgt \qty{-20}{\decibel}. Welche Bitrate kann nach dem Shannon-Hartley-Gesetz etwa maximal fehlerfrei übertragen werden?}{\textbf{\textcolor{DARCgreen}{ca.~\qty{39}{\bit}/s}}}
{\qty{0}{\bit}/s (Übertragung nicht möglich)}
{ca.~\qty[per-mode=symbol]{2,7}{\kilo\bit\per\s}}
{ca.~\qty[per-mode=symbol]{5,4}{\kilo\bit\per\s}}
\end{QQuestion}

}
\end{frame}

\begin{frame}Beispiel 3:

\begin{itemize}
  \item Ein Übertragungskanal mit einer Bandbreite von 10 MHz wird durch additives weißes Gaußsches Rauschen (AWGN) gestört. * Das Signal-Rausch-Verhältnis (SNR) beträgt 30 dB.
  \item Welche Bitrate kann nach dem Shannon-Hartley-Gesetz etwa maximal fehlerfrei übertragen werden?
  \end{itemize}
Durch ein SNR von 30db muss die maximal erreichbare Datenrate größer 10 Mbit/s sein. Es kann nur 100 Mbit/s richtig sein.

\end{frame}

\begin{frame}
\only<1>{
\begin{QQuestion}{AE419}{Ein Übertragungskanal mit einer Bandbreite von \qty{10}{\MHz} wird durch additives weißes Gaußsches Rauschen (AWGN) gestört. Das Signal-Rausch-Verhältnis (SNR) beträgt \qty{30}{\decibel}. Welche Bitrate kann nach dem Shannon-Hartley-Gesetz etwa maximal fehlerfrei übertragen werden?}{ca.~\qty[per-mode=symbol]{100}{\mega\bit\per\s}}
{ca.~\qty[per-mode=symbol]{10}{\mega\bit\per\s}}
{ca.~\qty[per-mode=symbol]{7}{\mega\bit\per\s}}
{ca.~\qty[per-mode=symbol]{8}{\mega\bit\per\s}}
\end{QQuestion}

}
\only<2>{
\begin{QQuestion}{AE419}{Ein Übertragungskanal mit einer Bandbreite von \qty{10}{\MHz} wird durch additives weißes Gaußsches Rauschen (AWGN) gestört. Das Signal-Rausch-Verhältnis (SNR) beträgt \qty{30}{\decibel}. Welche Bitrate kann nach dem Shannon-Hartley-Gesetz etwa maximal fehlerfrei übertragen werden?}{\textbf{\textcolor{DARCgreen}{ca.~\qty[per-mode=symbol]{100}{\mega\bit\per\s}}}}
{ca.~\qty[per-mode=symbol]{10}{\mega\bit\per\s}}
{ca.~\qty[per-mode=symbol]{7}{\mega\bit\per\s}}
{ca.~\qty[per-mode=symbol]{8}{\mega\bit\per\s}}
\end{QQuestion}

}
\end{frame}

\begin{frame}\end{frame}%ENDCONTENT
