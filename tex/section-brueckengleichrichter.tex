
\section{Brückengleichrichter}
\label{section:brueckengleichrichter}
\begin{frame}%STARTCONTENT

\only<1>{
\begin{question2x2}{AD305}{Welche der folgenden Auswahlantworten enthält die richtige Diodenanordnung und Polarität eines Brückengleichrichters?}{\DARCimage{1.0\linewidth}{69include}}
{\DARCimage{1.0\linewidth}{68include}}
{\DARCimage{1.0\linewidth}{67include}}
{\DARCimage{1.0\linewidth}{70include}}
\end{question2x2}

}
\only<2>{
\begin{question2x2}{AD305}{Welche der folgenden Auswahlantworten enthält die richtige Diodenanordnung und Polarität eines Brückengleichrichters?}{\DARCimage{1.0\linewidth}{69include}}
{\DARCimage{1.0\linewidth}{68include}}
{\textbf{\textcolor{DARCgreen}{\DARCimage{1.0\linewidth}{67include}}}}
{\DARCimage{1.0\linewidth}{70include}}
\end{question2x2}

}
\end{frame}

\begin{frame}
\only<1>{
\begin{PQuestion}{AD306}{Wie groß ist die Spannung am Siebkondensator $C_{\symup{S}}$ im Leerlauf, wenn die Netzwechselspannung von \qty{230}{\V} anliegt und das Windungsverhältnis 8:1 beträgt?}{etwa \qty{20}{\V}}
{etwa \qty{40}{\V}}
{etwa \qty{29}{\V}}
{etwa \qty{58}{\V}}
{\DARCimage{1.0\linewidth}{66include}}\end{PQuestion}

}
\only<2>{
\begin{PQuestion}{AD306}{Wie groß ist die Spannung am Siebkondensator $C_{\symup{S}}$ im Leerlauf, wenn die Netzwechselspannung von \qty{230}{\V} anliegt und das Windungsverhältnis 8:1 beträgt?}{etwa \qty{20}{\V}}
{\textbf{\textcolor{DARCgreen}{etwa \qty{40}{\V}}}}
{etwa \qty{29}{\V}}
{etwa \qty{58}{\V}}
{\DARCimage{1.0\linewidth}{66include}}\end{PQuestion}

}
\end{frame}

\begin{frame}
\frametitle{Lösungsweg}
\begin{itemize}
  \item gegeben: $U_P = 230V$
  \item gegeben: $\"{u} = 8:1$
  \item gegeben: $U_D = 0,6V$
  \item gesucht: $\hat{U}$
  \end{itemize}
    \pause
    $ü = \frac{U_P}{U_S} \Rightarrow U_S = \frac{U_P}{ü} = \frac{230V}{8} = 28,75V$
    \pause
    Im Leerlauf kann die Diodenspannung vernachlässigt werden.

$\hat{U} = U_S \cdot \sqrt{2} = 28,75V \cdot 1,41 \approx 40V$



\end{frame}%ENDCONTENT
