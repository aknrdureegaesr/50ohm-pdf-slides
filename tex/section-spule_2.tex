
\section{Spule II}
\label{section:spule_2}
\begin{frame}%STARTCONTENT

\only<1>{
\begin{QQuestion}{AA101}{Welche Einheit wird üblicherweise für die Impedanz verwendet?}{Farad}
{Ohm}
{Siemens}
{Henry}
\end{QQuestion}

}
\only<2>{
\begin{QQuestion}{AA101}{Welche Einheit wird üblicherweise für die Impedanz verwendet?}{Farad}
{\textbf{\textcolor{DARCgreen}{Ohm}}}
{Siemens}
{Henry}
\end{QQuestion}

}
\end{frame}

\begin{frame}
\only<1>{
\begin{QQuestion}{AC201}{In einer idealen Induktivität, die an einer Wechselspannungsquelle angeschlossen ist, eilt der Strom der angelegten Spannung~...}{um \qty{45}{\degree} voraus.}
{um \qty{90}{\degree} nach.}
{um \qty{45}{\degree} nach.}
{um \qty{90}{\degree} voraus.}
\end{QQuestion}

}
\only<2>{
\begin{QQuestion}{AC201}{In einer idealen Induktivität, die an einer Wechselspannungsquelle angeschlossen ist, eilt der Strom der angelegten Spannung~...}{um \qty{45}{\degree} voraus.}
{\textbf{\textcolor{DARCgreen}{um \qty{90}{\degree} nach.}}}
{um \qty{45}{\degree} nach.}
{um \qty{90}{\degree} voraus.}
\end{QQuestion}

}
\end{frame}

\begin{frame}
\only<1>{
\begin{QQuestion}{AC202}{Welches Vorzeichen hat der Blindwiderstand einer idealen Spule und von welchen physikalischen Größen hängt er ab? Der Blindwiderstand ist~...}{negativ und abhängig von der Induktivität und der anliegenden Frequenz.}
{positiv und unabhängig von der Induktivität und der anliegenden Frequenz.}
{positiv und abhängig von der Induktivität und der anliegenden Frequenz.}
{negativ und unabhängig von der Induktivität und der anliegenden Frequenz.}
\end{QQuestion}

}
\only<2>{
\begin{QQuestion}{AC202}{Welches Vorzeichen hat der Blindwiderstand einer idealen Spule und von welchen physikalischen Größen hängt er ab? Der Blindwiderstand ist~...}{negativ und abhängig von der Induktivität und der anliegenden Frequenz.}
{positiv und unabhängig von der Induktivität und der anliegenden Frequenz.}
{\textbf{\textcolor{DARCgreen}{positiv und abhängig von der Induktivität und der anliegenden Frequenz.}}}
{negativ und unabhängig von der Induktivität und der anliegenden Frequenz.}
\end{QQuestion}

}
\end{frame}

\begin{frame}
\only<1>{
\begin{QQuestion}{AC203}{Beim Anlegen einer Gleichspannung $U$ = \qty{1}{\V} an eine Spule messen Sie einen Strom. Wird der Strom beim Anlegen von einer Wechselspannung mit $U_{\symup{eff}}$ = \qty{1}{\V} größer oder kleiner?}{Beim Betrieb mit Gleichspannung wirkt nur der Gleichstromwiderstand der Spule. Beim Betrieb mit Wechselspannung wird der induktive Widerstand $X_{\symup{L}}$ wirksam und erhöht den Gesamtwiderstand. Der Strom wird kleiner.}
{Beim Betrieb mit Gleichspannung wirkt nur der Gleichstromwiderstand der Spule. Beim Betrieb mit Wechselspannung wirkt nur der kleinere induktive Widerstand $X_{\symup{L}}$. Der Strom wird größer.}
{Beim Betrieb mit Gleich- oder Wechselspannung wirkt nur der ohmsche Widerstand $X_{\symup{L}}$ der Spule. Der Strom bleibt gleich.}
{Beim Betrieb mit Wechselspannung wirkt nur der Wechselstromwiderstand der Spule. Beim Betrieb mit Gleichspannung wird nur der ohmsche Widerstand $X_{\symup{L}}$ wirksam. Der Strom wird größer.}
\end{QQuestion}

}
\only<2>{
\begin{QQuestion}{AC203}{Beim Anlegen einer Gleichspannung $U$ = \qty{1}{\V} an eine Spule messen Sie einen Strom. Wird der Strom beim Anlegen von einer Wechselspannung mit $U_{\symup{eff}}$ = \qty{1}{\V} größer oder kleiner?}{\textbf{\textcolor{DARCgreen}{Beim Betrieb mit Gleichspannung wirkt nur der Gleichstromwiderstand der Spule. Beim Betrieb mit Wechselspannung wird der induktive Widerstand $X_{\symup{L}}$ wirksam und erhöht den Gesamtwiderstand. Der Strom wird kleiner.}}}
{Beim Betrieb mit Gleichspannung wirkt nur der Gleichstromwiderstand der Spule. Beim Betrieb mit Wechselspannung wirkt nur der kleinere induktive Widerstand $X_{\symup{L}}$. Der Strom wird größer.}
{Beim Betrieb mit Gleich- oder Wechselspannung wirkt nur der ohmsche Widerstand $X_{\symup{L}}$ der Spule. Der Strom bleibt gleich.}
{Beim Betrieb mit Wechselspannung wirkt nur der Wechselstromwiderstand der Spule. Beim Betrieb mit Gleichspannung wird nur der ohmsche Widerstand $X_{\symup{L}}$ wirksam. Der Strom wird größer.}
\end{QQuestion}

}
\end{frame}

\begin{frame}
\only<1>{
\begin{QQuestion}{AC204}{Wie groß ist der Betrag des induktiven Blindwiderstands einer Spule mit \qty{3}{\micro\H} Induktivität bei einer Frequenz von \qty{100}{\MHz}?}{ca. \qty{1885}{\ohm}}
{ca. \qty{942,0}{\ohm}}
{ca. \qty{1885}{\kohm}}
{ca. \qty{1,942}{\ohm}}
\end{QQuestion}

}
\only<2>{
\begin{QQuestion}{AC204}{Wie groß ist der Betrag des induktiven Blindwiderstands einer Spule mit \qty{3}{\micro\H} Induktivität bei einer Frequenz von \qty{100}{\MHz}?}{\textbf{\textcolor{DARCgreen}{ca. \qty{1885}{\ohm}}}}
{ca. \qty{942,0}{\ohm}}
{ca. \qty{1885}{\kohm}}
{ca. \qty{1,942}{\ohm}}
\end{QQuestion}

}
\end{frame}

\begin{frame}
\frametitle{Lösungsweg}
\begin{itemize}
  \item gegeben: $L = 3\mu H$
  \item gegeben: $f = 100MHz$
  \item gesucht: $X_{\textrm{L}}$
  \end{itemize}
    \pause
    \begin{equation}\begin{split}\nonumber X_{\textrm{L}} &= \omega \cdot L = 2\pi \cdot f \cdot L\\ &= 2\pi \cdot 100MHz \cdot 3\mu H\\ &\approx 1885\Omega \end{split}\end{equation}



\end{frame}

\begin{frame}
\only<1>{
\begin{QQuestion}{AC205}{Wie groß ist die Induktivität einer Spule mit 14 Windungen, die auf einen Kern mit einer Induktivitätskonstante ($A_{\symup{L}}$-Wert) von \qty{1,5}{\nano\H} gewickelt ist?}{\qty{0,294}{\micro\H}}
{\qty{2,94}{\micro\H}}
{\qty{29,4}{\nano\H}}
{\qty{2,94}{\nano\H}}
\end{QQuestion}

}
\only<2>{
\begin{QQuestion}{AC205}{Wie groß ist die Induktivität einer Spule mit 14 Windungen, die auf einen Kern mit einer Induktivitätskonstante ($A_{\symup{L}}$-Wert) von \qty{1,5}{\nano\H} gewickelt ist?}{\textbf{\textcolor{DARCgreen}{\qty{0,294}{\micro\H}}}}
{\qty{2,94}{\micro\H}}
{\qty{29,4}{\nano\H}}
{\qty{2,94}{\nano\H}}
\end{QQuestion}

}
\end{frame}

\begin{frame}
\frametitle{Lösungsweg}
\begin{itemize}
  \item gegeben: $N = 14$
  \item gegeben: $A_{\textrm{L}} = 1,5nH$
  \item gesucht: $L$
  \end{itemize}
    \pause
    \begin{equation}\begin{split}\nonumber L &= N^2 \cdot A_{\textrm{L}}\\ &= 14^2 \cdot 1,5nH\\ &= 0,294\mu H \end{split}\end{equation}



\end{frame}

\begin{frame}
\only<1>{
\begin{QQuestion}{AC206}{Wie groß ist die Induktivität einer Spule mit 300 Windungen, die auf einen Kern mit einer Induktivitätskonstante ($A_{\symup{L}}$-Wert) von \qty{1250}{\nano\H} gewickelt ist?}{\qty{112,5}{\mH}}
{\qty{112,5}{\micro\H}}
{\qty{11,25}{\mH}}
{\qty{1,125}{\mH}}
\end{QQuestion}

}
\only<2>{
\begin{QQuestion}{AC206}{Wie groß ist die Induktivität einer Spule mit 300 Windungen, die auf einen Kern mit einer Induktivitätskonstante ($A_{\symup{L}}$-Wert) von \qty{1250}{\nano\H} gewickelt ist?}{\textbf{\textcolor{DARCgreen}{\qty{112,5}{\mH}}}}
{\qty{112,5}{\micro\H}}
{\qty{11,25}{\mH}}
{\qty{1,125}{\mH}}
\end{QQuestion}

}
\end{frame}

\begin{frame}
\frametitle{Lösungsweg}
\begin{itemize}
  \item gegeben: $N = 300$
  \item gegeben: $A_{\textrm{L}} = 1250nH$
  \item gesucht: $L$
  \end{itemize}
    \pause
    \begin{equation}\begin{split}\nonumber L &= N^2 \cdot A_{\textrm{L}}\\ &= 300^2 \cdot 1250nH\\ &= 112,5mH \end{split}\end{equation}



\end{frame}

\begin{frame}
\only<1>{
\begin{QQuestion}{AC207}{Mit einem Ringkern, dessen Induktivitätskonstante ($A_{\symup{L}}$-Wert) mit \qty{250}{\nano\H} angegeben ist, soll eine Spule mit einer Induktivität von \qty{2}{\mH} hergestellt werden. Wie groß ist die erforderliche Windungszahl etwa?}{89}
{3}
{2828}
{53}
\end{QQuestion}

}
\only<2>{
\begin{QQuestion}{AC207}{Mit einem Ringkern, dessen Induktivitätskonstante ($A_{\symup{L}}$-Wert) mit \qty{250}{\nano\H} angegeben ist, soll eine Spule mit einer Induktivität von \qty{2}{\mH} hergestellt werden. Wie groß ist die erforderliche Windungszahl etwa?}{\textbf{\textcolor{DARCgreen}{89}}}
{3}
{2828}
{53}
\end{QQuestion}

}
\end{frame}

\begin{frame}
\frametitle{Lösungsweg}
\begin{itemize}
  \item gegeben: $L = 2mH$
  \item gegeben: $A_{\textrm{L}} = 250nH$
  \item gesucht: $N$
  \end{itemize}
    \pause
    \begin{equation}\begin{align}\nonumber L &= N^2 \cdot A_{\textrm{L}}\\ \nonumber N &= \sqrt{\frac{L}{A_{\textrm{L}}}} = \sqrt{\frac{2mH}{250nH}} \\ \nonumber &= 89\ \textrm{Windungen} \end{align}\end{equation}



\end{frame}

\begin{frame}
\only<1>{
\begin{QQuestion}{AC208}{Ein Spulenkern hat eine Induktivitätskonstante ($A_{\symup{L}}$-Wert) von \qty{30}{\nano\H}. Wie groß ist die erforderliche Windungszahl zur Herstellung einer Induktivität von \qty{12}{\micro\H} in etwa?}{400}
{20}
{360}
{6}
\end{QQuestion}

}
\only<2>{
\begin{QQuestion}{AC208}{Ein Spulenkern hat eine Induktivitätskonstante ($A_{\symup{L}}$-Wert) von \qty{30}{\nano\H}. Wie groß ist die erforderliche Windungszahl zur Herstellung einer Induktivität von \qty{12}{\micro\H} in etwa?}{400}
{\textbf{\textcolor{DARCgreen}{20}}}
{360}
{6}
\end{QQuestion}

}
\end{frame}

\begin{frame}
\frametitle{Lösungsweg}
\begin{itemize}
  \item gegeben: $L = 12\mu H$
  \item gegeben: $A_{\textrm{L}} = 30nH$
  \item gesucht: $N$
  \end{itemize}
    \pause
    \begin{equation}\begin{align}\nonumber L &= N^2 \cdot A_{\textrm{L}}\\ \nonumber N &= \sqrt{\frac{L}{A_{\textrm{L}}}} = \sqrt{\frac{12\mu H}{30nH}} \\ \nonumber &= 20\ \textrm{Windungen} \end{align}\end{equation}



\end{frame}

\begin{frame}
\only<1>{
\begin{QQuestion}{AC209}{Neben dem induktiven Blindwiderstand treten in der mit Wechselstrom durchflossenen Spule auch Verluste auf, die rechnerisch in einem seriellen Verlustwiderstand zusammengefasst werden können. Als Maß für die Verluste in einer Spule wird auch~...}{der Verlustfaktor tan $\delta$ angegeben, der dem Kehrwert des Gütefaktors entspricht.}
{der relative Verlustwiderstand in Ohm pro Henry angegeben, mit dem die Spulengüte berechnet werden kann.}
{der relative Blindwiderstand in Ohm pro Henry angegeben, mit dem die Spulengüte berechnet werden kann.}
{der Verlustfaktor cos $\varphi$ angegeben, der dem Kehrwert des Gütefaktors entspricht.}
\end{QQuestion}

}
\only<2>{
\begin{QQuestion}{AC209}{Neben dem induktiven Blindwiderstand treten in der mit Wechselstrom durchflossenen Spule auch Verluste auf, die rechnerisch in einem seriellen Verlustwiderstand zusammengefasst werden können. Als Maß für die Verluste in einer Spule wird auch~...}{\textbf{\textcolor{DARCgreen}{der Verlustfaktor tan $\delta$ angegeben, der dem Kehrwert des Gütefaktors entspricht.}}}
{der relative Verlustwiderstand in Ohm pro Henry angegeben, mit dem die Spulengüte berechnet werden kann.}
{der relative Blindwiderstand in Ohm pro Henry angegeben, mit dem die Spulengüte berechnet werden kann.}
{der Verlustfaktor cos $\varphi$ angegeben, der dem Kehrwert des Gütefaktors entspricht.}
\end{QQuestion}

}
\end{frame}

\begin{frame}
\only<1>{
\begin{QQuestion}{AC210}{Um die Abstrahlungen der Spule eines abgestimmten Schwingkreises zu verringern, sollte die Spule~...}{einen abgestimmten Kunststoffkern aufweisen.}
{einen hohlen Kupferkern aufweisen.}
{in einem isolierenden Kunststoffgehäuse untergebracht werden.}
{in einem leitenden Metallgehäuse untergebracht werden.}
\end{QQuestion}

}
\only<2>{
\begin{QQuestion}{AC210}{Um die Abstrahlungen der Spule eines abgestimmten Schwingkreises zu verringern, sollte die Spule~...}{einen abgestimmten Kunststoffkern aufweisen.}
{einen hohlen Kupferkern aufweisen.}
{in einem isolierenden Kunststoffgehäuse untergebracht werden.}
{\textbf{\textcolor{DARCgreen}{in einem leitenden Metallgehäuse untergebracht werden.}}}
\end{QQuestion}

}
\end{frame}

\begin{frame}
\only<1>{
\begin{PQuestion}{AC211}{Das folgende Bild zeigt einen Kern, um den ein Kabel für den Bau einer Drossel gewickelt ist. Der Kern sollte üblicherweise aus~...}{Stahl bestehen.}
{Kunststoff bestehen.}
{Ferrit bestehen.}
{diamagnetischem Material bestehen.}
{\DARCimage{0.5\linewidth}{40include}}\end{PQuestion}

}
\only<2>{
\begin{PQuestion}{AC211}{Das folgende Bild zeigt einen Kern, um den ein Kabel für den Bau einer Drossel gewickelt ist. Der Kern sollte üblicherweise aus~...}{Stahl bestehen.}
{Kunststoff bestehen.}
{\textbf{\textcolor{DARCgreen}{Ferrit bestehen.}}}
{diamagnetischem Material bestehen.}
{\DARCimage{0.5\linewidth}{40include}}\end{PQuestion}

}
\end{frame}%ENDCONTENT
