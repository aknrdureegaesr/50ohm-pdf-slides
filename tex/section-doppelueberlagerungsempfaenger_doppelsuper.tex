
\section{Doppelüberlagerungsempfänger (Doppelsuper)}
\label{section:doppelueberlagerungsempfaenger_doppelsuper}
\begin{frame}%STARTCONTENT

\only<1>{
\begin{QQuestion}{AF112}{Welche Aussage ist für einen Doppelsuper richtig?}{Mit einer hohen ersten ZF erreicht man leicht eine gute Spiegelfrequenzunterdrückung.}
{Das von der Antenne aufgenommene Signal bleibt bis zum Demodulator in seiner Frequenz erhalten.}
{Mit einer niedrigen ersten ZF erreicht man leicht eine gute Spiegelfrequenzunterdrückung.}
{Mit einer niedrigen zweiten ZF erreicht man leicht eine gute Spiegelfrequenzunterdrückung.}
\end{QQuestion}

}
\only<2>{
\begin{QQuestion}{AF112}{Welche Aussage ist für einen Doppelsuper richtig?}{\textbf{\textcolor{DARCgreen}{Mit einer hohen ersten ZF erreicht man leicht eine gute Spiegelfrequenzunterdrückung.}}}
{Das von der Antenne aufgenommene Signal bleibt bis zum Demodulator in seiner Frequenz erhalten.}
{Mit einer niedrigen ersten ZF erreicht man leicht eine gute Spiegelfrequenzunterdrückung.}
{Mit einer niedrigen zweiten ZF erreicht man leicht eine gute Spiegelfrequenzunterdrückung.}
\end{QQuestion}

}
\end{frame}

\begin{frame}
\only<1>{
\begin{QQuestion}{AF113}{Welche Aussage ist für einen Doppelsuper richtig?}{Durch eine hohe erste ZF erreicht man leicht eine hohe Empfindlichkeit.}
{Mit einer niedrigen ersten ZF erreicht man leicht gute Werte bei der Kreuzmodulation.}
{Mit einer niedrigen zweiten ZF erreicht man leicht eine gute Trennschärfe.}
{Durch eine niedrige zweite ZF erreicht man leicht eine gute Spiegelselektion.}
\end{QQuestion}

}
\only<2>{
\begin{QQuestion}{AF113}{Welche Aussage ist für einen Doppelsuper richtig?}{Durch eine hohe erste ZF erreicht man leicht eine hohe Empfindlichkeit.}
{Mit einer niedrigen ersten ZF erreicht man leicht gute Werte bei der Kreuzmodulation.}
{\textbf{\textcolor{DARCgreen}{Mit einer niedrigen zweiten ZF erreicht man leicht eine gute Trennschärfe.}}}
{Durch eine niedrige zweite ZF erreicht man leicht eine gute Spiegelselektion.}
\end{QQuestion}

}
\end{frame}

\begin{frame}
\only<1>{
\begin{QQuestion}{AF114}{Welche Beziehungen der Zwischenfrequenzen zueinander sind für einen Kurzwellen-Doppelsuper vorteilhaft?}{Die 1. ZF liegt niedriger als die maximale Empfangsfrequenz. Nach der Filterung im Roofing-Filter (1. ZF) wird auf eine höhere 2. ZF heraufgemischt.}
{Die 1. ZF liegt höher als das Doppelte der maximalen Empfangsfrequenz. Nach der Filterung im Roofing-Filter (1. ZF) wird auf eine niedrigere 2. ZF heruntergemischt.}
{Die 1. ZF liegt unter der niedrigsten Empfangsfrequenz. Die 2. ZF liegt über der höchsten Empfangsfrequenz. }
{Die 1. ZF darf maximal die Hälfte der höchsten Empfangsfrequenz betragen. Die 2. ZF liegt höher als das Doppelte der niedrigsten Empfangsfrequenz.}
\end{QQuestion}

}
\only<2>{
\begin{QQuestion}{AF114}{Welche Beziehungen der Zwischenfrequenzen zueinander sind für einen Kurzwellen-Doppelsuper vorteilhaft?}{Die 1. ZF liegt niedriger als die maximale Empfangsfrequenz. Nach der Filterung im Roofing-Filter (1. ZF) wird auf eine höhere 2. ZF heraufgemischt.}
{\textbf{\textcolor{DARCgreen}{Die 1. ZF liegt höher als das Doppelte der maximalen Empfangsfrequenz. Nach der Filterung im Roofing-Filter (1. ZF) wird auf eine niedrigere 2. ZF heruntergemischt.}}}
{Die 1. ZF liegt unter der niedrigsten Empfangsfrequenz. Die 2. ZF liegt über der höchsten Empfangsfrequenz. }
{Die 1. ZF darf maximal die Hälfte der höchsten Empfangsfrequenz betragen. Die 2. ZF liegt höher als das Doppelte der niedrigsten Empfangsfrequenz.}
\end{QQuestion}

}
\end{frame}

\begin{frame}
\only<1>{
\begin{QQuestion}{AF116}{Wie groß sollte die Bandbreite des Filters für die 1. ZF in einem Doppelsuper sein?}{Mindestens so groß wie die größte benötigte Bandbreite der vorgesehenen Betriebsarten.}
{Mindestens so groß wie die doppelte Bandbreite der jeweiligen Betriebsart.}
{Mindestens so groß wie das breiteste zu empfangende Amateurband.}
{Sie muss den vollen Abstimmbereich des Empfängers umfassen.}
\end{QQuestion}

}
\only<2>{
\begin{QQuestion}{AF116}{Wie groß sollte die Bandbreite des Filters für die 1. ZF in einem Doppelsuper sein?}{\textbf{\textcolor{DARCgreen}{Mindestens so groß wie die größte benötigte Bandbreite der vorgesehenen Betriebsarten.}}}
{Mindestens so groß wie die doppelte Bandbreite der jeweiligen Betriebsart.}
{Mindestens so groß wie das breiteste zu empfangende Amateurband.}
{Sie muss den vollen Abstimmbereich des Empfängers umfassen.}
\end{QQuestion}

}
\end{frame}

\begin{frame}
\only<1>{
\begin{PQuestion}{AF209}{Folgende Schaltung stellt einen Doppelsuper dar. Welche Funktion haben die drei mit X, Y und Z gekennzeichneten Blöcke?}{X und Y sind Produktdetektoren, Z ist ein HF-Mischer.}
{X ist ein Mischer, Y ist ein Produktdetektor, Z ist ein Mischer.}
{X und Y sind Mischer, Z ist ein Produktdetektor.}
{X und Y sind Balancemischer, Z ist ein ZF-Verstärker.}
{\DARCimage{1.0\linewidth}{82include}}\end{PQuestion}

}
\only<2>{
\begin{PQuestion}{AF209}{Folgende Schaltung stellt einen Doppelsuper dar. Welche Funktion haben die drei mit X, Y und Z gekennzeichneten Blöcke?}{X und Y sind Produktdetektoren, Z ist ein HF-Mischer.}
{X ist ein Mischer, Y ist ein Produktdetektor, Z ist ein Mischer.}
{\textbf{\textcolor{DARCgreen}{X und Y sind Mischer, Z ist ein Produktdetektor.}}}
{X und Y sind Balancemischer, Z ist ein ZF-Verstärker.}
{\DARCimage{1.0\linewidth}{82include}}\end{PQuestion}

}
\end{frame}

\begin{frame}
\only<1>{
\begin{PQuestion}{AF117}{Folgende Schaltung stellt einen Doppelsuper dar. Welche Funktion haben die drei mit X, Y und Z gekennzeichneten Blöcke?}{X ist ein BFO, Y ist ein CO und Z ein VFO.}
{X ist ein VFO, Y ist ein BFO und Z ein CO.}
{X ist ein VFO, Y ist ein CO und Z ein BFO.}
{X ist ein BFO, Y ist ein VFO und Z ein CO.}
{\DARCimage{1.0\linewidth}{83include}}\end{PQuestion}

}
\only<2>{
\begin{PQuestion}{AF117}{Folgende Schaltung stellt einen Doppelsuper dar. Welche Funktion haben die drei mit X, Y und Z gekennzeichneten Blöcke?}{X ist ein BFO, Y ist ein CO und Z ein VFO.}
{X ist ein VFO, Y ist ein BFO und Z ein CO.}
{\textbf{\textcolor{DARCgreen}{X ist ein VFO, Y ist ein CO und Z ein BFO.}}}
{X ist ein BFO, Y ist ein VFO und Z ein CO.}
{\DARCimage{1.0\linewidth}{83include}}\end{PQuestion}

}
\end{frame}

\begin{frame}
\only<1>{
\begin{PQuestion}{AF210}{Welchen Frequenzbereich kann der VFO des im folgenden Blockschaltbild gezeichneten HF-Teils eines Empfängers haben?}{\qtyrange{20}{47}{\MHz} oder \qtyrange{62}{89}{\MHz}}
{\qtyrange{20}{47}{\MHz} oder \qtyrange{53}{80}{\MHz}}
{\qtyrange{23}{41}{\MHz} oder \qtyrange{53}{80}{\MHz}}
{\qtyrange{23}{41}{\MHz} oder \qtyrange{62}{89}{\MHz}}
{\DARCimage{1.0\linewidth}{93include}}\end{PQuestion}

}
\only<2>{
\begin{PQuestion}{AF210}{Welchen Frequenzbereich kann der VFO des im folgenden Blockschaltbild gezeichneten HF-Teils eines Empfängers haben?}{\qtyrange{20}{47}{\MHz} oder \qtyrange{62}{89}{\MHz}}
{\textbf{\textcolor{DARCgreen}{\qtyrange{20}{47}{\MHz} oder \qtyrange{53}{80}{\MHz}}}}
{\qtyrange{23}{41}{\MHz} oder \qtyrange{53}{80}{\MHz}}
{\qtyrange{23}{41}{\MHz} oder \qtyrange{62}{89}{\MHz}}
{\DARCimage{1.0\linewidth}{93include}}\end{PQuestion}

}
\end{frame}

\begin{frame}
\frametitle{Lösungsweg}
\begin{itemize}
  \item gegeben: $f_E = 3\dots30MHz$
  \item gegeben: $f_{ZF1} = 50MHz$
  \item gesucht: $f_{OSZ}$
  \end{itemize}
    \pause
    $f_{ZF} = |f_E − f_{OSZ}| \Rightarrow f_{OSZ} = f_{ZF} \pm f_{E}$
    \pause
    \begin{enumerate}
  \item[1] Lösung: $f_{OSZ} = f_{ZF} + f_{E} = 50MHz + 3\dots30MHz = 53\dots80MHz$
  \item[2] Lösung: $_{OSZ} = f_{ZF} -- f_{E} = 50MHz -- 3\dots30MHz = 47\dots20MHz$
  \end{enumerate}


\end{frame}

\begin{frame}
\only<1>{
\begin{PQuestion}{AF120}{Welche Frequenzen können die drei Oszillatoren des im folgenden Blockschaltbild gezeichneten Empfängers haben, wenn eine Frequenz von \qty{3,65}{\MHz} empfangen wird? Bei welcher Antwort sind alle drei Frequenzen richtig?}{VFO:~\qty{23,65}{\MHz} CO1:~\qty{59}{\MHz} CO2:~\qty{8,545}{\MHz}}
{VFO:~\qty{46,35}{\MHz} CO1:~\qty{41}{\MHz} CO2:~\qty{9,455}{\MHz}}
{VFO:~\qty{46,35}{\MHz} CO1:~\qty{41}{\MHz} CO2:~\qty{9,545}{\MHz}}
{VFO:~\qty{46,35}{\MHz} CO1:~\qty{40,545}{\MHz} CO2:~\qty{9,455}{\MHz}}
{\DARCimage{1.0\linewidth}{90include}}\end{PQuestion}

}
\only<2>{
\begin{PQuestion}{AF120}{Welche Frequenzen können die drei Oszillatoren des im folgenden Blockschaltbild gezeichneten Empfängers haben, wenn eine Frequenz von \qty{3,65}{\MHz} empfangen wird? Bei welcher Antwort sind alle drei Frequenzen richtig?}{VFO:~\qty{23,65}{\MHz} CO1:~\qty{59}{\MHz} CO2:~\qty{8,545}{\MHz}}
{\textbf{\textcolor{DARCgreen}{VFO:~\qty{46,35}{\MHz} CO1:~\qty{41}{\MHz} CO2:~\qty{9,455}{\MHz}}}}
{VFO:~\qty{46,35}{\MHz} CO1:~\qty{41}{\MHz} CO2:~\qty{9,545}{\MHz}}
{VFO:~\qty{46,35}{\MHz} CO1:~\qty{40,545}{\MHz} CO2:~\qty{9,455}{\MHz}}
{\DARCimage{1.0\linewidth}{90include}}\end{PQuestion}

}
\end{frame}

\begin{frame}
\frametitle{Lösungsweg}
\begin{columns}
    \begin{column}{0.48\textwidth}
    \begin{itemize}
  \item gegeben: $f_{E} = 3,65MHz$
  \item gegeben: $f_{ZF1} = 50MHz$
  \end{itemize}

    \end{column}
   \begin{column}{0.48\textwidth}
       \begin{itemize}
  \item gegeben: $f_{ZF2} = 9MHz$
  \item gegeben: $f_{NF} = 455kHz$
  \end{itemize}

   \end{column}
\end{columns}

\begin{itemize}
  \item gesucht: $f_{VFO}, f_{CO1}, f_{CO2}$
  \end{itemize}
    \pause
    $f_{ZF} = \begin{cases}f_E + f_{OSZ}\\ f_{OSZ} -- f_E\\ f_E -- f_{OSZ}\end{cases} \Rightarrow f_{OSZ} = \begin{cases}f_{ZF} -- f_E\\ f_E + f_{ZF}\\ f_E -- f_{ZF}\end{cases}$
    \pause
    $f_{VFO} = f_{ZF} -- f_E = 50MHz -- 3,65MHz = 46,35MHz$

$f_{VFO} = f_E \pm f_{ZF1} = 3,65MHz \pm 50MHz = \begin{cases}53,65MHz\\ \cancel{-46,35MHz}\end{cases}$



\end{frame}

\begin{frame}
    \pause
    $f_{CO1} = f_{ZF2} -- f_{ZF1} = 9MHz -- 50MHz = \cancel{-41MHz}$

$f_{CO1} = f_{ZF1} \pm f_{ZF2} = 50MHz \pm 9MHz = \begin{cases}59MHz\\ 41MHz\end{cases}$
    \pause
    $f_{CO2} = f_{NF} -- f_{ZF2} = 455kHz -- 9MHz = \cancel{-8,545MHz}$

$f_{CO2} = f_{ZF2} \pm f_{NF} = 9MHz \pm 455kHz = \begin{cases}9,455MHz\\ 8,545MHz\end{cases}$
    \pause
    VFO: $\bold{46,35MHz} \And 53,65MHz$, CO1: $\bold{41MHz} \And 59MHz$, CO2: $8,545MHz \And \bold{9,455MHz}$



\end{frame}

\begin{frame}
\only<1>{
\begin{PQuestion}{AF118}{Ein Doppelsuper hat eine erste ZF von \qty{9}{\MHz} und eine zweite ZF von \qty{460}{\kHz}. Die Empfangsfrequenz soll \qty{21,1}{\MHz} sein. Welche Frequenzen sind für den VFO und den CO erforderlich, wenn der VFO oberhalb und der CO unterhalb des jeweiligen Mischer-Eingangssignals schwingen sollen?}{Der VFO muss bei \qty{12,1}{\MHz} und der CO bei \qty{9,46}{\MHz} schwingen.}
{Der VFO muss bei \qty{30,1}{\MHz} und der CO bei \qty{8,54}{\MHz} schwingen.}
{Der VFO muss bei \qty{12,1}{\MHz} und der CO bei \qty{8,54}{\MHz} schwingen.}
{Der VFO muss bei \qty{30,1}{\MHz} und der CO bei \qty{9,46}{\MHz} schwingen.}
{\DARCimage{1.0\linewidth}{84include}}\end{PQuestion}

}
\only<2>{
\begin{PQuestion}{AF118}{Ein Doppelsuper hat eine erste ZF von \qty{9}{\MHz} und eine zweite ZF von \qty{460}{\kHz}. Die Empfangsfrequenz soll \qty{21,1}{\MHz} sein. Welche Frequenzen sind für den VFO und den CO erforderlich, wenn der VFO oberhalb und der CO unterhalb des jeweiligen Mischer-Eingangssignals schwingen sollen?}{Der VFO muss bei \qty{12,1}{\MHz} und der CO bei \qty{9,46}{\MHz} schwingen.}
{\textbf{\textcolor{DARCgreen}{Der VFO muss bei \qty{30,1}{\MHz} und der CO bei \qty{8,54}{\MHz} schwingen.}}}
{Der VFO muss bei \qty{12,1}{\MHz} und der CO bei \qty{8,54}{\MHz} schwingen.}
{Der VFO muss bei \qty{30,1}{\MHz} und der CO bei \qty{9,46}{\MHz} schwingen.}
{\DARCimage{1.0\linewidth}{84include}}\end{PQuestion}

}
\end{frame}

\begin{frame}
\frametitle{Lösungsweg}
\begin{columns}
    \begin{column}{0.48\textwidth}
    \begin{itemize}
  \item gegeben: $f_{E} = 21,1MHz$
  \item gegeben: $f_{ZF1} = 9MHz$
  \end{itemize}

    \end{column}
   \begin{column}{0.48\textwidth}
       \begin{itemize}
  \item gegeben: $f_{ZF2} = 460kHz$
  \end{itemize}

   \end{column}
\end{columns}

\begin{itemize}
  \item gesucht: $f_{VFO} \gt f_E, f_{CO} \lt f_{ZF1}$
  \end{itemize}
    \pause
    $f_{ZF} = \begin{cases}f_{OSZ} -- f_E\\ f_E -- f_{OSZ}\end{cases} \Rightarrow f_{OSZ} = \begin{cases}f_E + f_{ZF}\\ f_E -- f_{ZF}\end{cases}$
    \pause
    $f_{VFO} = f_E + f_{ZF1} = 21,1MHz + 9MHz = 30,1MHz$
    \pause
    $f_{CO} = f_{ZF1} -- f_{ZF2} = 9MHz -- 460kHz = 8,54MHz$



\end{frame}

\begin{frame}
\only<1>{
\begin{PQuestion}{AF119}{Ein Doppelsuper hat eine erste ZF von \qty{10,7}{\MHz} und eine zweite ZF von \qty{460}{\kHz}. Die Empfangsfrequenz soll \qty{28}{\MHz} sein. Welche Frequenzen sind für den VFO und den CO erforderlich, wenn die Oszillatoren oberhalb der Mischer-Eingangssignale schwingen sollen?}{Der VFO muss bei \qty{38,70}{\MHz} und der CO bei \qty{11,16}{\MHz} schwingen.}
{Der VFO muss bei \qty{10,24}{\MHz} und der CO bei \qty{17,30}{\MHz} schwingen.}
{Der VFO muss bei \qty{17,3}{\MHz} und der CO bei \qty{10,24}{\MHz} schwingen.}
{Der VFO muss bei \qty{28,460}{\MHz} und der CO bei \qty{39,16}{\MHz} schwingen.}
{\DARCimage{1.0\linewidth}{84include}}\end{PQuestion}

}
\only<2>{
\begin{PQuestion}{AF119}{Ein Doppelsuper hat eine erste ZF von \qty{10,7}{\MHz} und eine zweite ZF von \qty{460}{\kHz}. Die Empfangsfrequenz soll \qty{28}{\MHz} sein. Welche Frequenzen sind für den VFO und den CO erforderlich, wenn die Oszillatoren oberhalb der Mischer-Eingangssignale schwingen sollen?}{\textbf{\textcolor{DARCgreen}{Der VFO muss bei \qty{38,70}{\MHz} und der CO bei \qty{11,16}{\MHz} schwingen.}}}
{Der VFO muss bei \qty{10,24}{\MHz} und der CO bei \qty{17,30}{\MHz} schwingen.}
{Der VFO muss bei \qty{17,3}{\MHz} und der CO bei \qty{10,24}{\MHz} schwingen.}
{Der VFO muss bei \qty{28,460}{\MHz} und der CO bei \qty{39,16}{\MHz} schwingen.}
{\DARCimage{1.0\linewidth}{84include}}\end{PQuestion}

}
\end{frame}

\begin{frame}
\frametitle{Lösungsweg}
\begin{columns}
    \begin{column}{0.48\textwidth}
    \begin{itemize}
  \item gegeben: $f_{E} = 28MHz$
  \item gegeben: $f_{ZF1} = 10,7MHz$
  \end{itemize}

    \end{column}
   \begin{column}{0.48\textwidth}
       \begin{itemize}
  \item gegeben: $f_{ZF2} = 460kHz$
  \end{itemize}

   \end{column}
\end{columns}

\begin{itemize}
  \item gesucht: $f_{VFO} \gt f_E, f_{CO} \gt f_{ZF1}$
  \end{itemize}
    \pause
    $f_{ZF} = \begin{cases}f_{OSZ} -- f_E\\ f_E -- f_{OSZ}\end{cases} \Rightarrow f_{OSZ} = \begin{cases}f_E + f_{ZF}\\ f_E -- f_{ZF}\end{cases}$
    \pause
    $f_{VFO} = f_E + f_{ZF1} = 28MHz + 10,7MHz = 38,70MHz$
    \pause
    $f_{CO} = f_{ZF1} + f_{ZF2} = 10,7MHz + 460kHz = 11,16MHz$



\end{frame}%ENDCONTENT
