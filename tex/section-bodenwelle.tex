
\section{Bodenwelle}
\label{section:bodenwelle}
\begin{frame}%STARTCONTENT

\begin{columns}
    \begin{column}{0.48\textwidth}
    \begin{itemize}
  \item Die Bodenwelle reicht über den sichtbaren Horizont raus
  \item Folgt der Erdkrümmung
  \item Am besten für Frequenzen unter \qty{3}{\mega\hertz}
  \end{itemize}

    \end{column}
   \begin{column}{0.48\textwidth}
       
\begin{figure}
    \DARCimage{0.85\linewidth}{866include}
    \caption{\scriptsize Reichweite der Bodenwelle je nach Band}
    \label{e_reichweite_bodenwelle}
\end{figure}


   \end{column}
\end{columns}

\end{frame}

\begin{frame}
\frametitle{Reichweite}
\begin{itemize}
  \item Reichweite ist von Frequenz und Bodenbeschaffenheit abhängig
  \item Langwelle (\qty{30}{\kilo\hertz}–\qty{300}{\kilo\hertz}) bis zu \qty{1000}{\kilo\metre}, Mittelwelle (\qty{300}{\kilo\hertz}–\qty{3}{\mega\hertz}) bis zu \qty{250}{\kilo\metre}
  \item Gut nutzbar im \qty{160}{\metre}-Band
  \item Im \qty{10}{\metre}-Band für Kommunikation im Stadtbereich nutzbar
  \item VHF und höhere Frequenzen vernachlässigbar
  \end{itemize}

\end{frame}

\begin{frame}
\only<1>{
\begin{QQuestion}{EH211}{Die Ausbreitung der Wellen im \qty{160}{\m}-Band erfolgt tagsüber hauptsächlich~...}{über die Raumwelle, weil es in der Troposphäre durch Temperaturinversionen zu Reflexionen für die Frequenzen unter \qty{2}{\MHz} kommen kann.}
{über die Raumwelle, weil die Refraktion (Brechung) in der D-Region für Frequenzen bis zu \qty{2}{\MHz} besonders stark ist.}
{über Raum- und Bodenwelle, weil es bei den Frequenzen unter \qty{2}{\MHz} nur zu geringfügiger Phasenverschiebung zwischen reflektierter und direkter Welle kommt.}
{über die Bodenwelle, weil durch die Dämpfung der D-Region keine Raumwelle entstehen kann.}
\end{QQuestion}

}
\only<2>{
\begin{QQuestion}{EH211}{Die Ausbreitung der Wellen im \qty{160}{\m}-Band erfolgt tagsüber hauptsächlich~...}{über die Raumwelle, weil es in der Troposphäre durch Temperaturinversionen zu Reflexionen für die Frequenzen unter \qty{2}{\MHz} kommen kann.}
{über die Raumwelle, weil die Refraktion (Brechung) in der D-Region für Frequenzen bis zu \qty{2}{\MHz} besonders stark ist.}
{über Raum- und Bodenwelle, weil es bei den Frequenzen unter \qty{2}{\MHz} nur zu geringfügiger Phasenverschiebung zwischen reflektierter und direkter Welle kommt.}
{\textbf{\textcolor{DARCgreen}{über die Bodenwelle, weil durch die Dämpfung der D-Region keine Raumwelle entstehen kann.}}}
\end{QQuestion}

}
\end{frame}

\begin{frame}
\only<1>{
\begin{QQuestion}{EH212}{Welche der folgenden Aussagen trifft für KW-Funkverbindungen zu, die über Bodenwellen erfolgen?}{Die Bodenwelle folgt der Erdkrümmung und geht nicht über den geografischen Horizont hinaus. Sie wird in niedrigeren Frequenzbereichen stärker gedämpft als in höheren Frequenzbereichen.}
{Die Bodenwelle folgt der Erdkrümmung und geht nicht über den geografischen Horizont hinaus. Sie wird in höheren Frequenzbereichen stärker gedämpft als in niedrigeren Frequenzbereichen.}
{Die Bodenwelle folgt der Erdkrümmung und geht über den geografischen Horizont hinaus. Sie wird in niedrigeren Frequenzbereichen stärker gedämpft als in höheren Frequenzbereichen.}
{Die Bodenwelle folgt der Erdkrümmung und geht über den geografischen Horizont hinaus. Sie wird in höheren Frequenzbereichen stärker gedämpft als in niedrigeren Frequenzbereichen.}
\end{QQuestion}

}
\only<2>{
\begin{QQuestion}{EH212}{Welche der folgenden Aussagen trifft für KW-Funkverbindungen zu, die über Bodenwellen erfolgen?}{Die Bodenwelle folgt der Erdkrümmung und geht nicht über den geografischen Horizont hinaus. Sie wird in niedrigeren Frequenzbereichen stärker gedämpft als in höheren Frequenzbereichen.}
{Die Bodenwelle folgt der Erdkrümmung und geht nicht über den geografischen Horizont hinaus. Sie wird in höheren Frequenzbereichen stärker gedämpft als in niedrigeren Frequenzbereichen.}
{Die Bodenwelle folgt der Erdkrümmung und geht über den geografischen Horizont hinaus. Sie wird in niedrigeren Frequenzbereichen stärker gedämpft als in höheren Frequenzbereichen.}
{\textbf{\textcolor{DARCgreen}{Die Bodenwelle folgt der Erdkrümmung und geht über den geografischen Horizont hinaus. Sie wird in höheren Frequenzbereichen stärker gedämpft als in niedrigeren Frequenzbereichen.}}}
\end{QQuestion}

}

\end{frame}%ENDCONTENT
