
\section{Experimentelle Studien}
\label{section:experimentelle_studien}
\begin{frame}%STARTCONTENT
\begin{itemize}
  \item Für besondere experimentelle und technisch-wissenschaftliche Studien
  \item Zeitlich und im Berechtigungsumfang eingeschränkt
  \item Werden gemäß §~16 Absatz 2 Satz 2 AFuV betrieben
  \item Klasse~A: DA5AA bis DA5ZZZ
  \item Klasse~E: DA4AA bis DA4ZZZ
  \item Bei der BNetzA zu beantragen
  \end{itemize}
\end{frame}

\begin{frame}
\only<1>{
\begin{QQuestion}{BD102}{Sie hören die Station DA5XX. Um welche Art von Amateurfunkstelle handelt es sich? Es handelt sich um eine~...}{Kurzzeitzuteilung für einen ausländischen Funkamateur, der eine Amateurfunkprüfungsbescheinigung, aber kein individuelles Rufzeichen hat.}
{Versuchsfunkstelle, die zur Erprobung technischer oder wissenschaftlicher Entwicklungen betrieben wird.}
{exterritoriale deutsche Funkstelle des Amateurfunkdienstes oder des Amateurfunkdienstes über Satelliten.}
{Amateurfunkstelle, die für besondere experimentelle Studien gemäß § 16 Absatz 2 AFuV betrieben wird.}
\end{QQuestion}

}
\only<2>{
\begin{QQuestion}{BD102}{Sie hören die Station DA5XX. Um welche Art von Amateurfunkstelle handelt es sich? Es handelt sich um eine~...}{Kurzzeitzuteilung für einen ausländischen Funkamateur, der eine Amateurfunkprüfungsbescheinigung, aber kein individuelles Rufzeichen hat.}
{Versuchsfunkstelle, die zur Erprobung technischer oder wissenschaftlicher Entwicklungen betrieben wird.}
{exterritoriale deutsche Funkstelle des Amateurfunkdienstes oder des Amateurfunkdienstes über Satelliten.}
{\textbf{\textcolor{DARCgreen}{Amateurfunkstelle, die für besondere experimentelle Studien gemäß § 16 Absatz 2 AFuV betrieben wird.}}}
\end{QQuestion}

}
\end{frame}

\begin{frame}
\only<1>{
\begin{QQuestion}{VD116}{Für welche Zwecke sind Zuteilungen mit Ausnahmen von den technischen und betrieblichen Rahmenbedingungen der Amateurfunkverordnung (AFuV) möglich?}{Für Abgleicharbeiten und Messungen an Sendern ohne Abschlusswiderstand}
{Für Übungen zur Abwicklung des Funkverkehrs in Not- und Katastrophenfällen}
{Für besondere experimentelle und technisch-wissenschaftliche Studien mit einer Amateurfunkstelle}
{Für die Nutzung zusätzlicher Frequenzbereiche, die nicht im Frequenznutzungsplan für den Amateurfunkdienst ausgewiesen sind}
\end{QQuestion}

}
\only<2>{
\begin{QQuestion}{VD116}{Für welche Zwecke sind Zuteilungen mit Ausnahmen von den technischen und betrieblichen Rahmenbedingungen der Amateurfunkverordnung (AFuV) möglich?}{Für Abgleicharbeiten und Messungen an Sendern ohne Abschlusswiderstand}
{Für Übungen zur Abwicklung des Funkverkehrs in Not- und Katastrophenfällen}
{\textbf{\textcolor{DARCgreen}{Für besondere experimentelle und technisch-wissenschaftliche Studien mit einer Amateurfunkstelle}}}
{Für die Nutzung zusätzlicher Frequenzbereiche, die nicht im Frequenznutzungsplan für den Amateurfunkdienst ausgewiesen sind}
\end{QQuestion}

}
\end{frame}%ENDCONTENT
