
\section{Unerwünschte Aussendungen II}
\label{section:unerwuenschte_aussendungen_2}
\begin{frame}%STARTCONTENT

\frametitle{Oberwellen}
\begin{itemize}
  \item Ganzzahlige Vielfache der Grundfrequenz
  \item Entstehen durch Signalformen, die nicht sinusförmig sind, insbesondere bei Übersteuerung
  \item Beeinträchtigung anderer Funkdienste
  \item Können reduziert werden
  \end{itemize}
\end{frame}

\begin{frame}
\only<1>{
\begin{QQuestion}{EJ201}{Welche Signalform sollte der Träger einer hochfrequenten Schwingung haben, um Störungen durch Oberwellen zu vermeiden?}{kreisförmig}
{rechteckförmig}
{dreieckförmig}
{sinusförmig}
\end{QQuestion}

}
\only<2>{
\begin{QQuestion}{EJ201}{Welche Signalform sollte der Träger einer hochfrequenten Schwingung haben, um Störungen durch Oberwellen zu vermeiden?}{kreisförmig}
{rechteckförmig}
{dreieckförmig}
{\textbf{\textcolor{DARCgreen}{sinusförmig}}}
\end{QQuestion}

}
\end{frame}

\begin{frame}
\only<1>{
\begin{QQuestion}{EJ202}{Wie kann man hochfrequente Störungen reduzieren, die durch Harmonische hervorgerufen werden? Sie können reduziert werden durch ein~...}{ZF-Filter.}
{Nachbarkanalfilter.}
{Oberwellenfilter.}
{Hochpassfilter.}
\end{QQuestion}

}
\only<2>{
\begin{QQuestion}{EJ202}{Wie kann man hochfrequente Störungen reduzieren, die durch Harmonische hervorgerufen werden? Sie können reduziert werden durch ein~...}{ZF-Filter.}
{Nachbarkanalfilter.}
{\textbf{\textcolor{DARCgreen}{Oberwellenfilter.}}}
{Hochpassfilter.}
\end{QQuestion}

}
\end{frame}

\begin{frame}
\frametitle{Tiefpassfilter}
\begin{columns}
    \begin{column}{0.48\textwidth}
    \begin{itemize}
  \item Nur Frequenzen unterhalb einer bestimmten Grenzfrequenz werden durchgelassen
  \item Oberwellen können nicht passieren oder werden stark abgeschwächt
  \end{itemize}

    \end{column}
   \begin{column}{0.48\textwidth}
         
\begin{figure}
    \DARCimage{0.85\linewidth}{591include}
    \caption{\scriptsize Frequenzgang eines Tiefpassfilters}
    \label{tiefpass}
\end{figure}


   \end{column}
\end{columns}

\end{frame}

\begin{frame}
\only<1>{
\begin{QQuestion}{EJ204}{Welches Filter wäre zwischen Senderausgang und Antenne eingeschleift am besten zur Verringerung der Oberwellenausstrahlungen geeignet?}{Ein Antennenfilter}
{Ein Hochpassfilter}
{Ein Tiefpassfilter}
{Ein Sperrkreisfilter}
\end{QQuestion}

}
\only<2>{
\begin{QQuestion}{EJ204}{Welches Filter wäre zwischen Senderausgang und Antenne eingeschleift am besten zur Verringerung der Oberwellenausstrahlungen geeignet?}{Ein Antennenfilter}
{Ein Hochpassfilter}
{\textbf{\textcolor{DARCgreen}{Ein Tiefpassfilter}}}
{Ein Sperrkreisfilter}
\end{QQuestion}

}
\end{frame}

\begin{frame}
\only<1>{
\begin{QQuestion}{EJ205}{Um Oberwellenaussendungen eines UHF-Senders zu minimieren, sollte dem Gerät~...}{ein Notchfilter vorgeschaltet werden.}
{ein Hochpassfilter nachgeschaltet werden.}
{eine Bandsperre vorgeschaltet werden.}
{ein Tiefpassfilter nachgeschaltet werden.}
\end{QQuestion}

}
\only<2>{
\begin{QQuestion}{EJ205}{Um Oberwellenaussendungen eines UHF-Senders zu minimieren, sollte dem Gerät~...}{ein Notchfilter vorgeschaltet werden.}
{ein Hochpassfilter nachgeschaltet werden.}
{eine Bandsperre vorgeschaltet werden.}
{\textbf{\textcolor{DARCgreen}{ein Tiefpassfilter nachgeschaltet werden.}}}
\end{QQuestion}

}
\end{frame}

\begin{frame}
\only<1>{
\begin{question2x2}{EJ206}{Welche Schaltung wäre, zwischen Senderausgang und Antenne eingeschleift, am besten zur Verringerung der Oberwellenausstrahlungen geeignet?}{\DARCimage{1.0\linewidth}{161include}}
{\DARCimage{1.0\linewidth}{167include}}
{\DARCimage{1.0\linewidth}{165include}}
{\DARCimage{1.0\linewidth}{168include}}
\end{question2x2}

}
\only<2>{
\begin{question2x2}{EJ206}{Welche Schaltung wäre, zwischen Senderausgang und Antenne eingeschleift, am besten zur Verringerung der Oberwellenausstrahlungen geeignet?}{\DARCimage{1.0\linewidth}{161include}}
{\textbf{\textcolor{DARCgreen}{\DARCimage{1.0\linewidth}{167include}}}}
{\DARCimage{1.0\linewidth}{165include}}
{\DARCimage{1.0\linewidth}{168include}}
\end{question2x2}

}
\end{frame}

\begin{frame}
\only<1>{
\begin{question2x2}{EJ207}{Welche Charakteristik sollte ein Filter zur Verringerung der Oberwellen eines KW-Senders haben?}{\DARCimage{0.75\linewidth}{250include}}
{\DARCimage{0.75\linewidth}{251include}}
{\DARCimage{0.75\linewidth}{258include}}
{\DARCimage{0.75\linewidth}{259include}}
\end{question2x2}

}
\only<2>{
\begin{question2x2}{EJ207}{Welche Charakteristik sollte ein Filter zur Verringerung der Oberwellen eines KW-Senders haben?}{\textbf{\textcolor{DARCgreen}{\DARCimage{0.75\linewidth}{250include}}}}
{\DARCimage{0.75\linewidth}{251include}}
{\DARCimage{0.75\linewidth}{258include}}
{\DARCimage{0.75\linewidth}{259include}}
\end{question2x2}

}
\end{frame}

\begin{frame}
\only<1>{
\begin{question2x2}{EJ208}{Welche Filtercharakteristik würde sich am besten für den Ausgang eines KW-Mehrband-Senders eignen?}{\DARCimage{0.75\linewidth}{252include}}
{\DARCimage{0.75\linewidth}{251include}}
{\DARCimage{0.75\linewidth}{250include}}
{\DARCimage{0.75\linewidth}{253include}}
\end{question2x2}

}
\only<2>{
\begin{question2x2}{EJ208}{Welche Filtercharakteristik würde sich am besten für den Ausgang eines KW-Mehrband-Senders eignen?}{\DARCimage{0.75\linewidth}{252include}}
{\DARCimage{0.75\linewidth}{251include}}
{\textbf{\textcolor{DARCgreen}{\DARCimage{0.75\linewidth}{250include}}}}
{\DARCimage{0.75\linewidth}{253include}}
\end{question2x2}

}
\end{frame}

\begin{frame}
\only<1>{
\begin{QQuestion}{EJ203}{Was für ein Filter muss zwischen Transceiver und Antennenzuleitung eingefügt werden, um Oberwellen zu reduzieren?}{Hochpassfilter}
{Tiefpassfilter}
{CW-Filter}
{NF-Filter}
\end{QQuestion}

}
\only<2>{
\begin{QQuestion}{EJ203}{Was für ein Filter muss zwischen Transceiver und Antennenzuleitung eingefügt werden, um Oberwellen zu reduzieren?}{Hochpassfilter}
{\textbf{\textcolor{DARCgreen}{Tiefpassfilter}}}
{CW-Filter}
{NF-Filter}
\end{QQuestion}

}
\end{frame}

\begin{frame}
\frametitle{Hochpassfilter}
\begin{columns}
    \begin{column}{0.48\textwidth}
    \begin{itemize}
  \item Nur Frequenzen oberhalb einer bestimmten Grenzfrequenz werden durchgelassen
  \item Werden im Empfängereingang verwendet, damit tiefe Frequenzen nicht stören
  \end{itemize}

    \end{column}
   \begin{column}{0.48\textwidth}
          
\begin{figure}
    \DARCimage{0.85\linewidth}{592include}
    \caption{\scriptsize Frequenzgang eines Hochpassfilters}
    \label{hochpass}
\end{figure}


   \end{column}
\end{columns}

\end{frame}

\begin{frame}
\frametitle{Bandpassfilter}
\begin{columns}
    \begin{column}{0.48\textwidth}
    \begin{itemize}
  \item Bei Einbandsendern
  \item Sender im VHF/UHF/SHF-Bereich
  \item Signale aus der Signalaufbereitung unterhalb der Sendefrequenz unterdrücken
  \end{itemize}

    \end{column}
   \begin{column}{0.48\textwidth}
          
\begin{figure}
    \DARCimage{0.85\linewidth}{593include}
    \caption{\scriptsize Frequenzgang eines Bandpassfilters}
    \label{bandpass}
\end{figure}


   \end{column}
\end{columns}

\end{frame}

\begin{frame}
\frametitle{Arbeitspunkt}
\begin{itemize}
  \item Sender-Stufen und Leistungs-Endstufen sollen verzerrungsfrei arbeiten
  \item Nach Veränderung des Arbeitspunkts auf Linearität (saubere Sinus-Verstärkung) prüfen
  \item Aussendung auf Oberwellen prüfen
  \end{itemize}
\end{frame}

\begin{frame}
\only<1>{
\begin{QQuestion}{EF404}{Wann sollte ein Sender auf mögliche Oberwellenaussendungen überprüft werden?}{Vor jedem Sendebetrieb.
}
{Bei Empfang eines Störsignals.}
{Wenn der Arbeitspunkt der Endstufe neu justiert wurde.
}
{Wenn Splatter-Störungen zu hören sind.}
\end{QQuestion}

}
\only<2>{
\begin{QQuestion}{EF404}{Wann sollte ein Sender auf mögliche Oberwellenaussendungen überprüft werden?}{Vor jedem Sendebetrieb.
}
{Bei Empfang eines Störsignals.}
{\textbf{\textcolor{DARCgreen}{Wenn der Arbeitspunkt der Endstufe neu justiert wurde.
}}}
{Wenn Splatter-Störungen zu hören sind.}
\end{QQuestion}

}
\end{frame}%ENDCONTENT
