
\section{Analog vs. Digital}
\label{section:analog_vs_digital}
\begin{frame}%STARTCONTENT

\begin{columns}
    \begin{column}{0.48\textwidth}
    Bei der Informationsübertragung unterscheidet man grundsätzlich zwischen analogen und digitalen Verfahren.

\begin{itemize}
  \item \emph{Digital}: in Stufen, nur bestimmte Werte, keine Werte dazwischen
  \item \emph{Analog}: kontinuierlich, beliebige Zwischenwerte
  \end{itemize}

    \end{column}
   \begin{column}{0.48\textwidth}
       
\begin{figure}
    \DARCimage{0.85\linewidth}{411include}
    \caption{\scriptsize Digitales Signal (abgestuft)}
    \label{n_digital_einleitung_digitales_signal}
\end{figure}


\begin{figure}
    \DARCimage{0.85\linewidth}{408include}
    \caption{\scriptsize Analoges Signal (kontinuierlich)}
    \label{n_digital_einleitung_analoges_signal}
\end{figure}


   \end{column}
\end{columns}

\end{frame}%ENDCONTENT
