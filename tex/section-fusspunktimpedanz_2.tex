
\section{Fußpunktimpedanz II}
\label{section:fusspunktimpedanz_2}
\begin{frame}%STARTCONTENT

\only<1>{
\begin{QQuestion}{AG211}{Welchen Eingangs- bzw. Fußpunktwiderstand hat ein $\lambda$/2-Dipol in ungefähr einer Wellenlänge Höhe über dem Boden bei seiner Grundfrequenz?}{ca.~65~bis~\qty{75}{\ohm}}
{ca.~\qty{30}{\ohm}}
{ca.~\qty{120}{\ohm}}
{ca.~240~bis~\qty{300}{\ohm}}
\end{QQuestion}

}
\only<2>{
\begin{QQuestion}{AG211}{Welchen Eingangs- bzw. Fußpunktwiderstand hat ein $\lambda$/2-Dipol in ungefähr einer Wellenlänge Höhe über dem Boden bei seiner Grundfrequenz?}{\textbf{\textcolor{DARCgreen}{ca.~65~bis~\qty{75}{\ohm}}}}
{ca.~\qty{30}{\ohm}}
{ca.~\qty{120}{\ohm}}
{ca.~240~bis~\qty{300}{\ohm}}
\end{QQuestion}

}
\end{frame}

\begin{frame}
\only<1>{
\begin{QQuestion}{AG209}{Der Fusspunktwiderstand eines mittengespeisten $\lambda$/2-Dipols zeigt sich bei dessen Resonanzfrequenzen~...}{abwechselnd als kapazitiver oder induktiver Blindwiderstand.}
{im Wesentlichen als kapazitiver Blindwiderstand.}
{im Wesentlichen als induktiver Blindwiderstand.}
{im Wesentlichen als Wirkwiderstand.}
\end{QQuestion}

}
\only<2>{
\begin{QQuestion}{AG209}{Der Fusspunktwiderstand eines mittengespeisten $\lambda$/2-Dipols zeigt sich bei dessen Resonanzfrequenzen~...}{abwechselnd als kapazitiver oder induktiver Blindwiderstand.}
{im Wesentlichen als kapazitiver Blindwiderstand.}
{im Wesentlichen als induktiver Blindwiderstand.}
{\textbf{\textcolor{DARCgreen}{im Wesentlichen als Wirkwiderstand.}}}
\end{QQuestion}

}
\end{frame}

\begin{frame}
\only<1>{
\begin{QQuestion}{AG210}{Welche Fußpunktimpedanz hat ein $\lambda$/2-Dipol unterhalb und oberhalb seiner Grundfrequenz?}{Unterhalb der Grundfrequenz ist die Impedanz kapazitiv, oberhalb induktiv.}
{Unterhalb der Grundfrequenz ist die Impedanz induktiv, oberhalb kapazitiv.}
{Unterhalb der Grundfrequenz ist die Impedanz niedriger, oberhalb höher.}
{Unterhalb der Grundfrequenz ist die Impedanz höher, oberhalb niedriger.}
\end{QQuestion}

}
\only<2>{
\begin{QQuestion}{AG210}{Welche Fußpunktimpedanz hat ein $\lambda$/2-Dipol unterhalb und oberhalb seiner Grundfrequenz?}{\textbf{\textcolor{DARCgreen}{Unterhalb der Grundfrequenz ist die Impedanz kapazitiv, oberhalb induktiv.}}}
{Unterhalb der Grundfrequenz ist die Impedanz induktiv, oberhalb kapazitiv.}
{Unterhalb der Grundfrequenz ist die Impedanz niedriger, oberhalb höher.}
{Unterhalb der Grundfrequenz ist die Impedanz höher, oberhalb niedriger.}
\end{QQuestion}

}
\end{frame}%ENDCONTENT
