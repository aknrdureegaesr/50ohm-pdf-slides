
\section{Fernmeldegeheimnis und Abhörverbot}
\label{section:fernmeldegeheimnis_abhoerverbot}
\begin{frame}%STARTCONTENT

\frametitle{Fernmeldegeheimnis und Abhörverbot}
\begin{itemize}
  \item Bei Empfang, Verwertung oder Weitergabe von Nachrichten, die nicht für Funkamateure, die Allgemeinheit oder einen unbestimmten Personenkreis bestimmt sind, verletzt ein Funkamateur das Fernmeldegeheimnis.
  \end{itemize}
\begin{itemize}
  \item Er darf den Inhalt der Nachrichten sowie die Tatsache ihres Empfangs anderen nicht mitteilen. Das gilt nicht in Not- und Katastrophenfällen.
  \end{itemize}
\end{frame}

\begin{frame}\begin{itemize}
  \item Der Besitz und die Herstellung von Geräten, die einen anderen Gegenstand vortäuschen und dadurch besonders geeignet sind, das nicht öffentlich gesprochene Wort heimlich abzuhören („Wanzen“), ist verboten.
  \end{itemize}
\begin{itemize}
  \item Das Abhören des nicht-öffentlich gesprochenen Wortes ist ein Straftatbestand.
  \end{itemize}
\end{frame}

\begin{frame}
\only<1>{
\begin{QQuestion}{VE202}{Bei welcher Handlung verletzt ein Funkamateur das Fernmeldegeheimnis?}{Bei Empfang, Verwertung oder Weitergabe von Nachrichten, die nicht für Funkamateure, die Allgemeinheit oder einen unbestimmten Personenkreis bestimmt sind.}
{Bei Verwertung oder Weitergabe von Gesprächsinhalten und Daten aus Amateurfunkverbindungen, an denen der Funkamateur nicht selbst beteiligt war.}
{Bei Verwertung oder Weitergabe von Gesprächsinhalten und Daten aus Amateurfunkverbindungen, unabhängig davon, ob der Funkamateur selbst beteiligt war.}
{Bei Aufzeichnung und Weitergabe von Gesprächsinhalten und Daten aus Amateurfunkverbindungen, insbesondere, wenn die Weitergabe an Nicht-Funkamateure erfolgt.}
\end{QQuestion}

}
\only<2>{
\begin{QQuestion}{VE202}{Bei welcher Handlung verletzt ein Funkamateur das Fernmeldegeheimnis?}{\textbf{\textcolor{DARCgreen}{Bei Empfang, Verwertung oder Weitergabe von Nachrichten, die nicht für Funkamateure, die Allgemeinheit oder einen unbestimmten Personenkreis bestimmt sind.}}}
{Bei Verwertung oder Weitergabe von Gesprächsinhalten und Daten aus Amateurfunkverbindungen, an denen der Funkamateur nicht selbst beteiligt war.}
{Bei Verwertung oder Weitergabe von Gesprächsinhalten und Daten aus Amateurfunkverbindungen, unabhängig davon, ob der Funkamateur selbst beteiligt war.}
{Bei Aufzeichnung und Weitergabe von Gesprächsinhalten und Daten aus Amateurfunkverbindungen, insbesondere, wenn die Weitergabe an Nicht-Funkamateure erfolgt.}
\end{QQuestion}

}
\end{frame}

\begin{frame}
\only<1>{
\begin{QQuestion}{VE203}{Wie hat sich ein Funkamateur zu verhalten, der Nachrichten empfängt, die \underline{nicht} für Funkamateure, die Allgemeinheit oder einen unbestimmten Personenkreis bestimmt sind?}{Er hat sofort den Empfänger auszuschalten und die Bundesnetzagentur zu informieren.}
{Er darf Dritten zwar die Tatsache des Empfangs mitteilen, aber nicht den Inhalt und die näheren Umstände. Das gilt auch in Not- und Katastrophenfällen.}
{Er darf anderen Funkamateuren zwar die Tatsache des Empfangs mitteilen, aber nicht den Inhalt.}
{Er darf den Inhalt der Nachrichten sowie die Tatsache ihres Empfangs anderen nicht mitteilen. Das gilt nicht in Not- und Katastrophenfällen.}
\end{QQuestion}

}
\only<2>{
\begin{QQuestion}{VE203}{Wie hat sich ein Funkamateur zu verhalten, der Nachrichten empfängt, die \underline{nicht} für Funkamateure, die Allgemeinheit oder einen unbestimmten Personenkreis bestimmt sind?}{Er hat sofort den Empfänger auszuschalten und die Bundesnetzagentur zu informieren.}
{Er darf Dritten zwar die Tatsache des Empfangs mitteilen, aber nicht den Inhalt und die näheren Umstände. Das gilt auch in Not- und Katastrophenfällen.}
{Er darf anderen Funkamateuren zwar die Tatsache des Empfangs mitteilen, aber nicht den Inhalt.}
{\textbf{\textcolor{DARCgreen}{Er darf den Inhalt der Nachrichten sowie die Tatsache ihres Empfangs anderen nicht mitteilen. Das gilt nicht in Not- und Katastrophenfällen.}}}
\end{QQuestion}

}
\end{frame}

\begin{frame}
\only<1>{
\begin{QQuestion}{VE204}{Bei welchen der genannten Geräte sind nach dem Telekommunikation-Telemedien-Datenschutz-Gesetz (TTDSG) sowohl die Herstellung als auch der Besitz verboten? Bei~...}{Sendeanlagen, die einen anderen Gegenstand vortäuschen und somit zum Abhören des nichtöffentlich gesprochenen Wortes brauchbar sind}
{Scannern, die ein breitbandiges Abhören des Funkspektrums ermöglichen}
{Richtmikrofonen, die geeignet sind, das nichtöffentlich gesprochene Wort eines anderen unbemerkt abzuhören}
{digitalen Tonabspielgeräten, welche auch geeignet sind, Tonaufnahmen anzufertigen}
\end{QQuestion}

}
\only<2>{
\begin{QQuestion}{VE204}{Bei welchen der genannten Geräte sind nach dem Telekommunikation-Telemedien-Datenschutz-Gesetz (TTDSG) sowohl die Herstellung als auch der Besitz verboten? Bei~...}{\textbf{\textcolor{DARCgreen}{Sendeanlagen, die einen anderen Gegenstand vortäuschen und somit zum Abhören des nichtöffentlich gesprochenen Wortes brauchbar sind}}}
{Scannern, die ein breitbandiges Abhören des Funkspektrums ermöglichen}
{Richtmikrofonen, die geeignet sind, das nichtöffentlich gesprochene Wort eines anderen unbemerkt abzuhören}
{digitalen Tonabspielgeräten, welche auch geeignet sind, Tonaufnahmen anzufertigen}
\end{QQuestion}

}
\end{frame}

\begin{frame}
\only<1>{
\begin{QQuestion}{VE201}{Darf ein Funkamateur seine Amateurfunkstelle zum Abhören des nichtöffentlich gesprochenen Wortes verwenden?}{Der Funkamateur gilt als sachkundige Person und darf daher selbst entscheiden, auf welchen Frequenzen er hören darf.}
{Das Abhören des nichtöffentlich gesprochenen Wortes ist ein Straftatbestand.}
{Das Abhören des nichtöffentlich gesprochenen Wortes durch den Funkamateur bedarf einer besonderen Zulassung der BNetzA. }
{Der Funkamateur ist dazu berechtigt, wenn er dazu technisch zugelassene Empfänger benutzt.}
\end{QQuestion}

}
\only<2>{
\begin{QQuestion}{VE201}{Darf ein Funkamateur seine Amateurfunkstelle zum Abhören des nichtöffentlich gesprochenen Wortes verwenden?}{Der Funkamateur gilt als sachkundige Person und darf daher selbst entscheiden, auf welchen Frequenzen er hören darf.}
{\textbf{\textcolor{DARCgreen}{Das Abhören des nichtöffentlich gesprochenen Wortes ist ein Straftatbestand.}}}
{Das Abhören des nichtöffentlich gesprochenen Wortes durch den Funkamateur bedarf einer besonderen Zulassung der BNetzA. }
{Der Funkamateur ist dazu berechtigt, wenn er dazu technisch zugelassene Empfänger benutzt.}
\end{QQuestion}

}
\end{frame}%ENDCONTENT
