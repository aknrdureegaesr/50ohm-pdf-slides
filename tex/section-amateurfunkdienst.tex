
\section{Amateurfunkdienst}
\label{section:amateurfunkdienst}
\begin{frame}%STARTCONTENT

\frametitle{Amateurfunkdienst}
Da der Amateurfunk laut RR ein offizieller Funkdienst ist, beinhaltet er sinn- und verantwortungsvolle Aufgaben, als da wären:

\begin{itemize}
  \item Er dient zur eigenen Ausbildung
  \item Funkverkehr der Funkamateure untereinander
  \item Technische Studien
  \end{itemize}
\end{frame}

\begin{frame}Das Amateurfunkgesetz beschreibt es im Kern genauso, fügt aber folgende Details hinzu:

\begin{itemize}
  \item Den Aspekt der Völkerverständigung
  \item Die Unterstützung von Hilfsaktionen in Not- und Katastrophenfällen
  \end{itemize}

\end{frame}

\begin{frame}
\only<1>{
\begin{QQuestion}{VA101}{In welchem internationalen Regelwerk ist der Begriff \glqq Amateurfunkdienst\grqq{} definiert?}{In den Radio Regulations (RR) der ITU }
{In den Normen und Empfehlungen des ETSI }
{In den Empfehlungen der IARU }
{In den Regelungen der CEPT }
\end{QQuestion}

}
\only<2>{
\begin{QQuestion}{VA101}{In welchem internationalen Regelwerk ist der Begriff \glqq Amateurfunkdienst\grqq{} definiert?}{\textbf{\textcolor{DARCgreen}{In den Radio Regulations (RR) der ITU }}}
{In den Normen und Empfehlungen des ETSI }
{In den Empfehlungen der IARU }
{In den Regelungen der CEPT }
\end{QQuestion}

}
\end{frame}

\begin{frame}
\only<1>{
\begin{QQuestion}{VA102}{Wozu dient der Amateurfunkdienst nach der Begriffsbestimmung in den Radio Regulations (RR)?}{Zur Bereitstellung von Kommunikationsdienstleistungen in Gebieten mit fehlender Kommunikationsinfrastruktur}
{Zur Kommunikation der Funkamateure untereinander und mit anderen Funkdiensten}
{Zur eigenen Ausbildung, für den Funkverkehr der Funkamateure untereinander und für technische Studien}
{Zur Kommunikation von Funkamateuren mit Familienmitgliedern und Freunden}
\end{QQuestion}

}
\only<2>{
\begin{QQuestion}{VA102}{Wozu dient der Amateurfunkdienst nach der Begriffsbestimmung in den Radio Regulations (RR)?}{Zur Bereitstellung von Kommunikationsdienstleistungen in Gebieten mit fehlender Kommunikationsinfrastruktur}
{Zur Kommunikation der Funkamateure untereinander und mit anderen Funkdiensten}
{\textbf{\textcolor{DARCgreen}{Zur eigenen Ausbildung, für den Funkverkehr der Funkamateure untereinander und für technische Studien}}}
{Zur Kommunikation von Funkamateuren mit Familienmitgliedern und Freunden}
\end{QQuestion}

}
\end{frame}

\begin{frame}
\only<1>{
\begin{QQuestion}{VA103}{Wozu dient der Amateurfunkdienst über Satelliten nach der Begriffsbestimmung in den Radio Regulations (RR)?}{Der Ermittlung der Dämpfung der ionisierenden Regionen}
{Der Beobachtung des terrestrischen Wetters im Röntgenspektrum}
{Den gleichen Zwecken wie der übrige Amateurfunkdienst}
{Der Ermittlung der Dämpfung der reflektierenden Schichten im UHF-Bereich

}
\end{QQuestion}

}
\only<2>{
\begin{QQuestion}{VA103}{Wozu dient der Amateurfunkdienst über Satelliten nach der Begriffsbestimmung in den Radio Regulations (RR)?}{Der Ermittlung der Dämpfung der ionisierenden Regionen}
{Der Beobachtung des terrestrischen Wetters im Röntgenspektrum}
{\textbf{\textcolor{DARCgreen}{Den gleichen Zwecken wie der übrige Amateurfunkdienst}}}
{Der Ermittlung der Dämpfung der reflektierenden Schichten im UHF-Bereich

}
\end{QQuestion}

}
\end{frame}

\begin{frame}
\only<1>{
\begin{QQuestion}{VC102}{Im Sinne des Amateurfunkgesetzes ist der Amateurfunkdienst ein Funkdienst, der~...}{von Funkamateuren mit speziell dafür zugelassenen Funkgeräten auf allen im Frequenzplan ausgewiesenen Frequenzen und zur eigenen Weiterbildung ausgeübt werden darf.}
{auf allen im Frequenzplan ausgewiesenen Frequenzen Vorrang gegenüber anderen Funkdiensten genießt und zur Unterstützung von Hilfsaktionen in Not- und Katastrophenfällen ausgeübt werden darf.}
{von Funkamateuren untereinander, zu experimentellen und technisch-wissenschaftlichen Studien, zur eigenen Weiterbildung, zur Völkerverständigung und zur Unterstützung von Hilfsaktionen in Not- und Katastrophenfällen wahrgenommen wird.}
{von Funkamateuren aus persönlicher Neigung, aus gewerblich-wirtschaftlichen Interessen und zu technischen Studien wahrgenommen wird.}
\end{QQuestion}

}
\only<2>{
\begin{QQuestion}{VC102}{Im Sinne des Amateurfunkgesetzes ist der Amateurfunkdienst ein Funkdienst, der~...}{von Funkamateuren mit speziell dafür zugelassenen Funkgeräten auf allen im Frequenzplan ausgewiesenen Frequenzen und zur eigenen Weiterbildung ausgeübt werden darf.}
{auf allen im Frequenzplan ausgewiesenen Frequenzen Vorrang gegenüber anderen Funkdiensten genießt und zur Unterstützung von Hilfsaktionen in Not- und Katastrophenfällen ausgeübt werden darf.}
{\textbf{\textcolor{DARCgreen}{von Funkamateuren untereinander, zu experimentellen und technisch-wissenschaftlichen Studien, zur eigenen Weiterbildung, zur Völkerverständigung und zur Unterstützung von Hilfsaktionen in Not- und Katastrophenfällen wahrgenommen wird.}}}
{von Funkamateuren aus persönlicher Neigung, aus gewerblich-wirtschaftlichen Interessen und zu technischen Studien wahrgenommen wird.}
\end{QQuestion}

}
\end{frame}%ENDCONTENT
