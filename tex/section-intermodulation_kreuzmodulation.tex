
\section{Inter- und Kreuzmodulation}
\label{section:intermodulation_kreuzmodulation}
\begin{frame}%STARTCONTENT

\only<1>{
\begin{QQuestion}{AF217}{Welches Phänomen tritt bei einem gleichzeitigen Empfang zweier Signale an einer nicht linear arbeitenden Empfängerstufe auf?}{Dopplereffekt }
{Frequenzmodulation}
{erhöhter Signal-Rausch-Abstand}
{Intermodulation}
\end{QQuestion}

}
\only<2>{
\begin{QQuestion}{AF217}{Welches Phänomen tritt bei einem gleichzeitigen Empfang zweier Signale an einer nicht linear arbeitenden Empfängerstufe auf?}{Dopplereffekt }
{Frequenzmodulation}
{erhöhter Signal-Rausch-Abstand}
{\textbf{\textcolor{DARCgreen}{Intermodulation}}}
\end{QQuestion}

}
\end{frame}

\begin{frame}
\only<1>{
\begin{QQuestion}{AF219}{Wodurch wird Kreuzmodulation verursacht?}{Durch Übermodulation oder zu großen Hub.}
{Durch Reflexion der Oberwellen im Empfangsverstärker.}
{Durch die Übersteuerung eines Verstärkers.}
{Durch Vermischung eines starken unerwünschten Signals mit dem Nutzsignal.}
\end{QQuestion}

}
\only<2>{
\begin{QQuestion}{AF219}{Wodurch wird Kreuzmodulation verursacht?}{Durch Übermodulation oder zu großen Hub.}
{Durch Reflexion der Oberwellen im Empfangsverstärker.}
{Durch die Übersteuerung eines Verstärkers.}
{\textbf{\textcolor{DARCgreen}{Durch Vermischung eines starken unerwünschten Signals mit dem Nutzsignal.}}}
\end{QQuestion}

}
\end{frame}

\begin{frame}
\only<1>{
\begin{QQuestion}{AF222}{Wodurch kann die Qualität eines empfangenen Signals beispielsweise verringert werden?  }{Durch Betrieb des Empfängers an einem linear geregelten Netzteil}
{Durch Batteriebetrieb des Empfängers}
{Durch eine zu niedrige Rauschzahl des Empfängers}
{Durch starke HF-Signale auf einer sehr nahen Frequenz }
\end{QQuestion}

}
\only<2>{
\begin{QQuestion}{AF222}{Wodurch kann die Qualität eines empfangenen Signals beispielsweise verringert werden?  }{Durch Betrieb des Empfängers an einem linear geregelten Netzteil}
{Durch Batteriebetrieb des Empfängers}
{Durch eine zu niedrige Rauschzahl des Empfängers}
{\textbf{\textcolor{DARCgreen}{Durch starke HF-Signale auf einer sehr nahen Frequenz }}}
\end{QQuestion}

}
\end{frame}

\begin{frame}
\only<1>{
\begin{QQuestion}{AF218}{Was ist die Hauptursache für Intermodulationsprodukte in einem Empfänger?}{Es wird ein zu schmalbandiger Preselektor verwendet.}
{Der Empfänger ist nicht genau auf die Frequenz eingestellt.}
{Es wird ein zu schmalbandiges Quarzfilter verwendet.}
{Die HF-Stufe wird bei zunehmend großen Eingangssignalen zunehmend nichtlinear.}
\end{QQuestion}

}
\only<2>{
\begin{QQuestion}{AF218}{Was ist die Hauptursache für Intermodulationsprodukte in einem Empfänger?}{Es wird ein zu schmalbandiger Preselektor verwendet.}
{Der Empfänger ist nicht genau auf die Frequenz eingestellt.}
{Es wird ein zu schmalbandiges Quarzfilter verwendet.}
{\textbf{\textcolor{DARCgreen}{Die HF-Stufe wird bei zunehmend großen Eingangssignalen zunehmend nichtlinear.}}}
\end{QQuestion}

}
\end{frame}

\begin{frame}
\only<1>{
\begin{question2x2}{AF223}{Welche Konfiguration wäre für die Unterdrückung eines unerwünschten Störsignals am Eingang eines Empfängers hilfreich?}{\DARCimage{1.0\linewidth}{436include}}
{\DARCimage{1.0\linewidth}{435include}}
{\DARCimage{1.0\linewidth}{434include}}
{\DARCimage{1.0\linewidth}{437include}}
\end{question2x2}

}
\only<2>{
\begin{question2x2}{AF223}{Welche Konfiguration wäre für die Unterdrückung eines unerwünschten Störsignals am Eingang eines Empfängers hilfreich?}{\DARCimage{1.0\linewidth}{436include}}
{\DARCimage{1.0\linewidth}{435include}}
{\textbf{\textcolor{DARCgreen}{\DARCimage{1.0\linewidth}{434include}}}}
{\DARCimage{1.0\linewidth}{437include}}
\end{question2x2}

}
\end{frame}

\begin{frame}
\only<1>{
\begin{QQuestion}{AF221}{Welche Empfängereigenschaft beurteilt man mit dem Interception Point IP$_3$?}{Signal-Rausch-Verhältnis}
{Trennschärfe}
{Grenzempfindlichkeit}
{Großsignalfestigkeit}
\end{QQuestion}

}
\only<2>{
\begin{QQuestion}{AF221}{Welche Empfängereigenschaft beurteilt man mit dem Interception Point IP$_3$?}{Signal-Rausch-Verhältnis}
{Trennschärfe}
{Grenzempfindlichkeit}
{\textbf{\textcolor{DARCgreen}{Großsignalfestigkeit}}}
\end{QQuestion}

}
\end{frame}

\begin{frame}
\only<1>{
\begin{QQuestion}{AF220}{Wodurch erreicht man eine Verringerung von Intermodulation und Kreuzmodulation beim Empfang?}{Einschalten der Rauschsperre}
{Einschalten des Vorverstärkers}
{Einschalten des Noise-Blankers}
{Einschalten eines Dämpfungsgliedes vor den Empfängereingang}
\end{QQuestion}

}
\only<2>{
\begin{QQuestion}{AF220}{Wodurch erreicht man eine Verringerung von Intermodulation und Kreuzmodulation beim Empfang?}{Einschalten der Rauschsperre}
{Einschalten des Vorverstärkers}
{Einschalten des Noise-Blankers}
{\textbf{\textcolor{DARCgreen}{Einschalten eines Dämpfungsgliedes vor den Empfängereingang}}}
\end{QQuestion}

}
\end{frame}%ENDCONTENT
