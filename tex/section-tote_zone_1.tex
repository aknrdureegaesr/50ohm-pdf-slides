
\section{Tote Zone I}
\label{section:tote_zone_1}
\begin{frame}%STARTCONTENT

\begin{columns}
    \begin{column}{0.48\textwidth}
    \begin{itemize}
  \item Bereich, wo die \emph{Bodenwelle nicht mehr} hin gelangt
  \item Und die \emph{Raumwelle noch nicht} hingelangt
  \item Abhängig vom Reflexionswinkel der Raumwelle
  \item Funkstationen in der Toten Zone können mich nicht hören
  \end{itemize}

    \end{column}
   \begin{column}{0.48\textwidth}
       
\begin{figure}
    \DARCimage{0.85\linewidth}{741include}
    \caption{\scriptsize Tote Zone}
    \label{e_tote_zone}
\end{figure}


   \end{column}
\end{columns}

\end{frame}

\begin{frame}
\only<1>{
\begin{QQuestion}{EH201}{Unter der \glqq Toten Zone\grqq{} wird der Bereich verstanden,~...}{der durch die Bodenwelle erreicht wird und für die Raumwelle nicht zugänglich ist.}
{der durch die Bodenwelle überdeckt wird, so dass schwächere DX-Stationen zugedeckt werden.}
{der durch die Bodenwelle nicht mehr erreicht wird und durch die Raumwelle noch nicht erreicht wird.}
{der durch die Überlagerung der Bodenwelle mit der Raumwelle in einer Zone der gegenseitigen Auslöschung liegt.}
\end{QQuestion}

}
\only<2>{
\begin{QQuestion}{EH201}{Unter der \glqq Toten Zone\grqq{} wird der Bereich verstanden,~...}{der durch die Bodenwelle erreicht wird und für die Raumwelle nicht zugänglich ist.}
{der durch die Bodenwelle überdeckt wird, so dass schwächere DX-Stationen zugedeckt werden.}
{\textbf{\textcolor{DARCgreen}{der durch die Bodenwelle nicht mehr erreicht wird und durch die Raumwelle noch nicht erreicht wird.}}}
{der durch die Überlagerung der Bodenwelle mit der Raumwelle in einer Zone der gegenseitigen Auslöschung liegt.}
\end{QQuestion}

}
\end{frame}%ENDCONTENT
