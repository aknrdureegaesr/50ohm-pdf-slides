
\section{Spiegelfrequenzen}
\label{section:spiegelfrequenzen}
\begin{frame}%STARTCONTENT

\only<1>{
\begin{PQuestion}{AF201}{Welche Differenz liegt zwischen der HF-Nutzfrequenz und der Spiegelfrequenz?}{Das Doppelte der HF-Nutzfrequenz}
{Das Doppelte der ZF}
{Das Dreifache der ZF}
{Die HF-Nutzfrequenz plus der ZF}
{\DARCimage{1.0\linewidth}{80include}}\end{PQuestion}

}
\only<2>{
\begin{PQuestion}{AF201}{Welche Differenz liegt zwischen der HF-Nutzfrequenz und der Spiegelfrequenz?}{Das Doppelte der HF-Nutzfrequenz}
{\textbf{\textcolor{DARCgreen}{Das Doppelte der ZF}}}
{Das Dreifache der ZF}
{Die HF-Nutzfrequenz plus der ZF}
{\DARCimage{1.0\linewidth}{80include}}\end{PQuestion}

}
\end{frame}

\begin{frame}
\only<1>{
\begin{PQuestion}{AF202}{Der VCO schwingt auf \qty{134,9}{\MHz}. Die Empfangsfrequenz soll \qty{145,6}{\MHz} betragen. Welche Spiegelfrequenz kann Störungen beim Empfang verursachen?}{\qty{280,5}{\MHz} }
{\qty{134,9}{\MHz}}
{\qty{124,2}{\MHz}}
{\qty{156,3}{\MHz}}
{\DARCimage{1.0\linewidth}{80include}}\end{PQuestion}

}
\only<2>{
\begin{PQuestion}{AF202}{Der VCO schwingt auf \qty{134,9}{\MHz}. Die Empfangsfrequenz soll \qty{145,6}{\MHz} betragen. Welche Spiegelfrequenz kann Störungen beim Empfang verursachen?}{\qty{280,5}{\MHz} }
{\qty{134,9}{\MHz}}
{\textbf{\textcolor{DARCgreen}{\qty{124,2}{\MHz}}}}
{\qty{156,3}{\MHz}}
{\DARCimage{1.0\linewidth}{80include}}\end{PQuestion}

}
\end{frame}

\begin{frame}
\frametitle{Lösungsweg}
\begin{itemize}
  \item gegeben: $f_{OSZ} = 134,9MHz$
  \item gegeben: $f_E = 145,6MHz$
  \item gesucht: $f_S$
  \end{itemize}
    \pause
    $f_S = 2 \cdot f_{OSZ} -- f_E = 2 \cdot 134,9MHz -- 145,6MHz = 124,2MHz$



\end{frame}

\begin{frame}
\only<1>{
\begin{QQuestion}{AF203}{Der Quarzoszillator schwingt auf \qty{39}{\MHz}. Die Empfangsfrequenz soll \qty{28,3}{\MHz} betragen. Auf welcher Frequenz ist mit Spiegelfrequenzstörungen zu rechnen?}{\qty{67,3}{\MHz}}
{\qty{39}{\MHz}}
{\qty{49,7}{\MHz}}
{\qty{17,6}{\MHz}}
\end{QQuestion}

}
\only<2>{
\begin{QQuestion}{AF203}{Der Quarzoszillator schwingt auf \qty{39}{\MHz}. Die Empfangsfrequenz soll \qty{28,3}{\MHz} betragen. Auf welcher Frequenz ist mit Spiegelfrequenzstörungen zu rechnen?}{\qty{67,3}{\MHz}}
{\qty{39}{\MHz}}
{\textbf{\textcolor{DARCgreen}{\qty{49,7}{\MHz}}}}
{\qty{17,6}{\MHz}}
\end{QQuestion}

}
\end{frame}

\begin{frame}
\frametitle{Lösungsweg}
\begin{itemize}
  \item gegeben: $f_{OSZ} = 39MHz$
  \item gegeben: $f_E = 28,3MHz$
  \item gesucht: $f_S$
  \end{itemize}
    \pause
    $f_S = 2 \cdot f_{OSZ} -- f_E = 2 \cdot 39MHz -- 28,3MHz = 49,7MHz$



\end{frame}

\begin{frame}
\only<1>{
\begin{QQuestion}{AF204}{Wodurch wird beim Überlagerungsempfänger die Spiegelfrequenzdämpfung bestimmt?}{Durch die Selektion im ZF-Bereich}
{Durch die Demodulatorkennlinie}
{Durch die Vorselektion}
{Durch den Tiefpass im Audioverstärker}
\end{QQuestion}

}
\only<2>{
\begin{QQuestion}{AF204}{Wodurch wird beim Überlagerungsempfänger die Spiegelfrequenzdämpfung bestimmt?}{Durch die Selektion im ZF-Bereich}
{Durch die Demodulatorkennlinie}
{\textbf{\textcolor{DARCgreen}{Durch die Vorselektion}}}
{Durch den Tiefpass im Audioverstärker}
\end{QQuestion}

}
\end{frame}

\begin{frame}
\only<1>{
\begin{QQuestion}{AF106}{Welche Frequenzdifferenz besteht bei einem Einfachsuper immer zwischen der Empfangsfrequenz und der Spiegelfrequenz?}{Die doppelte Empfangsfrequenz}
{Die Frequenz des lokalen Oszillators}
{Die doppelte ZF}
{Die ZF}
\end{QQuestion}

}
\only<2>{
\begin{QQuestion}{AF106}{Welche Frequenzdifferenz besteht bei einem Einfachsuper immer zwischen der Empfangsfrequenz und der Spiegelfrequenz?}{Die doppelte Empfangsfrequenz}
{Die Frequenz des lokalen Oszillators}
{\textbf{\textcolor{DARCgreen}{Die doppelte ZF}}}
{Die ZF}
\end{QQuestion}

}
\end{frame}

\begin{frame}
\only<1>{
\begin{PQuestion}{AF107}{Ein Einfachsuperhet-Empfänger ist auf \qty{14,24}{\MHz} eingestellt. Der Lokaloszillator schwingt mit \qty{24,94}{\MHz} und liegt mit dieser Frequenz über der ZF. Wo können Spiegelfrequenzstörungen auftreten?}{\qty{10,7}{\MHz}}
{\qty{35,64}{\MHz}}
{\qty{3,54}{\MHz}}
{\qty{24,94}{\MHz}}
{\DARCimage{1.0\linewidth}{590include}}\end{PQuestion}

}
\only<2>{
\begin{PQuestion}{AF107}{Ein Einfachsuperhet-Empfänger ist auf \qty{14,24}{\MHz} eingestellt. Der Lokaloszillator schwingt mit \qty{24,94}{\MHz} und liegt mit dieser Frequenz über der ZF. Wo können Spiegelfrequenzstörungen auftreten?}{\qty{10,7}{\MHz}}
{\textbf{\textcolor{DARCgreen}{\qty{35,64}{\MHz}}}}
{\qty{3,54}{\MHz}}
{\qty{24,94}{\MHz}}
{\DARCimage{1.0\linewidth}{590include}}\end{PQuestion}

}
\end{frame}

\begin{frame}
\frametitle{Lösungsweg}
\begin{itemize}
  \item gegeben: $f_{OSZ} = 24,94MHz$
  \item gegeben: $f_E = 14,24MHz$
  \item gesucht: $f_S$
  \end{itemize}
    \pause
    $f_S = 2 \cdot f_{OSZ} -- f_E = 2 \cdot 24,94MHz -- 14,24MHz = 35,64MHz$



\end{frame}

\begin{frame}
\only<1>{
\begin{PQuestion}{AF108}{Ein Einfachsuper hat eine ZF von \qty{10,7}{\MHz} und ist auf \qty{28,5}{\MHz} abgestimmt. Der Oszillator des Empfängers schwingt oberhalb der Empfangsfrequenz. Welche Frequenz hat die Spiegelfrequenz?}{\qty{17,8}{\MHz}}
{\qty{7,1}{\MHz}}
{\qty{39,2}{\MHz}}
{\qty{49,9}{\MHz}}
{\DARCimage{1.0\linewidth}{590include}}\end{PQuestion}

}
\only<2>{
\begin{PQuestion}{AF108}{Ein Einfachsuper hat eine ZF von \qty{10,7}{\MHz} und ist auf \qty{28,5}{\MHz} abgestimmt. Der Oszillator des Empfängers schwingt oberhalb der Empfangsfrequenz. Welche Frequenz hat die Spiegelfrequenz?}{\qty{17,8}{\MHz}}
{\qty{7,1}{\MHz}}
{\qty{39,2}{\MHz}}
{\textbf{\textcolor{DARCgreen}{\qty{49,9}{\MHz}}}}
{\DARCimage{1.0\linewidth}{590include}}\end{PQuestion}

}
\end{frame}

\begin{frame}
\frametitle{Lösungsweg}
\begin{itemize}
  \item gegeben: $f_{ZF} = 10,7MHz$
  \item gegeben: $f_E = 28,5MHz$
  \item gesucht: $f_S$
  \end{itemize}
    \pause
    Bei $f_E < f_{OSZ}$:

$f_S = f_E + 2 \cdot f_{ZF} = 28,5MHz + 2 \cdot 10,7MHz = 49,9MHz$



\end{frame}

\begin{frame}
\only<1>{
\begin{QQuestion}{AF109}{Welchen Vorteil haben Kurzwellenempfänger mit einer sehr hohen ersten ZF-Frequenz (z.~B. \qty{50}{\MHz})?}{Man erhält einen Empfänger für Kurzwelle und gleichzeitig für Ultrakurzwelle.}
{Filter für \qty{50}{\MHz} haben eine höhere Trennschärfe als Filter mit niedrigerer Frequenz.}
{Ein solcher Empfänger hat eine höhere Großsignalfestigkeit.}
{Die Spiegelfrequenz liegt sehr weit außerhalb des Empfangsbereichs.}
\end{QQuestion}

}
\only<2>{
\begin{QQuestion}{AF109}{Welchen Vorteil haben Kurzwellenempfänger mit einer sehr hohen ersten ZF-Frequenz (z.~B. \qty{50}{\MHz})?}{Man erhält einen Empfänger für Kurzwelle und gleichzeitig für Ultrakurzwelle.}
{Filter für \qty{50}{\MHz} haben eine höhere Trennschärfe als Filter mit niedrigerer Frequenz.}
{Ein solcher Empfänger hat eine höhere Großsignalfestigkeit.}
{\textbf{\textcolor{DARCgreen}{Die Spiegelfrequenz liegt sehr weit außerhalb des Empfangsbereichs.}}}
\end{QQuestion}

}
\end{frame}

\begin{frame}
\only<1>{
\begin{QQuestion}{AF110}{Wodurch wird beim Überlagerungsempfänger mit einer ZF die Spiegelfrequenzunterdrückung hauptsächlich bestimmt?}{Durch die Verstärkung der ZF}
{Durch die Höhe der ZF}
{Durch die Bandbreite der ZF-Filter}
{Durch die NF-Bandbreite}
\end{QQuestion}

}
\only<2>{
\begin{QQuestion}{AF110}{Wodurch wird beim Überlagerungsempfänger mit einer ZF die Spiegelfrequenzunterdrückung hauptsächlich bestimmt?}{Durch die Verstärkung der ZF}
{\textbf{\textcolor{DARCgreen}{Durch die Höhe der ZF}}}
{Durch die Bandbreite der ZF-Filter}
{Durch die NF-Bandbreite}
\end{QQuestion}

}
\end{frame}

\begin{frame}
\only<1>{
\begin{QQuestion}{AF111}{Welchen Vorteil bietet eine hohe erste Zwischenfrequenz bei Überlagerungsempfängern?}{Sie reduziert Beeinflussungen des lokalen Oszillators durch Empfangssignale.}
{Sie ermöglicht eine hohe Spiegelfrequenzunterdrückung.}
{Sie vermeidet eine hohe Spiegelfrequenzunterdrückung.}
{Sie ermöglicht eine gute Nahselektion. }
\end{QQuestion}

}
\only<2>{
\begin{QQuestion}{AF111}{Welchen Vorteil bietet eine hohe erste Zwischenfrequenz bei Überlagerungsempfängern?}{Sie reduziert Beeinflussungen des lokalen Oszillators durch Empfangssignale.}
{\textbf{\textcolor{DARCgreen}{Sie ermöglicht eine hohe Spiegelfrequenzunterdrückung.}}}
{Sie vermeidet eine hohe Spiegelfrequenzunterdrückung.}
{Sie ermöglicht eine gute Nahselektion. }
\end{QQuestion}

}
\end{frame}%ENDCONTENT
