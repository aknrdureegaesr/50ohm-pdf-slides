
\section{Aufbau eines Empfängers}
\label{section:aufbau_empfaenger}
\begin{frame}%STARTCONTENT

\frametitle{1. Antenne}
\begin{columns}
    \begin{column}{0.48\textwidth}
    
\begin{figure}
    \DARCimage{0.85\linewidth}{736include}
    \caption{\scriptsize Blockdiagramm eines einfachen Empfängers}
    \label{aufbau_empfaenger_blockdiagramm}
\end{figure}


    \end{column}
   \begin{column}{0.48\textwidth}
       \begin{itemize}
  \item Nimmt Vielzahl von Funkwellen auf
  \item Leitet sie als elektrische Schwingungen weiter
  \end{itemize}

   \end{column}
\end{columns}

\end{frame}

\begin{frame}
\frametitle{2. Bandpassfilter}
\begin{columns}
    \begin{column}{0.48\textwidth}
    
\begin{figure}
    \DARCimage{0.85\linewidth}{736include}
    \caption{\scriptsize Blockdiagramm eines einfachen Empfängers}
    \label{aufbau_empfaenger_blockdiagramm}
\end{figure}


    \end{column}
   \begin{column}{0.48\textwidth}
       \begin{itemize}
  \item Lässt nur gewünschten Frequenzbereich durch
  \item Sperrt alle anderen ungewünschten Frequenzen
  \end{itemize}

   \end{column}
\end{columns}

\end{frame}

\begin{frame}
\frametitle{3. HF-Verstärker}
\begin{columns}
    \begin{column}{0.48\textwidth}
    
\begin{figure}
    \DARCimage{0.85\linewidth}{736include}
    \caption{\scriptsize Blockdiagramm eines einfachen Empfängers}
    \label{aufbau_empfaenger_blockdiagramm}
\end{figure}


    \end{column}
   \begin{column}{0.48\textwidth}
       \begin{itemize}
  \item Verstärkt das herausgefilterte Signal
  \end{itemize}

   \end{column}
\end{columns}

\end{frame}

\begin{frame}
\frametitle{4. Demodulator}
\begin{columns}
    \begin{column}{0.48\textwidth}
    
\begin{figure}
    \DARCimage{0.85\linewidth}{736include}
    \caption{\scriptsize Blockdiagramm eines einfachen Empfängers}
    \label{aufbau_empfaenger_blockdiagramm}
\end{figure}


    \end{column}
   \begin{column}{0.48\textwidth}
       \begin{itemize}
  \item Zurückgewinnung des ursprünglichen Signals, z.B. Sprachsignal
  \item Ergebnis ist das Niederfrequenz-Signal (NF)
  \end{itemize}

   \end{column}
\end{columns}

\end{frame}

\begin{frame}
\frametitle{5. NF-Verstärker}
\begin{columns}
    \begin{column}{0.48\textwidth}
    
\begin{figure}
    \DARCimage{0.85\linewidth}{736include}
    \caption{\scriptsize Blockdiagramm eines einfachen Empfängers}
    \label{aufbau_empfaenger_blockdiagramm}
\end{figure}


    \end{column}
   \begin{column}{0.48\textwidth}
       \begin{itemize}
  \item Verstärkt das demodulierte Signal
  \item NF-Verstärker zum Verstärken des Signals für den Lautsprecher
  \end{itemize}

   \end{column}
\end{columns}

\end{frame}

\begin{frame}
\frametitle{6. Lautsprecher}
\begin{columns}
    \begin{column}{0.48\textwidth}
    
\begin{figure}
    \DARCimage{0.85\linewidth}{736include}
    \caption{\scriptsize Blockdiagramm eines einfachen Empfängers}
    \label{aufbau_empfaenger_blockdiagramm}
\end{figure}


    \end{column}
   \begin{column}{0.48\textwidth}
       \begin{itemize}
  \item Wandelt elektrische Schwingung in Schallwelle um
  \item Signal wird wieder hörbar gemacht
  \end{itemize}

   \end{column}
\end{columns}

\end{frame}

\begin{frame}
\only<1>{
\begin{PQuestion}{NF201}{Was stellt folgendes Blockdiagramm dar?}{Sender}
{Tongenerator}
{Relaisfunkstelle}
{Empfänger}
{\DARCimage{1.0\linewidth}{525include}}\end{PQuestion}

}
\only<2>{
\begin{PQuestion}{NF201}{Was stellt folgendes Blockdiagramm dar?}{Sender}
{Tongenerator}
{Relaisfunkstelle}
{\textbf{\textcolor{DARCgreen}{Empfänger}}}
{\DARCimage{1.0\linewidth}{525include}}\end{PQuestion}

}
\end{frame}

\begin{frame}
\frametitle{Empfindlichkeit}
\begin{itemize}
  \item Je nach Aufbau haben Empfänger unterschiedliche Eigenschaften
  \item Wichtige Eigenschaft: \emph{Empfindlichkeit}
  \item Fähigkeit, schwache Signale zu empfangen
  \item Je empfindlicher, umso schwächere Signale können empfangen werden
  \end{itemize}
\end{frame}

\begin{frame}
\only<1>{
\begin{QQuestion}{NF303}{Worauf bezieht sich die Empfindlichkeit eines Empfängers?}{Auf die Fähigkeit, schwache Signale zu empfangen}
{Auf die Stabilität des VFO}
{Auf die Bandbreite des HF-Vorverstärkers}
{Auf die Fähigkeit, starke Signale zu unterdrücken}
\end{QQuestion}

}
\only<2>{
\begin{QQuestion}{NF303}{Worauf bezieht sich die Empfindlichkeit eines Empfängers?}{\textbf{\textcolor{DARCgreen}{Auf die Fähigkeit, schwache Signale zu empfangen}}}
{Auf die Stabilität des VFO}
{Auf die Bandbreite des HF-Vorverstärkers}
{Auf die Fähigkeit, starke Signale zu unterdrücken}
\end{QQuestion}

}
\end{frame}%ENDCONTENT
