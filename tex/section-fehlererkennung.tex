
\section{Fehlererkennung}
\label{section:fehlererkennung}
\begin{frame}%STARTCONTENT

\only<1>{
\begin{QQuestion}{AE411}{Eine digitale Übertragung wird durch ein einzelnes Prüfbit (Parity Bit) abgesichert. Der Empfänger stellt bei der Paritätsprüfung einen Übertragungsfehler fest. Wie viele Bits einschließlich des Prüfbits wurden fehlerhaft übertragen?}{Eine gerade Anzahl Bits}
{Eine ungerade Anzahl Bits}
{Mindestens zwei Bits}
{Maximal zwei Bits}
\end{QQuestion}

}
\only<2>{
\begin{QQuestion}{AE411}{Eine digitale Übertragung wird durch ein einzelnes Prüfbit (Parity Bit) abgesichert. Der Empfänger stellt bei der Paritätsprüfung einen Übertragungsfehler fest. Wie viele Bits einschließlich des Prüfbits wurden fehlerhaft übertragen?}{Eine gerade Anzahl Bits}
{\textbf{\textcolor{DARCgreen}{Eine ungerade Anzahl Bits}}}
{Mindestens zwei Bits}
{Maximal zwei Bits}
\end{QQuestion}

}
\end{frame}

\begin{frame}
\only<1>{
\begin{QQuestion}{AE412}{Eine digitale Übertragung wird durch ein einzelnes Prüfbit (Parity Bit) abgesichert. Der Empfänger stellt bei der Paritätsprüfung \underline{keinen} Übertragungsfehler fest. Was sagt dies über die Fehlerfreiheit der übertragenen Nutzdaten und des Prüfbits aus?}{Die Übertragung war fehlerfrei oder es ist eine gerade Anzahl an Bitfehlern aufgetreten.}
{Die Übertragung war fehlerfrei oder es ist eine ungerade Anzahl an Bitfehlern aufgetreten.}
{Die Übertragung war fehlerfrei.}
{Die Nutzdaten wurden fehlerfrei, das Prüfbit jedoch fehlerhaft übertragen.}
\end{QQuestion}

}
\only<2>{
\begin{QQuestion}{AE412}{Eine digitale Übertragung wird durch ein einzelnes Prüfbit (Parity Bit) abgesichert. Der Empfänger stellt bei der Paritätsprüfung \underline{keinen} Übertragungsfehler fest. Was sagt dies über die Fehlerfreiheit der übertragenen Nutzdaten und des Prüfbits aus?}{\textbf{\textcolor{DARCgreen}{Die Übertragung war fehlerfrei oder es ist eine gerade Anzahl an Bitfehlern aufgetreten.}}}
{Die Übertragung war fehlerfrei oder es ist eine ungerade Anzahl an Bitfehlern aufgetreten.}
{Die Übertragung war fehlerfrei.}
{Die Nutzdaten wurden fehlerfrei, das Prüfbit jedoch fehlerhaft übertragen.}
\end{QQuestion}

}
\end{frame}

\begin{frame}
\only<1>{
\begin{QQuestion}{AE410}{Was wird unter zyklischer Redundanzprüfung (CRC) verstanden?}{Wiederholte (zyklisch redundante) Prüfung der Amateurfunkanlage auf Fehler. }
{Die fortlaufende Prüfung eines zu übertragenden Datenstroms auf Redundanz.}
{Umlaufende (zyklische) Überwachung einer Frequenz durch mehrere Stationen.}
{Ein Prüfsummenverfahren zur Fehlererkennung in Datenblöcken variabler Länge.}
\end{QQuestion}

}
\only<2>{
\begin{QQuestion}{AE410}{Was wird unter zyklischer Redundanzprüfung (CRC) verstanden?}{Wiederholte (zyklisch redundante) Prüfung der Amateurfunkanlage auf Fehler. }
{Die fortlaufende Prüfung eines zu übertragenden Datenstroms auf Redundanz.}
{Umlaufende (zyklische) Überwachung einer Frequenz durch mehrere Stationen.}
{\textbf{\textcolor{DARCgreen}{Ein Prüfsummenverfahren zur Fehlererkennung in Datenblöcken variabler Länge.}}}
\end{QQuestion}

}
\end{frame}%ENDCONTENT
