
\section{Yagi-Uda-Antenne}
\label{section:yagi_uda_1}
\begin{frame}%STARTCONTENT

\begin{columns}
    \begin{column}{0.48\textwidth}
    
\begin{figure}
    \DARCimage{0.85\linewidth}{613include}
    \caption{\scriptsize Yagi-Uda-Antenne mit Einspeisung am Dipol am vorletzten Element}
    \label{n_yagi_uda}
\end{figure}


    \end{column}
   \begin{column}{0.48\textwidth}
       \begin{itemize}
  \item Vor und hinter dem Dipol werden leitende Stäbe geschickt angeordnet
  \item Bündelt Funkwellen in eine bestimmte Richtung
  \end{itemize}

   \end{column}
\end{columns}

\end{frame}

\begin{frame}
\only<1>{
\begin{PQuestion}{NG108}{Wie wird die dargestellte Antenne bezeichnet?}{Dipol-Antenne}
{Yagi-Uda-Antenne}
{Groundplane-Antenne}
{Endgespeiste Antenne}
{\DARCimage{0.5\linewidth}{613include}}\end{PQuestion}

}
\only<2>{
\begin{PQuestion}{NG108}{Wie wird die dargestellte Antenne bezeichnet?}{Dipol-Antenne}
{\textbf{\textcolor{DARCgreen}{Yagi-Uda-Antenne}}}
{Groundplane-Antenne}
{Endgespeiste Antenne}
{\DARCimage{0.5\linewidth}{613include}}\end{PQuestion}

}
\end{frame}%ENDCONTENT
