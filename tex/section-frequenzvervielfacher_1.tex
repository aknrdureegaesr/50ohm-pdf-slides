
\section{Frequenzvervielfacher I}
\label{section:frequenzvervielfacher_1}
\begin{frame}%STARTCONTENT

\begin{columns}
    \begin{column}{0.48\textwidth}
    \begin{itemize}
  \item Ein Oszillator schwingt nur auf einer Frequenz
  \item Um eine höhere Frequenz zu erhalten, kann diese ganzzahlig vervielfacht werden
  \item Rechts unten im Blockschaltbild ist der Multiplikator
  \end{itemize}

    \end{column}
   \begin{column}{0.48\textwidth}
       
\begin{figure}
    \DARCimage{0.85\linewidth}{316include}
    \caption{\scriptsize Frequenzvervielfacher nach einem Oszillator}
    \label{e_frequenzvervielfacher}
\end{figure}


   \end{column}
\end{columns}

\end{frame}

\begin{frame}
\only<1>{
\begin{PQuestion}{EF301}{Auf welcher Frequenz muss der Quarzoszillator schwingen, damit nach dem Blockschaltbild von der PA die Frequenz \qty{145,200}{\MHz} verstärkt wird?}{\qty{36,3}{\MHz}}
{\qty{12,1}{\MHz}}
{\qty{18,15}{\MHz}}
{\qty{24,2}{\MHz}}
{\DARCimage{1.0\linewidth}{316include}}\end{PQuestion}

}
\only<2>{
\begin{PQuestion}{EF301}{Auf welcher Frequenz muss der Quarzoszillator schwingen, damit nach dem Blockschaltbild von der PA die Frequenz \qty{145,200}{\MHz} verstärkt wird?}{\qty{36,3}{\MHz}}
{\textbf{\textcolor{DARCgreen}{\qty{12,1}{\MHz}}}}
{\qty{18,15}{\MHz}}
{\qty{24,2}{\MHz}}
{\DARCimage{1.0\linewidth}{316include}}\end{PQuestion}

}

\end{frame}

\begin{frame}
\only<1>{
\begin{PQuestion}{EF302}{Am Ausgang a dieser Frequenzaufbereitung wird eine Frequenz von \qty{21,360}{\MHz} gemessen. Welche Frequenz hat der VFO?}{\qty{3,560}{\MHz}}
{\qty{4,272}{\MHz}}
{\qty{7,120}{\MHz}}
{\qty{5,340}{\MHz}}
{\DARCimage{1.0\linewidth}{100include}}\end{PQuestion}

}
\only<2>{
\begin{PQuestion}{EF302}{Am Ausgang a dieser Frequenzaufbereitung wird eine Frequenz von \qty{21,360}{\MHz} gemessen. Welche Frequenz hat der VFO?}{\textbf{\textcolor{DARCgreen}{\qty{3,560}{\MHz}}}}
{\qty{4,272}{\MHz}}
{\qty{7,120}{\MHz}}
{\qty{5,340}{\MHz}}
{\DARCimage{1.0\linewidth}{100include}}\end{PQuestion}

}

\end{frame}

\begin{frame}
\only<1>{
\begin{PQuestion}{EF303}{Das Blockschaltbild stellt die Frequenzaufbereitung eines Mehrbandsenders dar. Welche Frequenz entsteht am Ausgang a, wenn der VFO auf \qty{3,51}{\MHz} eingestellt ist?}{\qty{21,06}{\MHz}}
{\qty{7,02}{\MHz}}
{\qty{14,04}{\MHz}}
{\qty{28,08}{\MHz}}
{\DARCimage{1.0\linewidth}{99include}}\end{PQuestion}

}
\only<2>{
\begin{PQuestion}{EF303}{Das Blockschaltbild stellt die Frequenzaufbereitung eines Mehrbandsenders dar. Welche Frequenz entsteht am Ausgang a, wenn der VFO auf \qty{3,51}{\MHz} eingestellt ist?}{\qty{21,06}{\MHz}}
{\qty{7,02}{\MHz}}
{\textbf{\textcolor{DARCgreen}{\qty{14,04}{\MHz}}}}
{\qty{28,08}{\MHz}}
{\DARCimage{1.0\linewidth}{99include}}\end{PQuestion}

}

\end{frame}%ENDCONTENT
