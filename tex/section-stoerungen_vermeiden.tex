
\section{Störungen vermeiden}
\label{section:stoerungen_vermeiden}
\begin{frame}%STARTCONTENT

\frametitle{Gründe für Störungen}
\begin{itemize}
  \item Unerwünschte Frequenzanteile, die nicht ausreichend unterdrückt werden
  \item Unzureichend abgeschirmte oder unzureichend geerdete Geräte 
  \item Die gewünschten Aussendungen selber
  \end{itemize}
\end{frame}

\begin{frame}
\frametitle{Störung}
\begin{columns}
    \begin{column}{0.48\textwidth}
    
\begin{figure}
    \DARCimage{0.85\linewidth}{745include}
    \caption{\scriptsize Störung des DVB-T2-Empfang eines Fernsehers durch die Oberschwingung einer Amateurfunkaussendung}
    \label{stoerungen_vermeiden_oberschwingung}
\end{figure}


    \end{column}
   \begin{column}{0.48\textwidth}
       \begin{itemize}
  \item Grenzwerte durch Amateurfunkanlage überschritten
  \item Z.B. Oberschwingung
  \end{itemize}

   \end{column}
\end{columns}

\end{frame}

\begin{frame}
\frametitle{Störende Beeinflussung}
\begin{columns}
    \begin{column}{0.48\textwidth}
    
\begin{figure}
    \DARCimage{0.85\linewidth}{744include}
    \caption{\scriptsize Einstrahlung über die Empfangsantenne}
    \label{stoerungen_vermeiden_einstrahlung}
\end{figure}


\begin{figure}
    \DARCimage{0.85\linewidth}{746include}
    \caption{\scriptsize Direkteinstrahlung in ein Gerät}
    \label{stoerungen_vermeiden_direkteinstrahlung}
\end{figure}


    \end{column}
   \begin{column}{0.48\textwidth}
       
\begin{figure}
    \DARCimage{0.85\linewidth}{747include}
    \caption{\scriptsize Einströmung über Anschlussleitungen}
    \label{stoerungen_vermeiden_einstroemung}
\end{figure}

\begin{itemize}
  \item Grenzwerte werden eingehalten
  \item Wege können einzeln oder zusammen auftreten
  \end{itemize}

   \end{column}
\end{columns}

\end{frame}

\begin{frame}
\frametitle{Umgang mit Störungen}
\begin{itemize}
  \item Nachbarschaftskonflikte vermeiden
  \item Höflich Hilfe zur Entstörung anbieten
  \end{itemize}
\end{frame}

\begin{frame}
\only<1>{
\begin{QQuestion}{NJ102}{Welche Reaktion ist angebracht, wenn ihr Nachbar sich über Störungen beklagt?}{Sie bieten höflich an, die erforderlichen Prüfungen in die Wege zu leiten.}
{Er sollte höflich darauf hingewiesen werden, dass es an seiner eigenen Einrichtung liegt.}
{Er sollte darauf hingewiesen werden, dass Sie hierfür nicht zuständig sind.}
{Sie bieten an, das örtlich zuständige Hauptzollamt zu benachrichtigen.}
\end{QQuestion}

}
\only<2>{
\begin{QQuestion}{NJ102}{Welche Reaktion ist angebracht, wenn ihr Nachbar sich über Störungen beklagt?}{\textbf{\textcolor{DARCgreen}{Sie bieten höflich an, die erforderlichen Prüfungen in die Wege zu leiten.}}}
{Er sollte höflich darauf hingewiesen werden, dass es an seiner eigenen Einrichtung liegt.}
{Er sollte darauf hingewiesen werden, dass Sie hierfür nicht zuständig sind.}
{Sie bieten an, das örtlich zuständige Hauptzollamt zu benachrichtigen.}
\end{QQuestion}

}
\end{frame}

\begin{frame}
\frametitle{Ursachenforschung}
\begin{itemize}
  \item Prüfen auf Behebung mit eigenen Mitteln
  \item Falls Ursache nicht ermittelt oder Störung nicht beseitigt werden kann $\rightarrow$ Nachbarn auf \emph{Funkstörungsannahme der BNetzA} (24/7~ aPhone ~0228~14~15~16) hinweisen
  \end{itemize}

\end{frame}

\begin{frame}
\only<1>{
\begin{QQuestion}{VE302}{Welche Reaktion ist angebracht, wenn Störungen im Fernseh- oder Rundfunkempfang beim Nachbarn nicht mit den zur Verfügung stehenden Mitteln beseitigt werden können?}{Sie empfehlen dem Nachbarn höflich, sich an die Bundesnetzagentur zur Prüfung der Störungsursache zu wenden.}
{Der Nachbar sollte höflich darauf hingewiesen werden, dass es an seiner eigenen Einrichtung liegt.}
{Der Nachbar sollte darauf hingewiesen werden, dass Sie hierfür nicht zuständig sind.}
{Sie benachrichtigen ihren Amateurfunkverband.}
\end{QQuestion}

}
\only<2>{
\begin{QQuestion}{VE302}{Welche Reaktion ist angebracht, wenn Störungen im Fernseh- oder Rundfunkempfang beim Nachbarn nicht mit den zur Verfügung stehenden Mitteln beseitigt werden können?}{\textbf{\textcolor{DARCgreen}{Sie empfehlen dem Nachbarn höflich, sich an die Bundesnetzagentur zur Prüfung der Störungsursache zu wenden.}}}
{Der Nachbar sollte höflich darauf hingewiesen werden, dass es an seiner eigenen Einrichtung liegt.}
{Der Nachbar sollte darauf hingewiesen werden, dass Sie hierfür nicht zuständig sind.}
{Sie benachrichtigen ihren Amateurfunkverband.}
\end{QQuestion}

}
\end{frame}

\begin{frame}
\frametitle{Ermittlung}
\begin{itemize}
  \item Kann einige Zeit in Anspruch nehmen
  \item Zur Wahrung des Nachbarschaftsfriedens Sendeleistung reduzieren
  \end{itemize}
\end{frame}

\begin{frame}
\only<1>{
\begin{QQuestion}{VE301}{Durch den Betrieb einer Amateurfunkstelle wird der Rundfunkempfang eines Nachbarn gestört. Welche Maßnahme kann der Funkamateur zur Wahrung des nachbarschaftlichen Friedens noch vor Einschaltung der Bundesnetzagentur durchführen?}{Er kann die Sendeleistung vorläufig reduzieren.}
{Er schaltet am Transceiver Passband-Tuning ein.}
{Er macht ausschließlich DX-Betrieb.}
{Er macht ausschließlich Split-Betrieb. }
\end{QQuestion}

}
\only<2>{
\begin{QQuestion}{VE301}{Durch den Betrieb einer Amateurfunkstelle wird der Rundfunkempfang eines Nachbarn gestört. Welche Maßnahme kann der Funkamateur zur Wahrung des nachbarschaftlichen Friedens noch vor Einschaltung der Bundesnetzagentur durchführen?}{\textbf{\textcolor{DARCgreen}{Er kann die Sendeleistung vorläufig reduzieren.}}}
{Er schaltet am Transceiver Passband-Tuning ein.}
{Er macht ausschließlich DX-Betrieb.}
{Er macht ausschließlich Split-Betrieb. }
\end{QQuestion}

}
\end{frame}

\begin{frame}
\frametitle{Überprüfung}
Falls Amateurfunkaussendungen die Ursache der Probleme sind, wird in drei Fälle unterschieden

\end{frame}

\begin{frame}
\frametitle{1. Fall}
\begin{itemize}
  \item Amateurfunkanlage wird \emph{nicht vorschriftsmäßig} betrieben
  \item Ggf. Anordnung einer kostenpflichtigen Betriebseinschränkung durch BNetzA
  \item Möglich ist eine Begrenzung der Sendeleistung
  \end{itemize}

\end{frame}

\begin{frame}
\frametitle{2. Fall}
\begin{itemize}
  \item Amateurfunkanlage wird \emph{vorschriftsmäßig} betrieben
  \item Feldstärke am betroffenen Gerät ist kleiner als Verträglichkeit durch die Störfestigkeit
  \item Betroffenes Gerät \emph{hält Störfestigkeit nicht ein}
  \item Verantwortung zur Behebung liegt beim Betreiber des betroffenen Geräts
  \item Funkamateur darf Sendebetrieb unverändert fortsetzen
  \end{itemize}
\end{frame}

\begin{frame}
\only<1>{
\begin{QQuestion}{VE305}{Durch den Betrieb einer Amateurfunkstelle auf \qty{145,550}{\MHz} wird der UKW-Rundfunkempfänger eines Nachbarn durch Direkteinstrahlung beeinträchtigt. Eine Überprüfung ergibt, dass der Funkamateur am Ort des beeinträchtigten Empfängers eine Feldstärke erzeugt, die den in der Norm empfohlenen Grenzwert für die Störfestigkeit von Geräten nicht erreicht. Was folgt daraus für den Funkamateur?}{Er kann seinen Funkbetrieb fortsetzen.}
{Er hat seine Sendeleistung so einzurichten, dass der Empfang nicht mehr beeinträchtigt wird.}
{Er kann seine Sendeleistung uneingeschränkt erhöhen.}
{Er hat den Betrieb seiner Amateurfunkstelle einzustellen.}
\end{QQuestion}

}
\only<2>{
\begin{QQuestion}{VE305}{Durch den Betrieb einer Amateurfunkstelle auf \qty{145,550}{\MHz} wird der UKW-Rundfunkempfänger eines Nachbarn durch Direkteinstrahlung beeinträchtigt. Eine Überprüfung ergibt, dass der Funkamateur am Ort des beeinträchtigten Empfängers eine Feldstärke erzeugt, die den in der Norm empfohlenen Grenzwert für die Störfestigkeit von Geräten nicht erreicht. Was folgt daraus für den Funkamateur?}{\textbf{\textcolor{DARCgreen}{Er kann seinen Funkbetrieb fortsetzen.}}}
{Er hat seine Sendeleistung so einzurichten, dass der Empfang nicht mehr beeinträchtigt wird.}
{Er kann seine Sendeleistung uneingeschränkt erhöhen.}
{Er hat den Betrieb seiner Amateurfunkstelle einzustellen.}
\end{QQuestion}

}
\end{frame}

\begin{frame}
\only<1>{
\begin{QQuestion}{VE306}{Durch den Betrieb einer Amateurfunkstelle auf \qty{144,250}{\MHz} wird der Kabelfernsehempfang eines Nachbarn beeinträchtigt. Eine Überprüfung ergibt, dass der Funkamateur am Ort der beeinträchtigten Empfangsanlage eine Feldstärke erzeugt, die den in der Norm empfohlenen Grenzwert für die Störfestigkeit von Kabelverteilanlagen nicht erreicht. Was folgt daraus für den Funkamateur?}{Er kann seinen Funkbetrieb fortsetzen.}
{Er hat den Betrieb seiner Amateurfunkstelle einzustellen.}
{Er hat seine Sendeleistung so einzurichten, dass der Empfang nicht mehr beeinträchtigt wird.}
{Er kann seine Sendeleistung uneingeschränkt erhöhen.}
\end{QQuestion}

}
\only<2>{
\begin{QQuestion}{VE306}{Durch den Betrieb einer Amateurfunkstelle auf \qty{144,250}{\MHz} wird der Kabelfernsehempfang eines Nachbarn beeinträchtigt. Eine Überprüfung ergibt, dass der Funkamateur am Ort der beeinträchtigten Empfangsanlage eine Feldstärke erzeugt, die den in der Norm empfohlenen Grenzwert für die Störfestigkeit von Kabelverteilanlagen nicht erreicht. Was folgt daraus für den Funkamateur?}{\textbf{\textcolor{DARCgreen}{Er kann seinen Funkbetrieb fortsetzen.}}}
{Er hat den Betrieb seiner Amateurfunkstelle einzustellen.}
{Er hat seine Sendeleistung so einzurichten, dass der Empfang nicht mehr beeinträchtigt wird.}
{Er kann seine Sendeleistung uneingeschränkt erhöhen.}
\end{QQuestion}

}
\end{frame}

\begin{frame}
\frametitle{3. Fall}
\begin{itemize}
  \item Amateurfunkanlage wird \emph{vorschriftsmäßig} betrieben
  \item Betroffenes Gerät \emph{hält Storfestigkeit ein}
  \item Konfliktfall: BNetzA ist befugt, eine Lösung \emph{in Zusammenarbeit mit allen Beteilgten} herzustellen
  \end{itemize}
\end{frame}

\begin{frame}
\only<1>{
\begin{QQuestion}{VE303}{Durch den Betrieb einer Amateurfunkstelle auf \qty{145,550}{\MHz} wird der UKW-Rundfunkempfang eines Nachbarn beeinträchtigt. Eine Überprüfung ergibt, dass sowohl die Amateurfunkstelle als auch die Rundfunkempfangsanlage vorschriftsmäßig betrieben werden. Womit muss der Funkamateur rechnen?}{Mit einer gebührenpflichtigen Betriebseinschränkung oder einem vollständigen Betriebsverbot für seine Amateurfunkstelle}
{Mit behördlichen Abhilfemaßnahmen in Zusammenarbeit mit den Beteiligten}
{Mit der Durchführung behördlicher Maßnahmen nach dem AFuG, wobei dem Funkamateur die Zulassung zur Teilnahme am Amateurfunkdienst entzogen werden kann}
{Mit einem Ordnungswidrigkeitenverfahren mit Betriebsverbot und Bußgeld auf der Grundlage des AFuG}
\end{QQuestion}

}
\only<2>{
\begin{QQuestion}{VE303}{Durch den Betrieb einer Amateurfunkstelle auf \qty{145,550}{\MHz} wird der UKW-Rundfunkempfang eines Nachbarn beeinträchtigt. Eine Überprüfung ergibt, dass sowohl die Amateurfunkstelle als auch die Rundfunkempfangsanlage vorschriftsmäßig betrieben werden. Womit muss der Funkamateur rechnen?}{Mit einer gebührenpflichtigen Betriebseinschränkung oder einem vollständigen Betriebsverbot für seine Amateurfunkstelle}
{\textbf{\textcolor{DARCgreen}{Mit behördlichen Abhilfemaßnahmen in Zusammenarbeit mit den Beteiligten}}}
{Mit der Durchführung behördlicher Maßnahmen nach dem AFuG, wobei dem Funkamateur die Zulassung zur Teilnahme am Amateurfunkdienst entzogen werden kann}
{Mit einem Ordnungswidrigkeitenverfahren mit Betriebsverbot und Bußgeld auf der Grundlage des AFuG}
\end{QQuestion}

}
\end{frame}

\begin{frame}
\only<1>{
\begin{QQuestion}{VE304}{Durch den Betrieb einer Amateurfunkstelle wird der Fernsehempfang eines Nachbarn beeinträchtigt. Eine Überprüfung ergibt, dass sowohl das Fernsehgerät als auch die Amateurfunkstelle die Vorschriften einhalten und Nachbesserungen nicht mehr möglich sind. Wozu ist die BNetzA in diesem Fall befugt?}{Die BNetzA kann Abhilfemaßnahmen in Zusammenarbeit mit den Beteiligten veranlassen.}
{Die BNetzA hat diesbezüglich keine Befugnisse.}
{Zum sofortigen Widerruf der Zulassung zum Amateurfunkdienst}
{Zur Einleitung eines Bußgeldverfahrens}
\end{QQuestion}

}
\only<2>{
\begin{QQuestion}{VE304}{Durch den Betrieb einer Amateurfunkstelle wird der Fernsehempfang eines Nachbarn beeinträchtigt. Eine Überprüfung ergibt, dass sowohl das Fernsehgerät als auch die Amateurfunkstelle die Vorschriften einhalten und Nachbesserungen nicht mehr möglich sind. Wozu ist die BNetzA in diesem Fall befugt?}{\textbf{\textcolor{DARCgreen}{Die BNetzA kann Abhilfemaßnahmen in Zusammenarbeit mit den Beteiligten veranlassen.}}}
{Die BNetzA hat diesbezüglich keine Befugnisse.}
{Zum sofortigen Widerruf der Zulassung zum Amateurfunkdienst}
{Zur Einleitung eines Bußgeldverfahrens}
\end{QQuestion}

}
\end{frame}

\begin{frame}
\frametitle{Anordnungen der BNetzA ohne Zusammenarbeit}
\begin{itemize}
  \item Zum Schutz von Empfangs- und Sendegeräten, die Sicherheitszwecken dienen
  \item Zum Schutz öffentlicher Telekommunikationsnetze, also beispielsweise dem Telefonnetz
  \item Zum Schutz von Leib und Leben einer Person oder Sachen von bedeutendem Wert
  \end{itemize}

\end{frame}%ENDCONTENT
