
\section{DX}
\label{section:dx}
\begin{frame}%STARTCONTENT
\begin{itemize}
  \item Funkverbindung über große Entfernung
  \item DX $\rightarrow$ \enquote{long distance} (aus Morsetelegrafie)
  \item Unterscheidung zwischen Kurzwelle und UKW
  \end{itemize}
\end{frame}

\begin{frame}
\frametitle{DX Kurzwelle}
\begin{columns}
    \begin{column}{0.48\textwidth}
    \begin{itemize}
  \item Kontakt mit Funkamateuren von einem anderen Kontinent
  \item Funkamateure vom selben Kontinent sollten nicht antworten
  \end{itemize}

    \end{column}
   \begin{column}{0.48\textwidth}
       
    \pause\QSOown{CQ DX}



   \end{column}
\end{columns}

\end{frame}

\begin{frame}
\only<1>{
\begin{QQuestion}{BB103}{Was bedeutet die betriebliche Abkürzung DX?}{Kleine Entfernung }
{Große Entfernung }
{Auf dem indirektem Weg}
{Auf dem direktem Weg}
\end{QQuestion}

}
\only<2>{
\begin{QQuestion}{BB103}{Was bedeutet die betriebliche Abkürzung DX?}{Kleine Entfernung }
{\textbf{\textcolor{DARCgreen}{Große Entfernung }}}
{Auf dem indirektem Weg}
{Auf dem direktem Weg}
\end{QQuestion}

}
\end{frame}

\begin{frame}
\only<1>{
\begin{QQuestion}{BE114}{Was bedeutet der im \qty{20}{\m}-Band gesendete Anruf \glqq CQ DX CQ DX DE HB9AFN HB9AFN K\grqq{}? HB9AFN sucht eine Verbindung mit~...}{einem anderen Kontinent und sollte durch europäische Funkamateure nicht angerufen werden.}
{dem Inland und sollte durch ausländische Funkamateure nicht angerufen werden.}
{Stationen in über \qty{500}{\km} Entfernung und sollte durch Funkamateure aus einer geringeren Entfernung nicht angerufen werden.}
{philippinischen Funkamateuren (Präfix \glqq DX\grqq{}) und sollte durch Funkamateure anderer Länder nicht angerufen werden.}
\end{QQuestion}

}
\only<2>{
\begin{QQuestion}{BE114}{Was bedeutet der im \qty{20}{\m}-Band gesendete Anruf \glqq CQ DX CQ DX DE HB9AFN HB9AFN K\grqq{}? HB9AFN sucht eine Verbindung mit~...}{\textbf{\textcolor{DARCgreen}{einem anderen Kontinent und sollte durch europäische Funkamateure nicht angerufen werden.}}}
{dem Inland und sollte durch ausländische Funkamateure nicht angerufen werden.}
{Stationen in über \qty{500}{\km} Entfernung und sollte durch Funkamateure aus einer geringeren Entfernung nicht angerufen werden.}
{philippinischen Funkamateuren (Präfix \glqq DX\grqq{}) und sollte durch Funkamateure anderer Länder nicht angerufen werden.}
\end{QQuestion}

}

\end{frame}

\begin{frame}
\only<1>{
\begin{QQuestion}{BB105}{Eine Station ruft in der Nacht auf \qty{3790}{\kHz} \glqq CQ DX\grqq{}. Wer soll antworten? Nur Stationen~...}{Stationen von anderen Kontinenten}
{mit DX-Präfix}
{im Nahbereich bis \qty{50}{\km} Entfernung}
{aus Deutschland}
\end{QQuestion}

}
\only<2>{
\begin{QQuestion}{BB105}{Eine Station ruft in der Nacht auf \qty{3790}{\kHz} \glqq CQ DX\grqq{}. Wer soll antworten? Nur Stationen~...}{\textbf{\textcolor{DARCgreen}{Stationen von anderen Kontinenten}}}
{mit DX-Präfix}
{im Nahbereich bis \qty{50}{\km} Entfernung}
{aus Deutschland}
\end{QQuestion}

}

\end{frame}

\begin{frame}
\frametitle{DX auf VHF/UHF}
\begin{itemize}
  \item Andere Kontinente erreicht man darüber nur sehr selten
  \item Deshalb Funkkontakte in erkennbar einigen hundert Kilometern Entfernung
  \end{itemize}

\end{frame}

\begin{frame}
\only<1>{
\begin{QQuestion}{BB104}{Eine Station ruft auf VHF/UHF \glqq CQ DX\grqq{}. Wer soll antworten?}{Stationen von anderen Kontinenten}
{Stationen auf den Philippinen}
{Stationen in mehr als einigen \qty{100}{\km} Entfernung}
{Stationen mit deutschem Präfix}
\end{QQuestion}

}
\only<2>{
\begin{QQuestion}{BB104}{Eine Station ruft auf VHF/UHF \glqq CQ DX\grqq{}. Wer soll antworten?}{Stationen von anderen Kontinenten}
{Stationen auf den Philippinen}
{\textbf{\textcolor{DARCgreen}{Stationen in mehr als einigen \qty{100}{\km} Entfernung}}}
{Stationen mit deutschem Präfix}
\end{QQuestion}

}
\end{frame}

\begin{frame}
\only<1>{
\begin{QQuestion}{BE109}{Eine Station ruft auf dem \qty{2}{\m}- oder dem \qty{70}{\cm}-Band \glqq CQ\grqq{} mit dem Zusatz \glqq DX\grqq{}. Wann sollten Sie antworten?}{Nur bei Stationen, die erkennbar einige hundert Kilometer entfernt sind.}
{Nur wenn die Entfernung zwischen beiden Stationen höchstens \qty{500}{\km} beträgt.}
{Nur wenn ich als hörende Station die rufende Station mit guter Lautstärke empfange.}
{Nur wenn es sich bei der anrufenden Station um eine außereuropäische Station handelt.}
\end{QQuestion}

}
\only<2>{
\begin{QQuestion}{BE109}{Eine Station ruft auf dem \qty{2}{\m}- oder dem \qty{70}{\cm}-Band \glqq CQ\grqq{} mit dem Zusatz \glqq DX\grqq{}. Wann sollten Sie antworten?}{\textbf{\textcolor{DARCgreen}{Nur bei Stationen, die erkennbar einige hundert Kilometer entfernt sind.}}}
{Nur wenn die Entfernung zwischen beiden Stationen höchstens \qty{500}{\km} beträgt.}
{Nur wenn ich als hörende Station die rufende Station mit guter Lautstärke empfange.}
{Nur wenn es sich bei der anrufenden Station um eine außereuropäische Station handelt.}
\end{QQuestion}

}
\end{frame}

\begin{frame}
\frametitle{Gezielter CQ-Ruf}
\begin{columns}
    \begin{column}{0.48\textwidth}
    \begin{itemize}
  \item Für Verbindungen in ein spezielles Land
  \item Landeskenner beim CQ-Ruf einsetzen
  \end{itemize}

    \end{column}
   \begin{column}{0.48\textwidth}
       
    \pause\QSOown{CQ VK/ZL}
    \pause
    Rufe Stationen in Australien oder Neuseeland




   \end{column}
\end{columns}

\end{frame}

\begin{frame}
\only<1>{
\begin{QQuestion}{BE110}{Sie hören 4U1ITU in Telefonie rufen: \glqq CQ VK/ZL this is 4U1ITU\grqq{}. Sollten Sie 4U1ITU anrufen, wenn Sie gerne ein QSO mit der Station führen würden?}{Ja! Aber nur wenn Sie geborener Australier oder Neuseeländer sind.}
{Ja! 4U1ITU in Australien/Neuseeland sucht eine Verbindung.}
{Nein! 4U1ITU sucht eine Verbindung mit Australien oder Neuseeland.}
{Nein! 4U1ITU sucht nur Verbindungen mit Indien oder Südafrika.}
\end{QQuestion}

}
\only<2>{
\begin{QQuestion}{BE110}{Sie hören 4U1ITU in Telefonie rufen: \glqq CQ VK/ZL this is 4U1ITU\grqq{}. Sollten Sie 4U1ITU anrufen, wenn Sie gerne ein QSO mit der Station führen würden?}{Ja! Aber nur wenn Sie geborener Australier oder Neuseeländer sind.}
{Ja! 4U1ITU in Australien/Neuseeland sucht eine Verbindung.}
{\textbf{\textcolor{DARCgreen}{Nein! 4U1ITU sucht eine Verbindung mit Australien oder Neuseeland.}}}
{Nein! 4U1ITU sucht nur Verbindungen mit Indien oder Südafrika.}
\end{QQuestion}

}

\end{frame}

\begin{frame}
\only<1>{
\begin{QQuestion}{BE113}{N4EAX ruft in Telegrafie: \glqq CQ DL CQ DL DE N4EAX N4EAX PSE K\grqq{}. Was beabsichtigt die Amateurfunkstelle damit?}{N4EAX sucht eine Verbindung mit einem Funkamateur, dessen Rufzeichen mit \glqq D\grqq{} oder \glqq L\grqq{} beginnt.}
{N4EAX sucht Verbindungen in digitalen Übertragungsverfahren (Data Link).}
{N4EAX sucht eine Verbindung mit einem Funkamateur in Deutschland.}
{N4EAX sucht Verbindungen mit Stationen bei Tageslicht (Day Light), um die Grayline-Bedingungen optimal auszunutzen.}
\end{QQuestion}

}
\only<2>{
\begin{QQuestion}{BE113}{N4EAX ruft in Telegrafie: \glqq CQ DL CQ DL DE N4EAX N4EAX PSE K\grqq{}. Was beabsichtigt die Amateurfunkstelle damit?}{N4EAX sucht eine Verbindung mit einem Funkamateur, dessen Rufzeichen mit \glqq D\grqq{} oder \glqq L\grqq{} beginnt.}
{N4EAX sucht Verbindungen in digitalen Übertragungsverfahren (Data Link).}
{\textbf{\textcolor{DARCgreen}{N4EAX sucht eine Verbindung mit einem Funkamateur in Deutschland.}}}
{N4EAX sucht Verbindungen mit Stationen bei Tageslicht (Day Light), um die Grayline-Bedingungen optimal auszunutzen.}
\end{QQuestion}

}
\end{frame}

\begin{frame}
\frametitle{Englische Sprache}
\begin{itemize}
  \item Internationale Verbindungen werden in der Regel in Englisch geführt
  \item Antwort auf Englisch geben
  \end{itemize}

\end{frame}

\begin{frame}
\only<1>{
\begin{QQuestion}{BE104}{EA6VQ ruft in Telefonie in englischer Sprache CQ. Ihr Rufzeichen ist DF1KW. Wie könnten Sie antworten?}{QRZ EA6VQ from DF1KW, over.}
{CQ CQ CQ de DF1KW for EA6VQ, please go ahead.}
{EA6VQ, es ruft Sie DF1KW, bitte kommen.}
{EA6VQ, this is DF1KW calling you.}
\end{QQuestion}

}
\only<2>{
\begin{QQuestion}{BE104}{EA6VQ ruft in Telefonie in englischer Sprache CQ. Ihr Rufzeichen ist DF1KW. Wie könnten Sie antworten?}{QRZ EA6VQ from DF1KW, over.}
{CQ CQ CQ de DF1KW for EA6VQ, please go ahead.}
{EA6VQ, es ruft Sie DF1KW, bitte kommen.}
{\textbf{\textcolor{DARCgreen}{EA6VQ, this is DF1KW calling you.}}}
\end{QQuestion}

}
\end{frame}

\begin{frame}
\frametitle{Unbeantworteter CQ DX}
\begin{itemize}
  \item Bleibt ein CQ DX Ruf lange Zeit unbeantwortet
  \item Empfehlung: Wechseln auf normalen CQ Ruf
  \item Kontakt zu Stationen aus der Umgebung aufnehmen
  \end{itemize}
\end{frame}

\begin{frame}
\frametitle{DX-Pedition}
\begin{itemize}
  \item Von besonderen Flecken der Erde DX-Funkaktivitäten durchführen
  \item Meistens an den entlegensten Orten der Welt
  \item Wird als \enquote{DX-Pedition} bezeichnet
  \end{itemize}

\end{frame}

\begin{frame}
\only<1>{
\begin{QQuestion}{BE312}{Was versteht man im Amateurfunk unter einer \glqq DX-Pedition\grqq{}?}{Es ist eine Zusammenstellung aller noch von Funkamateuren begehrten Länder.}
{Es ist eine weltweite Aktivitätswoche.}
{Es ist ein internationaler Funkwettbewerb.}
{Es ist eine Amateurfunkexpedition zu Ländern oder Inseln, die selten im Amateurfunk zu hören sind.}
\end{QQuestion}

}
\only<2>{
\begin{QQuestion}{BE312}{Was versteht man im Amateurfunk unter einer \glqq DX-Pedition\grqq{}?}{Es ist eine Zusammenstellung aller noch von Funkamateuren begehrten Länder.}
{Es ist eine weltweite Aktivitätswoche.}
{Es ist ein internationaler Funkwettbewerb.}
{\textbf{\textcolor{DARCgreen}{Es ist eine Amateurfunkexpedition zu Ländern oder Inseln, die selten im Amateurfunk zu hören sind.}}}
\end{QQuestion}

}
\end{frame}

\begin{frame}
\frametitle{Gründe für eine DX-Pedition}
\begin{itemize}
  \item Es gibt Diplomprogramme, bei denen mit unterschiedlichen Ländern gearbeitet werden muss
  \item Dafür benötigt man nachweislich Funkverbindungen mit 100 unterschiedlichen Ländern
  \item Mit DX-Peditionen können fehlende Länder ergänzt werden
  \end{itemize}

\end{frame}%ENDCONTENT
