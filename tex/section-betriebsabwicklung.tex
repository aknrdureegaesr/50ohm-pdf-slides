
\section{Betriebsabwicklung}
\label{section:betriebsabwicklung}
\begin{frame}%STARTCONTENT

\frametitle{Ablauf im Amateurfunk}
\begin{itemize}
  \item Es gibt keine verpflichtenden Vorgaben außer Nennung des Rufzeichens
  \item Es macht aber Sinn, sich an der Betriebsabwicklung zu orientieren
  \end{itemize}

\end{frame}

\begin{frame}
\frametitle{Freie Frequenz finden}
\begin{itemize}
  \item Frequenzen werden gemeinsam genutzt
  \item Erst hören, ob die Frequenz frei ist
  \item Zwei- bis dreimal kurz nachfragen, ob die Frequenz frei ist
  \end{itemize}

\end{frame}

\begin{frame}
\frametitle{Anruf starten}
    \pause
    
\frametitle{Allgemeiner Anruf}
\begin{itemize}
  \item Geht an \emph{alle} Stationen
  \item Beginnt mit der internationalen Abkürzung \emph{CQ}
  \end{itemize}
    \pause
    
\frametitle{Gezielter Anruf}
\begin{itemize}
  \item Antwort von einer bestimmten Station erwartet
  \end{itemize}
    \pause
    In der Antwort wird erst das Rufzeichen der anrufenden Station, dann das eigene genannt

\end{frame}

\begin{frame}
\frametitle{Allgemeiner Anruf}
    \pause\QSOown{Ist diese Frequenz frei? DL1PZ}\pause\QSOother{\emph{(keine Antwort)}}\pause\QSOown{Ist diese Frequenz frei? DL1PZ}\pause\QSOother{\emph{(keine Antwort)}}\pause\QSOown{CQ CQ hier ist DL1PZ mit einem allgemeinen Anruf, hier ist DL1PZ und hört.}\pause\QSOother{DL1PZ hier ist DL9MJ bitte kommen}


\end{frame}

\begin{frame}
\frametitle{Gezielter Anruf}
    \pause\QSOown{DL9MJ für DL1PZ bitte kommen}\pause\QSOother{DL1PZ hier ist DL9MJ}


\end{frame}

\begin{frame}
\only<1>{
\begin{QQuestion}{BB102}{Was bedeutet die betriebliche Abkürzung \glqq CQ\grqq{} im Amateurfunk?}{Allgemeiner Anruf}
{Telegrafie}
{Große Entfernung}
{Contest Query}
\end{QQuestion}

}
\only<2>{
\begin{QQuestion}{BB102}{Was bedeutet die betriebliche Abkürzung \glqq CQ\grqq{} im Amateurfunk?}{\textbf{\textcolor{DARCgreen}{Allgemeiner Anruf}}}
{Telegrafie}
{Große Entfernung}
{Contest Query}
\end{QQuestion}

}
\end{frame}

\begin{frame}
\only<1>{
\begin{QQuestion}{BE105}{Sie möchten einen Allgemeinen Anruf in Telefonie im \qty{10}{\m}-Band beginnen. Sie finden eine Frequenz, auf der Sie keine Signale hören. Wie gehen Sie vor?}{Ich frage zwei- bis dreimal, ob die Frequenz besetzt ist. Erfolgt keine Antwort, rufe ich CQ.}
{Ich beobachte die Frequenz für einige Sekunden. Wenn ich weiterhin keine Signale höre, rufe ich CQ.}
{Da ich auf der Frequenz kein Signal höre, kann ich mit meinem CQ-Ruf beginnen.}
{Ich stimme meinen Sender auf der Frequenz ab und starte dann meinen CQ-Ruf.}
\end{QQuestion}

}
\only<2>{
\begin{QQuestion}{BE105}{Sie möchten einen Allgemeinen Anruf in Telefonie im \qty{10}{\m}-Band beginnen. Sie finden eine Frequenz, auf der Sie keine Signale hören. Wie gehen Sie vor?}{\textbf{\textcolor{DARCgreen}{Ich frage zwei- bis dreimal, ob die Frequenz besetzt ist. Erfolgt keine Antwort, rufe ich CQ.}}}
{Ich beobachte die Frequenz für einige Sekunden. Wenn ich weiterhin keine Signale höre, rufe ich CQ.}
{Da ich auf der Frequenz kein Signal höre, kann ich mit meinem CQ-Ruf beginnen.}
{Ich stimme meinen Sender auf der Frequenz ab und starte dann meinen CQ-Ruf.}
\end{QQuestion}

}
\end{frame}

\begin{frame}
\only<1>{
\begin{QQuestion}{BE101}{Wie können Sie eine Amateurfunkverbindung zum Beispiel beginnen?}{Durch Benutzen der internationalen Betriebsabkürzung CQ bzw. mit einem allgemeinen Anruf; mit einem gezielten Anruf an eine bestimmte Station oder mit einer Antwort auf einen allgemeinen Anruf, jeweils mit Nennung des eigenen Rufzeichens.}
{Durch wiederholtes Aussenden der internationalen Q-Gruppe \glqq QRZ?\grqq{} mit angehängtem eigenen Rufzeichen und dem Abhören der Frequenz in den Sendepausen.}
{Durch mehrmaliges, bei schlechten Ausbreitungsbedingungen häufiges Aussenden der Abkürzung CQ, des eigenen Rufzeichens und der Q-Gruppe \glqq QTH\grqq{} mit Zwischenhören.}
{Durch das Aussenden Ihres Rufzeichens und des in der IARU festgelegten Auftasttones von \qty{1750}{\Hz}, durch den die abhörenden Stationen Ihren Verbindungswunsch erkennen.}
\end{QQuestion}

}
\only<2>{
\begin{QQuestion}{BE101}{Wie können Sie eine Amateurfunkverbindung zum Beispiel beginnen?}{\textbf{\textcolor{DARCgreen}{Durch Benutzen der internationalen Betriebsabkürzung CQ bzw. mit einem allgemeinen Anruf; mit einem gezielten Anruf an eine bestimmte Station oder mit einer Antwort auf einen allgemeinen Anruf, jeweils mit Nennung des eigenen Rufzeichens.}}}
{Durch wiederholtes Aussenden der internationalen Q-Gruppe \glqq QRZ?\grqq{} mit angehängtem eigenen Rufzeichen und dem Abhören der Frequenz in den Sendepausen.}
{Durch mehrmaliges, bei schlechten Ausbreitungsbedingungen häufiges Aussenden der Abkürzung CQ, des eigenen Rufzeichens und der Q-Gruppe \glqq QTH\grqq{} mit Zwischenhören.}
{Durch das Aussenden Ihres Rufzeichens und des in der IARU festgelegten Auftasttones von \qty{1750}{\Hz}, durch den die abhörenden Stationen Ihren Verbindungswunsch erkennen.}
\end{QQuestion}

}
\end{frame}

\begin{frame}
\only<1>{
\begin{QQuestion}{BE102}{Wie sollten Sie antworten, wenn jemand in Telefonie CQ ruft?}{Ich rufe ebenfalls CQ und nenne das Rufzeichen der rufenden Station mindestens dreimal, anschließend sage ich mindestens fünfmal: \glqq Hier ist (eigenes Rufzeichen buchstabieren)\grqq{}.}
{Ich nenne das Rufzeichen der rufenden Station mindestens fünfmal, und anschließend sage ich mindestens einmal: \glqq Hier ist (eigenes Rufzeichen buchstabieren)\grqq{}.}
{Ich nenne das Rufzeichen der rufenden Station einmal, anschließend sage ich einmal: \glqq Hier ist (eigenes Rufzeichen buchstabieren), bitte kommen\grqq{}.}
{Ich nenne mein Rufzeichen und fordere die rufende Station auf, auf einer anderen Frequenz weiter zu rufen (mindestens zweimal).}
\end{QQuestion}

}
\only<2>{
\begin{QQuestion}{BE102}{Wie sollten Sie antworten, wenn jemand in Telefonie CQ ruft?}{Ich rufe ebenfalls CQ und nenne das Rufzeichen der rufenden Station mindestens dreimal, anschließend sage ich mindestens fünfmal: \glqq Hier ist (eigenes Rufzeichen buchstabieren)\grqq{}.}
{Ich nenne das Rufzeichen der rufenden Station mindestens fünfmal, und anschließend sage ich mindestens einmal: \glqq Hier ist (eigenes Rufzeichen buchstabieren)\grqq{}.}
{\textbf{\textcolor{DARCgreen}{Ich nenne das Rufzeichen der rufenden Station einmal, anschließend sage ich einmal: \glqq Hier ist (eigenes Rufzeichen buchstabieren), bitte kommen\grqq{}.}}}
{Ich nenne mein Rufzeichen und fordere die rufende Station auf, auf einer anderen Frequenz weiter zu rufen (mindestens zweimal).}
\end{QQuestion}

}
\end{frame}

\begin{frame}
\frametitle{Unklare Verständigung}
    \pause\QSOown{D\emph{(krschkrsch)}MJ für DK5WP, bitte kommen}\pause\QSOother{Hier ist DL9MJ, wurde ich gerufen?}
    \pause
    Nachfragen, ob man gemeint war



\end{frame}

\begin{frame}
\only<1>{
\begin{QQuestion}{BE103}{Ihr Rufzeichen ist DH7RW. Sie hören unvollständig \glqq ...~7 Romeo Whiskey\grqq{}. Wie reagieren Sie?}{Ich antworte: \glqq Bitte QSY!\grqq{}.}
{Ich antworte: \glqq Bitte QSL!\grqq{}    }
{Ich antworte: \glqq Hier ist DH7RW, wurde ich gerufen?\grqq{}}
{Ich freue mich auf eine Antwort aus 7R - Algerien.}
\end{QQuestion}

}
\only<2>{
\begin{QQuestion}{BE103}{Ihr Rufzeichen ist DH7RW. Sie hören unvollständig \glqq ...~7 Romeo Whiskey\grqq{}. Wie reagieren Sie?}{Ich antworte: \glqq Bitte QSY!\grqq{}.}
{Ich antworte: \glqq Bitte QSL!\grqq{}    }
{\textbf{\textcolor{DARCgreen}{Ich antworte: \glqq Hier ist DH7RW, wurde ich gerufen?\grqq{}}}}
{Ich freue mich auf eine Antwort aus 7R - Algerien.}
\end{QQuestion}

}
\end{frame}

\begin{frame}
\frametitle{Anruf beenden}
\begin{itemize}
  \item Die Frequenz wird der anrufenden Station überlassen
  \item Falls die antwortende Station zwischendurch von einer weiteren Station gerufen wurde, soll sie sich mit dieser auf eine andere Frequenz einigen, um der bisherigen Station die Frequenz zurückzugeben
  \end{itemize}
\end{frame}

\begin{frame}
\only<1>{
\begin{QQuestion}{BE108}{Sie haben eine Funkverbindung mit einer vorher \glqq CQ\grqq{} rufenden Station beendet. Anschließend werden Sie von einer anderen Station gerufen. Wie verhalten Sie sich?}{Ich bleibe auf der Frequenz und tätige ein QSO mit der neu rufenden Station.}
{Ich verständige mich mit der neuen Gegenstation auf eine andere Frequenz und führe dort das QSO weiter.}
{Ich gehe etwa \qty{1}{\kHz} neben die bisherige Frequenz und rufe dort die anrufende Station.}
{Ich reagiere nicht auf den Anruf, weil die Frequenz der Station gehört, die CQ gerufen hat.}
\end{QQuestion}

}
\only<2>{
\begin{QQuestion}{BE108}{Sie haben eine Funkverbindung mit einer vorher \glqq CQ\grqq{} rufenden Station beendet. Anschließend werden Sie von einer anderen Station gerufen. Wie verhalten Sie sich?}{Ich bleibe auf der Frequenz und tätige ein QSO mit der neu rufenden Station.}
{\textbf{\textcolor{DARCgreen}{Ich verständige mich mit der neuen Gegenstation auf eine andere Frequenz und führe dort das QSO weiter.}}}
{Ich gehe etwa \qty{1}{\kHz} neben die bisherige Frequenz und rufe dort die anrufende Station.}
{Ich reagiere nicht auf den Anruf, weil die Frequenz der Station gehört, die CQ gerufen hat.}
\end{QQuestion}

}
\end{frame}%ENDCONTENT
