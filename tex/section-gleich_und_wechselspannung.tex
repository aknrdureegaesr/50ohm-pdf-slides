
\section{Gleich- und  Wechselspannung}
\label{section:gleich_und_wechselspannung}
\begin{frame}%STARTCONTENT

\frametitle{Einführung in die elektrische Spannung}
\begin{columns}
    \begin{column}{0.48\textwidth}
    
\begin{figure}
    \DARCimage{0.85\linewidth}{713include}
    \caption{\scriptsize Positiv und negativ geladene Teilchen gleichverteilt in einem Gegenstand.}
    \label{n_frequenz_elektrische_ladungen}
\end{figure}


    \end{column}
   \begin{column}{0.48\textwidth}
       \begin{itemize}
  \item Alle Stoffe bestehen aus winzigen Teilchen, die elektrisch geladen sind
  \item Manche sind \enquote{positiv} (Plus) geladen
  \item Manche sind \enquote{negativ} (Minus) geladen
  \end{itemize}

   \end{column}
\end{columns}

\end{frame}

\begin{frame}
\begin{columns}
    \begin{column}{0.48\textwidth}
    
\begin{figure}
    \DARCimage{0.85\linewidth}{710include}
    \caption{\scriptsize Anziehung und Abstoßung von Ladungen}
    \label{n_ladungen}
\end{figure}


    \end{column}
   \begin{column}{0.48\textwidth}
       \begin{itemize}
  \item Gleich geladene Teilchen stoßen sich ab
  \item Unterschiedliche Ladungen ziehen sich an
  \item Die meisten Gegenstände sind elektrisch ausgeglichen
  \end{itemize}

   \end{column}
\end{columns}

\end{frame}

\begin{frame}
\frametitle{Ladungstrennung}
\begin{itemize}
  \item Ladungen lassen sich gezielt trennen
  \item In einer Batterie, Solarzelle oder einem Windkraftwerk
  \item Ladungen versuchen wieder zusammen zu kommen
  \item Es liegt eine elektrische \emph{Spannung} vor
  \item Geräte zur Trennung von Ladungen heißen \emph{Spannungsquelle}
  \end{itemize}
\end{frame}

\begin{frame}
\frametitle{Spannungsquelle}
\begin{itemize}
  \item Der positiv geladene Anschluss heißt Pluspol
  \item Der negativ geladene Anchluss heißt Minuspol
  \item Die Spannung kann unterschiedlich groß sein
  \item Spannungsquellen, bei denen die Pole ständig zwischen positiver und negativer Spannung schwingen, erzeugen Wechselspannung
  \end{itemize}
    \pause
    Die elektrische Spannung wird in der Einheit $\text{Volt}$ mit der Abkürzung $V$ gemessen.



\end{frame}

\begin{frame}
\frametitle{Elektrischer Verbraucher}
\begin{columns}
    \begin{column}{0.48\textwidth}
    
\begin{figure}
    \DARCimage{0.85\linewidth}{714include}
    \caption{\scriptsize Die Pole einer Batterie, am Minus-Pol befindet sich ein Überschuss an negativen Ladungen und am Plus-Pol ein Überschuss an positiven Ladungen, die Pole der Batterie sind verbunden, daher kann der Strom durch den Verbraucher fließen.}
    \label{n_frequenz_strom_fliesst}
\end{figure}


    \end{column}
   \begin{column}{0.48\textwidth}
       \begin{itemize}
  \item Wird ein elektrischer Verbraucher zwischen beiden Polen angeschlossen, bewegen sich die Ladungen
  \item Es fließt ein elektrischer Strom
  \item Die Ladungsbewegung endet bei einem Ausgleich der Ladungsträger an den Polen
  \end{itemize}

   \end{column}
\end{columns}

\end{frame}%ENDCONTENT
