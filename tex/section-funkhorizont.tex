
\section{Funkhorizont}
\label{section:funkhorizont}
\begin{frame}%STARTCONTENT

\begin{columns}
    \begin{column}{0.48\textwidth}
    
\begin{figure}
    \DARCimage{0.85\linewidth}{484include}
    \caption{\scriptsize Ausbreitung}
    \label{n_funkhorizont}
\end{figure}


    \end{column}
   \begin{column}{0.48\textwidth}
       \begin{itemize}
  \item Funkwellen im VHF- und UHF-Bereich verhalten sich ähnlich wie Licht
  \item Licht reicht maximal bis zum geografischen (sichtbaren) Horizont
  \item Funkwellen schaffen ca. \qty{15}{\percent} mehr Reichweite
  \item Folgen ein wenig der Erdkrümmung
  \end{itemize}

   \end{column}
\end{columns}

\end{frame}

\begin{frame}\begin{itemize}
  \item Sichtverbindung für zuverlässige Funkverbindungen auf VHF, UHF und darüber
  \item Hohe Gebäude oder Berge stören
  \item Je höher die Antenne, umso größer die Reichweite
  \item Weite Verbindungen von Bergen statt aus dem Tal
  \end{itemize}
\end{frame}

\begin{frame}
\only<1>{
\begin{QQuestion}{NH301}{Wie weit etwa reicht der Funkhorizont im UKW-Bereich über den geografischen Horizont hinaus? Er reicht etwa~...}{doppelt so weit.}
{\qty{15}{\percent} weiter.}
{halb so weit.}
{bis zu viermal so weit.}
\end{QQuestion}

}
\only<2>{
\begin{QQuestion}{NH301}{Wie weit etwa reicht der Funkhorizont im UKW-Bereich über den geografischen Horizont hinaus? Er reicht etwa~...}{doppelt so weit.}
{\textbf{\textcolor{DARCgreen}{\qty{15}{\percent} weiter.}}}
{halb so weit.}
{bis zu viermal so weit.}
\end{QQuestion}

}
\end{frame}

\begin{frame}
\only<1>{
\begin{PQuestion}{NH303}{In dem folgenden Geländeprofil sei S ein Sender im \qty{2}{\m}-Band. Welche der Empfangsstationen E1 bis E4 wird das Signal des Senders wahrscheinlich am besten empfangen?}{$\text{E}_3$}
{$\text{E}_1$}
{$\text{E}_2$}
{$\text{E}_4$}
{\DARCimage{1.0\linewidth}{483include}}\end{PQuestion}

}
\only<2>{
\begin{PQuestion}{NH303}{In dem folgenden Geländeprofil sei S ein Sender im \qty{2}{\m}-Band. Welche der Empfangsstationen E1 bis E4 wird das Signal des Senders wahrscheinlich am besten empfangen?}{\textbf{\textcolor{DARCgreen}{$\text{E}_3$}}}
{$\text{E}_1$}
{$\text{E}_2$}
{$\text{E}_4$}
{\DARCimage{1.0\linewidth}{483include}}\end{PQuestion}

}
\end{frame}

\begin{frame}
\only<1>{
\begin{QQuestion}{NH302}{Wie wirkt sich die Antennenhöhe auf die Reichweite einer UKW-Verbindung aus? Die Reichweite steigt mit zunehmender Antennenhöhe, weil~...}{die quasi-optische Sichtweite zunimmt.}
{sie näher an der Ionosphäre ist.}
{dadurch steiler abgestrahlt werden kann.}
{in höheren Luftschichten die Temperatur sinkt.}
\end{QQuestion}

}
\only<2>{
\begin{QQuestion}{NH302}{Wie wirkt sich die Antennenhöhe auf die Reichweite einer UKW-Verbindung aus? Die Reichweite steigt mit zunehmender Antennenhöhe, weil~...}{\textbf{\textcolor{DARCgreen}{die quasi-optische Sichtweite zunimmt.}}}
{sie näher an der Ionosphäre ist.}
{dadurch steiler abgestrahlt werden kann.}
{in höheren Luftschichten die Temperatur sinkt.}
\end{QQuestion}

}
\end{frame}%ENDCONTENT
