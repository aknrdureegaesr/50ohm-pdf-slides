
\section{MUF und LUF II}
\label{section:muf_luf_2}
\begin{frame}%STARTCONTENT

\frametitle{Höchste brauchbare Frequenz (MUF)}
\begin{columns}
    \begin{column}{0.48\textwidth}
    \begin{itemize}
  \item Höchste Frequenz mit der eine Verbindung über die Raumwelle hergestellt werden kann
  \item Abhängig vom Abstrahlwinkel
  \item $MUF \approx \frac{f_c}{\sin(\alpha)}$
  \end{itemize}

    \end{column}
   \begin{column}{0.48\textwidth}
       
   \end{column}
\end{columns}

\end{frame}

\begin{frame}
\only<1>{
\begin{QQuestion}{AH206}{Die höchste Frequenz, bei der eine Kommunikation zwischen zwei Funkstellen über Raumwelle möglich ist, wird als~...}{höchste durchlässige Frequenz bezeichnet (LUF).}
{optimale Arbeitsfrequenz bezeichnet (f$_{opt}$, FOT).
}
{kritische Frequenz bezeichnet (f$_{krit}$, foF2).}
{höchste nutzbare Frequenz bezeichnet (MUF).}
\end{QQuestion}

}
\only<2>{
\begin{QQuestion}{AH206}{Die höchste Frequenz, bei der eine Kommunikation zwischen zwei Funkstellen über Raumwelle möglich ist, wird als~...}{höchste durchlässige Frequenz bezeichnet (LUF).}
{optimale Arbeitsfrequenz bezeichnet (f$_{opt}$, FOT).
}
{kritische Frequenz bezeichnet (f$_{krit}$, foF2).}
{\textbf{\textcolor{DARCgreen}{höchste nutzbare Frequenz bezeichnet (MUF).}}}
\end{QQuestion}

}
\end{frame}

\begin{frame}
\only<1>{
\begin{QQuestion}{AH207}{Wenn sich elektromagnetische Wellen zwischen zwei Orten durch ionosphärische Brechung ausbreiten, dann ist die MUF~...}{die vorgeschriebene nutzbare Frequenz.}
{der Mittelwert aus der höchsten und niedrigsten brauchbaren Frequenz.}
{die niedrigste brauchbare Frequenz.}
{die höchste brauchbare Frequenz.}
\end{QQuestion}

}
\only<2>{
\begin{QQuestion}{AH207}{Wenn sich elektromagnetische Wellen zwischen zwei Orten durch ionosphärische Brechung ausbreiten, dann ist die MUF~...}{die vorgeschriebene nutzbare Frequenz.}
{der Mittelwert aus der höchsten und niedrigsten brauchbaren Frequenz.}
{die niedrigste brauchbare Frequenz.}
{\textbf{\textcolor{DARCgreen}{die höchste brauchbare Frequenz.}}}
\end{QQuestion}

}
\end{frame}

\begin{frame}
\frametitle{Kritische Frequenz}
\begin{itemize}
  \item Bei \qty{90}{\degree} Abstrahlwinkel muss das Signal in der Ionosphäre eine \qty{180}{\degree}-Wendung vollziehen
  \item Kritische Frequenz f<sub>c</sub> bei der das Signal reflektiert wird
  \item MUF liefgt höher als f<sub>c</sub>, da in der Regel nicht senkrecht nach oben gesendet wird
  \end{itemize}

\end{frame}

\begin{frame}
\only<1>{
\begin{QQuestion}{AH208}{Die höchste brauchbare Frequenz (MUF) für eine Funkstrecke~...}{liegt tiefer als die kritische Frequenz, und zwar um so mehr, je steiler die Sendeantenne abstrahlt bzw. die Empfangsantenne aufnimmt.}
{liegt tiefer als die kritische Frequenz, und zwar um so mehr, je flacher die Sendeantenne abstrahlt bzw. die Empfangsantenne aufnimmt.}
{liegt höher als die kritische Frequenz, und zwar um so mehr, je flacher die Sendeantenne abstrahlt bzw. die Empfangsantenne aufnimmt.}
{ist nicht davon abhängig, wie flach die Sendeantenne abstrahlt bzw. die Empfangsantenne aufnimmt, sondern nur vom Zustand der Ionosphäre.}
\end{QQuestion}

}
\only<2>{
\begin{QQuestion}{AH208}{Die höchste brauchbare Frequenz (MUF) für eine Funkstrecke~...}{liegt tiefer als die kritische Frequenz, und zwar um so mehr, je steiler die Sendeantenne abstrahlt bzw. die Empfangsantenne aufnimmt.}
{liegt tiefer als die kritische Frequenz, und zwar um so mehr, je flacher die Sendeantenne abstrahlt bzw. die Empfangsantenne aufnimmt.}
{\textbf{\textcolor{DARCgreen}{liegt höher als die kritische Frequenz, und zwar um so mehr, je flacher die Sendeantenne abstrahlt bzw. die Empfangsantenne aufnimmt.}}}
{ist nicht davon abhängig, wie flach die Sendeantenne abstrahlt bzw. die Empfangsantenne aufnimmt, sondern nur vom Zustand der Ionosphäre.}
\end{QQuestion}

}
\end{frame}

\begin{frame}
\frametitle{Optimale Frequenz}
\begin{itemize}
  \item Kommerzielle Frequenzplanung verwendet eine \emph{Frequency of optimal transmition}, optimale Sendefrequenz
  \item Frequenz, die auf einem bestimmten Signalweg statistisch an \qty{90}{\percent} aller Tage eine Funkverbindung erlaubt
  \item Liegt \qty{15}{\percent} unter dem monatlichen Mittel der MUF
  \item $f_{\textrm{opt}} = \textrm{MUF}\cdot 0,85$
  \item Spielt für Amateurfunk keine große Rolle, da keine dauerhafte Verbindung aufgebaut wird
  \item Im Amateurfunk wird bis nahe an der MUF gearbeitet
  \end{itemize}
\end{frame}

\begin{frame}
\only<1>{
\begin{QQuestion}{AH209}{Wie groß ist die höchste nutzbare Frequenz (MUF) und die optimale Frequenz $f_{\symup{opt}}$, wenn die Antenne in einem Winkel von $45^\circ$ schräg nach oben strahlt und die kritische Frequenz $f_{k}$ \qty{3}{\MHz} beträgt?}{Die MUF liegt bei \qty{2,1}{\MHz} und $f_{\symup{opt}}$ bei \qty{2,5}{\MHz}.}
{Die MUF liegt bei \qty{2,1}{\MHz} und $f_{\symup{opt}}$ bei \qty{1,8}{\MHz}.}
{Die MUF liegt bei \qty{4,2}{\MHz} und $f_{\symup{opt}}$ bei \qty{3,6}{\MHz}.}
{Die MUF liegt bei \qty{4,2}{\MHz} und $f_{\symup{opt}}$ bei \qty{4,9}{\MHz}.}
\end{QQuestion}

}
\only<2>{
\begin{QQuestion}{AH209}{Wie groß ist die höchste nutzbare Frequenz (MUF) und die optimale Frequenz $f_{\symup{opt}}$, wenn die Antenne in einem Winkel von $45^\circ$ schräg nach oben strahlt und die kritische Frequenz $f_{k}$ \qty{3}{\MHz} beträgt?}{Die MUF liegt bei \qty{2,1}{\MHz} und $f_{\symup{opt}}$ bei \qty{2,5}{\MHz}.}
{Die MUF liegt bei \qty{2,1}{\MHz} und $f_{\symup{opt}}$ bei \qty{1,8}{\MHz}.}
{\textbf{\textcolor{DARCgreen}{Die MUF liegt bei \qty{4,2}{\MHz} und $f_{\symup{opt}}$ bei \qty{3,6}{\MHz}.}}}
{Die MUF liegt bei \qty{4,2}{\MHz} und $f_{\symup{opt}}$ bei \qty{4,9}{\MHz}.}
\end{QQuestion}

}
\end{frame}

\begin{frame}
\frametitle{Lösungsweg}
\begin{columns}
    \begin{column}{0.48\textwidth}
    \begin{itemize}
  \item gegeben: $\alpha = 45\degree$
  \item gegeben: $f_c = 3MHz$
  \end{itemize}

    \end{column}
   \begin{column}{0.48\textwidth}
       \begin{itemize}
  \item gesucht: MUF
  \item gesucht: $f_{\textrm{opt}}$
  \end{itemize}

   \end{column}
\end{columns}
\begin{columns}
    \begin{column}{0.48\textwidth}
    
    \pause
    \begin{equation}\begin{split} \nonumber MUF &\approx \frac{f_c}{\sin(\alpha)}\\ &\approx \frac{3MHz}{0,71}\\ &\approx 4,2MHz \end{split}\end{equation}




    \end{column}
   \begin{column}{0.48\textwidth}
       
    \pause
    \begin{equation}\begin{split} \nonumber f_{\textrm{opt}} &= \textrm{MUF}\cdot 0,85\\ &= 4,2MHz \cdot 0,85\\ &= 3,6MHz \end{split}\end{equation}




   \end{column}
\end{columns}

\end{frame}

\begin{frame}
\frametitle{Niedrigste brauchbare Frequenz (LUF)}
Niedrigste Frequenz mit der eine Verbindung über die Raumwelle hergestellt werden kann

\end{frame}

\begin{frame}
\only<1>{
\begin{QQuestion}{AH210}{Die LUF für eine Funkstrecke ist~...}{die niedrigste brauchbare Frequenz, bei der eine Verbindung über die Raumwelle hergestellt werden kann.}
{der Mittelwert der höchsten und niedrigsten brauchbaren Frequenz, bei der eine Verbindung über die Raumwelle hergestellt werden kann.}
{die gemessene brauchbare Frequenz, bei der eine Verbindung über die Raumwelle hergestellt werden kann.}
{die brauchbarste Frequenz, bei der eine Verbindung über die Raumwelle hergestellt werden kann.}
\end{QQuestion}

}
\only<2>{
\begin{QQuestion}{AH210}{Die LUF für eine Funkstrecke ist~...}{\textbf{\textcolor{DARCgreen}{die niedrigste brauchbare Frequenz, bei der eine Verbindung über die Raumwelle hergestellt werden kann.}}}
{der Mittelwert der höchsten und niedrigsten brauchbaren Frequenz, bei der eine Verbindung über die Raumwelle hergestellt werden kann.}
{die gemessene brauchbare Frequenz, bei der eine Verbindung über die Raumwelle hergestellt werden kann.}
{die brauchbarste Frequenz, bei der eine Verbindung über die Raumwelle hergestellt werden kann.}
\end{QQuestion}

}
\end{frame}

\begin{frame}
\only<1>{
\begin{QQuestion}{AH211}{Was bedeutet die Aussage: \glqq Die LUF für eine Funkstrecke liegt bei \qty{6}{\MHz}\grqq{}?}{Die mittlere Frequenz, die für Verbindungen über die Raumwelle genutzt werden kann, liegt bei \qty{6}{\MHz}.}
{Die höchste Frequenz, die für Verbindungen über die Raumwelle als noch brauchbar angesehen wird, liegt bei \qty{6}{\MHz}.}
{Die niedrigste Frequenz, die für Verbindungen über die Raumwelle als noch brauchbar angesehen wird, liegt bei \qty{6}{\MHz}.}
{Die optimale Frequenz, die für Verbindungen über die Raumwelle genutzt werden kann, liegt bei \qty{6}{\MHz}.}
\end{QQuestion}

}
\only<2>{
\begin{QQuestion}{AH211}{Was bedeutet die Aussage: \glqq Die LUF für eine Funkstrecke liegt bei \qty{6}{\MHz}\grqq{}?}{Die mittlere Frequenz, die für Verbindungen über die Raumwelle genutzt werden kann, liegt bei \qty{6}{\MHz}.}
{Die höchste Frequenz, die für Verbindungen über die Raumwelle als noch brauchbar angesehen wird, liegt bei \qty{6}{\MHz}.}
{\textbf{\textcolor{DARCgreen}{Die niedrigste Frequenz, die für Verbindungen über die Raumwelle als noch brauchbar angesehen wird, liegt bei \qty{6}{\MHz}.}}}
{Die optimale Frequenz, die für Verbindungen über die Raumwelle genutzt werden kann, liegt bei \qty{6}{\MHz}.}
\end{QQuestion}

}
\end{frame}%ENDCONTENT
