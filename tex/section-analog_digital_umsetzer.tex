
\section{Analog-Digital-Umsetzer (ADC)}
\label{section:analog_digital_umsetzer}
\begin{frame}%STARTCONTENT

\only<1>{
\begin{PQuestion}{AF620}{Welche Funktionen haben die einzelnen Blöcke im dargestellten Blockschaltbild eines digitalen Direktempfängers?}{1:~Abtastratengenerator, 2:~Antialiasing-Filter, 3:~Analog-Digital-Umsetzer}
{1:~Analog-Digital-Umsetzer, 2:~Antialiasing-Filter, 3:~Abtastratengenerator}
{1:~Analog-Digital-Umsetzer, 2:~Abtastratengenerator, 3:~Antialiasing-Filter}
{1:~Antialiasing-Filter, 2:~Abtastratengenerator, 3:~Analog-Digital-Umsetzer}
{\DARCimage{1.0\linewidth}{430include}}\end{PQuestion}

}
\only<2>{
\begin{PQuestion}{AF620}{Welche Funktionen haben die einzelnen Blöcke im dargestellten Blockschaltbild eines digitalen Direktempfängers?}{1:~Abtastratengenerator, 2:~Antialiasing-Filter, 3:~Analog-Digital-Umsetzer}
{1:~Analog-Digital-Umsetzer, 2:~Antialiasing-Filter, 3:~Abtastratengenerator}
{1:~Analog-Digital-Umsetzer, 2:~Abtastratengenerator, 3:~Antialiasing-Filter}
{\textbf{\textcolor{DARCgreen}{1:~Antialiasing-Filter, 2:~Abtastratengenerator, 3:~Analog-Digital-Umsetzer}}}
{\DARCimage{1.0\linewidth}{430include}}\end{PQuestion}

}
\end{frame}

\begin{frame}
\only<1>{
\begin{QQuestion}{AF607}{Warum kommt es in einem A/D-Umsetzer zu Quantisierungsfehlern?}{Die Bandbreite des Eingangssignals ist begrenzt.}
{Es steht nur eine begrenzte Anzahl diskreter Werte zur Verfügung.}
{Es können nur ganzzahlige Frequenzen verwendet werden.}
{Es können nur Werte zwischen 0 und 1 genutzt werden.}
\end{QQuestion}

}
\only<2>{
\begin{QQuestion}{AF607}{Warum kommt es in einem A/D-Umsetzer zu Quantisierungsfehlern?}{Die Bandbreite des Eingangssignals ist begrenzt.}
{\textbf{\textcolor{DARCgreen}{Es steht nur eine begrenzte Anzahl diskreter Werte zur Verfügung.}}}
{Es können nur ganzzahlige Frequenzen verwendet werden.}
{Es können nur Werte zwischen 0 und 1 genutzt werden.}
\end{QQuestion}

}
\end{frame}

\begin{frame}
\only<1>{
\begin{QQuestion}{AF608}{Wie viele Bereiche von Eingangswerten, z.\,B. Spannungen, kann ein A/D-Umsetzer mit \qty{8}{\bit} Auflösung maximal trennen?}{256}
{8}
{64}
{1024}
\end{QQuestion}

}
\only<2>{
\begin{QQuestion}{AF608}{Wie viele Bereiche von Eingangswerten, z.\,B. Spannungen, kann ein A/D-Umsetzer mit \qty{8}{\bit} Auflösung maximal trennen?}{\textbf{\textcolor{DARCgreen}{256}}}
{8}
{64}
{1024}
\end{QQuestion}

}
\end{frame}

\begin{frame}
\only<1>{
\begin{QQuestion}{AF621}{Bei einer Abtastung mit einem A/D-Umsetzer mit \qty{24}{\bit} Auflösung wird ein Oszillator mit starkem Taktzittern (Jitter) eingesetzt. Welche Auswirkung wird das Zittern haben?}{Aufgrund der großen Auflösung bleibt die Schwankung ohne Auswirkung.}
{Das Abschirmblech des A/D-Umsetzers wird durch Vibration störende Geräusche erzeugen.}
{Es entsteht zusätzliches Rauschen im Abtastergebnis.}
{Das Abtastergebnis wird verbessert (Dithering).}
\end{QQuestion}

}
\only<2>{
\begin{QQuestion}{AF621}{Bei einer Abtastung mit einem A/D-Umsetzer mit \qty{24}{\bit} Auflösung wird ein Oszillator mit starkem Taktzittern (Jitter) eingesetzt. Welche Auswirkung wird das Zittern haben?}{Aufgrund der großen Auflösung bleibt die Schwankung ohne Auswirkung.}
{Das Abschirmblech des A/D-Umsetzers wird durch Vibration störende Geräusche erzeugen.}
{\textbf{\textcolor{DARCgreen}{Es entsteht zusätzliches Rauschen im Abtastergebnis.}}}
{Das Abtastergebnis wird verbessert (Dithering).}
\end{QQuestion}

}
\end{frame}%ENDCONTENT
