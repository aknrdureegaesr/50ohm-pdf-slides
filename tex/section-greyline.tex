
\section{Greyline}
\label{section:greyline}
\begin{frame}%STARTCONTENT
\begin{itemize}
  \item Übergang zwischen Tag- und Nacht
  \item Für den Kurzwellenfunk interessant
  \end{itemize}
\end{frame}

\begin{frame}

\end{frame}

\begin{frame}
\begin{columns}
    \begin{column}{0.48\textwidth}
    Tag zu Nacht

\begin{itemize}
  \item D-Region wird abgebaut
  \item E-Region kann noch vorhanden sein
  \item F<sub>1</sub>-Region baut langsam ab
  \item F<sub>2</sub>-Region bleibt geschwächt bestehen
  \end{itemize}

    \end{column}
   \begin{column}{0.48\textwidth}
       Nacht zu Tag

\begin{itemize}
  \item D-Region baut erst auf, wenn Sonne in unteren Regionen angekommen
  \item E-Region baut sich langsam auf
  \item F<sub>1</sub>-Region vor E- und D-Region aufgebaut
  \item F<sub>2</sub>-Region wird wieder stärker
  \end{itemize}

   \end{column}
\end{columns}

\end{frame}

\begin{frame}
\frametitle{Greyline-DX}
\begin{itemize}
  \item Kurzwellen werden an der schwachen D-Region flach gebrochen und weniger gedämpft
  \item Die gebrochenen Kurzwellen werden in der F-Region flach reflektiert
  \item Hohe Skip-Distanz
  \item \emph{Greyline-DX} oder \emph{Twilight-DX}
  \end{itemize}
\end{frame}

\begin{frame}
\only<1>{
\begin{QQuestion}{EH213}{Bei der Ausbreitung auf Kurzwelle spielt die so genannte \glqq Greyline\grqq{} eine besondere Rolle. Was ist die \glqq Greyline\grqq{}?}{Die Zeit mit den besten Möglichkeiten für \glqq Short-Skip\grqq{}-Ausbreitung.}
{Die instabilen Ausbreitungsbedingungen in der Äquatorialzone.}
{Die Zone der Dämmerung um Sonnenauf- und -untergang herum.}
{Die Übergangszeit vor und nach dem Winter, in der sich die D-Region ab- und wieder aufbaut.}
\end{QQuestion}

}
\only<2>{
\begin{QQuestion}{EH213}{Bei der Ausbreitung auf Kurzwelle spielt die so genannte \glqq Greyline\grqq{} eine besondere Rolle. Was ist die \glqq Greyline\grqq{}?}{Die Zeit mit den besten Möglichkeiten für \glqq Short-Skip\grqq{}-Ausbreitung.}
{Die instabilen Ausbreitungsbedingungen in der Äquatorialzone.}
{\textbf{\textcolor{DARCgreen}{Die Zone der Dämmerung um Sonnenauf- und -untergang herum.}}}
{Die Übergangszeit vor und nach dem Winter, in der sich die D-Region ab- und wieder aufbaut.}
\end{QQuestion}

}
\end{frame}%ENDCONTENT
