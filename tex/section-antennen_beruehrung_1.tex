
\section{Berühren von Antennen I}
\label{section:antennen_beruehrung_1}
\begin{frame}%STARTCONTENT
\emph{Eine Sendeantenne in Betrieb berührt man nicht!}

\begin{itemize}
  \item Hohe Wechselspannungen
  \item Verursachen Herzrhythmusstörungen, Verbrennungen und andere Verletzungen
  \item Kann zum Tod führen
  \item Auch zu Sekundärunfall wie Sturz von der Leiter durch Erschrecken und Verkrampfen
  \end{itemize}
\end{frame}

\begin{frame}
\only<1>{
\begin{QQuestion}{EK202}{Welche möglichen Gefahren bestehen beim Berühren von im Sendebetrieb befindlichen Antennen?}{Keine, da durch den \glqq Skin-Effekt\grqq{} ein Stromfluss durch den menschlichen Körper verhindert wird.}
{Verletzungen und Verbrennungen durch hochfrequente Spannungen.}
{Keine, sofern die Antenne ordnungsgemäß über ein Blitzschutzsystem mit Erde verbunden ist.}
{Stromschlag durch die Gleichspannungsversorgung der Sender-Endstufe, die direkt am Antennenausgang anliegt.}
\end{QQuestion}

}
\only<2>{
\begin{QQuestion}{EK202}{Welche möglichen Gefahren bestehen beim Berühren von im Sendebetrieb befindlichen Antennen?}{Keine, da durch den \glqq Skin-Effekt\grqq{} ein Stromfluss durch den menschlichen Körper verhindert wird.}
{\textbf{\textcolor{DARCgreen}{Verletzungen und Verbrennungen durch hochfrequente Spannungen.}}}
{Keine, sofern die Antenne ordnungsgemäß über ein Blitzschutzsystem mit Erde verbunden ist.}
{Stromschlag durch die Gleichspannungsversorgung der Sender-Endstufe, die direkt am Antennenausgang anliegt.}
\end{QQuestion}

}
\end{frame}%ENDCONTENT
