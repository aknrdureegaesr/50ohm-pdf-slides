
\section{Formeln umstellen I}
\label{section:formeln_umstellen}
\begin{frame}%STARTCONTENT
Wir hatten bereits

$ U = R\cdot I $

Doch wie kommt man zu

$ R = \dfrac{U}{I} $

und

$ I = \dfrac{U}{R} $

?

\end{frame}

\begin{frame}
\frametitle{Mathematischer Ansatz}
$ U = R\cdot I $ soll nach $ I $ umgestellt werden.

Division auf beiden Seiten durch die Größe, die man auf der Seite mit dem Ziel \enquote{weg} haben möchte.
    \pause
    Division durch  $ R $: $\enspace \dfrac{U}{R} = \dfrac{\cancel{R}\cdot I}{\cancel{R}} \xRightarrow{kürzen} \dfrac{U}{R} = I $
    \pause
    Die Seiten dürfen getauscht werden:

$\dfrac{U}{R} = I \rArr I = \dfrac{U}{R} $



\end{frame}

\begin{frame}
\frametitle{Formeln kombinieren}
Wir kennen bereits

$ U = R\cdot I $ und $ P = U\cdot I $
    \pause
    Wenn jedoch $U$ nicht bekannt ist, dafür aber $R$ und $I$, reicht dieses zur Berechnung von $P$:

$ P = U\cdot I \xRightarrow{U einsetzen} P = R\cdot I\cdot I $

$ \rArr P = R\cdot I^2 $



\end{frame}%ENDCONTENT
