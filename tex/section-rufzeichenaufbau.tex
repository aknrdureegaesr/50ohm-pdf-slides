
\section{Aufbau von Rufzeichen}
\label{section:rufzeichenaufbau}
\begin{frame}%STARTCONTENT

\frametitle{Personengebundene Rufzeichen}
\begin{itemize}
  \item Jeder Funkamateur mit Zulassung zum Amateurfunkdienst in Deutschland
  \item Rufzeichen wird durch die BNetzA zugeteilt
  \item Weltweit eindeutig
  \item Darf nur durch den zugeteilten Funkamateur benutzt werden
  \end{itemize}
\end{frame}

\begin{frame}
\only<1>{
\begin{QQuestion}{VC116}{Welche personengebundenen Rufzeichen darf ein Funkamateur benutzen?}{Bei Nutzung einer fremden Station das personengebundene Rufzeichen des Stationsinhabers}
{Beliebige Rufzeichen}
{Nur ein von einer Amateurfunkvereinigung zugeteiltes Rufzeichen}
{Nur ein ihm von der Bundesnetzagentur zugeteiltes Rufzeichen}
\end{QQuestion}

}
\only<2>{
\begin{QQuestion}{VC116}{Welche personengebundenen Rufzeichen darf ein Funkamateur benutzen?}{Bei Nutzung einer fremden Station das personengebundene Rufzeichen des Stationsinhabers}
{Beliebige Rufzeichen}
{Nur ein von einer Amateurfunkvereinigung zugeteiltes Rufzeichen}
{\textbf{\textcolor{DARCgreen}{Nur ein ihm von der Bundesnetzagentur zugeteiltes Rufzeichen}}}
\end{QQuestion}

}
\end{frame}

\begin{frame}
\frametitle{Aufbau von Rufzeichen}
\begin{columns}
    \begin{column}{0.48\textwidth}
    
\begin{figure}
    \DARCimage{0.85\linewidth}{654include}
    \caption{\scriptsize Aufbau Rufzeichen}
    \label{n_amateurfunkstrationen_aufbau_rufzeichen}
\end{figure}


    \end{column}
   \begin{column}{0.48\textwidth}
       Drei Teile:

\begin{itemize}
  \item Präfix
  \item Ziffer
  \item Suffix
  \end{itemize}

   \end{column}
\end{columns}

\end{frame}

\begin{frame}
\frametitle{Präfix}
\begin{itemize}
  \item Ist länderspezifisch zugeordnet
  \item Zur Lokalisierung von Amateurfunkstationen
  \item Werden durch die Internatione Fernmeldeunion (International Telecommunication Union, ITU) festgelegt
  \item Stehen in den Radio Regulations (RR)
  \item Mitgliedsstaaten sollen dieses in nationales Recht umsetzen
  \end{itemize}
\end{frame}

\begin{frame}
\frametitle{Präfixe Deutschland}
\begin{itemize}
  \item DA bis DR
  \item Y2 bis Y9
  \end{itemize}

\end{frame}

\begin{frame}
\only<1>{
\begin{QQuestion}{VA406}{Wo sind die Präfixe für Amateurfunkrufzeichen international geregelt?}{Im Amateurfunkgesetz (AFuG)}
{In den Radio Regulations (RR)}
{In der Rufzeichenliste der BNetzA}
{In der Amateurfunkverordnung (AFuV)}
\end{QQuestion}

}
\only<2>{
\begin{QQuestion}{VA406}{Wo sind die Präfixe für Amateurfunkrufzeichen international geregelt?}{Im Amateurfunkgesetz (AFuG)}
{\textbf{\textcolor{DARCgreen}{In den Radio Regulations (RR)}}}
{In der Rufzeichenliste der BNetzA}
{In der Amateurfunkverordnung (AFuV)}
\end{QQuestion}

}
\end{frame}

\begin{frame}
\frametitle{Personengebundene Rufzeichen Deutschland}
\begin{columns}
    \begin{column}{0.48\textwidth}
    \begin{itemize}
  \item 2 Buchstaben Präfix
  \item eine Ziffer
  \item 2-3 Buchstaben Suffix
  \end{itemize}

    \end{column}
   \begin{column}{0.48\textwidth}
       \begin{table}
\begin{DARCtabular}{l}
     DL1ABC   \\
     DO5XYZ   \\
     DA0RC   \\
\end{DARCtabular}
\caption{Beispiel für personengebundene, deutsche Amateurfunkrufzeichen}
\label{n_amateurfunkstationen_rufzeichen_beispiele}
\end{table}

   \end{column}
\end{columns}

\end{frame}

\begin{frame}
\only<1>{
\begin{QQuestion}{VD203}{Wie werden personengebundene deutsche Amateurfunkrufzeichen gebildet? Sie bestehen aus~...}{einem 2-buchstabigen Suffix (Landeskenner), zwei Ziffern und einem 2- oder 3-buchstabigen Präfix.}
{einem 2-buchstabigen Präfix (Landeskenner), einer Ziffer und einem 2- oder 3-buchstabigen Suffix.}
{einem 1-buchstabigen Präfix (Landeskenner), einer oder zwei Ziffern und einem 1-, 2- oder 3-buchstabigen Suffix.}
{einem 2-buchstabigen Suffix (Landeskenner), einer Ziffer und einem 1-, 2- oder 3-buchstabigen Präfix.}
\end{QQuestion}

}
\only<2>{
\begin{QQuestion}{VD203}{Wie werden personengebundene deutsche Amateurfunkrufzeichen gebildet? Sie bestehen aus~...}{einem 2-buchstabigen Suffix (Landeskenner), zwei Ziffern und einem 2- oder 3-buchstabigen Präfix.}
{\textbf{\textcolor{DARCgreen}{einem 2-buchstabigen Präfix (Landeskenner), einer Ziffer und einem 2- oder 3-buchstabigen Suffix.}}}
{einem 1-buchstabigen Präfix (Landeskenner), einer oder zwei Ziffern und einem 1-, 2- oder 3-buchstabigen Suffix.}
{einem 2-buchstabigen Suffix (Landeskenner), einer Ziffer und einem 1-, 2- oder 3-buchstabigen Präfix.}
\end{QQuestion}

}
\end{frame}

\begin{frame}
\frametitle{Weitere Rufzeichen}
\begin{itemize}
  \item Fernbediente und automatisch arbeitende Amateurfunkstellen
  \item Klubstationen
  \end{itemize}
\end{frame}

\begin{frame}
\only<1>{
\begin{QQuestion}{VD202}{Welche Rufzeichenzuteilungsarten gibt es im Amateurfunk unter anderem?}{Personengebundene Rufzeichen, Klubstationsrufzeichen, Contestrufzeichen}
{Personengebundene Rufzeichen, Familienrufzeichen, Klubstationsrufzeichen}
{Personengebundene Rufzeichen, Rufzeichen für fernbediente und automatisch arbeitende Amateurfunkstellen, Klubstationsrufzeichen}
{Personengebundene Rufzeichen, Rufzeichen für fernbediente und automatisch arbeitende Amateurfunkstellen, Mobilfunkrufzeichen}
\end{QQuestion}

}
\only<2>{
\begin{QQuestion}{VD202}{Welche Rufzeichenzuteilungsarten gibt es im Amateurfunk unter anderem?}{Personengebundene Rufzeichen, Klubstationsrufzeichen, Contestrufzeichen}
{Personengebundene Rufzeichen, Familienrufzeichen, Klubstationsrufzeichen}
{\textbf{\textcolor{DARCgreen}{Personengebundene Rufzeichen, Rufzeichen für fernbediente und automatisch arbeitende Amateurfunkstellen, Klubstationsrufzeichen}}}
{Personengebundene Rufzeichen, Rufzeichen für fernbediente und automatisch arbeitende Amateurfunkstellen, Mobilfunkrufzeichen}
\end{QQuestion}

}
\end{frame}

\begin{frame}
\frametitle{Rufzeichenplan}
\begin{itemize}
  \item In Deutschland durch die BNetzA
  \item Nach Rufzeichenplan (\textcolor{DARCblue}{\faLink~\href{https://50ohm.de/rzp}{50ohm.de/rzp}})
  \item Verwendungszweck erkennbar anhand von Präfix, Ziffern und Suffix
  \item Steht während der Prüfung zur Verfügung
  \end{itemize}

\end{frame}

\begin{frame}
\only<1>{
\begin{QQuestion}{VD201}{In welchem Regelwerk sind die Vorgaben für die Bildung von Rufzeichen für den Amateurfunkdienst in Deutschland zu finden?}{Im Rufzeichenplan der Bundesnetzagentur (BNetzA)}
{Im Amateurfunkgesetz (AFuG)}
{Im Bundesgesetzblatt (BGBl)}
{In den Radio Regulations (RR) der ITU}
\end{QQuestion}

}
\only<2>{
\begin{QQuestion}{VD201}{In welchem Regelwerk sind die Vorgaben für die Bildung von Rufzeichen für den Amateurfunkdienst in Deutschland zu finden?}{\textbf{\textcolor{DARCgreen}{Im Rufzeichenplan der Bundesnetzagentur (BNetzA)}}}
{Im Amateurfunkgesetz (AFuG)}
{Im Bundesgesetzblatt (BGBl)}
{In den Radio Regulations (RR) der ITU}
\end{QQuestion}

}
\end{frame}%ENDCONTENT
