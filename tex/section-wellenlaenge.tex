
\section{Wellenlänge}
\label{section:wellenlaenge}
\begin{frame}%STARTCONTENT

\frametitle{Wellenlänge}
\begin{itemize}
  \item Der Abstand zwischen zwei gleichen Durchläufen einer Welle heißt \emph{Wellenlänge}
  \item Je größer die Frequenz, desto kleiner die Wellenlänge
  \end{itemize}
    \pause
    Die Wellenlänge wird mit dem griechischen Buchstaben $\lambda$ (Lambda) angegeben und in Meter ($m$) gemessen.



\end{frame}

\begin{frame}
\only<1>{
\begin{PQuestion}{NB403}{Was ist in der dargestellten Momentaufnahme einer Welle mit 2 markiert?}{Wellenlänge}
{Amplitude}
{Spannung}
{Strom}
{\DARCimage{1.0\linewidth}{628include}}\end{PQuestion}

}
\only<2>{
\begin{PQuestion}{NB403}{Was ist in der dargestellten Momentaufnahme einer Welle mit 2 markiert?}{\textbf{\textcolor{DARCgreen}{Wellenlänge}}}
{Amplitude}
{Spannung}
{Strom}
{\DARCimage{1.0\linewidth}{628include}}\end{PQuestion}

}
 \end{frame}

\begin{frame}
\only<1>{
\begin{QQuestion}{NA205}{Welche Einheit wird üblicherweise für die Wellenlänge verwendet?}{Meter pro Sekunde (m/s)}
{Meter (m)}
{Hertz (Hz)}
{Sekunde pro Meter (s/m)}
\end{QQuestion}

}
\only<2>{
\begin{QQuestion}{NA205}{Welche Einheit wird üblicherweise für die Wellenlänge verwendet?}{Meter pro Sekunde (m/s)}
{\textbf{\textcolor{DARCgreen}{Meter (m)}}}
{Hertz (Hz)}
{Sekunde pro Meter (s/m)}
\end{QQuestion}

}
 \end{frame}

\begin{frame}
\frametitle{Zusammenhang Frequenz – Wellenlänge}
\begin{itemize}
  \item Über die Lichtgeschwindigkeit
  \item Eine Welle mit einer Frequenz von \qty{1}{\hertz} breitet sich 300.\qty{000}{\kilo\metre} aus bevor der nächste Durchlauf beginnt
  \item Bei \qty{1000}{\hertz} sind es nur \qty{300}{\kilo\metre}
  \item Bei \qty{1}{\mega\hertz} sind es \qty{300}{\metre}
  \end{itemize}

\end{frame}

\begin{frame}$f[\textrm{MHz}] = \dfrac{300}{\lambda[\textrm{m}]} \quad\quad\quad \lambda[\textrm{m}] = \dfrac{300}{f[\textrm{MHz}]}$

\end{frame}

\begin{frame}
\frametitle{Beispiele}
Wellenlänge aus Frequenz

$\lambda[\text{m}] = \dfrac{300}{f[\text{MHz}]} = \dfrac{300}{145,3 \ \text{MHz}} \approx 2,06 \ \text{m}$
    \pause
    Frequenz aus Wellenlänge

$f[\text{MHz}] = \dfrac{300}{\lambda[\text{m}]} = \dfrac{300}{2,06 \ \text{m}} \approx 145,3 \ \text{MHz}$



\end{frame}

\begin{frame}
\frametitle{Band}
Statt der Frequenz wird häufig das gerundete Band angegeben

\begin{table}
\begin{DARCtabular}{lll}
     Frequenz  & Wellenlänge  & Band   \\
     \qty{28}{\mega\hertz} -- \qty{29,7}{\mega\hertz}  & \qty{10,7}{\metre} -- \qty{10,1}{\metre}  & \qty{10}{\metre}-Band   \\
     \qty{144}{\mega\hertz} -- \qty{146}{\mega\hertz}  & \qty{2,08}{\metre} -- \qty{2,05}{\metre}  & \qty{2}{\metre}-Band   \\
     \qty{430}{\mega\hertz} -- \qty{440}{\mega\hertz}  & \qty{68}{\centi\metre} -- \qty{70}{\centi\metre}  & \qty{70}{\centi\metre}-Band   \\
\end{DARCtabular}
\caption{Die drei Amateurfunkbänder, die für alle Klassen freigegeben sind}
\label{n_funkwellen_baender}
\end{table}
\end{frame}

\begin{frame}
\only<1>{
\begin{QQuestion}{NB302}{Welcher Frequenz $f$ entspricht in etwa eine Wellenlänge von \qty{2,08}{\m} im Freiraum?}{\qty{149}{\MHz}}
{\qty{144}{\MHz}}
{\qty{433}{\MHz}}
{\qty{437}{\MHz}}
\end{QQuestion}

}
\only<2>{
\begin{QQuestion}{NB302}{Welcher Frequenz $f$ entspricht in etwa eine Wellenlänge von \qty{2,08}{\m} im Freiraum?}{\qty{149}{\MHz}}
{\textbf{\textcolor{DARCgreen}{\qty{144}{\MHz}}}}
{\qty{433}{\MHz}}
{\qty{437}{\MHz}}
\end{QQuestion}

}
\end{frame}

\begin{frame}
\only<1>{
\begin{QQuestion}{NB303}{Welcher Wellenlänge $\lambda$ entspricht in etwa eine Frequenz von \qty{433,500}{\MHz} im Freiraum?}{\qty{2,06}{\m}}
{\qty{58,0}{\cm}}
{\qty{0,69}{\m}}
{\qty{198}{\cm}}
\end{QQuestion}

}
\only<2>{
\begin{QQuestion}{NB303}{Welcher Wellenlänge $\lambda$ entspricht in etwa eine Frequenz von \qty{433,500}{\MHz} im Freiraum?}{\qty{2,06}{\m}}
{\qty{58,0}{\cm}}
{\textbf{\textcolor{DARCgreen}{\qty{0,69}{\m}}}}
{\qty{198}{\cm}}
\end{QQuestion}

}
\end{frame}%ENDCONTENT
