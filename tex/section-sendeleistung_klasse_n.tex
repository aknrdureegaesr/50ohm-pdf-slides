
\section{Sendeleistung Klasse N}
\label{section:sendeleistung_klasse_n}
\begin{frame}%STARTCONTENT
\begin{itemize}
  \item 10~m-Band: 10~W ERP
  \item 2~m- und 70~cm-Band: 10~W EIRP
  \end{itemize}
    \pause
    Ein Funkgerät mit 5W Sendeleistung und einem Gewinnfaktor von 1,8 bezogen auf den isotropen Kugelstrahler darf damit betrieben werden:<br/>5~W $\cdot$ 1,8 = 9~W

\end{frame}

\begin{frame}
\only<1>{
\begin{QQuestion}{VD724}{Wie hoch ist die maximal zulässige isotrope Strahlungsleistung (EIRP) für Funkamateure mit der Zulassungsklasse~N im \qty{2}{\m}- und \qty{70}{\cm}-Band?}{\qty{100}{\W}}
{\qty{5}{\W}}
{\qty{10}{\W}}
{\qty{25}{\W}}
\end{QQuestion}

}
\only<2>{
\begin{QQuestion}{VD724}{Wie hoch ist die maximal zulässige isotrope Strahlungsleistung (EIRP) für Funkamateure mit der Zulassungsklasse~N im \qty{2}{\m}- und \qty{70}{\cm}-Band?}{\qty{100}{\W}}
{\qty{5}{\W}}
{\textbf{\textcolor{DARCgreen}{\qty{10}{\W}}}}
{\qty{25}{\W}}
\end{QQuestion}

}
\end{frame}

\begin{frame}
\only<1>{
\begin{QQuestion}{VD743}{Wie hoch ist die maximal zulässige effektive Strahlungsleistung (ERP) für Funkamateure mit der Zulassungsklasse~N im \qty{10}{\m}-Band?}{\qty{100}{\W}}
{\qty{5}{\W}}
{\qty{10}{\W}}
{\qty{25}{\W}}
\end{QQuestion}

}
\only<2>{
\begin{QQuestion}{VD743}{Wie hoch ist die maximal zulässige effektive Strahlungsleistung (ERP) für Funkamateure mit der Zulassungsklasse~N im \qty{10}{\m}-Band?}{\qty{100}{\W}}
{\qty{5}{\W}}
{\textbf{\textcolor{DARCgreen}{\qty{10}{\W}}}}
{\qty{25}{\W}}
\end{QQuestion}

}
\end{frame}

\begin{frame}
\only<1>{
\begin{QQuestion}{VD726}{Sie sind Inhaber einer Zulassung für den Amateurfunkdienst der Klasse N und nutzen ein Funkgerät mit 5 W Senderausgangsleistung. Dürfen Sie bei Sendebetrieb im \qty{2}{\m}-Band eine direkt angeschlossene Antenne mit Gewinnfaktor 1,8 bezogen auf den isotropen Kugelstrahler (entspricht 2,6 dBi Gewinn) verwenden?}{Nein, da sich eine Strahlungsleistung von über 10 W EIRP ergibt.}
{Ja, außer wenn die Amateurfunkstelle ortsfest betrieben wird.}
{Ja, da die Strahlungsleistung den Grenzwert von 10 W EIRP nicht überschreitet.}
{Nein, da ich Antennen mit Gewinn nicht benutzen darf}
\end{QQuestion}

}
\only<2>{
\begin{QQuestion}{VD726}{Sie sind Inhaber einer Zulassung für den Amateurfunkdienst der Klasse N und nutzen ein Funkgerät mit 5 W Senderausgangsleistung. Dürfen Sie bei Sendebetrieb im \qty{2}{\m}-Band eine direkt angeschlossene Antenne mit Gewinnfaktor 1,8 bezogen auf den isotropen Kugelstrahler (entspricht 2,6 dBi Gewinn) verwenden?}{Nein, da sich eine Strahlungsleistung von über 10 W EIRP ergibt.}
{Ja, außer wenn die Amateurfunkstelle ortsfest betrieben wird.}
{\textbf{\textcolor{DARCgreen}{Ja, da die Strahlungsleistung den Grenzwert von 10 W EIRP nicht überschreitet.}}}
{Nein, da ich Antennen mit Gewinn nicht benutzen darf}
\end{QQuestion}

}
\end{frame}

\begin{frame}
\only<1>{
\begin{QQuestion}{VD725}{Sie sind Inhaber einer Zulassung für den Amateurfunkdienst der Klasse N und nutzen ein Funkgerät mit 5 W Senderausgangsleistung. Dürfen Sie bei Sendebetrieb im \qty{2}{\m}-Band eine direkt angeschlossene Antenne mit Gewinnfaktor 2,5 bezogen auf den isotropen Kugelstrahler (entspricht 4,0 dBi Gewinn) verwenden?}{Ja, da die Senderausgangsleistung den Grenzwert von 10 W EIRP nicht überschreitet.}
{Nein, da ich Antennen mit Gewinn nicht benutzen darf.}
{Nein, da sich eine Strahlungsleistung von über 10 W EIRP ergibt.}
{Ja, sofern es sich um ein Handfunkgerät handelt.}
\end{QQuestion}

}
\only<2>{
\begin{QQuestion}{VD725}{Sie sind Inhaber einer Zulassung für den Amateurfunkdienst der Klasse N und nutzen ein Funkgerät mit 5 W Senderausgangsleistung. Dürfen Sie bei Sendebetrieb im \qty{2}{\m}-Band eine direkt angeschlossene Antenne mit Gewinnfaktor 2,5 bezogen auf den isotropen Kugelstrahler (entspricht 4,0 dBi Gewinn) verwenden?}{Ja, da die Senderausgangsleistung den Grenzwert von 10 W EIRP nicht überschreitet.}
{Nein, da ich Antennen mit Gewinn nicht benutzen darf.}
{\textbf{\textcolor{DARCgreen}{Nein, da sich eine Strahlungsleistung von über 10 W EIRP ergibt.}}}
{Ja, sofern es sich um ein Handfunkgerät handelt.}
\end{QQuestion}

}
\end{frame}%ENDCONTENT
