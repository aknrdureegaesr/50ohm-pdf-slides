
\section{Elektromagnetische Verträglichkeit}
\label{section:elektromagnetische_vertraeglichkeit}
\begin{frame}%STARTCONTENT

\frametitle{Beim Senden}
    \pause
    
\begin{columns}
    \begin{column}{0.48\textwidth}
    Funkwellen von

\begin{itemize}
  \item von Antennen
  \item von Transceivern
  \item von Zuleitungen
  \end{itemize}

    \end{column}
    \pause
    
   \begin{column}{0.48\textwidth}
       Elektrische Schwingungen gelangen in andere Leitungen

\begin{itemize}
  \item Zerstörungen von anderen elektronischen Geräten
  \item Geräusche aus Lautsprechern
  \item Internetausfall
  \item Fehler in Heizungssteuerung
  \end{itemize}

   \end{column}
\end{columns}



\end{frame}

\begin{frame}
\frametitle{Beim Senden}
Einhalten der Schutzanforderungen zur Gewährleistung der elektromagnetischen Verträglichkeit im Sinne des Gesetzes über die elektromagnetische Verträglichkeit von Betriebsmitteln (EMVG)

\end{frame}

\begin{frame}
\only<1>{
\begin{QQuestion}{VC118}{Was muss ein Funkamateur beim Betrieb seiner Amateurfunkstelle in Bezug auf die elektromagnetische Verträglichkeit beachten?}{Die Amateurfunkstelle darf nur aus baumustergeprüften Funkgeräten bestehen, die den Anforderungen des Gesetzes über Funkanlagen (FuAG) entsprechen.}
{Der Funkamateur benötigt für seine Amateurfunkstelle eine aktuelle  Verträglichkeitsbescheinigung der BNetzA.
}
{Der Funkamateur muss die Schutzanforderungen zur Gewährleistung der elektromagnetischen Verträglichkeit im Sinne des Gesetzes über die elektromagnetische Verträglichkeit von Betriebsmitteln (EMVG) einhalten.}
{Die Amateurfunkstelle muss von einem zertifizierten Elektromeister auf die Einhaltung der elektromagnetischen Verträglichkeit geprüft werden. Das Abnahmeprotokoll ist für die BNetzA bereitzuhalten.
}
\end{QQuestion}

}
\only<2>{
\begin{QQuestion}{VC118}{Was muss ein Funkamateur beim Betrieb seiner Amateurfunkstelle in Bezug auf die elektromagnetische Verträglichkeit beachten?}{Die Amateurfunkstelle darf nur aus baumustergeprüften Funkgeräten bestehen, die den Anforderungen des Gesetzes über Funkanlagen (FuAG) entsprechen.}
{Der Funkamateur benötigt für seine Amateurfunkstelle eine aktuelle  Verträglichkeitsbescheinigung der BNetzA.
}
{\textbf{\textcolor{DARCgreen}{Der Funkamateur muss die Schutzanforderungen zur Gewährleistung der elektromagnetischen Verträglichkeit im Sinne des Gesetzes über die elektromagnetische Verträglichkeit von Betriebsmitteln (EMVG) einhalten.}}}
{Die Amateurfunkstelle muss von einem zertifizierten Elektromeister auf die Einhaltung der elektromagnetischen Verträglichkeit geprüft werden. Das Abnahmeprotokoll ist für die BNetzA bereitzuhalten.
}
\end{QQuestion}

}
\end{frame}

\begin{frame}
\frametitle{Beim Empfang}
Funkamateur darf Störfestigkeit der eigenen Geräte selbst bestimmen. Die Abweichung vom EMVG ist ein Privileg.

\end{frame}

\begin{frame}
\only<1>{
\begin{QQuestion}{VC120}{Darf der Funkamateur bei Selbstbaugeräten von den grundlegenden Anforderungen zur Störfestigkeit im Sinne des Gesetzes über die elektromagnetische Verträglichkeit von Betriebsmitteln (EMVG) abweichen?}{Nein, die Störfestigkeit ist vorgegeben und muss eingehalten werden.}
{Ja, aber nur in Richtung Verbesserung der Störfestigkeit}
{Ja, er kann den Grad der Störfestigkeit seiner Geräte selbst bestimmen.}
{Nein, selbstgebaute Amateurfunkgeräte müssen im Bezug auf Störfestigkeit  kommerziell hergestellten Geräten entsprechen.}
\end{QQuestion}

}
\only<2>{
\begin{QQuestion}{VC120}{Darf der Funkamateur bei Selbstbaugeräten von den grundlegenden Anforderungen zur Störfestigkeit im Sinne des Gesetzes über die elektromagnetische Verträglichkeit von Betriebsmitteln (EMVG) abweichen?}{Nein, die Störfestigkeit ist vorgegeben und muss eingehalten werden.}
{Ja, aber nur in Richtung Verbesserung der Störfestigkeit}
{\textbf{\textcolor{DARCgreen}{Ja, er kann den Grad der Störfestigkeit seiner Geräte selbst bestimmen.}}}
{Nein, selbstgebaute Amateurfunkgeräte müssen im Bezug auf Störfestigkeit  kommerziell hergestellten Geräten entsprechen.}
\end{QQuestion}

}
\end{frame}

\begin{frame}
\only<1>{
\begin{QQuestion}{VC119}{Was gilt hinsichtlich der Störfestigkeit der Amateurfunkstelle nach dem Wortlaut des Amateurfunkgesetzes (AFuG)? }{Der Funkamateur darf von den grundlegenden Anforderungen nach dem Gesetz über die elektromagnetische Verträglichkeit von Betriebsmitteln (EMVG) abweichen und kann den Grad der Störfestigkeit seiner Amateurfunkstelle selbst bestimmen.}
{Der Funkamateur muss seine Amateurfunkstelle im Abstand von 2 Jahren einer Störfestigkeitsprüfung durch die BNetzA unterziehen lassen.}
{Amateurfunkstellen sind hinsichtlich ihrer Störfestigkeit anderen Betriebsmitteln gleichgestellt.}
{Amateurfunkstellen müssen elektromagnetische Störungen durch andere Betriebsmittel hinnehmen, selbst wenn diese nicht den grundlegenden Anforderungen nach dem Gesetz über die elektromagnetische Verträglichkeit von Betriebsmitteln (EMVG) entsprechen.}
\end{QQuestion}

}
\only<2>{
\begin{QQuestion}{VC119}{Was gilt hinsichtlich der Störfestigkeit der Amateurfunkstelle nach dem Wortlaut des Amateurfunkgesetzes (AFuG)? }{\textbf{\textcolor{DARCgreen}{Der Funkamateur darf von den grundlegenden Anforderungen nach dem Gesetz über die elektromagnetische Verträglichkeit von Betriebsmitteln (EMVG) abweichen und kann den Grad der Störfestigkeit seiner Amateurfunkstelle selbst bestimmen.}}}
{Der Funkamateur muss seine Amateurfunkstelle im Abstand von 2 Jahren einer Störfestigkeitsprüfung durch die BNetzA unterziehen lassen.}
{Amateurfunkstellen sind hinsichtlich ihrer Störfestigkeit anderen Betriebsmitteln gleichgestellt.}
{Amateurfunkstellen müssen elektromagnetische Störungen durch andere Betriebsmittel hinnehmen, selbst wenn diese nicht den grundlegenden Anforderungen nach dem Gesetz über die elektromagnetische Verträglichkeit von Betriebsmitteln (EMVG) entsprechen.}
\end{QQuestion}

}
\end{frame}

\begin{frame}
\frametitle{Maßnahmen}
Zur Einhaltung der vorgeschriebenen elektromagnetischen Verträglichkeit (EMV)

\begin{itemize}
  \item Abschirmen
  \item Erden
  \end{itemize}
    \pause
    Schutz vor Störungen in beide Richtungen



\end{frame}

\begin{frame}
\only<1>{
\begin{QQuestion}{NK101}{In Bezug auf EMV sollten HF-Stufen~...}{gut abgeschirmt werden.}
{eine besonders abgeschirmte Masseleitung erhalten.}
{in Kunststoff eingehüllt werden.}
{nur kapazitive Auskopplungen enthalten.}
\end{QQuestion}

}
\only<2>{
\begin{QQuestion}{NK101}{In Bezug auf EMV sollten HF-Stufen~...}{\textbf{\textcolor{DARCgreen}{gut abgeschirmt werden.}}}
{eine besonders abgeschirmte Masseleitung erhalten.}
{in Kunststoff eingehüllt werden.}
{nur kapazitive Auskopplungen enthalten.}
\end{QQuestion}

}
\end{frame}

\begin{frame}
\only<1>{
\begin{QQuestion}{NJ101}{Alle Geräte, die HF-Ströme übertragen, sollten~...}{möglichst gut geschirmt sein.}
{nicht geerdet sein.}
{über das Stromversorgungsnetz geerdet sein.}
{durch Kunststoffabdeckungen geschützt sein.}
\end{QQuestion}

}
\only<2>{
\begin{QQuestion}{NJ101}{Alle Geräte, die HF-Ströme übertragen, sollten~...}{\textbf{\textcolor{DARCgreen}{möglichst gut geschirmt sein.}}}
{nicht geerdet sein.}
{über das Stromversorgungsnetz geerdet sein.}
{durch Kunststoffabdeckungen geschützt sein.}
\end{QQuestion}

}
\end{frame}

\begin{frame}
\only<1>{
\begin{QQuestion}{NK102}{Um eine Amateurfunkstelle in Bezug auf EMV zu optimieren,~...}{sollten alle hochohmigen Erdverbindungen entfernt werden.}
{sollte der Sender mit der Wasserleitung im Haus verbunden werden.}
{sollten alle Einrichtungen mit einer guten HF-Erdung versehen werden.}
{sollte der Sender mit der Abwasserleitung im Haus verbunden werden.}
\end{QQuestion}

}
\only<2>{
\begin{QQuestion}{NK102}{Um eine Amateurfunkstelle in Bezug auf EMV zu optimieren,~...}{sollten alle hochohmigen Erdverbindungen entfernt werden.}
{sollte der Sender mit der Wasserleitung im Haus verbunden werden.}
{\textbf{\textcolor{DARCgreen}{sollten alle Einrichtungen mit einer guten HF-Erdung versehen werden.}}}
{sollte der Sender mit der Abwasserleitung im Haus verbunden werden.}
\end{QQuestion}

}
\end{frame}%ENDCONTENT
