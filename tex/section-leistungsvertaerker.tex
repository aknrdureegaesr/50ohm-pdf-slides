
\section{Leistungsverstärker}
\label{section:leistungsvertaerker}
\begin{frame}%STARTCONTENT

\only<1>{
\begin{PQuestion}{AF412}{Welche Art von Schaltung wird im folgenden Bild dargestellt? Es handelt sich um einen~...}{modulierbaren Oszillator.}
{selektiven Hochfrequenzverstärker.}
{Breitband-Gegentaktverstärker.}
{Breitband-Frequenzverdoppler.}
{\DARCimage{1.0\linewidth}{491include}}\end{PQuestion}

}
\only<2>{
\begin{PQuestion}{AF412}{Welche Art von Schaltung wird im folgenden Bild dargestellt? Es handelt sich um einen~...}{modulierbaren Oszillator.}
{selektiven Hochfrequenzverstärker.}
{\textbf{\textcolor{DARCgreen}{Breitband-Gegentaktverstärker.}}}
{Breitband-Frequenzverdoppler.}
{\DARCimage{1.0\linewidth}{491include}}\end{PQuestion}

}
\end{frame}

\begin{frame}
\only<1>{
\begin{PQuestion}{AF408}{Worum handelt es sich bei dieser Schaltung?}{Es handelt sich um einen selektiven HF-Verstärker.}
{Es handelt sich um einen selektiven Mischer.}
{Es handelt sich um einen breitbandigen NF-Verstärker.}
{Es handelt sich um einen frequenzvervielfachenden Oszillator.}
{\DARCimage{1.0\linewidth}{778include}}\end{PQuestion}

}
\only<2>{
\begin{PQuestion}{AF408}{Worum handelt es sich bei dieser Schaltung?}{\textbf{\textcolor{DARCgreen}{Es handelt sich um einen selektiven HF-Verstärker.}}}
{Es handelt sich um einen selektiven Mischer.}
{Es handelt sich um einen breitbandigen NF-Verstärker.}
{Es handelt sich um einen frequenzvervielfachenden Oszillator.}
{\DARCimage{1.0\linewidth}{778include}}\end{PQuestion}

}
\end{frame}

\begin{frame}
\only<1>{
\begin{PQuestion}{AF413}{Worum handelt es sich bei dieser Schaltung? Es handelt sich um einen...}{zweistufigen Breitband-HF-Verstärker.}
{selektiven Hochfrequenzverstärker.}
{Gegentakt-Verstärker im B-Betrieb.}
{zweistufigen LC-Oszillator.}
{\DARCimage{1.0\linewidth}{764include}}\end{PQuestion}

}
\only<2>{
\begin{PQuestion}{AF413}{Worum handelt es sich bei dieser Schaltung? Es handelt sich um einen...}{\textbf{\textcolor{DARCgreen}{zweistufigen Breitband-HF-Verstärker.}}}
{selektiven Hochfrequenzverstärker.}
{Gegentakt-Verstärker im B-Betrieb.}
{zweistufigen LC-Oszillator.}
{\DARCimage{1.0\linewidth}{764include}}\end{PQuestion}

}
\end{frame}

\begin{frame}
\only<1>{
\begin{PQuestion}{AF409}{Welchem Zweck dient die Anzapfung an X in der folgenden Schaltung?}{Sie bewirkt die notwendige Entkopplung für den Schwingungseinsatz der Oszillatorstufe.}
{Sie ermöglicht die Dreipunkt-Rückkopplung des Oszillators.}
{Sie dient zur Anpassung der Eingangsimpedanz dieser Stufe an die vorgelagerte Stufe.}
{Sie bewirkt eine stärkere Bedämpfung des Eingangsschwingkreises.}
{\DARCimage{1.0\linewidth}{779include}}\end{PQuestion}

}
\only<2>{
\begin{PQuestion}{AF409}{Welchem Zweck dient die Anzapfung an X in der folgenden Schaltung?}{Sie bewirkt die notwendige Entkopplung für den Schwingungseinsatz der Oszillatorstufe.}
{Sie ermöglicht die Dreipunkt-Rückkopplung des Oszillators.}
{\textbf{\textcolor{DARCgreen}{Sie dient zur Anpassung der Eingangsimpedanz dieser Stufe an die vorgelagerte Stufe.}}}
{Sie bewirkt eine stärkere Bedämpfung des Eingangsschwingkreises.}
{\DARCimage{1.0\linewidth}{779include}}\end{PQuestion}

}
\end{frame}

\begin{frame}
\only<1>{
\begin{PQuestion}{AF410}{Welchem Zweck dienen $C_1$ und $C_2$ in der folgenden Schaltung? Sie dienen zur...}{Impedanzanpassung. }
{Verhinderung der Schwingneigung.}
{Realisierung einer kapazitiven Dreipunktschaltung für den Oszillator.}
{Unterdrückung von Oberschwingungen.}
{\DARCimage{1.0\linewidth}{780include}}\end{PQuestion}

}
\only<2>{
\begin{PQuestion}{AF410}{Welchem Zweck dienen $C_1$ und $C_2$ in der folgenden Schaltung? Sie dienen zur...}{\textbf{\textcolor{DARCgreen}{Impedanzanpassung. }}}
{Verhinderung der Schwingneigung.}
{Realisierung einer kapazitiven Dreipunktschaltung für den Oszillator.}
{Unterdrückung von Oberschwingungen.}
{\DARCimage{1.0\linewidth}{780include}}\end{PQuestion}

}
\end{frame}

\begin{frame}
\only<1>{
\begin{PQuestion}{AF414}{Wozu dient der Transformator $T_1$ der folgenden Schaltung?}{Er dient der Anpassung des Ausgangswiderstandes der Emitterschaltung an den Eingang der folgenden Emitterschaltung.}
{Er dient der Anpassung des Ausgangswiderstandes der Emitterschaltung an den Eingang der folgenden Kollektorschaltung.}
{Er dient der Anpassung des Ausgangswiderstandes der Kollektorschaltung an den Eingang der folgenden Emitterschaltung.}
{Er dient der Anpassung des Ausgangswiderstandes der Kollektorschaltung an den Eingang der folgenden PA.}
{\DARCimage{1.0\linewidth}{765include}}\end{PQuestion}

}
\only<2>{
\begin{PQuestion}{AF414}{Wozu dient der Transformator $T_1$ der folgenden Schaltung?}{\textbf{\textcolor{DARCgreen}{Er dient der Anpassung des Ausgangswiderstandes der Emitterschaltung an den Eingang der folgenden Emitterschaltung.}}}
{Er dient der Anpassung des Ausgangswiderstandes der Emitterschaltung an den Eingang der folgenden Kollektorschaltung.}
{Er dient der Anpassung des Ausgangswiderstandes der Kollektorschaltung an den Eingang der folgenden Emitterschaltung.}
{Er dient der Anpassung des Ausgangswiderstandes der Kollektorschaltung an den Eingang der folgenden PA.}
{\DARCimage{1.0\linewidth}{765include}}\end{PQuestion}

}
\end{frame}

\begin{frame}
\only<1>{
\begin{PQuestion}{AF407}{Welche Funktion haben die mit X gekennzeichneten Bauteile in der folgenden Schaltung?}{Sie schützen den Transistor vor thermischer Überlastung.}
{Sie schützen den Transistor vor unerwünschten Rückkopplungen und filtern Eigenschwingungen des Transistors aus. }
{Sie dienen zur optimalen Einstellung des Arbeitspunktes für den Transistor.}
{Sie transformieren die Ausgangsimpedanz der vorhergehenden Stufe auf die Eingangsimpedanz des Transistors. }
{\DARCimage{1.0\linewidth}{768include}}\end{PQuestion}

}
\only<2>{
\begin{PQuestion}{AF407}{Welche Funktion haben die mit X gekennzeichneten Bauteile in der folgenden Schaltung?}{Sie schützen den Transistor vor thermischer Überlastung.}
{Sie schützen den Transistor vor unerwünschten Rückkopplungen und filtern Eigenschwingungen des Transistors aus. }
{Sie dienen zur optimalen Einstellung des Arbeitspunktes für den Transistor.}
{\textbf{\textcolor{DARCgreen}{Sie transformieren die Ausgangsimpedanz der vorhergehenden Stufe auf die Eingangsimpedanz des Transistors. }}}
{\DARCimage{1.0\linewidth}{768include}}\end{PQuestion}

}
\end{frame}

\begin{frame}
\only<1>{
\begin{PQuestion}{AF406}{Welche Funktion haben die mit X gekennzeichneten Bauteile in der folgenden Schaltung? Sie ~...}{dienen als Sperrkreis. }
{passen die Lastimpedanz an die gewünschte Impedanz für die Transistorschaltung an.}
{dienen der Trägerunterdrückung bei SSB-Modulation. }
{dienen als Bandsperre. }
{\DARCimage{1.0\linewidth}{769include}}\end{PQuestion}

}
\only<2>{
\begin{PQuestion}{AF406}{Welche Funktion haben die mit X gekennzeichneten Bauteile in der folgenden Schaltung? Sie ~...}{dienen als Sperrkreis. }
{\textbf{\textcolor{DARCgreen}{passen die Lastimpedanz an die gewünschte Impedanz für die Transistorschaltung an.}}}
{dienen der Trägerunterdrückung bei SSB-Modulation. }
{dienen als Bandsperre. }
{\DARCimage{1.0\linewidth}{769include}}\end{PQuestion}

}
\end{frame}

\begin{frame}
\only<1>{
\begin{PQuestion}{AF417}{Zu welchem Zweck dienen $T_1$ und $T_2$ in diesem HF-Leistungsverstärker?}{Zur Anpassung von \qty{50}{\ohm} an die hochohmige Eingangsimpedanz der Transistoren und die niederohmige Ausgangsimpedanz der Transistoren an \qty{50}{\ohm}.}
{Zur Anpassung von \qty{50}{\ohm} an die niederohmige Eingangsimpedanz der Transistoren und die niederohmige Ausgangsimpedanz der Transistoren an \qty{50}{\ohm}.}
{Zur Anpassung von \qty{50}{\ohm} an die niederohmige Eingangsimpedanz der Transistoren und die hochohmige Ausgangsimpedanz der Transistoren an \qty{50}{\ohm}.}
{Zur Anpassung von \qty{50}{\ohm} an die hochohmige Eingangsimpedanz der Transistoren und die hochohmige Ausgangsimpedanz der Transistoren an \qty{50}{\ohm}.}
{\DARCimage{1.0\linewidth}{786include}}\end{PQuestion}

}
\only<2>{
\begin{PQuestion}{AF417}{Zu welchem Zweck dienen $T_1$ und $T_2$ in diesem HF-Leistungsverstärker?}{Zur Anpassung von \qty{50}{\ohm} an die hochohmige Eingangsimpedanz der Transistoren und die niederohmige Ausgangsimpedanz der Transistoren an \qty{50}{\ohm}.}
{\textbf{\textcolor{DARCgreen}{Zur Anpassung von \qty{50}{\ohm} an die niederohmige Eingangsimpedanz der Transistoren und die niederohmige Ausgangsimpedanz der Transistoren an \qty{50}{\ohm}.}}}
{Zur Anpassung von \qty{50}{\ohm} an die niederohmige Eingangsimpedanz der Transistoren und die hochohmige Ausgangsimpedanz der Transistoren an \qty{50}{\ohm}.}
{Zur Anpassung von \qty{50}{\ohm} an die hochohmige Eingangsimpedanz der Transistoren und die hochohmige Ausgangsimpedanz der Transistoren an \qty{50}{\ohm}.}
{\DARCimage{1.0\linewidth}{786include}}\end{PQuestion}

}
\end{frame}

\begin{frame}
\only<1>{
\begin{QQuestion}{AF405}{Welche Funktion hat das Ausgangs-Pi-Filter eines HF-Senders?}{Es dient der besseren Oberwellenanpassung an die Antenne.}
{Es dient der Impedanztransformation und verbessert die Unterdrückung von Oberwellen.}
{Es dient der Verbesserung des Wirkungsgrads der Endstufe durch Änderung der ALC.}
{Es dient dem Schutz der Endstufe bei offener oder kurzgeschlossener Antennenbuchse.}
\end{QQuestion}

}
\only<2>{
\begin{QQuestion}{AF405}{Welche Funktion hat das Ausgangs-Pi-Filter eines HF-Senders?}{Es dient der besseren Oberwellenanpassung an die Antenne.}
{\textbf{\textcolor{DARCgreen}{Es dient der Impedanztransformation und verbessert die Unterdrückung von Oberwellen.}}}
{Es dient der Verbesserung des Wirkungsgrads der Endstufe durch Änderung der ALC.}
{Es dient dem Schutz der Endstufe bei offener oder kurzgeschlossener Antennenbuchse.}
\end{QQuestion}

}
\end{frame}

\begin{frame}
\only<1>{
\begin{QQuestion}{AF404}{Wozu dienen LC-Schaltungen unmittelbar hinter einem HF-Leistungsverstärker? Sie dienen zur...}{Unterdrückung des HF-Trägers bei SSB-Modulation. }
{optimalen Einstellung des Arbeitspunktes des HF-Leistungsverstärkers.}
{Verringerung der rücklaufenden Leistung bei Fehlanpassung der Antennenimpedanz.}
{frequenzabhängigen Transformation der Senderausgangsimpedanz auf die Antenneneingangsimpedanz und zur Unterdrückung von Oberschwingungen.}
\end{QQuestion}

}
\only<2>{
\begin{QQuestion}{AF404}{Wozu dienen LC-Schaltungen unmittelbar hinter einem HF-Leistungsverstärker? Sie dienen zur...}{Unterdrückung des HF-Trägers bei SSB-Modulation. }
{optimalen Einstellung des Arbeitspunktes des HF-Leistungsverstärkers.}
{Verringerung der rücklaufenden Leistung bei Fehlanpassung der Antennenimpedanz.}
{\textbf{\textcolor{DARCgreen}{frequenzabhängigen Transformation der Senderausgangsimpedanz auf die Antenneneingangsimpedanz und zur Unterdrückung von Oberschwingungen.}}}
\end{QQuestion}

}
\end{frame}

\begin{frame}
\only<1>{
\begin{QQuestion}{AF401}{Wie ist der Wirkungsgrad eines HF-Verstärkers definiert?}{Als Erhöhung der Ausgangsleistung  bezogen auf die Eingangsleistung.}
{Als Verhältnis der Stärke der erwünschten Aussendung zur Stärke der unerwünschten Aussendungen.}
{Als Verhältnis der HF-Leistung zu der Verlustleistung der Endstufenröhre bzw. des Endstufentransistors.}
{Als Verhältnis der HF-Ausgangsleistung zu der zugeführten Gleichstromleistung.}
\end{QQuestion}

}
\only<2>{
\begin{QQuestion}{AF401}{Wie ist der Wirkungsgrad eines HF-Verstärkers definiert?}{Als Erhöhung der Ausgangsleistung  bezogen auf die Eingangsleistung.}
{Als Verhältnis der Stärke der erwünschten Aussendung zur Stärke der unerwünschten Aussendungen.}
{Als Verhältnis der HF-Leistung zu der Verlustleistung der Endstufenröhre bzw. des Endstufentransistors.}
{\textbf{\textcolor{DARCgreen}{Als Verhältnis der HF-Ausgangsleistung zu der zugeführten Gleichstromleistung.}}}
\end{QQuestion}

}
\end{frame}

\begin{frame}
\only<1>{
\begin{PQuestion}{AF420}{Die Arbeitspunkteinstellung der LDMOS-Kurzwellen-PA erfolgt mit $R_3$. Wie verändert sich der Drainstrom, wenn $R_3$ in Richtung 3 verstellt wird?}{Der Drainstrom in beiden Transistoren verringert sich.}
{Der Drainstrom in beiden Transistoren erhöht sich.}
{Der Drainstrom steigt in $K_1$ und sinkt in $K_2$.}
{Der Drainstrom sinkt in $K_1$ und steigt in $K_2$.}
{\DARCimage{1.0\linewidth}{786include}}\end{PQuestion}

}
\only<2>{
\begin{PQuestion}{AF420}{Die Arbeitspunkteinstellung der LDMOS-Kurzwellen-PA erfolgt mit $R_3$. Wie verändert sich der Drainstrom, wenn $R_3$ in Richtung 3 verstellt wird?}{\textbf{\textcolor{DARCgreen}{Der Drainstrom in beiden Transistoren verringert sich.}}}
{Der Drainstrom in beiden Transistoren erhöht sich.}
{Der Drainstrom steigt in $K_1$ und sinkt in $K_2$.}
{Der Drainstrom sinkt in $K_1$ und steigt in $K_2$.}
{\DARCimage{1.0\linewidth}{786include}}\end{PQuestion}

}
\end{frame}

\begin{frame}
\only<1>{
\begin{PQuestion}{AF423}{Der Ruhestrom in der dargestellten VHF-LDMOS-PA soll erhöht werden. Welche Einstellungen sind vorzunehmen?}{$R_1$ in Richtung $U_\text{BIAS}$ und $R_2$ in Richtung GND verstellen.}
{$R_1$ und $R_2$ in Richtung GND verstellen.}
{$R_1$ und $R_2$ in Richtung $U_\text{BIAS}$ verstellen.}
{$R_1$ in Richtung GND und $R_2$ in Richtung $U_\text{BIAS}$ verstellen.}
{\DARCimage{1.0\linewidth}{783include}}\end{PQuestion}

}
\only<2>{
\begin{PQuestion}{AF423}{Der Ruhestrom in der dargestellten VHF-LDMOS-PA soll erhöht werden. Welche Einstellungen sind vorzunehmen?}{$R_1$ in Richtung $U_\text{BIAS}$ und $R_2$ in Richtung GND verstellen.}
{$R_1$ und $R_2$ in Richtung GND verstellen.}
{\textbf{\textcolor{DARCgreen}{$R_1$ und $R_2$ in Richtung $U_\text{BIAS}$ verstellen.}}}
{$R_1$ in Richtung GND und $R_2$ in Richtung $U_\text{BIAS}$ verstellen.}
{\DARCimage{1.0\linewidth}{783include}}\end{PQuestion}

}
\end{frame}

\begin{frame}
\only<1>{
\begin{PQuestion}{AF424}{Wie verändern sich die Drainströme in den beiden Endstufen-Transistoren, wenn der Schleifer von $R_4$ in Richtung $U_\text{BIAS}$ verstellt wird?}{Drainstrom in Transistor 1 sinkt und Drainstrom in Transistor 2 bleibt konstant.}
{Drainstrom in Transistor 1 steigt und Drainstrom in Transistor 2 steigt.}
{Drainstrom in Transistor 1 sinkt und Drainstrom in Transistor 2 sinkt.}
{Drainstrom in Transistor 1 steigt und Drainstrom in Transistor 2 bleibt konstant.}
{\DARCimage{1.0\linewidth}{784include}}\end{PQuestion}

}
\only<2>{
\begin{PQuestion}{AF424}{Wie verändern sich die Drainströme in den beiden Endstufen-Transistoren, wenn der Schleifer von $R_4$ in Richtung $U_\text{BIAS}$ verstellt wird?}{Drainstrom in Transistor 1 sinkt und Drainstrom in Transistor 2 bleibt konstant.}
{Drainstrom in Transistor 1 steigt und Drainstrom in Transistor 2 steigt.}
{Drainstrom in Transistor 1 sinkt und Drainstrom in Transistor 2 sinkt.}
{\textbf{\textcolor{DARCgreen}{Drainstrom in Transistor 1 steigt und Drainstrom in Transistor 2 bleibt konstant.}}}
{\DARCimage{1.0\linewidth}{784include}}\end{PQuestion}

}
\end{frame}

\begin{frame}
\only<1>{
\begin{PQuestion}{AF421}{Wie groß ist die Gate-Source-Spannung, wenn sich der Schleifer von $R_3$ am Anschlag~1 befindet?}{\qty{2,77}{\volt}}
{\qty{3,5}{\volt}}
{\qty{3,7}{\volt}}
{\qty{0,45}{\volt}}
{\DARCimage{1.0\linewidth}{786include}}\end{PQuestion}

}
\only<2>{
\begin{PQuestion}{AF421}{Wie groß ist die Gate-Source-Spannung, wenn sich der Schleifer von $R_3$ am Anschlag~1 befindet?}{\qty{2,77}{\volt}}
{\textbf{\textcolor{DARCgreen}{\qty{3,5}{\volt}}}}
{\qty{3,7}{\volt}}
{\qty{0,45}{\volt}}
{\DARCimage{1.0\linewidth}{786include}}\end{PQuestion}

}
\end{frame}

\begin{frame}
\frametitle{Lösungsweg}
\begin{columns}
    \begin{column}{0.48\textwidth}
    \begin{itemize}
  \item gegeben: $U_Z = 6,2V$
  \item gegeben: $R_2 = 270Ω$
  \item gegeben: $R_3 = 220Ω$
  \end{itemize}

    \end{column}
   \begin{column}{0.48\textwidth}
       \begin{itemize}
  \item gegeben: $R_4 = 6,8kΩ$
  \item gegeben: $R_6 = 150Ω$
  \item gesucht: $U_{GS}$
  \end{itemize}

   \end{column}
\end{columns}
    \pause
    $R_E = \frac{(R_3+R_6) \cdot R_4}{(R_3 + R_6) + R_4} = \frac{220Ω + 150Ω) \cdot 6,8kΩ}{220Ω + 150Ω + 6,8kΩ} = \frac{2,516MΩ^2}{7170Ω} = 351Ω$

$\frac{U_Z}{U_{GS}} = \frac{R_2 + R_E}{R_E} \Rightarrow \frac{6,2V}{U_{GS}} = \frac{270Ω+351Ω}{351Ω} = 1,77 \Rightarrow U_{GS} = \frac{6,2V}{1,77} = 3,50V$



\end{frame}

\begin{frame}
\only<1>{
\begin{PQuestion}{AF411}{Welchem Zweck dient X in der folgenden Schaltung?}{Zur Abstimmung}
{Zur HF-Entkopplung}
{Zur Wechselstromkopplung}
{Zur Kopplung mit der nächstfolgenden Stufe}
{\DARCimage{1.0\linewidth}{781include}}\end{PQuestion}

}
\only<2>{
\begin{PQuestion}{AF411}{Welchem Zweck dient X in der folgenden Schaltung?}{Zur Abstimmung}
{\textbf{\textcolor{DARCgreen}{Zur HF-Entkopplung}}}
{Zur Wechselstromkopplung}
{Zur Kopplung mit der nächstfolgenden Stufe}
{\DARCimage{1.0\linewidth}{781include}}\end{PQuestion}

}
\end{frame}

\begin{frame}
\only<1>{
\begin{PQuestion}{AF419}{Zu welchem Zweck dient die Schaltung der Spule, $C_2$ und $C_3$?}{Sie reduziert Brummspannungsanteile auf dem Sendesignal.}
{Sie reduziert HF-Anteile auf der Betriebsspannungsleitung.}
{Sie reduziert Oberschwingungen auf dem Sendesignal.}
{Sie wirkt als Pi-Filter für das Sendesignal.}
{\DARCimage{1.0\linewidth}{786include}}\end{PQuestion}

}
\only<2>{
\begin{PQuestion}{AF419}{Zu welchem Zweck dient die Schaltung der Spule, $C_2$ und $C_3$?}{Sie reduziert Brummspannungsanteile auf dem Sendesignal.}
{\textbf{\textcolor{DARCgreen}{Sie reduziert HF-Anteile auf der Betriebsspannungsleitung.}}}
{Sie reduziert Oberschwingungen auf dem Sendesignal.}
{Sie wirkt als Pi-Filter für das Sendesignal.}
{\DARCimage{1.0\linewidth}{786include}}\end{PQuestion}

}
\end{frame}

\begin{frame}
\only<1>{
\begin{PQuestion}{AF418}{Welche Funktion trifft für die Spule, $C_2$ und $C_3$ in der Schaltung zu?}{Tiefpass}
{Hochpass}
{Bandpass}
{Bandsperre}
{\DARCimage{1.0\linewidth}{786include}}\end{PQuestion}

}
\only<2>{
\begin{PQuestion}{AF418}{Welche Funktion trifft für die Spule, $C_2$ und $C_3$ in der Schaltung zu?}{\textbf{\textcolor{DARCgreen}{Tiefpass}}}
{Hochpass}
{Bandpass}
{Bandsperre}
{\DARCimage{1.0\linewidth}{786include}}\end{PQuestion}

}
\end{frame}

\begin{frame}
\only<1>{
\begin{PQuestion}{AF422}{Wozu dienen die mit X gekennzeichneten Spulen in der Schaltung?}{Sie verhindern die Entstehung von Oberschwingungen.}
{Sie verhindern ein Abfließen der Hochfrequenz in die Spannungsversorgung.}
{Sie dienen als Arbeitswiderstand für die Transistoren.}
{Sie transformieren die Ausgangsimpedanz der Transistoren auf \qty{50}{\ohm}.}
{\DARCimage{1.0\linewidth}{782include}}\end{PQuestion}

}
\only<2>{
\begin{PQuestion}{AF422}{Wozu dienen die mit X gekennzeichneten Spulen in der Schaltung?}{Sie verhindern die Entstehung von Oberschwingungen.}
{\textbf{\textcolor{DARCgreen}{Sie verhindern ein Abfließen der Hochfrequenz in die Spannungsversorgung.}}}
{Sie dienen als Arbeitswiderstand für die Transistoren.}
{Sie transformieren die Ausgangsimpedanz der Transistoren auf \qty{50}{\ohm}.}
{\DARCimage{1.0\linewidth}{782include}}\end{PQuestion}

}
\end{frame}

\begin{frame}
\only<1>{
\begin{PQuestion}{AF415}{Weshalb wurden jeweils $C_1$ und $C_2$, $C_3$ und $C_4$ sowie $C_5$ und $C_6$ parallel geschaltet?}{Der Kondensator geringer Kapazität dient jeweils zum Abblocken hoher Frequenzen, der Kondensator hoher Kapazität zum Abblocken niedriger Frequenzen.}
{Die Kapazität nur eines Kondensators reicht bei hohen Frequenzen nicht aus.}
{Der Kondensator mit der geringen Kapazität dient zur Siebung der niedrigen und der Kondensator mit der hohen Kapazität zur Siebung der hohen Frequenzen.}
{Zu einem Elektrolytkondensator muss immer ein keramischer Kondensator parallel geschaltet werden, weil er sonst bei hohen Frequenzen zerstört werden würde.}
{\DARCimage{1.0\linewidth}{766include}}\end{PQuestion}

}
\only<2>{
\begin{PQuestion}{AF415}{Weshalb wurden jeweils $C_1$ und $C_2$, $C_3$ und $C_4$ sowie $C_5$ und $C_6$ parallel geschaltet?}{\textbf{\textcolor{DARCgreen}{Der Kondensator geringer Kapazität dient jeweils zum Abblocken hoher Frequenzen, der Kondensator hoher Kapazität zum Abblocken niedriger Frequenzen.}}}
{Die Kapazität nur eines Kondensators reicht bei hohen Frequenzen nicht aus.}
{Der Kondensator mit der geringen Kapazität dient zur Siebung der niedrigen und der Kondensator mit der hohen Kapazität zur Siebung der hohen Frequenzen.}
{Zu einem Elektrolytkondensator muss immer ein keramischer Kondensator parallel geschaltet werden, weil er sonst bei hohen Frequenzen zerstört werden würde.}
{\DARCimage{1.0\linewidth}{766include}}\end{PQuestion}

}
\end{frame}

\begin{frame}
\only<1>{
\begin{PQuestion}{AF428}{Wie groß ist die Gesamtverstärkung des gesamten Sendezweigs ohne Berücksichtigung möglicher Kabelverluste?}{\qty{38}{\decibel}}
{\qty{48}{\decibel}}
{\qty{43}{\decibel}}
{\qty{59}{\decibel}}
{\DARCimage{1.0\linewidth}{470include}}\end{PQuestion}

}
\only<2>{
\begin{PQuestion}{AF428}{Wie groß ist die Gesamtverstärkung des gesamten Sendezweigs ohne Berücksichtigung möglicher Kabelverluste?}{\qty{38}{\decibel}}
{\textbf{\textcolor{DARCgreen}{\qty{48}{\decibel}}}}
{\qty{43}{\decibel}}
{\qty{59}{\decibel}}
{\DARCimage{1.0\linewidth}{470include}}\end{PQuestion}

}
\end{frame}

\begin{frame}
\frametitle{Lösungsweg}
\begin{itemize}
  \item gegeben: $P_1 = 0,3mW$ oder $-5dBm$
  \item gegeben: $P_2 = 20W$ oder $43dBm$
  \item gesucht: $g$
  \end{itemize}
    \pause
    $g = P_2 -- P_1 = 43dBm -- (-5dBm) = 43dBm + 5dBm = 48dB$
    \pause
    $g = 10 \cdot \log_{10}{(\frac{P_2}{P_1})}dB = 10 \cdot \log_{10}{(\frac{20W}{0,3mW})}dB \approx 48dB$



\end{frame}%ENDCONTENT
