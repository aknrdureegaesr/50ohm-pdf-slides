
\section{Strom- und Spannungsmessung III}
\label{section:strom_spannung_messung_3}
\begin{frame}%STARTCONTENT

\frametitle{Foliensatz in Arbeit}
2024-04-28: Die Inhalte werden noch aufbereitet.

Derzeit sind in diesem Abschnitt nur die Fragen sortiert enthalten.

Für das Selbststudium verweisen wir aktuell auf den Abschnitt Messtechnik im DARC Online Lehrgang (\textcolor{DARCblue}{\faLink~\href{https://www.darc.de/der-club/referate/ajw/lehrgang-ta/a16/}{www.darc.de/der-club/referate/ajw/lehrgang-ta/a16/}}) für die Prüfung bis Juni 2024. Bis auf die Fragen hat sich an der Thematik nichts geändert.

\end{frame}

\begin{frame}
\frametitle{Strom- und Spannungsmessung}
\end{frame}

\begin{frame}
\only<1>{
\begin{PQuestion}{AI101}{Wie sollten Strom- und Spannungsmessgeräte zur Feststellung der Gleichstrom-Eingangsleistung des dargestellten Endverstärkers (PA) angeordnet werden?}{Spannungsmessgerät bei 3, Strommessgerät bei 1.}
{Spannungsmessgerät bei 1, Strommessgerät bei 2.}
{Spannungsmessgerät bei 1, Strommessgerät bei 3.}
{Spannungsmessgerät bei 3, Strommessgerät bei 4.}
{\DARCimage{1.0\linewidth}{443include}}\end{PQuestion}

}
\only<2>{
\begin{PQuestion}{AI101}{Wie sollten Strom- und Spannungsmessgeräte zur Feststellung der Gleichstrom-Eingangsleistung des dargestellten Endverstärkers (PA) angeordnet werden?}{Spannungsmessgerät bei 3, Strommessgerät bei 1.}
{Spannungsmessgerät bei 1, Strommessgerät bei 2.}
{\textbf{\textcolor{DARCgreen}{Spannungsmessgerät bei 1, Strommessgerät bei 3.}}}
{Spannungsmessgerät bei 3, Strommessgerät bei 4.}
{\DARCimage{1.0\linewidth}{443include}}\end{PQuestion}

}
\end{frame}

\begin{frame}
\only<1>{
\begin{PQuestion}{AI102}{Für die Messung der Gleichstrom-Eingangsleistung werden verschiedene Messgeräte verwendet. Bei welchen der Instrumente in der Abbildung handelt es sich um Strommessgeräte?}{2, 3 und 4}
{1, 2 und 3}
{2, 4 und 1}
{1, 3 und 4}
{\DARCimage{1.0\linewidth}{443include}}\end{PQuestion}

}
\only<2>{
\begin{PQuestion}{AI102}{Für die Messung der Gleichstrom-Eingangsleistung werden verschiedene Messgeräte verwendet. Bei welchen der Instrumente in der Abbildung handelt es sich um Strommessgeräte?}{\textbf{\textcolor{DARCgreen}{2, 3 und 4}}}
{1, 2 und 3}
{2, 4 und 1}
{1, 3 und 4}
{\DARCimage{1.0\linewidth}{443include}}\end{PQuestion}

}
\end{frame}

\begin{frame}
\frametitle{Messgenauigkeit}
\end{frame}

\begin{frame}
\only<1>{
\begin{QQuestion}{AI103}{Ein Spannungs- und ein Strommessgerät werden für die Ermittlung der Gleichstromeingangsleistung einer Schaltung verwendet. Das Spannungsmessgerät zeigt \qty{10}{\V}, das Strommessgerät \qty{10}{\A} an. Falls beide dabei im Rahmen ihrer Messgenauigkeit jeweils einen um \qty{5}{\percent} zu geringen Wert anzeigen würden, würde man die elektrische Leistung um~...}{\qty{5}{\percent} zu hoch bestimmen.}
{\qty{5}{\percent} zu niedrig bestimmen.}
{\qty{10,25}{\percent} zu hoch bestimmen.}
{\qty{9,75}{\percent} zu niedrig bestimmen.}
\end{QQuestion}

}
\only<2>{
\begin{QQuestion}{AI103}{Ein Spannungs- und ein Strommessgerät werden für die Ermittlung der Gleichstromeingangsleistung einer Schaltung verwendet. Das Spannungsmessgerät zeigt \qty{10}{\V}, das Strommessgerät \qty{10}{\A} an. Falls beide dabei im Rahmen ihrer Messgenauigkeit jeweils einen um \qty{5}{\percent} zu geringen Wert anzeigen würden, würde man die elektrische Leistung um~...}{\qty{5}{\percent} zu hoch bestimmen.}
{\qty{5}{\percent} zu niedrig bestimmen.}
{\qty{10,25}{\percent} zu hoch bestimmen.}
{\textbf{\textcolor{DARCgreen}{\qty{9,75}{\percent} zu niedrig bestimmen.}}}
\end{QQuestion}

}
\end{frame}

\begin{frame}
\frametitle{Lösungsweg}
\begin{itemize}
  \item Prozentrechnung – die absoluten Werte sind nicht relevant
  \item gegeben: $U_{\textrm{Abw}}$ mit \qty{95}{\percent} vom Realwert
  \item gegeben: $I_{\textrm{Abw}}$ mit \qty{95}{\percent} vom Realwert
  \item gesucht: Abweichung der Leistung $P = U \cdot I$
  \end{itemize}
    \pause
    \begin{equation}\begin{split} \nonumber P_{\textrm{Abw}} &= 100\% -- (U_{\textrm{Abw}} \cdot I_{\textrm{Abw}})\\ &= 100\% -- (95\% \cdot 95\%)\\ &= 100\% -- 90,25\%\\ &= 9,75\% \end{split}\end{equation}



\end{frame}

\begin{frame}
\frametitle{Strom durch Multimeter}
\end{frame}

\begin{frame}
\only<1>{
\begin{QQuestion}{AI104}{Für ein digitales Multimeter ist folgende Angabe im Datenblatt zu finden: Kleinste Auflösung \qty{100}{\micro\V}, Innenwiderstand \qty{10}{\Mohm} in allen Messbereichen. Sie messen eine Spannung von \qty{0,5}{\V}. Welcher Strom fließt dabei durch das Multimeter?}{50~nA}
{10~nA}
{500~nA}
{200~nA}
\end{QQuestion}

}
\only<2>{
\begin{QQuestion}{AI104}{Für ein digitales Multimeter ist folgende Angabe im Datenblatt zu finden: Kleinste Auflösung \qty{100}{\micro\V}, Innenwiderstand \qty{10}{\Mohm} in allen Messbereichen. Sie messen eine Spannung von \qty{0,5}{\V}. Welcher Strom fließt dabei durch das Multimeter?}{\textbf{\textcolor{DARCgreen}{50~nA}}}
{10~nA}
{500~nA}
{200~nA}
\end{QQuestion}

}
\end{frame}

\begin{frame}
\frametitle{Lösungsweg}
\begin{itemize}
  \item gegeben: $U = 0,5V$
  \item gegeben: $R = 10M\Omega$
  \item gesucht: $I$
  \end{itemize}
    \pause
    \begin{equation} \nonumber I = \frac{U}{R} = \frac{0,5V}{10M\Omega} = 50nA \end{equation}



\end{frame}

\begin{frame}
\frametitle{Thermoumformer}
\begin{itemize}
  \item Messgerät, bei dem die abgestrahlte Wärme an einem Widerstand gemessen wird
  \item Aus der abgestrahlten Wärme wird mit einem Thermoelement eine Gleichspannung erzeugt, die gemessen werden kann
  \item Wird dann eingesetzt, wenn eine elektrische Messung nicht möglich ist, z.B. bei nicht-periodischen Signalen
  \end{itemize}
\end{frame}

\begin{frame}
\only<1>{
\begin{QQuestion}{AI105}{Zur genauen Messung der effektiven Leistung eines modulierten Signals bis in den oberen GHz-Bereich eignet sich~...}{ein Digitalmultimeter.}
{ein Oszillograf.}
{ein Messgerät mit Diodentastkopf.}
{ein Messgerät mit Thermoumformer.}
\end{QQuestion}

}
\only<2>{
\begin{QQuestion}{AI105}{Zur genauen Messung der effektiven Leistung eines modulierten Signals bis in den oberen GHz-Bereich eignet sich~...}{ein Digitalmultimeter.}
{ein Oszillograf.}
{ein Messgerät mit Diodentastkopf.}
{\textbf{\textcolor{DARCgreen}{ein Messgerät mit Thermoumformer.}}}
\end{QQuestion}

}
\end{frame}%ENDCONTENT
