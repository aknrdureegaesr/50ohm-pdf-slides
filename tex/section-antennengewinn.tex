
\section{Antennengewinn in dBi und dBd}
\label{section:antennengewinn}
\begin{frame}%STARTCONTENT

\frametitle{Richtwirkung}
\begin{itemize}
  \item \emph{Isotropstrahler}: Hypothetische Antenne, die in alle Richtungen gleich stark abstrahlt
  \item Eine reale Antenne weist eine Richtwirkung auf
  \item In bestimmten Richtungen stärker als der Isotropstrahler
  \item In bestimmten Richtungen schwächer als der Isotropstrahler
  \item Die \emph{Hauptstrahlrichtung} ist die Richtung mit dem maximalen Antennengewinn
  \end{itemize}
\end{frame}

\begin{frame}
\frametitle{Gewinn in dBi}
\begin{itemize}
  \item Gewinn in eine Richtung gegenüber dem Isotropstrahler
  \item Kann in dB angegeben werden
  \item Bei Bezug auf den Isotropstrahler wird \emph{dBi} verwendet
  \end{itemize}
\end{frame}

\begin{frame}
\only<1>{
\begin{QQuestion}{EG220}{Der Gewinn von Antennen wird häufig in dBi angegeben. Auf welche Vergleichsantenne bezieht man sich dabei? Man bezieht sich dabei auf den~...}{Halbwellenstrahler.}
{Isotropstrahler.}
{Horizontalstrahler.}
{Vertikalstrahler.}
\end{QQuestion}

}
\only<2>{
\begin{QQuestion}{EG220}{Der Gewinn von Antennen wird häufig in dBi angegeben. Auf welche Vergleichsantenne bezieht man sich dabei? Man bezieht sich dabei auf den~...}{Halbwellenstrahler.}
{\textbf{\textcolor{DARCgreen}{Isotropstrahler.}}}
{Horizontalstrahler.}
{Vertikalstrahler.}
\end{QQuestion}

}
\end{frame}

\begin{frame}
\frametitle{Gewinn eines Halbwellendipols}
\begin{itemize}
  \item Ein Halbwellendipol strahlt senkrecht zum Leiter um \qty{2,15}{\dB} stärker ab als ein Isotropstrahler
  \item Der Gewinn beträgt \qty{2,15}{\dBi}
  \end{itemize}
\end{frame}

\begin{frame}
\frametitle{Gewinn in dBd}
\begin{itemize}
  \item Bei anderen Antennen ist der Gewinn gegenüber einem Halbwellendipol interessant
  \item Bei Bezug auf den Halbwellendipol wird \emph{dBd} verwendet
  \item Ein Halbwellendipol hat in Hauptstrahlrichtung einen Gewinn von 0 dBd und \qty{2,15}{\dBi}
  \end{itemize}
\end{frame}

\begin{frame}
\only<1>{
\begin{QQuestion}{EG221}{Ein Antennenhersteller gibt den Gewinn einer Antenne mit 5 dBd an. Wie groß ist der Gewinn der Antenne in dBi?}{\qty{2,85}{\dBi}}
{\qty{5}{\dBi}}
{\qty{2,5}{\dBi}}
{\qty{7,15}{\dBi}}
\end{QQuestion}

}
\only<2>{
\begin{QQuestion}{EG221}{Ein Antennenhersteller gibt den Gewinn einer Antenne mit 5 dBd an. Wie groß ist der Gewinn der Antenne in dBi?}{\qty{2,85}{\dBi}}
{\qty{5}{\dBi}}
{\qty{2,5}{\dBi}}
{\textbf{\textcolor{DARCgreen}{\qty{7,15}{\dBi}}}}
\end{QQuestion}

}
\end{frame}%ENDCONTENT
