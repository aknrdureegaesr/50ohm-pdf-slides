
\section{Leiterwiderstand}
\label{section:leiterwiderstand}
\begin{frame}%STARTCONTENT

\frametitle{Foliensatz in Arbeit}
2024-04-28: Die Inhalte werden noch aufbereitet.

Derzeit sind in diesem Abschnitt nur die Fragen sortiert enthalten.

Für das Selbststudium verweisen wir aktuell auf den Abschnitt Messtechnik im DARC Online Lehrgang (\textcolor{DARCblue}{\faLink~\href{https://www.darc.de/der-club/referate/ajw/lehrgang-ta/a16/}{www.darc.de/der-club/referate/ajw/lehrgang-ta/a16/}}) für die Prüfung bis Juni 2024. Bis auf die Fragen hat sich an der Thematik nichts geändert.

\end{frame}

\begin{frame}
\frametitle{Widerstand von Drähten}
\begin{columns}
    \begin{column}{0.48\textwidth}
    $R = \frac{\rho\cdot l}{A_{\textrm{Dr}}}$

\begin{itemize}
  \item $l$: Drahtlänge
  \item $A_{\textrm{Dr}}$: Drahtquerschnitt
  \item $\rho$: Spezifischer Widerstand in Ωmm<sup>2</sup>/m
  \end{itemize}

    \end{column}
   \begin{column}{0.48\textwidth}
       
    \pause
    Kupfer: 0,018

Aluminium: 0,028

Gold: 0,022

Silber: 0,016

Zink: 0,11

Eisen: 0,1

Messing: 0,07




   \end{column}
\end{columns}

\end{frame}

\begin{frame}
\only<1>{
\begin{QQuestion}{AB101}{Welchen Widerstand hat ein Kupferdraht etwa, wenn der verwendete Draht eine Länge von \qty{1,8}{\m} und einen Durchmesser von \qty{0,2}{\mm} hat?}{\qty{0,26}{\ohm}}
{\qty{56,0}{\ohm}}
{\qty{1,02}{\ohm}}
{\qty{0,16}{\ohm}}
\end{QQuestion}

}
\only<2>{
\begin{QQuestion}{AB101}{Welchen Widerstand hat ein Kupferdraht etwa, wenn der verwendete Draht eine Länge von \qty{1,8}{\m} und einen Durchmesser von \qty{0,2}{\mm} hat?}{\qty{0,26}{\ohm}}
{\qty{56,0}{\ohm}}
{\textbf{\textcolor{DARCgreen}{\qty{1,02}{\ohm}}}}
{\qty{0,16}{\ohm}}
\end{QQuestion}

}
\end{frame}

\begin{frame}
\frametitle{Lösungsweg}
\begin{itemize}
  \item gegeben: $l = 1,8m$
  \item gegeben: $d = 0,2mm$
  \item gegeben: $\rho = 0,018 \frac{\Omega mm^2}{m}$
  \item gesucht: $R$
  \end{itemize}
    \pause
    \begin{equation} \nonumber A_{\textrm{Dr}} = \frac{d^2\cdot \pi}{4} = \frac{(0,2mm)^2 \cdot \pi}{4} = \frac{\pi}{100}mm^2 = 0,0314mm^2 \end{equation}
    \pause
    \begin{equation} \nonumber R = \frac{\rho\cdot l}{A_{\textrm{Dr}}} = \frac{0,018 \frac{\Omega mm^2}{m} \cdot 1,8m}{0,0314mm^2} \approx 1,02\Omega \end{equation}



\end{frame}

\begin{frame}
\only<1>{
\begin{QQuestion}{AB102}{Zwischen den Enden eines Kupferdrahtes mit einem Querschnitt von \qty{0,5}{\mm\squared} messen Sie einen Widerstand von \qty{1,5}{\ohm}. Wie lang ist der Draht etwa?}{\qty{41,7}{\m}}
{\qty{3,0}{\m}}
{\qty{4,2}{\m}}
{\qty{16,5}{\m}}
\end{QQuestion}

}
\only<2>{
\begin{QQuestion}{AB102}{Zwischen den Enden eines Kupferdrahtes mit einem Querschnitt von \qty{0,5}{\mm\squared} messen Sie einen Widerstand von \qty{1,5}{\ohm}. Wie lang ist der Draht etwa?}{\textbf{\textcolor{DARCgreen}{\qty{41,7}{\m}}}}
{\qty{3,0}{\m}}
{\qty{4,2}{\m}}
{\qty{16,5}{\m}}
\end{QQuestion}

}
\end{frame}

\begin{frame}
\frametitle{Lösungsweg}
\begin{itemize}
  \item gegeben: $A_{\textrm{Dr}} = 0,5mm^2$
  \item gegeben: $R = 1,5\Omega$
  \item gegeben: $\rho = 0,018 \frac{\Omega mm^2}{m}$
  \item gesucht: $l$
  \end{itemize}
    \pause
    \begin{equation}\begin{align} \nonumber R &= \frac{\rho\cdot l}{A_{\textrm{Dr}}}\\ \nonumber \Rightarrow l &= \frac{R\cdot A_{\textrm{Dr}}}{\rho} = \frac{1,5\Omega \cdot 0,5mm^2}{0,018 \frac{\Omega mm^2}{m}} \approx 41,7m \end{align}\end{equation}



\end{frame}

\begin{frame}
\frametitle{Temperaturkoeffizient}
\begin{itemize}
  \item Widerstand von Metallen steigt bei zunehemender Temperatur
  \end{itemize}
\end{frame}

\begin{frame}
\only<1>{
\begin{QQuestion}{AB103}{Wie ändert sich der Widerstand eines Metalls mit der Temperatur im Regelfall?}{Der Widerstand steigt mit zunehmender Temperatur, d. h. der Temperaturkoeffizient ist positiv.}
{Der Widerstand sinkt mit zunehmender Temperatur, d. h. der Temperaturkoeffizient ist negativ.}
{Der Widerstand ändert sich nicht mit zunehmender Temperatur, d. h. der Temperaturkoeffizient ist Null.}
{Der Widerstand oszilliert mit zunehmender Temperatur, d. h. der Temperaturkoeffizient ist komplex.}
\end{QQuestion}

}
\only<2>{
\begin{QQuestion}{AB103}{Wie ändert sich der Widerstand eines Metalls mit der Temperatur im Regelfall?}{\textbf{\textcolor{DARCgreen}{Der Widerstand steigt mit zunehmender Temperatur, d. h. der Temperaturkoeffizient ist positiv.}}}
{Der Widerstand sinkt mit zunehmender Temperatur, d. h. der Temperaturkoeffizient ist negativ.}
{Der Widerstand ändert sich nicht mit zunehmender Temperatur, d. h. der Temperaturkoeffizient ist Null.}
{Der Widerstand oszilliert mit zunehmender Temperatur, d. h. der Temperaturkoeffizient ist komplex.}
\end{QQuestion}

}
\end{frame}%ENDCONTENT
