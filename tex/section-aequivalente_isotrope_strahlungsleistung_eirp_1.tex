
\section{Äquivalente isotrope Strahlungsleistung (EIRP)}
\label{section:aequivalente_isotrope_strahlungsleistung_eirp_1}
\begin{frame}%STARTCONTENT

\frametitle{Kugelstrahler oder Isotroper Strahler}
\begin{columns}
    \begin{column}{0.48\textwidth}
    
\begin{figure}
    \DARCimage{0.85\linewidth}{751include}
    \caption{\scriptsize Isotroper Strahler in der Mitte einer Kugel, der an allen Stellen der Kugeloberfläche die gleiche Strahlungsleistung erzeugt}
    \label{n_Kugelstrahler}
\end{figure}


    \end{column}
   \begin{column}{0.48\textwidth}
       \begin{itemize}
  \item Theoretische Antenne
  \item Unendlich klein
  \item Funkwellen werden gleichmäßig in alle Richtungen abgestrahlt
  \item Es existiert keine Hauptstrahlrichtung
  \end{itemize}

   \end{column}
\end{columns}

\end{frame}

\begin{frame}
\frametitle{EIRP}
\begin{itemize}
  \item Bei Berechnung der Strahlungsleistung in Bezug zum isotropen Strahler wird von \enquote{äquivalenter isotroper Strahlungsleistung} gesprochen
  \item Englisch \enquote{equivalent isotropic radiated power} (EIRP)
  \end{itemize}
\end{frame}

\begin{frame}
\frametitle{Berechnung}
\begin{itemize}
  \item Erfolgt gleich zu ERP
  \item Hat eine Antennen einen Gewinnfaktor von 3 bezogen auf den isotropen Strahler, dann strahlt diese Antenne in Hauptstrahlrichtung dreimal so stark wie ein isotroper Strahler in jede beliebige Richtung
  \item Bei 5~W Sendeleistung auf Antenne mit Gewinnfaktor 3 gegenüber dem isotropen Strahler ergibt das die Strahlungsleistung 15~W EIRP
  \end{itemize}
\end{frame}

\begin{frame}
\frametitle{Bezug zum Halbwellendipol}
\begin{itemize}
  \item Halbwellendipol hat den Gewinnfaktor 1,64 gegenüber isotropen Strahler
  \item Antenne mit Gewinnfaktor 2 gegenüber Halbwellendipol hat einen Gewinnfaktor von 2 $\cdot$ 1,64 = 3,28 gegenüber isotropen Strahler
  \end{itemize}
\end{frame}

\begin{frame}
\only<1>{
\begin{QQuestion}{NG402}{Die gleichwertige isotrope Strahlungsleistung EIRP (Equivalent Isotropic Radiated Power) ist die von~...}{einem isotropen Strahler abgestrahlte Leistung, bezogen auf eine Antenne.}
{einer Antenne abgestrahlte Leistung, bezogen auf einen Halbwellendipol.}
{einem Halbwellendipol abgestrahlte Leistung, bezogen auf eine Antenne.}
{einer Antenne abgestrahlte Leistung, bezogen auf einen isotropen Strahler.}
\end{QQuestion}

}
\only<2>{
\begin{QQuestion}{NG402}{Die gleichwertige isotrope Strahlungsleistung EIRP (Equivalent Isotropic Radiated Power) ist die von~...}{einem isotropen Strahler abgestrahlte Leistung, bezogen auf eine Antenne.}
{einer Antenne abgestrahlte Leistung, bezogen auf einen Halbwellendipol.}
{einem Halbwellendipol abgestrahlte Leistung, bezogen auf eine Antenne.}
{\textbf{\textcolor{DARCgreen}{einer Antenne abgestrahlte Leistung, bezogen auf einen isotropen Strahler.}}}
\end{QQuestion}

}
\end{frame}%ENDCONTENT
