
\section{Personenschutz bei Richtantennen}
\label{section:personenschutzabstand_richtantennen}
\begin{frame}%STARTCONTENT

\only<1>{
\begin{QQuestion}{AK105}{An der Spitze Ihres Antennenmastes befindet sich eine Yagi-Uda-Antenne, deren Sicherheitsabstand in Hauptstrahlrichtung \qty{20}{\m} beträgt. Schräg unterhalb dieser Antenne befindet sich ein nicht kontrollierbarer Bereich. Sie ermitteln einen kritischen Winkel von \qty{40}{\degree}. Das vertikale Strahlungsdiagramm der Antenne weist bei diesem Winkel eine Dämpfung von \qty{6}{\decibel} auf. Auf welchen Wert verringert sich dann rechnerisch der Sicherheitsabstand bei \qty{40}{\degree}?}{Er verringert sich auf \qty{10}{\m}.}
{Er verringert sich auf \qty{3,33}{\m}.}
{Er verringert sich auf \qty{5,02}{\m}.}
{Er verringert sich nicht.}
\end{QQuestion}

}
\only<2>{
\begin{QQuestion}{AK105}{An der Spitze Ihres Antennenmastes befindet sich eine Yagi-Uda-Antenne, deren Sicherheitsabstand in Hauptstrahlrichtung \qty{20}{\m} beträgt. Schräg unterhalb dieser Antenne befindet sich ein nicht kontrollierbarer Bereich. Sie ermitteln einen kritischen Winkel von \qty{40}{\degree}. Das vertikale Strahlungsdiagramm der Antenne weist bei diesem Winkel eine Dämpfung von \qty{6}{\decibel} auf. Auf welchen Wert verringert sich dann rechnerisch der Sicherheitsabstand bei \qty{40}{\degree}?}{\textbf{\textcolor{DARCgreen}{Er verringert sich auf \qty{10}{\m}.}}}
{Er verringert sich auf \qty{3,33}{\m}.}
{Er verringert sich auf \qty{5,02}{\m}.}
{Er verringert sich nicht.}
\end{QQuestion}

}
\end{frame}

\begin{frame}
\frametitle{Lösungsweg}
\end{frame}%ENDCONTENT
