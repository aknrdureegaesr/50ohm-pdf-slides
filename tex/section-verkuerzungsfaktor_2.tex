
\section{Verkürzungsfaktor II}
\label{section:verkuerzungsfaktor_2}
\begin{frame}%STARTCONTENT

\only<1>{
\begin{QQuestion}{AG202}{Warum muss eine Antenne mechanisch etwas kürzer als der theoretisch errechnete Wert sein?}{Weil sich durch die mechanische Verkürzung die elektromagnetischen Wellen leichter von der Antenne ablösen. Dadurch steigt der Wirkungsgrad.}
{Weil sich diese Antenne nicht im idealen freien Raum befindet und weil die Antennenelemente nicht die Idealform des Kugelstrahlers besitzen. Kapazitive Einflüsse der Umgebung und die Abweichung von der idealen Kugelform verlängern die Antenne elektrisch.}
{Weil sich diese Antenne nicht im idealen freien Raum befindet und weil sie nicht unendlich dünn ist. Kapazitive Einflüsse der Umgebung und der Durchmesser des Strahlers verlängern die Antenne elektrisch.}
{Weil sich durch die mechanische Verkürzung der Verlustwiderstand eines Antennenstabes verringert. Dadurch steigt der Wirkungsgrad.}
\end{QQuestion}

}
\only<2>{
\begin{QQuestion}{AG202}{Warum muss eine Antenne mechanisch etwas kürzer als der theoretisch errechnete Wert sein?}{Weil sich durch die mechanische Verkürzung die elektromagnetischen Wellen leichter von der Antenne ablösen. Dadurch steigt der Wirkungsgrad.}
{Weil sich diese Antenne nicht im idealen freien Raum befindet und weil die Antennenelemente nicht die Idealform des Kugelstrahlers besitzen. Kapazitive Einflüsse der Umgebung und die Abweichung von der idealen Kugelform verlängern die Antenne elektrisch.}
{\textbf{\textcolor{DARCgreen}{Weil sich diese Antenne nicht im idealen freien Raum befindet und weil sie nicht unendlich dünn ist. Kapazitive Einflüsse der Umgebung und der Durchmesser des Strahlers verlängern die Antenne elektrisch.}}}
{Weil sich durch die mechanische Verkürzung der Verlustwiderstand eines Antennenstabes verringert. Dadurch steigt der Wirkungsgrad.}
\end{QQuestion}

}
\end{frame}

\begin{frame}
\only<1>{
\begin{QQuestion}{AG313}{Der Verkürzungsfaktor einer luftisolierten Paralleldrahtleitung ist~...}{unbestimmt.}
{0{,}1.}
{0{,}66.}
{ungefähr~1.}
\end{QQuestion}

}
\only<2>{
\begin{QQuestion}{AG313}{Der Verkürzungsfaktor einer luftisolierten Paralleldrahtleitung ist~...}{unbestimmt.}
{0{,}1.}
{0{,}66.}
{\textbf{\textcolor{DARCgreen}{ungefähr~1.}}}
\end{QQuestion}

}
\end{frame}

\begin{frame}
\only<1>{
\begin{QQuestion}{AG315}{Der Verkürzungsfaktor eines Koaxialkabels mit einem Dielektrikum aus massivem Polyethylen beträgt ungefähr~...}{1{,}0.}
{0{,}1.}
{0{,}8.}
{0{,}66.}
\end{QQuestion}

}
\only<2>{
\begin{QQuestion}{AG315}{Der Verkürzungsfaktor eines Koaxialkabels mit einem Dielektrikum aus massivem Polyethylen beträgt ungefähr~...}{1{,}0.}
{0{,}1.}
{0{,}8.}
{\textbf{\textcolor{DARCgreen}{0{,}66.}}}
\end{QQuestion}

}
\end{frame}

\begin{frame}
\only<1>{
\begin{QQuestion}{AG101}{Eine $\lambda$/2-Dipol-Antenne soll für \qty{14,2}{\MHz} aus Draht gefertigt werden. Es soll mit einem Verkürzungsfaktor von \num{0,95} gerechnet werden. Wie lang müssen die beiden Drähte der Dipol-Antenne jeweils sein?}{Je \qty{10,03}{\m}}
{Je \qty{10,56}{\m}}
{Je \qty{5,02}{\m}}
{Je \qty{5,28}{\m}}
\end{QQuestion}

}
\only<2>{
\begin{QQuestion}{AG101}{Eine $\lambda$/2-Dipol-Antenne soll für \qty{14,2}{\MHz} aus Draht gefertigt werden. Es soll mit einem Verkürzungsfaktor von \num{0,95} gerechnet werden. Wie lang müssen die beiden Drähte der Dipol-Antenne jeweils sein?}{Je \qty{10,03}{\m}}
{Je \qty{10,56}{\m}}
{\textbf{\textcolor{DARCgreen}{Je \qty{5,02}{\m}}}}
{Je \qty{5,28}{\m}}
\end{QQuestion}

}
\end{frame}

\begin{frame}
\frametitle{Lösungsweg}
\begin{itemize}
  \item gegeben: $f = 14,2MHz$
  \item gegeben: $k_v = 0,95$
  \item gegeben: $\frac{\lambda}{2}$-Dipol
  \item gesucht: $l_G$
  \end{itemize}
    \pause
    $l_E = \frac{1}{2} \cdot \frac{\lambda}{2} = \frac{1}{4} \cdot \frac{c}{f} = \frac{1}{4} \cdot \frac{3\cdot 10^8\frac{m}{s}}{14,2MHz} = \frac{1}{4} \cdot 21,13m = 5,28m$
    \pause
    $k_v = \frac{l_G}{l_E} \Rightarrow l_G = k_v \cdot l_E = 0,95 \cdot 5,28m = 5,02m$



\end{frame}

\begin{frame}
\only<1>{
\begin{QQuestion}{AG102}{Eine $\lambda$/2-Dipol-Antenne soll für \qty{7,1}{\MHz} aus Draht gefertigt werden. Wie lang müssen die beiden Drähte der Dipol-Antenne jeweils sein? Es soll hier mit einem Verkürzungsfaktor von \num{0,95} gerechnet werden.}{Je \qty{21,13}{\m}}
{Je \qty{10,56}{\m}}
{Je \qty{20,07}{\m}}
{Je \qty{10,04}{\m}}
\end{QQuestion}

}
\only<2>{
\begin{QQuestion}{AG102}{Eine $\lambda$/2-Dipol-Antenne soll für \qty{7,1}{\MHz} aus Draht gefertigt werden. Wie lang müssen die beiden Drähte der Dipol-Antenne jeweils sein? Es soll hier mit einem Verkürzungsfaktor von \num{0,95} gerechnet werden.}{Je \qty{21,13}{\m}}
{Je \qty{10,56}{\m}}
{Je \qty{20,07}{\m}}
{\textbf{\textcolor{DARCgreen}{Je \qty{10,04}{\m}}}}
\end{QQuestion}

}
\end{frame}

\begin{frame}
\frametitle{Lösungsweg}
\begin{itemize}
  \item gegeben: $f = 7,1MHz$
  \item gegeben: $k_v = 0,95$
  \item gegeben: $\frac{\lambda}{2}$-Dipol
  \item gesucht: $l_G$
  \end{itemize}
    \pause
    $l_E = \frac{1}{2} \cdot \frac{\lambda}{2} = \frac{1}{4} \cdot \frac{c}{f} = \frac{1}{4} \cdot \frac{3\cdot 10^8\frac{m}{s}}{7,1MHz} = \frac{1}{4} \cdot 42,25m = 10,56m$
    \pause
    $k_v = \frac{l_G}{l_E} \Rightarrow l_G = k_v \cdot l_E = 0,95 \cdot 10,56m = 10,04m$



\end{frame}

\begin{frame}
\only<1>{
\begin{QQuestion}{AG103}{Ein Drahtdipol hat eine Gesamtlänge von \qty{20}{\m}. Für welche Frequenz ist der Dipol in Resonanz, wenn mit einem Verkürzungsfaktor von \num{0,95} gerechnet wird?}{\qty{7,125}{\MHz}}
{\qty{6,768}{\MHz}}
{\qty{7,500}{\MHz}}
{\qty{7,000}{\MHz}}
\end{QQuestion}

}
\only<2>{
\begin{QQuestion}{AG103}{Ein Drahtdipol hat eine Gesamtlänge von \qty{20}{\m}. Für welche Frequenz ist der Dipol in Resonanz, wenn mit einem Verkürzungsfaktor von \num{0,95} gerechnet wird?}{\textbf{\textcolor{DARCgreen}{\qty{7,125}{\MHz}}}}
{\qty{6,768}{\MHz}}
{\qty{7,500}{\MHz}}
{\qty{7,000}{\MHz}}
\end{QQuestion}

}
\end{frame}

\begin{frame}
\frametitle{Lösungsweg}
\begin{itemize}
  \item gegeben: $l_G = 20m$
  \item gegeben: $k_v = 0,95$
  \item gegeben: Dipol
  \item gesucht: $f$
  \end{itemize}
    \pause
    $k_v = \frac{l_G}{l_E} \Rightarrow l_E = \frac{l_G}{k_v} = \frac{20m}{0,95} = 21,05m$
    \pause
    $l_E = \frac{\lambda}{2} = \frac{1}{2} \cdot \frac{c}{f} \Rightarrow f = \frac{1}{2} \cdot {c}{l_E} = 7,125MHz$



\end{frame}

\begin{frame}
\only<1>{
\begin{QQuestion}{AG104}{Eine $\lambda$/4-Groundplane-Antenne mit vier Radials soll für \qty{7,1}{\MHz} aus Drähten gefertigt werden. Für Strahlerelement und Radials kann mit einem Verkürzungsfaktor von \num{0,95} gerechnet werden. Wie lang müssen Strahlerelement und Radials jeweils sein?}{Strahlerelement:~\qty{20,06}{\m}, Radials:~je~\qty{20,06}{\m}}
{Strahlerelement:~\qty{21,13}{\m}, Radials:~je~\qty{21,13}{\m}}
{Strahlerelement:~\qty{10,56}{\m}, Radials:~je~\qty{10,56}{\m}}
{Strahlerelement:~\qty{10,04}{\m}, Radials:~je~\qty{10,04}{\m}}
\end{QQuestion}

}
\only<2>{
\begin{QQuestion}{AG104}{Eine $\lambda$/4-Groundplane-Antenne mit vier Radials soll für \qty{7,1}{\MHz} aus Drähten gefertigt werden. Für Strahlerelement und Radials kann mit einem Verkürzungsfaktor von \num{0,95} gerechnet werden. Wie lang müssen Strahlerelement und Radials jeweils sein?}{Strahlerelement:~\qty{20,06}{\m}, Radials:~je~\qty{20,06}{\m}}
{Strahlerelement:~\qty{21,13}{\m}, Radials:~je~\qty{21,13}{\m}}
{Strahlerelement:~\qty{10,56}{\m}, Radials:~je~\qty{10,56}{\m}}
{\textbf{\textcolor{DARCgreen}{Strahlerelement:~\qty{10,04}{\m}, Radials:~je~\qty{10,04}{\m}}}}
\end{QQuestion}

}
\end{frame}

\begin{frame}
\frametitle{Lösungsweg}
\begin{itemize}
  \item gegeben: $f = 7,1MHz$
  \item gegeben: $k_v = 0,95$
  \item gegeben: $\frac{\lambda}{4}$-Groundplane
  \item gesucht: $l_G$
  \end{itemize}
    \pause
    $l_E = \frac{\lambda}{4} = \frac{1}{4} \cdot \frac{c}{f} = \frac{1}{4} \cdot \frac{3\cdot 10^8\frac{m}{s}}{7,1MHz} = \frac{1}{4} \cdot 42,25m = 10,56m$
    \pause
    $k_v = \frac{l_G}{l_E} \Rightarrow l_G = k_v \cdot l_E = 0,95 \cdot 10,56m = 10,04m$



\end{frame}

\begin{frame}
\only<1>{
\begin{QQuestion}{AG105}{Eine 5/8-$\lambda$-Vertikalantenne soll für \qty{14,2}{\MHz} aus Draht hergestellt werden. Es soll mit einem Verkürzungsfaktor von \num{0,97} gerechnet werden. Wie lang muss der Draht insgesamt sein?}{\qty{10,03}{\m}}
{\qty{13,20}{\m}}
{\qty{12,80}{\m}}
{\qty{13,61}{\m}}
\end{QQuestion}

}
\only<2>{
\begin{QQuestion}{AG105}{Eine 5/8-$\lambda$-Vertikalantenne soll für \qty{14,2}{\MHz} aus Draht hergestellt werden. Es soll mit einem Verkürzungsfaktor von \num{0,97} gerechnet werden. Wie lang muss der Draht insgesamt sein?}{\qty{10,03}{\m}}
{\qty{13,20}{\m}}
{\textbf{\textcolor{DARCgreen}{\qty{12,80}{\m}}}}
{\qty{13,61}{\m}}
\end{QQuestion}

}
\end{frame}

\begin{frame}
\frametitle{Lösungsweg}
\begin{itemize}
  \item gegeben: $f = 14,2MHz$
  \item gegeben: $k_v = 0,97$
  \item gegeben: $\frac{5}{8}\lambda$-Vertikalantenne
  \item gesucht: $l_G$
  \end{itemize}
    \pause
    $l_E = \frac{5}{8}\lambda = \frac{5}{8} \cdot \frac{c}{f} = \frac{5}{8} \cdot \frac{3\cdot 10^8\frac{m}{s}}{14,2MHz} = \frac{5}{8} \cdot 21,13 = 13,20m$
    \pause
    $k_v = \frac{l_G}{l_E} \Rightarrow l_G = k_v \cdot l_E = 0,97 \cdot 13,20m = 12,80m$



\end{frame}

\begin{frame}
\only<1>{
\begin{QQuestion}{AG118}{Eine Delta-Loop-Antenne mit einer vollen Wellenlänge soll für \qty{7,1}{\MHz} aus Draht hergestellt werden. Es soll mit einem Korrekturfaktor von \num{1,02} gerechnet werden. Wie lang muss der Draht insgesamt sein?}{\qty{21,12}{\m}}
{\qty{42,25}{\m}}
{\qty{21,55}{\m}}
{\qty{43,10}{\m}}
\end{QQuestion}

}
\only<2>{
\begin{QQuestion}{AG118}{Eine Delta-Loop-Antenne mit einer vollen Wellenlänge soll für \qty{7,1}{\MHz} aus Draht hergestellt werden. Es soll mit einem Korrekturfaktor von \num{1,02} gerechnet werden. Wie lang muss der Draht insgesamt sein?}{\qty{21,12}{\m}}
{\qty{42,25}{\m}}
{\qty{21,55}{\m}}
{\textbf{\textcolor{DARCgreen}{\qty{43,10}{\m}}}}
\end{QQuestion}

}
\end{frame}

\begin{frame}
\frametitle{Lösungsweg}
\begin{itemize}
  \item gegeben: $f = 7,1MHz$
  \item gegeben: $k_v = 1,02$
  \item gegeben: Delta-Loop
  \item gesucht: $l_G$
  \end{itemize}
    \pause
    $l_E = \lambda = \frac{c}{f} = \frac{3\cdot 10^8\frac{m}{s}}{7,1MHz} = 42,23m$
    \pause
    $k_v = \frac{l_G}{l_E} \Rightarrow l_G = k_v \cdot l_E = 1,02 \cdot 42,23m = 43,10m$



\end{frame}

\begin{frame}
\only<1>{
\begin{QQuestion}{AG316}{Wie lang ist ein Koaxialkabel, das für eine ganze Wellenlänge bei \qty{145}{\MHz} zugeschnitten wurde, wenn der Verkürzungsfaktor 0,66 beträgt?}{\qty{0,68}{\m}}
{\qty{2,07}{\m}}
{\qty{1,37}{\m}}
{\qty{2,72}{\m}}
\end{QQuestion}

}
\only<2>{
\begin{QQuestion}{AG316}{Wie lang ist ein Koaxialkabel, das für eine ganze Wellenlänge bei \qty{145}{\MHz} zugeschnitten wurde, wenn der Verkürzungsfaktor 0,66 beträgt?}{\qty{0,68}{\m}}
{\qty{2,07}{\m}}
{\textbf{\textcolor{DARCgreen}{\qty{1,37}{\m}}}}
{\qty{2,72}{\m}}
\end{QQuestion}

}
\end{frame}

\begin{frame}
\frametitle{Lösungsweg}
\begin{itemize}
  \item gegeben: $f = 145MHz$
  \item gegeben: $k_v = 0,66$
  \item gesucht: $l_G$
  \end{itemize}
    \pause
    $l_E = \lambda = \frac{c}{f} = \frac{3\cdot 10^8\frac{m}{s}}{144MHz} = 2,08m$
    \pause
    $k_v = \frac{l_G}{l_E} \Rightarrow l_G = k_v \cdot l_E = 0,66 \cdot 2,08m = 1,37m$



\end{frame}%ENDCONTENT
