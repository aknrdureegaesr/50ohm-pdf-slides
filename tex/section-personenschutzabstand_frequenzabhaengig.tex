
\section{Frequenzabhängigkeit des Personenschutzabstands}
\label{section:personenschutzabstand_frequenzabhaengig}
\begin{frame}%STARTCONTENT
\begin{itemize}
  \item Der menschliche Körper kann hochfrequente Strahlung absorbieren (\textcolor{DARCblue}{\faLink~\href{https://50ohm.de/bfs}{50ohm.de/bfs}})
  \item Die Strahlung wird dabei in Wärme umgewandelt
  \item Thermoregulation des Körpers schafft begrenzt einen Ausgleich
  \end{itemize}
\end{frame}

\begin{frame}
\begin{columns}
    \begin{column}{0.48\textwidth}
    Eindringtiefe der Strahlung:

\begin{itemize}
  \item MHz ca. \qtyrange{10}{30}{\centi\metre}
  \item GHz wenige cm
  \item $>$\qty{10}{\giga\hertz} ca. $<$ \qty{1}{\milli\metre}
  \end{itemize}

    \end{column}
   \begin{column}{0.48\textwidth}
       \begin{itemize}
  \item Resonanz bei $\textrm{Körpergröße} \approx \frac{\lambda}{2}$
  \item Hohe Aufnahme von Strahlungsenergie bei Resonanz
  \item Deshalb sind die \emph{Feldstärkegrenzwerte} für den Schutz von Personen in elektromagnetischen Feldern \emph{von der Frequenz abhängig}
  \end{itemize}

   \end{column}
\end{columns}

\end{frame}

\begin{frame}
\only<1>{
\begin{QQuestion}{EK101}{Die Feldstärkegrenzwerte für den Schutz von Personen in elektromagnetischen Feldern sind von der Frequenz abhängig, weil~...}{auf den Amateurfunkbändern unterschiedlich hohe Sendeleistungen zugelassen sind.}
{niederfrequente elektromagnetische Felder energiereicher sind als hochfrequente.}
{die Fähigkeit des Körpers, hochfrequente Strahlung zu absorbieren, frequenzabhängig ist.}
{die spezifische Absorptionsrate bei einigen Frequenzen nicht messbar ist.}
\end{QQuestion}

}
\only<2>{
\begin{QQuestion}{EK101}{Die Feldstärkegrenzwerte für den Schutz von Personen in elektromagnetischen Feldern sind von der Frequenz abhängig, weil~...}{auf den Amateurfunkbändern unterschiedlich hohe Sendeleistungen zugelassen sind.}
{niederfrequente elektromagnetische Felder energiereicher sind als hochfrequente.}
{\textbf{\textcolor{DARCgreen}{die Fähigkeit des Körpers, hochfrequente Strahlung zu absorbieren, frequenzabhängig ist.}}}
{die spezifische Absorptionsrate bei einigen Frequenzen nicht messbar ist.}
\end{QQuestion}

}
\end{frame}%ENDCONTENT
