
\section{Senderausgangsleistung}
\label{section:senderausgangsleistung}
\begin{frame}%STARTCONTENT
\begin{itemize}
  \item Verpflichtung von Funkamateuren die Leistungsgrenzwerte ihrer Funkanlage einzuhalten
  \item Auf vielen Amateurfunkbändern gilt eine \emph{maximale Senderausgangsleistung} (\emph{PEP}, Peak-Envelope-Power) als Grenzwert
  \item Auch unerwünschte Aussendungen sind von Bedeutung
  \end{itemize}
\end{frame}

\begin{frame}
\frametitle{Messung von unerwünschten Aussendungen}
\begin{itemize}
  \item Am Senderausgang
  \item Unter Einzebiehung von Stehwellenmessgerät, Anpassgerät(e), Tiefpassfilter etc.
  \item Messsung von unerwünschen Aussendungen, die die Antenne erreichen können
  \end{itemize}
\end{frame}

\begin{frame}
\only<1>{
\begin{QQuestion}{EJ209}{Wie erfolgt die Messung der Leistungen, die zu unerwünschten Aussendungen führen?}{Die Messung erfolgt am Senderausgang mit einem hochohmigen HF-Tastkopf und angeschlossenem Transistorvoltmeter.}
{Die Messung erfolgt am Fußpunkt der im Funkbetrieb verwendeten Antenne unter Einbeziehung des gegebenenfalls verwendeten Antennenanpassgeräts.}
{Die Messung erfolgt am Ausgang der Antennenleitung unter Einbeziehung des im Funkbetrieb verwendeten Antennenanpassgeräts.}
{Die Messung erfolgt am Senderausgang unter Einbeziehung des gegebenenfalls verwendeten Stehwellenmessgeräts und des gegebenenfalls verwendeten Tiefpassfilters.}
\end{QQuestion}

}
\only<2>{
\begin{QQuestion}{EJ209}{Wie erfolgt die Messung der Leistungen, die zu unerwünschten Aussendungen führen?}{Die Messung erfolgt am Senderausgang mit einem hochohmigen HF-Tastkopf und angeschlossenem Transistorvoltmeter.}
{Die Messung erfolgt am Fußpunkt der im Funkbetrieb verwendeten Antenne unter Einbeziehung des gegebenenfalls verwendeten Antennenanpassgeräts.}
{Die Messung erfolgt am Ausgang der Antennenleitung unter Einbeziehung des im Funkbetrieb verwendeten Antennenanpassgeräts.}
{\textbf{\textcolor{DARCgreen}{Die Messung erfolgt am Senderausgang unter Einbeziehung des gegebenenfalls verwendeten Stehwellenmessgeräts und des gegebenenfalls verwendeten Tiefpassfilters.}}}
\end{QQuestion}

}
\end{frame}

\begin{frame}
\frametitle{Messung der Senderausgangsleistung}
\begin{itemize}
  \item Direkt am Senderausgang
  \item Ohne Zusatzgeräte, Filter oder Kabel
  \item Bei SSB $\rightarrow$ mit Modulation
  \item Ein- oder Zweitonaussteuerung, aber keine Sprache
  \item Messung der maximalen \emph{Hüllkurvenleistung} (PEP)
  \item Spitzenleistung des Senders bei maximaler Aussteuerung
  \end{itemize}

\end{frame}

\begin{frame}
\only<1>{
\begin{QQuestion}{EF401}{Die Ausgangsleistung eines Senders ist die unmittelbar nach~...}{dem Senderausgang gemessene Summe aus vorlaufender und rücklaufender Leistung.}
{dem Senderausgang gemessene Differenz aus vorlaufender und rücklaufender Leistung.}
{der Antenne messbaren Leistung, die durch ein Feldstärkenmessgerät im Nahfeld ermittelt werden kann.}
{dem Senderausgang messbare Leistung, bevor sie Zusatzgeräte durchläuft.}
\end{QQuestion}

}
\only<2>{
\begin{QQuestion}{EF401}{Die Ausgangsleistung eines Senders ist die unmittelbar nach~...}{dem Senderausgang gemessene Summe aus vorlaufender und rücklaufender Leistung.}
{dem Senderausgang gemessene Differenz aus vorlaufender und rücklaufender Leistung.}
{der Antenne messbaren Leistung, die durch ein Feldstärkenmessgerät im Nahfeld ermittelt werden kann.}
{\textbf{\textcolor{DARCgreen}{dem Senderausgang messbare Leistung, bevor sie Zusatzgeräte durchläuft.}}}
\end{QQuestion}

}
\end{frame}

\begin{frame}
\only<1>{
\begin{QQuestion}{EF402}{Wie und wo wird die Ausgangsleistung eines SSB-Senders gemessen? Die maximale Hüllkurvenleistung (PEP) wird gemessen...}{zwischen Antennentuner und Speisepunkt bei Sprachmodulation.}
{zwischen Antennentuner und Speisepunkt der Antenne mit unmoduliertem Träger.}
{direkt am Senderausgang bei Ein- oder Zweitonaussteuerung.}
{direkt am Senderausgang mit unmoduliertem Träger.}
\end{QQuestion}

}
\only<2>{
\begin{QQuestion}{EF402}{Wie und wo wird die Ausgangsleistung eines SSB-Senders gemessen? Die maximale Hüllkurvenleistung (PEP) wird gemessen...}{zwischen Antennentuner und Speisepunkt bei Sprachmodulation.}
{zwischen Antennentuner und Speisepunkt der Antenne mit unmoduliertem Träger.}
{\textbf{\textcolor{DARCgreen}{direkt am Senderausgang bei Ein- oder Zweitonaussteuerung.}}}
{direkt am Senderausgang mit unmoduliertem Träger.}
\end{QQuestion}

}
\end{frame}%ENDCONTENT
