
\section{Stehwellenmessgerät (SWR-Meter) II}
\label{section:swr_meter_2}
\begin{frame}%STARTCONTENT

\only<1>{
\begin{QQuestion}{AI401}{Ein Stehwellenmessgerät misst und vergleicht bei einer HF-Leitung im Sendebetrieb~...}{die Maximalleistung $P_{\symup{max}}$ am Richtkoppler und die Minimalspannung $U_{\symup{min}}$ auf der Leitung.}
{mittels der eingebauten Richtkoppler die vorhandenen Impedanzen in Vor- und Rückrichtung der Leitung.}
{den Phasenwinkel zwischen vorlaufender und rücklaufender Leistung am eingebauten Abschlusswiderstand der Richtkoppler.}
{die Ausgangsspannungen zweier in die Leitung eingeschleifter Richtkoppler, die in gegensätzlicher Richtung betrieben werden.  }
\end{QQuestion}

}
\only<2>{
\begin{QQuestion}{AI401}{Ein Stehwellenmessgerät misst und vergleicht bei einer HF-Leitung im Sendebetrieb~...}{die Maximalleistung $P_{\symup{max}}$ am Richtkoppler und die Minimalspannung $U_{\symup{min}}$ auf der Leitung.}
{mittels der eingebauten Richtkoppler die vorhandenen Impedanzen in Vor- und Rückrichtung der Leitung.}
{den Phasenwinkel zwischen vorlaufender und rücklaufender Leistung am eingebauten Abschlusswiderstand der Richtkoppler.}
{\textbf{\textcolor{DARCgreen}{die Ausgangsspannungen zweier in die Leitung eingeschleifter Richtkoppler, die in gegensätzlicher Richtung betrieben werden.  }}}
\end{QQuestion}

}
\end{frame}

\begin{frame}
\only<1>{
\begin{PQuestion}{AI402}{Bei dieser Schaltung handelt es sich um~...}{einen Absolutleistungsmesser.}
{ein Impedanzmessgerät.}
{ein Stehwellenmessgerät.}
{einen Absorptionsfrequenzmesser.}
{\DARCimage{1.0\linewidth}{499include}}\end{PQuestion}

}
\only<2>{
\begin{PQuestion}{AI402}{Bei dieser Schaltung handelt es sich um~...}{einen Absolutleistungsmesser.}
{ein Impedanzmessgerät.}
{\textbf{\textcolor{DARCgreen}{ein Stehwellenmessgerät.}}}
{einen Absorptionsfrequenzmesser.}
{\DARCimage{1.0\linewidth}{499include}}\end{PQuestion}

}
\end{frame}

\begin{frame}
\only<1>{
\begin{QQuestion}{AI403}{Zur Überprüfung eines Stehwellenmessgerätes wird dessen Ausgang mit einem HF-geeigneten \qty{150}{\ohm}-Lastwiderstand abgeschlossen. Welches Stehwellenverhältnis muss das Messgerät anzeigen, wenn die Impedanz von Messgerät und Sender \qty{50}{\ohm} beträgt?}{\num{2,5}}
{\num{3}}
{\num{3,33}}
{\num{2}}
\end{QQuestion}

}
\only<2>{
\begin{QQuestion}{AI403}{Zur Überprüfung eines Stehwellenmessgerätes wird dessen Ausgang mit einem HF-geeigneten \qty{150}{\ohm}-Lastwiderstand abgeschlossen. Welches Stehwellenverhältnis muss das Messgerät anzeigen, wenn die Impedanz von Messgerät und Sender \qty{50}{\ohm} beträgt?}{\num{2,5}}
{\textbf{\textcolor{DARCgreen}{\num{3}}}}
{\num{3,33}}
{\num{2}}
\end{QQuestion}

}
\end{frame}

\begin{frame}
\frametitle{Lösungsweg}
\begin{itemize}
  \item gegeben: $R_2 = 150Ω$
  \item gegeben: $Z = 50Ω$
  \item gesucht: $s$
  \end{itemize}
    \pause
    $s = \frac{R_2}{Z} = \frac{150Ω}{50Ω} = 3$



\end{frame}%ENDCONTENT
