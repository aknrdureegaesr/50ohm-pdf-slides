
\section{Übersteuerung}
\label{section:digimod_uebersteuerung}
\begin{frame}%STARTCONTENT

\begin{columns}
    \begin{column}{0.48\textwidth}
    \begin{itemize}
  \item Zu starkes Audiosignal am Eingang eines Senders $\rightarrow$ Oberschwingungen
  \item Links ist in Gelb das erwünschte Signal
  \item Rechts davon die unerwünschten Oberschwingungen
  \end{itemize}

    \end{column}
   \begin{column}{0.48\textwidth}
       
\begin{figure}
    \DARCimage{0.85\linewidth}{720include}
    \caption{\scriptsize Ein übersteuertes FT8-Signal, ganz links das erwünschte, rechts davon die unerwünschten Oberschwingungen}
    \label{uebersteuerung_ft8}
\end{figure}


   \end{column}
\end{columns}

\end{frame}

\begin{frame}\begin{itemize}
  \item Zu Verzerrungen durch Übersteuerung kann es auch im Sendeverstärker kommen
  \item Um das zu verhindern, verfügen viele Funkgeräte über eine automatische Pegelregelung (englisch: Automatic Level Control, ALC) $\rightarrow$ regelt Verstärkung automatisch runter
  \item Bei digitalen Übertragungsverfahren kann die ALC jedoch Problemen führen
  \item Das Signal könnte je nach Lautstärke oder Frequenz die ALC zu verschiedenen Zeitpunkten unterschiedlich stark auslösen $\rightarrow$ Amplitude wird unerwünscht verändert
  \end{itemize}

\end{frame}

\begin{frame}\begin{itemize}
  \item ALC-Probleme hängen von verschiedenen Faktoren ab
  \item Übertragungsverfahren
  \item Umsetzung der ALC im Transceiver (Reaktions- und Haltezeit)
  \item Anzeige der ALC im Transceiver
  \item $\rightarrow$ greift die ALC nicht ein, erzeugt sie keine Probleme
  \end{itemize}
\end{frame}

\begin{frame}
\only<1>{
\begin{QQuestion}{EJ218}{Wie sollte bei digitalen Übertragungsverfahren (z. B. FT8, JS8, PSK31) der NF-Pegel am Eingang eines Funkgerätes mit automatischer Pegelregelung (ALC) im SSB-Betrieb eingestellt sein, um Störungen zu vermeiden?}{\qty{18}{\decibel} höher als die Lautstärke, bei der die automatische Pegelregelung (ALC) eingreift.}
{So niedrig, dass die automatische Pegelregelung (ALC) nicht eingreift.}
{Alle Bedienelemente sind auf das Maximum einzustellen.}
{Die NF-Lautstärke muss $-\infty$~dB (also Null) betragen.}
\end{QQuestion}

}
\only<2>{
\begin{QQuestion}{EJ218}{Wie sollte bei digitalen Übertragungsverfahren (z. B. FT8, JS8, PSK31) der NF-Pegel am Eingang eines Funkgerätes mit automatischer Pegelregelung (ALC) im SSB-Betrieb eingestellt sein, um Störungen zu vermeiden?}{\qty{18}{\decibel} höher als die Lautstärke, bei der die automatische Pegelregelung (ALC) eingreift.}
{\textbf{\textcolor{DARCgreen}{So niedrig, dass die automatische Pegelregelung (ALC) nicht eingreift.}}}
{Alle Bedienelemente sind auf das Maximum einzustellen.}
{Die NF-Lautstärke muss $-\infty$~dB (also Null) betragen.}
\end{QQuestion}

}
\end{frame}

\begin{frame}
\only<1>{
\begin{QQuestion}{EJ217}{Was kann auftreten, wenn bei digitalen Übertragungsverfahren (z. B. RTTY, FT8, Olivia) die automatische Pegelregelung (ALC) eines Funkgerätes im SSB-Betrieb eingreift?}{Störungen von Stationen auf anderen Frequenzbändern}
{Störungen von Computern oder anderen digitalen Geräten}
{Störungen von Übertragungen auf Nachbarfrequenzen}
{Störungen von nachfolgenden Sendungen auf derselben Frequenz}
\end{QQuestion}

}
\only<2>{
\begin{QQuestion}{EJ217}{Was kann auftreten, wenn bei digitalen Übertragungsverfahren (z. B. RTTY, FT8, Olivia) die automatische Pegelregelung (ALC) eines Funkgerätes im SSB-Betrieb eingreift?}{Störungen von Stationen auf anderen Frequenzbändern}
{Störungen von Computern oder anderen digitalen Geräten}
{\textbf{\textcolor{DARCgreen}{Störungen von Übertragungen auf Nachbarfrequenzen}}}
{Störungen von nachfolgenden Sendungen auf derselben Frequenz}
\end{QQuestion}

}
\end{frame}

\begin{frame}
\only<1>{
\begin{QQuestion}{EJ219}{Was ist zu tun, wenn es bei digitalen Übertragungsverfahren zu Störungen kommt, weil die automatische Pegelregelung (ALC) eines Funkgerätes im SSB-Betrieb eingreift?}{Die Sendeleistung sollte erhöht werden.}
{Der NF-Pegel am Eingang des Funkgerätes sollte reduziert werden.}
{Das Oberwellenfilter sollte abgeschaltet werden.}
{Es sollte mit der RIT gegengesteuert werden.}
\end{QQuestion}

}
\only<2>{
\begin{QQuestion}{EJ219}{Was ist zu tun, wenn es bei digitalen Übertragungsverfahren zu Störungen kommt, weil die automatische Pegelregelung (ALC) eines Funkgerätes im SSB-Betrieb eingreift?}{Die Sendeleistung sollte erhöht werden.}
{\textbf{\textcolor{DARCgreen}{Der NF-Pegel am Eingang des Funkgerätes sollte reduziert werden.}}}
{Das Oberwellenfilter sollte abgeschaltet werden.}
{Es sollte mit der RIT gegengesteuert werden.}
\end{QQuestion}

}
\end{frame}%ENDCONTENT
