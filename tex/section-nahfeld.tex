
\section{Nahfeld}
\label{section:nahfeld}
\begin{frame}%STARTCONTENT

\only<1>{
\begin{QQuestion}{AK101}{Warum ist im Nahfeld einer Strahlungsquelle keine einfache Umrechnung zwischen den Feldgrößen~E und~H und damit auch keine vereinfachte Berechnung des Schutzabstandes möglich?}{Weil die elektrische und die magnetische Feldstärke im Nahfeld nicht senkrecht zur Ausbreitungsrichtung stehen und auf Grund des Einflusses der Erdoberfläche eine Phasendifferenz von größer \qty{180}{\degree} aufweisen.}
{Weil die elektrische und die magnetische Feldstärke im Nahfeld immer senkrecht aufeinander stehen und eine Phasendifferenz von \qty{90}{\degree} aufweisen.}
{Weil die elektrische und die magnetische Feldstärke im Nahfeld keine konstante Phasenbeziehung zueinander aufweisen.}
{Weil die elektrische und die magnetische Feldstärke im Nahfeld nicht exakt senkrecht aufeinander stehen und sich durch die nicht ideale Leitfähigkeit des Erdbodens am Sendeort der Feldwellenwiderstand des freien Raumes verändert.}
\end{QQuestion}

}
\only<2>{
\begin{QQuestion}{AK101}{Warum ist im Nahfeld einer Strahlungsquelle keine einfache Umrechnung zwischen den Feldgrößen~E und~H und damit auch keine vereinfachte Berechnung des Schutzabstandes möglich?}{Weil die elektrische und die magnetische Feldstärke im Nahfeld nicht senkrecht zur Ausbreitungsrichtung stehen und auf Grund des Einflusses der Erdoberfläche eine Phasendifferenz von größer \qty{180}{\degree} aufweisen.}
{Weil die elektrische und die magnetische Feldstärke im Nahfeld immer senkrecht aufeinander stehen und eine Phasendifferenz von \qty{90}{\degree} aufweisen.}
{\textbf{\textcolor{DARCgreen}{Weil die elektrische und die magnetische Feldstärke im Nahfeld keine konstante Phasenbeziehung zueinander aufweisen.}}}
{Weil die elektrische und die magnetische Feldstärke im Nahfeld nicht exakt senkrecht aufeinander stehen und sich durch die nicht ideale Leitfähigkeit des Erdbodens am Sendeort der Feldwellenwiderstand des freien Raumes verändert.}
\end{QQuestion}

}
\end{frame}

\begin{frame}
\only<1>{
\begin{QQuestion}{AK103}{In welchem Fall hat die folgende Formel zur Berechnung des Sicherheitsabstandes Gültigkeit und was sollten Sie tun, wenn die Gültigkeit nicht mehr sichergestellt ist? $d = \frac{\sqrt{\qty{30}{\ohm}\cdot P_{\symup{EIRP}}}}{E}$}{Die Formel gilt nur für Abstände $d > \frac{\lambda}{2\cdot\pi}$ bei horizontal polarisierten Antennen.
Bei kleineren Abständen und immer bei vertikal polarisierten Antennen muss der Sicherheitsabstand durch zum Beispiel Feldstärkemessungen oder Nahfeldberechnungen (Simulationen) ermittelt werden.}
{Im Bereich von Amateurfunkstellen ist der Unterschied zwischen Nah- und Fernfeld so gering, dass obige Formel, die eigentlich nur im Fernfeld gilt, trotzdem für alle Raumbereiche verwendet werden kann.}
{Die Formel gilt nur für Abstände $d > \frac{\lambda}{2\cdot\pi}$ bei den meisten Antennenformen (z.~B. Dipol-Antennen). Für Antennen, die z.~B. geometrisch klein im Verhältnis zur Wellenlänge sind und/oder in kürzerem Abstand zur Antenne muss der Sicherheitsabstand zum Beispiel durch Feldstärkemessungen oder Nahfeldberechnungen (Simulationen) ermittelt werden.}
{Die Formel gilt nur für Abstände $d > \frac{\lambda}{2\cdot\pi}$ bei vertikal polarisierten Antennen.
Bei kleineren Abständen und immer bei horizontal polarisierten Antennen muss der Sicherheitsabstand durch zum Beispiel Feldstärkemessungen oder Nahfeldberechnungen (Simulationen) ermittelt werden.}
\end{QQuestion}

}
\only<2>{
\begin{QQuestion}{AK103}{In welchem Fall hat die folgende Formel zur Berechnung des Sicherheitsabstandes Gültigkeit und was sollten Sie tun, wenn die Gültigkeit nicht mehr sichergestellt ist? $d = \frac{\sqrt{\qty{30}{\ohm}\cdot P_{\symup{EIRP}}}}{E}$}{Die Formel gilt nur für Abstände $d > \frac{\lambda}{2\cdot\pi}$ bei horizontal polarisierten Antennen.
Bei kleineren Abständen und immer bei vertikal polarisierten Antennen muss der Sicherheitsabstand durch zum Beispiel Feldstärkemessungen oder Nahfeldberechnungen (Simulationen) ermittelt werden.}
{Im Bereich von Amateurfunkstellen ist der Unterschied zwischen Nah- und Fernfeld so gering, dass obige Formel, die eigentlich nur im Fernfeld gilt, trotzdem für alle Raumbereiche verwendet werden kann.}
{\textbf{\textcolor{DARCgreen}{Die Formel gilt nur für Abstände $d > \frac{\lambda}{2\cdot\pi}$ bei den meisten Antennenformen (z.~B. Dipol-Antennen). Für Antennen, die z.~B. geometrisch klein im Verhältnis zur Wellenlänge sind und/oder in kürzerem Abstand zur Antenne muss der Sicherheitsabstand zum Beispiel durch Feldstärkemessungen oder Nahfeldberechnungen (Simulationen) ermittelt werden.}}}
{Die Formel gilt nur für Abstände $d > \frac{\lambda}{2\cdot\pi}$ bei vertikal polarisierten Antennen.
Bei kleineren Abständen und immer bei horizontal polarisierten Antennen muss der Sicherheitsabstand durch zum Beispiel Feldstärkemessungen oder Nahfeldberechnungen (Simulationen) ermittelt werden.}
\end{QQuestion}

}
\end{frame}%ENDCONTENT
