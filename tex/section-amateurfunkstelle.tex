
\section{Amateurfunkstelle}
\label{section:amateurfunkstelle}
\begin{frame}%STARTCONTENT

\frametitle{Amateurfunkstelle}
Die Radio Regulations (RR) legen fest:

\begin{itemize}
  \item Eine Funkstelle besteht nicht nur aus dem Empfänger und dem Sender an einem Ort, sondern auch jede Zusatzeinrichtung gehört dazu, die zum Betrieb erforderlich ist.
  \item Die allgemeine Definition einer Funkstelle der Radio Regulations (RR) gilt auch für Amateurfunkstellen. Die Radio Regulations legen daher die Amateurfunkstelle ganz einfach als \enquote{eine Funkstelle des Amateurfunkdienstes} fest.
  \end{itemize}
\end{frame}

\begin{frame}Das Amateurfunkgesetz beschreibt es im Kern genauso, fügt aber Details hinzu:

\begin{itemize}
  \item Eine Amateurfunkstelle besteht aus einer oder mehreren Sende- und Empfangsfunkanlagen einschließlich der Antennenanlagen und der zu ihrem Betrieb erforderlichen Zusatzeinrichtungen.
  \item Eine Amateurfunkstelle muss auf mindestens einer Amateurfunkfrequenz betrieben werden können.
  \end{itemize}
\end{frame}

\begin{frame}Inhaltlich ist laut Radio Regulations (RR) Funkverkehr zwischen Amateurfunkstellen verschiedener Länder auf Mitteilungen im Zusammenhang mit dem definitionsgemäßen Zweck des Amateurfunkdienstes und auf Bemerkungen persönlicher Art zu beschränken.

\end{frame}

\begin{frame}
\only<1>{
\begin{QQuestion}{VA201}{Wie ist die Funkstelle in den Radio Regulations (RR) sinngemäß definiert?}{Eine Funkstelle besteht aus mindestens einem Transceiver und einer Antenne einschließlich der Person, die diese bedient.}
{Der Begriff der Funkstelle ist in den Radio Regulations (RR) nicht definiert.}
{Eine Funkstelle besteht aus Sendern, Empfängern oder Transceivern ohne Zusatzeinrichtungen.}
{Eine Funkstelle besteht aus Sendern, Empfängern oder Transceivern und  Zusatzeinrichtungen, die zum Betrieb an einem Ort erforderlich sind.}
\end{QQuestion}

}
\only<2>{
\begin{QQuestion}{VA201}{Wie ist die Funkstelle in den Radio Regulations (RR) sinngemäß definiert?}{Eine Funkstelle besteht aus mindestens einem Transceiver und einer Antenne einschließlich der Person, die diese bedient.}
{Der Begriff der Funkstelle ist in den Radio Regulations (RR) nicht definiert.}
{Eine Funkstelle besteht aus Sendern, Empfängern oder Transceivern ohne Zusatzeinrichtungen.}
{\textbf{\textcolor{DARCgreen}{Eine Funkstelle besteht aus Sendern, Empfängern oder Transceivern und  Zusatzeinrichtungen, die zum Betrieb an einem Ort erforderlich sind.}}}
\end{QQuestion}

}
\end{frame}

\begin{frame}
\only<1>{
\begin{QQuestion}{VA202}{Wie ist die \glqq Amateurfunkstelle\grqq{} in den Radio Regulations (RR) definiert? Eine Amateurfunkstelle ist~...}{eine Funkstelle, die auf Amateurfunkfrequenzen sendet.}
{eine Funkstelle, die von Funkamateuren bedient wird.}
{eine Funkstelle des Amateurfunkdienstes.}
{eine Funkstelle mit Rufzeichen.}
\end{QQuestion}

}
\only<2>{
\begin{QQuestion}{VA202}{Wie ist die \glqq Amateurfunkstelle\grqq{} in den Radio Regulations (RR) definiert? Eine Amateurfunkstelle ist~...}{eine Funkstelle, die auf Amateurfunkfrequenzen sendet.}
{eine Funkstelle, die von Funkamateuren bedient wird.}
{\textbf{\textcolor{DARCgreen}{eine Funkstelle des Amateurfunkdienstes.}}}
{eine Funkstelle mit Rufzeichen.}
\end{QQuestion}

}
\end{frame}

\begin{frame}
\only<1>{
\begin{QQuestion}{VC103}{Nach dem Amateurfunkgesetz (AFuG) ist eine Amateurfunkstelle eine Funkstelle, die aus~...}{einer oder mehreren Sendefunkanlagen und Empfangsfunkanlagen einschließlich der Antennenanlagen und der zu ihrem Betrieb erforderlichen Zusatzeinrichtungen besteht und die auf mindestens einer der im Frequenzplan für den Amateurfunkdienst ausgewiesenen Frequenzen betrieben werden kann.}
{mehreren Sende- und Empfangsfunkanlagen besteht und die auf mindestens drei der im Frequenzplan für den Amateurfunkdienst ausgewiesenen Frequenzen oberhalb von \qty{30}{\MHz} betrieben werden kann.}
{mehreren Sende- und Empfangsfunkanlagen besteht und die auf mindestens drei der im Frequenzplan für den Amateurfunkdienst ausgewiesenen Frequenzen unterhalb von \qty{30}{\MHz} betrieben werden kann.}
{einer Empfangsfunkanlage einschließlich der Antennenanlagen und der zu ihrem Betrieb erforderlichen Zusatzeinrichtungen besteht und die auf mindestens einer der im Frequenzplan für den Amateurfunkdienst ausgewiesenen Frequenzen betrieben werden kann.}
\end{QQuestion}

}
\only<2>{
\begin{QQuestion}{VC103}{Nach dem Amateurfunkgesetz (AFuG) ist eine Amateurfunkstelle eine Funkstelle, die aus~...}{\textbf{\textcolor{DARCgreen}{einer oder mehreren Sendefunkanlagen und Empfangsfunkanlagen einschließlich der Antennenanlagen und der zu ihrem Betrieb erforderlichen Zusatzeinrichtungen besteht und die auf mindestens einer der im Frequenzplan für den Amateurfunkdienst ausgewiesenen Frequenzen betrieben werden kann.}}}
{mehreren Sende- und Empfangsfunkanlagen besteht und die auf mindestens drei der im Frequenzplan für den Amateurfunkdienst ausgewiesenen Frequenzen oberhalb von \qty{30}{\MHz} betrieben werden kann.}
{mehreren Sende- und Empfangsfunkanlagen besteht und die auf mindestens drei der im Frequenzplan für den Amateurfunkdienst ausgewiesenen Frequenzen unterhalb von \qty{30}{\MHz} betrieben werden kann.}
{einer Empfangsfunkanlage einschließlich der Antennenanlagen und der zu ihrem Betrieb erforderlichen Zusatzeinrichtungen besteht und die auf mindestens einer der im Frequenzplan für den Amateurfunkdienst ausgewiesenen Frequenzen betrieben werden kann.}
\end{QQuestion}

}
\end{frame}

\begin{frame}
\only<1>{
\begin{QQuestion}{VA302}{Was ist in den Radio Regulations (RR) hinsichtlich des Amateurfunkverkehrs festgelegt?}{Funkverkehr zwischen Amateurfunkstellen verschiedener Länder ist auf Mitteilungen im Zusammenhang mit dem definitionsgemäßen Zweck des Amateurfunkdienstes und auf Bemerkungen persönlicher Art zu beschränken.}
{Amateurfunkstellen ist die Teilnahme am Funkverkehr von Not- und Katastrophenfunkübungen nicht gestattet.}
{Es ist sicherzustellen, dass der Funkverkehr zwischen Amateurfunkstellen nicht für Dritte zugänglich gemacht wird.}
{Nachrichteninhalte im grenzüberschreitenden Amateurfunkverkehr sind auf rein technische Inhalte zu beschränken.}
\end{QQuestion}

}
\only<2>{
\begin{QQuestion}{VA302}{Was ist in den Radio Regulations (RR) hinsichtlich des Amateurfunkverkehrs festgelegt?}{\textbf{\textcolor{DARCgreen}{Funkverkehr zwischen Amateurfunkstellen verschiedener Länder ist auf Mitteilungen im Zusammenhang mit dem definitionsgemäßen Zweck des Amateurfunkdienstes und auf Bemerkungen persönlicher Art zu beschränken.}}}
{Amateurfunkstellen ist die Teilnahme am Funkverkehr von Not- und Katastrophenfunkübungen nicht gestattet.}
{Es ist sicherzustellen, dass der Funkverkehr zwischen Amateurfunkstellen nicht für Dritte zugänglich gemacht wird.}
{Nachrichteninhalte im grenzüberschreitenden Amateurfunkverkehr sind auf rein technische Inhalte zu beschränken.}
\end{QQuestion}

}
\end{frame}%ENDCONTENT
