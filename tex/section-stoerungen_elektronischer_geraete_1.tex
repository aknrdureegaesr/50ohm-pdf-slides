
\section{Störungen elektronischer Geräte I}
\label{section:stoerungen_elektronischer_geraete_1}
\begin{frame}%STARTCONTENT
\begin{itemize}
  \item Starke Sender führen zu unterschiedlichen Störungen und Beeinflussungen von elektronischen Geräten und Anlagen
  \item Ziel: Störungen vermeiden oder Ursachen durch Gegenmaßnahmen beseitigen
  \end{itemize}
\end{frame}

\begin{frame}
\frametitle{Einströmung}
\begin{itemize}
  \item Hochfrequenz gelangt durch Leitungen oder Kabel in ein Gerät
  \item Zum Beispiel über die Netzleitung, Antennenleitung, Lautsprecherkabel
  \end{itemize}
\end{frame}

\begin{frame}
\only<1>{
\begin{QQuestion}{EJ101}{In welchem Fall spricht man von Einströmungen? Einströmungen liegen dann vor, wenn Hochfrequenz~...}{wegen eines schlechten Stehwellenverhältnisses wieder zum Sender zurück strömt.}
{über das ungenügend abgeschirmte Gehäuse in die Elektronik gelangt.}
{über nicht genügend geschirmte Kabel zum Anpassgerät geführt wird.}
{über Leitungen oder Kabel in ein Gerät gelangt.}
\end{QQuestion}

}
\only<2>{
\begin{QQuestion}{EJ101}{In welchem Fall spricht man von Einströmungen? Einströmungen liegen dann vor, wenn Hochfrequenz~...}{wegen eines schlechten Stehwellenverhältnisses wieder zum Sender zurück strömt.}
{über das ungenügend abgeschirmte Gehäuse in die Elektronik gelangt.}
{über nicht genügend geschirmte Kabel zum Anpassgerät geführt wird.}
{\textbf{\textcolor{DARCgreen}{über Leitungen oder Kabel in ein Gerät gelangt.}}}
\end{QQuestion}

}
\end{frame}

\begin{frame}
\frametitle{Einstrahlung}
\begin{itemize}
  \item Hochfrequenz gelangt wegen ungenügend geschirmten Gehäuse in die Elektronik
  \item Führt dort zu Störungen
  \end{itemize}
\end{frame}

\begin{frame}
\only<1>{
\begin{QQuestion}{EJ102}{In welchem Fall spricht man von Einstrahlungen bei EMV? Einstrahlungen liegen dann vor, wenn die Hochfrequenz~...}{über das ungenügend abgeschirmte Gehäuse in die Elektronik gelangt.}
{über Leitungen oder Kabel in das gestörte Gerät gelangt.}
{über nicht genügend geschirmte Kabel zum gestörten Empfänger gelangt.}
{wegen eines schlechten Stehwellenverhältnisses wieder zum Sender zurück strahlt.}
\end{QQuestion}

}
\only<2>{
\begin{QQuestion}{EJ102}{In welchem Fall spricht man von Einstrahlungen bei EMV? Einstrahlungen liegen dann vor, wenn die Hochfrequenz~...}{\textbf{\textcolor{DARCgreen}{über das ungenügend abgeschirmte Gehäuse in die Elektronik gelangt.}}}
{über Leitungen oder Kabel in das gestörte Gerät gelangt.}
{über nicht genügend geschirmte Kabel zum gestörten Empfänger gelangt.}
{wegen eines schlechten Stehwellenverhältnisses wieder zum Sender zurück strahlt.}
\end{QQuestion}

}
\end{frame}

\begin{frame}
\frametitle{Störende Beeinflussung}
\begin{itemize}
  \item Kann trotz gesetzeskonformen Betrieb eines Senders beim Empfänger in Nähe auftreten
  \item Garagentorsteuerungen oder Funk-Autoschlüssel funktionieren nicht mehr wie gewohnt
  \item Störung von LED-Leuchten
  \end{itemize}
\end{frame}

\begin{frame}
\only<1>{
\begin{QQuestion}{EJ103}{Bereits durch die Aussendung des reinen Nutzsignals können in benachbarten Empfängern Störungen beim Empfang anderer Frequenzen auftreten. Dabei handelt es sich um eine~...}{Übersteuerung oder störende Beeinflussung.}
{Störung durch unerwünschte Aussendungen.}
{Störung durch unerwünschte Nebenaussendungen.}
{hinzunehmende Störung.}
\end{QQuestion}

}
\only<2>{
\begin{QQuestion}{EJ103}{Bereits durch die Aussendung des reinen Nutzsignals können in benachbarten Empfängern Störungen beim Empfang anderer Frequenzen auftreten. Dabei handelt es sich um eine~...}{\textbf{\textcolor{DARCgreen}{Übersteuerung oder störende Beeinflussung.}}}
{Störung durch unerwünschte Aussendungen.}
{Störung durch unerwünschte Nebenaussendungen.}
{hinzunehmende Störung.}
\end{QQuestion}

}
\end{frame}

\begin{frame}
\only<1>{
\begin{QQuestion}{EJ112}{Welches Gerät kann durch Aussendungen eines Amateurfunksenders störende Beeinflussungen zeigen?}{LED-Lampe mit Netzanschluss}
{Dampfbügeleisen mit Bimetall-Temperaturregler}
{Staubsauger mit Kollektormotor}
{Antennenrotor mit Wechselstrommotor}
\end{QQuestion}

}
\only<2>{
\begin{QQuestion}{EJ112}{Welches Gerät kann durch Aussendungen eines Amateurfunksenders störende Beeinflussungen zeigen?}{\textbf{\textcolor{DARCgreen}{LED-Lampe mit Netzanschluss}}}
{Dampfbügeleisen mit Bimetall-Temperaturregler}
{Staubsauger mit Kollektormotor}
{Antennenrotor mit Wechselstrommotor}
\end{QQuestion}

}
\end{frame}

\begin{frame}
\only<1>{
\begin{QQuestion}{EJ113}{Wie kommen Geräusche aus den Lautsprechern einer abgeschalteten Stereoanlage möglicherweise zustande?}{Durch Gleichrichtung starker HF-Signale in der NF-Endstufe der Stereoanlage.}
{Durch Gleichrichtung der ins Stromnetz eingestrahlten HF-Signale an den Dioden des Netzteils.}
{Durch Gleichrichtung abgestrahlter HF-Signale an PN-Übergängen in der NF-Vorstufe.}
{Durch eine Übersteuerung des Tuners mit dem über die Antennenzuleitung aufgenommenen HF-Signal.}
\end{QQuestion}

}
\only<2>{
\begin{QQuestion}{EJ113}{Wie kommen Geräusche aus den Lautsprechern einer abgeschalteten Stereoanlage möglicherweise zustande?}{\textbf{\textcolor{DARCgreen}{Durch Gleichrichtung starker HF-Signale in der NF-Endstufe der Stereoanlage.}}}
{Durch Gleichrichtung der ins Stromnetz eingestrahlten HF-Signale an den Dioden des Netzteils.}
{Durch Gleichrichtung abgestrahlter HF-Signale an PN-Übergängen in der NF-Vorstufe.}
{Durch eine Übersteuerung des Tuners mit dem über die Antennenzuleitung aufgenommenen HF-Signal.}
\end{QQuestion}

}
\end{frame}

\begin{frame}
\frametitle{Intermodulation}
\begin{itemize}
  \item Beim Auftreten von mehreren starken Empfangssignalen
  \item Z.B. TV-Sender und starke Amateurfunkstation in der Nachbarschaft
  \item Führt zu unerwünschten Oberwellen und deren Mischprodukten
  \item Durch Intermodulation werden Phantomsignale hervorgerufen
  \end{itemize}
\end{frame}

\begin{frame}
\only<1>{
\begin{QQuestion}{EJ120}{Welche Empfangs-Effekte werden durch Intermodulation hervorgerufen?}{Dem Empfangssignal ist ein pulsierendes Rauschen überlagert, das die Verständlichkeit beeinträchtigt.}
{Das Nutzsignal wird mit einem anderen Signal moduliert und dadurch verständlicher.}
{Es treten Phantomsignale auf, die selbst bei Einschalten eines Abschwächers in den HF-Signalweg nicht verschwinden.}
{Es treten Phantomsignale auf, die bei Abschalten einer der beteiligten Mischfrequenzen verschwindet.}
\end{QQuestion}

}
\only<2>{
\begin{QQuestion}{EJ120}{Welche Empfangs-Effekte werden durch Intermodulation hervorgerufen?}{Dem Empfangssignal ist ein pulsierendes Rauschen überlagert, das die Verständlichkeit beeinträchtigt.}
{Das Nutzsignal wird mit einem anderen Signal moduliert und dadurch verständlicher.}
{Es treten Phantomsignale auf, die selbst bei Einschalten eines Abschwächers in den HF-Signalweg nicht verschwinden.}
{\textbf{\textcolor{DARCgreen}{Es treten Phantomsignale auf, die bei Abschalten einer der beteiligten Mischfrequenzen verschwindet.}}}
\end{QQuestion}

}
\end{frame}

\begin{frame}
\frametitle{Oxidation}
\begin{itemize}
  \item Korrodierte Kontakte (Metall-Oxide) zwischen Metallen bilden Nichtlinearitäten durch Gleichricht-Effekte
  \item Unerwünschte Mischprodukte auf der Sende- und Empfangsseite
  \item Kann zu Störungen im Fernseh- und Rundfunkempfang führen
  \end{itemize}
\end{frame}

\begin{frame}
\only<1>{
\begin{QQuestion}{EJ121}{Ein korrodierter Anschluss an der Fernseh-Empfangsantenne des Nachbarn kann in Verbindung mit~...}{ dem Signal naher Sender unerwünschte Mischprodukte erzeugen, die den Fernsehempfang stören.}
{ dem Oszillatorsignal des Fernsehempfängers unerwünschte Mischprodukte erzeugen, die den Fernsehempfang stören.}
{ Einstreuungen aus dem Stromnetz durch Intermodulation Bild- und Tonstörungen hervorrufen.}
{dem Signal naher Sender parametrische Schwingungen erzeugen, die einen überhöhten Nutzsignalpegel hervorrufen.}
\end{QQuestion}

}
\only<2>{
\begin{QQuestion}{EJ121}{Ein korrodierter Anschluss an der Fernseh-Empfangsantenne des Nachbarn kann in Verbindung mit~...}{\textbf{\textcolor{DARCgreen}{ dem Signal naher Sender unerwünschte Mischprodukte erzeugen, die den Fernsehempfang stören.}}}
{ dem Oszillatorsignal des Fernsehempfängers unerwünschte Mischprodukte erzeugen, die den Fernsehempfang stören.}
{ Einstreuungen aus dem Stromnetz durch Intermodulation Bild- und Tonstörungen hervorrufen.}
{dem Signal naher Sender parametrische Schwingungen erzeugen, die einen überhöhten Nutzsignalpegel hervorrufen.}
\end{QQuestion}

}
\end{frame}

\begin{frame}
\frametitle{Erforderliche Sendeleistung}
\begin{itemize}
  \item Stets nur die für eine zufriedenstellende Kommunikation erforderliche Sendeleistung verwenden
  \item Zur Vermeidung von Störungen von Geräten
  \end{itemize}
\end{frame}

\begin{frame}
\only<1>{
\begin{QQuestion}{EJ104}{Um die Störwahrscheinlichkeit zu verringern, sollte die benutzte Sendeleistung~...}{nur auf den zulässigen Pegel eingestellt werden.}
{auf das für eine zufriedenstellende Kommunikation erforderliche Minimum eingestellt werden.}
{auf die für eine zufriedenstellende Kommunikation erforderlichen \qty{100}{\W} eingestellt werden.}
{die Hälfte des maximal zulässigen Pegels betragen.}
\end{QQuestion}

}
\only<2>{
\begin{QQuestion}{EJ104}{Um die Störwahrscheinlichkeit zu verringern, sollte die benutzte Sendeleistung~...}{nur auf den zulässigen Pegel eingestellt werden.}
{\textbf{\textcolor{DARCgreen}{auf das für eine zufriedenstellende Kommunikation erforderliche Minimum eingestellt werden.}}}
{auf die für eine zufriedenstellende Kommunikation erforderlichen \qty{100}{\W} eingestellt werden.}
{die Hälfte des maximal zulässigen Pegels betragen.}
\end{QQuestion}

}
\end{frame}

\begin{frame}
\only<1>{
\begin{QQuestion}{EJ105}{Bei einem Wohnort in einem Ballungsgebiet empfiehlt es sich, während der abendlichen Fernsehstunden~...}{mit keiner höheren Leistung zu senden, als für eine sichere Kommunikation erforderlich ist.}
{nur mit effektiver Leistung zu senden.}
{nur mit einer Hochgewinn-Richtantenne zu senden.}
{die Antenne unterhalb der Dachhöhe herabzulassen.}
\end{QQuestion}

}
\only<2>{
\begin{QQuestion}{EJ105}{Bei einem Wohnort in einem Ballungsgebiet empfiehlt es sich, während der abendlichen Fernsehstunden~...}{\textbf{\textcolor{DARCgreen}{mit keiner höheren Leistung zu senden, als für eine sichere Kommunikation erforderlich ist.}}}
{nur mit effektiver Leistung zu senden.}
{nur mit einer Hochgewinn-Richtantenne zu senden.}
{die Antenne unterhalb der Dachhöhe herabzulassen.}
\end{QQuestion}

}
\end{frame}

\begin{frame}
\frametitle{Übersteuerung}
\begin{itemize}
  \item Hohe Feldstärken durch hohe Sendeleistungen oder im Strahlungsbereich einer Antenne
  \item Empfänger und Empfangsstufen können übersteuert werden
  \item Verringert die Empfängerempfindlichkeit bis hin zur Blockierung
  \end{itemize}
\end{frame}

\begin{frame}
\only<1>{
\begin{QQuestion}{EJ106}{Eine \qty{432}{\MHz}-Sendeantenne mit hohem Gewinn ist unmittelbar auf eine Fernseh-Empfangsantenne gerichtet. Dies führt ggf. zu~...}{einer Übersteuerung eines TV-Empfängers.}
{Problemen mit dem \qty{432}{\MHz}-Empfänger.}
{Eigenschwingungen des \qty{432}{\MHz}-Senders.}
{dem Durchschlag des TV-Antennenkoaxialkabels.}
\end{QQuestion}

}
\only<2>{
\begin{QQuestion}{EJ106}{Eine \qty{432}{\MHz}-Sendeantenne mit hohem Gewinn ist unmittelbar auf eine Fernseh-Empfangsantenne gerichtet. Dies führt ggf. zu~...}{\textbf{\textcolor{DARCgreen}{einer Übersteuerung eines TV-Empfängers.}}}
{Problemen mit dem \qty{432}{\MHz}-Empfänger.}
{Eigenschwingungen des \qty{432}{\MHz}-Senders.}
{dem Durchschlag des TV-Antennenkoaxialkabels.}
\end{QQuestion}

}
\end{frame}

\begin{frame}
\only<1>{
\begin{QQuestion}{EJ107}{Wodurch können Sie die Übersteuerung eines Empfängers erkennen?}{Zeitweilige Blockierung der Frequenzeinstellung}
{Empfindlichkeitssteigerung}
{Auftreten von Pfeifstellen im gesamten Abstimmungsbereich}
{Rückgang der Empfindlichkeit}
\end{QQuestion}

}
\only<2>{
\begin{QQuestion}{EJ107}{Wodurch können Sie die Übersteuerung eines Empfängers erkennen?}{Zeitweilige Blockierung der Frequenzeinstellung}
{Empfindlichkeitssteigerung}
{Auftreten von Pfeifstellen im gesamten Abstimmungsbereich}
{\textbf{\textcolor{DARCgreen}{Rückgang der Empfindlichkeit}}}
\end{QQuestion}

}
\end{frame}

\begin{frame}
\frametitle{Weitere Maßnahmen}
\begin{itemize}
  \item Verringerung der Sendeleistung führt nicht immer zum Erfolg
  \item Das gestörte Gerät oder die Zuleitung könnte nicht genügend abgeschirmt sein
  \end{itemize}
\end{frame}

\begin{frame}
\only<1>{
\begin{QQuestion}{EJ108}{Wie sollte ein Abschirmgehäuse für HF-Baugruppen beschaffen sein?}{Möglichst geschlossenes Metallgehäuse }
{Kunststoffgehäuse mit niedriger Dielektrizitätszahl}
{Metallblech unter der HF-Baugruppe}
{Kunststoffgehäuse mit hoher Dielektrizitätszahl}
\end{QQuestion}

}
\only<2>{
\begin{QQuestion}{EJ108}{Wie sollte ein Abschirmgehäuse für HF-Baugruppen beschaffen sein?}{\textbf{\textcolor{DARCgreen}{Möglichst geschlossenes Metallgehäuse }}}
{Kunststoffgehäuse mit niedriger Dielektrizitätszahl}
{Metallblech unter der HF-Baugruppe}
{Kunststoffgehäuse mit hoher Dielektrizitätszahl}
\end{QQuestion}

}
\end{frame}

\begin{frame}
\only<1>{
\begin{QQuestion}{EJ109}{Falls sich eine Kurzwellen-Sendeantenne in der Nähe und parallel zu einer \qty{230}{\V}-Wechselstromleitung befindet,~...}{könnte erhebliche Überspannung im Netz erzeugt werden.}
{können harmonische Schwingungen erzeugt werden.}
{können Hochfrequenzströme ins Netz eingekoppelt werden.}
{kann \qty{50}{\Hz}-Modulation aller Signale auftreten.}
\end{QQuestion}

}
\only<2>{
\begin{QQuestion}{EJ109}{Falls sich eine Kurzwellen-Sendeantenne in der Nähe und parallel zu einer \qty{230}{\V}-Wechselstromleitung befindet,~...}{könnte erhebliche Überspannung im Netz erzeugt werden.}
{können harmonische Schwingungen erzeugt werden.}
{\textbf{\textcolor{DARCgreen}{können Hochfrequenzströme ins Netz eingekoppelt werden.}}}
{kann \qty{50}{\Hz}-Modulation aller Signale auftreten.}
\end{QQuestion}

}
\end{frame}

\begin{frame}
\only<1>{
\begin{QQuestion}{EJ111}{Um die Störwahrscheinlichkeit im eigenen Haus zu verringern, empfiehlt es sich vorzugsweise~...}{die Amateurfunkgeräte mittels des Schutzleiters zu erden.}
{Sendeantennen auf dem Dachboden zu errichten.}
{die Amateurfunkgeräte mit einem Wasserrohr zu verbinden.}
{für Sendeantennen eine separate HF-Erdleitung zu verwenden.}
\end{QQuestion}

}
\only<2>{
\begin{QQuestion}{EJ111}{Um die Störwahrscheinlichkeit im eigenen Haus zu verringern, empfiehlt es sich vorzugsweise~...}{die Amateurfunkgeräte mittels des Schutzleiters zu erden.}
{Sendeantennen auf dem Dachboden zu errichten.}
{die Amateurfunkgeräte mit einem Wasserrohr zu verbinden.}
{\textbf{\textcolor{DARCgreen}{für Sendeantennen eine separate HF-Erdleitung zu verwenden.}}}
\end{QQuestion}

}
\end{frame}

\begin{frame}
\frametitle{Nachbarschaftshilfe}
\begin{itemize}
  \item Hilfe dem Nachbarn anbieten
  \item Nur als letztes Mittel die Behörde einschalten
  \end{itemize}
\end{frame}

\begin{frame}
\only<1>{
\begin{QQuestion}{EJ124}{Die Bemühungen, die durch eine in der Nähe befindliche Amateurfunkstelle hervorgerufenen Fernsehstörungen zu verringern, sind fehlgeschlagen. Als nächster Schritt ist~...}{die Rückseite des Fernsehgeräts zu entfernen und das Gehäuse zu erden.}
{der Sender an die Bundesnetzagentur zu senden.}
{die zuständige Außenstelle der Bundesnetzagentur um Prüfung der Gegebenheiten zu bitten.}
{ein Fernsehtechniker des Fachhandwerks um Prüfung des Fernsehgeräts zu bitten.}
\end{QQuestion}

}
\only<2>{
\begin{QQuestion}{EJ124}{Die Bemühungen, die durch eine in der Nähe befindliche Amateurfunkstelle hervorgerufenen Fernsehstörungen zu verringern, sind fehlgeschlagen. Als nächster Schritt ist~...}{die Rückseite des Fernsehgeräts zu entfernen und das Gehäuse zu erden.}
{der Sender an die Bundesnetzagentur zu senden.}
{\textbf{\textcolor{DARCgreen}{die zuständige Außenstelle der Bundesnetzagentur um Prüfung der Gegebenheiten zu bitten.}}}
{ein Fernsehtechniker des Fachhandwerks um Prüfung des Fernsehgeräts zu bitten.}
\end{QQuestion}

}
\end{frame}

\begin{frame}
\frametitle{Filter}
\begin{itemize}
  \item Sowohl auf Seite des störenden Geräts als auch auf Seiten des gestörten Geräts einbauen
  \item Oberwellenaussendungen unterdrücken
  \item Hochpass oder Bandpass auf Empfängerseite 
  \item Übersteuerrung wird minimiert
  \end{itemize}
\end{frame}

\begin{frame}
\only<1>{
\begin{QQuestion}{EJ116}{Ein \qty{28}{\MHz}-Sender beeinflusst den Empfänger eines DVB-T2-Fernsehgerätes über dessen Antenneneingang. Was sollte zur Abhilfe vor den Antenneneingang des Fernsehgerätes eingeschleift werden?}{Ein Tiefpassfilter}
{Ein Hochpassfilter}
{Ein UHF-Abschwächer}
{Eine UHF-Bandsperre}
\end{QQuestion}

}
\only<2>{
\begin{QQuestion}{EJ116}{Ein \qty{28}{\MHz}-Sender beeinflusst den Empfänger eines DVB-T2-Fernsehgerätes über dessen Antenneneingang. Was sollte zur Abhilfe vor den Antenneneingang des Fernsehgerätes eingeschleift werden?}{Ein Tiefpassfilter}
{\textbf{\textcolor{DARCgreen}{Ein Hochpassfilter}}}
{Ein UHF-Abschwächer}
{Eine UHF-Bandsperre}
\end{QQuestion}

}
\end{frame}

\begin{frame}
\only<1>{
\begin{question2x2}{EJ117}{Eine KW-Amateurfunkstelle verursacht im Sendebetrieb in einem in der Nähe betriebenen Fernsehempfänger Störungen. Welches Filter schleifen Sie in das Fernsehantennenkabel ein, um die Störwahrscheinlichkeit zu verringern?}{\DARCimage{1.0\linewidth}{165include}}
{\DARCimage{1.0\linewidth}{166include}}
{\DARCimage{1.0\linewidth}{167include}}
{\DARCimage{1.0\linewidth}{168include}}
\end{question2x2}

}
\only<2>{
\begin{question2x2}{EJ117}{Eine KW-Amateurfunkstelle verursacht im Sendebetrieb in einem in der Nähe betriebenen Fernsehempfänger Störungen. Welches Filter schleifen Sie in das Fernsehantennenkabel ein, um die Störwahrscheinlichkeit zu verringern?}{\textbf{\textcolor{DARCgreen}{\DARCimage{1.0\linewidth}{165include}}}}
{\DARCimage{1.0\linewidth}{166include}}
{\DARCimage{1.0\linewidth}{167include}}
{\DARCimage{1.0\linewidth}{168include}}
\end{question2x2}

}
\end{frame}

\begin{frame}
\frametitle{Mantelwellensperren}
\begin{itemize}
  \item Sendesignal der Amateurfunkstation wird über den Schirm von Koaxialkabeln oder Zuleitungen in Empfänger oder Geräte in örtlicher Nähe eingekoppelt
  \item \emph{Mantelwellensperren} in Zuleitungen von Geräten einbauen
  \item Auch \emph{Drossel} genannt
  \item Ringkerne oder Klappferrite
  \item Weitere Möglichkeit: Verwendung von geschirmten Steuerkabeln
  \end{itemize}
\end{frame}

\begin{frame}
\only<1>{
\begin{QQuestion}{EJ118}{Durch eine Mantelwellendrossel in einem Fernseh-Antennenzuführungskabel~...}{werden Gleichtakt-HF-Störsignale unterdrückt.}
{werden niederfrequente Störsignale unterdrückt.}
{werden alle Wechselstromsignale unterdrückt.}
{wird Netzbrummen unterdrückt.}
\end{QQuestion}

}
\only<2>{
\begin{QQuestion}{EJ118}{Durch eine Mantelwellendrossel in einem Fernseh-Antennenzuführungskabel~...}{\textbf{\textcolor{DARCgreen}{werden Gleichtakt-HF-Störsignale unterdrückt.}}}
{werden niederfrequente Störsignale unterdrückt.}
{werden alle Wechselstromsignale unterdrückt.}
{wird Netzbrummen unterdrückt.}
\end{QQuestion}

}
\end{frame}

\begin{frame}
\only<1>{
\begin{QQuestion}{EJ119}{Die Signale eines \qty{144}{\MHz}-Senders werden in das Koax-Antennenkabel eines UKW-/DAB-Rundfunkempfängers induziert und verursachen Störungen. Eine Möglichkeit zur Verringerung der Störungen besteht darin,~...}{den \qty{144}{\MHz}-Sender mit einem Tiefpassfilter auszustatten.}
{die Erdverbindung des Senders abzuklemmen.}
{das Abschirmgeflecht am Antennenstecker des Empfängers abzuklemmen.}
{eine Mantelwellendrossel in das Kabel vor dem Rundfunkempfänger einzubauen.}
\end{QQuestion}

}
\only<2>{
\begin{QQuestion}{EJ119}{Die Signale eines \qty{144}{\MHz}-Senders werden in das Koax-Antennenkabel eines UKW-/DAB-Rundfunkempfängers induziert und verursachen Störungen. Eine Möglichkeit zur Verringerung der Störungen besteht darin,~...}{den \qty{144}{\MHz}-Sender mit einem Tiefpassfilter auszustatten.}
{die Erdverbindung des Senders abzuklemmen.}
{das Abschirmgeflecht am Antennenstecker des Empfängers abzuklemmen.}
{\textbf{\textcolor{DARCgreen}{eine Mantelwellendrossel in das Kabel vor dem Rundfunkempfänger einzubauen.}}}
\end{QQuestion}

}
\end{frame}

\begin{frame}
\only<1>{
\begin{QQuestion}{EJ115}{In einem Einfamilienhaus wird die Türsprechanlage durch den Betrieb eines nahen Senders gestört. Eine Möglichkeit zur Verringerung der Beeinflussungen besteht darin,~...}{die Länge des Kabels der Türsprechanlage zu verdoppeln.}
{für die Türsprechanlage ein geschirmtes Verbindungskabel zu verwenden.}
{für die Türsprechanlage eine Leitung mit niedrigerem Querschnitt zu verwenden.}
{für die Türsprechanlage eine Leitung mit versilberten Kupferdrähten zu verwenden.}
\end{QQuestion}

}
\only<2>{
\begin{QQuestion}{EJ115}{In einem Einfamilienhaus wird die Türsprechanlage durch den Betrieb eines nahen Senders gestört. Eine Möglichkeit zur Verringerung der Beeinflussungen besteht darin,~...}{die Länge des Kabels der Türsprechanlage zu verdoppeln.}
{\textbf{\textcolor{DARCgreen}{für die Türsprechanlage ein geschirmtes Verbindungskabel zu verwenden.}}}
{für die Türsprechanlage eine Leitung mit niedrigerem Querschnitt zu verwenden.}
{für die Türsprechanlage eine Leitung mit versilberten Kupferdrähten zu verwenden.}
\end{QQuestion}

}
\end{frame}

\begin{frame}
\only<1>{
\begin{QQuestion}{EJ114}{Bei der Musik-Anlage des Nachbarn wird Einströmung in die NF-Endstufe festgestellt. Eine mögliche Abhilfe wäre~...}{geschirmte Lautsprecherleitungen zu verwenden.}
{ein NF-Filter in das Koaxialkabel einzuschleifen.}
{einen Serienkondensator in die Lautsprecherleitung einzubauen.}
{ein geschirmtes Netzkabel für den Receiver zu verwenden.}
\end{QQuestion}

}
\only<2>{
\begin{QQuestion}{EJ114}{Bei der Musik-Anlage des Nachbarn wird Einströmung in die NF-Endstufe festgestellt. Eine mögliche Abhilfe wäre~...}{\textbf{\textcolor{DARCgreen}{geschirmte Lautsprecherleitungen zu verwenden.}}}
{ein NF-Filter in das Koaxialkabel einzuschleifen.}
{einen Serienkondensator in die Lautsprecherleitung einzubauen.}
{ein geschirmtes Netzkabel für den Receiver zu verwenden.}
\end{QQuestion}

}
\end{frame}

\begin{frame}
\frametitle{Logbuch}
\begin{itemize}
  \item Wenn die Funkanlage als Störquelle vermutet wird
  \item Freiwilligen Nachweis führen
  \item Ausschluss der Amateurfunkanlage als Störquelle
  \end{itemize}
\end{frame}

\begin{frame}
\only<1>{
\begin{QQuestion}{EJ122}{Ihr Nachbar beklagt sich über Störungen seines Fernsehempfangs und vermutet ihre Amateurfunkaussendungen als Ursache. Welcher erste Schritt bietet sich an?}{Sie verweisen den Nachbarn auf die Angebote von Internet-Streamingplattformen.}
{Sie überprüfen, ob der Nachbar sein Fernsehgerät ordnungsgemäß angemeldet hat.}
{Sie empfehlen die Erdung des Fernsehgerätes durch einen örtlichen Fachhändler.}
{Sie überprüfen den zeitlichen Zusammenhang der Störungen mit ihren Aussendungen.}
\end{QQuestion}

}
\only<2>{
\begin{QQuestion}{EJ122}{Ihr Nachbar beklagt sich über Störungen seines Fernsehempfangs und vermutet ihre Amateurfunkaussendungen als Ursache. Welcher erste Schritt bietet sich an?}{Sie verweisen den Nachbarn auf die Angebote von Internet-Streamingplattformen.}
{Sie überprüfen, ob der Nachbar sein Fernsehgerät ordnungsgemäß angemeldet hat.}
{Sie empfehlen die Erdung des Fernsehgerätes durch einen örtlichen Fachhändler.}
{\textbf{\textcolor{DARCgreen}{Sie überprüfen den zeitlichen Zusammenhang der Störungen mit ihren Aussendungen.}}}
\end{QQuestion}

}
\end{frame}

\begin{frame}
\frametitle{Schlechte Empfangsverhältnisse}
\begin{itemize}
  \item Z.B. TV-Zimmerantenne für Empfang
  \item Verwendung einer Außenantenne mit entsprechenden Vorfiltern
  \end{itemize}
\end{frame}

\begin{frame}
\only<1>{
\begin{QQuestion}{EJ123}{Beim Betrieb eines \qty{2}{\m}-Senders wird bei einem Nachbarn ein Fernsehempfänger gestört, der mit einer Zimmerantenne betrieben wird. Zur Behebung des Problems~...}{den Fernsehrundfunkempfänger zu wechseln.}
{ein doppelt geschirmtes Koaxialkabel für die Antennenleitung zu verwenden.}
{einen Vorverstärker in die Antennenleitung einzuschleifen.}
{schlagen Sie dem Nachbarn vor, eine außen angebrachte Fernsehantenne zu installieren.}
\end{QQuestion}

}
\only<2>{
\begin{QQuestion}{EJ123}{Beim Betrieb eines \qty{2}{\m}-Senders wird bei einem Nachbarn ein Fernsehempfänger gestört, der mit einer Zimmerantenne betrieben wird. Zur Behebung des Problems~...}{den Fernsehrundfunkempfänger zu wechseln.}
{ein doppelt geschirmtes Koaxialkabel für die Antennenleitung zu verwenden.}
{einen Vorverstärker in die Antennenleitung einzuschleifen.}
{\textbf{\textcolor{DARCgreen}{schlagen Sie dem Nachbarn vor, eine außen angebrachte Fernsehantenne zu installieren.}}}
\end{QQuestion}

}
\end{frame}

\begin{frame}
\frametitle{Übersteuerung}
\begin{itemize}
  \item Bei Übersteuerung von Sendern und Endstufen entstehen Nebenaussendungen
  \item Diese stören benachbarte Stationen
  \item Übersteuerung vermeiden
  \end{itemize}
\end{frame}

\begin{frame}
\only<1>{
\begin{QQuestion}{EJ213}{Die Übersteuerung eines Leistungsverstärkers führt~zu~...}{einem hohen Anteil an Nebenaussendungen.}
{lediglich geringen Verzerrungen beim Empfang.}
{einer besseren Verständlichkeit am Empfangsort.}
{einer Verringerung der Ausgangsleistung.}
\end{QQuestion}

}
\only<2>{
\begin{QQuestion}{EJ213}{Die Übersteuerung eines Leistungsverstärkers führt~zu~...}{\textbf{\textcolor{DARCgreen}{einem hohen Anteil an Nebenaussendungen.}}}
{lediglich geringen Verzerrungen beim Empfang.}
{einer besseren Verständlichkeit am Empfangsort.}
{einer Verringerung der Ausgangsleistung.}
\end{QQuestion}

}
\end{frame}

\begin{frame}
\only<1>{
\begin{QQuestion}{EJ214}{Ein SSB-Sender wird Störungen auf benachbarten Frequenzen hervorrufen, wenn~...}{die Ansteuerung der NF-Stufe zu gering ist.}
{das Antennenkabel unterbrochen ist.}
{der Leistungsverstärker übersteuert wird.}
{der Antennentuner falsch abgestimmt ist.}
\end{QQuestion}

}
\only<2>{
\begin{QQuestion}{EJ214}{Ein SSB-Sender wird Störungen auf benachbarten Frequenzen hervorrufen, wenn~...}{die Ansteuerung der NF-Stufe zu gering ist.}
{das Antennenkabel unterbrochen ist.}
{\textbf{\textcolor{DARCgreen}{der Leistungsverstärker übersteuert wird.}}}
{der Antennentuner falsch abgestimmt ist.}
\end{QQuestion}

}
\end{frame}

\begin{frame}
\frametitle{Frequenzstabilität}
\begin{itemize}
  \item Nicht stabile Oszillatoren können zu Aussendungen außerhalb der Bandgrenzen führen
  \item Das kann benachbarte Stationen stören
  \item Ursache z.B. Selbstbaugerät mit nicht quarzstabilisierten Oszillator
  \end{itemize}
\end{frame}

\begin{frame}
\only<1>{
\begin{QQuestion}{EJ216}{Welche unerwünschte Auswirkung kann mangelhafte Frequenzstabilität eines Senders haben?}{Aussendungen außerhalb der Bandgrenzen}
{Spannungsüberschläge in der Endstufe des Senders}
{Überlastung der Endstufe des Senders}
{Verstärkte Oberwellenaussendung innerhalb der Bandgrenzen}
\end{QQuestion}

}
\only<2>{
\begin{QQuestion}{EJ216}{Welche unerwünschte Auswirkung kann mangelhafte Frequenzstabilität eines Senders haben?}{\textbf{\textcolor{DARCgreen}{Aussendungen außerhalb der Bandgrenzen}}}
{Spannungsüberschläge in der Endstufe des Senders}
{Überlastung der Endstufe des Senders}
{Verstärkte Oberwellenaussendung innerhalb der Bandgrenzen}
\end{QQuestion}

}
\end{frame}

\begin{frame}
\frametitle{Bandbreite}
\begin{itemize}
  \item Überschreitung der zulässigen Bandbreite kann insbesondere bei AFSK-modulierten FM-Sendern geschehen
  \item Abhilfe durch Hub begrenzen
  \item Oder Aussteuerung der NF reduzieren
  \item Beachten bei Packet-Radio oder Digimodes
  \end{itemize}
\end{frame}

\begin{frame}
\only<1>{
\begin{QQuestion}{EJ212}{Sie modulieren Ihren FM-Sender mit einem AFSK-Signal (Niederfrequenzumtastung). Wie können Sie die Bandbreite der Aussendung reduzieren? Durch~...}{Anheben der Sendeleistung oder der ZF}
{Anheben des NF-Pegels oder des Frequenzhubs}
{Absenken der Sendeleistung oder der ZF}
{Absenken des NF-Pegels oder des Frequenzhubs}
\end{QQuestion}

}
\only<2>{
\begin{QQuestion}{EJ212}{Sie modulieren Ihren FM-Sender mit einem AFSK-Signal (Niederfrequenzumtastung). Wie können Sie die Bandbreite der Aussendung reduzieren? Durch~...}{Anheben der Sendeleistung oder der ZF}
{Anheben des NF-Pegels oder des Frequenzhubs}
{Absenken der Sendeleistung oder der ZF}
{\textbf{\textcolor{DARCgreen}{Absenken des NF-Pegels oder des Frequenzhubs}}}
\end{QQuestion}

}
\end{frame}%ENDCONTENT
