
\section{Wellenwiderstand}
\label{section:wellenwiderstand}
\begin{frame}%STARTCONTENT

\only<1>{
\begin{QQuestion}{AG305}{Eine offene Paralleldrahtleitung ist aus Draht mit einem Durchmesser d~=~\qty{2}{\mm} gefertigt. Der Abstand der parallelen Leiter beträgt a~=~\qty{20}{\cm}. Wie groß ist der Wellenwiderstand $Z_0$ der Leitung?}{ca. \qty{2,8}{\kohm}}
{ca. \qty{276}{\ohm}}
{ca. \qty{635}{\ohm}}
{ca. \qty{820}{\ohm}}
\end{QQuestion}

}
\only<2>{
\begin{QQuestion}{AG305}{Eine offene Paralleldrahtleitung ist aus Draht mit einem Durchmesser d~=~\qty{2}{\mm} gefertigt. Der Abstand der parallelen Leiter beträgt a~=~\qty{20}{\cm}. Wie groß ist der Wellenwiderstand $Z_0$ der Leitung?}{ca. \qty{2,8}{\kohm}}
{ca. \qty{276}{\ohm}}
{\textbf{\textcolor{DARCgreen}{ca. \qty{635}{\ohm}}}}
{ca. \qty{820}{\ohm}}
\end{QQuestion}

}
\end{frame}

\begin{frame}
\frametitle{Lösungsweg}
\begin{itemize}
  \item gegeben: $d = 2mm$
  \item gegeben: $a = 20cm$
  \item gegeben: $\epsilon_\mathrm{r} \approx 1$ für Luft
  \item gesucht: $Z$
  \end{itemize}
    \pause
    $Z = \dfrac{120Ω}{\sqrt{\epsilon_\mathrm{r}}} \cdot \ln{(\dfrac{2 \cdot a}{d})} = \dfrac{120Ω}{\sqrt{1}} \cdot \ln{(\dfrac{2 \cdot 200mm}{2mm})} \approx 635Ω$



\end{frame}

\begin{frame}
\only<1>{
\begin{QQuestion}{AG306}{Ein Koaxialkabel (luftisoliert) hat einen Innendurchmesser der Abschirmung von \qty{5}{\mm}. Der Außendurchmesser des inneren Leiters beträgt \qty{1}{\mm}. Wie groß ist der Wellenwiderstand $Z_0$ des Kabels?}{ca. \qty{123}{\ohm}}
{ca. \qty{60}{\ohm}}
{ca. \qty{50}{\ohm}}
{ca. \qty{97}{\ohm}}
\end{QQuestion}

}
\only<2>{
\begin{QQuestion}{AG306}{Ein Koaxialkabel (luftisoliert) hat einen Innendurchmesser der Abschirmung von \qty{5}{\mm}. Der Außendurchmesser des inneren Leiters beträgt \qty{1}{\mm}. Wie groß ist der Wellenwiderstand $Z_0$ des Kabels?}{ca. \qty{123}{\ohm}}
{ca. \qty{60}{\ohm}}
{ca. \qty{50}{\ohm}}
{\textbf{\textcolor{DARCgreen}{ca. \qty{97}{\ohm}}}}
\end{QQuestion}

}
\end{frame}

\begin{frame}
\frametitle{Lösungsweg}
\begin{itemize}
  \item gegeben: $D = 5mm$
  \item gegeben: $d = 1mm$
  \item gegeben: $\epsilon_\mathrm{r} \approx 1$ für Luft
  \item gesucht: $Z$
  \end{itemize}
    \pause
    $Z = \dfrac{60Ω}{\sqrt{\epsilon_\mathrm{r}}} \cdot \ln{(\dfrac{D}{d})} = \dfrac{60Ω}{\sqrt{1}} \cdot \ln{(\dfrac{5mm}{1mm})} \approx 97Ω$



\end{frame}

\begin{frame}
\only<1>{
\begin{QQuestion}{AG307}{Ein Koaxialkabel hat einen Innenleiterdurchmesser von \qty{0,7}{\mm}. Die Isolierung zwischen Innenleiter und Abschirmgeflecht besteht aus Polyethylen (PE) und sie hat einen Durchmesser von \qty{4,4}{\mm}. Der Außendurchmesser des Kabels ist \qty{7,4}{\mm}. Wie hoch ist der ungefähre Wellenwiderstand des Kabels?}{ca. \qty{75}{\ohm}}
{ca. \qty{20}{\ohm}}
{ca. \qty{50}{\ohm}}
{ca. \qty{95}{\ohm}}
\end{QQuestion}

}
\only<2>{
\begin{QQuestion}{AG307}{Ein Koaxialkabel hat einen Innenleiterdurchmesser von \qty{0,7}{\mm}. Die Isolierung zwischen Innenleiter und Abschirmgeflecht besteht aus Polyethylen (PE) und sie hat einen Durchmesser von \qty{4,4}{\mm}. Der Außendurchmesser des Kabels ist \qty{7,4}{\mm}. Wie hoch ist der ungefähre Wellenwiderstand des Kabels?}{\textbf{\textcolor{DARCgreen}{ca. \qty{75}{\ohm}}}}
{ca. \qty{20}{\ohm}}
{ca. \qty{50}{\ohm}}
{ca. \qty{95}{\ohm}}
\end{QQuestion}

}
\end{frame}

\begin{frame}
\frametitle{Lösungsweg}
\begin{itemize}
  \item gegeben: $d = 0,7mm$
  \item gegeben: $D = 4,4mm$
  \item gegeben: $\epsilon_\mathrm{r} = 2,29$
  \item gesucht: $Z$
  \end{itemize}
    \pause
    $Z = \dfrac{60Ω}{\sqrt{\epsilon_\mathrm{r}}} \cdot \ln{(\dfrac{D}{d})} = \dfrac{60Ω}{\sqrt{2,29}} \cdot \ln{(\dfrac{4,4mm}{0,7mm})} \approx 75Ω$



\end{frame}

\begin{frame}
\only<1>{
\begin{QQuestion}{AG304}{Eine Übertragungsleitung wird angepasst betrieben, wenn der Widerstand, mit dem sie abgeschlossen ist,~...}{den Wert des Wellenwiderstandes der Leitung aufweist.}
{\qty{50}{\ohm} beträgt.}
{ein ohmscher Wirkwiderstand ist.}
{eine offene Leitung darstellt.}
\end{QQuestion}

}
\only<2>{
\begin{QQuestion}{AG304}{Eine Übertragungsleitung wird angepasst betrieben, wenn der Widerstand, mit dem sie abgeschlossen ist,~...}{\textbf{\textcolor{DARCgreen}{den Wert des Wellenwiderstandes der Leitung aufweist.}}}
{\qty{50}{\ohm} beträgt.}
{ein ohmscher Wirkwiderstand ist.}
{eine offene Leitung darstellt.}
\end{QQuestion}

}
\end{frame}%ENDCONTENT
