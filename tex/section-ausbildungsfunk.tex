
\section{Ausbildungsfunkbetrieb}
\label{section:ausbildungsfunk}
\begin{frame}%STARTCONTENT

\frametitle{Ausbildungsfunkbetrieb}
Es gibt zum Zweck der Ausbildung die Ausnahme, dass auch Nicht-Funkamateure auf Amateurfunkfrequenzen senden dürfen.

Unter unmittelbarer Anleitung und Aufsicht eines zugelassenen Funkamateurs der Klasse~E oder A.

\end{frame}

\begin{frame}
\only<1>{
\begin{QQuestion}{VD303}{Nicht-Funkamateure dürfen am Ausbildungsfunkbetrieb~...}{auch an Wochenenden ohne besondere Auflagen teilnehmen.}
{nur an Klubstationen unter Aufsicht eines Funkamateurs mit zugeteiltem Rufzeichen der Klasse A oder E teilnehmen.}
{nur unter unmittelbarer Anleitung und Aufsicht eines Funkamateurs mit zugeteiltem Rufzeichen der Klasse A oder E teilnehmen.}
{auch ohne Anleitung und Aufsicht des ausbildenden Funkamateurs teilnehmen.}
\end{QQuestion}

}
\only<2>{
\begin{QQuestion}{VD303}{Nicht-Funkamateure dürfen am Ausbildungsfunkbetrieb~...}{auch an Wochenenden ohne besondere Auflagen teilnehmen.}
{nur an Klubstationen unter Aufsicht eines Funkamateurs mit zugeteiltem Rufzeichen der Klasse A oder E teilnehmen.}
{\textbf{\textcolor{DARCgreen}{nur unter unmittelbarer Anleitung und Aufsicht eines Funkamateurs mit zugeteiltem Rufzeichen der Klasse A oder E teilnehmen.}}}
{auch ohne Anleitung und Aufsicht des ausbildenden Funkamateurs teilnehmen.}
\end{QQuestion}

}
\end{frame}

\begin{frame}
\frametitle{Abwicklung Ausbildungsfunkbetrieb}
\begin{itemize}
  \item Der Auszubildende benutzt das Rufzeichen des Ausbilders und hängt den Zusatz „/T“ an: DG2RON/T
  \item Ausgesprochen wird das als „Trainee”
  \end{itemize}

\end{frame}

\begin{frame}
\only<1>{
\begin{QQuestion}{VD306}{Von wem ist während des Ausbildungsfunkbetriebs der Rufzeichenzusatz \glqq /T\grqq{} bzw. \glqq /Trainee\grqq{} zu benutzen?}{Vom Verantwortlichen der Schulstation}
{Vom Ausbilder}
{Vom Auszubildenden und vom Ausbilder}
{Vom Auszubildenden}
\end{QQuestion}

}
\only<2>{
\begin{QQuestion}{VD306}{Von wem ist während des Ausbildungsfunkbetriebs der Rufzeichenzusatz \glqq /T\grqq{} bzw. \glqq /Trainee\grqq{} zu benutzen?}{Vom Verantwortlichen der Schulstation}
{Vom Ausbilder}
{Vom Auszubildenden und vom Ausbilder}
{\textbf{\textcolor{DARCgreen}{Vom Auszubildenden}}}
\end{QQuestion}

}
\end{frame}

\begin{frame}
\only<1>{
\begin{QQuestion}{BD209}{Der Funkamateur mit dem Rufzeichen DL1PZ möchte Ausbildungsfunkbetrieb im Sprechfunk durchführen. Welches Rufzeichen darf der Auszubildende verwenden?}{DL1PZ/Trainee}
{DL1PZ/Ausbildung}
{Ausbildung/DL1PZ}
{Trainee/DL1PZ}
\end{QQuestion}

}
\only<2>{
\begin{QQuestion}{BD209}{Der Funkamateur mit dem Rufzeichen DL1PZ möchte Ausbildungsfunkbetrieb im Sprechfunk durchführen. Welches Rufzeichen darf der Auszubildende verwenden?}{\textbf{\textcolor{DARCgreen}{DL1PZ/Trainee}}}
{DL1PZ/Ausbildung}
{Ausbildung/DL1PZ}
{Trainee/DL1PZ}
\end{QQuestion}

}
 \end{frame}%ENDCONTENT
