
\section{Datenübertragungsrate}
\label{section:datenuebertragungsdrate}
\begin{frame}%STARTCONTENT
\begin{itemize}
  \item Die \emph{Bandbreite} ist der genutzte Frequenzbereich in \emph{Hz}
  \item Die \emph{Datenübertragungsrate} ist die je Zeiteinheit übertragene Datenmenge in \emph{Bit/s}
  \end{itemize}
\end{frame}

\begin{frame}\begin{itemize}
  \item In der Praxis erreichbare Datenübertragungsraten unterscheiden sich je nach Übertragungsverfahren und Funkbedingungen deutlich.
  \item WLAN und 5G unterstützen bei optimalen Bedingungen Datenübertragungsraten bis in den Bereich von Gigabit pro Sekunde.
  \item FT8 hingegen kann selbst unter widrigen Bedingungen eingesetzt werden, überträgt aber nur wenige Bit pro Sekunde.
  \end{itemize}
\end{frame}

\begin{frame}
\only<1>{
\begin{QQuestion}{EA106}{Welche Einheit wird üblicherweise für die Datenübertragungsrate verwendet?}{Dezibel (dB)}
{Baud (Bd)}
{Hertz (Hz)}
{Bit pro Sekunde (Bit/s)}
\end{QQuestion}

}
\only<2>{
\begin{QQuestion}{EA106}{Welche Einheit wird üblicherweise für die Datenübertragungsrate verwendet?}{Dezibel (dB)}
{Baud (Bd)}
{Hertz (Hz)}
{\textbf{\textcolor{DARCgreen}{Bit pro Sekunde (Bit/s)}}}
\end{QQuestion}

}
\end{frame}

\begin{frame}
\only<1>{
\begin{QQuestion}{EE401}{Welcher Unterschied besteht zwischen der Bandbreite und der Datenübertragungsrate?}{Als Bandbreite wird die übertragene Datenmenge (in Hz) und als Datenübertragungsrate die je Zeiteinheit übertragenen Symbole (in Baud) bezeichnet. }
{Als Bandbreite wird der genutzte Frequenzbereich (in Hz) und als Datenübertragungsrate die je Zeiteinheit übertragene Datenmenge (in Bit/s) bezeichnet.}
{Die Datenübertragungsrate (in Bit/s) entspricht der Symbolrate (in Baud). Die Bandbreite (in Hz) entspricht der maximal möglichen Datenübertragungsrate (in Bit/s).}
{Die Datenübertragungsrate (in Baud) entspricht der Symbolrate (in Bit/s). Die Bandbreite (in Hz) entspricht der minimal möglichen Datenübertragungsrate (in Baud).}
\end{QQuestion}

}
\only<2>{
\begin{QQuestion}{EE401}{Welcher Unterschied besteht zwischen der Bandbreite und der Datenübertragungsrate?}{Als Bandbreite wird die übertragene Datenmenge (in Hz) und als Datenübertragungsrate die je Zeiteinheit übertragenen Symbole (in Baud) bezeichnet. }
{\textbf{\textcolor{DARCgreen}{Als Bandbreite wird der genutzte Frequenzbereich (in Hz) und als Datenübertragungsrate die je Zeiteinheit übertragene Datenmenge (in Bit/s) bezeichnet.}}}
{Die Datenübertragungsrate (in Bit/s) entspricht der Symbolrate (in Baud). Die Bandbreite (in Hz) entspricht der maximal möglichen Datenübertragungsrate (in Bit/s).}
{Die Datenübertragungsrate (in Baud) entspricht der Symbolrate (in Bit/s). Die Bandbreite (in Hz) entspricht der minimal möglichen Datenübertragungsrate (in Baud).}
\end{QQuestion}

}
\end{frame}%ENDCONTENT
