
\section{Magnetisches Feld}
\label{section:h_feld}
\begin{frame}%STARTCONTENT

\frametitle{Stromdurchflossener Leiter}
\begin{columns}
    \begin{column}{0.48\textwidth}
    \begin{itemize}
  \item Fließt Strom durch einen Leiter, bilden sich konzentrische, magnetische Felder um den Leiter
  \end{itemize}

    \end{column}
   \begin{column}{0.48\textwidth}
       Grafik eines stromdurchflossenen Leiters mit konzentrischen magnetischen Feldlinien kommt noch


   \end{column}
\end{columns}

\end{frame}

\begin{frame}
\only<1>{
\begin{QQuestion}{EB201}{Wenn ein konstanter Gleichstrom durch einen gestreckten Leiter fließt, sind die~...}{magnetischen Feldlinien sternförmig um den Leiter.}
{elektrischen Feldlinien konzentrische Kreise um den Leiter.}
{magnetischen Feldlinien konzentrische Kreise um den Leiter.}
{elektrischen Feldlinien parallel zu den magnetischen Feldlinien um den Leiter.}
\end{QQuestion}

}
\only<2>{
\begin{QQuestion}{EB201}{Wenn ein konstanter Gleichstrom durch einen gestreckten Leiter fließt, sind die~...}{magnetischen Feldlinien sternförmig um den Leiter.}
{elektrischen Feldlinien konzentrische Kreise um den Leiter.}
{\textbf{\textcolor{DARCgreen}{magnetischen Feldlinien konzentrische Kreise um den Leiter.}}}
{elektrischen Feldlinien parallel zu den magnetischen Feldlinien um den Leiter.}
\end{QQuestion}

}
\end{frame}

\begin{frame}
\frametitle{Homogenes magnetisches Feld}
\begin{columns}
    \begin{column}{0.48\textwidth}
    \begin{itemize}
  \item Wird ein stromdurchflossener Leiter zu einer Zylinderspule aufgewickelt, entsteht im inneren ein homogenes magnetisches Feld (\emph{H-Feld})
  \item Eine Spule speichert magnetische Energie
  \end{itemize}

    \end{column}
   \begin{column}{0.48\textwidth}
       
\begin{figure}
    \DARCimage{0.85\linewidth}{50include}
    \caption{\scriptsize Magnetische Feldlinien in einer Zylinderspule}
    \label{e_h_feld_spule}
\end{figure}

\begin{itemize}
  \item Einheit: $\dfrac{A}{m}$
  \end{itemize}

   \end{column}
\end{columns}

\end{frame}

\begin{frame}
\only<1>{
\begin{PQuestion}{EB202}{Welches Feld stellt sich im Inneren einer langen Zylinderspule bei Fließen eines Gleichstroms näherungsweise ein?}{Homogenes magnetisches Feld}
{Homogenes elektrisches Feld}
{Konzentrisches magnetisches Feld}
{Zentriertes magnetisches Feld}
{\DARCimage{1.0\linewidth}{50include}}\end{PQuestion}

}
\only<2>{
\begin{PQuestion}{EB202}{Welches Feld stellt sich im Inneren einer langen Zylinderspule bei Fließen eines Gleichstroms näherungsweise ein?}{\textbf{\textcolor{DARCgreen}{Homogenes magnetisches Feld}}}
{Homogenes elektrisches Feld}
{Konzentrisches magnetisches Feld}
{Zentriertes magnetisches Feld}
{\DARCimage{1.0\linewidth}{50include}}\end{PQuestion}

}
\end{frame}

\begin{frame}
\only<1>{
\begin{QQuestion}{EA104}{Welche Einheit wird üblicherweise für die magnetische Feldstärke verwendet?}{Ampere pro Meter (A/m)}
{Watt pro Meter (W/m)}
{Volt pro Meter (V/m)}
{Henry pro Meter (H/m)}
\end{QQuestion}

}
\only<2>{
\begin{QQuestion}{EA104}{Welche Einheit wird üblicherweise für die magnetische Feldstärke verwendet?}{\textbf{\textcolor{DARCgreen}{Ampere pro Meter (A/m)}}}
{Watt pro Meter (W/m)}
{Volt pro Meter (V/m)}
{Henry pro Meter (H/m)}
\end{QQuestion}

}
\end{frame}

\begin{frame}
\frametitle{Ringkern}
\begin{columns}
    \begin{column}{0.48\textwidth}
    \begin{itemize}
  \item Leiter wird auf einen magnetisch leitenden Ringkern gewickelt, z.B. Eisen
  \item Vorteile: Platzsparend und stabiler
  \end{itemize}

    \end{column}
   \begin{column}{0.48\textwidth}
       
\begin{figure}
    \DARCimage{0.85\linewidth}{40include}
    \caption{\scriptsize Ringkernspule}
    \label{e_ringkern}
\end{figure}


   \end{column}
\end{columns}

\end{frame}

\begin{frame}
\begin{columns}
    \begin{column}{0.48\textwidth}
    \begin{itemize}
  \item Magnetische Feldstärke $H = \dfrac{I\cdot N}{l_m}$ in $\dfrac{A}{m}$
  \item mit $N$ als Wicklungsanzahl und $l_m$ mittlere Länge des Rings
  \end{itemize}

    \end{column}
   \begin{column}{0.48\textwidth}
       
\begin{figure}
    \DARCimage{0.85\linewidth}{40include}
    \caption{\scriptsize Ringkernspule}
    \label{e_ringkern}
\end{figure}


   \end{column}
\end{columns}

\end{frame}

\begin{frame}
\only<1>{
\begin{PQuestion}{EB203}{Ein Ringkern hat einen mittleren Durchmesser von \qty{2,6}{\cm} und trägt 6 Windungen Kupferdraht. Wie groß ist die mittlere magnetische Feldstärke im Ringkern, wenn der Strom \qty{2,5}{\A} beträgt?}{\qty{183,6}{\A}/m}
{\qty{1,836}{\A}/m}
{\qty{5769}{\A}/m}
{\qty{5,769}{\A}/m}
{\DARCimage{0.5\linewidth}{40include}}\end{PQuestion}

}
\only<2>{
\begin{PQuestion}{EB203}{Ein Ringkern hat einen mittleren Durchmesser von \qty{2,6}{\cm} und trägt 6 Windungen Kupferdraht. Wie groß ist die mittlere magnetische Feldstärke im Ringkern, wenn der Strom \qty{2,5}{\A} beträgt?}{\textbf{\textcolor{DARCgreen}{\qty{183,6}{\A}/m}}}
{\qty{1,836}{\A}/m}
{\qty{5769}{\A}/m}
{\qty{5,769}{\A}/m}
{\DARCimage{0.5\linewidth}{40include}}\end{PQuestion}

}

\end{frame}

\begin{frame}
\only<1>{
\begin{QQuestion}{EB204}{Welcher der nachfolgenden Werkstoffe ist bei Raumtemperatur ein ferromagnetischer Stoff?}{Kupfer}
{Chrom}
{Eisen}
{Aluminium}
\end{QQuestion}

}
\only<2>{
\begin{QQuestion}{EB204}{Welcher der nachfolgenden Werkstoffe ist bei Raumtemperatur ein ferromagnetischer Stoff?}{Kupfer}
{Chrom}
{\textbf{\textcolor{DARCgreen}{Eisen}}}
{Aluminium}
\end{QQuestion}

}
\end{frame}

\begin{frame}
\frametitle{Magnetfeld einer Antenne}
\begin{columns}
    \begin{column}{0.48\textwidth}
    \begin{itemize}
  \item An einer Antenne wirkt das Magnetfeld um den Leiter
  \item Hier an einer Vertikalantenne konzentrisch um die Antenne
  \end{itemize}

    \end{column}
   \begin{column}{0.48\textwidth}
       
\begin{figure}
    \DARCimage{0.85\linewidth}{192include}
    \caption{\scriptsize Magnetfeld an einer Vertikalantenne}
    \label{e_vertikalantenne_magnetfeld}
\end{figure}


   \end{column}
\end{columns}

\end{frame}

\begin{frame}
\only<1>{
\begin{PQuestion}{EB206}{Wie werden die mit X gekennzeichneten Feldlinien einer Vertikalantenne bezeichnet?}{Offene Feldlinien}
{Elektrische Feldlinien}
{Magnetische Feldlinien}
{Vertikale Feldlinien}
{\DARCimage{1.0\linewidth}{192include}}\end{PQuestion}

}
\only<2>{
\begin{PQuestion}{EB206}{Wie werden die mit X gekennzeichneten Feldlinien einer Vertikalantenne bezeichnet?}{Offene Feldlinien}
{Elektrische Feldlinien}
{\textbf{\textcolor{DARCgreen}{Magnetische Feldlinien}}}
{Vertikale Feldlinien}
{\DARCimage{1.0\linewidth}{192include}}\end{PQuestion}

}
\end{frame}%ENDCONTENT
