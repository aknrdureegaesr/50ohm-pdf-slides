
\section{Frequenzvervielfacher II}
\label{section:frequenzvervielfacher_2}
\begin{frame}%STARTCONTENT

\only<1>{
\begin{QQuestion}{AF311}{Nach welchem Prinzip arbeitet die analoge Frequenzvervielfachung?}{Das jeweils um plus und minus \qty{90}{\degree} phasenverschobene Signal wird einem additiven Mischer zugeführt, der die gewünschte Oberschwingungen erzeugt.}
{Das Signal wird einer nicht linearen Verzerrerstufe zugeführt und die gewünschte Oberschwingungen ausgefiltert.}
{Das Signal wird gefiltert und einem Ringmischer zugeführt, der die gewünschte Oberschwingungen erzeugt.}
{Das jeweils um plus und minus \qty{90}{\degree} phasenverschobene Signal wird einem multiplikativen Mischer zugeführt, der die gewünschte Oberschwingungen erzeugt.}
\end{QQuestion}

}
\only<2>{
\begin{QQuestion}{AF311}{Nach welchem Prinzip arbeitet die analoge Frequenzvervielfachung?}{Das jeweils um plus und minus \qty{90}{\degree} phasenverschobene Signal wird einem additiven Mischer zugeführt, der die gewünschte Oberschwingungen erzeugt.}
{\textbf{\textcolor{DARCgreen}{Das Signal wird einer nicht linearen Verzerrerstufe zugeführt und die gewünschte Oberschwingungen ausgefiltert.}}}
{Das Signal wird gefiltert und einem Ringmischer zugeführt, der die gewünschte Oberschwingungen erzeugt.}
{Das jeweils um plus und minus \qty{90}{\degree} phasenverschobene Signal wird einem multiplikativen Mischer zugeführt, der die gewünschte Oberschwingungen erzeugt.}
\end{QQuestion}

}
\end{frame}

\begin{frame}
\only<1>{
\begin{QQuestion}{AF313}{Wie sollten Frequenzvervielfacher in einer Sendeeinrichtung aufgebaut und betrieben werden?}{Sie sollten am Ausgang ein Hochpassfilter für das vervielfachte Signal besitzen.}
{Sie sollten gut abgeschirmt sein, um unerwünschte Abstrahlungen zu minimieren.}
{Sie sollten unbedingt im linearen Kennlinienabschnitt betrieben werden}
{Sie sollten sehr gut gekühlt werden.}
\end{QQuestion}

}
\only<2>{
\begin{QQuestion}{AF313}{Wie sollten Frequenzvervielfacher in einer Sendeeinrichtung aufgebaut und betrieben werden?}{Sie sollten am Ausgang ein Hochpassfilter für das vervielfachte Signal besitzen.}
{\textbf{\textcolor{DARCgreen}{Sie sollten gut abgeschirmt sein, um unerwünschte Abstrahlungen zu minimieren.}}}
{Sie sollten unbedingt im linearen Kennlinienabschnitt betrieben werden}
{Sie sollten sehr gut gekühlt werden.}
\end{QQuestion}

}
\end{frame}

\begin{frame}
\only<1>{
\begin{PQuestion}{AF312}{Worum handelt es sich bei dieser Schaltung?}{Oszillator}
{Frequenzvervielfacher}
{Frequenzteiler}
{Selbstschwingende Mischstufe}
{\DARCimage{1.0\linewidth}{489include}}\end{PQuestion}

}
\only<2>{
\begin{PQuestion}{AF312}{Worum handelt es sich bei dieser Schaltung?}{Oszillator}
{\textbf{\textcolor{DARCgreen}{Frequenzvervielfacher}}}
{Frequenzteiler}
{Selbstschwingende Mischstufe}
{\DARCimage{1.0\linewidth}{489include}}\end{PQuestion}

}
\end{frame}

\begin{frame}
\only<1>{
\begin{QQuestion}{AF314}{Ein quarzgesteuertes Funkgerät mit einer Ausgangsfrequenz von \qty{432}{\MHz} verursacht Störungen bei \qty{144}{\MHz}. Der Quarzoszillator des Funkgeräts schwingt auf einer Grundfrequenz bei \qty{12}{\MHz}.  Bei welcher Vervielfachungskombination kann die Störfrequenz von \qty{144}{\MHz} auftreten?  }{Grundfrequenz $\cdot 2 \cdot 3 \cdot 3 \cdot 2$}
{Grundfrequenz $\cdot 2 \cdot 2 \cdot 3 \cdot 3$}
{Grundfrequenz $\cdot 3 \cdot 3 \cdot 2\cdot 2$}
{Grundfrequenz $\cdot 3 \cdot 2 \cdot 3 \cdot 2$}
\end{QQuestion}

}
\only<2>{
\begin{QQuestion}{AF314}{Ein quarzgesteuertes Funkgerät mit einer Ausgangsfrequenz von \qty{432}{\MHz} verursacht Störungen bei \qty{144}{\MHz}. Der Quarzoszillator des Funkgeräts schwingt auf einer Grundfrequenz bei \qty{12}{\MHz}.  Bei welcher Vervielfachungskombination kann die Störfrequenz von \qty{144}{\MHz} auftreten?  }{Grundfrequenz $\cdot 2 \cdot 3 \cdot 3 \cdot 2$}
{\textbf{\textcolor{DARCgreen}{Grundfrequenz $\cdot 2 \cdot 2 \cdot 3 \cdot 3$}}}
{Grundfrequenz $\cdot 3 \cdot 3 \cdot 2\cdot 2$}
{Grundfrequenz $\cdot 3 \cdot 2 \cdot 3 \cdot 2$}
\end{QQuestion}

}
\end{frame}

\begin{frame}
\frametitle{Lösungsweg}
\begin{itemize}
  \item gegeben: $f_{Sender} = 432MHz$
  \item gegeben: $f_{Grund} = 12MHz$
  \item gegeben: $f_{QRM} = 144MHz$
  \item gesucht: Vervielfachungskombination
  \end{itemize}
    \pause
    $n = \frac{f_{Sender}}{f_{QRM}} = \frac{432MHz}{144MHz} = 3$
    \pause
    Es ist nur die Kombination aus $\textrm{Grundfrequenz}\,\cdot 2\cdot 2\cdot 3\cdot 3$ möglich, da diese als letzte eine Verdreifachung der Frequenz vornimmt.



\end{frame}

\begin{frame}Gegenprobe:

$$\begin{split}f_{Sender} &= f_{Grund}\cdot 2\cdot 2\cdot 3\cdot 3\\ &= 12MHz\cdot 2\cdot 2\cdot 3\cdot 3\\ &= 24MHz\cdot 2\cdot 3\cdot 3\\ &= 48MHz\cdot 3\cdot 3\\ &= \bold{144MHz}\cdot 3\\ &= 432MHz\end{split}\end{equation}

\end{frame}%ENDCONTENT
