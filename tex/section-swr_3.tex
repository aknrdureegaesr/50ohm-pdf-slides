
\section{Stehwellenverhältnis (SWR) III}
\label{section:swr_3}
\begin{frame}%STARTCONTENT

\only<1>{
\begin{QQuestion}{AG405}{Ein Kabel mit einem Wellenwiderstand von \qty{75}{\ohm} und vernachlässigbarer Dämpfung wird zur Speisung einer Faltdipol-Antenne verwendet. Welches SWR kann man auf der Leitung erwarten?}{ca. \num{1,5} bis \num{2}}
{\num{0,3}}
{ca. \num{3,2} bis \num{4}}
{\num{5,7}}
\end{QQuestion}

}
\only<2>{
\begin{QQuestion}{AG405}{Ein Kabel mit einem Wellenwiderstand von \qty{75}{\ohm} und vernachlässigbarer Dämpfung wird zur Speisung einer Faltdipol-Antenne verwendet. Welches SWR kann man auf der Leitung erwarten?}{ca. \num{1,5} bis \num{2}}
{\num{0,3}}
{\textbf{\textcolor{DARCgreen}{ca. \num{3,2} bis \num{4}}}}
{\num{5,7}}
\end{QQuestion}

}
\end{frame}

\begin{frame}
\frametitle{Lösungsweg}
\begin{itemize}
  \item gegeben: $Z = 75Ω$
  \item gegeben: $R_2 \approx 300Ω$ Widerstand Faltdipol
  \item gesucht: $s$
  \end{itemize}
    \pause
    $s = \frac{R_2}{Z} = \frac{300Ω}{75Ω} = 4$



\end{frame}

\begin{frame}
\only<1>{
\begin{QQuestion}{AG402}{Am Eingang einer angepassten HF-Übertragungsleitung werden \qty{100}{\W} HF-Leistung eingespeist. Die Dämpfung der Leitung beträgt \qty{3}{\decibel}. Welche Leistung wird bei Leerlauf oder Kurzschluss am Leitungsende reflektiert?}{\qty{25}{\W}}
{\qty{50}{\W}}
{\qty{50}{\W} bei Leerlauf und \qty{0}{\W} bei Kurzschluss}
{\qty{0}{\W} bei Leerlauf und \qty{50}{\W} bei Kurzschluss}
\end{QQuestion}

}
\only<2>{
\begin{QQuestion}{AG402}{Am Eingang einer angepassten HF-Übertragungsleitung werden \qty{100}{\W} HF-Leistung eingespeist. Die Dämpfung der Leitung beträgt \qty{3}{\decibel}. Welche Leistung wird bei Leerlauf oder Kurzschluss am Leitungsende reflektiert?}{\qty{25}{\W}}
{\textbf{\textcolor{DARCgreen}{\qty{50}{\W}}}}
{\qty{50}{\W} bei Leerlauf und \qty{0}{\W} bei Kurzschluss}
{\qty{0}{\W} bei Leerlauf und \qty{50}{\W} bei Kurzschluss}
\end{QQuestion}

}
\end{frame}

\begin{frame}
\only<1>{
\begin{QQuestion}{AG403}{In den Eingang einer Antennenleitung mit einer Dämpfung von \qty{3}{\decibel} werden \qty{10}{\W} HF-Leistung eingespeist. Mit der am Leitungsende angeschlossenen Antenne misst man am Leitungseingang ein SWR von 3. Mit einer künstlichen \qty{50}{\ohm}-Antenne am Leitungsende beträgt das SWR am Leitungseingang etwa 1. Was lässt sich aus diesen Messergebnissen schließen?}{Die Antenne ist fehlerhaft. Sie strahlt so gut wie keine HF-Leistung ab.}
{Die Antennenleitung ist fehlerhaft, an der Antenne kommt so gut wie keine HF-Leistung an.}
{Die Antennenanlage ist in Ordnung. Es werden etwa \qty{5}{\W} HF-Leistung abgestrahlt.}
{Die Antennenanlage ist in Ordnung. Es werden etwa \qty{3,75}{\W} HF-Leistung abgestrahlt.}
\end{QQuestion}

}
\only<2>{
\begin{QQuestion}{AG403}{In den Eingang einer Antennenleitung mit einer Dämpfung von \qty{3}{\decibel} werden \qty{10}{\W} HF-Leistung eingespeist. Mit der am Leitungsende angeschlossenen Antenne misst man am Leitungseingang ein SWR von 3. Mit einer künstlichen \qty{50}{\ohm}-Antenne am Leitungsende beträgt das SWR am Leitungseingang etwa 1. Was lässt sich aus diesen Messergebnissen schließen?}{\textbf{\textcolor{DARCgreen}{Die Antenne ist fehlerhaft. Sie strahlt so gut wie keine HF-Leistung ab.}}}
{Die Antennenleitung ist fehlerhaft, an der Antenne kommt so gut wie keine HF-Leistung an.}
{Die Antennenanlage ist in Ordnung. Es werden etwa \qty{5}{\W} HF-Leistung abgestrahlt.}
{Die Antennenanlage ist in Ordnung. Es werden etwa \qty{3,75}{\W} HF-Leistung abgestrahlt.}
\end{QQuestion}

}
\end{frame}

\begin{frame}
\only<1>{
\begin{QQuestion}{AG404}{Am Eingang einer Antennenleitung mit einer Dämpfung von \qty{5}{\decibel} werden \qty{10}{\W} HF-Leistung eingespeist. Mit der am Leitungsende angeschlossenen Antenne misst man am Leitungseingang ein SWR von 1. Welches SWR ist am Leitungseingang zu erwarten, wenn die Antenne abgeklemmt wird?}{Ein SWR von ca. 3,6}
{Ein SWR von ca. 1,92}
{Ein SWR von ca. 0, da sich vorlaufende und rücklaufende Leistung gegenseitig auslöschen}
{Ein SWR, das gegen unendlich geht, da am Ende der Leitung die gesamte HF-Leistung reflektiert wird}
\end{QQuestion}

}
\only<2>{
\begin{QQuestion}{AG404}{Am Eingang einer Antennenleitung mit einer Dämpfung von \qty{5}{\decibel} werden \qty{10}{\W} HF-Leistung eingespeist. Mit der am Leitungsende angeschlossenen Antenne misst man am Leitungseingang ein SWR von 1. Welches SWR ist am Leitungseingang zu erwarten, wenn die Antenne abgeklemmt wird?}{Ein SWR von ca. 3,6}
{\textbf{\textcolor{DARCgreen}{Ein SWR von ca. 1,92}}}
{Ein SWR von ca. 0, da sich vorlaufende und rücklaufende Leistung gegenseitig auslöschen}
{Ein SWR, das gegen unendlich geht, da am Ende der Leitung die gesamte HF-Leistung reflektiert wird}
\end{QQuestion}

}
\end{frame}

\begin{frame}
\frametitle{Lösungsweg}
\begin{itemize}
  \item gegeben: $P_V = 5W$
  \item gegeben: $a = 5dB$
  \item gesucht: $s$
  \end{itemize}
    \pause
    Dämpfung auf gesamtes Kabel für Hin- und Rückweg: 10dB

$P_R = 10dB \cdot P_V = \frac{5W}{10} = 0,5W$
    \pause
    $s = \frac{\sqrt{P_\mathrm{v}}+\sqrt{P_\mathrm{r}}}{\sqrt{P_\mathrm{v}}-\sqrt{P_\mathrm{r}}} = \frac{\sqrt{5W}+\sqrt{0,5W}}{\sqrt{5W}-\sqrt{0,5W}} = 1,92$



\end{frame}%ENDCONTENT
