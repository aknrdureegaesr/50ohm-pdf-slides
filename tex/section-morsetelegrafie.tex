
\section{Morsetelegrafie}
\label{section:morsetelegrafie}
\begin{frame}%STARTCONTENT
\begin{itemize}
  \item Ein- und Ausschalten eines Trägers
  \item Einführung eines \emph{Morsealphabets} 1838 durch Samuel Morse, optimiert durch Friedrich Clemens Gerke
  \item Morseprüfung lange Zeit Vorschrift für Funkamateure auf Kurzwelle
  \item Seit Mitte der 1990er legen Länder fest, ob \emph{Morseprüfung} notwendig ist
  \item Erst seit 2003 ist die Morseprüfung in Deutschland freiwillig
  \end{itemize}

\end{frame}

\begin{frame}\begin{table}
\begin{DARCtabular}{clclcl}
    ~  &~   &~   &~   &~   &~    \\
     A  & \MorseDit\MorseDah  & K  & \MorseDah\MorseDit\MorseDah  & U  & \MorseDit\MorseDit\MorseDah   \\
     B  & \MorseDah\MorseDit\MorseDit\MorseDit  & L  & \MorseDit\MorseDah\MorseDit\MorseDit  & V  & \MorseDit\MorseDit\MorseDit\MorseDah   \\
     C  & \MorseDah\MorseDit\MorseDah\MorseDit  & M  & \MorseDah\MorseDah  & W  & \MorseDit\MorseDah\MorseDah   \\
     D  & \MorseDah\MorseDit\MorseDit  & N  & \MorseDah\MorseDit  & X  & \MorseDah\MorseDit\MorseDit\MorseDah   \\
     E  & \MorseDit  & O  & \MorseDah\MorseDah\MorseDah  & Y  & \MorseDah\MorseDit\MorseDah\MorseDah   \\
     F  & \MorseDit\MorseDit\MorseDah\MorseDit  & P  & \MorseDit\MorseDah\MorseDah\MorseDit  & Z  & \MorseDah\MorseDah\MorseDit\MorseDit   \\
     G  & \MorseDah\MorseDah\MorseDit  & Q  & \MorseDah\MorseDah\MorseDit\MorseDah  & Ä  & \MorseDit\MorseDah\MorseDit\MorseDah   \\
     H  & \MorseDit\MorseDit\MorseDit\MorseDit  & R  & \MorseDit\MorseDah\MorseDit  & Ö  & \MorseDah\MorseDah\MorseDah\MorseDit   \\
     I  & \MorseDit\MorseDit  & S  & \MorseDit\MorseDit\MorseDit  & Ü  & \MorseDit\MorseDit\MorseDah\MorseDah   \\
     J  & \MorseDit\MorseDah\MorseDah\MorseDah  & T  & \MorseDah  & ẞ  & \MorseDit\MorseDit\MorseDit\MorseDah\MorseDah\MorseDit\MorseDit   \\
\end{DARCtabular}
\caption{Morsecode (Buchstaben)}
\label{n_morsetelegrafie_morsecode_buchstaben}
\end{table}
\end{frame}

\begin{frame}\begin{table}
\begin{DARCtabular}{clclcl}
    ~   &~   &~   &~   &~   &~    \\
     0  & \MorseDah\MorseDah\MorseDah\MorseDah\MorseDah  & 5  & \MorseDit\MorseDit\MorseDit\MorseDit\MorseDit  & /  & \MorseDah\MorseDit\MorseDit\MorseDah\MorseDit   \\
     1  & \MorseDit\MorseDah\MorseDah\MorseDah\MorseDah  & 6  & \MorseDah\MorseDit\MorseDit\MorseDit\MorseDit  & .  & \MorseDit\MorseDah\MorseDit\MorseDah\MorseDit\MorseDah   \\
     2  & \MorseDit\MorseDit\MorseDah\MorseDah\MorseDah  & 7  & \MorseDah\MorseDah\MorseDit\MorseDit\MorseDit  & ,  & \MorseDah\MorseDah\MorseDit\MorseDit\MorseDah\MorseDah   \\
     3  & \MorseDit\MorseDit\MorseDit\MorseDah\MorseDah  & 8  & \MorseDah\MorseDah\MorseDah\MorseDit\MorseDit  & ?  & \MorseDit\MorseDit\MorseDah\MorseDah\MorseDit\MorseDit   \\
     4  & \MorseDit\MorseDit\MorseDit\MorseDit\MorseDah  & 9  & \MorseDah\MorseDah\MorseDah\MorseDah\MorseDit  & =  & \MorseDah\MorseDit\MorseDit\MorseDit\MorseDah   \\
\end{DARCtabular}
\caption{Morsecode (Ziffern und Satzzeichen)}
\label{n_morsetelegrafie_morsecode_ziffern_satzzeichen}
\end{table}
\begin{table}
\begin{DARCtabular}{ll}
    ~   &~    \\
     Unterbrechung (BK)  & \MorseDah\MorseDit\MorseDit\MorseDit\MorseDah\MorseDit\MorseDah   \\
     Ende des Durchgangs (AR)   & \MorseDit\MorseDah\MorseDit\MorseDah\MorseDit   \\
     Ende der Sendung (SK)  & \MorseDit\MorseDit\MorseDit\MorseDah\MorseDit\MorseDah   \\
     Korrektur  & \MorseDit\MorseDit\MorseDit\MorseDit\MorseDit\MorseDit\MorseDit\MorseDit   \\
\end{DARCtabular}
\caption{Morsecode (besondere Zeichen, Auswahl)}
\label{n_morsetelegrafie_morsecode_spezial}
\end{table}
\end{frame}

\begin{frame}
\only<1>{
\begin{QQuestion}{VA304}{Was ist in den Radio Regulations (RR) bezüglich der Morsequalifikation für Funkamateure festgelegt?}{Wer Frequenzen unter \qty{30}{\MHz} nutzen will, muss eine Morseprüfung ablegen.}
{Bei einer Sendeleistung von mehr als \qty{100}{\W} benötigt der Funkamateur den Nachweis einer erfolgreich abgelegten Morseprüfung.}
{Die nationale Verwaltung eines jeden Landes legt eigenständig fest, ob eine Morseprüfung erforderlich ist.}
{In den Radio Regulations (RR) werden bezüglich der Morsequalifikation keine Regelungen getroffen.}
\end{QQuestion}

}
\only<2>{
\begin{QQuestion}{VA304}{Was ist in den Radio Regulations (RR) bezüglich der Morsequalifikation für Funkamateure festgelegt?}{Wer Frequenzen unter \qty{30}{\MHz} nutzen will, muss eine Morseprüfung ablegen.}
{Bei einer Sendeleistung von mehr als \qty{100}{\W} benötigt der Funkamateur den Nachweis einer erfolgreich abgelegten Morseprüfung.}
{\textbf{\textcolor{DARCgreen}{Die nationale Verwaltung eines jeden Landes legt eigenständig fest, ob eine Morseprüfung erforderlich ist.}}}
{In den Radio Regulations (RR) werden bezüglich der Morsequalifikation keine Regelungen getroffen.}
\end{QQuestion}

}
\end{frame}%ENDCONTENT
