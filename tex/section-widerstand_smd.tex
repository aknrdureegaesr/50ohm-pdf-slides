
\section{SMD-Widerstände}
\label{section:widerstand_smd}
\begin{frame}%STARTCONTENT

\begin{columns}
    \begin{column}{0.48\textwidth}
    \begin{itemize}
  \item SMD: Surface Mounted Device
  \item Widerstand in sehr kleiner Bauform
  \item Letzte Stelle des aufgedruckten Widerstandswerts gibt die Zehnerpotenz an
  \end{itemize}

    \end{column}
   \begin{column}{0.48\textwidth}
       
\begin{figure}
    \DARCimage{0.85\linewidth}{529include}
    \caption{\scriptsize SMD-Widerstand}
    \label{e_smd_widerstand}
\end{figure}


   \end{column}
\end{columns}

\end{frame}

\begin{frame}
\only<1>{
\begin{QQuestion}{EC114}{Wie wird in der Regel bei SMD-Widerständen der Widerstandswert angegeben?}{Auf dem Widerstand ist der Wert in Form von Zahlen abgedruckt, wobei die letzte Ziffer die Zehnerpotenz angibt.}
{Auf dem Widerstand ist der Wert in Form von Farbringen aufgedruckt, wobei der letzte Farbring die Toleranz angibt.}
{Auf dem Widerstand ist der Wert in Form von Farbringen aufgedruckt, wobei der letzte Farbring die Zehnerpotenz angibt.}
{Auf dem Widerstand ist der Wert in Form von Zahlen abgedruckt, wobei die angegebene Zahl dem Wert des Widerstands entspricht.}
\end{QQuestion}

}
\only<2>{
\begin{QQuestion}{EC114}{Wie wird in der Regel bei SMD-Widerständen der Widerstandswert angegeben?}{\textbf{\textcolor{DARCgreen}{Auf dem Widerstand ist der Wert in Form von Zahlen abgedruckt, wobei die letzte Ziffer die Zehnerpotenz angibt.}}}
{Auf dem Widerstand ist der Wert in Form von Farbringen aufgedruckt, wobei der letzte Farbring die Toleranz angibt.}
{Auf dem Widerstand ist der Wert in Form von Farbringen aufgedruckt, wobei der letzte Farbring die Zehnerpotenz angibt.}
{Auf dem Widerstand ist der Wert in Form von Zahlen abgedruckt, wobei die angegebene Zahl dem Wert des Widerstands entspricht.}
\end{QQuestion}

}
\end{frame}

\begin{frame}
\only<1>{
\begin{PQuestion}{EC115}{Welchen Wert hat der dargestellte SMD-Widerstand?}{\qty{103}{\ohm}}
{\qty{10}{\kohm}}
{\qty{1}{\kohm}}
{\qty{10,3}{\ohm}}
{\DARCimage{0.5\linewidth}{529include}}\end{PQuestion}

}
\only<2>{
\begin{PQuestion}{EC115}{Welchen Wert hat der dargestellte SMD-Widerstand?}{\qty{103}{\ohm}}
{\textbf{\textcolor{DARCgreen}{\qty{10}{\kohm}}}}
{\qty{1}{\kohm}}
{\qty{10,3}{\ohm}}
{\DARCimage{0.5\linewidth}{529include}}\end{PQuestion}

}
\end{frame}

\begin{frame}
\only<1>{
\begin{QQuestion}{EC116}{Welchen Wert hat ein SMD-Widerstand mit der Kennzeichnung 221?}{\qty{220}{\ohm}}
{\qty{221}{\ohm}}
{\qty{22,0}{\ohm}}
{\qty{22,1}{\ohm}}
\end{QQuestion}

}
\only<2>{
\begin{QQuestion}{EC116}{Welchen Wert hat ein SMD-Widerstand mit der Kennzeichnung 221?}{\textbf{\textcolor{DARCgreen}{\qty{220}{\ohm}}}}
{\qty{221}{\ohm}}
{\qty{22,0}{\ohm}}
{\qty{22,1}{\ohm}}
\end{QQuestion}

}
\end{frame}

\begin{frame}
\only<1>{
\begin{QQuestion}{EC117}{Welchen Wert hat ein SMD-Widerstand mit der Kennzeichnung 223?}{\qty{223}{\ohm}}
{\qty{22}{\kohm}}
{\qty{22,3}{\kohm}}
{\qty{220}{\ohm}}
\end{QQuestion}

}
\only<2>{
\begin{QQuestion}{EC117}{Welchen Wert hat ein SMD-Widerstand mit der Kennzeichnung 223?}{\qty{223}{\ohm}}
{\textbf{\textcolor{DARCgreen}{\qty{22}{\kohm}}}}
{\qty{22,3}{\kohm}}
{\qty{220}{\ohm}}
\end{QQuestion}

}
\end{frame}%ENDCONTENT
