
\section{Zehnerpotenzen}
\label{section:zehnerpotenzen}
\begin{frame}%STARTCONTENT

\frametitle{Große und kleine Werte}
\begin{itemize}
  \item Im Amateurfunk haben wir große und kleine Werte
  \item Um sich viele 0-en zu sparen, wurde bereits mit Einheitenvorsätzen abgekürzt, z.B. mit Milli (m) oder Kilo (k)
  \end{itemize}
\end{frame}

\begin{frame}
\frametitle{Zehnerpotenzen}
\begin{itemize}
  \item Einheitenvorsätze lassen sich in den meisten Taschenrechnern nicht direkt eingeben
  \item Stattdessen wird die Zehnerpotenz verwendet
  \item Kilo entspricht 1000 oder 10 $\cdot$ 10 $\cdot$ 10
  \item Abgekürzt $10^3$
  \end{itemize}
    \pause
    \vspace{1cm}
    \begin{exampleblock}{Beispiel}
        $\qty{1500}{\hertz} \rightarrow \qty{1,5}{\kilo\hertz} \rightarrow 1,5 \cdot 10^3~\qty{}{\hertz}$
        $\qty{1500000}{\hertz} \rightarrow \qty{1,5}{\mega\hertz} \rightarrow 1,5 \cdot 10^6~\qty{}{\hertz}$
    \end{exampleblock}
\end{frame}

\begin{frame}\begin{itemize}
  \item Milli entspricht  $\frac{1}{1000}$ oder $\frac{1}{10 \cdot 10 \cdot 10}$
  \item Abgekürzt 10<sup>-3</sup>
  \end{itemize}
    \pause
    \begin{exampleblock}{Beispiel}
        $\qty{0,0035}{\volt} \rightarrow \qty{3,5}{\milli\volt} \rightarrow 3,5 \cdot 10^{-3}~\qty{}{\volt}$
    \end{exampleblock}
\end{frame}

\begin{frame}
\frametitle{Einheitenvorsätze und Zehnerpotenzen}
\begin{table}
\begin{DARCtabular}{ccl}
     Bezeichnung  & Abkürzung  & Wert & ~             \\
     Pico         & p  & $10^{-12}$ & = 0,000000000001 \\
     Nano         & n  & $10^{-9} $ & = 0,000000001    \\
     Mikro        & µ  & $10^{-6} $ & = 0,000001       \\
     Milli        & m  & $10^{-3} $ & = 0,001          \\
                  &    & $10^{0}  $ & = 1              \\
     Kilo         & k  & $10^{3}  $ & = 1000           \\
     Mega         & M  & $10^{6}  $ & = 1000000        \\
     Giga         & G  & $10^{9}  $ & = 1000000000     \\
\end{DARCtabular}
\caption{Einheitenvorsätze für Zehnerpotenzen}
\label{e_einheitenvorzeichen}
\end{table}
\end{frame}

\begin{frame}
\frametitle{Taschenrechner}
\begin{columns}
    \begin{column}{0.73\textwidth}
    \begin{itemize}
  \item Taste \emph{EXP} oder $\cdot 10^x$
  \item Eintippen: \enquote{145,3 Exp 6}
  \item Taste \emph{ENG} verschiebt den Exponent um 3
  \end{itemize}

    \end{column}
   \begin{column}{0.25\textwidth}
       
\begin{figure}
    \includegraphics[width=0.85\textwidth]{foto/172}
    \caption{\scriptsize Verschiedene Darstellungen der Zahl 0,007 in einer Taschenrechner-App}
    \label{e_taschenrechner}
\end{figure}

   \end{column}
\end{columns}

\end{frame}

\begin{frame}
\only<1>{
\begin{QQuestion}{EA108}{\qty{0,00042}{\A} entspricht~...}{$420\cdot 10^6$ A.}
{$420\cdot 10^{-6}$ A.}
{$420\cdot 10^{-5}$ A.}
{$42\cdot 10^{-6}$ A.}
\end{QQuestion}

}
\only<2>{
\begin{QQuestion}{EA108}{\qty{0,00042}{\A} entspricht~...}{$420\cdot 10^6$ A.}
{\textbf{\textcolor{DARCgreen}{$420\cdot 10^{-6}$ A.}}}
{$420\cdot 10^{-5}$ A.}
{$42\cdot 10^{-6}$ A.}
\end{QQuestion}

}
\end{frame}

\begin{frame}
\only<1>{
\begin{QQuestion}{EA109}{\qty{0,042}{\A} entspricht~...}{$42\cdot 10^{-1}$ A.}
{$42\cdot 10^3$ A.}
{$42\cdot 10^{-2}$ A.}
{$42\cdot 10^{-3}$ A.}
\end{QQuestion}

}
\only<2>{
\begin{QQuestion}{EA109}{\qty{0,042}{\A} entspricht~...}{$42\cdot 10^{-1}$ A.}
{$42\cdot 10^3$ A.}
{$42\cdot 10^{-2}$ A.}
{\textbf{\textcolor{DARCgreen}{$42\cdot 10^{-3}$ A.}}}
\end{QQuestion}

}
\end{frame}

\begin{frame}
\only<1>{
\begin{QQuestion}{EA110}{\qty{4200000}{\Hz} entspricht~...}{$4,2\cdot 10^5$ Hz.}
{$4,2\cdot 10^6$ Hz.}
{$42\cdot 10^6$ Hz.}
{$42\cdot 10^{-5}$ Hz.}
\end{QQuestion}

}
\only<2>{
\begin{QQuestion}{EA110}{\qty{4200000}{\Hz} entspricht~...}{$4,2\cdot 10^5$ Hz.}
{\textbf{\textcolor{DARCgreen}{$4,2\cdot 10^6$ Hz.}}}
{$42\cdot 10^6$ Hz.}
{$42\cdot 10^{-5}$ Hz.}
\end{QQuestion}

}
\end{frame}

\begin{frame}
\only<1>{
\begin{QQuestion}{EA111}{\qty{0,01}{\mV} entspricht~...}{$1\cdot 10^{-7}$ V.}
{$10\cdot 10^{-6}$ V.}
{$10\cdot 10^{-5}$ V.}
{$0{,}01\cdot 10^{3}$ V.}
\end{QQuestion}

}
\only<2>{
\begin{QQuestion}{EA111}{\qty{0,01}{\mV} entspricht~...}{$1\cdot 10^{-7}$ V.}
{\textbf{\textcolor{DARCgreen}{$10\cdot 10^{-6}$ V.}}}
{$10\cdot 10^{-5}$ V.}
{$0{,}01\cdot 10^{3}$ V.}
\end{QQuestion}

}
\end{frame}

\begin{frame}
\only<1>{
\begin{QQuestion}{EA112}{\qty{0,002}{\Mohm} entspricht~...}{$20\cdot 10^{3} \Omega$.}
{$2\cdot 10^{3} \Omega$.}
{$2\cdot 10^{2} \Omega$.}
{$2000\cdot 10^{2} \Omega$.}
\end{QQuestion}

}
\only<2>{
\begin{QQuestion}{EA112}{\qty{0,002}{\Mohm} entspricht~...}{$20\cdot 10^{3} \Omega$.}
{\textbf{\textcolor{DARCgreen}{$2\cdot 10^{3} \Omega$.}}}
{$2\cdot 10^{2} \Omega$.}
{$2000\cdot 10^{2} \Omega$.}
\end{QQuestion}

}
\end{frame}

\begin{frame}
\only<1>{
\begin{QQuestion}{EA113}{$2\cdot 10^{-7}$ W entspricht~...}{\qty{20}{\micro\W}.}
{\qty{2}{\micro\W}.}
{\qty{0,2}{\micro\W}.}
{\qty{200}{\micro\W}.}
\end{QQuestion}

}
\only<2>{
\begin{QQuestion}{EA113}{$2\cdot 10^{-7}$ W entspricht~...}{\qty{20}{\micro\W}.}
{\qty{2}{\micro\W}.}
{\textbf{\textcolor{DARCgreen}{\qty{0,2}{\micro\W}.}}}
{\qty{200}{\micro\W}.}
\end{QQuestion}

}
\end{frame}

\begin{frame}
\only<1>{
\begin{QQuestion}{EA114}{$5 \cdot 10^{-1}$ W entspricht~...}{\qty{500}{\mW}.}
{\qty{5}{\W}.}
{\qty{-500}{\mW}.}
{\qty{-5}{\W}.}
\end{QQuestion}

}
\only<2>{
\begin{QQuestion}{EA114}{$5 \cdot 10^{-1}$ W entspricht~...}{\textbf{\textcolor{DARCgreen}{\qty{500}{\mW}.}}}
{\qty{5}{\W}.}
{\qty{-500}{\mW}.}
{\qty{-5}{\W}.}
\end{QQuestion}

}
\end{frame}

\begin{frame}
\only<1>{
\begin{QQuestion}{EA115}{0,22~μF entspricht~...}{\qty{22}{\pF}.}
{\qty{22}{\nF}.}
{\qty{220}{\pF}.}
{\qty{220}{\nF}.}
\end{QQuestion}

}
\only<2>{
\begin{QQuestion}{EA115}{0,22~μF entspricht~...}{\qty{22}{\pF}.}
{\qty{22}{\nF}.}
{\qty{220}{\pF}.}
{\textbf{\textcolor{DARCgreen}{\qty{220}{\nF}.}}}
\end{QQuestion}

}
\end{frame}

\begin{frame}
\only<1>{
\begin{QQuestion}{EA116}{\qty{3750}{\kHz} entspricht~...}{\qty{37500000}{\Hz}.}
{\qty{3,750}{\MHz}.}
{\qty{0,03750}{\GHz}.}
{\qty{0,3750}{\GHz}.}
\end{QQuestion}

}
\only<2>{
\begin{QQuestion}{EA116}{\qty{3750}{\kHz} entspricht~...}{\qty{37500000}{\Hz}.}
{\textbf{\textcolor{DARCgreen}{\qty{3,750}{\MHz}.}}}
{\qty{0,03750}{\GHz}.}
{\qty{0,3750}{\GHz}.}
\end{QQuestion}

}
\end{frame}%ENDCONTENT
