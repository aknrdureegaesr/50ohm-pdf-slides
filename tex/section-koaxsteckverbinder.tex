
\section{Koaxialsteckverbinder}
\label{section:koaxsteckverbinder}
\begin{frame}%STARTCONTENT

\begin{columns}
    \begin{column}{0.48\textwidth}
    \begin{itemize}
  \item Bestehen aus Innen- und Außenleiter
  \item Außengehäuse mit Außenleiter verbunden
  \item Innenleiter mit Kontaktstift oder Kontaktöffnung verbunden
  \item Verbindung durch Löten oder Crimpen
  \end{itemize}

    \end{column}
   \begin{column}{0.48\textwidth}
       \begin{itemize}
  \item \emph{Stecker}: Kontaktstift nach außen
  \item \emph{Kupplung}: Kontaktöffnung nach innen
  \item Sonderform \emph{Buchse}: In Gerät eingebaute Kupplung
  \end{itemize}

   \end{column}
\end{columns}

\end{frame}

\begin{frame}Häufige Koaxialsteckverbinder im Amateurfunk
\begin{columns}
    \begin{column}{0.48\textwidth}
    \begin{itemize}
  \item PL
  \item N
  \end{itemize}

    \end{column}
   \begin{column}{0.48\textwidth}
       \begin{itemize}
  \item BNC
  \item SMA
  \end{itemize}

   \end{column}
\end{columns}

\end{frame}

\begin{frame}
\frametitle{Hinweise zur Verwendung}
\begin{columns}
    \begin{column}{0.48\textwidth}
    \begin{itemize}
  \item Sorgsamer Umgang
  \item Fest verschrauben
  \item Innenleiter kann brechen
  \item Schirmung kann verrutschen
  \item Ggf. auf Kurzschluss prüfen
  \end{itemize}

    \end{column}
   \begin{column}{0.48\textwidth}
       \begin{itemize}
  \item Stecker passend zu Kabelstärke verwenden
  \item Stecker passend zu Kabeldurchmesser verwenden
  \end{itemize}

   \end{column}
\end{columns}

\end{frame}%ENDCONTENT
