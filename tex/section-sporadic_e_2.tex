
\section{Sporadic-E II}
\label{section:sporadic_e_2}
\begin{frame}%STARTCONTENT

\begin{columns}
    \begin{column}{0.48\textwidth}
    \begin{itemize}
  \item Regional begrenzte ungewöhnlich hohe Ionisation der E-Schicht
  \item Refraktion (Brechung) von Funkwellen in VHF und UHF
  \item Auch \qty{10}{\metre}-Band möglich
  \end{itemize}

    \end{column}
   \begin{column}{0.48\textwidth}
       
\begin{figure}
    \DARCimage{0.85\linewidth}{731include}
    \caption{\scriptsize Für den Amateurfunk relevante Schichten in der Atmosphäre}
    \label{e_atmosphaeren_schichten}
\end{figure}


   \end{column}
\end{columns}

\end{frame}

\begin{frame}
\only<1>{
\begin{QQuestion}{EH304}{Was verstehen Sie unter dem Begriff \glqq Sporadic-E\grqq{}?}{Kurzzeitig auftretende starke Reflexion von VHF-Signalen an Meteorbahnen innerhalb der E-Region.}
{Kurzfristige plötzliche Inversionsänderungen in der E-Region, die Fernausbreitung im VHF-Bereich ermöglichen.}
{Die Refraktion (Brechung) in lokal begrenzten Bereichen mit ungewöhnlich hoher Ionisation innerhalb der E-Region.}
{Lokal begrenzten kurzzeitigen Ausfall der Reflexion durch ungewöhnlich hohe Ionisation innerhalb der E-Region.}
\end{QQuestion}

}
\only<2>{
\begin{QQuestion}{EH304}{Was verstehen Sie unter dem Begriff \glqq Sporadic-E\grqq{}?}{Kurzzeitig auftretende starke Reflexion von VHF-Signalen an Meteorbahnen innerhalb der E-Region.}
{Kurzfristige plötzliche Inversionsänderungen in der E-Region, die Fernausbreitung im VHF-Bereich ermöglichen.}
{\textbf{\textcolor{DARCgreen}{Die Refraktion (Brechung) in lokal begrenzten Bereichen mit ungewöhnlich hoher Ionisation innerhalb der E-Region.}}}
{Lokal begrenzten kurzzeitigen Ausfall der Reflexion durch ungewöhnlich hohe Ionisation innerhalb der E-Region.}
\end{QQuestion}

}
\end{frame}

\begin{frame}
\frametitle{Short Skip}
\begin{columns}
    \begin{column}{0.48\textwidth}
    \begin{itemize}
  \item Funkverbindungen mit Sprungentfernungen unter \qty{1000}{\kilo\metre}
  \item Durch Refraktion an einer Sporadic-E-Schicht
  \item Insbesondere im \qty{10}{\metre}-Band
  \end{itemize}

    \end{column}
   \begin{column}{0.48\textwidth}
       
\begin{figure}
    \DARCimage{0.85\linewidth}{733include}
    \caption{\scriptsize Refraktion bei Sporadic-E}
    \label{e_sporadic_e}
\end{figure}


   \end{column}
\end{columns}

\end{frame}

\begin{frame}
\only<1>{
\begin{QQuestion}{EH218}{Unter dem Begriff \glqq Short Skip\grqq{} versteht man Funkverbindungen besonders im \qty{10}{\m}-Band mit Sprungentfernungen unter \qty{1000}{\km}, die~...}{durch Refraktion (Brechung) in sporadischen E-Regionen ermöglicht werden.}
{bei entsprechendem Abstrahlwinkel durch Refraktion (Brechung) in der F1-Region ermöglicht werden.}
{bei entsprechendem Abstrahlwinkel durch Refraktion (Brechung) in der F2-Region ermöglicht werden.}
{durch Refraktion (Brechung) in der hochionisierten D-Region ermöglicht werden.}
\end{QQuestion}

}
\only<2>{
\begin{QQuestion}{EH218}{Unter dem Begriff \glqq Short Skip\grqq{} versteht man Funkverbindungen besonders im \qty{10}{\m}-Band mit Sprungentfernungen unter \qty{1000}{\km}, die~...}{\textbf{\textcolor{DARCgreen}{durch Refraktion (Brechung) in sporadischen E-Regionen ermöglicht werden.}}}
{bei entsprechendem Abstrahlwinkel durch Refraktion (Brechung) in der F1-Region ermöglicht werden.}
{bei entsprechendem Abstrahlwinkel durch Refraktion (Brechung) in der F2-Region ermöglicht werden.}
{durch Refraktion (Brechung) in der hochionisierten D-Region ermöglicht werden.}
\end{QQuestion}

}
\end{frame}%ENDCONTENT
