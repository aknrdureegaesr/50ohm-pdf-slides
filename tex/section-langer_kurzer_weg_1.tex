
\section{Langer und kurzer Weg I}
\label{section:langer_kurzer_weg_1}
\begin{frame}%STARTCONTENT
\begin{itemize}
  \item Durch die Kugelform der Erde kann ein Ziel geradlinig über zwei Wege erreicht werden
  \item Funkwellen können sich je nach Ausbreitungsbedingungen besser über den längeren, indirekten Weg ausbreiten
  \end{itemize}
\end{frame}

\begin{frame}
\only<1>{
\begin{QQuestion}{EH217}{Was bedeutet die Aussage, dass ein Funkamateur in Deutschland mit \glqq VK\grqq{} auf dem \glqq langen Weg\grqq{} gearbeitet hat?}{Der Verbindungsweg mit Australien ist wegen der schlechten Ausbreitungsbedingungen erst nach langer Wartezeit zustande gekommen.}
{Die Verbindung mit Australien ist wegen der Ausbreitungsbedingungen auf langem direktem Weg über Südamerika hinweg zustande gekommen.}
{Die Verbindung mit Südamerika ist wegen der Ausbreitungsbedingungen auf dem indirekten und somit längeren Weg über Australien hinweg zustande gekommen.}
{Die Verbindung mit Australien ist wegen der Ausbreitungsbedingungen auf dem indirekten und somit längeren Weg über Südamerika hinweg zustande gekommen.}
\end{QQuestion}

}
\only<2>{
\begin{QQuestion}{EH217}{Was bedeutet die Aussage, dass ein Funkamateur in Deutschland mit \glqq VK\grqq{} auf dem \glqq langen Weg\grqq{} gearbeitet hat?}{Der Verbindungsweg mit Australien ist wegen der schlechten Ausbreitungsbedingungen erst nach langer Wartezeit zustande gekommen.}
{Die Verbindung mit Australien ist wegen der Ausbreitungsbedingungen auf langem direktem Weg über Südamerika hinweg zustande gekommen.}
{Die Verbindung mit Südamerika ist wegen der Ausbreitungsbedingungen auf dem indirekten und somit längeren Weg über Australien hinweg zustande gekommen.}
{\textbf{\textcolor{DARCgreen}{Die Verbindung mit Australien ist wegen der Ausbreitungsbedingungen auf dem indirekten und somit längeren Weg über Südamerika hinweg zustande gekommen.}}}
\end{QQuestion}

}

\end{frame}

\begin{frame}
\only<1>{
\begin{QQuestion}{EH216}{Was ist mit der Aussage \glqq Funkverkehr über den langen Weg (long path)\grqq{} gemeint?}{Die Funkverbindung läuft nicht über den direkten Weg zur Gegenstation, sondern über die dem kürzesten Weg entgegengesetzte Richtung.}
{Bei guten Ausbreitungsbedingungen treten mehrfache Refraktionen (Brechungen) mit vielen Sprüngen (hops) auf. Dann ist es möglich, sehr weite Entfernungen - \glqq lange Wege\grqq{} - zu überbrücken.}
{Bei guten Ausbreitungsbedingungen treten mehrfache Refraktionen (Brechungen) mit vielen Sprüngen (hops) auf. Sie hören dann Ihre eigenen Zeichen zeitverzögert als \glqq Echo\grqq{} im Empfänger wieder. Sie laufen also den \glqq langen Weg einmal um die Erde\grqq{}.}
{Bei sehr guten Ausbreitungsbedingungen liegen die reflektierenden Regionen in großer Höhe. Die Sprungdistanzen werden dann sehr groß, so dass sie die Reichweite der Bodenwelle um ein Vielfaches übertreffen. Dann kann man mit einem Sprung einen \glqq sehr langen Weg\grqq{} zurücklegen.}
\end{QQuestion}

}
\only<2>{
\begin{QQuestion}{EH216}{Was ist mit der Aussage \glqq Funkverkehr über den langen Weg (long path)\grqq{} gemeint?}{\textbf{\textcolor{DARCgreen}{Die Funkverbindung läuft nicht über den direkten Weg zur Gegenstation, sondern über die dem kürzesten Weg entgegengesetzte Richtung.}}}
{Bei guten Ausbreitungsbedingungen treten mehrfache Refraktionen (Brechungen) mit vielen Sprüngen (hops) auf. Dann ist es möglich, sehr weite Entfernungen - \glqq lange Wege\grqq{} - zu überbrücken.}
{Bei guten Ausbreitungsbedingungen treten mehrfache Refraktionen (Brechungen) mit vielen Sprüngen (hops) auf. Sie hören dann Ihre eigenen Zeichen zeitverzögert als \glqq Echo\grqq{} im Empfänger wieder. Sie laufen also den \glqq langen Weg einmal um die Erde\grqq{}.}
{Bei sehr guten Ausbreitungsbedingungen liegen die reflektierenden Regionen in großer Höhe. Die Sprungdistanzen werden dann sehr groß, so dass sie die Reichweite der Bodenwelle um ein Vielfaches übertreffen. Dann kann man mit einem Sprung einen \glqq sehr langen Weg\grqq{} zurücklegen.}
\end{QQuestion}

}
\end{frame}%ENDCONTENT
