
\section{Pile-up}
\label{section:pileup}
\begin{frame}%STARTCONTENT

\frametitle{Was ist ein Pile-Up?}
\begin{itemize}
  \item Der Begriff Pile-Up kommt vom englischen Verb \enquote{to pile up}, was soviel bedeutet wie \enquote{sich aufstapeln}.
  \item Was sich hier aufstapelt, sind die immer mehr werdenden Funkamateure, die alle das gleiche Ziel verfolgen: Eine Verbindung mit der begehrten Station herzustellen.
  \end{itemize}

\end{frame}

\begin{frame}
\frametitle{Umgang mit dem Pile-Up 1}
\begin{itemize}
  \item Manchmal wird der CQ-Ruf auf bestimmte Ziffern im Rufzeichen beschränkt, zum Beispiel \enquote{\emph{only number}~5}.
  \item Oder der Aufruf wird nach Ländern oder Kontinenten vorgenommen, zum Beispiel \enquote{Asia only}
  \item Selten wird \emph{Listenbetrieb} gemacht: Eine andere Station nimmt die Kontaktwünsche auf und erstellt eine Liste, die von der begehrten Station abgearbeitet wird.
  \end{itemize}
\end{frame}

\begin{frame}
\only<1>{
\begin{QQuestion}{BE305}{Was ist mit dem Begriff \glqq pile up\grqq{} gemeint? Im Amateurfunk meint man damit das gleichzeitige~...}{Anrufen einer begehrten Station durch viele Amateurfunkstellen.}
{Senden einer Station auf mehreren Amateurfunkfrequenzen.}
{Peilen einer Station mit mehreren übereinander angeordneten Richtantennen.}
{Hören einer Station mit vielen Remote-Stationen bei einem Contest.}
\end{QQuestion}

}
\only<2>{
\begin{QQuestion}{BE305}{Was ist mit dem Begriff \glqq pile up\grqq{} gemeint? Im Amateurfunk meint man damit das gleichzeitige~...}{\textbf{\textcolor{DARCgreen}{Anrufen einer begehrten Station durch viele Amateurfunkstellen.}}}
{Senden einer Station auf mehreren Amateurfunkfrequenzen.}
{Peilen einer Station mit mehreren übereinander angeordneten Richtantennen.}
{Hören einer Station mit vielen Remote-Stationen bei einem Contest.}
\end{QQuestion}

}
\end{frame}

\begin{frame}
\only<1>{
\begin{QQuestion}{BE306}{Eine begehrte Station ruft in Telefonie \glqq only number 3\grqq{}. Was ist damit gemeint? Die Station~...}{möchte, dass anrufende Stationen dreimal ihren Suffix durchgeben.}
{möchte jeweils drei rufende Stationen in eine Liste aufnehmen.}
{möchte Stationen mit dreistelligem Suffix aufrufen.}
{möchte Anrufe von Stationen mit der Ziffer 3 zwischen Präfix und Suffix.}
\end{QQuestion}

}
\only<2>{
\begin{QQuestion}{BE306}{Eine begehrte Station ruft in Telefonie \glqq only number 3\grqq{}. Was ist damit gemeint? Die Station~...}{möchte, dass anrufende Stationen dreimal ihren Suffix durchgeben.}
{möchte jeweils drei rufende Stationen in eine Liste aufnehmen.}
{möchte Stationen mit dreistelligem Suffix aufrufen.}
{\textbf{\textcolor{DARCgreen}{möchte Anrufe von Stationen mit der Ziffer 3 zwischen Präfix und Suffix.}}}
\end{QQuestion}

}
\end{frame}

\begin{frame}
\only<1>{
\begin{QQuestion}{BE307}{Was verstehen Sie bei einer seltenen Station unter der Aufforderung zu \glqq Listenbetrieb\grqq{}?}{Eine gut hörbare andere Station nimmt anrufende Stationen in eine Liste und ruft später diese Stationen zur Aufnahme einer Funkverbindung mit der seltenen Station auf.}
{Eine gut hörbare andere Station schickt per Internet Listen anrufender Stationen an die seltene Station.}
{Die seltene Station ruft Stationen nach einer Liste der Landeskenner alphabetisch auf.}
{Die seltene Station oder ihr QSL-Manager veröffentlicht eine Liste der gearbeiteten Stationen in den Amateurfunkzeitschriften.}
\end{QQuestion}

}
\only<2>{
\begin{QQuestion}{BE307}{Was verstehen Sie bei einer seltenen Station unter der Aufforderung zu \glqq Listenbetrieb\grqq{}?}{\textbf{\textcolor{DARCgreen}{Eine gut hörbare andere Station nimmt anrufende Stationen in eine Liste und ruft später diese Stationen zur Aufnahme einer Funkverbindung mit der seltenen Station auf.}}}
{Eine gut hörbare andere Station schickt per Internet Listen anrufender Stationen an die seltene Station.}
{Die seltene Station ruft Stationen nach einer Liste der Landeskenner alphabetisch auf.}
{Die seltene Station oder ihr QSL-Manager veröffentlicht eine Liste der gearbeiteten Stationen in den Amateurfunkzeitschriften.}
\end{QQuestion}

}
\end{frame}%ENDCONTENT
