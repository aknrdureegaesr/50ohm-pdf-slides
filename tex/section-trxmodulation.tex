
\section{Modulationseinstellungen am Funkgerät}
\label{section:trxmodulation}
\begin{frame}%STARTCONTENT
\begin{itemize}
  \item An vielen Funkgeräten gibt es einen Schalter, um die Modulationsart auszuwählen
  \item Meistens ist dieser mit „Mode“ beschriftet und erlaubt beispielsweise zwischen CW, AM, FM und SSB zu wählen
  \end{itemize}
\end{frame}

\begin{frame}
\only<1>{
\begin{QQuestion}{NE102}{In welcher der folgenden Antwortmöglichkeiten sind ausschließlich Modulationsarten enthalten? }{THOR, Olivia, FreeDV}
{RTTY, PSK31, SSTV}
{M17, FT8, JS8}
{SSB, FM, AM}
\end{QQuestion}

}
\only<2>{
\begin{QQuestion}{NE102}{In welcher der folgenden Antwortmöglichkeiten sind ausschließlich Modulationsarten enthalten? }{THOR, Olivia, FreeDV}
{RTTY, PSK31, SSTV}
{M17, FT8, JS8}
{\textbf{\textcolor{DARCgreen}{SSB, FM, AM}}}
\end{QQuestion}

}

\end{frame}

\begin{frame}\begin{itemize}
  \item Bei SSB ist zu beachten, das richtige Seitenband (LSB oder USB) auszuwählen
  \item Im Amateurfunk wird mit wenigen Ausnahmen unterhalb von \qty{10}{\mega\hertz} das untere Seitenband und ab \qty{10}{\mega\hertz} das obere Seitenband benutzt
  \end{itemize}

\end{frame}

\begin{frame}
\only<1>{
\begin{PQuestion}{NE209}{Die Darstellung zeigt das Display eines Transceivers. Was bedeutet die Anzeige \glqq USB\grqq{}?}{Der \glqq Untere Schmalband Betrieb\grqq{} ist aktiviert. }
{Der Transceiver empfängt in der Modulationsart SSB im unteren Seitenband.}
{Der Transceiver empfängt in der Modulationsart SSB im oberen Seitenband.}
{Die Unterspannung der Batterie ist erreicht.}
{\DARCimage{1.0\linewidth}{588include}}\end{PQuestion}

}
\only<2>{
\begin{PQuestion}{NE209}{Die Darstellung zeigt das Display eines Transceivers. Was bedeutet die Anzeige \glqq USB\grqq{}?}{Der \glqq Untere Schmalband Betrieb\grqq{} ist aktiviert. }
{Der Transceiver empfängt in der Modulationsart SSB im unteren Seitenband.}
{\textbf{\textcolor{DARCgreen}{Der Transceiver empfängt in der Modulationsart SSB im oberen Seitenband.}}}
{Die Unterspannung der Batterie ist erreicht.}
{\DARCimage{1.0\linewidth}{588include}}\end{PQuestion}

}
\end{frame}

\begin{frame}
\only<1>{
\begin{QQuestion}{BC202}{Welches Seitenband wird bei SSB-Telefonie nach IARU-Empfehlung im \qty{80}{m}-Band in der Regel benutzt?}{In der unteren Bandhälfte das untere Seitenband, in der oberen Bandhälfte das obere Seitenband.}
{Im Europaverkehr wird das untere, ansonsten das obere Seitenband benutzt.}
{Um den Nachteil der relativ niedrigen Sendefrequenz des \qty{80}{m}-Bandes auszugleichen, wird das obere Seitenband benutzt.}
{Im \qty{80}{m}-Band wird das untere Seitenband benutzt.}
\end{QQuestion}

}
\only<2>{
\begin{QQuestion}{BC202}{Welches Seitenband wird bei SSB-Telefonie nach IARU-Empfehlung im \qty{80}{m}-Band in der Regel benutzt?}{In der unteren Bandhälfte das untere Seitenband, in der oberen Bandhälfte das obere Seitenband.}
{Im Europaverkehr wird das untere, ansonsten das obere Seitenband benutzt.}
{Um den Nachteil der relativ niedrigen Sendefrequenz des \qty{80}{m}-Bandes auszugleichen, wird das obere Seitenband benutzt.}
{\textbf{\textcolor{DARCgreen}{Im \qty{80}{m}-Band wird das untere Seitenband benutzt.}}}
\end{QQuestion}

}
\end{frame}

\begin{frame}
\only<1>{
\begin{QQuestion}{BC203}{Welches Seitenband wird bei SSB-Telefonie nach Empfehlung der IARU im \qty{20}{m}-Band in der Regel benutzt?}{Um den Nachteil der relativ niedrigen Sendefrequenz des \qty{20}{m}-Bandes auszugleichen, wird das untere Seitenband benutzt.}
{Im Europaverkehr wird das untere, ansonsten das obere Seitenband benutzt.}
{Im \qty{20}{m}-Band wird das obere Seitenband benutzt.}
{In der unteren Bandhälfte das untere Seitenband, in der oberen Bandhälfte das obere Seitenband.}
\end{QQuestion}

}
\only<2>{
\begin{QQuestion}{BC203}{Welches Seitenband wird bei SSB-Telefonie nach Empfehlung der IARU im \qty{20}{m}-Band in der Regel benutzt?}{Um den Nachteil der relativ niedrigen Sendefrequenz des \qty{20}{m}-Bandes auszugleichen, wird das untere Seitenband benutzt.}
{Im Europaverkehr wird das untere, ansonsten das obere Seitenband benutzt.}
{\textbf{\textcolor{DARCgreen}{Im \qty{20}{m}-Band wird das obere Seitenband benutzt.}}}
{In der unteren Bandhälfte das untere Seitenband, in der oberen Bandhälfte das obere Seitenband.}
\end{QQuestion}

}
\end{frame}

\begin{frame}
\only<1>{
\begin{QQuestion}{NE211}{Im \qty{80}{\m}-Band wird bei Sprechfunk das Modulationsverfahren SSB \glqq Unteres Seitenband\grqq{} verwendet. Auf welchen \glqq MODE\grqq{} stellen Sie den Amateurfunk-Empfänger ein?}{LSB}
{USB}
{AM}
{SSB}
\end{QQuestion}

}
\only<2>{
\begin{QQuestion}{NE211}{Im \qty{80}{\m}-Band wird bei Sprechfunk das Modulationsverfahren SSB \glqq Unteres Seitenband\grqq{} verwendet. Auf welchen \glqq MODE\grqq{} stellen Sie den Amateurfunk-Empfänger ein?}{\textbf{\textcolor{DARCgreen}{LSB}}}
{USB}
{AM}
{SSB}
\end{QQuestion}

}
\end{frame}

\begin{frame}
\only<1>{
\begin{QQuestion}{NE210}{Im \qty{2}{\m}-Band wird das \glqq obere Seitenband\grqq{} verwendet. Auf welchen \glqq MODE\grqq{} stellen Sie den Amateurfunk-Transceiver ein?}{LSB}
{USB}
{FM}
{CW}
\end{QQuestion}

}
\only<2>{
\begin{QQuestion}{NE210}{Im \qty{2}{\m}-Band wird das \glqq obere Seitenband\grqq{} verwendet. Auf welchen \glqq MODE\grqq{} stellen Sie den Amateurfunk-Transceiver ein?}{LSB}
{\textbf{\textcolor{DARCgreen}{USB}}}
{FM}
{CW}
\end{QQuestion}

}
\end{frame}

\begin{frame}
\frametitle{Falsches Seitenband}
\begin{itemize}
  \item Wenn bei SSB das falsche Seitenband gewählt wird, dann ist die Sprache völlig unverständlich
  \item Ebenfalls ist es bei SSB wichtig, die Empfangsfrequenz sehr feinfühlig mit dem VFO-Drehknopf einzustellen
  \item Schon kleine Abweichungen von der richtigen Frequenz führen dazu, dass die Sprache unverständlich wird
  \end{itemize}

\end{frame}

\begin{frame}
\only<1>{
\begin{QQuestion}{NE212}{Sie können die Sprache beim SSB-Empfang nicht verstehen. Welche Vorgehensweise führt zum Ziel?}{Sie drehen am VFO-Knopf und drücken die TUNE-Taste.}
{Sie kontrollieren die Seitenbandeinstellung und drehen am VFO-Knopf.}
{Sie drehen am RIT-Knopf und drücken die PTT.}
{Sie beobachten das Wasserfalldiagramm und wechseln in die Modulationsart AM.}
\end{QQuestion}

}
\only<2>{
\begin{QQuestion}{NE212}{Sie können die Sprache beim SSB-Empfang nicht verstehen. Welche Vorgehensweise führt zum Ziel?}{Sie drehen am VFO-Knopf und drücken die TUNE-Taste.}
{\textbf{\textcolor{DARCgreen}{Sie kontrollieren die Seitenbandeinstellung und drehen am VFO-Knopf.}}}
{Sie drehen am RIT-Knopf und drücken die PTT.}
{Sie beobachten das Wasserfalldiagramm und wechseln in die Modulationsart AM.}
\end{QQuestion}

}
\end{frame}%ENDCONTENT
