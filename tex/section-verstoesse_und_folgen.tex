
\section{Verstöße und Folgen}
\label{section:verstoesse_und_folgen}
\begin{frame}%STARTCONTENT

\frametitle{Verstöße und Folgen}
Die Bundesnetzagentur kann bei Verstößen gegen AFuG oder AFuV eine Einschränkung des Betriebes oder die Außerbetriebnahme der Amateurfunkstelle anordnen. Wenn fortgesetzt gegen AFuG oder AFuV verstoßen wird, kann die Amateurfunkzulassung widerrufen werden.

\end{frame}

\begin{frame}Ordnungswidrigkeiten im Sinne des Amateurfunkgesetzes (AFuG) sind:

\begin{itemize}
  \item Betrieb ohne Zulassung und damit ohne Rufzeichen
  \item Geschäftsmäßiges Erbringen von Telekommunikationsdienstleistungen
  \item Nachrichtenübermittlung für und an Dritte
  \end{itemize}
    \pause
    Die Bundesnetzagentur kann einen Verstoß mit einer Geldbuße ahnden.



\end{frame}

\begin{frame}Eine Ordnungswidrigkeit gemäß Telekommunikationsgesetz (TKG) ist die Nutzung von Frequenzen ohne Frequenzzuteilung. Für den Funkamateur bedeutet das:

Wenn er mit seiner Amateurfunkanlage außerhalb der zugeteilten Amateurfunkbänder sendet, begeht er eine Ordnungswidrigkeit.

\end{frame}

\begin{frame}
\only<1>{
\begin{QQuestion}{VC122}{Die Bundesnetzagentur kann bei Verstößen gegen AFuG oder AFuV~...}{einen sofortigen Abbau der Amateurfunkstelle noch vor Ort anordnen.}
{eine Einschränkung des Betriebes oder die Außerbetriebnahme der Amateurfunkstelle anordnen.}
{ein Unbrauchbarmachen der Amateurfunkstelle durch Entnahme wichtiger Teile aus dem Sender anordnen.}
{eine kostenpflichtige fachliche Nachprüfung anordnen.}
\end{QQuestion}

}
\only<2>{
\begin{QQuestion}{VC122}{Die Bundesnetzagentur kann bei Verstößen gegen AFuG oder AFuV~...}{einen sofortigen Abbau der Amateurfunkstelle noch vor Ort anordnen.}
{\textbf{\textcolor{DARCgreen}{eine Einschränkung des Betriebes oder die Außerbetriebnahme der Amateurfunkstelle anordnen.}}}
{ein Unbrauchbarmachen der Amateurfunkstelle durch Entnahme wichtiger Teile aus dem Sender anordnen.}
{eine kostenpflichtige fachliche Nachprüfung anordnen.}
\end{QQuestion}

}
\end{frame}

\begin{frame}
\only<1>{
\begin{QQuestion}{VC123}{Was hat ein Funkamateur mit zugeteiltem Rufzeichen zu erwarten, wenn er fortgesetzt gegen AFuG oder AFuV verstößt?}{Ausbildungsfunkbetrieb wird untersagt}
{Den Widerruf der Amateurfunkzulassung}
{Eine kostenpflichtige Nachprüfung}
{Einzug des Amateurfunkzeugnisses}
\end{QQuestion}

}
\only<2>{
\begin{QQuestion}{VC123}{Was hat ein Funkamateur mit zugeteiltem Rufzeichen zu erwarten, wenn er fortgesetzt gegen AFuG oder AFuV verstößt?}{Ausbildungsfunkbetrieb wird untersagt}
{\textbf{\textcolor{DARCgreen}{Den Widerruf der Amateurfunkzulassung}}}
{Eine kostenpflichtige Nachprüfung}
{Einzug des Amateurfunkzeugnisses}
\end{QQuestion}

}
\end{frame}

\begin{frame}
\only<1>{
\begin{QQuestion}{VC124}{Welche der folgenden Handlungen sind Ordnungswidrigkeiten im Sinne des Amateurfunkgesetzes (AFuG) und können mit Bußgeldern von bis zu 5000 bzw. 10000 Euro geahndet werden?}{(1) Unzureichende Rufzeichennennung; (2) Überschreiten der zulässigen Bandbreite einer Aussendung; (3) Gleichzeitige Nutzung eines Rufzeichens von verschiedenen Standorten}
{(1) Überschreiten der zulässigen Sendeleistung; (2) Dauerhafte Verlegung der Amateurfunkstelle an einen anderen Standort; (3) Funkbetrieb ohne Mitführen der Zulassungsurkunde}
{(1) Betrieb ohne Zulassung und Zuteilung eines Rufzeichens; (2) Geschäftsmäßiges Erbringen von Telekommunikationsdiensten; (3) Nachrichtenübermittlung an Dritte}
{(1) Verschlüsselung von Amateurfunkverkehr zur Verschleierung des Inhalts; (2) Betrieb einer Remote-Station ohne Betriebsmeldung; (3) Überlassung des Rufzeichens an Dritte;}
\end{QQuestion}

}
\only<2>{
\begin{QQuestion}{VC124}{Welche der folgenden Handlungen sind Ordnungswidrigkeiten im Sinne des Amateurfunkgesetzes (AFuG) und können mit Bußgeldern von bis zu 5000 bzw. 10000 Euro geahndet werden?}{(1) Unzureichende Rufzeichennennung; (2) Überschreiten der zulässigen Bandbreite einer Aussendung; (3) Gleichzeitige Nutzung eines Rufzeichens von verschiedenen Standorten}
{(1) Überschreiten der zulässigen Sendeleistung; (2) Dauerhafte Verlegung der Amateurfunkstelle an einen anderen Standort; (3) Funkbetrieb ohne Mitführen der Zulassungsurkunde}
{\textbf{\textcolor{DARCgreen}{(1) Betrieb ohne Zulassung und Zuteilung eines Rufzeichens; (2) Geschäftsmäßiges Erbringen von Telekommunikationsdiensten; (3) Nachrichtenübermittlung an Dritte}}}
{(1) Verschlüsselung von Amateurfunkverkehr zur Verschleierung des Inhalts; (2) Betrieb einer Remote-Station ohne Betriebsmeldung; (3) Überlassung des Rufzeichens an Dritte;}
\end{QQuestion}

}
\end{frame}

\begin{frame}
\only<1>{
\begin{QQuestion}{VC125}{Was hat ein Funkamateur zu erwarten, der seine Amateurfunkstelle ordnungswidrig betreibt?}{Der Funkamateur hat mit dem Entzug des Amateurfunkzeugnisses zu rechnen.}
{Die Bundesnetzagentur kann einen Verstoß mit einer Geldbuße ahnden.}
{Der Funkamateur hat mit der Beschlagnahmung der Amateurfunkanlage durch die BNetzA zu rechnen.}
{Die Bundesnetzagentur kann eine kostenpflichtige Nachprüfung anordnen.}
\end{QQuestion}

}
\only<2>{
\begin{QQuestion}{VC125}{Was hat ein Funkamateur zu erwarten, der seine Amateurfunkstelle ordnungswidrig betreibt?}{Der Funkamateur hat mit dem Entzug des Amateurfunkzeugnisses zu rechnen.}
{\textbf{\textcolor{DARCgreen}{Die Bundesnetzagentur kann einen Verstoß mit einer Geldbuße ahnden.}}}
{Der Funkamateur hat mit der Beschlagnahmung der Amateurfunkanlage durch die BNetzA zu rechnen.}
{Die Bundesnetzagentur kann eine kostenpflichtige Nachprüfung anordnen.}
\end{QQuestion}

}
\end{frame}

\begin{frame}
\only<1>{
\begin{QQuestion}{VE103}{Welcher der nachfolgend genannten Tatbestände ist eine Ordnungswidrigkeit gemäß Telekommunikationsgesetz (TKG)?}{Nutzung von Frequenzen ohne Frequenzzuteilung}
{Nutzung von öffentlichen Telekommunikationseinrichtungen zur Vernetzung von Relaisfunkstellen}
{Die Übermittlung von Amateurfunknachrichten von oder an Dritte durch einen Funkamateur}
{Der Betrieb einer Amateurfunkstelle zu gewerblich-wirtschaftlichen Zwecken}
\end{QQuestion}

}
\only<2>{
\begin{QQuestion}{VE103}{Welcher der nachfolgend genannten Tatbestände ist eine Ordnungswidrigkeit gemäß Telekommunikationsgesetz (TKG)?}{\textbf{\textcolor{DARCgreen}{Nutzung von Frequenzen ohne Frequenzzuteilung}}}
{Nutzung von öffentlichen Telekommunikationseinrichtungen zur Vernetzung von Relaisfunkstellen}
{Die Übermittlung von Amateurfunknachrichten von oder an Dritte durch einen Funkamateur}
{Der Betrieb einer Amateurfunkstelle zu gewerblich-wirtschaftlichen Zwecken}
\end{QQuestion}

}
\end{frame}%ENDCONTENT
