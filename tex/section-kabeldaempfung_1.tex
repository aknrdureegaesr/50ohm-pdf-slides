
\section{Kabeldämpfung I}
\label{section:kabeldaempfung_1}
\begin{frame}%STARTCONTENT
\begin{itemize}
  \item Signalstärke eines Hochfrequenzsignals nimmt bei zunehmender Kabellänge ab
  \item Wird als \emph{Kabeldämpfung} bezeichnet
  \item Auch Stecker können das Signal dämpfen
  \item Ist unerwünscht
  \end{itemize}
\end{frame}

\begin{frame}
\begin{columns}
    \begin{column}{0.48\textwidth}
    \begin{itemize}
  \item Dämpfung wird in der Regel in Dezibel (dB) angegeben
  \item Wenn von Dämpfung gesprochen wird, bleibt die Zahl positiv
  \end{itemize}

    \end{column}
   \begin{column}{0.48\textwidth}
       \begin{itemize}
  \item Faktor zu dB-Umrechnung verwenden
  \item Oder in der Formelsammlung nachschlagen
  \end{itemize}

   \end{column}
\end{columns}

\end{frame}

\begin{frame}
\only<1>{
\begin{QQuestion}{EG309}{Am Ende einer Antennenleitung ist nur noch ein Viertel der Leistung vorhanden. Wie groß ist das Dämpfungsmaß des Kabels?}{\qty{3}{\decibel}}
{\qty{6}{\decibel}}
{\qty{10}{\decibel}}
{\qty{16}{\decibel}}
\end{QQuestion}

}
\only<2>{
\begin{QQuestion}{EG309}{Am Ende einer Antennenleitung ist nur noch ein Viertel der Leistung vorhanden. Wie groß ist das Dämpfungsmaß des Kabels?}{\qty{3}{\decibel}}
{\textbf{\textcolor{DARCgreen}{\qty{6}{\decibel}}}}
{\qty{10}{\decibel}}
{\qty{16}{\decibel}}
\end{QQuestion}

}
\end{frame}

\begin{frame}
\only<1>{
\begin{QQuestion}{EG310}{Am Ende einer Antennenleitung ist nur noch ein Zehntel der Leistung vorhanden. Wie groß ist das Dämpfungsmaß des Kabels?}{\qty{3}{\decibel}}
{\qty{10}{\decibel}}
{\qty{6}{\decibel}}
{\qty{16}{\decibel}}
\end{QQuestion}

}
\only<2>{
\begin{QQuestion}{EG310}{Am Ende einer Antennenleitung ist nur noch ein Zehntel der Leistung vorhanden. Wie groß ist das Dämpfungsmaß des Kabels?}{\qty{3}{\decibel}}
{\textbf{\textcolor{DARCgreen}{\qty{10}{\decibel}}}}
{\qty{6}{\decibel}}
{\qty{16}{\decibel}}
\end{QQuestion}

}
\end{frame}

\begin{frame}
\only<1>{
\begin{QQuestion}{EG308}{Eine HF-Ausgangsleistung von \qty{100}{\W} wird in eine angepasste Übertragungsleitung eingespeist. Am antennenseitigen Ende der Leitung beträgt die Leistung \qty{50}{\W} bei einem SWR von 1. Wie hoch ist die Leitungsdämpfung?}{\qty{-3}{\decibel}}
{\qty{-6}{\decibel}}
{\qty{3}{\decibel}}
{\qty{6}{\dBm}}
\end{QQuestion}

}
\only<2>{
\begin{QQuestion}{EG308}{Eine HF-Ausgangsleistung von \qty{100}{\W} wird in eine angepasste Übertragungsleitung eingespeist. Am antennenseitigen Ende der Leitung beträgt die Leistung \qty{50}{\W} bei einem SWR von 1. Wie hoch ist die Leitungsdämpfung?}{\qty{-3}{\decibel}}
{\qty{-6}{\decibel}}
{\textbf{\textcolor{DARCgreen}{\qty{3}{\decibel}}}}
{\qty{6}{\dBm}}
\end{QQuestion}

}
\end{frame}

\begin{frame}
\frametitle{Kabelverluste}
\begin{itemize}
  \item Alle Verluste, die in Kabeln entstehen
  \item Antenne und Verstärker verstärken das Signal, verändern jedoch nicht die Kabelverluste
  \end{itemize}
\end{frame}

\begin{frame}
\only<1>{
\begin{PQuestion}{EG307}{Die Skizze zeigt den Aufbau einer Amateurfunkstelle. Die Summe aller Kabelverluste in Dezibel betragen~...}{\qty{3}{\decibel}}
{\qty{-5}{\decibel}}
{\qty{5}{\decibel}}
{\qty{-3}{\decibel}}
{\DARCimage{1.0\linewidth}{439include}}\end{PQuestion}

}
\only<2>{
\begin{PQuestion}{EG307}{Die Skizze zeigt den Aufbau einer Amateurfunkstelle. Die Summe aller Kabelverluste in Dezibel betragen~...}{\qty{3}{\decibel}}
{\qty{-5}{\decibel}}
{\textbf{\textcolor{DARCgreen}{\qty{5}{\decibel}}}}
{\qty{-3}{\decibel}}
{\DARCimage{1.0\linewidth}{439include}}\end{PQuestion}

}
\end{frame}

\begin{frame}
\frametitle{Kabeldämpfungsdiagramm}
\begin{columns}
    \begin{column}{0.48\textwidth}
    \begin{itemize}
  \item Im Anhang der Formelsammlung
  \item Dämpfungen verschiedener Kabel in Abhängigkeit zur Frequenz
  \item Bezug auf 100m -- bei kürzeren Kabeln muss umgerechnet werden
  \end{itemize}

    \end{column}
   \begin{column}{0.48\textwidth}
       
\begin{figure}
    \DARCimage{0.85\linewidth}{202include}
    \caption{\scriptsize Kabeldämpfungsdiagramm im Anhang der Formelsammlung}
    \label{e_kabeldaempfung_diagramm}
\end{figure}


   \end{column}
\end{columns}

\end{frame}

\begin{frame}
\only<1>{
\begin{QQuestion}{EG312}{Welche Dämpfung ergibt sich auf der Grundlage des Kabeldämpfungsdiagramms für ein \qty{100}{\m} langes Koaxialkabel mit Voll-PE-Dielektrikum, \qty{4,95}{\mm} Durchmesser (Typ RG58), bei \qty{145}{\MHz}?}{\qty{0}{\decibel}}
{\qty{39}{\decibel}}
{\qty{1}{\decibel}}
{\qty{20}{\decibel}}
\end{QQuestion}

}
\only<2>{
\begin{QQuestion}{EG312}{Welche Dämpfung ergibt sich auf der Grundlage des Kabeldämpfungsdiagramms für ein \qty{100}{\m} langes Koaxialkabel mit Voll-PE-Dielektrikum, \qty{4,95}{\mm} Durchmesser (Typ RG58), bei \qty{145}{\MHz}?}{\qty{0}{\decibel}}
{\qty{39}{\decibel}}
{\qty{1}{\decibel}}
{\textbf{\textcolor{DARCgreen}{\qty{20}{\decibel}}}}
\end{QQuestion}

}
\end{frame}

\begin{frame}
\frametitle{Lösungsweg}
\begin{itemize}
  \item gesucht: Dämpfung für 100m RG58 Kabel bei \qty{145}{\mega\hertz}
  \item Lösung: Ablesen aus Diagramm
  \item Schnittpunkt der RG58 Linie mit \qty{145}{\mega\hertz} $\rightarrow$ \qty{20}{\dB}
  \end{itemize}
\end{frame}

\begin{frame}
\only<1>{
\begin{QQuestion}{EG311}{Ein \qty{100}{\m} langes Koaxialkabel hat eine Dämpfung von \qty{20}{\decibel} bei \qty{145}{\MHz}. Wie hoch ist die Dämpfung bei einer Länge von \qty{20}{\m}?}{\qty{5}{\decibel}}
{\qty{7,25}{\decibel}}
{\qty{4}{\decibel}}
{\qty{1,45}{\decibel}}
\end{QQuestion}

}
\only<2>{
\begin{QQuestion}{EG311}{Ein \qty{100}{\m} langes Koaxialkabel hat eine Dämpfung von \qty{20}{\decibel} bei \qty{145}{\MHz}. Wie hoch ist die Dämpfung bei einer Länge von \qty{20}{\m}?}{\qty{5}{\decibel}}
{\qty{7,25}{\decibel}}
{\textbf{\textcolor{DARCgreen}{\qty{4}{\decibel}}}}
{\qty{1,45}{\decibel}}
\end{QQuestion}

}
\end{frame}

\begin{frame}
\frametitle{Lösungsweg}
\begin{itemize}
  \item gesucht: Dämpfung für 20m bei 20dB Dämpfung auf 100m
  \item Lösung: Dreisatz
  \end{itemize}
$\dfrac{20dB}{100m} = \dfrac{x}{20m}$

$x = \dfrac{20dB\cdot 20m}{100m} = 4dB$

\end{frame}

\begin{frame}
\only<1>{
\begin{QQuestion}{EG313}{Welche Dämpfung ergibt sich auf der Grundlage des Kabeldämpfungsdiagramms für ein \qty{15}{\m} langes Koaxialkabel mit Voll-PE-Dielektrikum, \qty{4,95}{\mm} Durchmesser (Typ RG58), bei \qty{145}{\MHz}?}{\qty{2}{\decibel}}
{\qty{4}{\decibel}}
{\qty{3}{\decibel}}
{\qty{1}{\decibel}}
\end{QQuestion}

}
\only<2>{
\begin{QQuestion}{EG313}{Welche Dämpfung ergibt sich auf der Grundlage des Kabeldämpfungsdiagramms für ein \qty{15}{\m} langes Koaxialkabel mit Voll-PE-Dielektrikum, \qty{4,95}{\mm} Durchmesser (Typ RG58), bei \qty{145}{\MHz}?}{\qty{2}{\decibel}}
{\qty{4}{\decibel}}
{\textbf{\textcolor{DARCgreen}{\qty{3}{\decibel}}}}
{\qty{1}{\decibel}}
\end{QQuestion}

}
\end{frame}

\begin{frame}
\frametitle{Lösungsweg}
\begin{itemize}
  \item gesucht: Dämpfung für 15m RG58 Kabel bei \qty{145}{\mega\hertz}
  \item Lösung: Ablesen aus Diagramm und Dreisatz
  \item Schnittpunkt der RG58 Linie mit \qty{145}{\mega\hertz} $\rightarrow$ \qty{20}{\dB}
  \end{itemize}
$\dfrac{20dB}{100m} = \dfrac{x}{15m}$

$x = \dfrac{20dB\cdot 15m}{100m} = 3dB$

\end{frame}

\begin{frame}
\only<1>{
\begin{QQuestion}{EG314}{Welche Dämpfung ergibt sich auf der Grundlage des Kabeldämpfungsdiagramms für ein \qty{50}{\m} langes Koaxialkabel mit Voll-PE-Dielektrikum, \qty{2,8}{\mm} Durchmesser (Typ RG174), bei \qty{145}{\MHz}?}{\qty{12}{\decibel}}
{\qty{40}{\decibel}}
{\qty{68}{\decibel}}
{\qty{20}{\decibel}}
\end{QQuestion}

}
\only<2>{
\begin{QQuestion}{EG314}{Welche Dämpfung ergibt sich auf der Grundlage des Kabeldämpfungsdiagramms für ein \qty{50}{\m} langes Koaxialkabel mit Voll-PE-Dielektrikum, \qty{2,8}{\mm} Durchmesser (Typ RG174), bei \qty{145}{\MHz}?}{\qty{12}{\decibel}}
{\qty{40}{\decibel}}
{\qty{68}{\decibel}}
{\textbf{\textcolor{DARCgreen}{\qty{20}{\decibel}}}}
\end{QQuestion}

}
\end{frame}

\begin{frame}
\only<1>{
\begin{QQuestion}{EG315}{Welche Dämpfung ergibt sich auf der Grundlage des Kabeldämpfungsdiagramms für ein \qty{40}{\m} langes Koaxialkabel, PE-Schaum-Dielektrikum mit \qty{12,7}{\mm} Durchmesser, bei \qty{435}{\MHz}?}{\qty{0,8}{\decibel}}
{\qty{3,8}{\decibel}}
{\qty{1,8}{\decibel}}
{\qty{2,8}{\decibel}}
\end{QQuestion}

}
\only<2>{
\begin{QQuestion}{EG315}{Welche Dämpfung ergibt sich auf der Grundlage des Kabeldämpfungsdiagramms für ein \qty{40}{\m} langes Koaxialkabel, PE-Schaum-Dielektrikum mit \qty{12,7}{\mm} Durchmesser, bei \qty{435}{\MHz}?}{\qty{0,8}{\decibel}}
{\qty{3,8}{\decibel}}
{\qty{1,8}{\decibel}}
{\textbf{\textcolor{DARCgreen}{\qty{2,8}{\decibel}}}}
\end{QQuestion}

}
\end{frame}

\begin{frame}
\only<1>{
\begin{QQuestion}{EG316}{Welche Dämpfung ergibt sich auf der Grundlage des Kabeldämpfungsdiagramms für ein \qty{40}{\m} langes Koaxialkabel mit PE-Schaum-Dielektrikum und \qty{10,3}{\mm} Durchmesser im \qty{23}{\cm}-Band (\qty{1296}{\MHz})?}{\qty{8,2}{\decibel}}
{\qty{12,6}{\decibel}}
{\qty{10,4}{\decibel}}
{\qty{6,2}{\decibel}}
\end{QQuestion}

}
\only<2>{
\begin{QQuestion}{EG316}{Welche Dämpfung ergibt sich auf der Grundlage des Kabeldämpfungsdiagramms für ein \qty{40}{\m} langes Koaxialkabel mit PE-Schaum-Dielektrikum und \qty{10,3}{\mm} Durchmesser im \qty{23}{\cm}-Band (\qty{1296}{\MHz})?}{\textbf{\textcolor{DARCgreen}{\qty{8,2}{\decibel}}}}
{\qty{12,6}{\decibel}}
{\qty{10,4}{\decibel}}
{\qty{6,2}{\decibel}}
\end{QQuestion}

}
\end{frame}%ENDCONTENT
