
\section{Yagi-Uda-Antenne III}
\label{section:yagi_uda_3}
\begin{frame}%STARTCONTENT

\only<1>{
\begin{QQuestion}{AG212}{Die Impedanz des Strahlers eines Kurzwellenbeams richtet sich auch nach~...}{den Abständen zwischen Reflektor, Strahler und den Direktoren.}
{dem Strahlungswiderstand des Reflektors.}
{dem Widerstand des Zuführungskabels.}
{den Ausbreitungsbedingungen.}
\end{QQuestion}

}
\only<2>{
\begin{QQuestion}{AG212}{Die Impedanz des Strahlers eines Kurzwellenbeams richtet sich auch nach~...}{\textbf{\textcolor{DARCgreen}{den Abständen zwischen Reflektor, Strahler und den Direktoren.}}}
{dem Strahlungswiderstand des Reflektors.}
{dem Widerstand des Zuführungskabels.}
{den Ausbreitungsbedingungen.}
\end{QQuestion}

}
\end{frame}

\begin{frame}
\only<1>{
\begin{QQuestion}{AG222}{Worin unterscheidet sich eine Yagi-Uda-Antenne mit 11 Elementen von einer mit 3 Elementen? Bei der Antenne mit 11 Elementen ist~...}{der Strahlungswiderstand erhöht.}
{der Öffnungswinkel erhöht.}
{der Öffnungswinkel verringert.}
{das Vor-Rück-Verhältnis verringert.}
\end{QQuestion}

}
\only<2>{
\begin{QQuestion}{AG222}{Worin unterscheidet sich eine Yagi-Uda-Antenne mit 11 Elementen von einer mit 3 Elementen? Bei der Antenne mit 11 Elementen ist~...}{der Strahlungswiderstand erhöht.}
{der Öffnungswinkel erhöht.}
{\textbf{\textcolor{DARCgreen}{der Öffnungswinkel verringert.}}}
{das Vor-Rück-Verhältnis verringert.}
\end{QQuestion}

}
\end{frame}

\begin{frame}
\only<1>{
\begin{QQuestion}{AG126}{Für die Erzeugung von zirkularer Polarisation mit Yagi-Uda-Antennen wird eine horizontale und eine dazu um \qty{90}{\degree} um die Strahlungsachse gedrehte Yagi-Uda-Antenne zusammengeschaltet. Was ist dabei zu beachten, damit tatsächlich zirkulare Polarisation entsteht?}{Bei einer der Antennen muss die Welle um $\lambda$/4 verzögert werden. Dies kann entweder durch eine zusätzlich eingefügte Viertelwellen-Verzögerungsleitung oder durch mechanische \glqq Verschiebung\grqq{} beider Yagi-Uda-Antennen um $\lambda$/4 gegeneinander hergestellt werden.}
{Bei einer der Antennen muss die Welle um $\lambda$/2 verzögert werden. Dies kann entweder durch eine zusätzlich eingefügte $\lambda$/2-Verzögerungsleitung oder durch mechanische \glqq Verschiebung\grqq{} beider Yagi-Uda-Antennen um $\lambda$/2 gegeneinander hergestellt werden.}
{Die Zusammenschaltung der Antennen muss über eine Halbwellen-Lecherleitung erfolgen. Zur Anpassung an den Wellenwiderstand muss zwischen der Speiseleitung und den Antennen noch ein $\lambda$/4-Transformationsstück eingefügt werden.}
{Die kreuzförmig angeordneten Elemente der beiden Antennen sind um \qty{45}{\degree} zu verdrehen, so dass in der Draufsicht ein liegendes Kreuz gebildet wird. Die Antennen werden über Leitungsstücke gleicher Länge parallel geschaltet. Die Anpassung erfolgt mit einem Symmetrierglied.}
\end{QQuestion}

}
\only<2>{
\begin{QQuestion}{AG126}{Für die Erzeugung von zirkularer Polarisation mit Yagi-Uda-Antennen wird eine horizontale und eine dazu um \qty{90}{\degree} um die Strahlungsachse gedrehte Yagi-Uda-Antenne zusammengeschaltet. Was ist dabei zu beachten, damit tatsächlich zirkulare Polarisation entsteht?}{\textbf{\textcolor{DARCgreen}{Bei einer der Antennen muss die Welle um $\lambda$/4 verzögert werden. Dies kann entweder durch eine zusätzlich eingefügte Viertelwellen-Verzögerungsleitung oder durch mechanische \glqq Verschiebung\grqq{} beider Yagi-Uda-Antennen um $\lambda$/4 gegeneinander hergestellt werden.}}}
{Bei einer der Antennen muss die Welle um $\lambda$/2 verzögert werden. Dies kann entweder durch eine zusätzlich eingefügte $\lambda$/2-Verzögerungsleitung oder durch mechanische \glqq Verschiebung\grqq{} beider Yagi-Uda-Antennen um $\lambda$/2 gegeneinander hergestellt werden.}
{Die Zusammenschaltung der Antennen muss über eine Halbwellen-Lecherleitung erfolgen. Zur Anpassung an den Wellenwiderstand muss zwischen der Speiseleitung und den Antennen noch ein $\lambda$/4-Transformationsstück eingefügt werden.}
{Die kreuzförmig angeordneten Elemente der beiden Antennen sind um \qty{45}{\degree} zu verdrehen, so dass in der Draufsicht ein liegendes Kreuz gebildet wird. Die Antennen werden über Leitungsstücke gleicher Länge parallel geschaltet. Die Anpassung erfolgt mit einem Symmetrierglied.}
\end{QQuestion}

}
\end{frame}%ENDCONTENT
