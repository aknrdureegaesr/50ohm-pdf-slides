
\section{Vielfachzugriff}
\label{section:vielfachzugriff}
\begin{frame}%STARTCONTENT

\frametitle{TDMA}
\begin{columns}
    \begin{column}{0.48\textwidth}
    \begin{itemize}
  \item Time Division Multiple Access – Zeitmultiplexverfahren
  \item Die digitalen Nutzdaten werden getrennt und nacheinander über die dieselbe Frequenz gesandt
  \item Am Empfänger wird der Datenstrom wieder zusammengesetzt
  \end{itemize}

    \end{column}
   \begin{column}{0.48\textwidth}
       
\begin{figure}
    \DARCimage{0.85\linewidth}{844include}
    \caption{\scriptsize Zeitmultiplexverfahren mit drei Signalen}
    \label{e_vielfachzugriff_tdma}
\end{figure}


   \end{column}
\end{columns}

\end{frame}

\begin{frame}
\only<1>{
\begin{QQuestion}{EE409}{Wie werden bei Zeitmultiplexverfahren (TDMA) mehrere Signale gleichzeitig übertragen?}{Im schnellen zeitlichen Wechsel auf derselben Frequenz}
{Zeitgleich auf unterschiedlichen Frequenzen}
{Zeitgleich mit Spreizcodierung im selben Frequenzbereich}
{Zeitgleich auf unterschiedlichen Wegen}
\end{QQuestion}

}
\only<2>{
\begin{QQuestion}{EE409}{Wie werden bei Zeitmultiplexverfahren (TDMA) mehrere Signale gleichzeitig übertragen?}{\textbf{\textcolor{DARCgreen}{Im schnellen zeitlichen Wechsel auf derselben Frequenz}}}
{Zeitgleich auf unterschiedlichen Frequenzen}
{Zeitgleich mit Spreizcodierung im selben Frequenzbereich}
{Zeitgleich auf unterschiedlichen Wegen}
\end{QQuestion}

}
\end{frame}

\begin{frame}
\frametitle{CDMA}
\begin{columns}
    \begin{column}{0.48\textwidth}
    \begin{itemize}
  \item Code Division Multiple Access – Codemultiplexverfahren
  \item Die digitalen Nutzdaten werden mit einem digitalen Code codiert (gemischt)
  \item Am Empfänger wird derselbe digitale Code zum decodieren verwendet
  \end{itemize}

    \end{column}
   \begin{column}{0.48\textwidth}
       
\begin{figure}
    \DARCimage{0.85\linewidth}{846include}
    \caption{\scriptsize Codemultiplexverfahren mit drei Signalen}
    \label{e_vielfachzugriff_cdma}
\end{figure}


   \end{column}
\end{columns}

\end{frame}

\begin{frame}
\only<1>{
\begin{QQuestion}{EE411}{Wie werden bei Codemultiplexverfahren (CDMA) mehrere Signale gleichzeitig übertragen?}{Im schnellen zeitlichen Wechsel auf derselben Frequenz}
{Zeitgleich auf unterschiedlichen Frequenzen}
{Zeitgleich mit Spreizcodierung im selben Frequenzbereich}
{Zeitgleich auf unterschiedlichen Wegen}
\end{QQuestion}

}
\only<2>{
\begin{QQuestion}{EE411}{Wie werden bei Codemultiplexverfahren (CDMA) mehrere Signale gleichzeitig übertragen?}{Im schnellen zeitlichen Wechsel auf derselben Frequenz}
{Zeitgleich auf unterschiedlichen Frequenzen}
{\textbf{\textcolor{DARCgreen}{Zeitgleich mit Spreizcodierung im selben Frequenzbereich}}}
{Zeitgleich auf unterschiedlichen Wegen}
\end{QQuestion}

}
\end{frame}

\begin{frame}
\frametitle{FDMA}
\begin{columns}
    \begin{column}{0.48\textwidth}
    \begin{itemize}
  \item Frequency Division Multiple Access – Frequenzmultiplexverfahren
  \item Das digitale Signal wird auf mehrere Frequenzen aufgeteilt
  \item Dadurch kann mehr Bandbreite verwendet werden
  \end{itemize}

    \end{column}
   \begin{column}{0.48\textwidth}
       
\begin{figure}
    \DARCimage{0.85\linewidth}{845include}
    \caption{\scriptsize Frequenzmultiplexverfahren mit drei Signalen}
    \label{e_vielfachzugriff_cdma}
\end{figure}


   \end{column}
\end{columns}

\end{frame}

\begin{frame}
\only<1>{
\begin{QQuestion}{EE410}{Wie werden bei Frequenzmultiplexverfahren (FDMA) mehrere Signale gleichzeitig übertragen?}{Zeitgleich auf unterschiedlichen Frequenzen}
{Im schnellen zeitlichen Wechsel auf derselben Frequenz}
{Zeitgleich mit Spreizcodierung im selben Frequenzbereich}
{Zeitgleich auf unterschiedlichen Wegen}
\end{QQuestion}

}
\only<2>{
\begin{QQuestion}{EE410}{Wie werden bei Frequenzmultiplexverfahren (FDMA) mehrere Signale gleichzeitig übertragen?}{\textbf{\textcolor{DARCgreen}{Zeitgleich auf unterschiedlichen Frequenzen}}}
{Im schnellen zeitlichen Wechsel auf derselben Frequenz}
{Zeitgleich mit Spreizcodierung im selben Frequenzbereich}
{Zeitgleich auf unterschiedlichen Wegen}
\end{QQuestion}

}
\end{frame}%ENDCONTENT
